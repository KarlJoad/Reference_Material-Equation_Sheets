\documentclass[10pt,letterpaper,final,twocolumn,twoside,notitlepage]{article}
\usepackage[margin=.5in]{geometry} % 1/2 inch margins on all pages
\usepackage[utf8]{inputenc} % Define the input encoding
\usepackage[USenglish]{babel} % Define language used
\usepackage{amsmath}
\usepackage{amsfonts}
\usepackage{amssymb}
\usepackage{amsthm} % Gives us plain, definition, and remark to use in \theoremstyle{style}
\usepackage{graphicx}

\usepackage{hyperref} % Generate hyperlinks to referenced items
\usepackage[noabbrev,nameinlink]{cleveref} % Fancy cross-references in the document everywhere
\usepackage{nameref} % Can make references by name to places
\usepackage{subcaption} % Allows for multiple figures in one Figure environment
\usepackage{siunitx} % Gives us ways to typeset units for stuff
\usepackage{enumitem} % Provides [noitemsep, nolistsep] for more compact lists
\usepackage{chngcntr} % Allows us to tamper with the counter a little more
\usepackage{empheq} % Allow boxing of equations in special math environments
\usepackage{tcolorbox} % Allows us to create boxes of various types for examples
\usepackage{tikz} % Allows us to create TikZ and PGF Pictures
%\usepackage{ctable} % Greater control over tables and how they look

%\graphicspath{{./Drawings/MS_201}} % Uncomment this to use pictures in this document
\numberwithin{equation}{section} % Uncomment this to number equations with section numbers too

\author{Karl Hallsby}
\title{MS 201 Equations}

\begin{document}
\section{Exam 1 Equations} \label{sec:Exam 1}
	\subsection{Ch. 2 - Interatomic Forces}
		\begin{equation}
			E_{N} = E_{A} + E_{R} = -\frac{A}{r} + \frac{B}{r^{n}}
		\end{equation}
		\begin{equation}
			A = \frac{1}{4 \pi \epsilon_{0}} \left( Z_{1} e \right) \left( Z_{2} e \right)
		\end{equation}
		\begin{itemize}[noitemsep]
			\item This equation works for both $A$ and $B$
			\item $\epsilon_{0}$ = $8.85 \times 10^{-12} \si{\farad / \meter}$
			\item $e$ = $1.602 \times 10^{-19} \si{\coulomb}$
			\item $r$ - Radius in \si{\meter}
		\end{itemize}

		\begin{equation}
			F_{A} = \frac{\left( 1.602 \times 10^{-19} \right)^{2}}{4 \pi \left( 8.85 \times 10^{-12} \right) r^{2}} \left( \lVert Z_{1} \rVert \right) \left( \lVert Z_{2} \rVert \right)
		\end{equation}
		\begin{itemize}[noitemsep]
			\item $F_{A}$ - Force of Attraction
			\item $r$ - Distance in \si{\meter}
			\item $Z$ - Number of Valence Electrons
			\item $F_{A}$ - Interatomic Force in \si{\newton}
			\item $-F_{A} = F_{R}$ - Attractive and Repulsive Force Equal and Opposite
		\end{itemize}

		\begin{align}
			\text{Force} &= \frac{dE}{dr} \\
			\text{Elastic Modulus} &= \frac{dF}{dr}
		\end{align}
	
		\begin{equation}
			\text{\%IC} = \left( 1 - e^{\frac{\left( x_{A}-x_{B} \right)^{2}}{4}} \right) \times 100\%
		\end{equation}
		\begin{itemize}[noitemsep]
			\item $\text{\%IC}$ - \% Ionic Character
			\item $x$ - Electronegativities
		\end{itemize}
	
	\subsection{Ch. 3 - Structures of Metals/Ceramics}
		\subsubsection{Lattice Parameters}
			\begin{align}
				a_{\text{BCC}} &= \frac{4r}{\sqrt{3}} \\
				a_{\text{FCC}} &= \frac{4r}{\sqrt{2}} \\
				a_{\text{HCP}} &= \frac{c}{1.633} 
			\end{align}
			\begin{itemize}[noitemsep]
				\item $a$ - Lattice Parameter
				\item $r$ - Radius of atom
			\end{itemize}

		\subsubsection{Volume of Hexagonal Prism}
			\begin{equation}
				V_{H} = \frac{3 \sqrt{3}}{2} a^{2} h
			\end{equation}

		\subsubsection{Densities}
			\begin{equation}
				\rho = \frac{nA}{V_{C} N_{A}}
			\end{equation}
			\begin{itemize}[noitemsep]
				\item $n$ - Number of atoms/unit cell
				\item $A$ - Molar Mass of Material
				\item $V_{C}$ - Volume of Unit Cell in \si{\centi \meter \cubic}
				\item $N_{A}$ - Avogadro's Number ($6.022 \times 10^{23}$)
			\end{itemize}

			\begin{equation}
				\text{Planar Density} = \frac{\frac{\text{Atoms}}{\text{2D Unit Area}}}{\frac{\text{Area}}{\text{2D Repeat Unit}}}
			\end{equation}
			\begin{equation}
				\text{Linear Density} = \frac{\text{\# of Atoms in a Direction}}{\text{Magnitude of Linear Vector}}
			\end{equation}
			\begin{itemize}[noitemsep]
				\item The repeat units/vector magnitude are in terms of atomic radii
			\end{itemize}

			\begin{equation}
				\text{APF} = \frac{\frac{\text{Atoms}}{\text{Unit Cell}} \left( \frac{4}{3} \pi \left( \text{atom radius} \right)^{3} \right)}{\text{Unit Cell Volume}}
			\end{equation}	

		\subsubsection{Thermal Expansion}
			\begin{equation}
				\frac{\Delta L}{L_{0}} = \alpha \left( T_{2} - T_{1} \right)
			\end{equation}
			\begin{itemize}[noitemsep]
				\item $E \uparrow$, $T_{m} \uparrow$
				\item $E \uparrow$, $\alpha \downarrow$
			\end{itemize}

		\subsubsection{Convert between Coordinates}
			\begin{equation}
				\left[ XYZ \right] = \left[ a_{1} a_{2} a_{3} c \right]
				\begin{aligned}
					a_{1} &= \frac{1}{3} \left( 2X - Y \right) \\
					a_{2} &= \frac{1}{3} \left( 2Y - X \right) \\
					a_{3} &= - \left( a_{1} + a_{2} \right) \\
					c &= Z \\
					a_{1} + a_{2} + a_{3} &= 0 \\
				\end{aligned}
			\end{equation}
	
		\subsubsection{Planes}
			\begin{enumerate}[noitemsep]
				\item Given $x$, $y$, $z$ as intersects
				\item Convert to $\frac{1}{x}$, $\frac{1}{y}$, $\frac{1}{z}$
				\item Reduce to smallest common denominator
				\item Leave as $\left( \frac{1}{x} \frac{1}{y} \frac{1}{z} \right)$
			\end{enumerate}
	
		\subsubsection{Light Refraction}
			\begin{equation}
				D = \frac{n \lambda}{2 \sin \theta}
			\end{equation}
			\begin{itemize}[noitemsep]
				\item $n$ = 1
				\item $\lambda$ - Wavelength in \si{\nano \meter}
				\item $\theta$ - Angle of Incidence
				\item $\theta$ is usually given as $2 \theta$. Be careful
			\end{itemize}
	
		\subsubsection{Randoms}
			\begin{equation}
				D_{HKL} = \frac{a}{\sqrt{h^{2}+k^{2}+l^{2}}}
			\end{equation}
			\begin{itemize}[noitemsep]
				\item ONLY for cubic structures
			\end{itemize}

\section{Exam 2 Equations}
	\subsection{Ch. 4 Imperfections}
		\subsubsection{Vacancies}
		Number of Vacancies:
			\begin{equation}
				N_{V} = \text{Vacancy Frac.} \times \frac{N_{A} \rho_{\text{Ele}}}{A_{\text{Ele}}}
			\end{equation}
			\begin{itemize}[noitemsep]
				\item $A_{\text{Ele}}$ - Atomic Weight (\si{\gram / \mole})
				\item $\rho_{\text{Ele}}$ - Element Density (\si{\gram / \cubic \centi \meter})
				\item $N_{A}$ - Avogadro's Number ($6.022 \times 10^{23}$)
			\end{itemize}
		Number of Potential Vacancy Sites:
			\begin{equation}
				N = \frac{\rho_{\text{Ele}} N_{A}}{A_{\text{Ele}}}
			\end{equation}
			\begin{itemize}[noitemsep]
				\item $A_{\text{Ele}}$ - Atomic Weight (\si{\gram / \mole})
				\item $\rho_{\text{Ele}}$ - Element Density (\si{\gram / \cubic \centi \meter})
				\item $N_{A}$ - Avogadro's Number ($6.022 \times 10^{23}$)
			\end{itemize}
		Grains per Area:
			\begin{equation}
				n_{M} = \left( \frac{100}{M} \right)^{2} = 2^{G-1}
			\end{equation}
			\begin{itemize}[noitemsep]
				\item $n_{M}$ - Grains/in\textsuperscript{2}
				\item $G$ - Grain Size Number
				\begin{itemize}[noitemsep]
					\item $G = -6.6457 \log \left( \bar{\ell} \right) - 3.298$ for \si{\milli \meter}
					\item $G = -6.6353 \log \left( \bar{\ell} \right) - 12.6$ for in.
				\end{itemize}
			\end{itemize}
		
		\subsubsection{Mean Intercept Length}
			\begin{equation}
				\bar{\ell} = \frac{L_{T}}{P M}
			\end{equation}
			\begin{itemize}[noitemsep]
				\item $L_{T}$ - Total Length of all Lines
				\item $P$ - Number of Grain Intersections
				\item $M$ - Magnification
					\begin{equation}
						M = \frac{\text{Scale Length}}{\text{\# on Scale Bar}}
					\end{equation}
			\end{itemize}
		
		\subsubsection{Complete Substitution}
		To have a complete substitution, it must be:
			\begin{itemize}[noitemsep]
				\item $\Delta r < 15\%$
				\item Electronegativity $\leq .4$
				\item SAME Crystal Structure
				\item SAME Valence
			\end{itemize}
		
		\subsubsection{Edges}
		Burger's Vector:
			\begin{itemize}[noitemsep]
				\item Burger's Vector
				\begin{itemize}[noitemsep]
					\item $\perp$ for Edge Dislocations
					\item $\parallel$ for Screw Dislocations
				\end{itemize}
				\item Twin Boundary - Symmetric Around Fault
				\item Stacking Fault - NOT Symmetric Around Fault
				\begin{itemize}[noitemsep]
					\item High Angle - $E \uparrow$
					\item Low Angle - $E \downarrow$
				\end{itemize}
			\end{itemize}
		
	\subsection{Ch. 5 Diffusion}
		\subsubsection{Diffusion Coefficient}
			\begin{equation}
				D = D_{0} \times e^{\frac{-Q_{d}}{RT}}
			\end{equation}
			\begin{itemize}[noitemsep]
				\item $D$ - Diffusion Coefficient (\si{\meter \squared / \second})
				\item $Q_{d}$ - Activation Energy (\si{\joule / \mole}, \si{\electronvolt / atom})
				\item $R$ - 8.314 (\si{\joule / \mole \kelvin})
				\item $T$ - Temperature (\si{\kelvin})
			\end{itemize}
			\begin{equation}
				D_{1} t_{1} = D_{2} t_{2}
			\end{equation}
			\begin{itemize}[noitemsep]
				\item $D$ - Diffusion Coefficients
				\item $t$ - Time
			\end{itemize}
		
		\subsubsection{Flux}
			\begin{equation}
				J = -D A \frac{dC}{dx}
			\end{equation}
			\begin{itemize}[noitemsep]
				\item For Steady State Diffusion
				\item $D$ - Diffusion Coefficient
				\item $dC$ - $\Delta$ Concentration (Low-High)
				\item $dx$ - Distance to Cross
				\item $A$ - Area
			\end{itemize}
		
		\subsubsection{Diffusion Concentration}
			\begin{equation}
				\frac{C \left( x,t \right) - C_{0}}{C_{s}-C_{0}} = 1 - \text{erf} \left( \frac{x}{2 \sqrt{Dt}} \right)
			\end{equation}
			\begin{equation}
				Z = \frac{x}{2 \sqrt{Dt}} = \frac{z-\text{point below}}{\text{point above}-\text{point below}}
			\end{equation}
			\begin{itemize}[noitemsep]
				\item $C \left( x,t \right)$ - Concentration at a point AND time
				\item $C_{0}$  - Initial Concentration
				\item $C_{2}$ - Surface Concentration of \textsc{diffusing} species
				\item $x$ - Position
				\item $D$ - Diffusion Coefficient
				\item $t$ - Time
			\end{itemize}

\section{Exam 3 Equations} \label{sec:Exam 3 Equations}
	\subsection{Ch. 6 - Mechanical Properties} \label{subsec:Ch 6}
		\subsubsection{Stress} \label{subsubsec:Stress}
			\paragraph{Tensile Stress} \label{par:Tensile Stress}
			\begin{equation} \label{eq:Tensile Stress}
				\sigma = \frac{F_{t}}{A_{0}}
			\end{equation}
			\begin{itemize}[noitemsep]
				\item Think of amount of force required to pull ends of paper apart
				\item $\sigma$ has units $lb_{f}/in^{2}$ or \si{\newton / \meter \squared}
				\item $F_{t}$ - Normal Force
				\item $A_{0}$ - Original Cross-Sectional Area
				\item $\sigma < 0$ - Compressive Force
				\item $\sigma > 0$ - Tensile Force
			\end{itemize}
		
			\paragraph{Shear Stress} \label{par:Shear Stress}
				\begin{equation} \label{eq:Shear Stress}
					\tau = \frac{F_{S}}{A_{0}}
				\end{equation}
				\begin{itemize}[noitemsep]
					\item Think of amount of force required to rip a piece of paper
					\item $\tau$ has units $lb_{f}/in^{2}$ or \si{\newton / \meter \squared}
					\item $F_{S}$ - Shear Force
					\item $A_{0}$ - Original Cross-Sectional Area
				\end{itemize}
		
		\subsubsection{Strain} \label{subsubsec:Strain}
			\paragraph{Tensile Strain} \label{par:Tensile Strain}
				\begin{equation} \label{eq:Tensile Strain}
					\varepsilon = \frac{\delta}{L_{0}} = \frac{L - L_{0}}{L_{0}}
				\end{equation}
				\begin{itemize}[noitemsep]
					\item Think of pulling ends of paper apart and seeing how much it stretches
					\item $\varepsilon$ - \nameref{par:Tensile Strain}
					\item $\delta$ - Change in Length of Material
					\item $L_{0}$ - Original Length of Material
				\end{itemize}
		
			\paragraph{Shear Strain} \label{par:Shear Strain}
				\begin{equation} \label{eq:Shear Strain}
					\gamma = \frac{\Delta x}{y} = \tan \theta
				\end{equation}
				\begin{itemize}[noitemsep]
					\item Change of length of material compared to height when ripped apart
					\item $\gamma$ - \nameref{par:Shear Strain}
					\item $\Delta x$ - Change in length
					\item $y$ - Height of Material Tested
				\end{itemize}
		
		\subsubsection{Moduli}
			\paragraph{Young's Modulus} \label{par:Youngs Modulus}
				\begin{equation} \label{eq:Young's Modulus}
					E = \frac{\sigma}{\varepsilon} = \frac{dF}{dr}
				\end{equation}
				\begin{itemize}[noitemsep]
					\item How Stiff a Material is from being pulled apart
					\item $E$ - \nameref{par:Youngs Modulus}
					\item $\sigma$ - \nameref{par:Tensile Stress}
					\item $\varepsilon$ - \nameref{par:Tensile Strain}
				\end{itemize}
			
			\paragraph{Elastic Shear Modulus} \label{par:Elastic Shear Modulus}
				\begin{equation} \label{eq:Elastic Shear Modulus}
					G = \frac{\tau}{\gamma}
				\end{equation}
				\begin{itemize}[noitemsep]
					\item How Stiff a material is from Ripping
					\item $\tau$ - \nameref{par:Shear Stress}
					\item $\gamma$ - \nameref{par:Shear Strain}
				\end{itemize}
			
			\paragraph{Elastic Bulk Modulus} \label{par:Elastic Bulk Modulus}
				\begin{equation} \label{eq:Elastic Bulk Modulus}
					K = -P \frac{V_{0}}{\Delta V}
				\end{equation}
				\begin{itemize}[noitemsep]
					\item $K$ - \nameref{par:Elastic Bulk Modulus}
					\item $P$ - 
					\item $V_{0}$ - Original Volume
					\item $\Delta V$ - Change in Volume
				\end{itemize}
					
			\paragraph{Poisson's Ratio} \label{par:Poissons Ratio}
				\begin{equation}
					v = -\frac{\varepsilon_{L}}{\varepsilon}
				\end{equation}
				\begin{itemize}[noitemsep]
					\item $v$ - \nameref{par:Poissons Ratio}
					\item $\varepsilon_{L}$ - \nameref{par:Tensile Strain} at the length $L$
					\item $\varepsilon$ - \nameref{par:Tensile Strain}
				\end{itemize}
			
		\subsubsection{Isotropic Materials} \label{subsubsec:Isotropic Materials}
		If a material is isotropic, these equations apply to \nameref{par:Elastic Shear Modulus} and \nameref{par:Elastic Bulk Modulus}.
			\begin{equation} \label{eq:Isotropic Elastic Shear Modulus}
				G = \frac{E}{2 \left( 1+v \right)}
			\end{equation}
			\begin{equation} \label{eq:Isotropic Elastic Bulk Modulus}
				K = \frac{E}{3 \left( 1-2v \right)}
			\end{equation}
			\begin{itemize}[noitemsep]
				\item $G$ - Elastic Shear Modulus of isotropic material
				\item $K$ - Elastic Bulk Modulus of isotropic material
				\item $E$ - Young's Modulus of material
				\item $v$ - \nameref{par:Poissons Ratio} of Material
			\end{itemize}
			
		\subsubsection{Deflection} \label{subsubsec:Deflection}
			\begin{equation} \label{eq:Deflection}
				\delta = \frac{F L_{0}}{E A_{0}}
			\end{equation}
			\begin{itemize}[noitemsep]
				\item $F$ - Force Applied
				\item $L_{0}$ - Original Length of Material
				\item $E$ - \nameref{par:Youngs Modulus}
				\item $A_{0}$ - Original Cross-Sectional Area of Material
			\end{itemize}
			
			\paragraph{Simple Tension} \label{par:Simple Tension}
				\begin{equation} \label{eq:Simple Tension}
					\delta_{L} = -v \frac{F w_{0}}{E A_{0}}
				\end{equation}
				\begin{itemize}[noitemsep]
					\item $delta_{L}$ - \nameref{par:Deflection}
					\item $v$ - \nameref{par:Poissons Ratio}
					\item $F$ - Force Applied
					\item $w_{0}$ - Width of Thing applying the force
					\item $E$ - \nameref{par:Youngs Modulus}
					\item $A_{0}$ - Original Cross-Sectional Area of Material
				\end{itemize}
		
		\subsubsection{Simple Torsion} \label{subsubsec:Simple Torsion}
			\begin{equation} \label{eq:Simple Torsion}
				\alpha = \frac{2 M L_{0}}{\pi \left( r_{0} \right)^{4} G}
			\end{equation}
			\begin{itemize}[noitemsep]
				\item $\alpha$ - \nameref{subsubsec:Simple Torsion}
				\item $M$ - 
				\item $L_{0}$ - Original Length of Material
				\item $r_{0}$ - 
				\item $G$ - \nameref{par:Elastic Shear Modulus}
			\end{itemize}
			
		\subsubsection{Working with Stress Curve}
			\begin{itemize}[noitemsep]
				\item $\sigma_{y}$ = Yield Strength
				\item Tensile Strength = Max Height on Curve (Plastic Deformation)
				\item Toughness = Area Beneath the Stress Curve (Energy Absorbed)
				\item \nameref{par:Percent Elongation}
				\item \nameref{par:Percent Reduction Area}
				\item $U_{r} \cong \frac{1}{2} \sigma_{y} \varepsilon_{y}$ = Resilience (Energy Absorbed in Elastic Deformation)
				\item $\sigma_{T} = \frac{F}{A_{0}} = K \varepsilon_{T}^{n}$ = True Stress
				\item $\epsilon_{T} = \ln \left( \frac{L}{L_{0}} \right)$ = True Strain
				\item $\sigma_{\text{Working}} = \frac{\sigma_{y}}{N}$ = Safety Measure Measure
			\end{itemize}
			
			\paragraph{Percent Elongation (\%EL)} \label{par:Percent Elongation}
				\begin{equation} \label{eq:Percent Elongation}
					\%EL = \frac{L_{f} - L_{0}}{L_{0}} \times 100\%
				\end{equation}
				\begin{itemize}[noitemsep]
					\item $\%EL$ - Percent Elongation
					\item $L_{f}$ - Final Length of Material
					\item $L_{0}$ - Starting Length of Material
				\end{itemize}
			
			\paragraph{Percent Reduction in Area} \label{par:Percent Reduction Area}
				\begin{equation} \label{eq:Percent Reduction Area}
					\%RA = \frac{A_{0} - A_{f}}{A_{0}} \times 100\%
				\end{equation}
				\begin{itemize}[noitemsep]
					\item $\%RA$ - Percent Reduction in Area
					\item $A_{f}$ - Final Cross-Sectional Area of Material
					\item $A_{0}$ - Starting Cross-Sectional Area of Material
				\end{itemize}
		
		\subsubsection{Random Equations}
			\begin{equation}
				\text{HB} = \frac{2P}{\pi D \left( D - \sqrt{D^{2} - d^{2}} \right)}
			\end{equation}
			\begin{equation}
				\text{HV} = 1.854 \times \frac{P}{d^{2}}
			\end{equation}
			\begin{equation}
				\text{HK} = 14.2 \times \frac{P}{l^{2}}
			\end{equation}
			
	\subsection{Ch. 7 - Deformation \& Strengthening Mechanisms} \label{subsec:Ch 7}
		\subsubsection{Burger's Vector Explained} \label{subsubsec:Burgers Vector Explained}
			\begin{equation}
				\begin{aligned}
					\lVert \vec{b} \rVert &= \frac{a}{2} \sqrt{u^{2} + v^{2} + w^{2}} \\
					\vec{b}_{BCC} &= \frac{a}{2} \left[ 111 \right] \\
					\vec{b}_{FCC} &= \frac{a}{2} \left[ 110 \right] \\
					\vec{b}_{HCP} &= \frac{a}{2} \left[ 11 \bar{2} 0 \right] \\
				\end{aligned}
			\end{equation}
			
		\subsubsection{Angled Stresses} \label{subsubsec:Angled Stresses}
			\begin{equation} \label{eq:Angled Stresses}
				\phi / \lambda = \arccos \left( \frac{u_{1}u_{2} + v_{1}v_{2} + w_{1}w_{2}}{\sqrt{\left( u_{1}^{2} + v_{1}^{2} + w_{1}^{2} \right) \left( u_{2}^{2} + v_{2}^{2} + w_{2}^{2} \right)}} \right)
			\end{equation}
			\begin{itemize}[noitemsep]
				\item $\phi$ - Angle of stress to normal
				\item $\lambda$ - Angle of stress to slip direction
			\end{itemize}
		
		\subsubsection{Strengthening Mechanisms} \label{subsubsec:Strengthening Mechanisms}
			\begin{enumerate}[noitemsep]
				\item \nameref{par:Grain-Size Reduction} - Increase grain boundaries, increase misalignment
				
				\item \nameref{par:Solid-Solution} - Add interstitial atoms
				
				\item \nameref{par:Cold Work/Annealing} - Increase the number of dislocations
				
				\item \nameref{par:Precipitate Strengthening} - Shear precipitate or bend the slip line
			\end{enumerate}
		
			\paragraph{Grain-Size Reduction} \label{par:Grain-Size Reduction}
				\begin{equation} \label{eq:Grain-Size Reduction}
					\sigma_{y} = \sigma_{0} + kdy^{\frac{-1}{2}}
				\end{equation}
				\begin{itemize}[noitemsep]
					\item $\rho_{\text{disloc}} \uparrow = \sigma_{y} \uparrow$
					\item $\rho_{\text{disloc}} = \frac{\text{Total disloc. Length}}{\text{Unit Volume}}$
				\end{itemize}
			
			\paragraph{Solid-Solution} \label{par:Solid-Solution}
				\begin{equation} \label{eq:Solid-Solution}
					\sigma_{y} = c^{\frac{1}{2}}
				\end{equation}
				\begin{itemize}[noitemsep]
					\item $c$ - Impurity Concentration
				\end{itemize}
			
			\paragraph{Cold Work/Annealing} \label{par:Cold Work/Annealing}
				\begin{equation} \label{eq:Cold Work}
					\text{\%CW} = \frac{A_{0} - A}{A_{0}} \times 100
				\end{equation}
				
				\begin{equation} \label{eq:Cold Work/Annealing}
					d^{n} - d_{0}^{n} = kt
				\end{equation}
				\begin{itemize}[noitemsep]
					\item \Cref{eq:Cold Work/Annealing} only works if heat treatment occurs
					\item $n=2$, usually
				\end{itemize}
			
			\paragraph{Precipitation Strengthening} \label{par:Precipitate Strengthening}
				\begin{equation} \label{eq:Precipitate Strengthening}
					\sigma_{y} = \frac{1}{s}
				\end{equation}
				\begin{itemize}[noitemsep]
					\item $s$ - Space in pinning sites
				\end{itemize}

\section{Exam 4 Equations} \label{sec:Exam 4 Equations}
	\subsection{Ch. 9 - Phase Diagrams} \label{subsec:Ch9 Phase Diagrams}
		\begin{align} 
			W_{L} &= \frac{s}{R+S} \\
			W_{\alpha} &= \frac{R}{R+S}
		\end{align}
		\begin{itemize}[noitemsep]
			\item Hypo [Eutectic Composition] - Before
			\item Hyper [Eutectic Composition] - After
			\item Eutectic $\rightarrow$ $S \rightleftharpoons \alpha + \beta$, Solid $\rightleftharpoons$ 2 solids
			
			\item Eutectoid $\rightarrow$ $S \rightleftharpoons \alpha + \beta$, Solid $\rightleftharpoons$ 2 solids
			
			\item Peritectic $\rightarrow$ $S \rightleftharpoons \alpha + L$, Solid $\rightleftharpoons$ 1 Solid, 1 Liquid
			
			\item Proeutectoid Ferrite - Temp > Forming temp of eutectoid composition
			\item Eutectoid Ferrite - Temp < Forming temp of eutectoid composition
		\end{itemize}
	
	\subsection{Ch. 10 - Phase Transformations} \label{subsec:Ch10 Phase Transformations}
		\begin{equation}
			\begin{aligned}
				r^{*} &= \frac{-2 \gamma T_{m}}{\Delta H_{F} \Delta T} \\
				\Delta T &= \left( T_{m} - T \right) k \\
			\end{aligned}
		\end{equation}
		\begin{equation}
			\Delta G_{T} = 4 \pi r^{2} \gamma + \frac{4}{3} \pi r^{3} \left( \frac{\Delta H_{f} \left( T_{m} - T \right)}{T_{m}} \right)
		\end{equation}
		\begin{equation}
			\Delta G^{*} = \left( \frac{16 \pi \gamma^{3} T_{m}^{2}}{3 \Delta H_{f}^{2}} \right) \cdot \frac{1}{\left( T_{m} - T \right)^{2}}
		\end{equation}
		\begin{equation}
			y = 1-e^{-kt^{n}}
		\end{equation}
		\begin{equation}
			\begin{aligned}
				\ln \left( \ln \left( \frac{1}{1-y} \right) \right) = \ln \left( k \right) + n \ln \left( t \right)
				\text{Rate} &= \frac{1}{t_{.5}} \\
			\end{aligned}
		\end{equation}
		
		\begin{itemize}[noitemsep]
			\item Coarse Pearlite - Heat $\rightarrow$ Furnace Cool
			\item Fine Pearlite - Heat $\rightarrow$ Air Cool
			\item Spheroidite - Heat for a long time @ eutectoid Temp, then furnace cooled
			\item Martensite - Heat $\rightarrow$ Quench
			\item Tempered Martensite - Heat $\rightarrow$ Quench $\rightarrow$ Heat $\rightarrow$ Air cool
		\end{itemize}
		Ductility \emph{\textsc{increases}} as you go \emph{up} this list. \\
		Tensile Strength \emph{\textsc{increases}} as you go \emph{down} this list.

\section{Exam 5 Equations}
	\subsection{Ch. 12 - Mechanical Properties of Ceramics} \label{subsec:Ch12 Mechanical Properties of Ceramics}
	Overall, ceramics are categorized by:
		\begin{enumerate}[noitemsep]
			\item Ionic Bonds
			\item Few Slip Systems
			\item Dislocations \emph{\textbf{\textsc{cannot}}} move
		\end{enumerate}
	
		\subsubsection{Strength of Ceramics} \label{subsubsec:Ceramics Strength}
			\begin{equation} \label{eq:Ceramics Max Strength}
				\sigma_{\text{Max}} = 2 \sigma_{0} \sqrt{\frac{a}{\rho}}
			\end{equation}
			\begin{itemize}[noitemsep]
				\item $\sigma_{\text{Max}}$ - Maximum Strength
				\item $a$ - Half major axis length
				\item $\rho$ - Radius of Crack tip
			\end{itemize}
		
			\begin{equation} \label{eq:Ceramics Fracture Strength}
				\sigma_{\text{FS}} = A \sqrt{\frac{E \gamma}{a}}
			\end{equation}
			\begin{itemize}[noitemsep]
				\item $\sigma_{\text{FS}}$ - Fracture Strength
				\item $\gamma$ - Surface Energy
				\item $A$ - Constant
			\end{itemize}
			
			\begin{equation} \label{eq:Flexural Strength}
				\sigma_{\text{FXS}} = \frac{3 F_{f} L}{2 b d^{2}}
			\end{equation}
			\begin{equation}
				\sigma_{\text{FXS}} = \frac{F_{f} L}{\pi R^{3}}
			\end{equation}
			\begin{itemize}[noitemsep]
				\item $\sigma_{\text{FXS}}$ - Flexural Strength
			\end{itemize}
	
		\subsubsection{Elasticity of Ceramics} \label{subsubsec:Ceramics Elasticity}
			\begin{equation}
				E = \frac{F}{\delta} \cdot \frac{L^{3}}{4bd^{2}}
			\end{equation}
			\begin{equation}
				E = \frac{F}{\delta} \cdot \frac{L^{3}}{12 \pi R^{4}}
			\end{equation}
			\begin{itemize}[noitemsep]
				\item $\delta$ - Midpoint Deflection
			\end{itemize}
		
		
	\subsection{Ch. 15 - Mechanical Properties of Polymers} \label{subsec:Ch15 Mechanical Properties of Polymers}
		\begin{equation}
			\sigma_{\text{TS}} = \sigma_{\text{TS}\infty} - \frac{A}{\bar{M}_{n}}
		\end{equation}
		\subsubsection{Molecular Weight} \label{subsubsec:Polymer Molecular Weight}
			\begin{itemize}[noitemsep]
				\item $E$ - No relationship
				\item $\sigma_{\text{TS}}$ - Increases
				\item More carbon chain entanglement
			\end{itemize}
		
		\subsubsection{Degree of Crystallinity} \label{subsubsec:Polymer Degree of Crystallinity}
			\begin{itemize}[noitemsep]
				\item $E$ - Increases
				\item $\sigma_{\text{TS}}$ - Increases
				\item Stronger secondary bonds between carbon chains
			\end{itemize}
		
		\subsubsection{Deformation by Drawing} \label{subsubsec:Polymer Deformation by Drawing}
			\begin{itemize}[noitemsep]
				\item $E$ - Increases
				\item $\sigma_{\text{TS}}$ - Increases
				\item Carbon chains are aligning, effectively testing the Carbon-Carbon bonds
			\end{itemize}
		
			\paragraph{Annealing/Pre-Drawing} \label{par:Polymer Annealing/Pre-Drawing}
				\begin{itemize}[noitemsep]
					\item $E$ - Increases
					\item $\sigma_{\text{TS}}$ - Increases
				\end{itemize}
			
			\paragraph{Post-Drawing} \label{par:Polymer Post-Drawing}
				\begin{itemize}[noitemsep]
					\item $E$ - Decreases
					\item $\sigma_{\text{TS}}$ - Decreases
				\end{itemize}
	
	\subsection{Ch. 18 - Electrical Properties}
		\subsubsection{Resistivity/Conductance} \label{subsubsec:Resistivity/Conductance}
			\begin{equation} \label{eq:Resistivity}
				\rho = \frac{RA}{\ell}
			\end{equation}
			\begin{equation} \label{eq:Conductance}
				\sigma = \frac{1}{\rho} = \frac{\ell}{RA}
			\end{equation}
			\begin{itemize}[noitemsep]
				\item $\rho$ - Resistivity
				\item $\sigma$ - Conductance
				\item $R$ - Resistance
				\item $A$ - Cross-Sectional Area
				\item $\ell$ - Length
			\end{itemize}
			\begin{equation} \label{eq:Total Resistivity}
				\rho = \rho_{\text{Thermal}} + \rho_{\text{Impurity}} + \rho_{\text{Deform}}
			\end{equation}
		
		\subsubsection{Conductance with Doping} \label{subsubsec:Conductance with Doping}
			\begin{equation} \label{eq:Conductance with Doping}
				\sigma = n \lvert e \rvert \mu_{e} + p \lvert e \rvert \mu_{h}
			\end{equation}
			\begin{itemize}[noitemsep]
				\item $p=0$ In Metals
				\item If $p=n$ it is intrinsic
				\item $e$ = $1.602 \times 10^{-19} \si{\coulomb}$
			\end{itemize}
		
		\subsubsection{Electron Drift Velocity} \label{subsubsec:Electron Drift Velocity}
			\begin{equation} \label{eq:Electron Drift Velocity}
				v_{d} = \frac{\mu_{e}}{\mu_{h}} \xi
			\end{equation}
			\begin{equation} \label{eq:Time for Electron to Travel}
				t = \frac{\ell}{v_{d}}
			\end{equation}

%====================================APPENDIX====================================
\appendix
\counterwithin{equation}{section}

\subsection{Trigonometry} \label{app:Trig}
	\subsubsection{Trigonometric Formulas} \label{subsubsec:Trig Formulas}
		\begin{equation} \label{eq:Sin plus Sin with diff Angles}
			\sin \left( \alpha \right) + \sin \left( \beta \right) = 2 \sin \left( \frac{\alpha + \beta}{2} \right) \cos\left( \frac{\alpha - \beta}{2} \right)  
		\end{equation}
		\begin{equation} \label{eq:Cosine-Sine Product}
			\cos \left( \theta \right) \sin \left( \theta \right) = \frac{1}{2} \sin \left( 2 \theta \right)
		\end{equation}
	
	\subsubsection{Euler Equivalents of Trigonometric Functions} \label{subsubsec:Euler Equivalents}
		\begin{equation} \label{eq:Euler Sin}
			\sin \left( x \right) = \frac{e^{\imath x} + e^{-\imath x}}{2}
		\end{equation}
		\begin{equation} \label{eq:Euler Cos}
			\cos \left( x \right) = \frac{e^{\imath x} - e^{-\imath x}}{2 \imath}
		\end{equation}
		\begin{equation} \label{eq:Euler Sinh}
			\sinh \left( x \right) = \frac{e^{x} - e^{-x}}{2}
		\end{equation}
		\begin{equation} \label{eq:Euler Cosh}
			\cosh \left( x \right) = \frac{e^{x} + e^{-x}}{2}
		\end{equation}

\end{document}