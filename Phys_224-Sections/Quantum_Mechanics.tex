\section{Quantum Mechanics} \label{sec:Quantum Mech}
Quantum mechanics may seem strange or weird, but it really isn't.
	\begin{definition}[Hamiltonian] \label{def:Hamiltonian}
		The \emph{hamiltonian} is a function that contains all knowable information about a particle or system in classical mechanics.
		\begin{equation} \label{eq:Hamiltonian}
			\text{Hamiltonian} = K+U
		\end{equation}
		\begin{itemize}[noitemsep, nolistsep]
			\item $K$ is the Kinetic Energy of the particle of system of particles.
			\begin{itemize}[noitemsep, nolistsep]
				\item $K = \frac{1}{2} mv^{2}$
				\item This contains the velocity of the system, which includes direction.
			\end{itemize}
			\item $U$ is the Potential Energy of the particle or system of particles.
			\begin{itemize}[noitemsep, nolistsep]
				\item $U = mg \left( y_{2} - y_{1} \right)$
				\item Contains the force of the system, its location, etc.
			\end{itemize}
		\end{itemize}
	\end{definition}

	\subsection{Photoelectric Effect} \label{subsec:Photoelectric Effect}
		\begin{definition}[Photoelectric Effect] \label{def:Photoelectric Effect}
			Occurs when light shines on an object.
			The object will lose electrons $e^{-}$ if the light has a great enough frequency.
			This \emph{is not} a function of intensity or brightness, only frequency.
			\begin{equation} \label{eq:Photoelectric Effect}
				\begin{aligned}
					K_{\text{Max}} &= hf - \Phi \\
					K_{\text{Max}} = E - \Phi &\Rightarrow K_{\text{Max}} + \Phi = E \\
				\end{aligned}
			\end{equation}
			\begin{itemize}[noitemsep, nolistsep]
				\item $K_{\text{Max}} = e V_{\text{Stop}}$ is Kinetic Energy of released electrons (\si{\joule})
					\begin{itemize}[noitemsep, nolistsep]
						\item $e$ is charge of electron
						\item $V_{\text{Stop}}$ is the stopping potential
					\end{itemize}
				\item $h$ is Planck's Constant
				\item $f$ is Frequency of the incident wave (\si{\hertz})
				\item $\Phi$ is the Work Function (\si{\electronvolt})
			\end{itemize}
		\end{definition}
	The \nameref{eq:Kinetic Energy of Matter Wave} is given in Equation~\eqref{eq:Kinetic Energy of Matter Wave}.
		
	\subsection{Schr\"{o}dinger's Equation} \label{subsec:Schrodingers Equation}
		\begin{definition}[de Broglie Wavelength] \label{def:de Broglie Wavelength}
			The \emph{de Broglie wavelength, $\lambda$}, is the wavelength that is \emph{associated} with a particle.
			In other words, a particle does \emph{\textbf{not}} have a wavelength, but the equations describing the particle's actions have a wavelength.
			\begin{equation} \label{eq:de Broglie Wavelength}
				\lambda = \frac{h}{p}
			\end{equation}
			\begin{note} \label{note:de Broglie Wavelength Momentum}
				As a direction consequence of the \nameref{def:de Broglie Wavelength}, Equation~\eqref{eq:de Broglie Wavelength}:
				\begin{equation}
					p = \frac{h}{\lambda}
				\end{equation}
				\begin{itemize}[noitemsep, nolistsep]
					\item $h$ is Planck's Constant
					\item $\lambda$ is the \nameref{def:de Broglie Wavelength}
				\end{itemize}
			For electromagnetic waves, Equation~\eqref{eq:Momentum of Electromagnetic Wave} holds true.
				\begin{equation} \label{eq:Momentum of Electromagnetic Wave}
					p = \frac{E}{c}
				\end{equation}
			For matter waves, Equation~\eqref{eq:Kinetic Energy of Matter Wave} holds true.
				\begin{equation} \label{eq:Kinetic Energy of Matter Wave}
					K = \frac{p^{2}}{2m}
				\end{equation}
			\end{note}
		\end{definition}
		\begin{definition}[Schr\"{o}dinger's Equation] \label{def:Schrodinger's Equation}
			\emph{Schr\"{o}dinger's equation} is a 3-dimensional Cartesian, time-independent way to express the equivalent of a \nameref{def:Hamiltonian} for quantum effects, namely in terms of waves (Denoted with $\Psi$ or $\psi$).
			There are 2 equivalent ways to express \nameref{def:Schrodinger's Equation}.
			The first way, Equation~\eqref{eq:Schrodinger's Equation, Form 1}, is the most general form.
			The second way, Equation~\eqref{eq:Schrodinger's Equation, Form 2}, has expanded the nabla operator, $\nabla$.
			\begin{equation} \label{eq:Schrodinger's Equation, Form 1}
				\frac{- \hbar^{2}}{2m} \cdot \nabla^{2} \Psi + V\Psi = E\Psi
			\end{equation}
			\begin{equation} \label{eq:Schrodinger's Equation, Form 2}
				\frac{-\hbar}{2m} \left( \frac{\partial^{2} \Psi}{\partial x^{2}} + \frac{\partial^{2} \Psi}{\partial y^{2}} + \frac{\partial^{2} \Psi}{\partial z^{2}} \right) + V\Psi = E\Psi
			\end{equation}
			\begin{itemize}[noitemsep, nolistsep]
				\item $E$ is the \emph{\textbf{total}} energy of the system
				\item $V$ is the \emph{\textbf{potential energy}} of the system. It may be:
				\begin{enumerate}[noitemsep, nolistsep]
					\item Constant
					\item A function of time and/or space
				\end{enumerate}
				\item $\Psi$ is a solution to \nameref{def:Schrodinger's Equation}, and contains all that can be known about the system
				\item $\frac{-\hbar^{2}}{2m} \cdot \nabla\Psi$ is the \emph{\textbf{kinetic energy}} operator
				\begin{itemize}[noitemsep, nolistsep]
					\item $\nabla^{2} = \frac{\partial^{2}}{\partial x^{2}} + \frac{\partial^{2}}{\partial y^{2}} + \frac{\partial^{2}}{\partial z^{2}}$ is in the Cartesian coordinate system ($x$, $y$, $z$), and is called ``del squared'', or ``2nd-degree nabla.''
				\end{itemize}
				\item For each observable system, there exists a corresponding mathematical operator.
			\end{itemize}
			\begin{note} \label{note:Psi Probability Distribution Function of Particle Location}
				The probability distribution function of the position of a particle in the system can be found according to Equation~\eqref{eq:Psi Probability Distribution Function of Particle Location}.
				This is the multiplication of $\Psi$ with its complex conjugate, $\Psi^{*}$.
				\begin{equation} \label{eq:Psi Probability Distribution Function of Particle Location}
					\text{PDF} = \Psi \cdot \Psi^{*}
				\end{equation}
			\end{note}
		\end{definition}
		\subsubsection{Properties of Schr\"{o}dinger's Equation} \label{subsubsec:Properties of Schrodingers Equation}
			Solutions, $\Psi$, to \nameref{def:Schrodinger's Equation} have the following characteristics:
			\begin{enumerate}[noitemsep, nolistsep]
				\item They are single-valued
				\item They are piecewise continuous values
				\item $\frac{\partial \Psi}{\partial x}$, $\frac{\partial \Psi}{\partial y}$, $\frac{\partial \Psi}{\partial z}$ are all continuous
				\item $\iiint \Psi \Psi^{*} \partial x \partial y \partial z$ is square integrable, or $\iiint \Psi \Psi^{*} \partial x \partial y \partial z \neq \pm \infty$
			\end{enumerate}

		\begin{proof}[Rough Justification of \nameref{def:Schrodinger's Equation}] \label{proof:Schrodinger's Equation Justification}
		If $\Psi$ is to be a wave function, it must satisfy the following differential equation:
		\begin{equation*}
			\frac{\partial^{2} \Psi}{\partial x^{2}} = \frac{1}{v^{2}} \frac{\partial^{2} \Psi}{\partial t^{2}}
		\end{equation*}
		Since we are looking for \nameref{eq:Schrodinger's Equation, Form 1}, in which all the dynamical observables are contained within $\Psi$, we will set $v=c$, the upper limit of information transfer.
		We will first consider (without loss of generality) one dimension, the $x$-dimension.
		Then,
		\begin{equation*}
			\frac{\partial^{2} \Psi}{\partial x^{2}} = \frac{1}{v^{2}} \frac{\partial^{2} \Psi}{\partial t^{2}}
		\end{equation*}
		As a test solution, we will propose
		\begin{equation*}
			\Psi = A e^{-i \left( kx-\omega t \right)}
		\end{equation*}
		With respect to $x$:
		\begin{align*}
			\frac{\partial \Psi}{\partial x} &= A \left( -ik \right) e^{-i \left( kx-\omega t \right)} \\
			\frac{\partial^{2} \Psi}{\partial x^{2}} &= A \left( -ik \right)^{2} e^{-i \left( kx-\omega t \right)} = \left( -ik \right)^{2} \Psi = k^{2} \Psi \\
		\end{align*}
		Therefore, $\Psi$ is a satisfactory solution in $x$.
		With respect to $t$:
		\begin{align*}
			\frac{\partial \Psi}{\partial t} &= A \left( i\omega \right) e^{-i \left( kx-\omega t \right)} \\
			\frac{\partial^{2} \Psi}{\partial t^{2}} &= A \left( i\omega \right)^{2} e^{-i \left( kx-\omega t \right)} \\
		\end{align*}
		Therefore, $\Psi$ is a satisfactory solution in $t$.
		Thus, $\Psi$ is a solution to $\frac{\partial^{2} \Psi}{\partial x^{2}} = \frac{1}{c^{2}} \frac{\partial^{2} \Psi}{\partial t^{2}}$.
		(Also, $\frac{1}{c^{2}} = \frac{k^{2}}{\omega^{2}}$). \newline
		In our solution, $\Psi = A e^{-i \left( kx-\omega t \right)} $, k has its usual meaning:
		\begin{equation*} 
			k = \frac{2 \pi}{\lambda}
		\end{equation*}
		\emph{Where the wavelength, $\lambda$, is to be the \nameref{def:de Broglie Wavelength}, i.e. the wavelength that is associated with a particle.}
		In otherwords, a particle does not have a wavelength, but the equations describing the particle's action have a wavelength.
		\begin{align*}
			\frac{\partial \Psi}{\partial x} &= A \left( -ik \right) e^{-i \left( kx-\omega t \right)} \\
			&= A \left( -ik \right) e^{-i \left( \frac{2\pi}{\lambda}x- 2\pi ft \right)} \\
			&= A \left( -ik \right) e^{-i \frac{h}{h} \left( \frac{2\pi}{\lambda}x- 2\pi ft \right)} \\
			&= A \left( -ik \right) e^{-i \frac{2\pi}{h} \left( \frac{h}{\lambda}x- hft \right)} \\
		\end{align*}
		\emph{According to the \nameref{def:de Broglie Wavelength}, $p = \frac{h}{\lambda}$; according to Einstein, $E = hf$}; substituting:
		\begin{align*}
			\frac{\partial \Psi}{\partial x} &= -i \left( \frac{2 \pi}{h} \right) \left( \frac{h}{\lambda} \right) A e^{-i \frac{2\pi}{h} \left( \frac{h}{\lambda}x- hft \right)} \\ 
			\frac{\partial \Psi}{\partial x} &= -i \left( \frac{2 \pi}{h} \right) \left( p \right) A e^{-i \frac{2\pi}{h} \left( px- Et \right)} \\
		\end{align*}
		Define ($\hbar$ is pronounced ``h-bar''):
		\begin{equation*}
			\hbar = \frac{h}{2 \pi}
		\end{equation*}
		This leads us to:
		\begin{align*}
			\frac{\partial \Psi}{\partial x} &= \frac{i}{\hbar} \left( p \right) \Psi \\
			\frac{-\hbar}{i} \frac{\partial \Psi}{\partial x} &= p \Psi \\
		\end{align*}
		Here, it is clear that $p$ does not equal the classical mechanical $p = mv$. Rather, $p$ is an \emph{operator} such that:
		\begin{equation*}
			\hat{p} = \frac{\hbar}{i} \frac{\partial}{\partial x}
		\end{equation*}
		It is possible to construct other operators that correspond to observable dynamical variables in a similar way. (Operators are expressed with a ``hat'' above the variable name). \newline
		In classical mechanics, the kinetic energy of a particle can be written as:
		\begin{equation*}
			K = \frac{\vec{p} \cdot \vec{p}}{2m}
		\end{equation*}
		Since there exists a quantum mechanical operator that corresponds to any observable, we can write the kinetic energy operator for a one-dimensional system as:
		\begin{equation*}
			\widehat{K} = \frac{\hat{p} \cdot \hat{p}}{2m}
			= \frac{\frac{-\hbar}{i} \frac{\partial}{\partial x} \cdot \frac{-\hbar}{i} \frac{\partial}{\partial x}}{2m}
			= \frac{\hbar^{2}}{2m} \frac{\partial^{2}}{\partial x^{2}}
		\end{equation*}
		\begin{center}
			\textbf{\Large Extension to Three Dimensions}
		\end{center}
		Recall that the ``del-squared'' can be written as
		\begin{equation*}
			\nabla^{2} = \frac{\partial^{2}}{\partial x^{2}} + \frac{\partial^{2}}{\partial y^{2}} + \frac{\partial^{2}}{\partial z^{2}}
		\end{equation*}
		So, the kinetic energy part of the 3-dimensional Hamiltonian for a wave can be written as:
		\begin{equation*}
			\widehat{K} =  \frac{-\hbar}{2m} \left( \frac{\partial^{2}}{\partial x^{2}} + \frac{\partial^{2}}{\partial y^{2}} + \frac{\partial^{2}}{\partial z^{2}} \right)
		\end{equation*}
		Define $\Psi_{x,y,z} \left( x,y,z \right)$:
		\begin{equation*}
			\Psi_{x,y,z} \left( x,y,z \right) = \langle \Psi_{x} \left( x \right),\Psi_{y} \left( y \right),\Psi_{z} \left( z \right) \rangle
		\end{equation*}
		Therefore, 
		\begin{align*}
			\widehat{K}
			&= -\frac{\hbar^{2}}{2m} \left( \frac{\partial^{2}}{\partial x^{2}} + \frac{\partial^{2}}{\partial y^{2}}  + \frac{\partial^{2}}{\partial z^{2}} \right) \Psi_{x,y,z} \left( x,y,z \right) \\
			&= -\frac{\hbar^{2}}{2m} \left( \frac{\partial^{2}}{\partial x^{2}} + \frac{\partial^{2}}{\partial y^{2}}  + \frac{\partial^{2}}{\partial z^{2}} \right) \langle \Psi_{x} \left( x \right),\Psi_{y} \left( y \right),\Psi_{z} \left( z \right) \rangle \\
			&= -\frac{\hbar^{2}}{2m} \left[ \left( \frac{\partial^{2} \Psi_{x} \left( x \right)}{\partial x^{2}} \right) + \left(\frac{\partial^{2} \Psi_{y} \left( y \right)}{\partial y^{2}}\right) + \left(\frac{\partial^{2} \Psi_{z} \left( z \right)}{\partial z^{2}}\right) \right] \\
		\end{align*}
	\end{proof}
	
		{\Large \emph{\textbf{This means that the 3-Dimensional problem is broken down into 3 1-Dimensional problems!}} Each dimension can be approached exactly the same way as above.}
	
	\subsection{Operators} \label{subsec:Operators}
		\begin{definition}[Operator] \label{def:Operators}
			In quantum mechanics, for each observable dynamical variable, there exists a corresponding \emph{operator}, which, if applied to the solution of the \nameref{eq:Schrodinger's Equation, Form 1}, will provide a value for that dynamical variable.
			\nameref{def:Operators} do something to the function immediately after the operator.
			An operator is nothing other than shorthand for instruction to do something to a function or number. For example, some common operators are:
			\begin{enumerate}[noitemsep, nolistsep]
				\item $+$
				\item $-$
				\item $\times$
				\item $\div$
				\item $\frac{\partial^{2}}{\partial t^{2}}$
				\item $\int$
				\item $\sum$
				\item $\nabla^{2}$
			\end{enumerate}
			Operators have some vector-like properties, but they are not vectors.
			Further, they do not always always \emph{commute}.
			Part of the main confusion arising from the public press about quantum mechanics is as statement like ``\textellipsis $A \times B$ is equal to $B \times A$.''
			This is simply sensationalizing the fact that some operators have commutative properties.
			One example is:
			\begin{equation*}
				y \times \frac{\partial f \left( y \right)}{\partial y} \neq \frac{\partial \left[ x \times f \left( y \right) \right]}{\partial y}
			\end{equation*}
			For each classical observable, there exists a corresponding quantum mechanical operator.
			Some examples of simple quantum mechanical operators are:
			\begin{itemize}[noitemsep, nolistsep] \label{item:List of Quantum Mechanical Operators}
				\item Kinetic Energy: $\frac{-\hbar^{2}}{2m} \nabla^{2}$
				\item Momentum: $\frac{-\hbar}{i} \nabla$
				\item Potential Energy: $V$ is simply multiplication by $V$
				\item Angular Momentum: $L_{z} = \mathbf{r} \times \frac{-\hbar}{i} \nabla^{2}$
			\end{itemize}
			Further, some, but not all, operators are linear.
			Fortunately, the quantum mechanics kinetic energy operators, $\nabla^{2} = \frac{\partial^{2}}{\partial x^{2}} + \frac{\partial^{2}}{\partial y^{2}} + \frac{\partial^{2}}{\partial z^{2}}$ is a linear operator.
			This allows us to \emph{reduce a 3-dimensional problem into 3 1-dimensional problems.}
		\end{definition}
	
	\subsection{Eigenvalues} \label{subsec:Eigenvalues}
		\begin{definition}[Eigenvalue] \label{def:Eigenvalues}
			Let $\widehat{\mathbf{O}}$ be some quantum mechanical operator.
			If $O$ and $\Psi$ are \emph{eigenvalues} and eigenfunctions of $\widehat{\mathbf{O}} \Psi = o \Psi$, the result $o$ will always be a precise number.
			\emph{Eigenvalues} represent the value of precise, exact values of the dynamical variables corresponding to the eigenfunction of the operator.
		\end{definition}
	The \emph{\textbf{easy}} way to find the value of a dynamical variable $o$, given the corresponding operator, $\widehat{\mathbf{O}}$ is as follows.
		\begin{proof}[Solve for Eigenvalue] \label{proof:Solve for Eigenvalue}
	Let $\Psi$ be a normalized function, i.e.
	\begin{equation*}
		\iiint \Psi \Psi^{*} dx dy dz =1
	\end{equation*}
	Then,
	\begin{equation*}
		\widehat{\mathbf{O}} \Psi = o \Psi
	\end{equation*}
	Multiply both sides by $\Psi^{*}$, the complex conjugate of $\Psi$.
	\begin{equation*}
		\Psi^{*} \widehat{\mathbf{O}} \Psi = \Psi^{*} o \Psi
	\end{equation*}
	Integrate both sides over the space $\Omega$.
	\begin{equation*}
		\int \Psi^{*} \widehat{\mathbf{O}} \Psi d\Omega = \int \Psi^{*} o \Psi d\Omega
		= o \int Psi^{*} \Psi d\Omega = o \text{, since } int Psi^{*} \Psi d\Omega = 1
	\end{equation*}
\end{proof}
	
	\subsection{Expectation Values} \label{subsec:Expectation Values}
		\begin{definition}[Expectation Values] \label{def:Expectation Values}
			If $o$ and $\Psi$ are \textbf{not} eigenvalues and eigenfunctions of $\widehat{\mathbf{O}}$, e.g. $\widehat{\mathbf{O}} \Psi = z \phi$, where $z$ is some number, and $\phi$ is some function not equal to $\Psi$, the result $z$ will be the average of many possible observations. \newline
			This is usually written as $\langle z \rangle$ and is called an \emph{expectation value}. The expectation value is the expected average value of a number of measurements for a dynamical variable described by the operator $\widehat{\mathbf{O}}$.
		\end{definition}
	
	\subsection{Determine Dynamical Variables in Quantum Mechanics} \label{subsec:Determine Dynamical Variables in Quantum Mechanics}
		This section has an example where we solve for dynamical variables.
		Finding $\Psi$, the wave function that satisfies the \nameref{eq:Schrodinger's Equation, Form 1} for the one-dimensional infinite square well potential (particle in a box) is presented in the physics textbooks.
Now we wish to find values for dynamical observables.
Starting with the energy $E$.

Recall the \nameref{subsubsec:Properties of Schrodingers Equation}, Section~\ref{subsubsec:Properties of Schrodingers Equation}.
The solution for the \nameref{eq:Schrodinger's Equation, Form 1} in a box of length $L$ is
\begin{equation*}
	\Psi_{n} = \sqrt{\frac{2}{L}} \sin \left( \frac{n \pi x}{L} \right) \text{, where $n$ is the quantum number of the wave function}
\end{equation*}
First, we operate on $\Psi_{n}$ with the kinetic energy operator $\widehat{K}$:
\begin{equation*}
	\widehat{K} \Psi_{n} = E \Psi_{n} = \varepsilon_{n} \Psi_{n}
\end{equation*}
$\varepsilon_{n}$ is the kinetic energy since $E$; the total energy is all kinetic with quantized values $varepsilon_{n}$. \newline
Rewrite $\widehat{K} \Psi_{n}$.
\begin{align*}
	\widehat{K} \Psi_{n} &= -\frac{\hbar^{2}}{2m} \frac{d^{2} \Psi_{n}}{dx^{2}} = \varepsilon_{n} \Psi_{n} \\
	&= -\frac{\hbar^{2}}{2m} \frac{d^{2}}{dx^{2}} \left[ \sqrt{\frac{2}{L}} \sin \left( \frac{n \pi x}{L} \right) \right] = \varepsilon_{n} \Psi_{n} \\
\end{align*}
Multiply both sides by $\Psi_{n}^{*}$, the complex conjugate of $\Psi_{n}$ (which in this case $\Psi_{n}^{*} = \Psi_{n}$ since $\Psi_{n}$ is real) \emph{\textbf{before the operator}} (since operators may not commute nor be distributive):
\begin{align*}
	\Psi_{n}^{*} \widehat{K} \Psi_{n} &= \Psi_{n}^{*} \left( -\frac{\hbar^{2}}{2m} \cdot \frac{d^{2} \Psi_{n}}{dx^{2}} \right) = \Psi_{n}^{*} \varepsilon_{n} \Psi_{n} \\
	&= -\sqrt{\frac{2}{L}} \sin \left( \frac{n \pi x}{L} \right) \cdot \left( -\frac{\hbar^{2}}{2m} \right) \frac{d^2}{dx^{2}} \left[ \sqrt{\frac{2}{L}} \sin \left( \frac{n \pi x}{L} \right) \right] \\
	&= \varepsilon_{n} \sqrt{\frac{2}{L}} \left( \sin \left( \frac{n \pi x}{L} \right) \right) \frac{n^{2} \pi^{2}}{L^{2}} \sqrt{\frac{2}{L}} \sin \left( \frac{n \pi x}{L} \right) \\
\end{align*}
Now integrate over both sides:
\begin{align*}
	\int \Psi_{n} \left( -\frac{\hbar^{2}}{2m} \frac{d^{2} \Psi_{n}}{dx^{2}} \right) dx 
	&= \int \Psi_{n}^{*} \varepsilon_{n} \Psi_{n} dx \\
	&= \int \sqrt{\frac{2}{L}} \sin \left( \frac{n \pi x}{L} \right) \cdot \left( -\frac{\hbar^{2}}{2m} \frac{d^{2}}{dx^{2}} \left[ \sqrt{\frac{2}{L}} \sin \left( \frac{n \pi x}{L} \right) \right] \right) dx \\
\end{align*}
Take the second derivative of $\Psi_{n}$ with respect to $x$:
\begin{equation*}
	\frac{d^{2}}{dx^{2}} \left[ \sqrt{\frac{2}{L}} \sin \left( \frac{n \pi x}{L} \right) = \frac{n^{2} \pi^{2}}{L^{2}} \sqrt{\frac{2}{L}} \sin \left( \frac{n \pi x}{L} \right) \right]
\end{equation*}
Substitute into the integral
\begin{equation*}
	\int \sqrt{\frac{2}{L}} \sin \left( \frac{n \pi x}{L} \right) \left( -\frac{\hbar^{2}}{2m} \right) \left( -\frac{n^{2} \pi^{2}}{L^{2}} \right) \sqrt{\frac{2}{L}} \sin \left( \frac{n \pi x}{L} \right) dx
	= \int \sqrt{\frac{2}{L}} \sin \left( \frac{n \pi x}{L} \right) \varepsilon_{n} \sqrt{\frac{2}{L}} \sin \left( \frac{n \pi x}{L} \right) dx
\end{equation*}
Pull out the constants
\begin{equation*}
	\left( -\frac{\hbar^{2}}{2m} \right) \left( -\frac{n^{2} \pi^{2}}{L^{2}} \right) \int \sqrt{\frac{2}{L}} \sin \left( \frac{n \pi x}{L} \right) \sqrt{\frac{2}{L}} \sin \left( \frac{n \pi x}{L} \right) dx
	= \varepsilon_{n} \int \sqrt{\frac{2}{L}} \sin \left( \frac{n \pi x}{L} \right) \sqrt{\frac{2}{L}} \sin \left( \frac{n \pi x}{L} \right) dx
\end{equation*}
Recall that 
\begin{equation*}
	\int \sqrt{\frac{2}{L}} \sin \left( \frac{n \pi x}{L} \right) \sqrt{\frac{2}{L}} \sin \left( \frac{n \pi x}{L} \right) dx = 1
\end{equation*}
So, $\left( \frac{\hbar^{2}}{2m} \right) \left( \frac{n^{2} \pi^{2}}{L^{2}} \right) = \varepsilon_{n}$ is the kinetic energy of a particle with $n$ as its quantum number.
		
	\subsection{Heisenberg's Uncertainty Principle} \label{subsec:Heisenberg's Uncertainty Principle}
		\begin{definition} \label{def:Heisenberg's Uncertainty Principle}
			The \nameref{def:Heisenberg's Uncertainty Principle} states that measured values cannot be assigned to the position vector, $\vec{r}$ and the momentum vector, $\vec{p}$ simultaneously with unlimited precision.
			\begin{equation} \label{eq:Heisenberg's Uncertainty Principle}
				\begin{aligned}
					\delta x \cdot \delta p_{x} &\geq \hbar \\
					\delta y \cdot \delta p_{y} &\geq \hbar \\
					\delta z \cdot \delta p_{z} &\geq \hbar \\
				\end{aligned}
			\end{equation}
			\begin{itemize}[noitemsep, nolistsep]
				\item $\delta x$ and $\delta p_{x}$ represent the uncertainty of the value from the given values of $\vec{r}_{x}x$ and $\vec{p}_{x}$.
				\item $\hbar$ is defined in Equation~\eqref{eq:hbar Definition}.
			\end{itemize}
		\end{definition}