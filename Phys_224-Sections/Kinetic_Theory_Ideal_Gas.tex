\section{Kinetic Theory of Ideal Gases} \label{sec:Kinetic Theory of Ideal Gases}
	\begin{definition}[Ideal Gas] \label{def:Ideal Gas}
		An \emph{ideal gas} is a gas that obeys the ideal gas law.
		\begin{equation} \label{eq:Ideal Gas Law}
			pV = nRT \text{, } R \approxeq 8.31 \si{\joule / \mole~\kelvin}
		\end{equation}
	\end{definition}
	\begin{align} \label{eq:Total Internal Energy of Ideal Gas}
		E_{Internal} &= K_{Translate} + K_{Rotate} \\
		\Delta E_{Internal} &= \Delta K_{Translate} + \Delta K_{Rotate} 
	\end{align}
	\begin{equation} \label{eq:Vrms of Ideal Gas}
		v_{rms} = \sqrt{\frac{3RT}{M}} \text{ M=molar mass of gas}
	\end{equation}
	
	\subsection{Mean Free Path} \label{subsec:Mean Free Path of Ideal Gas}
		\begin{definition}[Mean Free Path]
			The \emph{mean free path}, $\lambda$ is the average distance traversed by a molecule between collisions.
			\begin{equation} \label{eq:Mean Free Path of Ideal Gas}
				\lambda = \frac{1}{\sqrt{2} \pi d^{2} \frac{N}{V}}
			\end{equation}
			Where:
			\begin{itemize}[noitemsep, nolistsep]
				\item $d$ is the diameter of the atoms, or distance between centers during collision (\si{\meter})
				\item $N$ is the number of molecules (\si{\mole})
				\item $V$ is the volume of gas being handled (\si{\liter})
				\item $\lambda$ is the wavelength, the same as in $v=f\lambda$
			\end{itemize}
			\begin{note}
				If all the particles, except 1 are stationary, then you can use:
				\begin{equation} \label{eq:Mean Free Path of Stationary Ideal Gas}
					\lambda = \frac{1}{\pi d^{2} \frac{N}{V}}
				\end{equation}
			\end{note}
		\end{definition}
	
	\subsection{Work Done by Ideal Gases} \label{subsec:Work Done By Ideal Gas}
		\subsubsection{Isothermally} \label{subsubsec:Work Done Isothermally}
			\begin{equation}
				\begin{aligned}
					W &= \int \vec{F} \vec{ds} \\
					W &= \int_{V_{1}}^{V_{2}} p \left( V,T \right) dV \\
					W &= \int_{V_{1}}^{V_{2}} \frac{nRT}{V} dV \\
					W &= nRT \int_{V_{1}}^{V_{2}} \frac{1}{V} dV \\
					W &= nRT \ln \left( \frac{V_{2}}{V_{1}} \right) \\
				\end{aligned}
			\end{equation}
		\subsubsection*{Constant Pressure} \label{subsubsec:Work Done Under Constant Pressure}
			\begin{equation}
				W = p \left(V_{final}-V_{init} \right)
			\end{equation}
			
	\subsection{Translational Kinetic Energy} \label{subsec:Translational Kinetic Energy}
		\begin{definition}[Degrees of Freedom] \label{def:Degrees of Freedom}
			\emph{Degrees of freedom} represent the number of variables that are needed to describe a system. Represented with $d$, occasionally.
			\begin{note}
				In an ideal gas, these are a means to store energy.
			\end{note}
		\end{definition}
		\begin{equation} \label{eq:Translational Kinetic Energy}
			\begin{aligned}
				K_{Translate} &= \frac{3}{2}nRT \\
				\Delta K_{Translate} &= \frac{3}{2}nR \Delta T \\
			\end{aligned}
		\end{equation}
		\begin{table}[h!]
			\centering
			\begin{tabular}{c|c|c|c}
				 & Translational & Rotational & Total \\ \hline
				Monatomic & 3 & 0 & 3 \\ \hline
				Diatomic & 3 & 2 & 5 \\ \hline
				Polyatomic & 3 & 3 & 6 \\
			\end{tabular}
			\caption{Degrees of Freedom Table for Gases}
			\label{tab:Degrees of Freedom}
		\end{table}
		\begin{itemize}[noitemsep, nolistsep]
			\item An ideal gas has \emph{\textbf{ONLY}} kinetic energy
			\item Completely elastic collisions
		\end{itemize}
		
		\begin{equation} \label{eq:Internal Energy of Ideal Gas}
			E_{int} = \frac{DoF}{2} nRT \text{, where } DoF=\text{Degrees of Freedom}
		\end{equation}
		
	\subsection{Molar Specific Heats of Ideal Gases} \label{subsec:Molar Specific Heats of Ideal Gases}
		\subsubsection{Molar Specific Heat @ Constant Volume} \label{subsubsec:Molar Specific Heat at Constant Volume}
			\begin{equation} \label{eq:Molar Specific Heat at Constant Volume}
				\begin{aligned}
					C_{V} &= \frac{\Delta E}{n \Delta T} \\
					C_{V} &= \frac{dE}{dT} \\
					C_{V} &= \left( \frac{DoF}{2} \right) R \\
				\end{aligned}
			\end{equation}
			\begin{equation} \label{eq:Heat with Molar Specific Heat at Constant Volume}
				Q = n C_{V} \Delta T
			\end{equation}
			
		\subsubsection{Molar Specific Heat @ Constant Pressure} \label{Molar Specific Heat at Constant Pressure}
			\begin{equation} \label{eq:Molar Specific Heat at Constant Pressure}
				C_{P} = C_{V} + R
			\end{equation}
			\begin{equation} \label{eq:Heat with Molar Specific Heat at Constant Pressure}
			Q = n C_{P} \Delta T
			\end{equation}
	
	\subsection{Adiabatic Processes in Ideal Gases} \label{subsec:Adiabatic Processes in Ideal Gases}
		\begin{definition}[Adiabatic Process] \label{def:Adiabatic Processes in Ideal Gases}
			An \emph{adiabatic process} is one in which no heat exchange occurs, namely:
			\begin{equation} \label{eq:Adiabatic Processes in Ideal Gases}
				\begin{aligned}
					dE_{Internal} &= dQ - dW \\
					dQ &= 0 \\
					dE_{Internal} &= -dW \\
				\end{aligned}
			\end{equation}
		\end{definition}
		This leads to:
		\begin{equation}
			pV^{\gamma} = \text{Constant}, \gamma = \frac{C_{P}}{C_{V}}
		\end{equation}
		\begin{equation} \label{eq:Adiabatic Pressure Volume}
			p_{1}V_{1}^{\gamma} = p_{2}V_{2}^{\gamma}
		\end{equation}
		\begin{equation} \label{eq:Adiabatic Temperature Volume}
			T_{1}V_{1}^{\gamma -1} = T_{2}V_{2}^{\gamma -1} 
		\end{equation}
		You get Equation~\eqref{eq:Adiabatic Temperature Volume} by plugging $p = \frac{nRT}{V}$ into Equation~\eqref{eq:Adiabatic Pressure Volume}.