\section{Linguistics} \label{sec:Linguistics}
\begin{definition}[Linguistics] \label{def:Linguistics}
  \emph{Linguistics} is the study and the description of human languages.
  Linguistics have been developed since ancient times and the Middle Ages.
  \emph{Modern Linguistics} developed between the end of the 19\textsuperscript{th} century and beginning of the 20\textsuperscript{th} century.
  It founder and prominent figure was Ferdinand de Saussure.
\end{definition}

There is something else that we will be working with, called \nameref{def:Computational Linguistics}.
\begin{definition}[Computational Linguistics] \label{def:Computational Linguistics}
  \emph{Computational Linguistics} is a subset of both \nameref{def:Linguistics} and computer science.
  Its goal is to design mathematical models of language structures enabling the automation of language processing by a computer.
  We can consider computational linguistics as the formalization of linguistic theories and models, or their implementation in a machine.
  New linguistic theories can be developed with the aid of a computer too.
\end{definition}

Historically, there are 3 disciplines of \nameref{sec:Linguistics}.
\begin{enumerate}[noitemsep, nolistsep]
  \item \nameref{subsec:Phonetics}
  \item \nameref{subsec:Words}
  \item \nameref{subsec:Syntax}
\end{enumerate}

\subsection{Phonetics} \label{subsec:Phonetics}
\begin{definition}[Phonetics] \label{def:Phonetics}
  \emph{Phonetics} concerns the production and perception of acoustic sounds that form the speech signal.
  In every language sounds can be classified into a finite set of \nameref{def:Phonemes}
\end{definition}

\begin{definition}[Phonemes] \label{def:Phonemes}
  \emph{Phonemes} are the building blocks of \nameref{def:Phonetics}.
  Traditionally, \nameref{def:Phonemes} include \nameref{def:Vowels} and \nameref{def:Consonants}.
  \nameref{def:Phonemes} are assembled into \nameref{def:Syllables} to build words.
  
  Examples include: \emph{pa, pi, po}.
\end{definition}

\begin{definition}[Vowels] \label{def:Vowels}
  \emph{Vowels} are a speech sound that is produced by a comparatively open configuration of the vocal tract, with vibration of the vocal cords, but without audible friction.
  They are a unit of the sound system of a language that forms the nucleus of \nameref{def:Syllables}.
  
  Examples include: \emph{a, e, i, o}.
\end{definition}

\begin{definition}[Consonants] \label{def:Consonants}
  \emph{Consonants} are a speech sound in which the breath is at least partly obstructed and which can be combined with \nameref{def:Vowels} to form \nameref{def:Syllables}.
  
  Examples include: \emph{p, f, r, m}.
\end{definition}

\begin{definition}[Syllables] \label{def:Syllables}
  \emph{Syllables} are a unit of pronunciation having one vowel sound, with or without surrounding consonants, forming the whole or part of a word.
\end{definition}

\subsection{Words} \label{subsec:Words}
\begin{definition}[Words] \label{def:Words}
  \emph{Words}
\end{definition}

\subsection{Syntax} \label{subsec:Syntax}
\begin{definition}[Syntax] \label{def:Syntax}
  \emph{Syntax}
\end{definition}