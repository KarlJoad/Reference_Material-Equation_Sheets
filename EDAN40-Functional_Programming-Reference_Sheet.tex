\documentclass[10pt,letterpaper,final,twoside,notitlepage]{article}
\usepackage[margin=.5in]{geometry} % 1/2 inch margins on all pages
\usepackage[utf8]{inputenc} % Define the input encoding
\usepackage[USenglish]{babel} % Define language used
\usepackage{amsmath,amsfonts,amssymb}
\usepackage{amsthm} % Gives us plain, definition, and remark to use in \theoremstyle{style}
\usepackage{mathtools} % Allow for text and math in align* environment.
\usepackage{thmtools}
\usepackage{thm-restate}
\usepackage{graphicx}

\usepackage[
backend=biber,
style=alphabetic,
citestyle=authoryear]{biblatex} % Must include citation somewhere in document to print bibliography
\usepackage{hyperref} % Generate hyperlinks to referenced items
\usepackage[nottoc]{tocbibind} % Prints the Reference/Bibliography in TOC as well
\usepackage[noabbrev,nameinlink]{cleveref} % Fancy cross-references in the document everywhere
\usepackage{nameref} % Can make references by name to places
\usepackage{caption} % Allows for greater control over captions in figure, algorithm, table, etc. environments
\usepackage{subcaption} % Allows for multiple figures in one Figure environment
\usepackage[binary-units=true]{siunitx} % Gives us ways to typeset units for stuff
\usepackage{csquotes} % Context-sensitive quotation facilities
\usepackage{enumitem} % Provides [noitemsep, nolistsep] for more compact lists
\usepackage{chngcntr} % Allows us to tamper with the counter a little more
\usepackage{empheq} % Allow boxing of equations in special math environments
\usepackage[x11names]{xcolor} % Gives access to coloring text in environments or just text, MUST be before tikz
\usepackage{tcolorbox} % Allows us to create boxes of various types for examples
\usepackage{tikz} % Allows us to create TikZ and PGF Pictures
\usepackage{ctable} % Greater control over tables and how they look
\usepackage{diagbox} % Allow us to have shared diagonal cells in tables
\usepackage{multirow} % Allow us to have a single cell in a table span multiple rows
\usepackage{titling} % Put document information throughout the document programmatically
\usepackage[linesnumbered,ruled,vlined]{algorithm2e} % Allows us to write algorithms in a nice style.

\counterwithin{figure}{section}
\counterwithin{table}{section}
\counterwithin{equation}{section}
\counterwithin{algocf}{section}
\crefname{algocf}{algorithm}{algorithms}
\Crefname{algocf}{Algorithm}{Algorithms}
\setcounter{secnumdepth}{4}
\setcounter{tocdepth}{4} % Include \paragraph in toc
\crefname{paragraph}{paragraph}{paragraphs}
\Crefname{paragraph}{Paragraph}{Paragraphs}

% Create a theorem environment
\theoremstyle{plain}
\newtheorem{theorem}{Theorem}[section]
% Create a numbered theorem-like environment for lemmas
\newtheorem{lemma}{Lemma}[theorem]

% Create a definition environment
\theoremstyle{definition}
\newtheorem{definition}{Defn}
\newtheorem{corollary}{Corollary}[section]
% \begin{definition}[Term] \label{def:}
%   Make sure the term is emphasized with \emph{term}.
%   This ensures that if \emph is changed, it shows up everywhere
% \end{definition}

% Create a numbered remark environment numbered based on definition
% NOTE: This version of remark MUST go inside a definition environment
\theoremstyle{remark}
\newtheorem{remark}{Remark}[definition]
%\counterwithin{definition}{subsection} % Uncomment to have definitions use section.subsection numbering

% Create an unnumbered remark environment for general use
% NOTE: This version of remark has NO restrictions on placement
\newtheorem*{remark*}{Remark}

% Create a special list that handles properties. It can be broken and restarted
\newlist{propertylist}{enumerate}{1} % {Name}{Template}{Max Depth}
% [newlistname, LevelsToApplyTo]{formatting options}
\setlist[propertylist, 1]{label=\textbf{(\roman*)}, ref=\textbf{(\roman*)}, noitemsep, nolistsep}
\crefname{propertylisti}{property}{properties}
\Crefname{propertylisti}{Property}{Properties}

% Create a special list that handles enumerate starting with lower letters. Breakable/Restartable.
\newlist{boldalphlist}{enumerate}{1} % {Name}{Template}{Max Depth}
% [newlistname, LevelsToApplyTo]{formatting options}
\setlist[boldalphlist, 1]{label=\textbf{(\alph*)}, ref=\alph*, noitemsep, nolistsep} % Set options

\newlist{nocrefenumerate}{enumerate}{1} % {Name}{Template}{Max Depth}
% [newlistname, LevelsToApplyTo]{formatting options}
\setlist[nocrefenumerate, 1]{label=(\arabic*), ref=(\arabic*), noitemsep, nolistsep}

% Create a list that allows for deeper nesting of numbers. By default enumerate only allows depth=4.
\newlist{nestednums}{enumerate}{6}
% [newlistname, LevelsToApplyTo]{formatting options}
\setlist[nestednums]{noitemsep, label*=\arabic*.}

\tcbuselibrary{breakable} % Allow tcolorboxes to be broken across pages
% Create a tcolorbox for examples
% /begin{example}[extra name]{NAME}
% Create a tcolorbox for examples
% Argument #1 is optional, given by [], that is the textbook's problem number
% Argument #2 is mandatory, given by {}, that is the title for the example
% Avoid putting special characters, (), [], {}, ",", etc. in the title.
\newtcolorbox[auto counter,
number within=section,
number format=\arabic,
crefname={example}{examples}, % Define reference format for cref (No Capitals)
Crefname={Example}{Examples}, % Reference format for cleveref (With Capitals)
]{example}[2][]{ % The [2][] Means the first argument is optional
  width=\textwidth,
  title={Example \thetcbcounter: #2. #1}, % Parentheses and commas are not well supported
  fonttitle=\bfseries,
  label={ex:#2},
  nameref=#2,
  colbacktitle=white!100!black,
  coltitle=black!100!white,
  colback=white!100!black,
  upperbox=visible,
  lowerbox=visible,
  sharp corners=all,
  breakable
}

% Create a tcolorbox for general use
\newtcolorbox[% auto counter,
% number within=section,
% number format=\arabic,
% crefname={example}{examples}, % Define reference format for cref (No Capitals)
% Crefname={Example}{Examples}, % Reference format for cleveref (With Capitals)
]{blackbox}{
  width=\textwidth,
  % title={},
  fonttitle=\bfseries,
  % label={},
  % nameref=,
  colbacktitle=white!100!black,
  coltitle=black!100!white,
  colback=white!100!black,
  upperbox=visible,
  lowerbox=visible,
  sharp corners=all
}

% Redefine the 'end of proof' symbol to be a black square, not blank
\renewcommand{\qedsymbol}{$\blacksquare$} % Change proofs to have black square at end

% Common Mathematical Stuff
\newcommand{\Abs}[1]{\ensuremath{\lvert #1 \rvert}}
\newcommand{\DNE}{\ensuremath{\mathrm{DNE}}} % Used when limit of function Does Not Exist

% Complex Numbers functions
\renewcommand{\Re}{\operatorname{Re}} % Redefine to use the command, but not the fraktur version
\renewcommand{\Im}{\operatorname{Im}} % Redefine to use the command, but not the fraktur version
\newcommand{\Real}[1]{\ensuremath{\Re \lbrace #1 \rbrace}}
\newcommand{\Imag}[1]{\ensuremath{\Im \lbrace #1 \rbrace}}
\newcommand{\Conjugate}[1]{\ensuremath{\overline{#1}}}
\newcommand{\Modulus}[1]{\ensuremath{\lvert #1 \rvert}}
\DeclareMathOperator{\PrincipalArg}{\ensuremath{Arg}}

% Math Operators that are useful to abstract the written math away to one spot
% Number Sets
\DeclareMathOperator{\RealNumbers}{\ensuremath{\mathbb{R}}}
\DeclareMathOperator{\AllIntegers}{\ensuremath{\mathbb{Z}}}
\DeclareMathOperator{\PositiveInts}{\ensuremath{\mathbb{Z}^{+}}}
\DeclareMathOperator{\NegativeInts}{\ensuremath{\mathbb{Z}^{-}}}
\DeclareMathOperator{\NaturalNumbers}{\ensuremath{\mathbb{N}}}
\DeclareMathOperator{\ComplexNumbers}{\ensuremath{\mathbb{C}}}
\DeclareMathOperator{\RationalNumbers}{\ensuremath{\mathbb{Q}}}

% Calculus operators
\DeclareMathOperator*{\argmax}{argmax} % Thin Space and subscripts are UNDER in display

% Signal and System Functions
\DeclareMathOperator{\UnitStep}{\mathcal{U}}
\DeclareMathOperator{\sinc}{sinc} % sinc(x) = (sin(pi x)/(pi x))

% Transformations
\DeclareMathOperator{\Lapl}{\mathcal{L}} % Declare a Laplace symbol to be used

% Logical Operators
\DeclareMathOperator{\XOR}{\oplus}

% x86 CPU Registers
\newcommand{\rbpRegister}{\texttt{\%rbp}}
\newcommand{\rspRegister}{\texttt{\%rsp}}
\newcommand{\ripRegister}{\texttt{\%rip}}
\newcommand{\raxRegister}{\texttt{\%rax}}
\newcommand{\rbxRegister}{\texttt{\%rbx}}

%%% Local Variables:
%%% mode: latex
%%% TeX-master: shared
%%% End:


% These packages are more specific to certain documents, but will be availabe in the template
% \usepackage{esint} % Provides us with more types of integral symbols to use
% \usepackage[outputdir=./TeX_Output]{minted} % Allow us to nicely typeset 300+ programming languages
% This document must be compiled with the -shell-escape flag if the packages above are uncommented

% \graphicspath{{./Drawings/Course}} % Uncomment this to use pictures in this document
\addbibresource{./Bibliographies/EDAN40-Functional_Programming.bib}

% Math Operators that are useful to abstract the written math away to one spot
% These are supposed to be document-specific mathematical operators that will make life easier
% Many fundamental operators are defined in Reference_Sheet_Preamble.tex

\begin{titlepage}
  \title{EDAN40/EDAN95: Functional Programming --- Reference Sheet}
  \author{Karl Hallsby}
  \date{Last Edited: \today} % We want to inform people when this document was last edited
\end{titlepage}

\begin{document}
\pagenumbering{gobble}
\maketitle
\pagenumbering{roman} % i, ii, iii on beginning pages, that don't have content
\tableofcontents
\clearpage
\listoftheorems[ignoreall, show={definition, Definition}]
\clearpage
\pagenumbering{arabic} % 1,2,3 on content pages

\nocite{*}

%====================================APPENDIX====================================
\appendix
\counterwithin{definition}{subsection}

\clearpage
\section{Complex Numbers}\label{sec:Complex_Numbers}
\begin{definition}[Complex Number]\label{def:Complex_Number}
  A \emph{complex number} is a hyper real number system.
  This means that two real numbers, $a, b \in \RealNumbers$, are used to construct the set of complex numbers, denoted $\ComplexNumbers$.

  A complex number is written, in Cartesian form, as shown in \Cref{eq:Complex_Number} below.
  \begin{equation}\label{eq:Complex_Number}
    z = a \pm ib
  \end{equation}
  where
  \begin{equation}\label{eq:Imaginary_Value}
    i = \sqrt{-1}
  \end{equation}

  \begin{remark*}[$i$ vs. $j$ for Imaginary Numbers]
    Complex numbers are generally denoted with either $i$ or $j$.
    Electrical engineering regularly makes use of $j$ as the imaginary value.
    This is because alternating current $i$ is already taken, so $j$ is used as the imaginary value instad.
  \end{remark*}
\end{definition}

\subsection{Parts of a Complex Number}\label{subsec:Complex_Number_Parts}
A \nameref{def:Complex_Number} is made of up 2 parts:
\begin{enumerate}[noitemsep]
\item \nameref{def:Real_Part}
\item \nameref{def:Imaginary_Part}
\end{enumerate}

\begin{definition}[Real Part]\label{def:Real_Part}
  The \emph{real part} of an imaginary number, denoted with the $\Re$ operator, is the portion of the \nameref{def:Complex_Number} with no part of the imaginary value $i$ present.

  If $z = x + iy$, then
  \begin{equation}\label{eq:Real_Part}
    \Real{z} = x
  \end{equation}

  \begin{remark}[Alternative Notation]\label{rmk:Real_Part_Alternative_Notation}
    The \nameref{def:Real_Part} of a number sometimes uses a slightly different symbol for denoting the operation.
    It is:
    \begin{equation*}
      \mathfrak{Re}
    \end{equation*}
  \end{remark}
\end{definition}

\begin{definition}[Imaginary Part]\label{def:Imaginary_Part}
  The \emph{imaginary part} of an imaginary number, denoted with the $\Im$ operator, is the portion of the \nameref{def:Complex_Number} where the imaginary value $i$ is present.

  If $z = x + iy$, then
  \begin{equation}\label{eq:Imaginary_Part}
    \Imag{z} = y
  \end{equation}

  \begin{remark}[Alternative Notation]\label{rmk:Imaginary_Part_Alternative_Notation}
    The \nameref{def:Imaginary_Part} of a number sometimes uses a slightly different symbol for denoting the operation.
    It is:
    \begin{equation*}
      \mathfrak{Im}
    \end{equation*}
  \end{remark}
\end{definition}

\subsection{Binary Operations}\label{subsec:Binary_Operations}

%%% Local Variables:
%%% mode: latex
%%% TeX-master: shared
%%% End:


\subsection{Complex Conjugates}\label{app:Complex_Conjugates}
\begin{definition}[Complex Conjugate]\label{def:Complex_Conjugate}
  The conjugate of a complex number is called its \emph{complex conjugate}.
  The complex conjugate of a complex number is the number with an equal real part and an imaginary part equal in magnitude but opposite in sign.
  If we have a complex number as shown below,
  \begin{equation*}
    z = a \pm bi
  \end{equation*}

  then, the conjugate is denoted and calculated as shown below.
  \begin{equation}\label{eq:Complex_Conjugates}
    \Conjugate{z} = a \mp bi
  \end{equation}
\end{definition}

The \nameref{def:Complex_Conjugate} can also be denoted with an asterisk ($*$).
This is generally done for complex functions, rather than single variables.
\begin{equation}\label{eq:Complex_Conjugates_Asterisk}
  z^{*} = \Conjugate{z}
\end{equation}

%%% Local Variables:
%%% mode: latex
%%% TeX-master: shared
%%% End:


\subsection{Geometry of Complex Numbers}\label{subsec:Geometry_Complex_Numbers}
So far, we have viewed \nameref{def:Complex_Number}s only algebraically.
However, we can also view them geometrically as points on a 2 dimensional \nameref{def:Argand_Plane}.

\begin{definition}[Argand Plane]\label{def:Argand_Plane}
  An \emph{Argane Plane} is a standard two dimensional plane whose points are all elements of the complex numbers, $z \in \ComplexNumbers$.
  This is taken from Descarte's definition of a completely real plane.

  The Argand plane contains 2 lines that form the axes, that indicate the real component and the imaginary component of the complex number specified.
\end{definition}

A \nameref{def:Complex_Number} can be viewed as a point in the \nameref{def:Argand_Plane}, where the \nameref{def:Real_Part} is the ``$x$''-component and the \nameref{def:Imaginary_Part} is the ``$y$''-component.

By plotting this, you see that we form a right triangle, so we can find the hypotenuse of that triangle.
This hypotenuse is the distance the point $p$ is from the origin, refered to as the \nameref{def:Complex_Number_Modulus}.
\begin{remark*}
  When working with \nameref{def:Complex_Number}s geometrically, we refer to the points, where they are defined like so:
  \begin{equation*}
    z = x + iy = p(x, y)
  \end{equation*}

  Note that $p$ is \textbf{not} a function of $x$ and $y$.
  Those are the values that inform us \textbf{where} $p$ is located on the \nameref{def:Argand_Plane}.
\end{remark*}

\subsubsection{Modulus of a Complex Number}\label{subsubsec:Complex_Number_Modulus}
\begin{definition}[Modulus]\label{def:Complex_Number_Modulus}
  The \emph{modulus} of a \nameref{def:Complex_Number} is the distance from the origin to the complex point $p$.
  This is based off the Pythagorean Theorem.
  \begin{equation}\label{eq:Complex_Number_Modulus}
    \begin{aligned}
      {\lvert z \rvert}^{2} = x^{2} + y^{2} &= z \Conjugate{z} \\
      \lvert z \rvert &= \sqrt{x^{2} + y^{2}}
    \end{aligned}
  \end{equation}
\end{definition}

\begin{propertylist}
\item The \emph{Law of Moduli} states that $\lvert z w \rvert = \lvert z \rvert \lvert w \rvert$.\label{prop:Law_of_Moduli}.
\end{propertylist}

We can prove \Cref{prop:Law_of_Moduli} using an algebraic identity.
\begin{proof}[Prove \Cref*{prop:Law_of_Moduli}]
  Let $z$ and $w$ be complex numbers ($z, w \in \ComplexNumbers$).
  We are asked to prove
  \begin{equation*}
    \lvert z w \rvert = \lvert z \rvert \lvert w \rvert
  \end{equation*}

  But, it is actually easier to prove
  \begin{equation*}
    {\lvert z w \rvert}^{2} = {\lvert z \rvert}^{2} {\lvert w \rvert}^{2}
  \end{equation*}

  We start by simplifying the ${\lvert z w \rvert}^{2}$ equation above.
  \begin{align*}
    {\lvert z w \rvert}^{2} &= {\lvert z \rvert}^{2} {\lvert w \rvert}^{2} \\
    \intertext{Using the definition of the \nameref{def:Complex_Number_Modulus} of a \nameref{def:Complex_Number} in \Cref{eq:Complex_Number_Modulus}, we can expand the modulus.}
                            &= (z w) (\Conjugate{z w}) \\
    \intertext{Using \Cref{prop:Complex_Conjugate_Split} for multiplication allows us to do the next step.}
                            &= (z w) (\Conjugate{z} \Conjugate{w}) \\
    \intertext{Using Multiplicative Associativity and Multiplicative Commutativity, we can simplify this further.}
                            &= (z \Conjugate{z}) (w \Conjugate{w}) \\
                            &= {\lvert z \rvert}^{2} {\lvert w \rvert}^{2}
  \end{align*}

  Note how we never needed to define $z$ or $w$, so this is as general a result as possible.
\end{proof}

\paragraph{Algebraic Effects of the Modulus' \Cref*{prop:Law_of_Moduli}}\label{par:Law_of_Moduli-Algebraic_Effects}
For this section, let $z = x_{1} + iy_{1}$ and $w = x_{2} + iy_{2}$.
Now,
\begin{align*}
  z w &= (x_{1}x_{2} - y_{1}y_{2}) + i(x_{1}y_{2} + x_{2}y_{1}) \\
  {\lvert z w \rvert}^{2} &= {(x_{1}x_{2} - y_{1}y_{2})}^{2} + {(x_{1}y_{2} + x_{2}y_{1})}^{2} \\
      &= \left( x_{1}^{2} + x_{2}^{2} \right) \left( x_{2}^{2} + y_{2}^{2} \right) \\
      &= {\lvert z \rvert}^{2} {\lvert w \rvert}^{2}
\end{align*}

However, the Law of Moduli (\Cref{prop:Law_of_Moduli}) does \textbf{not} hold for a hyper complex number system one that uses 2 or more imaginaries, i.e.\ $z = a + iy + jz$.
But, the Law of Moduli (\Cref{prop:Law_of_Moduli}) \textbf{does} hold for hyper complex number system that uses 3 imaginaries, $a = z + iy + jz + k \ell$.

\paragraph{Conceptual Effects of the Modulus' \Cref*{prop:Law_of_Moduli}}\label{par:Law_of_Moduli-Conceptual_Effects}
We are interested in seeing if $\lvert z w \rvert = (x_{1}^{2} + y_{1}^{2})(x_{2}^{2}+y_{2}^{2})$ can be extended to more complex terms (3 terms in the complex number).

However, Langrange proved that the equation below \textbf{always} holds.
Note that the $z$ below has no relation to the $z$ above.
\begin{equation*}
  (x_{1} + y_{1} + z_{1}) \neq X^{2} + Y^{2} + Z^{2}
\end{equation*}

%%% Local Variables:
%%% mode: latex
%%% TeX-master: shared
%%% End:


\subsection{Circles and Complex Numbers}\label{subsec:Circles_Complex_Numbers}
We need to define both a center and a radius, just like with regular purely real values.
\Cref{eq:Circles_Complex_Numbers} defines the relation required for a circle using \nameref{def:Complex_Number}s.
\begin{equation}\label{eq:Circles_Complex_Numbers}
  \lvert z - a \rvert = r
\end{equation}

\begin{example}[Lecture 2, Example 1]{Convert to Circle}
  Given the expression below, find the location of the center of the circle and the radius of the circle?
  \begin{equation*}
    \lvert 5 iz + 10 \rvert = 7
  \end{equation*}
  \tcblower{}
  This is just a matter of simplification and moving terms around.
  \begin{align*}
    \lvert 5 iz + 10 \rvert &= 7 \\
    \lvert 5i (z + \frac{10}{5i}) \rvert &= 7 \\
    \lvert 5i (z + \frac{2}{i}) \rvert &= 7 \\
    \lvert 5i (z + \frac{2}{i} \frac{-i}{-i}) \rvert &= 7 \\
    \lvert 5i (z - 2i) \rvert &= 7 \\
    \intertext{Now using the Law of Moduli (\Cref{prop:Law_of_Moduli}) $\lvert a b \rvert = \lvert a \rvert \lvert b \rvert$, we can simplify out the extra imaginary term.}
    \lvert 5i \rvert \lvert z-2i \rvert &= 7 \\
    5 \lvert z - 2i \rvert &= 7 \\
    \lvert z - 2i \rvert = \frac{7}{5}
  \end{align*}

  Thus, the circle formed by the equation $\lvert 5 iz + 10 \rvert = 7$ is actually $\lvert z - 2i \rvert = \frac{7}{5}$, with a center at $a = 2i$ and a radius of $\frac{7}{5}$.
\end{example}

\subsubsection{Annulus}\label{subsubsec:Annulus}
\begin{definition}[Annulus]\label{def:Annulus}
  An \emph{annulus} is a region that is bounded by 2 concentric circles.
  This takes the form of \Cref{eq:Annulus}.
  \begin{equation}\label{eq:Annulus}
    r_{1} \leq \lvert z - a \rvert \leq r_{2}
  \end{equation}

  In \Cref{eq:Annulus}, each of the $\leq$ symbols could also be replaced with $<$.
  This leads to 3 different possibilities for the annulus:
  \begin{enumerate}[noitemsep]
  \item If both inequality symbols are $\leq$, then it is a \textbf{Closed Annulus}.
  \item If both inequality symbols are $<$, then it is an \textbf{Open Annulus}.
  \item If \textbf{only one} inequality symbol $<$ and the other $\leq$, then it is not an \textbf{Open Annulus}.
  \end{enumerate}
\end{definition}


%%% Local Variables:
%%% mode: latex
%%% TeX-master: shared
%%% End:



%%% Local Variables:
%%% mode: latex
%%% TeX-master: shared
%%% End:

\clearpage
\subsection{Trigonometry} \label{app:Trig}
	\subsubsection{Trigonometric Formulas} \label{subsubsec:Trig Formulas}
		\begin{equation} \label{eq:Sin plus Sin with diff Angles}
			\sin \left( \alpha \right) + \sin \left( \beta \right) = 2 \sin \left( \frac{\alpha + \beta}{2} \right) \cos\left( \frac{\alpha - \beta}{2} \right)  
		\end{equation}
		\begin{equation} \label{eq:Cosine-Sine Product}
			\cos \left( \theta \right) \sin \left( \theta \right) = \frac{1}{2} \sin \left( 2 \theta \right)
		\end{equation}
	
	\subsubsection{Euler Equivalents of Trigonometric Functions} \label{subsubsec:Euler Equivalents}
		\begin{equation} \label{eq:Euler Sin}
			\sin \left( x \right) = \frac{e^{\imath x} + e^{-\imath x}}{2}
		\end{equation}
		\begin{equation} \label{eq:Euler Cos}
			\cos \left( x \right) = \frac{e^{\imath x} - e^{-\imath x}}{2 \imath}
		\end{equation}
		\begin{equation} \label{eq:Euler Sinh}
			\sinh \left( x \right) = \frac{e^{x} - e^{-x}}{2}
		\end{equation}
		\begin{equation} \label{eq:Euler Cosh}
			\cosh \left( x \right) = \frac{e^{x} + e^{-x}}{2}
		\end{equation}

\clearpage
\section{Calculus}\label{app:Calculus}
\subsection{L'Hopital's Rule}\label{subsec:LHopitals_Rule}
L'Hopital's Rule can be used to simplify and solve expressions regarding limits that yield irreconcialable results.
\begin{lemma}[L'Hopital's Rule]\label{lemma:LHopitals_Rule}
  If the equation
  \begin{equation*}
    \lim\limits_{x \rightarrow a} \frac{f(x)}{g(x)} =
    \begin{cases}
      \frac{0}{0} \\
      \frac{\infty}{\infty} \\
    \end{cases}
  \end{equation*}
  then \Cref{eq:LHopitals_Rule} holds.
  \begin{equation}\label{eq:LHopitals_Rule}
    \lim\limits_{x \rightarrow a} \frac{f(x)}{g(x)} = \lim\limits_{x \rightarrow a} \frac{f'(x)}{g'(x)}
  \end{equation}
\end{lemma}

\subsection{Fundamental Theorems of Calculus}\label{subsec:Fundamental Theorem of Calculus}
\begin{definition}[First Fundamental Theorem of Calculus]\label{def:1st Fundamental Theorem of Calculus}
  The \emph{first fundamental theorem of calculus} states that, if $f$ is continuous on the closed interval $\left[ a,b \right]$ and $F$ is the indefinite integral of $f$ on $\left[ a,b \right]$, then

  \begin{equation}\label{eq:1st Fundamental Theorem of Calculus}
    \int_{a}^{b}f \left( x \right) dx = F \left( b \right) - F \left( a \right)
  \end{equation}
\end{definition}

\begin{definition}[Second Fundamental Theorem of Calculus]\label{def:2nd Fundamental Theorem of Calculus}
  The \emph{second fundamental theorem of calculus} holds for $f$ a continuous function on an open interval $I$ and $a$ any point in $I$, and states that if $F$ is defined by

  \begin{equation*}
    F \left( x \right) = \int_{a}^{x} f \left( t \right) dt,
  \end{equation*}
  then
  \begin{equation}\label{eq:2nd Fundamental Theorem of Calculus}
    \begin{aligned}
      \frac{d}{dx} \int_{a}^{x} f \left( t \right) dt &= f \left( x \right) \\
      F' \left( x \right) &= f \left( x \right) \\
    \end{aligned}
  \end{equation}
\end{definition}

\begin{definition}[argmax]\label{def:argmax}
  The arguments to the \emph{argmax} function are to be maximized by using their derivatives.
  You must take the derivative of the function, find critical points, then determine if that critical point is a global maxima.
  This is denoted as
  \begin{equation*}\label{eq:argmax}
    \argmax_{x}
  \end{equation*}
\end{definition}

\subsection{Rules of Calculus}\label{subsec:Rules of Calculus}
\subsubsection{Chain Rule}\label{subsubsec:Chain Rule}
\begin{definition}[Chain Rule]\label{def:Chain Rule}
  The \emph{chain rule} is a way to differentiate a function that has 2 functions multiplied together.

  If
  \begin{equation*}
    f(x) = g(x) \cdot h(x)
  \end{equation*}
  then,
  \begin{equation}\label{eq:Chain Rule}
    \begin{aligned}
      f'(x) &= g'(x) \cdot h(x) + g(x) \cdot h'(x) \\
      \frac{df(x)}{dx} &= \frac{dg(x)}{dx} \cdot g(x) + g(x) \cdot \frac{dh(x)}{dx} \\
    \end{aligned}
  \end{equation}
\end{definition}

\subsection{Useful Integrals}\label{subsec:Useful_Integrals}
\begin{equation}\label{eq:Cosine_Indefinite_Integral}
  \int \cos(x) \; dx = \sin(x)
\end{equation}

\begin{equation}\label{eq:Sine_Indefinite_Integral}
  \int \sin(x) \; dx = -\cos(x)
\end{equation}

\begin{equation}\label{eq:x_Cosine_Indefinite_Integral}
  \int x \cos(x) \; dx = \cos(x) + x \sin(x)
\end{equation}
\Cref{eq:x_Cosine_Indefinite_Integral} simplified with Integration by Parts.

\begin{equation}\label{eq:x_Sine_Indefinite_Integral}
  \int x \sin(x) \; dx = \sin(x) - x \cos(x)
\end{equation}
\Cref{eq:x_Sine_Indefinite_Integral} simplified with Integration by Parts.

\begin{equation}\label{eq:x_Squared_Cosine_Indefinite_Integral}
  \int x^{2} \cos(x) \; dx = 2x \cos(x) + (x^{2} - 2) \sin(x)
\end{equation}
\Cref{eq:x_Squared_Cosine_Indefinite_Integral} simplified by using Integration by Parts twice.

\begin{equation}\label{eq:x_Squared_Sine_Indefinite_Integral}
  \int x^{2} \sin(x) \; dx = 2x \sin(x) - (x^{2} - 2) \cos(x)
\end{equation}
\Cref{eq:x_Squared_Sine_Indefinite_Integral} simplified by using Integration by Parts twice.

\begin{equation}\label{eq:Exponential_Cosine_Indefinite_Integral}
  \int e^{\alpha x} \cos(\beta x) \; dx = \frac{e^{\alpha x} \bigl( \alpha \cos(\beta x) + \beta \sin(\beta x) \bigr)}{\alpha^{2} + \beta^{2}} + C
\end{equation}

\begin{equation}\label{eq:Exponential_Sine_Indefinite_Integral}
  \int e^{\alpha x} \sin(\beta x) \; dx = \frac{e^{\alpha x} \bigl( \alpha \sin(\beta x) - \beta \cos(\beta x) \bigr)}{\alpha^{2}+\beta^{2}} + C
\end{equation}

\begin{equation}\label{eq:Exponential_Indefinite_Integral}
  \int e^{\alpha x} \; dx = \frac{e^{\alpha x}}{\alpha}
\end{equation}

\begin{equation}\label{eq:x_Exponential_Indefinite_Integral}
  \int x e^{\alpha x} \; dx = e^{\alpha x} \left( \frac{x}{\alpha} - \frac{1}{\alpha^{2}} \right)
\end{equation}
\Cref{eq:x_Exponential_Indefinite_Integral} simplified with Integration by Parts.

\begin{equation}\label{eq:Inverse_x_Indefinite_Integral}
  \int \frac{dx}{\alpha + \beta x} = \int \frac{1}{\alpha + \beta x} \; dx = \frac{1}{\beta} \ln (\alpha + \beta x)
\end{equation}

\begin{equation}\label{eq:Inverse_x_Squared_Indefinite_Integral}
  \int \frac{dx}{\alpha^{2} + \beta^{2} x^{2}} = \int \frac{1}{\alpha^{2} + \beta^{2} x^{2}} \; dx = \frac{1}{\alpha \beta} \arctan \left( \frac{\beta x}{\alpha} \right)
\end{equation}

\begin{equation}\label{eq:a_Exponential_Indefinite_Integral}
  \int \alpha^{x} \; dx = \frac{\alpha^{x}}{\ln(\alpha)}
\end{equation}

\begin{equation}\label{eq:a_Exponential_Derivative}
  \frac{d}{dx} \alpha^{x} = \frac{d\alpha^{x}}{dx} = \alpha^{x} \ln(x)
\end{equation}

\subsection{Leibnitz's Rule}\label{subsec:Leibnitzs_Rule}
\begin{lemma}[Leibnitz's Rule]\label{lemma:Leibnitzs_Rule}
  Given
  \begin{equation*}
    g(t) = \int_{a(t)}^{b(t)} f(x, t) \, dx
  \end{equation*}
  with $a(t)$ and $b(t)$ differentiable in $t$ and $\frac{\partial f(x, t)}{\partial t}$ continuous in both $t$ and $x$, then
  \begin{equation}\label{eq:Leibnitzs_Rule}
    \frac{d}{dt} g(t) = \frac{d g(t)}{dt} = \int_{a(t)}^{b(t)} \frac{\partial f(x, t)}{\partial t} \, dx + f \bigl[ b(t), t \bigr] \, \frac{d b(t)}{dt} - f \bigl[ a(t), t \bigr] \, \frac{d a(t)}{dt}
  \end{equation}
\end{lemma}



\clearpage
\section{Laplace Transform}\label{app:Laplace_Transform}
\subsection{Laplace Transform}\label{subsec:Laplace_Transform}
\begin{definition}[Laplace Transform]\label{def:Laplace_Transform}
  The \emph{Laplace transformation} operation is denoted as $\Lapl \lbrace x(t) \rbrace$ and is defined as
  \begin{equation}\label{eq:Laplace_Transform}
    X(s) = \int\limits_{-\infty}^{\infty} x(t) e^{-st} dt
  \end{equation}
\end{definition}

\subsection{Inverse Laplace Transform}\label{subsec:Inverse_Laplace_Transform}
\begin{definition}[Inverse Laplace Transform]\label{def:Inverse_Laplace_Transform}
  The \emph{inverse Laplace transformation} operation is denoted as $\Lapl^{-1} \lbrace X(s) \rbrace$ and is defined as
  \begin{equation}\label{eq:Inverse_Laplace_Transform}
    x(t) = \frac{1}{2j \pi} \int_{\sigma-\infty}^{\sigma+\infty} X(s) e^{st} \, ds
  \end{equation}
\end{definition}

\subsection{Properties of the Laplace Transform}\label{subsec:Laplace_Transform_Properties}
\subsubsection{Linearity}\label{subsubsec:Laplace_Linearity}
The \nameref{def:Laplace_Transform} is a linear operation, meaning it obeys the laws of linearity.
This means \Cref{eq:Laplace_Linearity} must hold.
\begin{subequations}\label{eq:Laplace_Linearity}
  \begin{equation}\label{eq:Laplace_Linearity_Time}
    x(t) = \alpha_{1} x_{1}(t) + \alpha_{2} x_{2}(t)
  \end{equation}
  \begin{equation}\label{eq:Laplace_Linearity_Frequency}
    X(s) = \alpha_{1} X_{1}(s) + \alpha_{2} X_{2}(s)
  \end{equation}
\end{subequations}

\subsubsection{Time Scaling}\label{subsubsec:Laplace_Time_Scaling}
Scaling in the time domain (expanding or contracting) yields a slightly different transform.
However, this only makes sense for $\alpha > 0$ in this case.
This is seen in \Cref{eq:Laplace_Time_Scaling}.
\begin{equation}\label{eq:Laplace_Time_Scaling}
  \Lapl \bigl\lbrace x(\alpha t) \bigr\rbrace = \frac{1}{\alpha} X \left( \frac{s}{\alpha} \right)
\end{equation}

\subsubsection{Time Shift}\label{subsubsec:Laplace_Time_Shift}
Shifting in the time domain means to change the point at which we consider $t=0$.
\Cref{eq:Laplace_Time_Shifting} below holds for shifting both forward in time and backward.
\begin{equation}\label{eq:Laplace_Time_Shifting}
  \Lapl \bigl\lbrace x(t-a) \bigr\rbrace = X(s) e^{-a s}
\end{equation}

\subsubsection{Frequency Shift}\label{subsubsec:Laplace_Frequency_Shift}
Shifting in the frequency domain means to change the complex exponential in the time domain.
\begin{equation}\label{eq:Laplace_Frequency_Shift}
  \Lapl^{-1} \bigl\lbrace X(s-a) \bigr\rbrace = x(t)e^{at}
\end{equation}

\subsubsection{Integration in Time}\label{subsubsec:Laplace_Time_Integration}
Integrating in time is equivalent to scaling in the frequency domain.
\begin{equation}\label{eq:Laplace_Time_Integration}
  \Lapl \left\lbrace \int_{0}^{t} x(\lambda) \, d\lambda \right\rbrace = \frac{1}{s} X(s)
\end{equation}

\subsubsection{Frequency Multiplication}\label{subsubsec:Laplace_Frequency_Multiplication}
Multiplication of two signals in the frequency domain is equivalent to a convolution of the signals in the time domain.
\begin{equation}\label{eq:Laplace_Frequency_Multiplication}
  \Lapl \bigl\lbrace x(t) * v(t) \bigr\rbrace = X(s) V(s)
\end{equation}

\subsubsection{Relation to Fourier Transform}\label{subsubsec:Fourier_Transform_Relation}
The Fourier transform looks and behaves very similarly to the Laplace transform.
In fact, if $X(\omega)$ exists, then \Cref{eq:Fourier_Laplace_Transform_Relation} holds.
\begin{equation}\label{eq:Fourier_Laplace_Transform_Relation}
  X(s) = X(\omega) \vert_{\omega = \frac{s}{j}}
\end{equation}

\subsection{Theorems}\label{subsec:Laplace_Theorems}
There are 2 theorems that are most useful here:
\begin{enumerate}[noitemsep]
\item \nameref{thm:Laplace_Initial_Value_Theorem}
\item \nameref{thm:Laplace_Final_Value_Theorem}
\end{enumerate}


%%% Local Variables:
%%% mode: latex
%%% TeX-master: shared
%%% End:


% To make this print, you must include a citation somewhere in the document
\clearpage
\printbibliography{}
\end{document}