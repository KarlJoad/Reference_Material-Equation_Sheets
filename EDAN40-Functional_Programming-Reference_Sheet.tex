\documentclass[10pt,letterpaper,final,twoside,notitlepage]{article}
\usepackage[margin=.5in]{geometry} % 1/2 inch margins on all pages
\usepackage[utf8]{inputenc} % Define the input encoding
\usepackage[USenglish]{babel} % Define language used
\usepackage{amsmath}
\usepackage{mathtools} % Allow for text and math in align* environment.
\usepackage{amsfonts}
\usepackage{amssymb}
\usepackage{amsthm} % Gives us plain, definition, and remark to use in \theoremstyle{style}
\usepackage{thmtools}
\usepackage{thm-restate}
\usepackage{graphicx}

\usepackage[
backend=biber,
style=alphabetic,
citestyle=authoryear]{biblatex} % Must include citation somewhere in document to print bibliography
\usepackage{hyperref} % Generate hyperlinks to referenced items
\usepackage[nottoc]{tocbibind} % Prints the Reference/Bibliography in TOC as well
\usepackage[noabbrev,nameinlink]{cleveref} % Fancy cross-references in the document everywhere
\usepackage{nameref} % Can make references by name to places
\usepackage{caption} % Allows for greater control over captions in figure, algorithm, table, etc. environments
\usepackage{subcaption} % Allows for multiple figures in one Figure environment
\usepackage[binary-units=true]{siunitx} % Gives us ways to typeset units for stuff
\usepackage{csquotes} % Context-sensitive quotation facilities
\usepackage{enumitem} % Provides [noitemsep, nolistsep] for more compact lists
\usepackage{chngcntr} % Allows us to tamper with the counter a little more
\usepackage{empheq} % Allow boxing of equations in special math environments
\usepackage[x11names]{xcolor} % Gives access to coloring text in environments or just text, MUST be before tikz
\usepackage{tcolorbox} % Allows us to create boxes of various types for examples
\usepackage{tikz} % Allows us to create TikZ and PGF Pictures
\usepackage{ctable} % Greater control over tables and how they look
\usepackage{multirow} % Allow us to have a single cell in a table span multiple rows
\usepackage{titling} % Put document information throughout the document programmatically
\usepackage[linesnumbered,ruled,vlined]{algorithm2e} % Allows us to write algorithms in a nice style.

\counterwithin{figure}{section}
\counterwithin{table}{section}
\counterwithin{equation}{section}
\counterwithin{algocf}{section}
\crefname{algocf}{algorithm}{algorithms}
\Crefname{algocf}{Algorithm}{Algorithms}
\setcounter{secnumdepth}{4}
\setcounter{tocdepth}{4} % Include \paragraph in toc
\crefname{paragraph}{paragraph}{paragraphs}
\Crefname{paragraph}{Paragraph}{Paragraphs}

% Create a theorem environment
\theoremstyle{plain}
\newtheorem{theorem}{Theorem}[section]
% Create a numbered theorem-like environment for lemmas
\newtheorem{lemma}{Lemma}[theorem]

% Create a definition environment
\theoremstyle{definition}
\newtheorem{definition}{Defn}
\newtheorem{corollary}{Corollary}[section]
% \begin{definition}[Term] \label{def:}
% 		Make sure the term is emphasized with \emph{term}.
%		This ensures that if \emph is changed, it shows up everywhere
% \end{definition}

% Create a numbered remark environment numbered based on definition
% NOTE: This version of remark MUST go inside a definition environment
\theoremstyle{remark}
\newtheorem{remark}{Remark}[definition]
%\counterwithin{definition}{subsection} % Uncomment to have definitions use section.subsection numbering

% Create an unnumbered remark environment for general use
% NOTE: This version of remark has NO restrictions on placement
\newtheorem*{remark*}{Remark}

% Create a special list that handles properties. It can be broken and restarted
\newlist{propertylist}{enumerate}{1} % {Name}{Template}{Max Depth}
% [newlistname, LevelsToApplyTo]{formatting options}
\setlist[propertylist, 1]{label=\textbf{(\roman*)}, ref=\textbf{(\roman*)}, noitemsep, nolistsep}
\crefname{propertylisti}{property}{properties}
\Crefname{propertylisti}{Property}{Properties}

% Create a special list that handles enumerate starting with lower letters. Breakable/Restartable.
\newlist{boldalphlist}{enumerate}{1} % {Name}{Template}{Max Depth}
% [newlistname, LevelsToApplyTo]{formatting options}
\setlist[boldalphlist, 1]{label=\textbf{(\alph*)}, ref=\alph*, noitemsep, nolistsep} % Set options

\newlist{nocrefenumerate}{enumerate}{1} % {Name}{Template}{Max Depth}
% [newlistname, LevelsToApplyTo]{formatting options}
\setlist[nocrefenumerate, 1]{label=(\arabic*), ref=(\arabic*), noitemsep, nolistsep}

% Create a list that allows for deeper nesting of numbers. By default enumerate only allows depth=4.
\newlist{nestednums}{enumerate}{6}
% [newlistname, LevelsToApplyTo]{formatting options}
\setlist[nestednums]{noitemsep, label*=\arabic*.}

\tcbuselibrary{breakable} % Allow tcolorboxes to be broken across pages
% Create a tcolorbox for examples
% /begin{example}[extra name]{NAME}
% Create a tcolorbox for examples
% Argument #1 is optional, given by [], that is the textbook's problem number
% Argument #2 is mandatory, given by {}, that is the title for the example
% Avoid putting special characters, (), [], {}, ",", etc. in the title.
\newtcolorbox[auto counter,
number within=section,
number format=\arabic,
crefname={example}{examples}, % Define reference format for cref (No Capitals)
Crefname={Example}{Examples}, % Reference format for cleveref (With Capitals)
]{example}[2][]{ % The [2][] Means the first argument is optional
  width=\textwidth,
  title={Example \thetcbcounter: #2. #1}, % Parentheses and commas are not well supported
  fonttitle=\bfseries,
  label={ex:#2},
  nameref=#2,
  colbacktitle=white!100!black,
  coltitle=black!100!white,
  colback=white!100!black,
  upperbox=visible,
  lowerbox=visible,
  sharp corners=all,
  breakable
}

% Create a tcolorbox for general use
\newtcolorbox[% auto counter,
% number within=section,
% number format=\arabic,
% crefname={example}{examples}, % Define reference format for cref (No Capitals)
% Crefname={Example}{Examples}, % Reference format for cleveref (With Capitals)
]{blackbox}{
  width=\textwidth,
  % title={},
  fonttitle=\bfseries,
  % label={},
  % nameref=,
  colbacktitle=white!100!black,
  coltitle=black!100!white,
  colback=white!100!black,
  upperbox=visible,
  lowerbox=visible,
  sharp corners=all
}

% Redefine the 'end of proof' symbol to be a black square, not blank
\renewcommand\qedsymbol{$\blacksquare$} % Change proofs to have black square at end

\renewcommand{\Re}{\operatorname{Re}} % Redefine to use the command, but not the fraktur version
\renewcommand{\Im}{\operatorname{Im}} % Redefine to use the command, but not the fraktur version
% Math Operators that are useful to abstract the written math away to one spot
\DeclareMathOperator{\RealNumbers}{\mathbb{R}}
\newcommand{\TextRealNumbers}{$\RealNumbers$}
\DeclareMathOperator{\AllIntegers}{\mathbb{Z}}
\newcommand{\TextAllIntegers}{$\AllIntegers$}
\DeclareMathOperator{\PositiveInts}{\mathbb{Z}^{+}}
\newcommand{\TextPositiveInts}{$\PositiveInts$}
\DeclareMathOperator{\NegativeInts}{\mathbb{Z}^{-}}
\newcommand{\TextNegativeInts}{$\NegativeInts$}
\DeclareMathOperator{\NaturalNumbers}{\mathbb{N}}
\newcommand{\TextNaturalNumbers}{$\NaturalNumbers$}
\DeclareMathOperator{\ComplexNumbers}{\mathbb{C}}
\newcommand{\TextComplexNumbers}{$\ComplexNumbers$}
\DeclareMathOperator{\RationalNumbers}{\mathbb{Q}}
\newcommand{\TextRationalNumbers}{$\RationalNumbers$}
\DeclareMathOperator*{\argmax}{argmax} % Thin Space and subscripts are UNDER in display
\DeclareMathOperator{\Lapl}{\mathcal{L}} % Declare a Laplace symbol to be used
\DeclareMathOperator{\UnitStep}{\mathcal{U}}
\DeclareMathOperator{\sinc}{sinc} % sinc(x) = (sin(pi x)/(pi x))
\DeclareMathOperator{\XOR}{\oplus}

\newcommand{\rbpRegister}{\texttt{\%rbp}}
\newcommand{\rspRegister}{\texttt{\%rsp}}
\newcommand{\ripRegister}{\texttt{\%rip}}
\newcommand{\raxRegister}{\texttt{\%rax}}
\newcommand{\rbxRegister}{\texttt{\%rbx}}

%%% Local Variables:
%%% mode: latex
%%% TeX-master: shared
%%% End:


% These packages are more specific to certain documents, but will be availabe in the template
% \usepackage{esint} % Provides us with more types of integral symbols to use
\usepackage[outputdir=./TeX_Output]{minted} % Allow us to nicely typeset 300+ programming languages
% This document must be compiled with the -shell-escape flag if the packages above are uncommented

% \graphicspath{{./Drawings/Course}} % Uncomment this to use pictures in this document
\addbibresource{./Bibliographies/EDAN40-Functional_Programming.bib}

% Math Operators that are useful to abstract the written math away to one spot
% These are supposed to be document-specific mathematical operators that will make life easier
% Many fundamental operators are defined in Reference_Sheet_Preamble.tex

\usemintedstyle{emacs} % Best for use on white backgrounds. Use for inline-code
% This macro creates the 2 minted environments for kernel source code with common options
\def\mintedhaskellargs{
  frame=lines, % Surround the source code with lines on top and bottom
  linenos, % We want to show line numbers for each line in the margin
  % Colors used here are xcolor X11 colors.
  % style=fruity, % Use the fruity color scheme. Best for use on black backgrounds. Use for code blocks.
  % bgcolor=black, % Set the background used
  style=emacs,
  bgcolor=white,
  autogobble=true, % Automatically remove shared indentation from files
  breaklines=true, % Break lines that are too long at convenient locations
}
\newcommand{\makenewmintedfiles}[1]{
  \newminted[haskellsource]{haskell}{#1} % Use with \begin{minted}{kernelsource} code \end{minted}

  \newmintedfile[haskellsourcefile]{haskell}{#1} % Use with \kernelsourcefile[additional-options]{Filename}
}
\expandafter\makenewmintedfiles\expandafter{\mintedhaskellargs}
\newmintinline[haskellinline]{haskell}{% Use with \kernelinline{code}
  style=emacs,
  bgcolor=white,
}

\crefname{lstlisting}{listing}{listings}
\Crefname{lstlisting}{Listing}{Listings}

\begin{titlepage}
  \title{EDAN40/EDAN95: Functional Programming --- Reference Sheet \\ Lund University}
  \author{Karl Hallsby}
  \date{Last Edited: \today} % We want to inform people when this document was last edited
\end{titlepage}

\begin{document}
\pagenumbering{gobble}
\maketitle
\pagenumbering{roman} % i, ii, iii on beginning pages, that don't have content
\tableofcontents
\clearpage
\listoftheorems[ignoreall, show={definition, Definition}]
\clearpage
\listoflistings{}
\clearpage
\pagenumbering{arabic} % 1,2,3 on content pages

\nocite{*}

\section{Introduction}\label{sec:Introduction}
This section is dedicated to giving a small introduction to functional programming.
Functional Programming is a style of programming, nothing else.
In this style, the basic method of computation is the evaluation of expressions as arguments to functions, which themselves return expressions.

\begin{quote}
  ``Functional programming is so called because a program consists entirely of functions. [$\ldots$]
  These functions are much like ordinary mathematical functions [$\ldots$] defined by ordinary equations'' (John Hughes)
\end{quote}

If you want to view all possible language categories, visit \href{https://en.wikipedia.org/wiki/Programming_paradigm}{Wikipedia's Programming Paradigms}.

\begin{definition}[Imperative Programming Language]\label{def:Imperative_Programming_Language}
  \emph{Imperative programming languages} have a programming paradigm that uses statements that change a program's state.
  An imperative program consists of commands for the computer to perform.
  Imperative programming focuses on describing how a program operates.
\end{definition}

\begin{restatable}[Functional Programming Language]{definition}{defFunctionalProgrammingLanguage}\label{def:Functional_Programming_Language}
  \emph{Functional programming languages} treat computation as the evaluation of mathematical functions and avoids changing-state and mutable data.
  It is a declarative programming paradigm in that programming is done with expressions or declarations instead of statements.
  In functional code, the output value of a function depends only on its arguments, so calling a function with the same value for an argument always produces the same result.

  This is in contrast to \nameref{def:Imperative_Programming_Language}s where, in addition to a function's arguments, global program state can affect a function's resulting value.
  Eliminating side effects, that is, changes in state that do not depend on the function inputs, can make understanding a program easier, which is one of the key motivations for the development of functional programming.

  Because of its close relationship to mathematics, it is much easier to develop mathematical techniques for reasoning about and proving the behavior of programs developed in functional languages.
  These techniques are important tools for helping us to ensure that programs work properly without having to resort to tedious testing and debugging which can only show the presence of errors, never their absence.
  Moreover, they provide important tools for documenting the reasoning that went into the formulation of a program, making the code easier to understand and maintain.

  \begin{remark}[Course Language]
    The languages of use in this course is Haskell.
    It is a purely functional language that supports impure actions with \nameref{def:Monad}s.
  \end{remark}
\end{restatable}

Functional programming is very nice because it allows us to perform certain actions that are quite natural quite easily.
For example,
\begin{itemize}[noitemsep]
\item \nameref{subsubsec:Higher_Order_Functions}
  \begin{itemize}[noitemsep]
  \item Functions that take functions as arguments and return functions as expressions
  \item Used frequently
  \item Currying
  \item How to use effectively?
  \end{itemize}

\item \nameref{subsubsec:Infinite_Data_Structures}
  \begin{itemize}[noitemsep]
  \item Nice idea that is easily proven in functional languages
  \end{itemize}

\item Lazy evaluation (This is a function unique to Haskell)
  \begin{itemize}[noitemsep]
  \item Only evaluate expressions \textbf{ONLY WHEN NEEDED}
  \item This also allow us to deal with idea of infinite data structures
  \end{itemize}
\end{itemize}

\subsection{Rewrite Semantics}\label{subsec:Rewrite_Semantics}
One of the key strengths of \nameref{def:Functional_Programming_Language}s is the fact we can easily perform \nameref{def:Rewrite_Semantics} on any given \nameref{def:Expression}.

\begin{definition}[Rewrite Semantics]\label{def:Rewrite_Semantics}
  \emph{Rewrite semantics} is the process of rewriting and deconstructing an \nameref{def:Expression} into its constiuent parts.
  Rewrite semantics answers the question ``How do we extract values from functions?''
\end{definition}

\begin{listing}[h!tbp]
\haskellsourcefile{./EDAN40-Functional_Programming-Sections/Introduction/Code/Rewrite_Factorial.hs}
\caption{\nameref{def:Rewrite_Semantics} of a Factorial Function}
\label{lst:Rewrite_Semantics}
\end{listing}

\begin{definition}[Expression]\label{def:Expression}
  An \emph{expression} is a combination of one or more \nameref{def:Operand}s and operators that the programming language interprets (according to its particular rules of precedence and of association) and computes to produce another value.

  \begin{remark}[Overloading]\label{rmk:Overloading_Expressions}
    An expression can be overloaded if there is more than one definition for an operator.
  \end{remark}
\end{definition}

\begin{definition}[Operand]\label{def:Operand}
  An \emph{operand} is a:
  \begin{itemize}[noitemsep]
  \item Constant
  \item Variable
  \item Another \nameref{def:Expression}
  \item Result from function calls
  \end{itemize}
\end{definition}

%%% Local Variables:
%%% mode: latex
%%% TeX-master: "../../EDAN40-Functional_Programming-Reference_Sheet"
%%% End:


\subsection{Paradigm Differences}\label{subsec:Paradigm_Differences}
Functional programming is a completely different paradigm of programming than traditional imperative programming.
One of the biggest differences is that \textbf{side effects are NOT allowed}.

\subsubsection{Side Effects}\label{subsubsec:Side_Effects}
Side effects are typically defined as being function-local.
So, we can assign variables, make lists, etc. \textbf{so long as the effects are destroyed upon leaving the function}.
Additionally, nothing globally usable can/should be changed.

\begin{listing}[h!tbp]
\begin{minted}[frame=lines,linenos,style=emacs,autogobble=true,breaklines=true]{c}
public int f(int x) {
	int t1 = g(x) + g(x);
	int t2 = 2 * g(x);
	return t1-t2;
}
// We should probably get 0 back.
// f(x) = t1-t2 = g(x) + g(x) - 2*g(x) = 0

// But, if g(x) is defined like so,
public int g(int x) {
	int y = input.nextInt();
	return y;
}
// The two instances of g(x) (g(x) + g(x)) can be different values,
// This invalidates the result we reached made earlier.
\end{minted}
\caption{C-Like Code with Side Effects}
\label{lst:Side_Effects}
\end{listing}

\subsubsection{Syntactic Differences}\label{subsubsec:Syntactic_Differences}
The \texttt{=} symbol has different meanings in \nameref{def:Functional_Programming_Language}s.
In functional languages, \texttt{=}, is the mathematical definition of equivalence.
Whereas in \nameref{def:Imperative_Programming_Language}s, \texttt{=} is the assignment of values to memory locations.

Typically, \nameref{def:Functional_Programming_Language}s do not have a way to directly access memory, since that is an inherently stateful change, breaking the rules of ``side-effect free''.
However, ``variables'' \textbf{do} exist, but they are different.
\begin{itemize}[noitemsep]
\item Variables are \textbf{NAMED} expressions, not locations in memory
\item When ``reassigning'' a variable, the old value that name pointed to is discarded, and a new one created.
\end{itemize}

\subsubsection{Tendency Towards Recursion}\label{subsubsec:Tendency_Recursion}
Most \nameref{def:Functional_Programming_Language}s use recursion more than they use iteration.
This is possible because recursion can express all solutions that iteration can, but that does not hold true the other way around.
Recursion is also intimately tied to the computability of an \nameref{def:Expression}.

Take the code snippet below as an example.
It sums all values from a list of arbitrary size by taking the front element of the provided list (\texttt{x}) and adding that to the results of adding the rest of the list (\texttt{xs}) together.

\begin{listing}[h!tbp]
\haskellsourcefile{./EDAN40-Functional_Programming-Sections/Introduction/Code/sum1.hs}
\caption{Basic List Summation}
\label{lst:Recursion_List_Summation}
\end{listing}

\subsubsection{Higher-Order Functions}\label{subsubsec:Higher_Order_Functions}
Similarly to what we defined in \Cref{lst:Recursion_List_Summation}, say we want to define the operations:
\begin{itemize}[noitemsep]
\item Multiplying all elements together
\item Finding if any elements are \texttt{True}.
\item Finding if all the elements are \texttt{True}.
\end{itemize}

It would look like the code shown below.
The code from \Cref{lst:Recursion_List_Summation} will be included.
\begin{listing}[h!tbp]
\haskellsourcefile{./EDAN40-Functional_Programming-Sections/Introduction/Code/Many_Funcs_No_Higher_Order.hs}
\caption{List Comprehension Functions, No Higher-Order Functions Used}
\label{lst:Many_Funcs_No_Higher_Order}
\end{listing}

If you look at each of the functions, you will notice something in common between all of them.
\begin{itemize}[noitemsep]
\item There is a default value, depending on the operation, for when the list is empty.
\item There is an operation applied between the current element and,
\item The rest of the list is recursively operated upon.
\end{itemize}

If we instead used a higher-order function, we can define all of those functions with just one higher-order function.
\begin{listing}[h!tbp]
\haskellsourcefile{./EDAN40-Functional_Programming-Sections/Introduction/Code/Many_Funcs_Higher_Order.hs}
\caption{List Comprehension Functions, Higher-Order Functions Used}
\label{lst:Many_Funcs_Higher_Order}
\end{listing}

\subsubsection{Infinite Data Structures}\label{subsubsec:Infinite_Data_Structures}
One of the benefits of lazy evaluation, and allowing higher-order functions, is that infinite data structures can be created.
So, we could have a list of \textbf{all} integers, but we will not run out of memory (probably).
Because of lazy evaluation, the values from these infinite data structures are computed \textbf{on when needed}.

For example, we find all prime numbers, starting with 2, using the Eratosthenes Sieve method (\Cref{lst:Infinite_Data_Structure}).
This method states we take \textbf{ALL} integers, starting from 2
\begin{enumerate}[noitemsep]
\item Make a list out of them.
\item Take the first element out.
\item Remove all multiples of that number.
\item Put that number into a list of primes.
\item Repeat from step 2, until you find all the prime numbers you want.
\end{enumerate}

In Haskell, this looks like:
\begin{listing}[h!tbp]
\haskellsourcefile{./EDAN40-Functional_Programming-Sections/Introduction/Code/Eratosthenes_Primes.hs}
\caption{Infinite Data Structure, All Primes by Eratosthenes Sieve}
\label{lst:Infinite_Data_Structure}
\end{listing}

%%% Local Variables:
%%% mode: latex
%%% TeX-master: "../../EDAN40-Functional_Programming-Reference_Sheet"
%%% End:


\subsection{Language Basics}\label{subsec:Lang_Basics}
All of the functions and operations presented below come from \textbf{The Standard Prelude}.
The library file \emph{Prelude.hs} is loaded first by the REPL (Read, Evaluate, Print, Loop) environment that we will use.
It defines:
\begin{itemize}[noitemsep]
\item \nameref{subsubsec:Math_Ops}
\item \nameref{subsubsec:List_Ops}
\item And other conveniences for writing Haskell.
\end{itemize}

\subsubsection{Mathematical Operations}\label{subsubsec:Math_Ops}
\emph{Prelude.hs} defines the basic mathematical \textbf{integer} functions of:
\begin{itemize}[noitemsep]
\item Addition
\item Subtraction
\item Multiplication
\item Division
\item Exponentiation
\end{itemize}
\begin{listing}[h!tbp]
\begin{haskellsource}
> 2+3
5
> 2-3
-1
> 2*3
6
>7 `div` 2
3
> 2^3
8
\end{haskellsource}
\caption{Integer Mathematical Operations}
\label{lst:Int_Math_Ops}
\end{listing}

\paragraph{Precedences}\label{par:Math_Precedences}
Just like in normal mathematics, there exists a precedence to disambiguate mathematical expressions containing multiple, different operations.
In order of highest-to-lowest precedence:
\begin{enumerate}[noitemsep]
\item Negation
\item Exponentiation
\item Multiplication and Division
\item Addition and Subtraction
\end{enumerate}

\paragraph{Associativity}\label{par:Math_Associativity}
Just like in normal mathematics, there are rules associativity rules to disambiguate mathematical expressions containing multiple of the same operations.
There are only 2 types of associativity, left and right.
\begin{enumerate}[noitemsep]
\item Left Associative:
  \begin{itemize}[noitemsep]
  \item Everything else.
  \item Addition. $2+3+4 = (2+3)+4$
  \item Subtraction. $2-3-4 = (2-3)-4$
  \item Multiplication. $2*3*4 = (2*3)*4$
  \item Division. $2 \div 3 \div 4 = (2 \div 3) \div 4$
  \end{itemize}
\item Right Associative:
  \begin{itemize}[noitemsep]
  \item Exponentiation. $2^{3^{4}} = 2^{(3^{4})}$
  \item Negation. $--2 = -(-2) = 2$
  \end{itemize}
\end{enumerate}

\begin{remark*}[Types of Associativity]
  Technically, there are 3 types of associativity.
  \begin{enumerate}[noitemsep]
  \item Left-Associative
  \item Right-Associative
  \item Non-Associative
  \end{enumerate}

  Non-associativity means that it does not have an implicit associativity rule associated with it.
  It could also mean it is neither left-, nor right-associative.
\end{remark*}

\subsubsection{List Operations}\label{subsubsec:List_Ops}
\emph{Prelude.hs} also defines the basic list operations that we will need.
To denote a list in Haskell, the elements are comma-delimited inside of square braces.
For example, the mathematical list (set) of integers 1 to 3 $\lbrace 1, 2, 3 \rbrace$ is written in Haskell like so \haskellinline{[1, 2, 3]}.

Lists are a homogenous data structure.
It stores several \textbf{elements of the same type}.
So, we can have a list of integers or a list of characters but we can't have a list with both integers and characters.
The most common list operations are shown below.

\paragraph{\texorpdfstring{\haskellinline{head}}{\texttt{head}}}\label{par:List_Head_Function}
Get the \emph{head} of a list. Return the first element of a non-empty list. Remove all elements other than the first element.
If the list \textbf{is} empty, then an Exception is returned.
See \Cref{lst:List_Head_Function}.
\begin{listing}[h!tbp]
\begin{haskellsource}
  > head [1, 2, 3, 4, 5]
  1
\end{haskellsource}
\caption{Haskell \haskellinline{head} Function}
\label{lst:List_Head_Function}
\end{listing}

\paragraph{\texorpdfstring{\haskellinline{tail}}{\texttt{tail}}}\label{par:List_Tail_Function}
Get the \emph{tail} of a list. Return the second through $n$th elements of a non-empty list. Remove the first element.
If the list \textbf{is} empty, then an Exception is returned.
See \Cref{lst:List_Tail_Function}.
\begin{listing}[h!tbp]
\begin{haskellsource}
  > tail [1, 2, 3, 4, 5]
  [2, 3, 4, 5]
\end{haskellsource}
\caption{Haskell \haskellinline{tail} Function}
\label{lst:List_Tail_Function}
\end{listing}

\paragraph{\texorpdfstring{\haskellinline{last}}{\texttt{last}}}\label{par:List_Last_Function}
Get the \emph{last} element in a list.
See \Cref{lst:List_Last_Function}.
\begin{listing}[h!tbp]
\begin{haskellsource}
  > last [1, 2, 3, 4, 5]
  5
\end{haskellsource}
\caption{Haskell \haskellinline{last} Function}
\label{lst:List_Last_Function}
\end{listing}

\paragraph{\texorpdfstring{\haskellinline{init}}{\texttt{init}}}\label{par:List_Init_Function}
Get the \emph{init}ial portion of the list, namely all elements except the last one.
See \Cref{lst:List_Init_Function}.
\begin{listing}[h!tbp]
\begin{haskellsource}
  > init [1, 2, 3, 4, 5]
  [1, 2, 3, 4]
\end{haskellsource}
\caption{Haskell \haskellinline{init} Function}
\label{lst:List_Init_Function}
\end{listing}

\paragraph{\texorpdfstring{Selection, \haskellinline{!!}}{Selection, \texttt{!!}}}\label{par:List_Select_Function}
Select the $n$th element of a list. Lists in Haskell are zero-indexed.
See \Cref{lst:List_Select_Function}.
\begin{listing}[h!tbp]
\begin{haskellsource}
  > [1, 2, 3, 4, 5] !! 2
  3
\end{haskellsource}
\caption{Haskell \haskellinline{!!} Function}
\label{lst:List_Select_Function}
\end{listing}

\paragraph{\texorpdfstring{\haskellinline{take}}{\texttt{take}}}\label{par:List_Take_Function}
\emph{Take} the first $n$ elements of a list.
See \Cref{lst:List_Take_Function}.
\begin{listing}[h!tbp]
\begin{haskellsource}
  > take 3 [1, 2, 3, 4, 5]
  [1, 2, 3]
\end{haskellsource}
\caption{Haskell \haskellinline{take} Function}
\label{lst:List_Take_Function}
\end{listing}

\paragraph{\texorpdfstring{\haskellinline{drop}}{\texttt{drop}}}\label{par:List_Drop_Function}
\emph{Drop} the first $n$ elements of a list.
See \Cref{lst:List_Drop_Function}.
\begin{listing}[h!tbp]
\begin{haskellsource}
  > drop 3 [1, 2, 3, 4, 5]
  [4, 5]
\end{haskellsource}
\caption{Haskell \haskellinline{drop} Function}
\label{lst:List_Drop_Function}
\end{listing}

\paragraph{\texorpdfstring{Appending Lists to Lists, \haskellinline{++}}{Appending Lists to Lists}}\label{par:List_Append_Function}
Append the second list to the end of the first list.
See \Cref{lst:List_Append_Function}.
\begin{listing}[h!tbp]
\begin{haskellsource}
  > [1, 2, 3, 4] ++ [9, 10, 11, 12]
  [1, 2, 3, 4, 9, 10, 11, 12]
\end{haskellsource}
\caption{Haskell \haskellinline{++} Function}
\label{lst:List_Append_Function}
\end{listing}

\begin{remark*}
  Be careful of this function.
  It runs in $O(n_{1})$-like time, where $n_{1}$ is the length of the first list.
\end{remark*}

\paragraph{\texorpdfstring{Constructing Lists, \haskellinline{:}}{\texttt{cons}tructing Lists}}\label{par:List_Cons_Function}
To construct lists, they need to be composed from single expressions.
This is done with the \texttt{cons} function.
See \Cref{lst:List_Cons_Function}.
\begin{listing}[h!tbp]
\begin{haskellsource}
  > 8:[1, 2, 3, 4]
  [8, 1, 2, 3, 4]
\end{haskellsource}
\caption{Haskell \haskellinline{:} Function}
\label{lst:List_Cons_Function}
\end{listing}

\paragraph{\texorpdfstring{\haskellinline{length}}{\texttt{length}}}\label{par:List_Length_Function}
To get the \emph{length} of a list, use \Cref{lst:List_Length_Function}.
\begin{listing}[h!tbp]
\begin{haskellsource}
  > length [1, 2, 3, 4, 5]
  5
\end{haskellsource}
\caption{Haskell \haskellinline{length} Function}
\label{lst:List_Length_Function}
\end{listing}

\paragraph{\texorpdfstring{\haskellinline{sum}}{\texttt{sum}}}\label{par:List_Sum_Function}
The \emph{sum} function is used to find the sum of all elements in a list.
See \Cref{lst:List_Sum_Function}.
\begin{listing}[h!tbp]
\begin{haskellsource}
  > sum [1, 2, 3, 4, 5]
  15
\end{haskellsource}
\caption{Haskell \haskellinline{sum} Function}
\label{lst:List_Sum_Function}
\end{listing}

\paragraph{\texorpdfstring{\haskellinline{product}}{\texttt{product}}}\label{par:List_Product_Function}
The \emph{product} function is used to find the product of all elements in a list.
See \Cref{lst:List_Product_Function}.
\begin{listing}[h!tbp]
\begin{haskellsource}
  > product [1, 2, 3, 4, 5]
  120
\end{haskellsource}
\caption{Haskell \haskellinline{product} Function}
\label{lst:List_Product_Function}
\end{listing}

\paragraph{\texorpdfstring{\haskellinline{reverse}}{\texttt{reverse}}}\label{par:List_Reverse_Function}
The \emph{reverse} function is used to reverse the order of the elements in a list.
See \Cref{lst:List_Reverse_Function}.
\begin{listing}[h!tbp]
\begin{haskellsource}
  > reverse [1, 2, 3, 4, 5]
  [5, 4, 3, 2, 1]
\end{haskellsource}
\caption{Haskell \haskellinline{reverse} Function}
\label{lst:List_Reverse_Function}
\end{listing}

\subsubsection{Function Application}\label{subsubsec:Function_Application}
Like in mathematics, functions can be used in expressions, and are treated as first-class objects.
This means they have the same properties as regular variables, for almost all intents and purposes.
For example, the equation
\begin{equation*}
  f(a,b) + cd
\end{equation*}
would be translated to Haskell like so
\begin{haskellsource}
  f a b + c * d
\end{haskellsource}

To ensure that functions are handled in Haskell like they are in mathematics, they have the highest precedence in an expression.
This means that
\begin{haskellsource}
  f a + b
\end{haskellsource}
means
\begin{equation*}
  f(a) + b
\end{equation*}
in mathematics.

\Cref{tab:Function_Parens} illustrates the use of parentheses to ensure Haskell functions are interpretted like their mathematical counterparts.

\begin{table}[h!tbp]
  \centering
  \begin{tabular}{cc}
    \toprule
    Mathematics & Haskell \\
    \midrule
    $f(x)$ & \haskellinline{f x} \\
    $f(x, y)$ & \haskellinline{f x y} \\
    $f \bigl( g(x) \bigr)$ & \haskellinline{f (g x)} \\
    $f \bigl(x, g(y) \bigr)$ & \haskellinline{f x (g x)} \\
    $f(x) g(y)$ & \haskellinline{f x * g y} \\
    \bottomrule
  \end{tabular}
  \caption{Parentheses Used with Functions}
  \label{tab:Function_Parens}
\end{table}

Note that parentheses are still required in the Haskell expression \haskellinline{f (g x)} above, because \haskellinline{f g x} on its own would be interpreted as the application of the function \texttt{f} to two arguments \texttt{g} and \texttt{x}, whereas the intention is that \texttt{f} is applied to one argument, namely the result of applying the function \texttt{g} to an argument \texttt{x}.

\subsubsection{Haskell Files/Scripts}\label{subsubsec:Haskell_Scripts}
New functions can be defined within a script, a text file comprising a sequence of definitions.
By convention, Haskell scripts usually have a \texttt{.hs} file extension on their filename.

If you load a script into a REPL environment, the \emph{Prelude.hs} library is already loaded for you, so you can work with that directly.
To load a file, you use the \texttt{:load} command at the REPL.\@
Once loaded, you can call all the functions in the script at the REPL line.

If you edit the script, save it, and want your changes to be reflected in the REPL, you must \texttt{:reload} the REPL.\@

Some basic REPL commands are shown in \Cref{tab:Basic_REPL_Commands}

\begin{table}[h!tbp]
  \centering
  \begin{tabular}{ll}
    \toprule
    Command & Meaning \\
    \midrule
    \texttt{:load \emph{name}} or \texttt{:l \emph{name}} & Load script \texttt{\emph{name}} \\
    \texttt{:reload} or \texttt{:r} & Reload the current scripts \\
    \texttt{:type \emph{expr}} or \texttt{:t \emph{expr}} & Show the type of \texttt{\emph{expr}} \\
    \texttt{:?} & Show all possible commands\\
    \texttt{:quit} or \texttt{:q} & Quit the REPL \\
    \bottomrule
  \end{tabular}
  \caption{Basic REPL Commands}
  \label{tab:Basic_REPL_Commands}
\end{table}

\paragraph{Naming Conventions}\label{par:Naming_Conventions}
There are some conventions and requirements when it comes to naming expressions in Haskell.

\subparagraph{Function Naming Conventions}\label{subpar:Function_Naming_Conventions}
Functions \textbf{\emph{MUST}}:
\begin{itemize}[noitemsep]
\item Start with a \textbf{LOWER}-case letter
\item Every subsequent character in the name can be upper-, lower-case, a number, underscores, or single quotes (\texttt{'}).
\end{itemize}


%%% Local Variables:
%%% mode: latex
%%% TeX-master: "../../EDAN40-Functional_Programming-Reference_Sheet"
%%% End:


%%% Local Variables:
%%% mode: latex
%%% TeX-master: "../EDAN40-Functional_Programming-Reference_Sheet"
%%% End:


\section{Monads}\label{sec:Monads}
So far, we have only written Haskell programs as \nameref{def:Batch_Program}s.
However, most people today like \nameref{def:Interactive_Program}s.

\begin{definition}[Batch Program]\label{def:Batch_Program}
  A \emph{batch program} is a program that takes some input from the user at the beginning of program execution and returns some result afterwards.
  During the computations, there is no further interaction between user and program.
\end{definition}

\begin{definition}[Monad]\label{def:Monad}
  \emph{Monads} are fairly unique to Haskell.
\end{definition}

%%% Local Variables:
%%% mode: latex
%%% TeX-master: "../EDAN40-Functional_Programming-Reference_Sheet"
%%% End:


%====================================APPENDIX====================================
\appendix
\counterwithin{definition}{subsection}

\clearpage
\section{Complex Numbers}
	\begin{equation} \label{eq:Exponential to Rectangular}
		A e^{-ix} = A \left[ \cos \left( x \right) + i\sin \left( x \right) \right]
	\end{equation}

\clearpage
\subsection{Trigonometry} \label{app:Trig}
	\subsubsection{Trigonometric Formulas} \label{subsubsec:Trig Formulas}
		\begin{equation} \label{eq:Sin plus Sin with diff Angles}
			\sin \left( \alpha \right) + \sin \left( \beta \right) = 2 \sin \left( \frac{\alpha + \beta}{2} \right) \cos\left( \frac{\alpha - \beta}{2} \right)  
		\end{equation}
		\begin{equation} \label{eq:Cosine-Sine Product}
			\cos \left( \theta \right) \sin \left( \theta \right) = \frac{1}{2} \sin \left( 2 \theta \right)
		\end{equation}

\clearpage
\subsection{Calculus} \label{app:Calculus}
	\subsubsection{Fundamental Theorems of Calculus} \label{subsubsec:Fundamental Theorem of Calculus}
		\begin{definition}[First Fundamental Theorem of Calculus] \label{def:1st Fundamental Theorem of Calculus}
			The \emph{first fundamental theorem of calculus} states that, if $f$ is continuous on the closed interval $\left[ a,b \right]$ and $F$ is the indefinite integral of $f$ on $\left[ a,b \right]$, then 
			\begin{equation} \label{eq:1st Fundamental Theorem of Calculus}
				\int_{a}^{b}f \left( x \right) dx = F \left( b \right) - F \left( a \right)
			\end{equation}
		\end{definition}
		\begin{definition}[Second Fundamental Theorem of Calculus] \label{def:2nd Fundamental Theorem of Calculus}
			The \emph{second fundamental theorem of calculus} holds for $f$ a continuous function on an open interval $I$ and $a$ any point in $I$, and states that if $F$ is defined by
			\begin{equation*}
				F \left( x \right) = \int_{a}^{x} f \left( t \right) dt,
			\end{equation*}
			then
			\begin{equation} \label{eq:2nd Fundamental Theorem of Calculus}
				\begin{aligned}
					\frac{d}{dx} \int_{a}^{x} f \left( t \right) dt &= f \left( x \right) \\
					F' \left( x \right) &= f \left( x \right) \\
				\end{aligned}
			\end{equation}
		\end{definition}

\clearpage
\section{Laplace Transform}\label{app:Laplace_Transform}
\subsection{Laplace Transform}\label{subsec:Laplace_Transform}
\begin{definition}[Laplace Transform]\label{def:Laplace_Transform}
  The \emph{Laplace transformation} operation is denoted as $\Lapl \lbrace x(t) \rbrace$ and is defined as
  \begin{equation}\label{eq:Laplace_Transform}
    X(s) = \int\limits_{-\infty}^{\infty} x(t) e^{-st} dt
  \end{equation}
\end{definition}

\subsection{Inverse Laplace Transform}\label{subsec:Inverse_Laplace_Transform}
\begin{definition}[Inverse Laplace Transform]\label{def:Inverse_Laplace_Transform}
  The \emph{inverse Laplace transformation} operation is denoted as $\Lapl^{-1} \lbrace X(s) \rbrace$ and is defined as
  \begin{equation}\label{eq:Inverse_Laplace_Transform}
    x(t) = \frac{1}{2j \pi} \int_{\sigma-\infty}^{\sigma+\infty} X(s) e^{st} \, ds
  \end{equation}
\end{definition}



% To make this print, you must include a citation somewhere in the document
\clearpage
\printbibliography{}
\end{document}