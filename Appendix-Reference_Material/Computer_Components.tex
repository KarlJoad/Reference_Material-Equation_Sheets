\section{Computer Components}\label{app:Computer_Components}
\subsection{Central Processing Unit}\label{subsec:CPU}
\begin{definition}[Central Processing Unit]\label{def:CPU}
  The \emph{Central Processing Unit}, \emph{CPU}, is a chip that performs all actions in the computer.
  It calculates mathematical and logical values and acts based on them.
  It has several components built onto it, and can be thought of as the ``brain'' of the computer.

  The design of a CPU determines some of the functionality it has.
  Therefore, more specialized processors can be made for special tasks, and more general processors can be built to handle a wide variety of calculations.
\end{definition}

\subsubsection{Registers}\label{subsubsec:Registers}
\begin{definition}[Register]\label{def:Register}
  A \emph{register} is a data storage mechanism built directly onto the \nameref{def:CPU}.
  It is several hundred times faster than the system \nameref{def:Memory}.
  Registers are generally used when the currently running program is performing calculations.
  Since they are so fast, they are used as both source and destination operands in instructions.

  \begin{remark}
    Depending on the \nameref{def:CPU} architecture, there may be cases when \nameref{def:Register}s behave slightly differently between processors.
    This is something that can only be found by checking the \nameref{def:CPU} manufacturer's documentation.
  \end{remark}
\end{definition}

\subsubsection{Program Counter}\label{subsubsec:Program_Counter}
\subsubsection{Arithmetic Logic Unit}\label{subsubsec:ALU}
\subsubsection{Cache}\label{subsubsec:CPU_Cache}

\subsection{Memory}\label{subsec:Memory}
\begin{definition}[Memory]\label{def:Memory}
  \emph{Memory}, or \emph{RAM} (\emph{Random Access Memory}), is a \nameref{def:Volatile} data storage mechanism.
  It is directly connected to the \nameref{def:CPU}.
  This is the location that the \nameref{def:CPU} writes to when it cannot or should not store something in the \nameref{def:CPU}'s \nameref{def:Register}s.

  \begin{remark}[Volatility]
    \nameref{def:Memory} is volatile because each of the cells is a small capacitor.
    In between the clock cycles on the \nameref{def:CPU} and \nameref{def:Memory}, the capacitors discharge.
    On the clock cycle, the capacitors are refreshed with electrical power, which does one of 2 things:
    \begin{enumerate}[noitemsep]
    \item Keep the data bits the same, 1 to 1.
    \item Update the data bits from 0 to 1.
    \end{enumerate}
  \end{remark}
\end{definition}

\begin{definition}[Volatile]\label{def:Volatile}
  If a data storage mechanism is called \emph{volatile}, it means that once the storage mechanism loses power, the data is lost.
  This is in contrast to \nameref{def:Non-Volatile} data storage mechanisms.
\end{definition}

\subsection{Disk}\label{subsec:Disk}
\begin{definition}[Non-Volatile]\label{def:Non-Volatile}
  If a data storage mechanism is called \emph{non-volatile}, it means that once the storage device loses power, the data is still safely stored.
  This is in contrast to \nameref{def:Volatile} data storage mechanisms.
\end{definition}

\subsection{Fetch-Execute Cycle}\label{subsec:Fetch_Execute_Cycle}
%%% Local Variables:
%%% mode: latex
%%% TeX-master: shared
%%% End:
