\section{Complex Numbers}\label{sec:Complex_Numbers}
\begin{definition}[Complex Number]\label{def:Complex_Number}
  A \emph{complex number} is a hyper real number system.
  This means that two real numbers, $a, b \in \RealNumbers$, are used to construct the set of complex numbers, denoted $\ComplexNumbers$.

  A complex number is written, in Cartesian form, as shown in \Cref{eq:Complex_Number} below.
  \begin{equation}\label{eq:Complex_Number}
    z = a \pm ib
  \end{equation}
  where
  \begin{equation}\label{eq:Imaginary_Value}
    i = \sqrt{-1}
  \end{equation}

  \begin{remark*}[$i$ vs. $j$ for Imaginary Numbers]
    Complex numbers are generally denoted with either $i$ or $j$.
    Electrical engineering regularly makes use of $j$ as the imaginary value.
    This is because alternating current $i$ is already taken, so $j$ is used as the imaginary value instad.
  \end{remark*}
\end{definition}

\subsection{Parts of a Complex Number}\label{subsec:Complex_Number_Parts}
A \nameref{def:Complex_Number} is made of up 2 parts:
\begin{enumerate}[noitemsep]
\item \nameref{def:Real_Part}
\item \nameref{def:Imaginary_Part}
\end{enumerate}

\begin{definition}[Real Part]\label{def:Real_Part}
  The \emph{real part} of an imaginary number, denoted with the $\Re$ operator, is the portion of the \nameref{def:Complex_Number} with no part of the imaginary value $i$ present.

  If $z = x + iy$, then
  \begin{equation}\label{eq:Real_Part}
    \Real{z} = x
  \end{equation}

  \begin{remark}[Alternative Notation]\label{rmk:Real_Part_Alternative_Notation}
    The \nameref{def:Real_Part} of a number sometimes uses a slightly different symbol for denoting the operation.
    It is:
    \begin{equation*}
      \mathfrak{Re}
    \end{equation*}
  \end{remark}
\end{definition}

\begin{definition}[Imaginary Part]\label{def:Imaginary_Part}
  The \emph{imaginary part} of an imaginary number, denoted with the $\Im$ operator, is the portion of the \nameref{def:Complex_Number} where the imaginary value $i$ is present.

  If $z = x + iy$, then
  \begin{equation}\label{eq:Imaginary_Part}
    \Imag{z} = y
  \end{equation}

  \begin{remark}[Alternative Notation]\label{rmk:Imaginary_Part_Alternative_Notation}
    The \nameref{def:Imaginary_Part} of a number sometimes uses a slightly different symbol for denoting the operation.
    It is:
    \begin{equation*}
      \mathfrak{Im}
    \end{equation*}
  \end{remark}
\end{definition}

\subsection{Binary Operations}\label{subsec:Binary_Operations}
The question here is if we are given 2 complex numbers, how should these binary operations work such that we end up with just one resulting complex number.
There are only 2 real operations that we need to worry about, and the other 3 can be defined in terms of these two:
\begin{enumerate}[noitemsep]
\item \nameref{subsubsec:Complex_Number-Addition}
\item \nameref{subsubsec:Complex_Number-Multiplication}
\end{enumerate}

For the sections below, assume:
\begin{align*}
  z &= x_{1} + iy_{1} \\
  w &= x_{2} + iy_{2}
\end{align*}

\subsubsection{Addition}\label{subsubsec:Complex_Number-Addition}
The addition operation, still denoted with the $+$ symbol is done pairwise.
You should treat $i$ like a variable in regular algebra, and not move it around.

\begin{equation}\label{eq:Complex_Number-Addition}
  z+w \coloneqq (x_{1}+x_{2}) + i(y_{1}+y_{2})
\end{equation}

\subsubsection{Multiplication}\label{subsubsec:Complex_Number-Multiplication}
The multiplication operation, like in traditional algebra, usually lacks a multiplication symbol.
You should treat $i$ like a variable in regular algebra, and not move it around.

\begin{equation}\label{eq:Complex_Number-Multiplication}
  \begin{aligned}
    zw &\coloneqq (x_{1} + iy_{1}) (x_{2} + iy_{2}) \\
    &= (x_{1}x_{2}) + (iy_{1}x_{2}) + (ix_{1}y_{2}) + (i^{2}y_{1}y_{2}) \\
    &= (x_{1}x_{2}) + i(y_{1}x_{2} + x_{1}y_{2}) + (-1 y_{1}y_{2}) \\
    &= (x_{1}x_{2} - y_{1}y_{2}) + i(y_{1}x_{2} + x_{1}y_{2}) \\
  \end{aligned}
\end{equation}

%%% Local Variables:
%%% mode: latex
%%% TeX-master: shared
%%% End:


\subsection{Complex Conjugates}\label{app:Complex_Conjugates}
\begin{definition}[Complex Conjugate]\label{def:Complex_Conjugate}
  The conjugate of a complex number is called its \emph{complex conjugate}.
  The complex conjugate of a complex number is the number with an equal real part and an imaginary part equal in magnitude but opposite in sign.
  If we have a complex number as shown below,
  \begin{equation*}
    z = a \pm bi
  \end{equation*}

  then, the conjugate is denoted and calculated as shown below.
  \begin{equation}\label{eq:Complex_Conjugates}
    \Conjugate{z} = a \mp bi
  \end{equation}
\end{definition}

The \nameref{def:Complex_Conjugate} can also be denoted with an asterisk ($*$).
This is generally done for complex functions, rather than single variables.
\begin{equation}\label{eq:Complex_Conjugates_Asterisk}
  z^{*} = \Conjugate{z}
\end{equation}

%%% Local Variables:
%%% mode: latex
%%% TeX-master: shared
%%% End:


\subsection{Geometry of Complex Numbers}\label{subsec:Geometry_Complex_Numbers}

%%% Local Variables:
%%% mode: latex
%%% TeX-master: shared
%%% End:


\subsection{Circles and Complex Numbers}\label{subsec:Circles_Complex_Numbers}

%%% Local Variables:
%%% mode: latex
%%% TeX-master: shared
%%% End:


\subsection{Polar Form}\label{subsec:Polar_Form}
The polar form of a \nameref{def:Complex_Number} is an alternative, but equally useful way to express a complex number.
In polar form, we express the distance the complex number is from the origin and the angle it sits at from the real axis.
This is seen in \Cref{eq:Polar_Form}.
\begin{equation}\label{eq:Polar_Form}
  z = r \bigl( \cos(\theta) + i \sin(\theta) \bigr)
\end{equation}

\begin{remark*}
  Note that in the definition of polar form (\Cref{eq:Polar_Form}), there is no allowance for the radius, $r$, to be negative.
  You must fix this by figuring out the angle change that is required for the radius to become positive.
\end{remark*}

Thus,
\begin{align*}
  r &= \lvert z \rvert \\
  \theta &= \arg(z) \\
\end{align*}


%%% Local Variables:
%%% mode: latex
%%% TeX-master: shared
%%% End:


\subsection{Roots of a Complex Number}\label{subsec:Complex_Roots}
\nameref{def:de_Moivers_Law} also applies to finding \textbf{roots} of a \nameref{def:Complex_Number}.

\begin{equation}\label{eq:Complex_Roots}
  z^{\frac{1}{n}} = r^{\frac{1}{n}} \biggl( \cos \left( \frac{\arg{z}}{n} \right) + i \sin \left( \frac{\arg{z}}{n} \right) \biggr)
\end{equation}

\begin{remark*}
  As the entire $\arg{z}$ term is being divided by $n$, the $2 \pi k$ is \textbf{ALSO} divided by $n$.
\end{remark*}


%%% Local Variables:
%%% mode: latex
%%% TeX-master: shared
%%% End:


\subsection{Arguments}\label{subsec:Complex_Number_Arguments}
There are 2 types of arguments that we can talk about for a \nameref{def:Complex_Number}.
\begin{enumerate}[noitemsep]
\item The \nameref{def:Complex_Number_Argument}
\item The \nameref{def:Principal_Argument}
\end{enumerate}


%%% Local Variables:
%%% mode: latex
%%% TeX-master: shared
%%% End:


\subsection{Complex Exponentials}\label{subsec:Complex_Exponentials}
The definition of an exponential with a \nameref{def:Complex_Number} as its exponent is defined in \Cref{eq:Complex_Exponential}.
\begin{equation}\label{eq:Complex_Exponential}
  e^{z} = e^{x + iy} = e^{x} \bigl( \cos(y) + i \sin(y) \bigr)
\end{equation}

If instead of $e$ as the base, we have some value $a$, then we have \Cref{eq:Complex_Exponential_Diff_Base}.
\begin{equation}\label{eq:Complex_Exponential_Diff_Base}
  \begin{aligned}
    a^{z} &= e^{z \ln(a)} \\
    &= e^{\Real{z \ln(a)}} \Bigl( \cos \bigl(\Imag{z \ln(a)} \bigr) + i \sin \bigl(\Imag{z \ln(a)} \bigl) \Bigr)
  \end{aligned}
\end{equation}

In the case of \Cref{eq:Complex_Exponential}, $z$ can be presented in either Cartesian or polar form, they are equivalent.

\begin{example}[Lecture 3]{Simplify Simple Complex Exponential}
  Simplify the expression below, then find its \nameref{def:Complex_Number_Modulus}, \nameref{def:Complex_Number_Argument}, and its \nameref{def:Principal_Argument}?
  \begin{equation*}
    e^{-1 + i\sqrt{3}}
  \end{equation*}
  \tcblower{}
  If we look at the exponent on the exponential, we see
  \begin{equation*}
    z = -1 + i\sqrt{3}
  \end{equation*}
  which means
  \begin{align*}
    x &= -1 \\
    y &= \sqrt{3}
  \end{align*}

  With this information, we can simplify the expression \textbf{just} by observation, with no calculations required.
  \begin{equation*}
    e^{-1 + i\sqrt{3}} = e^{-1} \bigl( \cos(\sqrt{3}) + i \sin(\sqrt{3}) \bigr)
  \end{equation*}

  Now, we can solve the other 3 parts of this example \textbf{by observation}.
  \begin{align*}
    \Modulus{e^{-1 + i\sqrt{3}}} &= \Modulus{e^{-1} \bigl( \cos(\sqrt{3}) + i \sin(\sqrt{3}) \bigr)} \\
                                 &= e^{-1} \\
    \arg \left( e^{-1 + i\sqrt{3}} \right) &= \arg \left( e^{-1} \bigl( \cos(\sqrt{3}) + i \sin(\sqrt{3}) \bigr) \right) \\
                                 &= \sqrt{3} + 2 \pi k \\
    \PrincipalArg \left( e^{-1 + i\sqrt{3}} \right) &= \PrincipalArg \left( e^{-1} \bigl( \cos(\sqrt{3}) + i \sin(\sqrt{3}) \bigr) \right) \\
                                 &= \sqrt{3}
  \end{align*}
\end{example}

\begin{example}[Lecture 3]{Simplify Complex Exponential Exponent}
  Given $z = e^{-e^{-i}}$, what is this expression in polar form, what is its \nameref{def:Complex_Number_Modulus}, its \nameref{def:Complex_Number_Argument}, and its \nameref{def:Principal_Argument}?
  \tcblower{}
  We start by simplifying the exponent of the base exponential, i.e.\ $e^{-i}$.
  \begin{align*}
    e^{-i} &= e^{0 - i} \\
           &= e^{0} \bigl( \cos(-1) + i \sin(-1) \bigr) \\
           &= 1 \bigl( \cos(-1) + i \sin(-1) \bigr)
  \end{align*}

  Now, with that exponent simplified, we can solve the main question.
  \begin{align*}
    e^{-e^{-i}} &= e^{-1 \bigl( \cos(-1) + i \sin(-1) \bigr)} \\
                &= e^{-1 \bigl( \cos(1) - i \sin(1) \bigr)} \\
    &= e^{-\cos(1) + i \sin(1)} \\
    \intertext{If we refer back to \Cref{eq:Complex_Exponential}, then it becomes obvious what $x$ and $y$ are.}
    x &= -\cos(1) \\
    y &= \sin(1) \\
    e^{-e^{-i}} &= e^{-\cos(1)} \Bigl( \cos \bigl( \sin(1) \bigr) + i \sin \bigl( \sin(1) \bigr) \Bigr)
  \end{align*}

  Now that we have ``simplified'' this exponential, we can solve the other 3 questions by \textbf{observation}.
  \begin{align*}
    \Modulus{e^{-e^{-i}}} &= \Modulus{e^{-\cos(1)} \Bigl( \cos \bigl( \sin(1) \bigr) + i \sin \bigl( \sin(1) \bigr) \Bigr)} \\
                          &= e^{-\cos(1)} \\
    \arg \left( e^{-e^{-i}} \right) &= \arg \left( e^{-\cos(1)} \Bigl( \cos \bigl( \sin(1) \bigr) + i \sin \bigl( \sin(1) \bigr) \Bigr) \right) \\
                          &= \sin(1) + 2 \pi k \\
    \PrincipalArg \left( e^{-e^{-i}} \right) &= \PrincipalArg \left( e^{-\cos(1)} \Bigl( \cos \bigl( \sin(1) \bigr) + i \sin \bigl( \sin(1) \bigr) \Bigr) \right) \\
                          &= \sin(1)
  \end{align*}
\end{example}

\begin{example}[Lecture 3]{Non-e Complex Exponential}
  Find all values of $z=1^{i}$?
  \tcblower{}
  Use \Cref{eq:Complex_Exponential_Diff_Base} to simplify this to a base of $e$, where we can use the usual \Cref{eq:Complex_Exponential} to solve this.
  \begin{align*}
    a^{z} &= e^{z \ln(a)} \\
    1^{i} &= e^{i \ln(1)} \\
    \intertext{Simplify the logarithm in the exponent first, $\ln(1)$.}
    \ln(1) &= \log_{e} \Modulus{1} + i \arg(1) \\
          &= \log_{e}(1) + i (0 + 2\pi k) \\
          &= 0 + 2\pi k i \\
          &= 2\pi k i \\
    \intertext{Now, plug $\ln(1)$ back into the exponent, and solve the exponential.}
    e^{i (2\pi k i)} &= e^{2\pi k i^{2}} \\
          &= e^{2\pi k (-1)} \\
    z &= e^{-2\pi k}
  \end{align*}

  Thus, all values of $z = e^{-2\pi k}$ where $k = 0, 1, \ldots$.
\end{example}


%%% Local Variables:
%%% mode: latex
%%% TeX-master: shared
%%% End:


\subsection{Complex Logarithms}\label{subsec:Complex_Logarithms}
There are some denotational changes that need to be made for this to work.
The traditional real-number natural logarithm $\ln$ needs to be redefined to its defining form $\log_{e}$.

With that denotational change, we can now use $\ln$ for the \nameref{def:Complex_Logarithm}.


%%% Local Variables:
%%% mode: latex
%%% TeX-master: shared
%%% End:



%%% Local Variables:
%%% mode: latex
%%% TeX-master: shared
%%% End: