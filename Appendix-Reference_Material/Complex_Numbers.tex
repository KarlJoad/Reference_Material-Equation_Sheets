\section{Complex Numbers}\label{sec:Complex_Numbers}
\begin{definition}[Complex Number]\label{def:Complex_Number}
  A \emph{complex number} is a hyper real number system.
  This means that two real numbers, $a, b \in \RealNumbers$, are used to construct the set of complex numbers, denoted $\ComplexNumbers$.

  A complex number is written, in Cartesian form, as shown in \Cref{eq:Complex_Number} below.
  \begin{equation}\label{eq:Complex_Number}
    z = a \pm ib
  \end{equation}
  where
  \begin{equation}\label{eq:Imaginary_Value}
    i = \sqrt{-1}
  \end{equation}

  \begin{remark*}[$i$ vs. $j$ for Imaginary Numbers]
    Complex numbers are generally denoted with either $i$ or $j$.
    Electrical engineering regularly makes use of $j$ as the imaginary value.
    This is because alternating current $i$ is already taken, so $j$ is used as the imaginary value instad.
  \end{remark*}
\end{definition}

\subsection{Parts of a Complex Number}\label{subsec:Complex_Number_Parts}
A \nameref{def:Complex_Number} is made of up 2 parts:
\begin{enumerate}[noitemsep]
\item \nameref{def:Real_Part}
\item \nameref{def:Imaginary_Part}
\end{enumerate}

\begin{definition}[Real Part]\label{def:Real_Part}
  The \emph{real part} of an imaginary number, denoted with the $\Re$ operator, is the portion of the \nameref{def:Complex_Number} with no part of the imaginary value $i$ present.

  If $z = x + iy$, then
  \begin{equation}\label{eq:Real_Part}
    \Real{z} = x
  \end{equation}

  \begin{remark}[Alternative Notation]\label{rmk:Real_Part_Alternative_Notation}
    The \nameref{def:Real_Part} of a number sometimes uses a slightly different symbol for denoting the operation.
    It is:
    \begin{equation*}
      \mathfrak{Re}
    \end{equation*}
  \end{remark}
\end{definition}

\begin{definition}[Imaginary Part]\label{def:Imaginary_Part}
  The \emph{imaginary part} of an imaginary number, denoted with the $\Im$ operator, is the portion of the \nameref{def:Complex_Number} where the imaginary value $i$ is present.

  If $z = x + iy$, then
  \begin{equation}\label{eq:Imaginary_Part}
    \Imag{z} = y
  \end{equation}

  \begin{remark}[Alternative Notation]\label{rmk:Imaginary_Part_Alternative_Notation}
    The \nameref{def:Imaginary_Part} of a number sometimes uses a slightly different symbol for denoting the operation.
    It is:
    \begin{equation*}
      \mathfrak{Im}
    \end{equation*}
  \end{remark}
\end{definition}

\subsection{Binary Operations}\label{subsec:Binary_Operations}

%%% Local Variables:
%%% mode: latex
%%% TeX-master: shared
%%% End:


\subsection{Complex Conjugates}\label{app:Complex_Conjugates}
\begin{definition}[Complex Conjugate]\label{def:Complex_Conjugate}
  The conjugate of a complex number is called its \emph{complex conjugate}.
  The complex conjugate of a complex number is the number with an equal real part and an imaginary part equal in magnitude but opposite in sign.
  If we have a complex number as shown below,
  \begin{equation*}
    z = a \pm bi
  \end{equation*}

  then, the conjugate is denoted and calculated as shown below.
  \begin{equation}\label{eq:Complex_Conjugates}
    \Conjugate{z} = a \mp bi
  \end{equation}
\end{definition}

The \nameref{def:Complex_Conjugate} can also be denoted with an asterisk ($*$).
This is generally done for complex functions, rather than single variables.
\begin{equation}\label{eq:Complex_Conjugates_Asterisk}
  z^{*} = \Conjugate{z}
\end{equation}

%%% Local Variables:
%%% mode: latex
%%% TeX-master: shared
%%% End:


\subsection{Geometry of Complex Numbers}\label{subsec:Geometry_Complex_Numbers}
So far, we have viewed \nameref{def:Complex_Number}s only algebraically.
However, we can also view them geometrically as points on a 2 dimensional \nameref{def:Argand_Plane}.

\begin{definition}[Argand Plane]\label{def:Argand_Plane}
  An \emph{Argane Plane} is a standard two dimensional plane whose points are all elements of the complex numbers, $z \in \ComplexNumbers$.
  This is taken from Descarte's definition of a completely real plane.

  The Argand plane contains 2 lines that form the axes, that indicate the real component and the imaginary component of the complex number specified.
\end{definition}

A \nameref{def:Complex_Number} can be viewed as a point in the \nameref{def:Argand_Plane}, where the \nameref{def:Real_Part} is the ``$x$''-component and the \nameref{def:Imaginary_Part} is the ``$y$''-component.

By plotting this, you see that we form a right triangle, so we can find the hypotenuse of that triangle.
This hypotenuse is the distance the point $p$ is from the origin, refered to as the \nameref{def:Complex_Number_Modulus}.
\begin{remark*}
  When working with \nameref{def:Complex_Number}s geometrically, we refer to the points, where they are defined like so:
  \begin{equation*}
    z = x + iy = p(x, y)
  \end{equation*}

  Note that $p$ is \textbf{not} a function of $x$ and $y$.
  Those are the values that inform us \textbf{where} $p$ is located on the \nameref{def:Argand_Plane}.
\end{remark*}

\subsubsection{Modulus of a Complex Number}\label{subsubsec:Complex_Number_Modulus}
\begin{definition}[Modulus]\label{def:Complex_Number_Modulus}
  The \emph{modulus} of a \nameref{def:Complex_Number} is the distance from the origin to the complex point $p$.
  This is based off the Pythagorean Theorem.
  \begin{equation}\label{eq:Complex_Number_Modulus}
    \begin{aligned}
      {\lvert z \rvert}^{2} = x^{2} + y^{2} &= z \Conjugate{z} \\
      \lvert z \rvert &= \sqrt{x^{2} + y^{2}}
    \end{aligned}
  \end{equation}
\end{definition}

\begin{propertylist}
\item The \emph{Law of Moduli} states that $\lvert z w \rvert = \lvert z \rvert \lvert w \rvert$.\label{prop:Law_of_Moduli}.
\end{propertylist}

We can prove \Cref{prop:Law_of_Moduli} using an algebraic identity.
\begin{proof}[Prove \Cref*{prop:Law_of_Moduli}]
  Let $z$ and $w$ be complex numbers ($z, w \in \ComplexNumbers$).
  We are asked to prove
  \begin{equation*}
    \lvert z w \rvert = \lvert z \rvert \lvert w \rvert
  \end{equation*}

  But, it is actually easier to prove
  \begin{equation*}
    {\lvert z w \rvert}^{2} = {\lvert z \rvert}^{2} {\lvert w \rvert}^{2}
  \end{equation*}

  We start by simplifying the ${\lvert z w \rvert}^{2}$ equation above.
  \begin{align*}
    {\lvert z w \rvert}^{2} &= {\lvert z \rvert}^{2} {\lvert w \rvert}^{2} \\
    \intertext{Using the definition of the \nameref{def:Complex_Number_Modulus} of a \nameref{def:Complex_Number} in \Cref{eq:Complex_Number_Modulus}, we can expand the modulus.}
                            &= (z w) (\Conjugate{z w}) \\
    \intertext{Using \Cref{prop:Complex_Conjugate_Split} for multiplication allows us to do the next step.}
                            &= (z w) (\Conjugate{z} \Conjugate{w}) \\
    \intertext{Using Multiplicative Associativity and Multiplicative Commutativity, we can simplify this further.}
                            &= (z \Conjugate{z}) (w \Conjugate{w}) \\
                            &= {\lvert z \rvert}^{2} {\lvert w \rvert}^{2}
  \end{align*}

  Note how we never needed to define $z$ or $w$, so this is as general a result as possible.
\end{proof}

\paragraph{Algebraic Effects of the Modulus' \Cref*{prop:Law_of_Moduli}}\label{par:Law_of_Moduli-Algebraic_Effects}
For this section, let $z = x_{1} + iy_{1}$ and $w = x_{2} + iy_{2}$.
Now,
\begin{align*}
  z w &= (x_{1}x_{2} - y_{1}y_{2}) + i(x_{1}y_{2} + x_{2}y_{1}) \\
  {\lvert z w \rvert}^{2} &= {(x_{1}x_{2} - y_{1}y_{2})}^{2} + {(x_{1}y_{2} + x_{2}y_{1})}^{2} \\
      &= \left( x_{1}^{2} + x_{2}^{2} \right) \left( x_{2}^{2} + y_{2}^{2} \right) \\
      &= {\lvert z \rvert}^{2} {\lvert w \rvert}^{2}
\end{align*}

However, the Law of Moduli (\Cref{prop:Law_of_Moduli}) does \textbf{not} hold for a hyper complex number system one that uses 2 or more imaginaries, i.e.\ $z = a + iy + jz$.
But, the Law of Moduli (\Cref{prop:Law_of_Moduli}) \textbf{does} hold for hyper complex number system that uses 3 imaginaries, $a = z + iy + jz + k \ell$.

\paragraph{Conceptual Effects of the Modulus' \Cref*{prop:Law_of_Moduli}}\label{par:Law_of_Moduli-Conceptual_Effects}
We are interested in seeing if $\lvert z w \rvert = (x_{1}^{2} + y_{1}^{2})(x_{2}^{2}+y_{2}^{2})$ can be extended to more complex terms (3 terms in the complex number).

However, Langrange proved that the equation below \textbf{always} holds.
Note that the $z$ below has no relation to the $z$ above.
\begin{equation*}
  (x_{1} + y_{1} + z_{1}) \neq X^{2} + Y^{2} + Z^{2}
\end{equation*}

%%% Local Variables:
%%% mode: latex
%%% TeX-master: shared
%%% End:


\subsection{Circles and Complex Numbers}\label{subsec:Circles_Complex_Numbers}
We need to define both a center and a radius, just like with regular purely real values.
\Cref{eq:Circles_Complex_Numbers} defines the relation required for a circle using \nameref{def:Complex_Number}s.
\begin{equation}\label{eq:Circles_Complex_Numbers}
  \lvert z - a \rvert = r
\end{equation}

\begin{example}[Lecture 2, Example 1]{Convert to Circle}
  Given the expression below, find the location of the center of the circle and the radius of the circle?
  \begin{equation*}
    \lvert 5 iz + 10 \rvert = 7
  \end{equation*}
  \tcblower{}
  This is just a matter of simplification and moving terms around.
  \begin{align*}
    \lvert 5 iz + 10 \rvert &= 7 \\
    \lvert 5i (z + \frac{10}{5i}) \rvert &= 7 \\
    \lvert 5i (z + \frac{2}{i}) \rvert &= 7 \\
    \lvert 5i (z + \frac{2}{i} \frac{-i}{-i}) \rvert &= 7 \\
    \lvert 5i (z - 2i) \rvert &= 7 \\
    \intertext{Now using the Law of Moduli (\Cref{prop:Law_of_Moduli}) $\lvert a b \rvert = \lvert a \rvert \lvert b \rvert$, we can simplify out the extra imaginary term.}
    \lvert 5i \rvert \lvert z-2i \rvert &= 7 \\
    5 \lvert z - 2i \rvert &= 7 \\
    \lvert z - 2i \rvert = \frac{7}{5}
  \end{align*}

  Thus, the circle formed by the equation $\lvert 5 iz + 10 \rvert = 7$ is actually $\lvert z - 2i \rvert = \frac{7}{5}$, with a center at $a = 2i$ and a radius of $\frac{7}{5}$.
\end{example}

\subsubsection{Annulus}\label{subsubsec:Annulus}
\begin{definition}[Annulus]\label{def:Annulus}
  An \emph{annulus} is a region that is bounded by 2 concentric circles.
  This takes the form of \Cref{eq:Annulus}.
  \begin{equation}\label{eq:Annulus}
    r_{1} \leq \lvert z - a \rvert \leq r_{2}
  \end{equation}

  In \Cref{eq:Annulus}, each of the $\leq$ symbols could also be replaced with $<$.
  This leads to 3 different possibilities for the annulus:
  \begin{enumerate}[noitemsep]
  \item If both inequality symbols are $\leq$, then it is a \textbf{Closed Annulus}.
  \item If both inequality symbols are $<$, then it is an \textbf{Open Annulus}.
  \item If \textbf{only one} inequality symbol $<$ and the other $\leq$, then it is not an \textbf{Open Annulus}.
  \end{enumerate}
\end{definition}


%%% Local Variables:
%%% mode: latex
%%% TeX-master: shared
%%% End:


\subsection{Polar Form}\label{subsec:Polar_Form}
The polar form of a \nameref{def:Complex_Number} is an alternative, but equally useful way to express a complex number.
In polar form, we express the distance the complex number is from the origin and the angle it sits at from the real axis.
This is seen in \Cref{eq:Polar_Form}.
\begin{equation}\label{eq:Polar_Form}
  z = r \bigl( \cos(\theta) + i \sin(\theta) \bigr)
\end{equation}

\begin{remark*}
  Note that in the definition of polar form (\Cref{eq:Polar_Form}), there is no allowance for the radius, $r$, to be negative.
  You must fix this by figuring out the angle change that is required for the radius to become positive.
\end{remark*}

Thus,
\begin{align*}
  r &= \lvert z \rvert \\
  \theta &= \arg(z) \\
\end{align*}

\begin{example}[Lecture 2, Example 1]{Find Polar Coordinates from Cartesian Coordinates}
  Find the complex number's $z = -\sqrt{3} + i$ polar coordinates?
  \tcblower{}
  We start by finding the radius of $z$ (modulus of $z$).
  \begin{align*}
    r &= \lvert z \rvert \\
      &= \sqrt{\Real{z}^{2} + \Imag{z}^{2}} \\
      &= \sqrt{{(-\sqrt{3})}^{2} + 1^{2}} \\
      &= \sqrt{3 + 1} \\
      &= \sqrt{4} \\
      &= 2
  \end{align*}
  Thus, the point is $2$ units away from the origin, the radius is 2 $r=2$.

  Now, we need to find the angle, the argument, of the \nameref{def:Complex_Number}.
  \begin{align*}
    \cos(\theta) &= \frac{-\sqrt{3}}{2} \\
    \theta &= \cos^{-1} \left( \frac{-\sqrt{3}}{2} \right) \\
                 &= \frac{5 \pi}{6} \\
  \end{align*}

  Now that we have one angle for the point, we also need to consider the possibility that there have been an unknown amount of rotations around the entire plane, meaning there have been $2 \pi k$, where $k = 0, 1, \ldots$.

  We now have all the information required to reconstruct this point using polar coordinates:
  \begin{align*}
    r &= 2 \\
    \theta &= \frac{5 \pi}{6} \\
    \arg(z) &= \frac{5 \pi}{6} + 2\pi k
  \end{align*}
\end{example}

\subsubsection{Converting Between Cartesian and Polar Forms}\label{subsubsec:Convert_Cartesian_Polar}
Using \Cref{eq:Polar_Form} and \Cref{eq:Complex_Number}, it is easy to see the relation between $r$, $\theta$, $x$, and $y$.

\begin{align*}
  \shortintertext{Definition of a \nameref{def:Complex_Number} in Cartesian form.}
  z &= x + iy \\
  \shortintertext{Definition of a \nameref{def:Complex_Number} in polar form.}
  z &= r \bigl( \cos(\theta) + i \sin(\theta) \bigr) \\
    &= r \cos(\theta) + i r \sin(\theta) \\
\end{align*}

Thus,
\begin{equation}\label{eq:Convert_Cartesian_Polar}
  \begin{aligned}
    x &= r \cos(\theta) \\
    y &= r \sin(\theta)
  \end{aligned}
\end{equation}


%%% Local Variables:
%%% mode: latex
%%% TeX-master: shared
%%% End:


\subsection{Roots of a Complex Number}\label{subsec:Complex_Roots}

%%% Local Variables:
%%% mode: latex
%%% TeX-master: shared
%%% End:


\subsection{Arguments}\label{subsec:Complex_Number_Arguments}
There are 2 types of arguments that we can talk about for a \nameref{def:Complex_Number}.
\begin{enumerate}[noitemsep]
\item The \nameref{def:Complex_Number_Argument}
\item The \nameref{def:Principal_Argument}
\end{enumerate}

\begin{definition}[Argument]\label{def:Complex_Number_Argument}
  The \emph{argument} of a \nameref{def:Complex_Number} refers to \textbf{all} possible angles that can satisfy the angle requirement of a \nameref{def:Complex_Number}.
\end{definition}

\begin{example}[Lecture 3, Example 1]{Argument of Complex Number}
  If $z = -1 - i$, then what is its \textbf{\nameref{def:Complex_Number_Argument}}?
  \tcblower{}
  You can plot this value on the \nameref{def:Argand_Plane} and find the angle graphically/geometrically, or you can ``cheat'' and use $\tan^{-1}$ (so long as you correct for the proper quadrant).
  I will ``cheat'', as I cannot plot easily.
  \begin{align*}
    z &= -1 - i \\
    \arg(z) &= \tan(\theta) = \frac{-i}{-1} \\
      &= \frac{\pi}{4} \\
    \shortintertext{Remember to correct for the proper quadrant. We are in quadrant IV.}
      &= \frac{5 \pi}{4} \\
    \intertext{Now, we have to account for \textbf{all} possible angles that form this angle.}
    \arg(z) &= \frac{5 \pi}{4} + 2 \pi k
  \end{align*}

  Thus, the argument of $z = -1 - i$ is $\arg(z) = \frac{5 \pi}{4} + 2 \pi k$.
\end{example}

\begin{definition}[Principal Argument]\label{def:Principal_Argument}
  The \emph{principal argument} is the exact or reference angle of the \nameref{def:Complex_Number}.
  By convention, the principal Argument of a complex number $z$ is defined to be bounded like so: $-\pi < \PrincipalArg(z) \leq \pi$.
\end{definition}


%%% Local Variables:
%%% mode: latex
%%% TeX-master: shared
%%% End:


\subsection{Complex Exponentials}\label{subsec:Complex_Exponentials}
The definition of an exponential with a \nameref{def:Complex_Number} as its exponent is defined in \Cref{eq:Complex_Exponential}.
\begin{equation}\label{eq:Complex_Exponential}
  e^{z} = e^{x + iy} = e^{x} \bigl( \cos(y) + i \sin(y) \bigr)
\end{equation}

If instead of $e$ as the base, we have some value $a$, then we have \Cref{eq:Complex_Exponential_Diff_Base}.
\begin{equation}\label{eq:Complex_Exponential_Diff_Base}
  \begin{aligned}
    a^{z} &= e^{z \ln(a)} \\
    &= e^{\Real{z \ln(a)}} \Bigl( \cos \bigl(\Imag{z \ln(a)} \bigr) + i \sin \bigl(\Imag{z \ln(a)} \bigl) \Bigr)
  \end{aligned}
\end{equation}

In the case of \Cref{eq:Complex_Exponential}, $z$ can be presented in either Cartesian or polar form, they are equivalent.

\begin{example}[Lecture 3]{Simplify Simple Complex Exponential}
  Simplify the expression below, then find its \nameref{def:Complex_Number_Modulus}, \nameref{def:Complex_Number_Argument}, and its \nameref{def:Principal_Argument}?
  \begin{equation*}
    e^{-1 + i\sqrt{3}}
  \end{equation*}
  \tcblower{}
  If we look at the exponent on the exponential, we see
  \begin{equation*}
    z = -1 + i\sqrt{3}
  \end{equation*}
  which means
  \begin{align*}
    x &= -1 \\
    y &= \sqrt{3}
  \end{align*}

  With this information, we can simplify the expression \textbf{just} by observation, with no calculations required.
  \begin{equation*}
    e^{-1 + i\sqrt{3}} = e^{-1} \bigl( \cos(\sqrt{3}) + i \sin(\sqrt{3}) \bigr)
  \end{equation*}

  Now, we can solve the other 3 parts of this example \textbf{by observation}.
  \begin{align*}
    \Modulus{e^{-1 + i\sqrt{3}}} &= \Modulus{e^{-1} \bigl( \cos(\sqrt{3}) + i \sin(\sqrt{3}) \bigr)} \\
                                 &= e^{-1} \\
    \arg \left( e^{-1 + i\sqrt{3}} \right) &= \arg \left( e^{-1} \bigl( \cos(\sqrt{3}) + i \sin(\sqrt{3}) \bigr) \right) \\
                                 &= \sqrt{3} + 2 \pi k \\
    \PrincipalArg \left( e^{-1 + i\sqrt{3}} \right) &= \PrincipalArg \left( e^{-1} \bigl( \cos(\sqrt{3}) + i \sin(\sqrt{3}) \bigr) \right) \\
                                 &= \sqrt{3}
  \end{align*}
\end{example}

\begin{example}[Lecture 3]{Simplify Complex Exponential Exponent}
  Given $z = e^{-e^{-i}}$, what is this expression in polar form, what is its \nameref{def:Complex_Number_Modulus}, its \nameref{def:Complex_Number_Argument}, and its \nameref{def:Principal_Argument}?
  \tcblower{}
  We start by simplifying the exponent of the base exponential, i.e.\ $e^{-i}$.
  \begin{align*}
    e^{-i} &= e^{0 - i} \\
           &= e^{0} \bigl( \cos(-1) + i \sin(-1) \bigr) \\
           &= 1 \bigl( \cos(-1) + i \sin(-1) \bigr)
  \end{align*}

  Now, with that exponent simplified, we can solve the main question.
  \begin{align*}
    e^{-e^{-i}} &= e^{-1 \bigl( \cos(-1) + i \sin(-1) \bigr)} \\
                &= e^{-1 \bigl( \cos(1) - i \sin(1) \bigr)} \\
    &= e^{-\cos(1) + i \sin(1)} \\
    \intertext{If we refer back to \Cref{eq:Complex_Exponential}, then it becomes obvious what $x$ and $y$ are.}
    x &= -\cos(1) \\
    y &= \sin(1) \\
    e^{-e^{-i}} &= e^{-\cos(1)} \Bigl( \cos \bigl( \sin(1) \bigr) + i \sin \bigl( \sin(1) \bigr) \Bigr)
  \end{align*}

  Now that we have ``simplified'' this exponential, we can solve the other 3 questions by \textbf{observation}.
  \begin{align*}
    \Modulus{e^{-e^{-i}}} &= \Modulus{e^{-\cos(1)} \Bigl( \cos \bigl( \sin(1) \bigr) + i \sin \bigl( \sin(1) \bigr) \Bigr)} \\
                          &= e^{-\cos(1)} \\
    \arg \left( e^{-e^{-i}} \right) &= \arg \left( e^{-\cos(1)} \Bigl( \cos \bigl( \sin(1) \bigr) + i \sin \bigl( \sin(1) \bigr) \Bigr) \right) \\
                          &= \sin(1) + 2 \pi k \\
    \PrincipalArg \left( e^{-e^{-i}} \right) &= \PrincipalArg \left( e^{-\cos(1)} \Bigl( \cos \bigl( \sin(1) \bigr) + i \sin \bigl( \sin(1) \bigr) \Bigr) \right) \\
                          &= \sin(1)
  \end{align*}
\end{example}

\begin{example}[Lecture 3]{Non-e Complex Exponential}
  Find all values of $z=1^{i}$?
  \tcblower{}
  Use \Cref{eq:Complex_Exponential_Diff_Base} to simplify this to a base of $e$, where we can use the usual \Cref{eq:Complex_Exponential} to solve this.
  \begin{align*}
    a^{z} &= e^{z \ln(a)} \\
    1^{i} &= e^{i \ln(1)} \\
    \intertext{Simplify the logarithm in the exponent first, $\ln(1)$.}
    \ln(1) &= \log_{e} \Modulus{1} + i \arg(1) \\
          &= \log_{e}(1) + i (0 + 2\pi k) \\
          &= 0 + 2\pi k i \\
          &= 2\pi k i \\
    \intertext{Now, plug $\ln(1)$ back into the exponent, and solve the exponential.}
    e^{i (2\pi k i)} &= e^{2\pi k i^{2}} \\
          &= e^{2\pi k (-1)} \\
    z &= e^{-2\pi k}
  \end{align*}

  Thus, all values of $z = e^{-2\pi k}$ where $k = 0, 1, \ldots$.
\end{example}

\subsubsection{Complex Conjugates of Exponentials}\label{app:Exponential_Complex_Conjugates}
\begin{equation}\label{eq:Exponential_Complex_Conjugate}
  \Conjugate{e^{z}} = e^{\Conjugate{z}}
\end{equation}

%%% Local Variables:
%%% mode: latex
%%% TeX-master: shared
%%% End:



%%% Local Variables:
%%% mode: latex
%%% TeX-master: shared
%%% End: