\section{Complex Numbers}\label{sec:Complex_Numbers}
\begin{definition}[Complex Number]\label{def:Complex_Number}
  A \emph{complex number} is a hyper real number system.
  This means that two real numbers, $a, b \in \RealNumbers$, are used to construct the set of complex numbers, denoted $\ComplexNumbers$.

  A complex number is written, in Cartesian form, as shown in \Cref{eq:Complex_Number} below.
  \begin{equation}\label{eq:Complex_Number}
    z = a \pm ib
  \end{equation}
  where
  \begin{equation}\label{eq:Imaginary_Value}
    i = \sqrt{-1}
  \end{equation}

  \begin{remark*}[$i$ vs. $j$ for Imaginary Numbers]
    Complex numbers are generally denoted with either $i$ or $j$.
    Electrical engineering regularly makes use of $j$ as the imaginary value.
    This is because alternating current $i$ is already taken, so $j$ is used as the imaginary value instad.
  \end{remark*}
\end{definition}

\subsection{Parts of a Complex Number}\label{subsec:Complex_Number_Parts}
A \nameref{def:Complex_Number} is made of up 2 parts:
\begin{enumerate}[noitemsep]
\item \nameref{def:Real_Part}
\item \nameref{def:Imaginary_Part}
\end{enumerate}

\begin{definition}[Real Part]\label{def:Real_Part}
  The \emph{real part} of an imaginary number, denoted with the $\Re$ operator, is the portion of the \nameref{def:Complex_Number} with no part of the imaginary value $i$ present.

  If $z = x + iy$, then
  \begin{equation}\label{eq:Real_Part}
    \Real{z} = x
  \end{equation}

  \begin{remark}[Alternative Notation]\label{rmk:Real_Part_Alternative_Notation}
    The \nameref{def:Real_Part} of a number sometimes uses a slightly different symbol for denoting the operation.
    It is:
    \begin{equation*}
      \mathfrak{Re}
    \end{equation*}
  \end{remark}
\end{definition}

\begin{definition}[Imaginary Part]\label{def:Imaginary_Part}
  The \emph{imaginary part} of an imaginary number, denoted with the $\Im$ operator, is the portion of the \nameref{def:Complex_Number} where the imaginary value $i$ is present.

  If $z = x + iy$, then
  \begin{equation}\label{eq:Imaginary_Part}
    \Imag{z} = y
  \end{equation}

  \begin{remark}[Alternative Notation]\label{rmk:Imaginary_Part_Alternative_Notation}
    The \nameref{def:Imaginary_Part} of a number sometimes uses a slightly different symbol for denoting the operation.
    It is:
    \begin{equation*}
      \mathfrak{Im}
    \end{equation*}
  \end{remark}
\end{definition}

\subsection{Binary Operations}\label{subsec:Binary_Operations}

%%% Local Variables:
%%% mode: latex
%%% TeX-master: shared
%%% End:


\subsection{Complex Conjugates}\label{app:Complex_Conjugates}
\begin{definition}[Complex Conjugate]\label{def:Complex_Conjugate}
  The conjugate of a complex number is called its \emph{complex conjugate}.
  The complex conjugate of a complex number is the number with an equal real part and an imaginary part equal in magnitude but opposite in sign.
  If we have a complex number as shown below,
  \begin{equation*}
    z = a \pm bi
  \end{equation*}

  then, the conjugate is denoted and calculated as shown below.
  \begin{equation}\label{eq:Complex_Conjugates}
    \Conjugate{z} = a \mp bi
  \end{equation}
\end{definition}

The \nameref{def:Complex_Conjugate} can also be denoted with an asterisk ($*$).
This is generally done for complex functions, rather than single variables.
\begin{equation}\label{eq:Complex_Conjugates_Asterisk}
  z^{*} = \Conjugate{z}
\end{equation}

%%% Local Variables:
%%% mode: latex
%%% TeX-master: shared
%%% End:



%%% Local Variables:
%%% mode: latex
%%% TeX-master: shared
%%% End: