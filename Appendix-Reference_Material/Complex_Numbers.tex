\section{Complex Numbers}\label{sec:Complex_Numbers}
\begin{definition}[Complex Number]\label{def:Complex_Number}
  A \emph{complex number} is a hyper real number system.
  This means that two real numbers, $a, b \in \RealNumbers$, are used to construct the set of complex numbers, denoted $\ComplexNumbers$.

  A complex number is written, in Cartesian form, as shown in \Cref{eq:Complex_Number} below.
  \begin{equation}\label{eq:Complex_Number}
    z = a \pm ib
  \end{equation}
  where
  \begin{equation}\label{eq:Imaginary_Value}
    i = \sqrt{-1}
  \end{equation}

  \begin{remark*}[$i$ vs. $j$ for Imaginary Numbers]
    Complex numbers are generally denoted with either $i$ or $j$.
    Electrical engineering regularly makes use of $j$ as the imaginary value.
    This is because alternating current $i$ is already taken, so $j$ is used as the imaginary value instad.
  \end{remark*}
\end{definition}

\end{definition}

The complex conjugate can also be denoted with an asterisk ($*$).
This is generally done for complex functions, rather than single variables.
\begin{equation}\label{eq:Complex_Conjugates_Asterisk}
  z^{*} = \overline{z}
\end{equation}

\subsubsection{Complex Conjugates of Exponentials}\label{app:Exponential_Complex_Conjugates}
\begin{equation}\label{eq:Exponential_Complex_Conjugates-e}
  \overline{e^{z}} = e^{\overline{z}}
\end{equation}

\begin{equation}\label{eq:Exponential_Complex_Conjugates-log}
  \overline{\log(z)} = \log(\overline{z})
\end{equation}

\subsubsection{Complex Conjugates of Sinusoids}\label{app:Sinusoid_Complex_Conjugates}
Since sinusoids can be represented by complex exponentials, as shown in \Cref{subsec:Euler Equivalents}, we could calculate their complex conjugate.

\begin{equation}\label{eq:Sinusoid_Complex_Conjugate-Cosine}
  \begin{aligned}
    \overline{\cos(x)} &= \cos(x) \\
    &= \frac{1}{2} \left( e^{ix} + e^{-ix} \right) \\
  \end{aligned}
\end{equation}

\begin{equation}\label{eq:Sinusoid_Complex_Conjugate-Sine}
  \begin{aligned}
    \overline{\sin(x)} &= \sin(x) \\
    &= \frac{1}{2i} \left( e^{ix} - e^{-ix} \right) \\
  \end{aligned}
\end{equation}

%%% Local Variables:
%%% mode: latex
%%% TeX-master: shared
%%% End: