\subsection{Complex Exponentials}\label{subsec:Complex_Exponentials}
The definition of an exponential with a \nameref{def:Complex_Number} as its exponent is defined in \Cref{eq:Complex_Exponential}.
\begin{equation}\label{eq:Complex_Exponential}
  e^{z} = e^{x + iy} = e^{x} \bigl( \cos(y) + i \sin(y) \bigr)
\end{equation}

If instead of $e$ as the base, we have some value $a$, then we have \Cref{eq:Complex_Exponential_Diff_Base}.
\begin{equation}\label{eq:Complex_Exponential_Diff_Base}
  \begin{aligned}
    a^{z} &= e^{z \ln(a)} \\
    &= e^{\Real{z \ln(a)}} \Bigl( \cos \bigl(\Imag{z \ln(a)} \bigr) + i \sin \bigl(\Imag{z \ln(a)} \bigl) \Bigr)
  \end{aligned}
\end{equation}

In the case of \Cref{eq:Complex_Exponential}, $z$ can be presented in either Cartesian or polar form, they are equivalent.


%%% Local Variables:
%%% mode: latex
%%% TeX-master: shared
%%% End:
