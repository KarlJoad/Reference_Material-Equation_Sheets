\subsection{Complex Exponentials}\label{subsec:Complex_Exponentials}
The definition of an exponential with a \nameref{def:Complex_Number} as its exponent is defined in \Cref{eq:Complex_Exponential}.
\begin{equation}\label{eq:Complex_Exponential}
  e^{z} = e^{x + iy} = e^{x} \bigl( \cos(y) + i \sin(y) \bigr)
\end{equation}

If instead of $e$ as the base, we have some value $a$, then we have \Cref{eq:Complex_Exponential_Diff_Base}.
\begin{equation}\label{eq:Complex_Exponential_Diff_Base}
  \begin{aligned}
    a^{z} &= e^{z \ln(a)} \\
    &= e^{\Real{z \ln(a)}} \Bigl( \cos \bigl(\Imag{z \ln(a)} \bigr) + i \sin \bigl(\Imag{z \ln(a)} \bigl) \Bigr)
  \end{aligned}
\end{equation}

In the case of \Cref{eq:Complex_Exponential}, $z$ can be presented in either Cartesian or polar form, they are equivalent.

\begin{example}[Lecture 3]{Simplify Simple Complex Exponential}
  Simplify the expression below, then find its \nameref{def:Complex_Number_Modulus}, \nameref{def:Complex_Number_Argument}, and its \nameref{def:Principal_Argument}?
  \begin{equation*}
    e^{-1 + i\sqrt{3}}
  \end{equation*}
  \tcblower{}
  If we look at the exponent on the exponential, we see
  \begin{equation*}
    z = -1 + i\sqrt{3}
  \end{equation*}
  which means
  \begin{align*}
    x &= -1 \\
    y &= \sqrt{3}
  \end{align*}

  With this information, we can simplify the expression \textbf{just} by observation, with no calculations required.
  \begin{equation*}
    e^{-1 + i\sqrt{3}} = e^{-1} \bigl( \cos(\sqrt{3}) + i \sin(\sqrt{3}) \bigr)
  \end{equation*}

  Now, we can solve the other 3 parts of this example \textbf{by observation}.
  \begin{align*}
    \Modulus{e^{-1 + i\sqrt{3}}} &= \Modulus{e^{-1} \bigl( \cos(\sqrt{3}) + i \sin(\sqrt{3}) \bigr)} \\
                                 &= e^{-1} \\
    \arg \left( e^{-1 + i\sqrt{3}} \right) &= \arg \left( e^{-1} \bigl( \cos(\sqrt{3}) + i \sin(\sqrt{3}) \bigr) \right) \\
                                 &= \sqrt{3} + 2 \pi k \\
    \PrincipalArg \left( e^{-1 + i\sqrt{3}} \right) &= \PrincipalArg \left( e^{-1} \bigl( \cos(\sqrt{3}) + i \sin(\sqrt{3}) \bigr) \right) \\
                                 &= \sqrt{3}
  \end{align*}
\end{example}


%%% Local Variables:
%%% mode: latex
%%% TeX-master: shared
%%% End:
