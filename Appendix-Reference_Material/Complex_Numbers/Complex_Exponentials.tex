\subsection{Complex Exponentials}\label{subsec:Complex_Exponentials}
The definition of an exponential with a \nameref{def:Complex_Number} as its exponent is defined in \Cref{eq:Complex_Exponential}.
\begin{equation}\label{eq:Complex_Exponential}
  e^{z} = e^{x + iy} = e^{x} \bigl( \cos(y) + i \sin(y) \bigr)
\end{equation}

If instead of $e$ as the base, we have some value $a$, then we have \Cref{eq:Complex_Exponential_Diff_Base}.
\begin{equation}\label{eq:Complex_Exponential_Diff_Base}
  \begin{aligned}
    a^{z} &= e^{z \ln(a)} \\
    &= e^{\Real{z \ln(a)}} \Bigl( \cos \bigl(\Imag{z \ln(a)} \bigr) + i \sin \bigl(\Imag{z \ln(a)} \bigl) \Bigr)
  \end{aligned}
\end{equation}

In the case of \Cref{eq:Complex_Exponential}, $z$ can be presented in either Cartesian or polar form, they are equivalent.

\begin{example}[Lecture 3]{Simplify Simple Complex Exponential}
  Simplify the expression below, then find its \nameref{def:Complex_Number_Modulus}, \nameref{def:Complex_Number_Argument}, and its \nameref{def:Principal_Argument}?
  \begin{equation*}
    e^{-1 + i\sqrt{3}}
  \end{equation*}
  \tcblower{}
  If we look at the exponent on the exponential, we see
  \begin{equation*}
    z = -1 + i\sqrt{3}
  \end{equation*}
  which means
  \begin{align*}
    x &= -1 \\
    y &= \sqrt{3}
  \end{align*}

  With this information, we can simplify the expression \textbf{just} by observation, with no calculations required.
  \begin{equation*}
    e^{-1 + i\sqrt{3}} = e^{-1} \bigl( \cos(\sqrt{3}) + i \sin(\sqrt{3}) \bigr)
  \end{equation*}

  Now, we can solve the other 3 parts of this example \textbf{by observation}.
  \begin{align*}
    \Modulus{e^{-1 + i\sqrt{3}}} &= \Modulus{e^{-1} \bigl( \cos(\sqrt{3}) + i \sin(\sqrt{3}) \bigr)} \\
                                 &= e^{-1} \\
    \arg \left( e^{-1 + i\sqrt{3}} \right) &= \arg \left( e^{-1} \bigl( \cos(\sqrt{3}) + i \sin(\sqrt{3}) \bigr) \right) \\
                                 &= \sqrt{3} + 2 \pi k \\
    \PrincipalArg \left( e^{-1 + i\sqrt{3}} \right) &= \PrincipalArg \left( e^{-1} \bigl( \cos(\sqrt{3}) + i \sin(\sqrt{3}) \bigr) \right) \\
                                 &= \sqrt{3}
  \end{align*}
\end{example}

\begin{example}[Lecture 3]{Simplify Complex Exponential Exponent}
  Given $z = e^{-e^{-i}}$, what is this expression in polar form, what is its \nameref{def:Complex_Number_Modulus}, its \nameref{def:Complex_Number_Argument}, and its \nameref{def:Principal_Argument}?
  \tcblower{}
  We start by simplifying the exponent of the base exponential, i.e.\ $e^{-i}$.
  \begin{align*}
    e^{-i} &= e^{0 - i} \\
           &= e^{0} \bigl( \cos(-1) + i \sin(-1) \bigr) \\
           &= 1 \bigl( \cos(-1) + i \sin(-1) \bigr)
  \end{align*}

  Now, with that exponent simplified, we can solve the main question.
  \begin{align*}
    e^{-e^{-i}} &= e^{-1 \bigl( \cos(-1) + i \sin(-1) \bigr)} \\
                &= e^{-1 \bigl( \cos(1) - i \sin(1) \bigr)} \\
    &= e^{-\cos(1) + i \sin(1)} \\
    \intertext{If we refer back to \Cref{eq:Complex_Exponential}, then it becomes obvious what $x$ and $y$ are.}
    x &= -\cos(1) \\
    y &= \sin(1) \\
    e^{-e^{-i}} &= e^{-\cos(1)} \Bigl( \cos \bigl( \sin(1) \bigr) + i \sin \bigl( \sin(1) \bigr) \Bigr)
  \end{align*}

  Now that we have ``simplified'' this exponential, we can solve the other 3 questions by \textbf{observation}.
  \begin{align*}
    \Modulus{e^{-e^{-i}}} &= \Modulus{e^{-\cos(1)} \Bigl( \cos \bigl( \sin(1) \bigr) + i \sin \bigl( \sin(1) \bigr) \Bigr)} \\
                          &= e^{-\cos(1)} \\
    \arg \left( e^{-e^{-i}} \right) &= \arg \left( e^{-\cos(1)} \Bigl( \cos \bigl( \sin(1) \bigr) + i \sin \bigl( \sin(1) \bigr) \Bigr) \right) \\
                          &= \sin(1) + 2 \pi k \\
    \PrincipalArg \left( e^{-e^{-i}} \right) &= \PrincipalArg \left( e^{-\cos(1)} \Bigl( \cos \bigl( \sin(1) \bigr) + i \sin \bigl( \sin(1) \bigr) \Bigr) \right) \\
                          &= \sin(1)
  \end{align*}
\end{example}

\begin{example}[Lecture 3]{Non-e Complex Exponential}
  Find all values of $z=1^{i}$?
  \tcblower{}
  Use \Cref{eq:Complex_Exponential_Diff_Base} to simplify this to a base of $e$, where we can use the usual \Cref{eq:Complex_Exponential} to solve this.
  \begin{align*}
    a^{z} &= e^{z \ln(a)} \\
    1^{i} &= e^{i \ln(1)} \\
    \intertext{Simplify the logarithm in the exponent first, $\ln(1)$.}
    \ln(1) &= \log_{e} \Modulus{1} + i \arg(1) \\
          &= \log_{e}(1) + i (0 + 2\pi k) \\
          &= 0 + 2\pi k i \\
          &= 2\pi k i \\
    \intertext{Now, plug $\ln(1)$ back into the exponent, and solve the exponential.}
    e^{i (2\pi k i)} &= e^{2\pi k i^{2}} \\
          &= e^{2\pi k (-1)} \\
    z &= e^{-2\pi k}
  \end{align*}

  Thus, all values of $z = e^{-2\pi k}$ where $k = 0, 1, \ldots$.
\end{example}


%%% Local Variables:
%%% mode: latex
%%% TeX-master: shared
%%% End:
