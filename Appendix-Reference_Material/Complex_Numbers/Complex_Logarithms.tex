\subsection{Complex Logarithms}\label{subsec:Complex_Logarithms}
There are some denotational changes that need to be made for this to work.
The traditional real-number natural logarithm $\ln$ needs to be redefined to its defining form $\log_{e}$.

With that denotational change, we can now use $\ln$ for the \nameref{def:Complex_Logarithm}.

\begin{definition}[Complex Logarithm]\label{def:Complex_Logarithm}
  The \emph{complex logarithm} is defined in \Cref{eq:Complex_Logarithm}.
  The only requirement for this equation to hold true is that $w \neq 0$.

  \begin{equation}\label{eq:Complex_Logarithm}
    \begin{aligned}
      e^{z} &= w \\
      z &= \ln(w) \\
      &= \log_{e} \Modulus{w} + i \arg(w)
    \end{aligned}
  \end{equation}

  \begin{remark}
    The \nameref{def:Complex_Logarithm} is different than it's purely-real cousin because we allow negative numbers to be input.
    This means it is more general, but we must lose the precision of the purely-real logarithm.
    This means that each nonzero number has infinitely many logarithms.
  \end{remark}
\end{definition}

\begin{example}[Lecture 3]{All Complex Logarithms of Simple Expression}
  What are \textbf{all} \nameref{def:Complex_Logarithm}s of $z = -1$?
  \tcblower{}
  We can apply the definition of a \nameref{def:Complex_Logarithm} (\Cref{eq:Complex_Logarithm}) directly.
  \begin{align*}
    \ln(z) &= \log_{e} \Modulus{z} + i \arg(z) \\
           &= \log_{e} \Modulus{-1} + i \arg(-1) \\
           &= \log_{e} (1) + i (\pi + 2 \pi k) \\
           &= 0 + i (\pi + 2 \pi k) \\
           &= i (\pi + 2 \pi k)
  \end{align*}

  Thus, all logarithms of $z = -1$ are defined by the expression $i (\pi + 2 \pi k)$, $k = 0, 1, \ldots$.

  \begin{remark*}
    You can see the loss of specificity in the \nameref{def:Complex_Logarithm} because the variable $k$ is still present in the final answer.
  \end{remark*}
\end{example}

\begin{example}[Lecture 3]{All Complex Logarithms of Complex Logarithm}
  What are \textbf{all} the \nameref{def:Complex_Logarithm}s of $z = \ln(1)$?
  \tcblower{}
  We start by simplifying $z$, before finding $\ln(z)$.
  We can make use of \Cref{eq:Complex_Logarithm}, to simplify this value.
  \begin{align*}
    \ln(w) &= \log_{e} \Modulus{w} + i \arg(w) \\
    \ln(1) &= \log_{e} \Modulus{1} + i \arg(1) \\
           &= \log_{e} 1 + i (0 + 2 \pi k) \\
           &= 0 + 2 \pi k i \\
           &= 2\pi k i
  \end{align*}

  Now that we have simplified $z$, we can solve for $\ln(z)$.
  \begin{align*}
    \ln(z) &= \ln(2\pi k i) \\
           &= \log_{e} \Modulus{2\pi k i} + i \arg(2\pi k i) \\
           &= \log_{e} (2\pi \Abs{k}) + \left( i
             \begin{cases}
               \frac{\pi}{2} + 2\pi m & k > 0 \\
               \frac{-\pi}{2} + 2\pi m & k < 0 \\
             \end{cases} \right)
    \shortintertext{The $\Abs{k}$ is the \textbf{absolute value} of $k$, because $k$ is an integer.}
  \end{align*}

  Thus, our solution of $\ln \bigl( \ln(1) \bigr) = \log_{e} (2\pi \Abs{k}) + \left( i
             \begin{cases}
               \frac{\pi}{2} + 2\pi m & k > 0 \\
               \frac{-\pi}{2} + 2\pi m & k < 0 \\
             \end{cases} \right)$.
\end{example}

\subsubsection{Complex Conjugates of Logarithms}\label{app:Logarithms_Complex_Conjugates}
\begin{equation}\label{eq:Logarithms_Complex_Conjugate}
  \Conjugate{\log(z)} = \log(\Conjugate{z})
\end{equation}

%%% Local Variables:
%%% mode: latex
%%% TeX-master: shared
%%% End:
