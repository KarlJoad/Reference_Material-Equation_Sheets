\subsection{Complex Logarithms}\label{subsec:Complex_Logarithms}
There are some denotational changes that need to be made for this to work.
The traditional real-number natural logarithm $\ln$ needs to be redefined to its defining form $\log_{e}$.

With that denotational change, we can now use $\ln$ for the \nameref{def:Complex_Logarithm}.

\begin{definition}[Complex Logarithm]\label{def:Complex_Logarithm}
  The \emph{complex logarithm} is defined in \Cref{eq:Complex_Logarithm}.
  The only requirement for this equation to hold true is that $w \neq 0$.

  \begin{equation}\label{eq:Complex_Logarithm}
    \begin{aligned}
      e^{z} &= w \\
      z &= \ln(w) \\
      &= \log_{e} \Modulus{w} + i \arg(w)
    \end{aligned}
  \end{equation}

  \begin{remark}
    The \nameref{def:Complex_Logarithm} is different than it's purely-real cousin because we allow negative numbers to be input.
    This means it is more general, but we must lose the precision of the purely-real logarithm.
    This means that each nonzero number has infinitely many logarithms.
  \end{remark}
\end{definition}


%%% Local Variables:
%%% mode: latex
%%% TeX-master: shared
%%% End:
