\subsection{Polar Form}\label{subsec:Polar_Form}
The polar form of a \nameref{def:Complex_Number} is an alternative, but equally useful way to express a complex number.
In polar form, we express the distance the complex number is from the origin and the angle it sits at from the real axis.
This is seen in \Cref{eq:Polar_Form}.
\begin{equation}\label{eq:Polar_Form}
  z = r \bigl( \cos(\theta) + i \sin(\theta) \bigr)
\end{equation}

\begin{remark*}
  Note that in the definition of polar form (\Cref{eq:Polar_Form}), there is no allowance for the radius, $r$, to be negative.
  You must fix this by figuring out the angle change that is required for the radius to become positive.
\end{remark*}

Thus,
\begin{align*}
  r &= \lvert z \rvert \\
  \theta &= \arg(z) \\
\end{align*}


%%% Local Variables:
%%% mode: latex
%%% TeX-master: shared
%%% End:
