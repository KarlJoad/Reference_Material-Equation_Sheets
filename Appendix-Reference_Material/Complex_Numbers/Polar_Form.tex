\subsection{Polar Form}\label{subsec:Polar_Form}
The polar form of a \nameref{def:Complex_Number} is an alternative, but equally useful way to express a complex number.
In polar form, we express the distance the complex number is from the origin and the angle it sits at from the real axis.
This is seen in \Cref{eq:Polar_Form}.
\begin{equation}\label{eq:Polar_Form}
  z = r \bigl( \cos(\theta) + i \sin(\theta) \bigr)
\end{equation}

\begin{remark*}
  Note that in the definition of polar form (\Cref{eq:Polar_Form}), there is no allowance for the radius, $r$, to be negative.
  You must fix this by figuring out the angle change that is required for the radius to become positive.
\end{remark*}

Thus,
\begin{align*}
  r &= \lvert z \rvert \\
  \theta &= \arg(z) \\
\end{align*}

\begin{example}[Lecture 2, Example 1]{Find Polar Coordinates from Cartesian Coordinates}
  Find the complex number's $z = -\sqrt{3} + i$ polar coordinates?
  \tcblower{}
  We start by finding the radius of $z$ (modulus of $z$).
  \begin{align*}
    r &= \lvert z \rvert \\
      &= \sqrt{\Real{z}^{2} + \Imag{z}^{2}} \\
      &= \sqrt{{(-\sqrt{3})}^{2} + 1^{2}} \\
      &= \sqrt{3 + 1} \\
      &= \sqrt{4} \\
      &= 2
  \end{align*}
  Thus, the point is $2$ units away from the origin, the radius is 2 $r=2$.

  Now, we need to find the angle, the argument, of the \nameref{def:Complex_Number}.
  \begin{align*}
    \cos(\theta) &= \frac{-\sqrt{3}}{2} \\
    \theta &= \cos^{-1} \left( \frac{-\sqrt{3}}{2} \right) \\
                 &= \frac{5 \pi}{6} \\
  \end{align*}

  Now that we have one angle for the point, we also need to consider the possibility that there have been an unknown amount of rotations around the entire plane, meaning there have been $2 \pi k$, where $k = 0, 1, \ldots$.

  We now have all the information required to reconstruct this point using polar coordinates:
  \begin{align*}
    r &= 2 \\
    \theta &= \frac{5 \pi}{6} \\
    \arg(z) &= \frac{5 \pi}{6} + 2\pi k
  \end{align*}
\end{example}

\subsubsection{Converting Between Cartesian and Polar Forms}\label{subsubsec:Convert_Cartesian_Polar}
Using \Cref{eq:Polar_Form} and \Cref{eq:Complex_Number}, it is easy to see the relation between $r$, $\theta$, $x$, and $y$.

\begin{align*}
  \shortintertext{Definition of a \nameref{def:Complex_Number} in Cartesian form.}
  z &= x + iy \\
  \shortintertext{Definition of a \nameref{def:Complex_Number} in polar form.}
  z &= r \bigl( \cos(\theta) + i \sin(\theta) \bigr) \\
    &= r \cos(\theta) + i r \sin(\theta) \\
\end{align*}

Thus,
\begin{equation}\label{eq:Convert_Cartesian_Polar}
  \begin{aligned}
    x &= r \cos(\theta) \\
    y &= r \sin(\theta)
  \end{aligned}
\end{equation}

\subsubsection{Benefits of Polar Form}\label{subsubsec:Polar_Form_Benefits}
Polar form is good for multiplication of \nameref{def:Complex_Number}s because of the way $\sin$ and $\cos$ multiply together.
The Cartesian form is good for addition and subtraction.
Take the examples below to show what I mean.

\paragraph{Multiplication}\label{par:Polar_Form_Multiplication}
For multiplication, the radii are multiplied together, and the angles are added.
\begin{equation}\label{eq:Polar_Form_Muliplication}
  \Bigl( r_{1} \bigl( \cos(\theta) + i \sin(\theta) \bigr) \Bigr) \Bigl( r_{2} \bigl( \cos(\phi) + i \sin(\phi) \bigr) \Bigr) = r_{1}r_{2} \bigl( \cos(\theta + \phi) + i \sin(\theta + \phi) \bigr)
\end{equation}


%%% Local Variables:
%%% mode: latex
%%% TeX-master: shared
%%% End:
