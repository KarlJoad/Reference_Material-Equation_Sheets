\subsection{Roots of a Complex Number}\label{subsec:Complex_Roots}
\nameref{def:de_Moivers_Law} also applies to finding \textbf{roots} of a \nameref{def:Complex_Number}.

\begin{equation}\label{eq:Complex_Roots}
  z^{\frac{1}{n}} = r^{\frac{1}{n}} \biggl( \cos \left( \frac{\arg{z}}{n} \right) + i \sin \left( \frac{\arg{z}}{n} \right) \biggr)
\end{equation}

\begin{remark*}
  As the entire $\arg{z}$ term is being divided by $n$, the $2 \pi k$ is \textbf{ALSO} divided by $n$.
\end{remark*}

Roots of a \nameref{def:Complex_Number} satisfy \Cref{eq:Complex_Root_Requirement}.
To demonstrate that equation, $z = r \bigl( \cos(\theta) + i \sin(\theta) \bigr)$ and $w = \rho \bigl( \cos(\phi) + i \sin(\phi) \bigr)$.
\begin{equation}\label{eq:Complex_Root_Requirement}
  w^{n} = z
\end{equation}
A $w$ that satisfies \Cref{eq:Complex_Root_Requirement} is an $n$th root of $z$.

\begin{example}[Lecture 2, Example 2]{Roots of a Complex Number}
  Find the cube roots of $z = -\sqrt{3} + i$?
  \tcblower{}
  From \Cref{ex:Find Polar Coordinates from Cartesian Coordinates}, we know that the polar form of $z$ is
  \begin{equation*}
    z = 2 \biggl( \cos \left( \frac{5 \pi}{6} + 2 \pi k \right) + i \sin \left( \frac{5 \pi}{6} + 2 \pi k \right) \biggr)
  \end{equation*}

  Because the question is asking for \textbf{cube} roots, that means there are 3 roots.
  Using \Cref{eq:Complex_Roots}, we can find the general form of the roots.
  \begin{align*}
    z &= 2 \biggl( \cos \left( \frac{5 \pi}{6} + 2 \pi k \right) + i \sin \left( \frac{5 \pi}{6} + 2 \pi k \right) \biggr) \\
    z^{\frac{1}{3}} &= \sqrt[3]{2} \left( \cos \Biggl( \frac{1}{3} \left( \frac{5 \pi}{6} + 2 \pi k \right) \Biggr) + i \sin \Biggl( \frac{1}{3} \left( \frac{5 \pi}{6} + 2 \pi k \right)  \Biggr) \right) \\
    &= \sqrt[3]{2} \biggl( \cos \left( \frac{\pi + 12 \pi k}{18} \right) + i \sin \left( \frac{\pi + 12 \pi k}{18} \right) \biggr)
  \end{align*}

  Now that we have a general equation for \textbf{all} possible cube roots, we need to find all the unique ones.
  This is because after $k=n$ roots, the roots start to repeat themselves, because the $2 \pi k$ part of the expression becomes effective, making the angle a full rotation.
  We simply enumerate $k \in \PositiveInts$, so $k = 0, 1, 2, \ldots$.
  \begin{align*}
    \shortintertext{k = 0}
    \sqrt[3]{2} \biggl( \cos \left( \frac{\pi + 12 \pi (0)}{18} \right) + i \sin \left( \frac{\pi + 12 \pi (0)}{18} \right) \biggr) &= \sqrt[3]{2} \biggl( \cos \left( \frac{\pi}{18} \right) + i \sin \left( \frac{\pi}{18} \right) \biggr) \\
    \shortintertext{k = 1}
    \sqrt[3]{2} \biggl( \cos \left( \frac{\pi + 12 \pi (1)}{18} \right) + i \sin \left( \frac{\pi + 12 \pi (1)}{18} \right) \biggr) &= \sqrt[3]{2} \biggl( \cos \left( \frac{13 \pi}{18} \right) + i \sin \left( \frac{13 \pi}{18} \right) \biggr) \\
    \shortintertext{k = 2}
    \sqrt[3]{2} \biggl( \cos \left( \frac{\pi + 12 \pi (2)}{18} \right) + i \sin \left( \frac{\pi + 12 \pi (2)}{18} \right) \biggr) &= \sqrt[3]{2} \biggl( \cos \left( \frac{25 \pi}{18} \right) + i \sin \left( \frac{25 \pi}{18} \right) \biggr) \\
    \shortintertext{k = 3}
    \sqrt[3]{2} \biggl( \cos \left( \frac{\pi + 12 \pi (3)}{18} \right) + i \sin \left( \frac{\pi + 12 \pi (3)}{18} \right) \biggr) &= \sqrt[3]{2} \biggl( \cos \left( \frac{\pi}{18} + \frac{36 \pi}{18} \right) + i \sin \left( \frac{\pi}{18} + \frac{36 \pi}{18} \right) \biggr) \\
                                                                                                                                    &= \sqrt[3]{2} \biggl( \cos \left( \frac{\pi}{18} + 2 \pi \right) + i \sin \left( \frac{\pi}{18} + 2 \pi \right) \biggr) \\
                                                                                                                                      &= \sqrt[3]{2} \biggl( \cos \left( \frac{\pi}{18} \right) + i \sin \left( \frac{\pi}{18} \right) \biggr)
  \end{align*}

  Thus, the 3 cube roots of $z$ are:
  \begin{align*}
    z_{1}^{\frac{1}{3}} &= \sqrt[3]{2} \biggl( \cos \left( \frac{\pi}{18} \right) + i \sin \left( \frac{\pi}{18} \right) \biggr) \\
    z_{2}^{\frac{1}{3}} &= \sqrt[3]{2} \biggl( \cos \left( \frac{13 \pi}{18} \right) + i \sin \left( \frac{13 \pi}{18} \right) \biggr) \\
    z_{3}^{\frac{1}{3}} &= \sqrt[3]{2} \biggl( \cos \left( \frac{25 \pi}{18} \right) + i \sin \left( \frac{25 \pi}{18} \right) \biggr)
  \end{align*}
\end{example}

%%% Local Variables:
%%% mode: latex
%%% TeX-master: shared
%%% End:
