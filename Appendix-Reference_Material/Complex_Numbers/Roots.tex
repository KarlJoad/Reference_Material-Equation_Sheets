\subsection{Roots of a Complex Number}\label{subsec:Complex_Roots}
\nameref{def:de_Moivers_Law} also applies to finding \textbf{roots} of a \nameref{def:Complex_Number}.

\begin{equation}\label{eq:Complex_Roots}
  z^{\frac{1}{n}} = r^{\frac{1}{n}} \biggl( \cos \left( \frac{\arg{z}}{n} \right) + i \sin \left( \frac{\arg{z}}{n} \right) \biggr)
\end{equation}

\begin{remark*}
  As the entire $\arg{z}$ term is being divided by $n$, the $2 \pi k$ is \textbf{ALSO} divided by $n$.
\end{remark*}

Roots of a \nameref{def:Complex_Number} satisfy \Cref{eq:Complex_Root_Requirement}.
To demonstrate that equation, $z = r \bigl( \cos(\theta) + i \sin(\theta) \bigr)$ and $w = \rho \bigl( \cos(\phi) + i \sin(\phi) \bigr)$.
\begin{equation}\label{eq:Complex_Root_Requirement}
  w^{n} = z
\end{equation}
A $w$ that satisfies \Cref{eq:Complex_Root_Requirement} is an $n$th root of $z$.


%%% Local Variables:
%%% mode: latex
%%% TeX-master: shared
%%% End:
