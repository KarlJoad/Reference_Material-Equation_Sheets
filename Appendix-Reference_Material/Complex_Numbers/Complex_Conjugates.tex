\subsection{Complex Conjugates}\label{app:Complex_Conjugates}
\begin{definition}[Complex Conjugate]\label{def:Complex_Conjugate}
  The conjugate of a complex number is called its \emph{complex conjugate}.
  The complex conjugate of a complex number is the number with an equal real part and an imaginary part equal in magnitude but opposite in sign.
  If we have a complex number as shown below,
  \begin{equation*}
    z = a \pm bi
  \end{equation*}

  then, the conjugate is denoted and calculated as shown below.
  \begin{equation}\label{eq:Complex_Conjugates}
    \Conjugate{z} = a \mp bi
  \end{equation}
\end{definition}

The \nameref{def:Complex_Conjugate} can also be denoted with an asterisk ($*$).
This is generally done for complex functions, rather than single variables.
\begin{equation}\label{eq:Complex_Conjugates_Asterisk}
  z^{*} = \Conjugate{z}
\end{equation}

\subsubsection{Notable Complex Conjugate Expressions}\label{subsubsec:Complex_Conjugate_Notable_Expressions}
There are 2 interesting things that we can perform with \textit{just} the concept of a \nameref{def:Complex_Number} and a \nameref{def:Complex_Conjugate}:
\begin{enumerate}
\item $z \Conjugate{z}$
\item $\frac{z}{\Conjugate{z}}$
\end{enumerate}

The first is interesting because of this simplification:
\begin{align*}
  z \Conjugate{z} &= (x + iy) (x - iy) \\
                 &= x^{2} - xyi + xyi - i^{2} y^{2} \\
                 &= x^{2} - (-1) y^{2} \\
                 &= x^{2} + y^{2}
\end{align*}

Thus,
\begin{equation}\label{eq:Complex_Number_Mult_Conjugate}
  z \Conjugate{z} = x^{2} + y^{2}
\end{equation}
which is interesting because, in comparison to the input values, the output is completely real.

%%% Local Variables:
%%% mode: latex
%%% TeX-master: shared
%%% End:
