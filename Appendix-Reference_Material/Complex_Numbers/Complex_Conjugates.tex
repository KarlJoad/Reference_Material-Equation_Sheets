\subsection{Complex Conjugates}\label{app:Complex_Conjugates}
\begin{definition}[Complex Conjugate]\label{def:Complex_Conjugate}
  The conjugate of a complex number is called its \emph{complex conjugate}.
  The complex conjugate of a complex number is the number with an equal real part and an imaginary part equal in magnitude but opposite in sign.
  If we have a complex number as shown below,
  \begin{equation*}
    z = a \pm bi
  \end{equation*}

  then, the conjugate is denoted and calculated as shown below.
  \begin{equation}\label{eq:Complex_Conjugates}
    \Conjugate{z} = a \mp bi
  \end{equation}
\end{definition}

The \nameref{def:Complex_Conjugate} can also be denoted with an asterisk ($*$).
This is generally done for complex functions, rather than single variables.
\begin{equation}\label{eq:Complex_Conjugates_Asterisk}
  z^{*} = \Conjugate{z}
\end{equation}

\subsubsection{Notable Complex Conjugate Expressions}\label{subsubsec:Complex_Conjugate_Notable_Expressions}
There are 2 interesting things that we can perform with \textit{just} the concept of a \nameref{def:Complex_Number} and a \nameref{def:Complex_Conjugate}:
\begin{enumerate}
\item $z \Conjugate{z}$
\item $\frac{z}{\Conjugate{z}}$
\end{enumerate}

The first is interesting because of this simplification:
\begin{align*}
  z \Conjugate{z} &= (x + iy) (x - iy) \\
                 &= x^{2} - xyi + xyi - i^{2} y^{2} \\
                 &= x^{2} - (-1) y^{2} \\
                 &= x^{2} + y^{2}
\end{align*}

Thus,
\begin{equation}\label{eq:Complex_Number_Mult_Conjugate}
  z \Conjugate{z} = x^{2} + y^{2}
\end{equation}
which is interesting because, in comparison to the input values, the output is completely real.

The other interesting \nameref{def:Complex_Conjugate} is dividing a \nameref{def:Complex_Number} by its conjugate.
\begin{align*}
  \frac{z}{\Conjugate{z}} &= \frac{x + iy}{x - iy} \\
  \intertext{We want to have this end up in a form of $a + ib$, so we multiply the entire fraction by $z$, to cause the denominator to be completely real.}
  z \left( \frac{z}{\Conjugate{z}} \right) &= \frac{z^{2}}{z \Conjugate{z}} \\
  \intertext{Using our solution from \Cref{eq:Complex_Number_Mult_Conjugate}:}
                         &= \frac{{(x + iy)}^{2}}{x^{2}+y^{2}} \\
                         &= \frac{x^{2} + 2xyi + i^{2}y^{2}}{x^{2}+y^{2}} \\
  \intertext{By breaking up the fraction's numerator, we can more easily recognize this to be teh Cartesian form of the \nameref{def:Complex_Number}.}
                         &= \frac{\left( x^{2} - y^{2} \right) + 2xyi}{x^{2}+y^{2}} \\
                         &= \frac{x^{2}-y^{2}}{x^{2}+y^{2}} + \frac{2xyi}{x^{2}+y^{2}} \\
\end{align*}

This is an interesting development because, unlike the multiplication of a \nameref{def:Complex_Number} by its \nameref{def:Complex_Conjugate}, the division of these two values does \textbf{not} yield a purely real number.
\begin{equation}\label{eq:Complex_Number_Div_Conjugate}
  \frac{z}{\Conjugate{z}} = \frac{x^{2}-y^{2}}{x^{2}+y^{2}} + \frac{2xyi}{x^{2}+y^{2}}
\end{equation}

\subsubsection{Properties of Complex Conjugates}\label{subsubsec:Complex_Conjugates_Properties}
Conjugation follows some of the traditional algebraic properties that you are already familiar with, namely commutativity.

First, start by defining some expressions so that we can prove some of these properties:
\begin{align*}
  z &= x + iy \\
  \Conjugate{z} &= x - iy
\end{align*}

\begin{propertylist}
\item The conjugation operation is commutative.
\item The conjugation operation can be distributed over addition and multiplication.\label{prop:Complex_Conjugate_Split}
  \begin{align*}
    \Conjugate{z+w} &= \Conjugate{z} + \Conjugate{w} \\
    \Conjugate{zw} &= \Conjugate{z} \Conjugate{w}
  \end{align*}
\end{propertylist}

\Cref{prop:Complex_Conjugate_Split} can be proven by just performing a simplification.
\begin{proof}[Prove \Cref*{prop:Complex_Conjugate_Split}]
  Let $z$ and $w$ be complex numbers ($z, w \in \ComplexNumbers$) where $z = x_{1} + iy_{1}$ and $w = x_{2} + iy_{2}$.
  Prove that $\Conjugate{z + w} = \Conjugate{z} + \Conjugate{w}$.

  We start by simplifying the left-hand side of the equation ($\Conjugate{z+w}$).
  \begin{align*}
    \Conjugate{z + w} &= \Conjugate{(x_{1} + iy_{1}) + (x_{2} + iy_{2})} \\
                     &= \Conjugate{(x_{1} + x_{2}) + i(y_{1} + y_{2})} \\
                     &= (x_{1} + x_{2}) - i(y_{1} + y_{2}) \\
  \end{align*}

  Now, we simplify the other side ($\Conjugate{z} + \Conjugate{w}$).
  \begin{align*}
    \Conjugate{z} + \Conjugate{w} &= \Conjugate{(x_{1} + iy_{1})} + \Conjugate{(x_{2} + iy_{2})} \\
                                &= (x_{1} - iy_{1}) + (x_{2} - iy_{2}) \\
                                &= (x_{1} + x_{2}) - i(y_{1} + y_{2})
  \end{align*}

  We can see that both sides are equivalent, thus the addition portion of \Cref{prop:Complex_Conjugate_Split} is correct.

  \begin{remark*}
    The proof of the multiplication portion of \Cref{prop:Complex_Conjugate_Split} is left as an exercise to the reader.
    However, that proof is quite similar to this proof of addition.
  \end{remark*}
\end{proof}

%%% Local Variables:
%%% mode: latex
%%% TeX-master: shared
%%% End:
