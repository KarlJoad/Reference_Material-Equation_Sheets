\subsection{Complex Trigonometry}\label{subsec:Complex_Trigonometry}
For the equations below, $z \in \ComplexNumbers$.
These equations are based on Euler's relationship, \Cref{subsec:Euler Equivalents}
\begin{equation}\label{eq:Complex_Cosine}
  \cos(z) = \frac{e^{iz} + e^{-iz}}{2}
\end{equation}

\begin{equation}\label{eq:Complex_Sine}
  \sin(z) = \frac{e^{iz} - e^{-iz}}{2i}
\end{equation}

\begin{example}[Lecture 3]{Simplify Complex Sinusoid}
  Solve for $z$ in the equation $\cos(z) = 5$?
  \tcblower{}
  We start by using the defintion of complex cosine \Cref{eq:Complex_Cosine}.
  \begin{align*}
    \cos(z) &= 5 \\
    \frac{e^{iz} + e^{-iz}}{2} &= 5 \\
    e^{iz} + e^{-iz} &= 10 \\
    e^{iz} \left( e^{iz} + e^{-iz} \right) &= e^{iz} (10) \\
    {e^{iz}}^{2} + 1 &= 10e^{iz} \\
    {e^{iz}}^{2} -10e^{iz} + 1 = 0 \\
    \shortintertext{Solve this quadratic equation by using the Quadratic Equation.}
    e^{iz} &= \frac{-(-10) \pm \sqrt{{(-10)}^{2} - 4(1)(1)}}{2(1)} \\
            &= \frac{10 \pm \sqrt{100 - 4}}{2} \\
            &= \frac{10 \pm \sqrt{96}}{2} \\
            &= \frac{10 \pm 4 \sqrt{6}}{2} \\
            &= 5 \pm 2 \sqrt{6} \\
    \intertext{Use the definition of complex logarithms to simplify the exponential.}
    iz &= \ln(5 \pm 2 \sqrt{6}) \\
            &= \log_{e} \Modulus{5 \pm 2 \sqrt {6}} + i \arg(5 \pm 2 \sqrt{6}) \\
            &= \log_{e} \Modulus{5 \pm 2 \sqrt {6}} + i (0 + 2\pi k) \\
            &= \log_{e} \Modulus{5 \pm 2 \sqrt {6}} + 2\pi k i \\
    z &= \frac{1}{i} \left( \log_{e} \Modulus{5 \pm 2 \sqrt {6}} + 2\pi k i \right) \\
            &= \frac{-i}{-i} \frac{1}{i} \left( \log_{e} \Modulus{5 \pm 2 \sqrt {6}} \right) + 2\pi k \\
    &= 2\pi k - i\log_{e} \Modulus{5 \pm 2 \sqrt{6}}
  \end{align*}

  Thus, $z = 2\pi k - i\log_{e} \Modulus{5 \pm 2 \sqrt{6}}$.
\end{example}

\subsubsection{Complex Angle Sum and Difference Identities}\label{subsec:Complex_Angle_Sum_Difference_Identities}
Because the definitions of sine and cosine are unsatisfactory in their Euler definitions, we can use angle sum and difference formulas and their Euler definitions to yield a set of Cartesian equations.
\begin{equation}\label{eq:Cos_Angle_Sum_Difference}
  \cos(x + iy) = \bigl( \cos(x) \cosh(y) \bigr) - i \bigl( \sin(x) \sinh(y) \bigr)
\end{equation}

\begin{equation}\label{eq:Sin_Angle_Sum_Difference}
  \sin(x + iy) = \bigl( \sin(x) \cosh(y) \bigr) + i \bigl( \cos(x) \sinh(y) \bigr)
\end{equation}

\begin{example}[Lecture 3]{Simplify Trigonometric Exponential}
  Simpify $z = e^{\cos(2 + 3i)}$, and find $z$'s \nameref{def:Complex_Number_Modulus}, \nameref{def:Complex_Number_Argument}, and \nameref{def:Principal_Argument}?
  \tcblower{}
  We start by simplifying the $\cos$ using \Cref{eq:Cos_Angle_Sum_Difference}.
  \begin{align*}
    \cos(x + iy) &= \bigl( \cos(x) \cosh(y) \bigr) - i \bigl( \sin(x) \sinh(y) \bigr) \\
    \cos(2 + 3i) &= \bigl( \cos(2) \cosh(3) \bigr) - i \bigl( \sin(2) \sinh(3) \bigr) \\
  \end{align*}

  Now that we have put the $\cos$ into a Cartesian form, one that is usable with \Cref{eq:Complex_Exponential}, we can solve this.
  \begin{align*}
    e^{z} &= e^{x + iy} = e^{x} \bigl( \cos(y) + i \sin(y) \bigr) \\
    x &= \cos(2) \cosh(3) \\
    y &= -\sin(2) \sinh(3) \\
    e^{\cos(2) \cosh(3) - i \sin(2) \sinh(3)} &= e^{\cos(2) \cosh(3)} \Bigl( \cos \bigl( -\sin(2) \sinh(3) \bigr) + i \sin \bigl( -\sin(2) \sinh(3) \bigr) \Bigr)
  \end{align*}

  Now that we have simplified $z$, we can solve for the modulus, argument, and principal argument \textbf{by observation}.
  \begin{align*}
    \Modulus{z} &= \Modulus{e^{\cos(2) \cosh(3)} \Bigl( \cos \bigl( -\sin(2) \sinh(3) \bigr) + i \sin \bigl( -\sin(2) \sinh(3) \bigr) \Bigr)} \\
                &= e^{\cos(2) \cosh(3)} \\
    \arg(z) &= \arg(e^{\cos(2) \cosh(3)} \Bigl( \cos \bigl( -\sin(2) \sinh(3) \bigr) + i \sin \bigl( -\sin(2) \sinh(3) \bigr) \Bigr)) \\
                &= -\sin(2) \sinh(3) + 2\pi k \\
    \PrincipalArg(z) &= \PrincipalArg(e^{\cos(2) \cosh(3)} \Bigl( \cos \bigl( -\sin(2) \sinh(3) \bigr) + i \sin \bigl( -\sin(2) \sinh(3) \bigr) \Bigr)) \\
                &= -\sin(2) \sinh(3) \\
  \end{align*}
\end{example}


%%% Local Variables:
%%% mode: latex
%%% TeX-master: shared
%%% End:
