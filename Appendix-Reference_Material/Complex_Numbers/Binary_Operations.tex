\subsection{Binary Operations}\label{subsec:Binary_Operations}
The question here is if we are given 2 complex numbers, how should these binary operations work such that we end up with just one resulting complex number.
There are only 2 real operations that we need to worry about, and the other 3 can be defined in terms of these two:
\begin{enumerate}[noitemsep]
\item \nameref{subsubsec:Complex_Number-Addition}
\item \nameref{subsubsec:Complex_Number-Multiplication}
\end{enumerate}

For the sections below, assume:
\begin{align*}
  z &= x_{1} + iy_{1} \\
  w &= x_{2} + iy_{2}
\end{align*}

\subsubsection{Addition}\label{subsubsec:Complex_Number-Addition}
The addition operation, still denoted with the $+$ symbol is done pairwise.
You should treat $i$ like a variable in regular algebra, and not move it around.

\begin{equation}\label{eq:Complex_Number-Addition}
  z+w \coloneqq (x_{1}+x_{2}) + i(y_{1}+y_{2})
\end{equation}

\subsubsection{Multiplication}\label{subsubsec:Complex_Number-Multiplication}
The multiplication operation, like in traditional algebra, usually lacks a multiplication symbol.
You should treat $i$ like a variable in regular algebra, and not move it around.

\begin{equation}\label{eq:Complex_Number-Addition}
  \begin{aligned}
    zw &\coloneqq (x_{1} + iy_{1}) (x_{2} + iy_{2}) \\
    &= (x_{1}x_{2}) + (iy_{1}x_{2}) + (ix_{1}y_{2}) + (i^{2}y_{1}y_{2}) \\
    &= (x_{1}x_{2}) + i(y_{1}x_{2} + x_{1}y_{2}) + (-1 y_{1}y_{2}) \\
    &= (x_{1}x_{2} - y_{1}y_{2}) + i(y_{1}x_{2} + x_{1}y_{2}) \\
  \end{aligned}
\end{equation}

%%% Local Variables:
%%% mode: latex
%%% TeX-master: shared
%%% End:
