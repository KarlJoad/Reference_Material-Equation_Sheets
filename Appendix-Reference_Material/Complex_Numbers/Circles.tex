\subsection{Circles and Complex Numbers}\label{subsec:Circles_Complex_Numbers}
We need to define both a center and a radius, just like with regular purely real values.
\Cref{eq:Circles_Complex_Numbers} defines the relation required for a circle using \nameref{def:Complex_Number}s.
\begin{equation}\label{eq:Circles_Complex_Numbers}
  \lvert z - a \rvert = r
\end{equation}

\begin{example}[Lecture 2, Example 1]{Convert to Circle}
  Given the expression below, find the location of the center of the circle and the radius of the circle?
  \begin{equation*}
    \lvert 5 iz + 10 \rvert = 7
  \end{equation*}
  \tcblower{}
  This is just a matter of simplification and moving terms around.
  \begin{align*}
    \lvert 5 iz + 10 \rvert &= 7 \\
    \lvert 5i (z + \frac{10}{5i}) \rvert &= 7 \\
    \lvert 5i (z + \frac{2}{i}) \rvert &= 7 \\
    \lvert 5i (z + \frac{2}{i} \frac{-i}{-i}) \rvert &= 7 \\
    \lvert 5i (z - 2i) \rvert &= 7 \\
    \intertext{Now using the Law of Moduli (\Cref{prop:Law_of_Moduli}) $\lvert a b \rvert = \lvert a \rvert \lvert b \rvert$, we can simplify out the extra imaginary term.}
    \lvert 5i \rvert \lvert z-2i \rvert &= 7 \\
    5 \lvert z - 2i \rvert &= 7 \\
    \lvert z - 2i \rvert = \frac{7}{5}
  \end{align*}

  Thus, the circle formed by the equation $\lvert 5 iz + 10 \rvert = 7$ is actually $\lvert z - 2i \rvert = \frac{7}{5}$, with a center at $a = 2i$ and a radius of $\frac{7}{5}$.
\end{example}

\subsubsection{Annulus}\label{subsubsec:Annulus}
\begin{definition}[Annulus]\label{def:Annulus}
  An \emph{annulus} is a region that is bounded by 2 concentric circles.
  This takes the form of \Cref{eq:Annulus}.
  \begin{equation}\label{eq:Annulus}
    r_{1} \leq \lvert z - a \rvert \leq r_{2}
  \end{equation}

  In \Cref{eq:Annulus}, each of the $\leq$ symbols could also be replaced with $<$.
  This leads to 3 different possibilities for the annulus:
  \begin{enumerate}[noitemsep]
  \item If both inequality symbols are $\leq$, then it is a \textbf{Closed Annulus}.
  \item If both inequality symbols are $<$, then it is an \textbf{Open Annulus}.
  \item If \textbf{only one} inequality symbol $<$ and the other $\leq$, then it is not an \textbf{Open Annulus}.
  \end{enumerate}
\end{definition}

The concept of an \nameref{def:Annulus} can be extended to angles and arguments of a \nameref{def:Complex_Number}.
A general example of this is shown below.
\begin{equation*}
  \theta_{1} \leq \arg(z) \leq \theta_{2}
\end{equation*}

Angular Annuli follow all the same rules as regular annuli.

%%% Local Variables:
%%% mode: latex
%%% TeX-master: shared
%%% End:
