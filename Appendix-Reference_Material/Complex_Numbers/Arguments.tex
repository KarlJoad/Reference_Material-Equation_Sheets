\subsection{Arguments}\label{subsec:Complex_Number_Arguments}
There are 2 types of arguments that we can talk about for a \nameref{def:Complex_Number}.
\begin{enumerate}[noitemsep]
\item The \nameref{def:Complex_Number_Argument}
\item The \nameref{def:Principal_Argument}
\end{enumerate}

\begin{definition}[Argument]\label{def:Complex_Number_Argument}
  The \emph{argument} of a \nameref{def:Complex_Number} refers to \textbf{all} possible angles that can satisfy the angle requirement of a \nameref{def:Complex_Number}.
\end{definition}

\begin{example}[Lecture 3, Example 1]{Argument of Complex Number}
  If $z = -1 - i$, then what is its \textbf{\nameref{def:Complex_Number_Argument}}?
  \tcblower{}
  You can plot this value on the \nameref{def:Argand_Plane} and find the angle graphically/geometrically, or you can ``cheat'' and use $\tan^{-1}$ (so long as you correct for the proper quadrant).
  I will ``cheat'', as I cannot plot easily.
  \begin{align*}
    z &= -1 - i \\
    \arg(z) &= \tan(\theta) = \frac{-i}{-1} \\
      &= \frac{\pi}{4} \\
    \shortintertext{Remember to correct for the proper quadrant. We are in quadrant IV.}
      &= \frac{5 \pi}{4} \\
    \intertext{Now, we have to account for \textbf{all} possible angles that form this angle.}
    \arg(z) &= \frac{5 \pi}{4} + 2 \pi k
  \end{align*}

  Thus, the argument of $z = -1 - i$ is $\arg(z) = \frac{5 \pi}{4} + 2 \pi k$.
\end{example}


%%% Local Variables:
%%% mode: latex
%%% TeX-master: shared
%%% End:
