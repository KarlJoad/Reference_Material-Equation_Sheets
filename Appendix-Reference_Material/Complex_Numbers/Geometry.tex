\subsection{Geometry of Complex Numbers}\label{subsec:Geometry_Complex_Numbers}
So far, we have viewed \nameref{def:Complex_Number}s only algebraically.
However, we can also view them geometrically as points on a 2 dimensional \nameref{def:Argand_Plane}.

\begin{definition}[Argand Plane]\label{def:Argand_Plane}
  An \emph{Argane Plane} is a standard two dimensional plane whose points are all elements of the complex numbers, $z \in \ComplexNumbers$.
  This is taken from Descarte's definition of a completely real plane.

  The Argand plane contains 2 lines that form the axes, that indicate the real component and the imaginary component of the complex number specified.
\end{definition}

A \nameref{def:Complex_Number} can be viewed as a point in the \nameref{def:Argand_Plane}, where the \nameref{def:Real_Part} is the ``$x$''-component and the \nameref{def:Imaginary_Part} is the ``$y$''-component.

By plotting this, you see that we form a right triangle, so we can find the hypotenuse of that triangle.
This hypotenuse is the distance the point $p$ is from the origin, refered to as the \nameref{def:Complex_Number_Modulus}.
\begin{remark*}
  When working with \nameref{def:Complex_Number}s geometrically, we refer to the points, where they are defined like so:
  \begin{equation*}
    z = x + iy = p(x, y)
  \end{equation*}

  Note that $p$ is \textbf{not} a function of $x$ and $y$.
  Those are the values that inform us \textbf{where} $p$ is located on the \nameref{def:Argand_Plane}.
\end{remark*}

\subsubsection{Modulus of a Complex Number}\label{subsubsec:Complex_Number_Modulus}
\begin{definition}[Modulus]\label{def:Complex_Number_Modulus}
  The \emph{modulus} of a \nameref{def:Complex_Number} is the distance from the origin to the complex point $p$.
  This is based off the Pythagorean Theorem.
  \begin{equation}\label{eq:Complex_Number_Modulus}
    \begin{aligned}
      {\lvert z \rvert}^{2} = x^{2} + y^{2} &= z \Conjugate{z} \\
      \lvert z \rvert &= \sqrt{x^{2} + y^{2}}
    \end{aligned}
  \end{equation}
\end{definition}

\begin{propertylist}
\item The \emph{Law of Moduli} states that $\lvert z w \rvert = \lvert z \rvert \lvert w \rvert$.\label{prop:Law_of_Moduli}.
\end{propertylist}

We can prove \Cref{prop:Law_of_Moduli} using an algebraic identity.
\begin{proof}[Prove \Cref*{prop:Law_of_Moduli}]
  Let $z$ and $w$ be complex numbers ($z, w \in \ComplexNumbers$).
  We are asked to prove
  \begin{equation*}
    \lvert z w \rvert = \lvert z \rvert \lvert w \rvert
  \end{equation*}

  But, it is actually easier to prove
  \begin{equation*}
    {\lvert z w \rvert}^{2} = {\lvert z \rvert}^{2} {\lvert w \rvert}^{2}
  \end{equation*}

  We start by simplifying the ${\lvert z w \rvert}^{2}$ equation above.
  \begin{align*}
    {\lvert z w \rvert}^{2} &= {\lvert z \rvert}^{2} {\lvert w \rvert}^{2} \\
    \intertext{Using the definition of the \nameref{def:Complex_Number_Modulus} of a \nameref{def:Complex_Number} in \Cref{eq:Complex_Number_Modulus}, we can expand the modulus.}
                            &= (z w) (\Conjugate{z w}) \\
    \intertext{Using \Cref{prop:Complex_Conjugate_Split} for multiplication allows us to do the next step.}
                            &= (z w) (\Conjugate{z} \Conjugate{w}) \\
    \intertext{Using Multiplicative Associativity and Multiplicative Commutativity, we can simplify this further.}
                            &= (z \Conjugate{z}) (w \Conjugate{w}) \\
                            &= {\lvert z \rvert}^{2} {\lvert w \rvert}^{2}
  \end{align*}

  Note how we never needed to define $z$ or $w$, so this is as general a result as possible.
\end{proof}

\paragraph{Algebraic Effects of the Modulus' \Cref*{prop:Law_of_Moduli}}\label{par:Law_of_Moduli-Algebraic_Effects}
For this section, let $z = x_{1} + iy_{1}$ and $w = x_{2} + iy_{2}$.
Now,
\begin{align*}
  z w &= (x_{1}x_{2} - y_{1}y_{2}) + i(x_{1}y_{2} + x_{2}y_{1}) \\
  {\lvert z w \rvert}^{2} &= {(x_{1}x_{2} - y_{1}y_{2})}^{2} + {(x_{1}y_{2} + x_{2}y_{1})}^{2} \\
      &= \left( x_{1}^{2} + x_{2}^{2} \right) \left( x_{2}^{2} + y_{2}^{2} \right) \\
      &= {\lvert z \rvert}^{2} {\lvert w \rvert}^{2}
\end{align*}

However, the Law of Moduli (\Cref{prop:Law_of_Moduli}) does \textbf{not} hold for a hyper complex number system one that uses 2 or more imaginaries, i.e.\ $z = a + iy + jz$.
But, the Law of Moduli (\Cref{prop:Law_of_Moduli}) \textbf{does} hold for hyper complex number system that uses 3 imaginaries, $a = z + iy + jz + k \ell$.


%%% Local Variables:
%%% mode: latex
%%% TeX-master: shared
%%% End:
