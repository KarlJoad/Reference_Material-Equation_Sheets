\subsection{Geometry of Complex Numbers}\label{subsec:Geometry_Complex_Numbers}
So far, we have viewed \nameref{def:Complex_Number}s only algebraically.
However, we can also view them geometrically as points on a 2 dimensional \nameref{def:Argand_Plane}.

\begin{definition}[Argand Plane]\label{def:Argand_Plane}
  An \emph{Argane Plane} is a standard two dimensional plane whose points are all elements of the complex numbers, $z \in \ComplexNumbers$.
  This is taken from Descarte's definition of a completely real plane.

  The Argand plane contains 2 lines that form the axes, that indicate the real component and the imaginary component of the complex number specified.
\end{definition}

A \nameref{def:Complex_Number} can be viewed as a point in the \nameref{def:Argand_Plane}, where the \nameref{def:Real_Part} is the ``$x$''-component and the \nameref{def:Imaginary_Part} is the ``$y$''-component.

By plotting this, you see that we form a right triangle, so we can find the hypotenuse of that triangle.
This hypotenuse is the distance the point $p$ is from the origin, refered to as the \nameref{def:Complex_Number_Modulus}.
\begin{remark*}
  When working with \nameref{def:Complex_Number}s geometrically, we refer to the points, where they are defined like so:
  \begin{equation*}
    z = x + iy = p(x, y)
  \end{equation*}

  Note that $p$ is \textbf{not} a function of $x$ and $y$.
  Those are the values that inform us \textbf{where} $p$ is located on the \nameref{def:Argand_Plane}.
\end{remark*}

\subsubsection{Modulus of a Complex Number}\label{subsubsec:Complex_Number_Modulus}
\begin{definition}[Modulus]\label{def:Complex_Number_Modulus}
  The \emph{modulus} of a \nameref{def:Complex_Number} is the distance from the origin to the complex point $p$.
  This is based off the Pythagorean Theorem.
  \begin{equation}\label{eq:Complex_Number_Modulus}
    \begin{aligned}
      {\lvert z \rvert}^{2} = x^{2} + y^{2} &= z \Conjugate{z} \\
      \lvert z \rvert &= \sqrt{x^{2} + y^{2}}
    \end{aligned}
  \end{equation}
\end{definition}


%%% Local Variables:
%%% mode: latex
%%% TeX-master: shared
%%% End:
