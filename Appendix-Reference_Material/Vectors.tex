\section{Vectors}\label{sec:Vectors}
\begin{definition}[Vector]\label{def:Vector}
  A \emph{vector} is a way to show both magnitude of displacement and direction of displacement.
  Vectors are drawn as rays.
  \begin{remark}
    \nameref{def:Vector}s and \nameref{def:Scalar}s may seem similar, but are different.
  \end{remark}
\end{definition}

\begin{definition}[Scalar]\label{def:Scalar}
  A \emph{scalar} is a way to show \emph{\textbf{ONLY}} the magnitude of a displacement, without any direction information.
\end{definition}

\subsection{Vector Properties}\label{subsec:Vector Properties}
\begin{propertylist}
  \item $\vec{A} + \vec{B} = \vec{C}$
  \item $\vec{0} = \langle 0, 0, 0, \ldots, 0 \rangle$
  \item $\vec{A} + \vec{0} = \vec{A}$
  \item $\vec{A} + -\vec{A} = \vec{0}$
  \item $\left( \vec{A} + \vec{B} \right) + \vec{C} = \vec{A} + \left( \vec{B} + \vec{C} \right)$
  \item $\vec{A} + \vec{B} = \vec{B} + \vec{A}$
  \item Magnitude of vector: $\lVert \vec{A} \rVert = \sqrt{A_{x}^{2} + A_{y}^{2} + A_{z}^{2}}$
\end{propertylist}

\subsubsection{Getting Components}\label{subsubsec:Vector Components}
Getting the components of a vector involves solving the imaginary pythagorean triangle around the vector.

For a 2-dimensional vector, $\vec{V}$, you have the components $\langle V_{x}, V_{y} \rangle$.
You find their values with this equation:
\begin{equation}\label{eq:Vector Components}
  \begin{aligned}
    V_{x} &= V \cos \theta \\
    V_{y} &= V \sin \theta
  \end{aligned}
\end{equation}

\subsubsection{3D Unit Vectors}\label{subsubsec:3D Unit Vectors}
3-dimensional vectors shouldn't be any too crazy by this point.
They are just another variable that can be thrown around in the vector.
However, the three \nameref{subsubsec:3D Unit Vectors} are special.
You can also use these to describe any lower-dimensional vector as well.
\begin{equation}
  \begin{aligned}
    \hat{\imath} &= \langle 1, 0, 0 \rangle \\
    \hat{\jmath} &= \langle 0, 1, 0 \rangle \\
    \hat{k} &= \langle 0, 0, 1 \rangle
  \end{aligned}
\end{equation}

\subsubsection{Addition}\label{subsubsec:Vector Addition}
Vectors are additive, and are done from head-to-tail.
This means that
\begin{equation}\label{eq:Vector Addition}
  \vec{A} + \vec{B} = \vec{C}
\end{equation}

This means that in 3-dimensional vectors, they are added like this:
\begin{equation}\label{eq:3D Vector Addition}
  \begin{aligned}
    \vec{A} &= \langle A_{x}, A_{y}, A_{z} \rangle \\
    \vec{B} &= \langle B_{x}, B_{y}, B_{z} \rangle \\
    \vec{A} + \vec{B} &= \langle A_{x}+B_{x}, A_{y}+B_{y}, A_{z}+B_{z} \rangle
  \end{aligned}
\end{equation}

\subsubsection{Scalar Multiplication}\label{subsubsec:Scalar Vector Multiplication}
When applying multiplication between a scalar and a vector, you perform \nameref{subsubsec:Scalar Vector Multiplication}.
\begin{equation}\label{eq:Scalar Vector Multiplication}
  2 \times \vec{V} = 2 \langle V_{x}, V_{y} \rangle = \langle 2V_{x}, 2V_{y}, 2V_{z} \rangle
\end{equation}

This means that you do \emph{\textbf{NOT}} modify the direction of the vector, you only change its magnitude.

\subsubsection{Scalar (Dot) Product}\label{subsubsec:Dot Product}
The \nameref{subsubsec:Dot Product} is the first of two ways to multiply 2 vectors.
The other is the \nameref{subsubsec:Cross Product}.
There are 2 ways to calculate the \nameref{subsubsec:Dot Product}.

The first involves using the magnitudes of each vector and multiplying those by the cosine of the angle between them.
\begin{equation}\label{eq:Dot Product Magnitudes}
  \vec{A} \cdot \vec{B} = \lVert \vec{A} \rVert \lVert \vec{B} \rVert \cos \left( \theta \right)
\end{equation}

The second is done by adding the product of each component of each vector.
\begin{equation}\label{eq:Dot Product Components}
  \vec{A} \cdot \vec{B} = A_{x}B_{x} + A_{y}B_{y} + A_{z}B_{z}
\end{equation}

\begin{remark*}
  This means that when you apply the \nameref{subsubsec:Dot Product} to 2 vectors, you return a \nameref{def:Scalar}.
\end{remark*}

\paragraph{Properties of \nameref{subsubsec:Dot Product}}\label{par:Dot Product Properties}
\begin{propertylist}
  \item $( \vec{A} )^{2} = \vec{A} \cdot \vec{A}$
  \item $\vec{A} \cdot \vec{B} = \vec{B} \cdot \vec{A}$
\end{propertylist}

\subsubsection{Vector (Cross) Product}\label{subsubsec:Cross Product}
The \nameref{subsubsec:Cross Product} is the second of two ways to multiply 2 vectors.
The other is the \nameref{subsubsec:Dot Product}.
There are 2 ways to calculate the \nameref{subsubsec:Cross Product}.

The first involves using the magnitudes of each vector and multiplying those by the sine of the angle between them.
\begin{equation}\label{eq:Cross Product Magnitudes}
  \vec{A} \cross \vec{B} = \lVert \vec{A} \rVert \lVert \vec{B} \rVert \sin \theta
\end{equation}

The second is done by taking the determinant of a $2 \times 2$ or $3 \times 3$ matrix.
\begin{equation}\label{eq:Cross Product Determinant}
  \begin{aligned}
    \vec{A} \cross \vec{B}
    &= \det \begin{bmatrix}
      \hat{\imath} & \hat{\jmath} & \hat{k} \\
      A_{x} & A_{y} & A_{z} \\
      B_{x} & B_{y} & B_{z}
    \end{bmatrix} \\
    &= \begin{vmatrix}
      \hat{\imath} & \hat{\jmath} & \hat{k} \\
      A_{x} & A_{y} & A_{z} \\
      B_{x} & B_{y} & B_{z}
    \end{vmatrix} \\
    &= \left( A_{y}B_{z}-A_{z}B_{y} \right) \hat{\imath} - \left( A_{x}B_{z}-A_{z}B_{x} \right) \hat{\jmath} + \left( A_{x}B_{y} - A_{y}B_{x} \right) \hat{k} \\
    &= \bigl\langle A_{y}B_{z}-A_{z}B_{y}, - \left( A_{x}B_{z}+A_{z}B_{x} \right), A_{x}B_{y}-A_{y}B_{x} \bigr\rangle
  \end{aligned}
\end{equation}

\begin{remark*}
  This means that when you apply the \nameref{subsubsec:Cross Product} to 2 vectors, you return a \nameref{def:Vector}.
\end{remark*}

\paragraph{Properties of \nameref{subsubsec:Cross Product}}\label{par:Cross Product Properties}
\begin{propertylist}
  \item $\vec{A} \cross \vec{A} = \vec{0}$
  \item $\vec{A} \cross \vec{B} = - \left( \vec{B} \cross \vec{A} \right)$
  \item $\vec{A} \cross \left( \vec{B} \cross \vec{C} \right) = \vec{B} \left( \vec{A} \cdot \vec{C}\right) - \vec{C} \left( \vec{A} \cdot \vec{B} \right)$
  \item $\vec{A} \cdot \left( \vec{B} \cross \vec{C} \right) = \vec{C} \left( \vec{A} \cross \vec{B} \right) = \vec{B} \cdot \left( \vec{C} \cross \vec{A} \right)$
\end{propertylist}
