\section{Trigonometry} \label{app:Trig}
	\subsection{Trigonometric Formulas} \label{subsec:Trig Formulas}
		\begin{equation} \label{eq:Sin plus Sin with diff Angles}
			\sin \left( \alpha \right) + \sin \left( \beta \right) = 2 \sin \left( \frac{\alpha + \beta}{2} \right) \cos\left( \frac{\alpha - \beta}{2} \right)  
		\end{equation}
		\begin{equation} \label{eq:Cosine-Sine Product}
			\cos \left( \theta \right) \sin \left( \theta \right) = \frac{1}{2} \sin \left( 2 \theta \right)
		\end{equation}
	
	\subsection{Euler Equivalents of Trigonometric Functions} \label{subsec:Euler Equivalents}
		\begin{equation} \label{eq:Euler Complex}
			e^{\pm \imath \alpha} = \cos \left( \alpha \right) \pm \imath \sin \left( \alpha \right)
		\end{equation}
		\begin{equation} \label{eq:Euler Sin}
			\sin \left( x \right) = \frac{e^{\imath x} + e^{-\imath x}}{2}
		\end{equation}
		\begin{equation} \label{eq:Euler Cos}
			\cos \left( x \right) = \frac{e^{\imath x} - e^{-\imath x}}{2 \imath}
		\end{equation}
		\begin{equation} \label{eq:Euler Sinh}
			\sinh \left( x \right) = \frac{e^{x} - e^{-x}}{2}
		\end{equation}
		\begin{equation} \label{eq:Euler Cosh}
			\cosh \left( x \right) = \frac{e^{x} + e^{-x}}{2}
		\end{equation}
	
	\subsection{Angle Sum and Difference Identities} \label{subsec:Angle Sum and Difference Identities}
		\begin{equation} \label{eq:Sin Angle Sum and Difference}
			\sin \left( \alpha \pm \beta \right) = \sin \left( \alpha \right) \cos \left( \beta \right) \pm \cos \left( \alpha \right) \sin \left( \beta \right)
		\end{equation}
		\begin{equation} \label{eq:Cos Angle Sum and Difference}
			\cos \left( \alpha \pm \beta \right) = \cos \left( \alpha \right) \cos \left( \beta \right) \mp \sin \left( \alpha \right) \sin \left( \beta \right)
		\end{equation}
	
	\subsection{Double-Angle Formulae} \label{subsec:Double-Angle Formulae}
		\begin{equation} \label{eq:Sin Double-Angle}
			\sin \left( 2 \alpha \right) = 2 \sin \left( \alpha \right) \cos \left( \alpha \right)
		\end{equation}
		\begin{equation} \label{eq:Cos Double-Angle}
			\cos \left( 2 \alpha \right) = \cos^{2} \left( \alpha \right) - \sin^{2} \left( \alpha \right)
		\end{equation}
		
	\subsection{Half-Angle Formulae} \label{subsec:Half-Angle Formulae}
		\begin{equation} \label{eq:Sin Half-Angle}
			\sin \left( \frac{\alpha}{2} \right) = \sqrt{\frac{1-\cos \left( \alpha \right)}{2}}
		\end{equation}
		\begin{equation} \label{eq:Cos Half-Angle}
			\cos \left( \frac{\alpha}{2} \right) = \sqrt{\frac{1+\cos \left( \alpha \right)}{2}}
		\end{equation}
	
	\subsection{Exponent Reduction Formulae} \label{subsec:Exponent Reduction Formula}
		\begin{equation} \label{eq:Sin Squared Reduction}
			\sin^{2} \left( \alpha \right) = \frac{1-\cos \left( 2 \alpha \right)}{2}
		\end{equation}
		\begin{equation} \label{eq:Cos Squared Reduction}
			\cos^{2} \left( \alpha \right) = \frac{1+\cos \left( 2 \alpha \right)}{2}
		\end{equation}
	
	\subsection{Product-to-Sum Identities} \label{subsec:Product-to-Sum Identities}
		\begin{equation} \label{eq:Cos-Cos Product-to-Sum Identity}
			2 \cos \left( \alpha \right) \cos \left( \beta \right) = \cos \left( \alpha - \beta \right) + \cos \left( \alpha + \beta \right)
		\end{equation}
		\begin{equation} \label{eq:Sin-Sin Product-to-Sum Identity}
			2 \sin \left( \alpha \right) \sin \left( \beta \right) = \cos \left( \alpha - \beta \right) - \cos \left( \alpha + \beta \right)
		\end{equation}
		\begin{equation} \label{eq:Sin-Cos Product-to-Sum Identity}
			2 \sin \left( \alpha \right) \cos \left( \beta \right) = \sin \left( \alpha + \beta \right) + \sin \left( \alpha - \beta \right)
		\end{equation}
		\begin{equation} \label{eq:Cos-Sin Product-to-Sum Identity}
			2 \cos \left( \alpha \right) \sin \left( \beta \right) = \sin \left( \alpha + \beta \right) - \sin \left( \alpha - \beta \right)
		\end{equation}
	
	\subsection{Sum-to-Product Identities} \label{subsec:Sum-to-Product Identities}
		\begin{equation} \label{eq:Sin Sum-to-Product Identity}
			\sin \left( \alpha \right) \pm \sin \left( \beta \right) = 2 \sin \left( \frac{ \alpha \pm \beta}{2} \right) \cos \left( \frac{\alpha \mp \beta}{2} \right)
		\end{equation}
		\begin{equation} \label{eq:Cos+Cos Sum-to-Product Identity}
			\cos \left( \alpha \right) + \cos \left( \alpha \right) = 2 \cos \left( \frac{\alpha + \beta}{2} \right) \cos \left( \frac{\alpha - \beta}{2} \right)
		\end{equation}
		\begin{equation} \label{eq:Cos-Cos Sum-to-Product Identity}
			\cos \left( \alpha \right) - \cos \left( \beta \right) = -2 \sin \left( \frac{\alpha + \beta}{2} \right) \sin \left( \frac{\alpha - \beta}{2} \right)
		\end{equation}
	
	\subsection{Pythagorean Theorem for Trig} \label{subsec:Pythagorean Theorem for Trig}
		\begin{equation} \label{eq:Pythagorean Theorem for Trig}
			\cos^{2} \left( \alpha \right) + \sin^{2} \left( \alpha \right) = 1^{2}
		\end{equation}
		
	\subsection{Rectangular to Polar} \label{subsec:Rectangular to Polar}
		\begin{equation} \label{eq:Rectangular to Polar-Magnitude}
			a + \imath b = \sqrt{a^{2}+b^{2}} e^{\imath \theta} = re^{\imath \theta}
		\end{equation}
		\begin{equation} \label{eq:Rectangular to Polar-Angle}
			\theta = \begin{cases}
				\arctan \left( \frac{b}{a} \right) & a>0 \\
				\pi - \arctan \left( \frac{b}{a} \right) & a<0
			\end{cases}
		\end{equation}
		
	\subsection{Polar to Rectangular} \label{subsec:Polar to Rectangular}
		\begin{equation} \label{eq:Polar to Rectangular}
			re^{\imath \theta} = r \cos \left( \theta \right) + \imath r \sin \left( \theta \right)
		\end{equation}