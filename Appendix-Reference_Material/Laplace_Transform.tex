\section{Laplace Transform}\label{app:Laplace_Transform}
\subsection{Laplace Transform}\label{subsec:Laplace_Transform}
\begin{definition}[Laplace Transform]\label{def:Laplace_Transform}
  The \emph{Laplace transformation} operation is denoted as $\Lapl \lbrace x(t) \rbrace$ and is defined as
  \begin{equation}\label{eq:Laplace_Transform}
    X(s) = \int\limits_{-\infty}^{\infty} x(t) e^{-st} dt
  \end{equation}
\end{definition}

\subsection{Inverse Laplace Transform}\label{subsec:Inverse_Laplace_Transform}
\begin{definition}[Inverse Laplace Transform]\label{def:Inverse_Laplace_Transform}
  The \emph{inverse Laplace transformation} operation is denoted as $\Lapl^{-1} \lbrace X(s) \rbrace$ and is defined as
  \begin{equation}\label{eq:Inverse_Laplace_Transform}
    x(t) = \frac{1}{2j \pi} \int_{\sigma-\infty}^{\sigma+\infty} X(s) e^{st} \, ds
  \end{equation}
\end{definition}

\subsection{Properties of the Laplace Transform}\label{subsec:Laplace_Transform_Properties}
\subsubsection{Linearity}\label{subsubsec:Laplace_Linearity}
The \nameref{def:Laplace_Transform} is a linear operation, meaning it obeys the laws of linearity.
This means \Cref{eq:Laplace_Linearity} must hold.
\begin{subequations}\label{eq:Laplace_Linearity}
  \begin{equation}\label{eq:Laplace_Linearity_Time}
    x(t) = \alpha_{1} x_{1}(t) + \alpha_{2} x_{2}(t)
  \end{equation}
  \begin{equation}\label{eq:Laplace_Linearity_Frequency}
    X(s) = \alpha_{1} X_{1}(s) + \alpha_{2} X_{2}(s)
  \end{equation}
\end{subequations}

\subsubsection{Time Scaling}\label{subsubsec:Laplace_Time_Scaling}
\subsubsection{Time Shift}\label{subsubsec:Laplace_Time_Shift}
\subsubsection{Frequency Shift}\label{subsubsec:Laplace_Frequency_Shift}

