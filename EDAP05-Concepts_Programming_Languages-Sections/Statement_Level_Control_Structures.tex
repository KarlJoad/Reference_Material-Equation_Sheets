\section{Statement-Level Control Structures}\label{sec:Statement_Level_Control_Structures}
\begin{definition}[Control Statement]\label{def:Control_Statement}
  \emph{Control statement}s provide the ability to:
  \begin{itemize}[noitemsep]
  \item Select alternative control flow paths of statement execution
  \item Cause repeated execution of statements or sequences of statements
  \end{itemize}
\end{definition}

\begin{definition}[Control Structure]\label{def:Control_Structure}
  A \emph{control structure} is a \nameref{def:Control_Statement} and the collection of statements whose execution it controls.
\end{definition}

\subsection{Selection Statements}\label{subsec:Selection_Statements}
\begin{definition}[Selection Statement]\label{def:Selection_Statement}
  A \emph{selection statement} provides a means of choosing between 2 or more execution paths in a program.

  There are 2 general categories of these:
  \begin{enumerate}[noitemsep]
  \item \nameref{subsubsec:2_Way_Selection}
  \item \nameref{subsubsec:N_Way_Selection}
  \end{enumerate}
\end{definition}

\subsubsection{2-Way Selection}\label{subsubsec:2_Way_Selection}
There are several ways to syntactically design a 2-Way selection statement, but they all follow the same basic steps.
\begin{verbatim}
if control_expression
   then clause
   else clause
\end{verbatim}

\paragraph{Design Issues}\label{par:2_Way_Selection-Design_Issues}
\paragraph{The Control Expression}\label{par:2_Way_Selection-Control_Expression}
\paragraph{Clause Form}\label{par:2_Way_Selection-Clause_Form}
\paragraph{Nesting Selectors}\label{par:2_Way_Selection-Nesting}
\paragraph{Selector Expressions}\label{par:2_Way_Selection-Selector_Expressions}

\subsubsection{N Way Selection}\label{subsubsec:N_Way_Selection}
\paragraph{Design Issues}\label{par:N_Way_Selection-Design_Issues}
\paragraph{Examples of Multiple Selectors}\label{par:N_Way_Selection-Examples}

\subsection{Iterative Statements}\label{subsec:Iterative_Statements}
\begin{definition}[Iterative Statement]\label{def:Iterative_Statement}
  An \emph{iterative statement} is one that causes a statement or collection of statements to be executed zero ore more times.
  These are frequently called \emph{loops}.

  There are 2 questions that need to be answered to design iterative statements.
  \begin{enumerate}[noitemsep]
  \item How is the iteration controlled?
  \item Where should the control mechanism appear in the loop statement, i.e.\ when is it executed in relation to the code in the statement body?
  \end{enumerate}

  \begin{remark}[Iteration Statement]\label{rmk:Iteration_Statement}
    An \emph{iteration statement} is the combination of the \nameref{def:Iterative_Statement-Body} and the control mechanism.
  \end{remark}
\end{definition}

\nameref{subsubsec:Counter_Controlled_Loops} and \nameref{subsubsec:Counter_Controlled_Loops}, and a combination of the 2, are the main ways to control an \nameref{def:Iterative_Statement}.
The 2 main choices of when to execute the control mechanism of the loop are:
\begin{enumerate}[noitemsep]
\item At the beginning of the loop, before the body. \nameref{def:Iterative_Statement-Pretest_Control}
\item At the end of the loop, after the body. \nameref{def:Iterative_Statement-Posttest_Control}
\end{enumerate}

\begin{definition}[Body]\label{def:Iterative_Statement-Body}
  The \emph{body} of an \nameref{def:Iterative_Statement} is the collection of statements whose execution is controlled by the iteration station statement.
\end{definition}

\begin{definition}[Pretest]\label{def:Iterative_Statement-Pretest_Control}
  A \emph{pretest} \nameref{def:Iterative_Statement} means the test for loop completion, the control mechanism, is evaluated before the loop \nameref{def:Iterative_Statement-Body} executes.
\end{definition}

\begin{definition}[Posttest]\label{def:Iterative_Statement-Posttest_Control}
  A \emph{posttest} \nameref{def:Iterative_Statement} means the test for loop completion, the control mechanism, is evaluated after the loop \nameref{def:Iterative_Statement-Body} executes.
\end{definition}

\subsubsection{Counter-Controlled Loops}\label{subsubsec:Counter_Controlled_Loops}
\begin{definition}[Counter Iteration Statement]\label{def:Counter_Iteration_Statement}
  A \emph{counter iteration statement} has:
  \begin{itemize}[noitemsep]
  \item A \emph{loop variable}, where the count value is maintained
  \item The \emph{loop parameters}. These include
    \begin{itemize}[noitemsep]
    \item A means of specifying the \emph{initial} and \emph{terminal} values of the loop variable
    \item The \emph{step size}, the difference between sequential loop variable values
    \end{itemize}
  \end{itemize}
\end{definition}

\nameref{subsubsec:Logically_Controlled_Loops} are more \textbf{general} than \nameref{subsubsec:Counter_Controlled_Loops}, they are not necessarily more commonly used.
The greater complexity of \nameref{subsubsec:Counter_Controlled_Loops} means that their design is more demanding.

\begin{remark*}[Machine Instructions]
  Sometimes there are machine-level instructions for performing \nameref{subsubsec:Counter_Controlled_Loops}.
  However, these are rare, because a language can outlive a processor's ISA.\@
\end{remark*}

\paragraph{Design Issues}\label{par:Counter_Controlled_Loops-Design_Issues}
\begin{itemize}[noitemsep]
\item What are the \nameref{def:Data_Type}, and \nameref{def:Variable_Scope} of the loop variable?
\item Should it be legal for the loop variable or loop parameters to be changed in the loop, and if so, does it affect the loop control?
\item Should the loop parameters be evaluated only once, or once for each iteration?
\end{itemize}

\paragraph{The \texttt{for} Statement of Ada}\label{par:Counter_Controlled_Loops-Ada}
The Ada \texttt{for} statement can be generalized to
\begin{minted}[frame=lines,linenos]{ada}
for variable in [reverse] discrete_range loop
    -- Body statements
end loop;
\end{minted}
\begin{itemize}[noitemsep]
\item The \texttt{discrete\textunderscore{}range} is a \nameref{def:Subrange_Type} of an integer or \nameref{def:Enumeration_Type}, such as \texttt{1..10} or \texttt{Monday..Friday}.
\item The \texttt{reverse} \nameref{def:Reserved_Word} indicates the values of the \texttt{discrete\textunderscore{}range} are assigned in reverse order.
\item The \texttt{variable} is only in scope of the loop
  \begin{itemize}[noitemsep]
  \item It is implicitly declared at the beginning of the \texttt{for} statement
  \item and implicitly undeclared after loop termination
  \item It also shadows/masks variables with the same name as it
  \end{itemize}
\item The \texttt{body} cannot assign the loop variable a new value
\end{itemize}

The operational semantics are given below.
\begin{verbatim}
  [define for_var (its type is that of the discrete range)]
  [evaluate discrete range]
loop:
  if [there are no elements left in the discrete range] goto out
  for_var = [next element of discrete range]
  [loop body]
  goto loop
out:
  [undefine for_var]
\end{verbatim}

\paragraph{The \texttt{for} Statement of the C-Based Languages}\label{par:Counter_Controlled_Loops-C_Langs}
The C \texttt{for} statement can be generalized to
\begin{minted}[frame=lines,linenos]{c}
for (expression_1; expression_2; expression_3) {
    // Body statements
}
\end{minted}
\begin{itemize}[noitemsep]
\item \texttt{expression\textunderscore{}1} is usually an assignment statement to create the loop variable, which can produce results
  \begin{itemize}[noitemsep]
  \item This is only evaluated once, before the loop begins
  \item In early versions of C, these could \textbf{not} be definitions (\texttt{int count = 0;})
  \end{itemize}
\item \texttt{expression\textunderscore{}2} is the loop control, it makes comparisons against the loop variable
  \begin{itemize}[noitemsep]
  \item This is evaluated before every iteration of the loop body
  \end{itemize}
\item \texttt{expression\textunderscore{}3} manipulates the loop variable
  \begin{itemize}[noitemsep]
  \item This is evaluated after every iteration of the loop body
  \end{itemize}
\item All 3 of these expressions are optional.
  \begin{itemize}[noitemsep]
  \item Having no \texttt{expression\textunderscore{}2} means the comparison is always true, which may result in an infinite loop
  \item If \texttt{expression\textunderscore{}1} is absent, no loop variable is initialized, but that doesn't stop other \nameref{def:Variable}s declared elsewhere from being part of the control.
  \end{itemize}
\end{itemize}

Because C expressions can be used as statements, expression evaluations are shown as statements in this operational semantics.
\begin{verbatim}
  expression_1
loop:
  if expression_2 = 0 goto out
  [loop body]
  expression_3
  goto loop
out:
  ...
\end{verbatim}

C's design choices for its \texttt{for} loop are as follows:
\begin{itemize}[noitemsep]
\item No explicit loop variables or loop parameters
\item All involved variables can be changed in the loop body
\item Expressions are evaluated in the order shown before.
  \begin{itemize}[noitemsep]
  \item There can even be multiple expressions present in each of the \texttt{expression}s.
  \end{itemize}
\item You can branch into a C \texttt{for} loop body
\end{itemize}

\paragraph{The \texttt{for} Statement of Python}\label{par:Counter_Controlled_Loops-Python}
The Python \texttt{for} statement can be generalized to
\begin{minted}[frame=lines,linenos]{python3}
for loop_variable in object:
    # Body statements
[else:
    # Else clause]
\end{minted}

\begin{itemize}[noitemsep]
\item The \texttt{loop\textunderscore{}variable} is assigned the one of the values in the object, which is often a range, for each iteration.
\item The \texttt{else} clause is executed if the loop terminates \textbf{normally}
\end{itemize}

\subsubsection{Logically Controlled Loops}\label{subsubsec:Logically_Controlled_Loops}
Collections of statements must be executed repeatedly, but should be controlled by a Boolean expression rather than a counter.
Logically controlled loops are convenient for this.
They are also more general than \nameref{subsubsec:Counter_Controlled_Loops}.

\paragraph{Design Issues}\label{par:Logically_Controlled_Loops-Design_Issues}
\begin{itemize}[noitemsep]
\item Should the control be \nameref{def:Iterative_Statement-Pretest_Control} or \nameref{def:Iterative_Statement-Posttest_Control}?
\item Should the logically controlled loop be a special form of a counting loop or a separate statement?
\end{itemize}

\paragraph{Examples}\label{par:Logically_Controlled_Loops-Examples}
C-based programming languages have both \nameref{def:Iterative_Statement-Pretest_Control} and \nameref{def:Iterative_Statement-Posttest_Control} \nameref{subsubsec:Logically_Controlled_Loops}.
These loops have these respective forms:
\begin{minted}[frame=lines,linenos]{c}
while (control_expression) {
    // Loop body
}
\end{minted}
\begin{minted}[frame=lines,linenos]{c}
do
    // Loop body
while (control_expression);
\end{minted}

The \nameref{def:Iterative_Statement-Pretest_Control} (\texttt{while}) executes as long as the \texttt{control\textunderscore{}expression} evaluates to true.
The \nameref{def:Iterative_Statement-Posttest_Control} (\texttt{do-while}) executes as long as the \texttt{control\textunderscore{}expression} evaluates to true.
However, the \texttt{do-while} loop will cause the loop body to be executed at least once.

The operational semantics are given below, with the \texttt{while} coming first, then the \texttt{do-while}.
\begin{verbatim}
loop:
  if control-expression is falst go to outlive
  [loop body]
  goto loop
out:
  ...
\end{verbatim}

\begin{verbatim}
loop:
  [loop body]
  if control_expression is true goto loop
\end{verbatim}

In C, it is possible to branch into these statement's bodies.
Java's \texttt{while} and \texttt{do-while} statements are similar, but they require the \texttt{control\textunderscore{}expression} be \texttt{boolean} type.

\subsubsection{Iteration Based on Data Structures}\label{subsubsec:Iteration_Based_on_Data_Structures}
A general data-based iteration statement uses a user-defined data structure and user-defined iteration to go through the elements in the structure.
An example of this is in Python, shown below
\begin{minted}[frame=lines,linenos]{python3}
for count in range(0, 9, 2)
    # Body
\end{minted}
This would set the value of \texttt{count} to \texttt{0, 2, 4, 6, 8} before each iteration.

The user-defined iterator function must be history sensitive.
It should also terminate when it fails to find more elements.

User-defined \nameref{rmk:Iteration_Statement}s are more commonly used and more important in object-oriented programming, because of the heavy use of abstract data types for data structures.
In Java, this is done by having a class implement the \texttt{Iterable} interface.
It is called the \texttt{foreach} loop, and is written as
\begin{minted}[frame=lines,linenos]{java}
for (String myElement : myList) {
    ...
}
\end{minted}

%%% Local Variables:
%%% mode: latex
%%% TeX-master: "../EDAP05-Concepts_Programming_Languages-Reference_Sheet"
%%% End:
