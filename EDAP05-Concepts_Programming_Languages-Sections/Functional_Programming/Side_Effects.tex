\subsection{\nameref*{def:Function_Side_Effect}s in Functional Languages}\label{subsec:Functional_Function_Side_Effects}
In a strict definition of \nameref{def:Functional_Programming_Language}s, \nameref{def:Function_Side_Effect}s are not allowed.
However, \nameref{def:Function_Side_Effect}s include:
\begin{itemize}[noitemsep]
\item Writing a file to disk
\item Sending packets over the network
\item Writing to standard output
\end{itemize}

However, if a functional language did not include these, the language is severely hampered in its usability.\footnote{Some functional languages are like this, but they are not widely used, so they are not discussed.}
To get ``around'' this issue, there are 3 ways functional languages can handle these non-pure actions.
\begin{enumerate}[noitemsep]
\item The language was \emph{imperative first}.
  These are languages that:
  \begin{itemize}[noitemsep]
  \item Were originally \nameref{def:Imperative_Programming_Language}s, like C, C++, Java
  \item Can be adapted to behave like functional languages if they are written correctly.
  \end{itemize}
\item The language was \emph{functional first}.
  These are languages that:
  \begin{itemize}[noitemsep]
  \item Are by default true \nameref{def:Functional_Programming_Language}s, like Haskell, F\#, the ML-family
  \item Allow \nameref{def:Function_Side_Effect}s to be contained within the functions themselves, and not outside them.
  \item Has the compiler track these functions with \nameref{def:Function_Side_Effect}s so they are handled correctly.
  \end{itemize}
\item The language is \emph{monadic}.
  There is only a single language that supports this model right now, Haskell.
  A monad is:
  \begin{itemize}[noitemsep]
  \item An abstraction or generalization over ordered things
  \item These things may have side-effects, which are tagged.
  \item Monads allow for an imperative-like state to be introduced to a functional environment. However, the state exists \emph{ONLY} for the monad's function(s) and its/their children.
  \item Monads may also allow for state to be present throughout a program, but permission to access the state must be passed to functions, like an \nameref{def:Actual_Parameter}.
  \end{itemize}
\end{enumerate}

\subsubsection{Building Side Effects into a Language}\label{subsubsec:Building_Side_Effects_to_Language}
Since inference rules describe \nameref{def:Semantics} in a ``pure'' mathematical way, side effects become additional evaluation results.
This means our evaluation relation now relates the tuple $(expr, \text{previous state})$ with the tuple $(val, \text{state afterwards})$.

For example, to model printing, you could write
\begin{equation}\label{eq:Semantic_Side_Effect_Printing}
  \frac{(e, P) \EvaluatesTo (v, P')}{(\mathtt{PRINT\,} e, P) \EvaluatesTo (v, P'+[v])}
\end{equation}
\begin{description}[noitemsep]
\item $e$, The expression to be evaluated and printed.
\item $P$, The previous state of the system, in this case, the list of all previously printed numbers.
\item $v$, The value the expression $e$ evaluates to.
\item $P'$, The state of the system after printing, in this case, the list of all printed numbers up to this one.
\item $P'+[v]$, The list of printed numbers up to this point, with the value $v$ appended.
\end{description}

\begin{remark*}
  In \Cref{eq:Semantic_Side_Effect_Printing}, $\mathtt{PRINT} x$ would also return the evaluation result of $x$, so $\mathtt{PRINT} (\mathtt{PRINT} 7)$ would print the number 7 twice.
\end{remark*}

\subsubsection{Modelling Side Effect Updating of Variables}\label{subsubsec:Model_Side_Effect_Variable_Update}
In most \nameref{def:Imperative_Programming_Language}s, variables can be updated, i.e.\ have their value changed from under them.
This is different than in \nameref{def:Functional_Programming_Language}s, which technically create a new variable with the same name and assign the new value.

Modelling updateable variables is a bit more tricky but still possible.
The challenge here is that our previous notion of \nameref{def:Semantic_Environment}s is no longer powerful enough, because a subexpression may change global state, as seen in the \texttt{MYSTERY} code below.

\inputminted[frame=lines,linenos]{pascal}{./EDAP05-Concepts_Programming_Languages-Sections/Functional_Programming/Code/Global_State_Change.pas}

There are two models for modelling this.
The simpler one turns the environment into a parameter of the evaluation relation, so that you get
\begin{equation}\label{eq:Simple_Side_Effect_Variable_Update}
  (e, E) \EvaluatesTo (v, E')
\end{equation}

However, the model in \Cref{eq:Simple_Side_Effect_Variable_Update} is too weak for real languages like Java or Scala, since those lanugages permit \nameref{def:Aliasing}.

The second model can support \nameref{def:Aliasing}.
To do this, we usually use a two-step look-up process, where the first lookup (with the \nameref{def:Semantic_Environment} $E$) determines a variable's memory address $a$.
The second step uses a so-called store to translate the address to a value.
This gives rules like the following:
\begin{subequations}\label{eq:Strong_Side_Effect_Variable_Update}
  \begin{equation}\label{subeq:Strong_Side_Effect_Variable_Update-Read_Variable}
    \frac{E[x] = a \:\: S[a] = v}{E \vdash (x, S) \EvaluatesTo (v, S)} \SemanticRuleName{Read Variable}
  \end{equation}
  \begin{equation}\label{subeq:Strong_Side_Effect_Variable_Update-Update_Variable}
    \frac{E \vdash (x, S) \EvaluatesTo (v, S') \:\: E[x] = a}{E \vdash (x := e, S) \EvaluatesTo (v, S'[a \mapsto v])} \SemanticRuleName{Update Variable}
  \end{equation}
  \begin{equation}\label{subeq:Strong_Side_Effect_Variable_Update-Sequence}
  \frac{E \vdash (S, s_{1}) \EvaluatesTo (v_{1}, S') \:\: E \vdash (S', s_{2}) \EvaluatesTo (v_{2}, S'')}{E \vdash (s_{1}; s_{2}, S) \EvaluatesTo (v_{2}, S'')} \SemanticRuleName{Sequence}
  \end{equation}
\end{subequations}

With this kind of model it becomes easy to model \nameref{def:Pass_By_Reference} versus \nameref{def:Pass_By_Value} \nameref{def:Semantics}, or to describe the \texttt{\&} and \texttt{*} operators from C and C++:
\begin{equation}\label{eq:C_Get_Address}
  \frac{E[x] = a}{E \vdash (\&x, S) \EvaluatesTo (a, s)} \SemanticRuleName{Get Address}
\end{equation}

\begin{equation}\label{eq:C_Dereference}
  \frac{E \vdash (e, S) \EvaluatesTo (a, S') \:\: v = S'[a]}{E \vdash (*e, S) \EvaluatesTo (v, S)} \SemanticRuleName{Dereference Address}
\end{equation}

%%% Local Variables:
%%% mode: latex
%%% TeX-master: "../../EDAP05-Concepts_Programming_Languages-Reference_Sheet"
%%% End:
