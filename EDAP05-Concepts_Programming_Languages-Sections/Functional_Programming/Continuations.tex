\subsection{Continuations}\label{subsec:Continuations}
\begin{definition}[Continuation]\label{def:Continuation}
  \emph{Continuation}s allow us to abstract in the opposite direction of \nameref{def:Variable}s.
  They allow us to abstract over the \underline{context in which the values get used}.

  They describe the required input to an expression (variables, values, other expressions, etc.), any computations required by the expression (adding a value, computing another expression, etc.) the command to be executed on a line (the instruction, \texttt{print}, etc.) and then the next statement.

  For example, in the code snippet of \texttt{Mystery} below, there is a continuation on each line where a \nameref{def:Variable} is used in place of a value.
\begin{minted}[frame=lines,linenos]{pascal}
VAR x : INTEGER;
VAR y : INTEGER
BEGIN
  ...
  PRINT x + 1;
  PRINT y
END
\end{minted}
  So, there are continuations on lines 5 and 6.
  The one on line 6 is ``take an integer value, \texttt{PRINT} that value, and then \texttt{END} the program''.
  The continuation on line 5 is ``take an integer value, add 1 to it, \texttt{PRINT} the result, then \texttt{PRINT} y, and finally \texttt{END} the program''.
\end{definition}

\inputminted[frame=lines,linenos]{sml}{./EDAP05-Concepts_Programming_Languages-Sections/Functional_Programming/Code/20191216-Continuation_Loop.sml}

%%% Local Variables:
%%% mode: latex
%%% TeX-master: "../../EDAP05-Concepts_Programming_Languages-Reference_Sheet"
%%% End:
