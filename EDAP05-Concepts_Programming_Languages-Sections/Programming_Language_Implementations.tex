\section{Programming Language Implementations}\label{sec:Lang_Implementations}
There are 3 main ways for a programming language to be implemented:
\begin{enumerate}[noitemsep]
\item \nameref{subsec:Interpretation}
\item \nameref{subsec:Compilation}
\item \nameref{subsec:Hybrid_Implementation}
\end{enumerate}

There are benefits and drawbacks for each of these implementations:
\begin{table}[h!]
  \centering
  \begin{tabular}{cccc}
    \toprule
    Property & \nameref{subsec:Interpretation} & \nameref{subsec:Compilation} & \nameref{subsec:Hybrid_Implementation} \\
    \midrule
    Execution Performance & Slow & Fast & Fast \\
    Turnaround & Fast & Slow (Compile and Link) & Fast (Compile when needed) \\
    Language Flexibility & High & Limited & High \\
    \bottomrule
  \end{tabular}
  \caption{Pros and Cons for Programming Language Implementations}
  \label{tab:Lang_Implementations_Pros_Cons}
\end{table}

There is a trade-off to be made between:
\begin{itemize}[noitemsep]
\item Language flexibility
\item CPU time / RAM time
\end{itemize}

\subsection{Interpretation}\label{subsec:Interpretation}
\begin{definition}[Interpretation]\label{def:Interpretation}
  If a programming language is implemented with \emph{interpretation}, is \emph{interpreted}, then there is an intermediate program that runs between the source code and what the CPU can run on.
  This \emph{interpreter} reads the high-level source code, then alternates between:
  \begin{itemize}[noitemsep]
  \item Figure out next command
    \begin{itemize}[noitemsep]
    \item This means that the current instruction is parsed in
    \item Equivalent commands are generated in the CPU-specific or VM-specific instruction sets from the high-level source code
    \end{itemize}
  \item Execute Command
  \end{itemize}
\end{definition}

Some examples of languages with a \nameref{def:Interpretation} implementation are:
\begin{itemize}[noitemsep]
\item Python
\item Perl
\item Ruby
\item Bash
\item AWK
\item $\cdots$
\end{itemize}

\subsection{Compilation}\label{subsec:Compilation}
\begin{definition}[Compilation]\label{def:Compilation}
  If a programming language is implemented with \emph{compilation}, is \emph{compiled}, then there are several programs that must be run before the high-level source code can be run.
  \begin{enumerate}[noitemsep]
  \item The \nameref{def:Compiler}
  \item The \nameref{def:Assembler}
  \item The \nameref{def:Linker}
  \item The \nameref{def:Loader}
  \end{enumerate}
\end{definition}

\begin{definition}[Compiler]\label{def:Compiler}
  The \emph{compiler} is the main program needed in a compiled language implementation.
  It is responsible for taking the high-level source code written in some language, and converting it to assembly code, which can then be run through an \nameref{def:Assembler}.

  The steps involved in a compiler are:
  \begin{enumerate}[noitemsep]
  \item Lexical Analysis/Tokenizing: Convert the input file into a set of tokens
  \item Syntactic Analysis/Parsing: Convert the tokens into a tree representing all the tokens in the program
  \item Semantic Analysis: Interpret the program and ensure that everything expressed in the program is correct.
    \begin{itemize}[noitemsep]
    \item This is where compile-time errors are \textbf{usually} caught. Though, this is just a generalization.
    \item Type analysis is handled here for instance
    \end{itemize}
    
  \item Optimize the Code: The output assembly code could be optimized before actually making the output. Take care of that here.
  \item Output Assembly: With the potentially optimized machine-equivalent code from our program, write out the equivalent assembly, and finish the compilation process.
  \end{enumerate}

  \begin{remark}
    The specifics of a \nameref{def:Compiler}'s implementation are \textbf{not} discussed in this course, but it is useful to know the basics of the compilation process.
    For both the implementation details, please refer to \href{run:./EDAN65-Compilers-Reference_Sheet.pdf}{EDAN65:Compilers-Reference Material}.
  \end{remark}
\end{definition}

\begin{definition}[Assembler]\label{def:Assembler}
  The \emph{assembler} is an intermediate program used after the \nameref{def:Compiler} has been run.
  The assembler takes the assembly code that the \nameref{def:Compiler} outputs and applies a one-to-one mapping.
  Since all assembly code is just an abstraction and humanization of machine code in a one-to-one mapping fashion, the assembler takes the assembly code and converts it to its equivalent machine code.

  \begin{remark}
    This particular program is not discussed heavily in this course.
  \end{remark}
\end{definition}

\begin{definition}[Linker]\label{def:Linker}
  The \emph{linker} is an intermediate program, that may be provided by the operating system or may be provided by that language implementation's tooling.
  It is run after the \nameref{def:Compiler} and/or the \nameref{def:Assembler} have been run.
  \begin{itemize}[noitemsep]
  \item Provided by operating system
    \begin{itemize}[noitemsep]
    \item If the programming language implementation relies on the operating system and critical portions of the system.
    \end{itemize}
  \item Provided by the language implementation's tooling
    \begin{itemize}[noitemsep]
    \item If the implementation provides certain libraries, it will likely have their own linker too.
    \end{itemize}
  \end{itemize}

  \begin{remark}
    This particular program is not discussed in this course.
  \end{remark}
\end{definition}

\begin{definition}[Loader]\label{def:Loader}
  The \emph{loader} is the program provided by the operating system that loads the specified program into main memory and begins execution.
  
  \begin{remark}
    This particular program is not discussed in this course.
  \end{remark}
\end{definition}

Some examples of languages with a \nameref{def:Compilation} implementation are:
\begin{itemize}[noitemsep]
\item C
\item C++
\item SML
\item Haskell
\item FORTRAN
\item $\cdots$
\end{itemize}

\subsection{Hybrid Implementation}\label{subsec:Hybrid_Implementation}
\begin{definition}[Hybrid Implementation]\label{def:Hybrid_Implementation}
  A programming language can be implemented with a \emph{hybrid implementation}.
  This means that it takes some aspects of a language implemented by \nameref{def:Interpretation} and some aspects of the language implemented with \nameref{def:Compilation}.

  For example, Java does this with their Just-In-Time (JIT) compilation scheme.
\end{definition}

One way to do this is with \nameref{subsubsec:Dynamic_Compilation}.
\subsubsection{Dynamic Compilation}\label{subsubsec:Dynamic_Compilation}
\begin{itemize}[noitemsep]
\item Idea: behind dynamic compilation is that code is compiled \emph{while executing}.
\item Theory: The best of \nameref{def:Interpretation} and \nameref{def:Compilation} worlds.
\item Practice:
  \begin{itemize}[noitemsep]
  \item Difficult to build
  \item Memory usage can increase (sometimes dramatically)
  \item Performance can be higher than pre-compiled code, because only the code needed is compiled.
  \end{itemize}
\end{itemize}

Some examples of these are:
\begin{itemize}[noitemsep]
\item Java
\item Scala
\item C\#
\item JavaScript
\item $\cdots$
\end{itemize}


%%% Local Variables:
%%% mode: latex
%%% TeX-master: "../EDAP05-Concepts_Programming_Languages-Reference_Sheet"
%%% End:
