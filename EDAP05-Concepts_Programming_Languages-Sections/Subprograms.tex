\section{Subprograms}\label{sec:Subprograms}
This is a way to perform \nameref{par:Process_Abstraction}.
This generally improves \nameref{subsec:Readability} and \nameref{subsec:Reliability}.

\subsection{General Characteristics}\label{subsec:Suprogram_Characteristics}
\begin{itemize}[noitemsep]
\item Each subprogram has a single entry point
\item The calling program unit is suspended during the execution of the called subprogram, which implies there is only one subprogram in execution at any given time
\item Control always returns to the caller whenthe subprogram terminates
\end{itemize}

Alternatives to these generalizations result in coroutines and concurrent units.

\subsection{Basic Definitions}\label{subsec:Subprogram_Definitions}
\begin{definition}[Subprogram Definition]\label{def:Subprogram_Definition}
  A \emph{subprogram definition} describes the interface to and the actions of the subprogram abstraction
\end{definition}

\begin{definition}[Subprogram Call]\label{def:Subprogram_Call}
  A \emph{subprogram call} is the explicit request that a specific subprogram be executed.

  \begin{remark}[Call]\label{rmk:Subprogram_Call}
    This is generally shortened to just a \emph{call}.
  \end{remark}
\end{definition}

\begin{definition}[Active]\label{def:Subprogram_Active}
  A subprogram is said to be \emph{active} if it has been called, but not yet completed its execution.
\end{definition}

\begin{definition}[Subprogram Header]\label{def:Subprogram_Header}
  A \emph{subprogram header} is the first part of a \nameref{def:Subprogram_Definition}.
  This serves several purposes:
  \begin{enumerate}[noitemsep]
  \item Specifies that the following syntactic unit is a \nameref{def:Subprogram_Definition} of some kind.
  \item Provides a name for the subprogram, if it's not an anonymous subprogram.
  \item Optionally specify a list of parameters.
  \end{enumerate}
\end{definition}

\begin{definition}[Subprogram Body]\label{def:Subprogram_Body}
  The \emph{subprogram body} defines the actions that the subprogram takes when there is a \nameref{def:Subprogram_Call}.
  The body may be delimited with curly-braces, \texttt{\{} and \texttt{\}}.
  It may be whitespace delimited, Python.
  It may also have an \texttt{end} statement that ends the execution of that block.
\end{definition}

Python is unique in that its \nameref{def:Subprogram_Definition}s can be executed in control-statement blocks.
For example,
\begin{minted}[frame=lines,linenos]{python3}
if conditional_expression :
    def func(...):
        ...
else:
    def func(...):
        ...
\end{minted}
This means that there are 2 possible \nameref{def:Subprogram_Definition}s possible during runtime, and which one is currently valid depends on the result of the \texttt{conditional\textunderscore{}expression}.

\begin{definition}[Parameter Profile]\label{def:Subprogram_Parameter_Profile}
  The \emph{parameter profile} of a subprogram contains the number, order, and types of its \nameref{def:Formal_Parameter}s.
\end{definition}

\begin{definition}[Protocol]\label{def:Subprogram_Protocol}
  The \emph{protocol} of a subprogram is its \nameref{def:Subprogram_Parameter_Profile}, and if its a function, its return type.
\end{definition}

\begin{definition}[Subprogram Declaration]\label{def:Subprogram_Declaration}
  A \emph{subprogram declaration} is the act of providing type and name information, but not giving any \nameref{def:Subprogram_Body}.
  This is needed in languages that do not allow forward references to subprograms.

  \begin{remark}[Prototype]\label{rmk:Subprogram_Prototype}
    In C/C++, if a subprogram needs to be declared, it is called a \emph{prototype}.
    These are generally specified in \emph{header} files, with a file extension of \texttt{.h}.
  \end{remark}
\end{definition}

\subsection{Parameters}\label{subsec:Subprogram_Parameters}
Subprograms typically want access to nonlocal data to perform their computations.
There are 2 ways to gain access to this nonlocal data:
\begin{enumerate}[noitemsep]
\item Direct access to nonlocal \nameref{def:Variable}s (\nameref{def:Global_Variable}s)
\item Parameter passing
\end{enumerate}

Data that is passed to the subprogram as a parameter is acccessed through names local to the subprogram.
Parameter passing is more flexible, because if direct access is used, new storage needs to be allocated for computation results.
Direct access also leads to issues with \nameref{def:Variable}s being visible to places they shouldn't be.
Pure \nameref{def:Functional_Programming_Language}s avoid this by having all their data being immutable.

\begin{definition}[Formal Parameter]\label{def:Formal_Parameter}
  A \emph{formal parameter} are the parameters present in the \nameref{def:Subprogram_Header}.
  These are sometimes thought of as ``dummy variables'' because they aren't normal \nameref{def:Variable}s.
  They are only bound to storage when the subprogram is called, and that storage is often through some other program \nameref{def:Variable}s.

  \begin{remark}[Parameter]\label{rmk:Parameter}
    Sometimes \nameref{def:Formal_Parameter}s are just called \emph{parameter}s, usually when \nameref{def:Actual_Parameter}s are called \nameref{rmk:Argument}s.
  \end{remark}
\end{definition}

\begin{definition}[Actual Parameter]\label{def:Actual_Parameter}
  An \emph{actual parameter} is the parameter that is bound to the \nameref{def:Formal_Parameter} of a subprogram.
  
  \begin{remark}[Argument]\label{rmk:Argument}
    Sometimes \nameref{def:Actual_Parameter}s are called \emph{argument}s, usually when \nameref{def:Formal_Parameter}s are called \nameref{rmk:Parameter}s.
  \end{remark}
\end{definition}

The binding of \nameref{def:Actual_Parameter}s to \nameref{def:Formal_Parameter}s is usually done by position.
So, the first \nameref{def:Actual_Parameter} is bound to the first \nameref{def:Formal_Parameter}.
However, this is only a good method when the number of parameters is small.

When the \nameref{def:Formal_Parameter} list gets long, it is hard to get all of the \nameref{def:Actual_Parameter}s in the right order.
One solution is to use \nameref{def:Keyword_Parameter}s.

\begin{definition}[Keyword Parameter]\label{def:Keyword_Parameter}
  \emph{Keyword parameter}s have the name of the \nameref{def:Formal_Parameter} usable by the \nameref{def:Actual_Parameter} to bind the \nameref{def:Variable_Value}.

  For example, if \texttt{sumer} has the \nameref{def:Formal_Parameter}s \texttt{length}, \texttt{list}, and \texttt{sum}:
\begin{minted}[frame=lines,linenos]{python3}
sumer(length = my_length, list = my_array, sum = my_sum)
\end{minted}

  Advantages and Disadvantages of keyword parameters:
  \begin{itemize}[noitemsep]
  \item Advantages
    \begin{itemize}[noitemsep]
    \item No need to remember \nameref{def:Formal_Parameter} order.
    \end{itemize}
  \item Disadvantages
    \begin{itemize}[noitemsep]
    \item Need to remember the name of the \nameref{def:Formal_Parameter}s.
    \end{itemize}
  \end{itemize}

  \begin{remark}[End of \nameref{def:Actual_Parameter}]\label{rmk:Keyword_Parameters_at_End}
    In an \nameref{def:Actual_Parameter} list, all parameters after a \nameref{def:Keyword_Parameter} \textbf{must} be keyworded, because the list may not be well-formed enough for parameters to line up by position.
  \end{remark}
\end{definition}

Languages that support default values on \nameref{def:Formal_Parameter}s handle them differently.
In Python, regular \nameref{def:Formal_Parameter}s and ones with default values can be in any order.
However, in C++, which does not support \nameref{def:Keyword_Parameter}s, \nameref{def:Formal_Parameter}s with default values must be at the end of the \nameref{def:Subprogram_Header}.
This is illustrated in the next 2 code blocks, which are in Python and C++, respectively.
\begin{minted}[frame=lines,linenos]{python3}
def compute_pay(income, exemptions = 1, tax_rate)
\end{minted}

\begin{minted}[frame=lines,linenos]{c++}
float compute_pay(float income, float tax_rate, int exemptions = 1)
\end{minted}

Other languages have more varied and interesting ways to pass \nameref{def:Actual_Parameter}s to subprograms.
Look at the language specification for more details.

\subsection{Local Referencing Environments}\label{subsec:Local_Referencing_Environments}
Issues related to \nameref{def:Variable}s defined within subprograms.

\subsubsection{Local Variables}\label{subsubsec:Local_Variables}
The definition of \nameref{def:Local_Variable}s is given in \Cref{def:Local_Variable}.
These can be either \nameref{def:Static_Variable_Binding_Lifetime} or \nameref{def:Stack-Dynamic_Variable_Binding_Lifetime}.

If \nameref{def:Local_Variable}s are \nameref{def:Stack-Dynamic_Variable_Binding_Lifetime}, they are bound to storage when the subprogram begins and unbound when that execution terminates.
The advantages and disadvantages of \nameref{def:Stack-Dynamic_Variable_Binding_Lifetime} are:
\begin{itemize}[noitemsep]
\item Advantages
  \begin{itemize}[noitemsep]
  \item Allows for recursive subprograms
  \item Inactive subprograms can share \nameref{def:Memory} with the active subprogram
  \end{itemize}
\item Disadvantages
  \begin{itemize}[noitemsep]
  \item The cost of the time requried to allocate, initialize, and deallocate these variables
  \item The indirect \nameref{def:Memory} accesses to the data
  \item When all \nameref{def:Variable}s are \nameref{def:Stack-Dynamic_Variable_Binding_Lifetime}, subprograms cannot be history sensitive.
  \end{itemize}
\end{itemize}

However, the advantages and disadvantages of \nameref{def:Static_Variable_Binding_Lifetime} are:
\begin{itemize}[noitemsep]
\item Advantages
  \begin{itemize}[noitemsep]
  \item No runtime overhead to allocate/deallocate the storage
  \item Direct \nameref{def:Memory} access (Absolute addressing)
  \end{itemize}
\item Disadvantages
  \begin{itemize}[noitemsep]
  \item Inability to support recursion
  \item Cannot share \nameref{def:Memory} with inactive subprograms.
  \end{itemize}
\end{itemize}

Most contemporary programming languages make their \nameref{def:Local_Variable}s \nameref{def:Stack-Dynamic_Variable_Binding_Lifetime} by defautlt.
However, this can usually be overridden with a \texttt{static} keyword.

\subsubsection{Nested Subprograms}\label{subsubsec:Nested_Subprograms}
The idea was to create a hierarchy of logic and \nameref{def:Variable_Scope}s.
The motivation was that if a subprogram is only used within one other subprogram, why not place it there, and hide it from the rest of the program?

\nameref{def:Static_Scoping} is usually used in languages that allow nested subprograms.
Language support of this featuer depends \textbf{heavily} on the language.
You will have to chek the language specification to find out if the programming language supports them.

\subsection{Parameter-Passing Methods}\label{subsec:Parameter_Passing_Methods}
How are \nameref{def:Actual_Parameter}s passed to the subprograms?

\subsubsection{Semantic Models of Parameter Passing}\label{subsubsec:Semantic_Models_Parameter_Passing}
\nameref{def:Formal_Parameter}s are characterized by 1 of 3 distinct \nameref{def:Semantics} models:
\begin{enumerate}[noitemsep]
\item The \nameref{def:Formal_Parameter}s receive data from the corresponding \nameref{def:Actual_Parameter}. This is called \textbf{\nameref{def:Parameter_Passing-In_Mode}}.
\item The \nameref{def:Formal_Parameter}s  can transmit the computed data back to the \nameref{def:Actual_Parameter}. This is called \textbf{\nameref{def:Parameter_Passing-Out_Mode}}.
\item They can do both 1 and 2. This is called \textbf{\nameref{def:Parameter_Passing-Inout_Mode}}.
\end{enumerate}

\begin{definition}[In Mode]\label{def:Parameter_Passing-In_Mode}
  \nameref{def:Formal_Parameter}s can receive data from the corresponding \nameref{def:Actual_Parameter}.
  This is called \emph{in mode}.

  This is generally used when passing \nameref{def:Actual_Parameter}s to a subprogram.
\end{definition}

\begin{definition}[Out Mode]\label{def:Parameter_Passing-Out_Mode}
  \nameref{def:Formal_Parameter}s transmit the computed data back to the corresponding \nameref{def:Actual_Parameter}s.
  This is called \emph{out mode}.

  This is similar to a return value, but acts \emph{only} on the actual parameters.
\end{definition}

\begin{definition}[In/Out Mode]\label{def:Parameter_Passing-Inout_Mode}
  The \nameref{def:Formal_Parameter}s can both transmit and receive data from the corresponding \nameref{def:Actual_Parameter}s.
  This is called \emph{in/out mode}.

  This is generally used when a \nameref{def:Subprogram_Call} takes in \nameref{def:Actual_Parameter}s and returns values in those same parameters.
\end{definition}

There are 2 conceptual models of how data transfers take place in parameter transmission:
\begin{enumerate}[noitemsep]
\item An actual value is copied (to the caller, to the callee, or both).
\item Or an access path is transmitted. This is usually a pointer/reference.
\end{enumerate}

\subsubsection{Implementation Models of Parameter Passing}\label{subsubsec:Implementation_Models_Parameter_Passing}
A great variety of implementations for parameter passing have been put together.
Here, we list some of them, and discuss their respective advantages and disadvantages.

\begin{remark*}[Call-by-...]
  All of these models can have the ``Pass-by'' replaced with ``Call-by'', which means the same thing.
\end{remark*}

\paragraph{Pass-by-Value}\label{par:Parameter_Passing-Pass_By_Value}
\begin{definition}[Pass-by-Value]\label{def:Pass_By_Value}
  When a parameter is \emph{passed-by-value}, the value of the \nameref{def:Actual_Parameter} is used to initialize the corresponding \nameref{def:Formal_Parameter}, which then acts as a \nameref{def:Local_Variable} in the subprogram.
  This implements \nameref{def:Parameter_Passing-In_Mode} \nameref{def:Semantics}.

  Passing a parameter by value is typically done by copying the \nameref{def:Actual_Parameter}'s \nameref{def:Variable_Value} for the \nameref{def:Formal_Parameter}.
  This means we don't have to make the \nameref{def:Memory} cell read-only, because making cells read-only can be difficult.

  The advantages and disadvantages are:
  \begin{itemize}[noitemsep]
  \item Advantages
    \begin{itemize}[noitemsep]
    \item Fast to copy scalars in both linkage cost and access time
    \end{itemize}
  \item Disadvantages
    \begin{itemize}[noitemsep]
    \item Additional \nameref{def:Memory} is required for the \nameref{def:Formal_Parameter}'s new value.
    \item The \nameref{def:Actual_Parameter} must be copied to the storage area for the corresponding \nameref{def:Formal_Parameter}
    \item Difficult to implement by transmitting access paths
    \item This copying can be expensive if the \nameref{def:Actual_Parameter} is large, like an array with many elements.
    \end{itemize}
  \end{itemize}
\end{definition}

\paragraph{Pass-by-Result}\label{par:Parameter_Passing-Pass_By_Result}
\begin{definition}[Pass-by-Result]\label{def:Pass_By_Result}
  \emph{Pass-by-result} is an implementation for \nameref{def:Parameter_Passing-Out_Mode} parameters.
  When a parameter is passed-by-result, no \nameref{def:Variable_Value} is transmitted to the subprogram.
  The corresponding \nameref{def:Formal_Parameter} acts as a \nameref{def:Local_Variable}, but before control is transferred back to the caller, the \nameref{def:Formal_Parameter}'s result is transmitted back to the caller's \nameref{def:Actual_Parameter}.

  The advantages and disadvantages of pass-by-result are fairly similar to \nameref{def:Pass_By_Value}, but with some extra disadvantages.
  \begin{itemize}[noitemsep]
  \item Advantages
    \begin{itemize}[noitemsep]
    \item Fast to copy scalars
    \end{itemize}
  \item Disadvantages
    \begin{itemize}[noitemsep]
    \item If the values are returned by copying the value into the \nameref{def:Actual_Parameter}, extra storage and copy operations are required.
    \item Difficult to transmit an access path.
      \begin{itemize}[noitemsep]
      \item Need to ensure the initial value of the \nameref{def:Actual_Parameter} is \textbf{not} used in the subprogram.
      \end{itemize}
    \item There can be an \nameref{def:Actual_Parameter} collision.
\begin{minted}[frame=lines,linenos]{csharp}
void Fixer(out int x, out int y) {
  x = 17;
  y = 35;
}
...
f.Fixer(out a, out a); // What gets assigned first? Is a=17 or a=35? Who knows?
\end{minted}
    \item A similar issue arises when the implementor can choose between 2 different times to evaluate the addresses of the \nameref{def:Actual_Parameter}s. This is illustrated below.
\begin{minted}[frame=lines,linenos]{csharp}
void DoIt(out int x, int index) {
  x = 17;
  index = 42;
}
...
sub = 21;
f.doIt(out list[sub], out sub); // What gets assigned to list[sub], because sub is unknown: sub=21? sub=42?
\end{minted}
    \end{itemize}
  \end{itemize}
\end{definition}

\paragraph{Pass-by-Value-Result}\label{par:Parameter_Passing-Pass_By_Value_Result}
\begin{definition}[Pass-by-Value-Result]\label{def:Pass_By_Value_Result}
  \emph{Pass-by-value-result} is an implementation of the \nameref{def:Parameter_Passing-Inout_Mode} model in which actual values are copied.
  It is a combination of \nameref{def:Pass_By_Value} and \nameref{def:Pass_By_Result}.
  The \nameref{def:Variable_Value} of the \nameref{def:Actual_Parameter} is used to initialize the \nameref{def:Formal_Parameter}, which then acts as a \nameref{def:Local_Variable}.
  At subprogram termination, the value of the \nameref{def:Formal_Parameter}'s \nameref{def:Local_Variable} is copied back to the \nameref{def:Formal_Parameter}, then copied back to the \nameref{def:Actual_Parameter}.

  \begin{remark}[Pass-by-Copy]\label{rmk:Pass_By_Copy}
    \nameref{def:Pass_By_Value_Result} is sometimes called \emph{pass-by-copy}, because the \nameref{def:Actual_Parameter} is copied to the \nameref{def:Formal_Parameter} at the subprogram's start, and then copied back at the subprogram's termination.
  \end{remark}

  The advantages and disadvantages of pass-by-value-result are shared with both \nameref{def:Pass_By_Value} and \nameref{def:Pass_By_Result}.
\end{definition}

\paragraph{Pass-by-Reference}\label{par:Parameter_Passing-Pass_By_Reference}
\begin{definition}[Pass-by-Reference]\label{def:Pass_By_Reference}
  \emph{Pass-by-reference} is a second implementation model for \nameref{def:Parameter_Passing-Inout_Mode}.
  In this case, rather than copying the actual data \nameref{def:Variable_Value}s back and forth, pass-by-reference transmits an access path, usually an address/pointer/reference, to the called subprogram.
  This allows the subprogram to access the \textbf{SAME} value as the calling program.

  The advantages and disadvantages of this implementation are:
  \begin{itemize}[noitemsep]
  \item Advantages
    \begin{itemize}[noitemsep]
    \item The passing process is efficient, in terms of time and space.
      \begin{itemize}[noitemsep]
      \item No duplicate space is required
      \item There is also no copying required
      \end{itemize}
    \end{itemize}
  \item Disadvantages
    \begin{itemize}[noitemsep]
    \item Access to the \nameref{def:Formal_Parameter}s will be slower than \nameref{def:Pass_By_Value}, because of the indirect addressing required.
    \item There may be inadvertent and erroneous changes to the \nameref{def:Actual_Parameter}'s \nameref{def:Variable_Value}.
    \item \nameref{def:Aliasing} can occur.
      \begin{itemize}[noitemsep]
      \item This harms \nameref{subsec:Readability} and \nameref{subsec:Reliability}.
      \end{itemize}
\begin{minted}[frame=lines,linenos]{c++}
void func(int &first, int &second);
func(total, total); // When using first or second, they both point to the same ``total'' variable
fun(list[i], list[j]); // Assuming i==j is true, list[i] and list[j] both point to the same value
\end{minted}
    \item Program verification is more difficult.
    \end{itemize}
  \end{itemize}
\end{definition}

\paragraph{Pass-by-Name}\label{par:Parameter_Passing-Pass_By_Name}
\begin{definition}[Pass-by-Name]\label{def:Pass_By_Name}
  \emph{Pass-by-name} is an \nameref{def:Parameter_Passing-Inout_Mode} parameter transmission method that does not correspond to a single implementation model.
  When \nameref{def:Actual_Parameter} are passed by name, every occurrence of the actual parameter is textually substituted for the corresponding \nameref{def:Formal_Parameter}.
  A pass-by-name \nameref{def:Formal_Parameter} is bound to an access method at the time of the program call, but the actual binding to a \nameref{def:Variable_Value} or an \nameref{def:Variable_Address} is delayed until the \nameref{def:Formal_Parameter} is assigned or referenced.

  Implementing this requires a subprogram be passed to the called subprogram to evaluate the \nameref{def:Variable_Address} or \nameref{def:Variable_Value} of the \nameref{def:Formal_Parameter}.

  The advantages and disadvantages of this are:
  \begin{itemize}[noitemsep]
  \item Advantages
    \begin{itemize}[noitemsep]
    \item \nameref{def:Formal_Parameter}s are lazily evaluated, so they will only be evaluated once they are needed.
    \item However, the \nameref{def:Formal_Parameter} must be evaluated \textbf{EACH TIME}, making this a very inefficient method.
    \item Can add \nameref{subsubsec:Expressivity}.
    \end{itemize}
  \item Disadvantages
    \begin{itemize}[noitemsep]
    \item Pass-by-name parameters are difficult to implement
    \item Pass-by-name parameters are inefficient
    \item Add significant complexity to the program
      \begin{itemize}[noitemsep]
      \item Reduced \nameref{subsec:Readability}
      \item Reduced \nameref{subsec:Reliability}
      \end{itemize}
    \end{itemize}
  \end{itemize}

  \begin{remark}[Languages using \nameref*{def:Pass_By_Name}]\label{rmk:Langauges_Using_Pass_By_Name}
    There are no widely-used high-level languages that use \nameref{def:Pass_By_Name}.
    However, it is used at compile time by macros is assembly languages and for generic parameters of generic subprograms.
  \end{remark}
\end{definition}

\paragraph{Pass-by-Need}\label{par:Parameter_Passing-Pass_By_Need}
\begin{definition}[Pass-by-Need]\label{def:Pass_By_Need}
  This is a fairly niche way to pass \nameref{def:Actual_Parameter}s around to a subprogram's \nameref{def:Formal_Parameter}s.
  It is only used by the Haskell language.

  This is \textbf{VERY} similar to the \nameref{def:Pass_By_Name} implementation, but instead of evaluating the \nameref{def:Actual_Parameter} each time the \nameref{def:Formal_Parameter} appears, the \nameref{def:Actual_Parameter} is evaluated \emph{AT MOST ONCE}.
  Then, every subsequent occurrence of the \nameref{def:Actual_Parameter} has it value substituted by \nameref{def:Referential_Transparency}, or if its a function call, the result of the \nameref{def:Function_Pure} function.

  The advantages and disadvantages of pass-by-need are:
  \begin{itemize}[noitemsep]
  \item Advantages
    \begin{itemize}[noitemsep]
    \item More time-efficient than \nameref{def:Pass_By_Name}
    \end{itemize}
  \item Disadvantages
    \begin{itemize}[noitemsep]
    \item Less flexity in the presence of updating \nameref{def:Variable}s.
      \begin{itemize}[noitemsep]
      \item Haskell avoids this problem because its ``variables'' are immutable, which makes this a moot point.
      \end{itemize}
    \end{itemize}
  \end{itemize}
\end{definition}

\subsection{Parameters That Are Subprograms}\label{subsec:Parameters_are_Subprograms}
Being able to use \nameref{sec:Subprograms} as \nameref{def:Actual_Parameter}s in a program can help simplify the programming of certain problems.
There are 2 major complications of this desire:
\begin{enumerate}[noitemsep]
\item The \nameref{def:Type_Checking} of the parameters of the activations of the subprogram that was passed as a parameter.
\item In languages that allow \nameref{subsubsec:Nested_Subprograms}, there is a question of what \nameref{def:Referencing_Environment} should be used. There are 3 possible solutions:
  \begin{enumerate}[noitemsep]
  \item \emph{\nameref{def:Shallow_Binding}}
  \item \emph{\nameref{def:Deep_Binding}}
  \item \emph{\nameref{def:Ad_Hoc_Binding}}
  \end{enumerate}
\end{enumerate}

\begin{definition}[Shallow Binding]\label{def:Shallow_Binding}
  In \emph{shallow binding}, the \nameref{def:Referencing_Environment} of the call statement that enacts the subprogram is used.

  The \nameref{def:Referencing_Environment} where the call statement to the passed subprogram occurs is \nameref{def:Referencing_Environment} when the subprogram is executed.

  \begin{remark}[Shallow Binding Program Output]\label{rmk:Shallow_Binding-Program_Output}
    In the program below, the output of \texttt{sub2} is \texttt{4}, because shallow binding binds the \nameref{def:Referencing_Environment} of \texttt{sub2} to \texttt{sub4}'s \nameref{def:Referencing_Environment}, where \texttt{sub2} was executed.
  \end{remark}
\end{definition}

\begin{definition}[Deep Binding]\label{def:Deep_Binding}
  In \emph{deep binding}, the \nameref{def:Referencing_Environment} of the definition of the passed subprogram is used.

  The \nameref{def:Referencing_Environment} of the definition of the subprogram is used as the \nameref{def:Referencing_Environment}.

  \begin{remark}[Deep Binding Program Output]\label{rmk:Deep_Binding-Program_Output}
    In the program below, the output of \texttt{sub2} is \texttt{1}, because deep binding binds the \nameref{def:Referencing_Environment} of \texttt{sub2} to \texttt{sub1}'s \nameref{def:Referencing_Environment}, where \texttt{sub2} was defined.
  \end{remark}
\end{definition}

\begin{definition}[Ad-Hoc Binding]\label{def:Ad_Hoc_Binding}
  In \emph{ad-hod binding}, the \nameref{def:Referencing_Environment} of the call statement that \emph{passed} the program as an \nameref{def:Actual_Parameter} is used.

  The \nameref{def:Referencing_Environment} where the call statement passes the subprogram is used as the \nameref{def:Referencing_Environment} for the subprogram.

  \begin{remark}[Ad-Hoc Binding Program Output]\label{rmk:Ad_Hoc_Binding-Program_Output}
    In the program below, the output of \texttt{sub2} is \texttt{3}, because ad-hoc binding binds the \nameref{def:Referencing_Environment} of \texttt{sub2} to \texttt{sub3}'s \nameref{def:Referencing_Environment}, where \texttt{sub2} was originally passed as a subprogram.
  \end{remark}
\end{definition}

This code block is written with JavaScript's syntax, but depending on the binding used, the output of \texttt{sub2} will be different.
\begin{minted}[frame=lines,linenos]{javascript}
function sub1() {
  var x;
  function sub2() {
    alert(x); // Creates a box with the value of x
  };
  function sub3() {
    var x;
    x = 3;
    sub4(sub2);
  };
  function sub4(subx) {
    var x;
    x = 4;
    subx();
  };
  x = 1;
  sub3();
};
\end{minted}

\subsection{Closures}\label{subsec:Closures}
\begin{definition}[Closure]\label{def:Closure}
  A \emph{closure} is a subprogram and the \nameref{def:Referencing_Environment} where it was defined.
  The \nameref{def:Referencing_Environment} is needed if the subprogram can be called from any arbitrary place in the program.

  \begin{remark}[\nameref*{subsubsec:Static_Scope}d, No Nested Subprograms]
    If a statically-scoped programming language does not support nested subprograms, then it usually doesn't support \nameref{def:Closure}s either.
    All of the \nameref{def:Variable}s in the \nameref{def:Referencing_Environment} of a subprogram are accessible, regardless of the place in the program where the subprogram is called.
  \end{remark}
\end{definition}

When subprograms can be nested, the subprogram can use its \nameref{def:Local_Variable}s, the \nameref{def:Global_Variable}s, and any \nameref{def:Variable}s declared in parent subprograms.
This isn't an issue when the nested subprogram can only be called where the enclosing scopes are active and \nameref{def:Visible_Variable}.

However, if the nested subprogram can be called from elsewhere.
This can happen if the subprogram is passed as a parameter or assigned to a variable.
This also means that the subprogram could be called after its parent subprograms have terminated, meaning some of the \nameref{def:Variable}s defined there are no longer available.
To prevent this, we have to have special \nameref{def:Variable}s which are said to have \emph{\nameref{def:Unlimited_Extent}}.

\begin{definition}[Unlimited Extent]\label{def:Unlimited_Extent}
  A \nameref{def:Variable} whose lifetime is that of the whole program, and are required by nested subprograms said to have \emph{unlimited extent}.

  These variables are usually heap-dynamic, rather than \nameref{def:Stack-Dynamic_Variable_Binding_Lifetime}s.
\end{definition}

The code snippet below is an example of a \nameref{def:Closure} and its \nameref{def:Variable}s having \nameref{def:Unlimited_Extent}.
\begin{minted}[frame=lines,linenos]{csharp}
static Func<int, int> makeAdder(int x) {
  return delegate(int y) { return x + y;};
}
...
Func<int, int> Add10 = makeAdder(10); // Makes a function that adds 10 to a passed integer parameter
Func<int, int> Add5 = makeAdder(5); // Makes a function that adds 5 to a passed integer parameter
Console.WriteLine("Add 10 to 20: {0}", Add10(20)); // Prints "Add 10 to 20: 30"
Console.WriteLine("{0}", Add5(20)); // Prints 25
\end{minted}

%%% Local Variables:
%%% mode: latex
%%% TeX-master: "../EDAP05-Concepts_Programming_Languages-Reference_Sheet"
%%% End:
