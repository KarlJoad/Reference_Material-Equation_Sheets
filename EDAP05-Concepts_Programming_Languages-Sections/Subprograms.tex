\section{Subprograms}\label{sec:Subprograms}
This is a way to perform \nameref{par:Process_Abstraction}.
This generally improves \nameref{subsec:Readability} and \nameref{subsec:Reliability}.

\subsection{General Characteristics}\label{subsec:Suprogram_Characteristics}
\begin{itemize}[noitemsep]
\item Each subprogram has a single entry point
\item The calling program unit is suspended during the execution of the called subprogram, which implies there is only one subprogram in execution at any given time
\item Control always returns to the caller whenthe subprogram terminates
\end{itemize}

Alternatives to these generalizations result in coroutines and concurrent units.

\subsection{Basic Definitions}\label{subsec:Subprogram_Definitions}
\begin{definition}[Subprogram Definition]\label{def:Subprogram_Definition}
  A \emph{subprogram definition} describes the interface to and the actions of the subprogram abstraction
\end{definition}

\begin{definition}[Subprogram Call]\label{def:Subprogram_Call}
  A \emph{subprogram call} is the explicit request that a specific subprogram be executed.

  \begin{remark}[Call]\label{rmk:Subprogram_Call}
    This is generally shortened to just a \emph{call}.
  \end{remark}
\end{definition}

\begin{definition}[Active]\label{def:Subprogram_Active}
  A subprogram is said to be \emph{active} if it has been called, but not yet completed its execution.
\end{definition}

\begin{definition}[Subprogram Header]\label{def:Subprogram_Header}
  A \emph{subprogram header} is the first part of a \nameref{def:Subprogram_Definition}.
  This serves several purposes:
  \begin{enumerate}[noitemsep]
  \item Specifies that the following syntactic unit is a \nameref{def:Subprogram_Definition} of some kind.
  \item Provides a name for the subprogram, if it's not an anonymous subprogram.
  \item Optionally specify a list of parameters.
  \end{enumerate}
\end{definition}

\begin{definition}[Subprogram Body]\label{def:Subprogram_Body}
  The \emph{subprogram body} defines the actions that the subprogram takes when there is a \nameref{def:Subprogram_Call}.
  The body may be delimited with curly-braces, \texttt{\{} and \texttt{\}}.
  It may be whitespace delimited, Python.
  It may also have an \texttt{end} statement that ends the execution of that block.
\end{definition}

Python is unique in that its \nameref{def:Subprogram_Definition}s can be executed in control-statement blocks.
For example,
\begin{minted}[frame=lines,linenos]{python3}
if conditional_expression :
    def func(...):
        ...
else:
    def func(...):
        ...
\end{minted}
This means that there are 2 possible \nameref{def:Subprogram_Definition}s possible during runtime, and which one is currently valid depends on the result of the \texttt{conditional\textunderscore{}expression}.

\begin{definition}[Parameter Profile]\label{def:Subprogram_Parameter_Profile}
  The \emph{parameter profile} of a subprogram contains the number, order, and types of its \nameref{def:Formal_Parameter}s.
\end{definition}

\begin{definition}[Protocol]\label{def:Subprogram_Protocol}
  The \emph{protocol} of a subprogram is its \nameref{def:Subprogram_Parameter_Profile}, and if its a function, its return type.
\end{definition}

\begin{definition}[Subprogram Declaration]\label{def:Subprogram_Declaration}
  A \emph{subprogram declaration} is the act of providing type and name information, but not giving any \nameref{def:Subprogram_Body}.
  This is needed in languages that do not allow forward references to subprograms.

  \begin{remark}[Prototype]\label{rmk:Subprogram_Prototype}
    In C/C++, if a subprogram needs to be declared, it is called a \emph{prototype}.
    These are generally specified in \emph{header} files, with a file extension of \texttt{.h}.
  \end{remark}
\end{definition}

\subsection{Parameters}\label{subsec:Subprogram_Parameters}
Subprograms typically want access to nonlocal data to perform their computations.
There are 2 ways to gain access to this nonlocal data:
\begin{enumerate}[noitemsep]
\item Direct access to nonlocal \nameref{def:Variable}s (\nameref{def:Global_Variable}s)
\item Parameter passing
\end{enumerate}

Data that is passed to the subprogram as a parameter is acccessed through names local to the subprogram.
Parameter passing is more flexible, because if direct access is used, new storage needs to be allocated for computation results.
Direct access also leads to issues with \nameref{def:Variable}s being visible to places they shouldn't be.
Pure \nameref{def:Functional_Programming_Language}s avoid this by having all their data being immutable.

\begin{definition}[Formal Parameter]\label{def:Formal_Parameter}
  A \emph{formal parameter} are the parameters present in the \nameref{def:Subprogram_Header}.
  These are sometimes thought of as ``dummy variables'' because they aren't normal \nameref{def:Variable}s.
  They are only bound to storage when the subprogram is called, and that storage is often through some other program \nameref{def:Variable}s.

  \begin{remark}[Parameter]\label{rmk:Parameter}
    Sometimes \nameref{def:Formal_Parameter}s are just called \emph{parameter}s, usually when \nameref{def:Actual_Parameter}s are called \nameref{rmk:Argument}s.
  \end{remark}
\end{definition}

\begin{definition}[Actual Parameter]\label{def:Actual_Parameter}
  An \emph{actual parameter} is the parameter that is bound to the \nameref{def:Formal_Parameter} of a subprogram.
  
  \begin{remark}[Argument]\label{rmk:Argument}
    Sometimes \nameref{def:Actual_Parameter}s are called \emph{argument}s, usually when \nameref{def:Formal_Parameter}s are called \nameref{rmk:Parameter}s.
  \end{remark}
\end{definition}

The binding of \nameref{def:Actual_Parameter}s to \nameref{def:Formal_Parameter}s is usually done by position.
So, the first \nameref{def:Actual_Parameter} is bound to the first \nameref{def:Formal_Parameter}.
However, this is only a good method when the number of parameters is small.

When the \nameref{def:Formal_Parameter} list gets long, it is hard to get all of the \nameref{def:Actual_Parameter}s in the right order.
One solution is to use \nameref{def:Keyword_Parameter}s.

\begin{definition}[Keyword Parameter]\label{def:Keyword_Parameter}
  \emph{Keyword parameter}s have the name of the \nameref{def:Formal_Parameter} usable by the \nameref{def:Actual_Parameter} to bind the \nameref{def:Variable_Value}.

  For example, if \texttt{sumer} has the \nameref{def:Formal_Parameter}s \texttt{length}, \texttt{list}, and \texttt{sum}:
\begin{minted}[frame=lines,linenos]{python3}
sumer(length = my_length, list = my_array, sum = my_sum)
\end{minted}

  Advantages and Disadvantages of keyword parameters:
  \begin{itemize}[noitemsep]
  \item Advantages
    \begin{itemize}[noitemsep]
    \item No need to remember \nameref{def:Formal_Parameter} order.
    \end{itemize}
  \item Disadvantages
    \begin{itemize}[noitemsep]
    \item Need to remember the name of the \nameref{def:Formal_Parameter}s.
    \end{itemize}
  \end{itemize}

  \begin{remark}[End of \nameref{def:Actual_Parameter}]\label{rmk:Keyword_Parameters_at_End}
    In an \nameref{def:Actual_Parameter} list, all parameters after a \nameref{def:Keyword_Parameter} \textbf{must} be keyworded, because the list may not be well-formed enough for parameters to line up by position.
  \end{remark}
\end{definition}

Languages that support default values on \nameref{def:Formal_Parameter}s handle them differently.
In Python, regular \nameref{def:Formal_Parameter}s and ones with default values can be in any order.
However, in C++, which does not support \nameref{def:Keyword_Parameter}s, \nameref{def:Formal_Parameter}s with default values must be at the end of the \nameref{def:Subprogram_Header}.
This is illustrated in the next 2 code blocks, which are in Python and C++, respectively.
\begin{minted}[frame=lines,linenos]{python3}
def compute_pay(income, exemptions = 1, tax_rate)
\end{minted}

\begin{minted}[frame=lines,linenos]{c++}
float compute_pay(float income, float tax_rate, int exemptions = 1)
\end{minted}

Other languages have more varied and interesting ways to pass \nameref{def:Actual_Parameter}s to subprograms.
Look at the language specification for more details.

\subsection{Local Referencing Environments}\label{subsec:Local_Referencing_Environments}
Issues related to \nameref{def:Variable}s defined within subprograms.

\subsubsection{Local Variables}\label{subsubsec:Local_Variables}
The definition of \nameref{def:Local_Variable}s is given in \Cref{def:Local_Variable}.
These can be either \nameref{def:Static_Variable_Binding_Lifetime} or \nameref{def:Stack-Dynamic_Variable_Binding_Lifetime}.

If \nameref{def:Local_Variable}s are \nameref{def:Stack-Dynamic_Variable_Binding_Lifetime}, they are bound to storage when the subprogram begins and unbound when that execution terminates.
The advantages and disadvantages of \nameref{def:Stack-Dynamic_Variable_Binding_Lifetime} are:
\begin{itemize}[noitemsep]
\item Advantages
  \begin{itemize}[noitemsep]
  \item Allows for recursive subprograms
  \item Inactive subprograms can share \nameref{def:Memory} with the active subprogram
  \end{itemize}
\item Disadvantages
  \begin{itemize}[noitemsep]
  \item The cost of the time requried to allocate, initialize, and deallocate these variables
  \item The indirect \nameref{def:Memory} accesses to the data
  \item When all \nameref{def:Variable}s are \nameref{def:Stack-Dynamic_Variable_Binding_Lifetime}, subprograms cannot be history sensitive.
  \end{itemize}
\end{itemize}

However, the advantages and disadvantages of \nameref{def:Static_Variable_Binding_Lifetime} are:
\begin{itemize}[noitemsep]
\item Advantages
  \begin{itemize}[noitemsep]
  \item No runtime overhead to allocate/deallocate the storage
  \item Direct \nameref{def:Memory} access (Absolute addressing)
  \end{itemize}
\item Disadvantages
  \begin{itemize}[noitemsep]
  \item Inability to support recursion
  \item Cannot share \nameref{def:Memory} with inactive subprograms.
  \end{itemize}
\end{itemize}

Most contemporary programming languages make their \nameref{def:Local_Variable}s \nameref{def:Stack-Dynamic_Variable_Binding_Lifetime} by defautlt.
However, this can usually be overridden with a \texttt{static} keyword.

\subsubsection{Nested Subprograms}\label{subsubsec:Nested_Subprograms}
The idea was to create a hierarchy of logic and \nameref{def:Variable_Scope}s.
The motivation was that if a subprogram is only used within one other subprogram, why not place it there, and hide it from the rest of the program?

\nameref{def:Variable_Static_Scoping} is usually used in languages that allow nested subprograms.
Language support of this featuer depends \textbf{heavily} on the language.
You will have to chek the language specification to find out if the programming language supports them.

\subsection{Parameter-Passing Methods}\label{subsec:Parameter_Passing_Methods}
How are \nameref{def:Actual_Parameter}s passed to the subprograms?

\subsubsection{Semantic Models of Parameter Passing}\label{subsubsec:Semantic_Models_Parameter_Passing}
\nameref{def:Formal_Parameter}s are characterized by 1 of 3 distinct \nameref{def:Semantics} models:
\begin{enumerate}[noitemsep]
\item They can received data from the corresponding \nameref{def:Actual_Parameter}. This is called \textbf{\nameref{def:Parameter_Passing-In_Mode}}.
\item They can transmit data to the \nameref{def:Actual_Parameter}. This is called \textbf{\nameref{def:Parameter_Passing-Out_Mode}}.
\item They can do both 1 and 2. This is called \textbf{\nameref{def:Parameter_Passing-Inout_Mode}}.
\end{enumerate}

\begin{definition}[In Mode]\label{def:Parameter_Passing-In_Mode}
  \nameref{def:Formal_Parameter}s can receive data from the corresponding \nameref{def:Actual_Parameter}.
  This is called \emph{in mode}.

  This is generally used when passing \nameref{def:Actual_Parameter}s to a subprogram.
\end{definition}

\begin{definition}[Out Mode]\label{def:Parameter_Passing-Out_Mode}
  \nameref{def:Formal_Parameter}s can transmit data to the corresponding \nameref{def:Actual_Parameter}s.
  This is called \emph{out mode}.

  This is generally used when returning a value from a subprogram.
\end{definition}

\begin{definition}[In/Out Mode]\label{def:Parameter_Passing-Inout_Mode}
  The \nameref{def:Formal_Parameter}s can both transmit and receive data from the corresponding \nameref{def:Actual_Parameter}s.
  This is called \emph{in/out mode}.

  This is generally used when a \nameref{def:Subprogram_Call} takes in \nameref{def:Actual_Parameter}s and returns a value.
\end{definition}

There are 2 conceptual models of how data transfers take place in parameter transmission:
\begin{enumerate}[noitemsep]
\item An actual value is copied (to the caller, to the callee, or both).
\item Or an access path is transmitted. This is usually a pointer/reference.
\end{enumerate}

\subsubsection{Implementation Models of Parameter Passing}\label{subsubsec:Implementation_Models_Parameter_Passing}
A great variety of implementations for parameter passing have been put together.
Here, we list some of them, and discuss their respective advantages and disadvantages.

\paragraph{Pass-by-Value}\label{par:Parameter_Passing-Pass_By_Value}
\begin{definition}[Pass-by-Value]\label{def:Pass_By_Value}
  When a parameter is \emph{passed-by-value}, the value of the \nameref{def:Actual_Parameter} is used to initialize the corresponding \nameref{def:Formal_Parameter}, which then acts as a \nameref{def:Local_Variable} in the subprogram.
  This implements \nameref{def:Parameter_Passing-In_Mode} \nameref{def:Semantics}.

  Passing a parameter by value is typically done by copying the \nameref{def:Actual_Parameter}'s \nameref{def:Variable_Value} for the \nameref{def:Formal_Parameter}.
  This means we don't have to make the \nameref{def:Memory} cell read-only, because making cells read-only can be difficult.

  The advantages and disadvantages are:
  \begin{itemize}[noitemsep]
  \item Advantages
    \begin{itemize}[noitemsep]
    \item Fast to copy scalars in both linkage cost and access time
    \end{itemize}
  \item Disadvantages
    \begin{itemize}[noitemsep]
    \item Additional \nameref{def:Memory} is required for the \nameref{def:Formal_Parameter}'s new value.
    \item The \nameref{def:Actual_Parameter} must be copied to the storage area for the corresponding \nameref{def:Formal_Parameter}
    \item This copying can be expensive if the \nameref{def:Actual_Parameter} is large, like an array with many elements.
    \end{itemize}
  \end{itemize}
\end{definition}

\paragraph{Pass-by-Result}\label{par:Parameter_Passing-Pass_By_Result}
\paragraph{Pass-by-Value-Result}\label{par:Parameter_Passing-Pass_By_Value_Result}
\paragraph{Pass-by-Reference}\label{par:Parameter_Passing-Pass_By_Reference}
\paragraph{Pass-by-Name}\label{par:Parameter_Passing-Pass_By_Name}
\paragraph{Pass-by-Need}\label{par:Parameter_Passing-Pass_By_Need}

%%% Local Variables:
%%% mode: latex
%%% TeX-master: "../EDAP05-Concepts_Programming_Languages-Reference_Sheet"
%%% End:
