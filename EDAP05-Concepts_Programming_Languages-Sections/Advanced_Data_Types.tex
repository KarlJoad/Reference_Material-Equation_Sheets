\section{Advanced Data Types}\label{sec:Advanced_Data_Types}
\nameref{def:Static_Type_Checking} is favorable because it allows us to catch \nameref{def:Type_Error}s early.
Overall, this contributes to the robustness and \nameref{subsec:Reliability} of a language.
We attempt to assign a \nameref{def:Data_Type} to every value and \nameref{def:Expression}.
This is usually done with static typing rules as automation and explicit use specification.

However, static typing cannot solve all our issues in programming.
For instance, if we wanted to exclude division by zero, we might require that the divisor be non-zero.
However, this limits our ability to do computation.

There are 3 main trade-offs that we need to make with \nameref{def:Data_Type}s:
\begin{enumerate}[noitemsep]
\item \textbf{Precision}: How accurately can we make types capture the behaviour of a value, \nameref{def:Expression}, \nameref{sec:Subprograms}, or other program concept?
  This concept ties into the question of Increasing \nameref{subsec:Reliability}.
\item \textbf{Automation}: How much type-checking can we do while still being certain that the type-checking mechanism will eventually (and, ideally, quickly) finish?
  This concept ties into the question of Reducing Compile-Time \nameref{subsec:Cost} and more generally Reducing Development \nameref{subsec:Cost}.
\item \textbf{User-Friendliness}: At what point does writing and reading types become too unwieldy to be practical for users?
  This concept ties into several of the \nameref{subsec:Readability} and \nameref{subsec:Writability} criteria.
\end{enumerate}

\begin{definition}[Polymorphism]\label{def:Polymorphism}
  \emph{Polymorphism} is the idea that part of a program can have multiple forms, i.e.\ a single program handles multiple cases of input types.

\subsection{Ad-Hoc Polymorphism}\label{subsec:Ad_Hoc_Polymorphism}
\begin{definition}[Ad-Hoc Polymorphism]\label{def:Ad_Hoc_Polymorphism}
  
  There are 3 types of polymorphism that are discussed here:
  \begin{enumerate}[noitemsep]
  \item \nameref{subsec:Parametric_Polymorphism}
  \item \nameref{subsec:Ad_Hoc_Polymorphism}
  \item \nameref{subsec:Subtype_Polymorphism}
  \end{enumerate}
\end{definition}

\subsection{Subtype Polymorphism}\label{subsec:Subtype_Polymorphism}
\begin{definition}[Subtype Polymorphism]\label{def:Subtype_Polymorphism}
  
\begin{definition}[Conservative]
  If a \nameref{def:Type_System} cannot precisely express the possible values a \nameref{def:Data_Type} can take, and forces us to describe the values more generally, this is a \emph{conservative} \nameref{def:Type_System}.
\end{definition}

\subsection{Parametric Polymorphism}\label{subsec:Parametric_Polymorphism}
\begin{definition}[Parametric Polymorphism]\label{def:Parametric_Polymorphism}
  \emph{Parametric polymorphism} can be summarized with the following phrase; ``This value has a \nameref{def:Data_Type}, but you don't need to know what it is''.

  \begin{remark}[Generics]\label{rmk:Generics}
    Outside of the \nameref{def:Functional_Programming_Language} world, \nameref{def:Parametric_Polymorphism} is usually refered to as \emph{generics}.
    Subprograms that make use of these generics (\nameref{def:Type_Parameter}s) are called \emph{generic subprograms}.
  \end{remark}
\end{definition}

An example of this is a function that creates a list of fixed size where all elements are initialized with a programmer-specified initial value.
The code shown below is \emph{invalid} Scala code, but illustrates the concept.
\inputminted[frame=lines,linenos]{scala}{./EDAP05-Concepts_Programming_Languages-Sections/Advanced_Data_Types/Code/No_Polymorphism_Initial_List.scala}

This is even more apparent if we write a couple of identity functions.
These are functions that take their parameter and just return it, and do nothing else.
Again, written in Scala, they are:
\inputminted[frame=lines,linenos]{scala}{./EDAP05-Concepts_Programming_Languages-Sections/Advanced_Data_Types/Code/No_Polymorphism_Identity_Functions.scala}

The way to solve this is with \nameref{def:Type_Parameter}s.
\begin{definition}[Type Parameter]\label{def:Type_Parameter}
  The \emph{type parameter} language mechanism abstracts over types that are already present in the program, so we can use these types without knowing their exact form.
  In many ways, type parameters mirror traditional subprogram \nameref{subsec:Subprogram_Parameters}.

  There are 2 forms of type parameters:
  \begin{enumerate}[noitemsep]
  \item \nameref{def:Formal_Type_Parameter}s
  \item \nameref{def:Actual_Type_Parameter}s
  \end{enumerate}
\end{definition}

Using the definition of a \nameref{def:Type_Parameter} system, we can rewrite the identity functions, in Scala, as:
\inputminted[frame=lines,linenos]{scala}{./EDAP05-Concepts_Programming_Languages-Sections/Advanced_Data_Types/Code/Polymorphism_Identity_Functions.scala}

For those familiar with Java, this \nameref{def:Type_Parameter} system has the syntax
\inputminted[frame=lines,linenos]{scala}{./EDAP05-Concepts_Programming_Languages-Sections/Advanced_Data_Types/Code/Polymorphism_Identity_Functions.java}

\begin{remark*}
  The code examples of \nameref{def:Parametric_Polymorphism} above make it seem like it is only possible to write these \nameref{def:Subprogram_Definition}s with just one \nameref{def:Formal_Type_Parameter}.
  However, just like normal subprogram \nameref{subsec:Subprogram_Parameters}, you can pass as many \nameref{def:Actual_Type_Parameter}s as you would like.

  Take this use of a HashMap in Java as an example.
  \inputminted[frame=lines,linenos]{scala}{./EDAP05-Concepts_Programming_Languages-Sections/Advanced_Data_Types/Code/Polymorphism_HashMap.java}
\end{remark*}

\begin{definition}[Formal Type Parameter]\label{def:Formal_Type_Parameter}
  A \emph{formal type parameter} are the \nameref{def:Type_Parameter}s present in the \nameref{def:Subprogram_Header}.
  These are type variables that can be used freely throughout the body in all the places where the \nameref{def:Data_Type} can vary, but should act the same.

  \begin{remark}[Type Parameter]\label{rmk:Type_Parameter}
    \nameref{def:Formal_Type_Parameter}s are sometimes colloquially referred to as \emph{type parameter}s.
    In this case, they are related to the definition of \nameref{def:Type_Parameter}s (\Cref{def:Type_Parameter}), but they are only they ``placeholder'' types.
  \end{remark}
\end{definition}

\begin{definition}[Actual Type Parameter]\label{def:Actual_Type_Parameter}
  The \emph{actual type parameter} is the actual \nameref{def:Data_Type} provided to a function that uses/requires a \nameref{def:Formal_Type_Parameter}.
  
  \begin{remark}[Type Variable]\label{rmk:Type_Variable}
    \nameref{def:Actual_Type_Parameter}s are sometimes colloquially referred to as \emph{type variable}s.
    These share a relationship with normal \nameref{def:Variable}s, but they do not necessarily have all the same properties as normal \nameref{def:Variable}s.
    However, these functions can ``vary'' depending on what the programmer specifies.
  \end{remark}
\end{definition}

%%% Local Variables:
%%% mode: latex
%%% TeX-master: "../../EDAP05-Concepts_Programming_Languages-Reference_Sheet"
%%% End:


%%% Local Variables:
%%% mode: latex
%%% TeX-master: "../EDAP05-Concepts_Programming_Languages-Reference_Sheet"
%%% End:
