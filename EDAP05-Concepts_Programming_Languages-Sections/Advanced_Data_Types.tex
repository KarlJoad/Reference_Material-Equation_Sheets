\section{Advanced Data Types}\label{sec:Advanced_Data_Types}
\nameref{def:Static_Type_Checking} is favorable because it allows us to catch \nameref{def:Type_Error}s early.
Overall, this contributes to the robustness and \nameref{subsec:Reliability} of a language.
We attempt to assign a \nameref{def:Data_Type} to every value and \nameref{def:Expression}.
This is usually done with static typing rules as automation and explicit use specification.

However, static typing cannot solve all our issues in programming.
For instance, if we wanted to exclude division by zero, we might require that the divisor be non-zero.
However, this limits our ability to do computation.

There are 3 main trade-offs that we need to make with \nameref{def:Data_Type}s:
\begin{enumerate}[noitemsep]
\item \textbf{Precision}: How accurately can we make types capture the behaviour of a value, \nameref{def:Expression}, \nameref{sec:Subprograms}, or other program concept?
  This concept ties into the question of Increasing \nameref{subsec:Reliability}.
\item \textbf{Automation}: How much type-checking can we do while still being certain that the type-checking mechanism will eventually (and, ideally, quickly) finish?
  This concept ties into the question of Reducing Compile-Time \nameref{subsec:Cost} and more generally Reducing Development \nameref{subsec:Cost}.
\item \textbf{User-Friendliness}: At what point does writing and reading types become too unwieldy to be practical for users?
  This concept ties into several of the \nameref{subsec:Readability} and \nameref{subsec:Writability} criteria.
\end{enumerate}

\begin{definition}[Polymorphism]\label{def:Polymorphism}
  \emph{Polymorphism} is the idea that part of a program can have multiple forms, i.e.\ a single program handles multiple cases of input types.

\subsection{Ad-Hoc Polymorphism}\label{subsec:Ad_Hoc_Polymorphism}
\begin{definition}[Ad-Hoc Polymorphism]\label{def:Ad_Hoc_Polymorphism}
  
  There are 3 types of polymorphism that are discussed here:
  \begin{enumerate}[noitemsep]
  \item \nameref{subsec:Parametric_Polymorphism}
  \item \nameref{subsec:Ad_Hoc_Polymorphism}
  \item \nameref{subsec:Subtype_Polymorphism}
  \end{enumerate}
\end{definition}

\subsection{Subtype Polymorphism}\label{subsec:Subtype_Polymorphism}
\begin{definition}[Subtype Polymorphism]\label{def:Subtype_Polymorphism}
  
\begin{definition}[Conservative]
  If a \nameref{def:Type_System} cannot precisely express the possible values a \nameref{def:Data_Type} can take, and forces us to describe the values more generally, this is a \emph{conservative} \nameref{def:Type_System}.
\end{definition}

\subsection{Parametric Polymorphism}\label{subsec:Parametric_Polymorphism}
\begin{definition}[Parametric Polymorphism]\label{def:Parametric_Polymorphism}
  \emph{Parametric polymorphism} can be summarized with the following phrase; ``This value has a \nameref{def:Data_Type}, but you don't need to know what it is''.

  \begin{remark}[Generics]\label{rmk:Generics}
    Outside of the \nameref{def:Functional_Programming_Language} world, \nameref{def:Parametric_Polymorphism} is usually refered to as \emph{generics}.
    Subprograms that make use of these generics are called \emph{generic subprograms}.
  \end{remark}
\end{definition}

%%% Local Variables:
%%% mode: latex
%%% TeX-master: "../../EDAP05-Concepts_Programming_Languages-Reference_Sheet"
%%% End:


%%% Local Variables:
%%% mode: latex
%%% TeX-master: "../EDAP05-Concepts_Programming_Languages-Reference_Sheet"
%%% End:
