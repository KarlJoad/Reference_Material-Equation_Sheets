\section{Abstract Data Types and Encapsulation Constructs}\label{sec:Abstract_Data_Types_Encapsulation_Constructs}
\subsection{The Concept of Abstraction}\label{subsec:Concept_Abstraction}
\begin{definition}[Abstraction]\label{def:Abstraction}
  An \emph{abstraction} is a view or representation of an entity that includes only the most significant attributes.
  In a general sense, abstraction allows one to collect instances of entities into groups in which their common attributes need not be considered, and their unique attributes separate entities which may be from the same group.

  There is:
  \begin{itemize}[noitemsep]
  \item \nameref{par:Process_Abstraction}
  \item \nameref{par:Data_Abstraction}
  \end{itemize}
\end{definition}

\subsection{Introduction to Data Abstraction}\label{subsec:Intro_Data_Abstraction}
An \emph{\nameref{def:Abstract_Data_Type}} is a data structure, in the form of a record, but includes subprograms that manipulate its data.
It is an enclosure that only includes the data representations of one specific \nameref{def:Data_Type}, and the subprograms provide operations for that type.
This allows unnecessary details of the type to be hidden from units outside the enclosure.

\begin{definition}[Object]\label{def:Object}
  An instance of an \nameref{def:Abstract_Data_Type} is called an \emph{object}.
\end{definition}

\nameref{def:Abstract_Data_Type}s are used to combat program complexity by grouping things together similarly to how we would group them in the real-world as humans.

\subsubsection{Floating-Point as an Abstract Data Type}\label{subsubsec:Floating_Point_Abstract_Data_Type}
\subsubsection{User-Defined Abstract Data Types}\label{subsubsec:User_Defined_Abstract_Data_Types}

\subsection{Design Issues for Abstract Data Types}\label{subsec:Abstract_Data_Type_Design_Issues}

\subsection{Language Examples}\label{subsec:Abstract_Data_Type_Lang_Examples}
\subsubsection{Abstract Data Types in Ada}\label{subsubsec:Abstract_Data_Types_Ada}
\paragraph{Encapsulation}\label{par:Ada_Encapsulation}
\paragraph{Information Hiding}\label{par:Ada_Info_Hiding}
\paragraph{Example}\label{par:Ada_Abstract_Data_Type_Example}
\paragraph{Evaluation}\label{par:Ada_Abstract_Data_Type_Evaluation}

\subsubsection{Abstract Data Types in C++}\label{subsubsec:Abstract_Data_Types_C++}
\paragraph{Encapsulation}\label{par:C++_Encapsulation}
\paragraph{Information Hiding}\label{par:C++_Info_Hiding}
\paragraph{Constructors and Destructors}\label{par:C++_Constructors_Destructors}
\paragraph{Example}\label{par:C++_Abstract_Data_Type_Example}
\paragraph{Evaluation}\label{par:C++_Abstract_Data_Type_Evaluation}

\subsubsection{Abstract Data Types in Objective-C}\label{subsubsec:Abstract_Data_Types_Objective_C}
\paragraph{Encapsulation}\label{par:Objective_C_Encapsulation}
\paragraph{Information Hiding}\label{par:Objective_C_Info_Hiding}
\paragraph{Example}\label{par:Objective_C_Abstract_Data_Type_Example}
\paragraph{Evaluation}\label{par:Objective_C_Abstract_Data_Type_Evaluation}

\subsubsection{Abstract Data Types in Java}\label{subsubsec:Abstract_Data_Types_Java}
\paragraph{Example}\label{par:Java_Abstract_Data_Type_Example}
\paragraph{Evaluation}\label{par:Java_Abstract_Data_Type_Evaluation}

\subsubsection{Abstract Data Types in C\#}\label{subsubsec:Abstract_Data_Types_CSharp}
\paragraph{Encapsulation}\label{par:Objective_C_Encapsulation}
\paragraph{Information Hiding}\label{par:Objective_C_Info_Hiding}

\subsubsection{Abstract Data Types in Ruby}\label{subsubsec:Abstract_Data_Types_Ruby}
\paragraph{Encapsulation}\label{par:Ruby_Encapsulation}
\paragraph{Information Hiding}\label{par:Ruby_Info_Hiding}
\paragraph{Example}\label{par:Ruby_Abstract_Data_Type_Example}
\paragraph{Evaluation}\label{par:Ruby_Abstract_Data_Type_Evaluation}
%%% Local Variables:
%%% mode: latex
%%% TeX-master: "../EDAP05-Concepts_Programming_Languages-Reference_Sheet"
%%% End:
