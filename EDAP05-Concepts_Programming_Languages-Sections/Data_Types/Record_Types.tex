\subsection{Record Types}\label{subsec:Record_Data_Types}
\begin{definition}[Record]\label{def:Record_Data_Type}
  A \emph{record} is an aggregate of data elements in which the individual elements are identified by names and accessed through offsets from the beginning of the structure, similar to arrays.
  The offset from the head of the record to reach any \nameref{def:Record_Data_Type_Field} is known statically, because the sizes and \nameref{def:Variable_Type}s are known at compile time.

  Records are used to model a collection of data in which the individual elements, the \nameref{def:Record_Data_Type_Field}s are not of the same \nameref{def:Variable_Type} or size.

  \begin{remark}[Clarification]
    An important clarification here is that the \nameref{def:Record_Data_Type}, defined in \Cref{def:Record_Data_Type} is \textbf{NOT} a record in a database \textbf{in any way}.
  \end{remark}

  \begin{remark}
    A \nameref{def:Record_Data_Type} is similar to a heterogeneous array, but they differ in one key way.
    A heterogeneous array is an array of pointers to areas of \nameref{def:Memory} that may be discontinuous.
    However, all the \nameref{def:Record_Data_Type_Field}s in a \nameref{def:Record_Data_Type} all reside in adjacent \nameref{def:Memory} locations.
  \end{remark}

  \begin{remark}[\nameref*{def:Record_Data_Type} vs. Object]
    A \nameref{def:Record_Data_Type} and an object are quite similar.
    However, the differences between them depend on the language.
  \end{remark}
\end{definition}

\begin{definition}[Field]\label{def:Record_Data_Type_Field}
  A \emph{field} is an element a \nameref{def:Record_Data_Type}.
  These are fixed length, with fixed \nameref{def:Variable_Type}, meaning the \nameref{def:Memory} address can be statically calculated to reach any \nameref{def:Record_Data_Type_Field} in the \nameref{def:Record_Data_Type}.

  Fields are referenced by their identifier, rather than an index.
\end{definition}

\subsubsection{Definitions of \nameref*{def:Record_Data_Type}s}\label{subsubsec:Definitions_of_Records}
There are 2 design questions that need to be asked when defining \nameref{def:Record_Data_Type}s.
\begin{enumerate}[noitemsep]
\item What is the syntactic form of references to \nameref{def:Record_Data_Type_Field}s?
\item Are elliptical references allowed?
\end{enumerate}

Below are 2 blocks of code, the first from COBOL, the second from Ada.
Both describe an employee.

\inputminted[frame=lines,linenos]{cobol}{./EDAP05-Concepts_Programming_Languages-Sections/Data_Types/Code/Employee_Record.cbl}
The numbers \texttt{01}, \texttt{02}, and \texttt{05} are \emph{level numbers}, which indicate relative hierarchical values.
Any line that is followed by a line with a higher-level number is itself a record.
\texttt{PICTURE} clauses show the formats of the storage locations.
\texttt{X(20)} is a 20 character alphanumeric string and \texttt{99V99} is a 4 decimal digit number with the dcimal in the middle.

However, Ada does not have the level numbers like COBOL, so they allow for nesting \nameref{def:Record_Data_Type} structures inside \nameref{def:Record_Data_Type} declarations.
\inputminted[frame=lines,linenos]{cobol}{./EDAP05-Concepts_Programming_Languages-Sections/Data_Types/Code/Employee_Record.ada}

In Java and C\#, \nameref{def:Record_Data_Type}s can be defined as data classes, whith nested \nameref{def:Record_Data_Type}s defined as nested classes.
Lua's tables serve this purpose.

\subsubsection{References to \nameref*{def:Record_Data_Type} \nameref*{def:Record_Data_Type_Field}s}\label{subsubsec:References_to_Record_Fields}
There are many ways to refer to individual \nameref{def:Record_Data_Type_Field}s.
We will look at the way COBOL referenced \nameref{def:Record_Data_Type_Field}s, and the \nameref{def:Record_Data_Type_Field_Access-Dot_Notation}.

\begin{definition}[Dot Notation]\label{def:Record_Data_Type_Field_Access-Dot_Notation}
  Most programming languages use \emph{dot notation} for \nameref{def:Record_Data_Type_Field} references, where components necessary to reach the \nameref{def:Record_Data_Type_Field} are connected with periods.
  The outermost \nameref{def:Record_Data_Type} goes on the left, and gets more specific as it grows to the right.
\end{definition}

There are 2 examples below of accessing the middle name of the Employee \nameref{def:Record_Data_Type}s we made in the previous section.

\begin{minted}[frame=lines,linenos]{cobol}
MIDDLE OF EMPLOYEE-NAME OF EMLOYEE-RECORD
\end{minted}

\begin{minted}[frame=lines,linenos]{ada}
Employee_Record.Employee_Name.Middle;
\end{minted}

There are 2 ways to make a reference to a \nameref{def:Record_Data_Type} \nameref{def:Record_Data_Type_Field}.
\begin{enumerate}[noitemsep]
\item \nameref{def:Fully_Qualified_Reference}s
\item \nameref{def:Elliptical_Reference}s
\end{enumerate}

\begin{definition}[Fully Qualified Reference]\label{def:Fully_Qualified_Reference}
  A \emph{fully qualified reference} to a record field is one in which to access a \nameref{def:Record_Data_Type_Field}, the programmer \textbf{MUST} specify all intermediate \nameref{def:Record_Data_Type}s to go through.

  An alternative to fully qualified reference is the \nameref{def:Elliptical_Reference}.
\end{definition}

\begin{definition}[Elliptical Reference]\label{def:Elliptical_Reference}
  An \emph{elliptical reference} allows a programmer to specify the \nameref{def:Record_Data_Type_Field} and omit any to all parent \nameref{def:Record_Data_Type}s; so long as the resulting reference in unambiguous in the referencing environment (the \nameref{def:Variable_Scope}).

  These are a programmer convenience, but are \textbf{incredibly} difficult to compile, because of the alaborate data structures and procedures required to correctly identify the referenced field.
  There is also a slight loss of \nameref{subsec:Readability}.
\end{definition}

\subsubsection{Evaluation of \nameref*{subsec:Record_Data_Types}}\label{subsubsec:Evaluation_of_Record_Types}
\nameref{def:Record_Data_Type}s are valuable to programming and programming languages.
Their design is straightforward and use is safe.

\nameref{def:Record_Data_Type}s and \nameref{def:Array}s are quite similar, but differ in some key ways
\begin{itemize}[noitemsep]
\item \nameref{def:Array}s:
  \begin{itemize}[noitemsep]
  \item All data values have the same \nameref{def:Variable_Type}. This allows for easy subscripting of \nameref{def:Memory} addresses
  \end{itemize}
\item \nameref{def:Record_Data_Type}
  \begin{itemize}[noitemsep]
  \item When the collection of data values is heterogeneous
  \item The different data \nameref{def:Record_Data_Type_Field}s are not processes the same way
  \item The \nameref{def:Record_Data_Type_Field}s are not processed in any particular order
  \item \nameref{def:Record_Data_Type_Field}s are like named subscripts
  \item Since \nameref{def:Record_Data_Type_Field} names are static, they provide efficient access to fields.
  \end{itemize}
\end{itemize}

\subsubsection{Implementation of \nameref*{subsec:Record_Data_Types}}\label{subsubsec:Implementation_of_Record_Types}
The \nameref{def:Record_Data_Type_Field}s of \nameref{def:Record_Data_Type}s are stored in adjacent \nameref{def:Memory} locations.
But because each \nameref{def:Record_Data_Type_Field} may be of a different size, the offset address, relative to the beginning of the \nameref{def:Record_Data_Type} is associated with each field.
These are calculated at compile-time.
This way, there are no runtime calculations that need to be done.

%%% Local Variables:
%%% mode: latex
%%% TeX-master: "../../EDAP05-Concepts_Programming_Languages-Reference_Sheet"
%%% End:
