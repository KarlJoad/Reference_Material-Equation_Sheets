\subsection{Character String Types}\label{subsec:Character_String_Types}
\begin{definition}[Character String Type]\label{def:Character_String_Type}
  A \emph{character string type} is one in which the values consist of sequences of characters.
  In most programming languages, these are simply refered to as \texttt{string}s.
\end{definition}

\subsubsection{Design Issues}\label{subsubsec:Character_String_Types_Design_Issues}
There are 2 questions that need to be answered when designing a language implementation when it comes to strings.
\begin{itemize}[noitemsep]
\item Should strings be a special kind of character array or a primitive type?
\item Should strings have static or dynamic lengths?
\end{itemize}

\subsubsection{Strings and Their Operations}\label{subsubsec:String_Types_and_Ops}
The most common string operations are:
\begin{itemize}[noitemsep]
\item Assignment
  \begin{itemize}[noitemsep]
  \item What happens when a string is longer than expected? C/C++'s \texttt{strcpy} function
  \end{itemize}
\item Concatenation
\item Substring Reference
  \begin{itemize}[noitemsep]
  \item Discussed more in the context of arrays, where substring references are called slices.
  \end{itemize}
\item Comparison
  \begin{itemize}[noitemsep]
  \item How do we compare 2 strings, where one is longer than the other?
  \end{itemize}
\item Pattern Matching
\end{itemize}

In C and C++, strings are terminated with the null character, \texttt{00}.
This way we do not need to track the length of a string.

Object-Oriented Languages (Java, Ruby, C\#) use classes to represent strings.
The only field in these objects is a constant string.

Python supports strings as a primitive type, and supports array-like operations on them.

Some languages have \nameref{def:Regular_Expression}s built in, like Perl, JavaScript, Ruby, and PHP.\@
Others have libraries that handle \nameref{def:Regular_Expression}s.

\begin{definition}[Regular Expression]\label{def:Regular_Expression}
    A \emph{regular expression}, sometimes called a \emph{regex} is a way to define a sequence of characters to form strings.
\end{definition}

\subsubsection{String Length Options}\label{subsubsec:String_Type_Length_Options}
\begin{definition}[Static Length String]\label{def:Static_Length_String}
  A \emph{static length string} has its length set at the time of string creation.
  It is static, in that the length cannot be changed later in the program's execution.
\end{definition}

\begin{definition}[Dynamic Length String]\label{def:Dynamic_Length_String}
  A \emph{dynamic length string} has its length set at the time of string creation.
  However, strings can change their length, and there is no set maximum size they can have.
\end{definition}

\begin{definition}[Limited Dynamic Length String]\label{def:Limited_Dynamic_Length_String}
  A \emph{dynamic length string} has its length set at the time of string creation.
  However, the string can be redefined later in the program, so long as the new string is the same length or shorter than when the string \nameref{def:Variable} was defined.
\end{definition}

\subsubsection{Evaluation}\label{subsubsec:String_Type_Evaluation}
Primitive string type implementations would require there to be predefined functions for many string operations.
If there aren't, then programming in that language becomes more cumbersome.

\nameref{def:Dynamic_Length_String}s are the most flexible, but the overhead of their implementation should be weighed against that flexibility.

\subsubsection{Implementation of Character String Types}\label{subsubsec:Implementation_of_Character_String_Types}
Software is used to implement string storage, retrieval, and manipulation.
When a language uses character arrays to store character string types, the language usually supplies few operations.

A \nameref{def:Descriptor} for a \nameref{def:Static_Length_String} has 3 fields:
\begin{enumerate}[noitemsep]
\item Name of the type
\item The type's length in characters
\item Address of the first character
\end{enumerate}

A \nameref{def:Descriptor} for a \nameref{def:Limited_Dynamic_Length_String} has 4 fields:
\begin{enumerate}[noitemsep]
\item Name of the type
\item The type's maximum length in characters
\item The length of the currently stored string
\item The address of the first character
\end{enumerate}

A \nameref{def:Descriptor} for a \nameref{def:Dynamic_Length_String} is more difficult to handle because of its dynamic nature.
There are 3 approaches to storing these:
\begin{enumerate}[noitemsep]
\item Strings stored in a linked list. If the string gets longer, individual nodes can be allocated from anywhere in the \nameref{def:Heap}.
  \begin{itemize}[noitemsep]
  \item A drawback of this is that extra storage of the links
  \item The necessary complexity of string operations
  \end{itemize}
\item Store strings as arrays of pointers to individual characters on the \nameref{def:Heap}
  \begin{itemize}[noitemsep]
  \item This uses more memory, but processing is faster than the linked list approach.
  \end{itemize}
\item Store complete strings in adjacent cells, and when a new longer string comes along, store the whole thing in a new area in the \nameref{def:Heap} and deallocate the old location.
  \begin{itemize}[noitemsep]
  \item Less storage required compared to the linked list approach
  \item Allocation and deallocation of the string is more difficult
  \end{itemize}
\end{enumerate}

%%% Local Variables:
%%% mode: latex
%%% TeX-master: "../../EDAP05-Concepts_Programming_Languages-Reference_Sheet"
%%% End:
