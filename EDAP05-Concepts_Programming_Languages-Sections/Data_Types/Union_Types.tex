\subsection{Union Types}\label{subsec:Union_Types}
\begin{definition}[Union]\label{def:Union_Type}
  A \emph{union} is a \nameref{def:Data_Type} whose \nameref{def:Variable}s may store different \nameref{def:Variable_Type} \nameref{def:Variable_Value}s at different times during program execution.
  \textbf{However, ONLY ONE value is stored at a time.}

  This means that in a C-program, which uses \nameref{def:Free_Union}s will only hold one thing at a time, and \textbf{based on the field that you call in the union, the program will interpret the bit pattern differently}.
  This site gives a good exposition of this point: \href{https://www.tutorialspoint.com/cprogramming/c_unions.htm}{Greater Exposition}.

  The union will allocate the maximum space required by all the \nameref{def:Variable_Type}s of the union, and only use the parts it needs, based on what \nameref{def:Variable_Type}s are actually in use.
\end{definition}

\subsubsection{Design Issues}\label{subsubsec:Union_Types-Design_Issues}
\begin{itemize}[noitemsep]
\item Should \nameref{def:Type_Checking} be required?
  \begin{itemize}[noitemsep]
  \item This \nameref{def:Type_Checking} will have to be dynamic, running during program execution.
  \end{itemize}
\item Should \nameref{def:Union_Type}s be embedded in \nameref{def:Record_Data_Type}?
\end{itemize}

\subsubsection{Discriminated vs. Free Unions}\label{subsubsec:Union_Types-Discriminated_vs_Free}
The \texttt{union} construct in C/C++ is used to specify \nameref{def:Union_Type} structures.
These are called \nameref{def:Free_Union}s.
These stand in stark contrast to \nameref{def:Discriminated_Union}s.
\begin{definition}[Free Union]\label{def:Free_Union}
  A \emph{free union} is a \nameref{def:Union_Type} that have \textbf{\textit{NO}} \nameref{def:Type_Checking} enforced on their use.
  For example, this C snippet:
  \inputminted[frame=lines,linenos]{c}{./EDAP05-Concepts_Programming_Languages-Sections/Data_Types/Code/Free_Union.c}
  The last assignment is not type checked, because the the current \nameref{def:Variable_Type} of \texttt{el1} cannot be checked.
  So, \texttt{27} is assigned to the \texttt{float} variable \texttt{x}, which is nonsense.
\end{definition}

\begin{definition}[Discriminated Union]\label{def:Discriminated_Union}
  A \emph{discriminated union} makes use of a \emph{tag} or \emph{discriminant} as a \nameref{def:Data_Type} indicator.
  These allow for the \nameref{def:Type_Checking} of \nameref{def:Union_Type}s during runtime.
\end{definition}

\subsubsection{Ada \nameref*{subsec:Union_Types}}\label{subsubsec:Ada_Union_Types}
Ada allows the use to specify variables for a variant \nameref{def:Record_Data_Type} that will only store one of the possible \nameref{def:Variable_Type} values in the variant.

\begin{definition}[Constrained Variant Variable]\label{def:Union_Type-Constrained_Variant_Variable}
  A \emph{constrained variant variable} is when a language allows the programmer to specify the types present in a variant \nameref{def:Data_Type}, allowing for static \nameref{def:Type_Checking}.
  These enforce that variants can only be changed by assigning the entire record at a time.

  This is shown in the code snippet below.
  \inputminted[frame=lines,linenos]{ada}{./EDAP05-Concepts_Programming_Languages-Sections/Data_Types/Code/Discriminated_Union.ada}
\end{definition}

\subsubsection{Evaluation}\label{subsubsec:Union_Types-Evaluation}
\nameref{def:Union_Type}s are potentially unsafe constructs in some languages, because they cannot be type checked.
C and C++ are not strongly typed for this reason.
However, Ada, ML, Haskell, and F\# are strongly typed, because they can perform \nameref{def:Type_Checking} on \nameref{def:Union_Type}s.
Some languages, like Java and C\# do not even include the ability to construct a \nameref{def:Union_Type}.

\subsubsection{Implementation of \nameref*{subsec:Union_Types}}\label{subsubsec:Union_Types-Implementation}
\nameref{def:Union_Type}s are implemented by using the same address for a single \nameref{def:Union_Type}, no matter its variant.
Enough storage is allocated for the largest possible variant.
Then, depending on the variant, the space will be used according to how the \nameref{def:Union_Type} was defined.
The tag of the \nameref{def:Union_Type} variant is stored in its \nameref{def:Descriptor}.

%%% Local Variables:
%%% mode: latex
%%% TeX-master: "../../EDAP05-Concepts_Programming_Languages-Reference_Sheet"
%%% End:
