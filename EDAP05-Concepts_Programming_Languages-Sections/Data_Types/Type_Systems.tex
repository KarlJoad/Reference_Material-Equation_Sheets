\subsection{Type Systems}\label{subsec:Type_Systems}
\begin{definition}[Type System]\label{def:Type_System}
  \emph{Type System}s are systems that relate the concepts of \emph{\nameref{def:Data_Type}s}, \emph{\nameref{def:Expression}s}, and \emph{values}.
  They describe rules for ensuring these concepts are correct.

  There are some properties that type systems should have.
  \begin{propertylist}
  \item The type system should allow us to predict the output of computations. Meaning it $e \EvaluatesTo \SemanticValue{v}$, then we can assign types to both $e$ and to the result $\SemanticValue{v}$.
  \end{propertylist}

  \begin{remark}[Expressions]
    In this portion of the document, an \nameref{def:Expression} refers to
    \begin{equation*}
      e \EvaluatesTo \SemanticValue{v}
    \end{equation*}
  \end{remark}
\end{definition}

\begin{definition}[Data Type]\label{def:Data_Type}
  A \emph{data type} defines a subset of data values.
  It also specifies the set of operations possible on those values.
  The data types present in a language used for a particular problem should closely mirror the objects in the real-world the program is solving.

  They can be mathematically defined as
  \begin{equation}\label{eq:Data_Type}
    \SemanticValue{v} : \DataType
  \end{equation}
  where $\SemanticValue{v}$ is a value and $\DataType$ is a type.

  \begin{remark}[Has the Type]\label{rmk:Has_the_Type}
    To say ``$\SemanticValue{v}$ has the type $\DataType$''
    \begin{equation}\label{eq:Has_the_Type}
      \SemanticValue{v} \in \DataType
    \end{equation}
  \end{remark}
\end{definition}

User-defined data types allow for:
\begin{itemize}[noitemsep]
\item Improved readability with better named \nameref{def:Data_Type}s.
\item Improved modifiability with programmers having to change just one common data type somewhere for a large change throughout a program.
\end{itemize}

If we take user-defined \nameref{def:Data_Type}s further, we end up with \emph{abstract data types}.
These force an interface for a particular data type, which is then visible to the user, and the data and background operations are hidden away.

Because of the wide variety of \nameref{def:Data_Type}s present today, it is more useful to think about \nameref{def:Variable}s in terms of \nameref{def:Descriptor}.

\begin{definition}[Descriptor]\label{def:Descriptor}
  A \emph{descriptor} is the collection of attributes of a \nameref{def:Variable}.
  In an implementation, a descriptor is an area of \nameref{def:Memory} that stores the attributes of a \nameref{def:Variable}.

  There are a 2 cases for these:
  \begin{enumerate}[noitemsep]
  \item If all attributes are static, then they are known at compile-time, and the \nameref{def:Compiler} can use the symbol table to construct everything.
  \item If all attributes are dynamic, then the symbol table and all attributes must be stored in \nameref{def:Memory} during program execution.
  \end{enumerate}

  Descriptors are used for \nameref{def:Type_Checking} and building the code for \nameref{def:Variable_Memory_Allocation} and \nameref{def:Variable_Memory_Deallocation} operations.
\end{definition}

\begin{definition}[Type Error]\label{def:Type_Error}
  A \emph{type error} is an attempt to perform an operation that requires an input value of type $\DataType$ with a value $\SemanticValue{v}$ even though $\SemanticValue{v} : \DataType$ does not hold.
\end{definition}

\begin{definition}[Type Preservation]\label{def:Type_Preservation}
  A type system has the \emph{type preservation} (or \emph{subject reduction}) property if for any $e \EvaluatesTo \SemanticValue{v}$, $e : \DataType$ implies $\SemanticValue{v} : \DataType$.

  There are 3 properties we want to have a type preserving type system to have:
  \begin{propertylist}
  \item \emph{Type Preservation}: The predictions of the type system agree with the evaluation rules.\label{prop:Type_Preservation-Preservation}
  \item \emph{Progress}: The type system only assigns a type if the evaluation rules will not get ``stuck'' due to a missing semantic rule. This is not the same as guaranteeing that the program itself terminates, meaning but it does guarantee that the language implementation will never run into a situation in which it doesn't know what to do next.\label{prop:Type_Preservation-Progress}
  \item \emph{Termination}: We want the type system to be \emph{decidable}, that is, we want an automatic mechanism that performs type checking.\label{prop:Type_Preservation-Termination}
  \end{propertylist}
\end{definition}

\subsubsection{Strong and Weak Typing}\label{subsubsec:Strong_Weak_Typing}
\begin{definition}[Strong Type Checking]\label{def:Strong_Type_Checking}
  \emph{Strong type checking} is a property of a programming language if and only if
  \begin{propertylist}
  \item All values have a \nameref{def:Data_Type}
  \item \textbf{Any} \nameref{def:Type_Error}s and prevents the operations that caused the \nameref{def:Type_Error} from taking place, \textbf{BEFORE EXECUTION}
  \end{propertylist}

  A language that has strong type checking is called a \emph{strongly-typed language}.
  Such languages are:
  \begin{itemize}[noitemsep]
  \item Haskell
  \item Python
  \item Ruby
  \item JavaScript (This is somewhat debatable)
  \end{itemize}

  \begin{remark}[Cost]\label{rmk:Strong_Type_Checking_Cost}
    \nameref{def:Strong_Type_Checking} improves the \nameref{subsec:Reliability} of a language, at a \nameref{subsec:Cost} to the language's \nameref{subsubsec:Expressivity}, and sometimes also at a \nameref{subsec:Cost} to the language's execution time.

    Usually this improvement to \nameref{subsec:Reliability} is worth it.
    However, sometimes it is necessary to write directly to memory without regard to the \nameref{def:Data_Type} system.
    These languages provide ``backdoors'' to do this.
  \end{remark}
\end{definition}

\begin{definition}[Weak Type Checking]\label{def:Weak_Type_Checking}
  \emph{Weak type checking} is the absence of \nameref{def:Strong_Type_Checking}.
  Meaning, that an operation that causes a \nameref{def:Type_Error} does \textbf{NOT} always get prevented.
  In some languages, like JavaScript, the operation would return some default value and continue processing.

  A language that has weak type checking is called a \emph{weakly-typed language}.
  Such languages are:
  \begin{itemize}[noitemsep]
  \item C
  \item C++
  \end{itemize}
\end{definition}

\subsubsection{When to Type Check}\label{subsubsec:When_Type_Check}
\begin{definition}[Static Type Checking]\label{def:Static_Type_Checking}
  Any \nameref{def:Type_Checking} that is performed before runtime (at compile time, as part of the static semantics) is called \emph{static type checking}.

  For example, the following code \textbf{WILL NOT EVEN COMPILE} until the type error is fixed.
  \inputminted[frame=lines,linenos]{scala}{./EDAP05-Concepts_Programming_Languages-Sections/Language_Critique/Code/Static_Type_Check.scala}

  Languages that perform static type checking are called \emph{statically typed language}s.
  Such languages are:
  \begin{itemize}[noitemsep]
  \item Haskell
  \item Standard ML
  \item C, C++, C\#, Objective-C, Java, Kotlin, etc.
  \end{itemize}

  \begin{remark}[Undecidability of Static Type Checking]\label{rmk:Static_Type_Checking-Undecidability}
    Not all \nameref{def:Data_Type}s can be checked statically.
    For example, the subscript used in an array access or division by a variable that may be zero must be checked dynamically.

    This deference is actually \textbf{required} for a language to be strongly-typed.
  \end{remark}
\end{definition}

\begin{definition}[Dynamic Type Checking]\label{def:Dynamic_Type_Checking}
  Any \nameref{def:Type_Checking} that is performed at runtime is called \emph{dynamic type checking}.

  For example, the following code \textbf{will} run, but will throw a type error \textbf{DURING RUNTIME}.
  \inputminted[frame=lines,linenos]{javascript}{./EDAP05-Concepts_Programming_Languages-Sections/Language_Critique/Code/Dynamic_Type_Check.js}

  Languages that perform dynamic type checking are called \emph{dynamically typed language}s.
  Such languages are:
  \begin{itemize}[noitemsep]
  \item Python
  \item JavaScript
  \end{itemize}
\end{definition}

\subsubsection{Type Checking}\label{subsubsec:Type_System_Checking}
The \nameref{def:Type_System} only allows an \nameref{def:Expression} \textbf{if and only if} it can assign a \nameref{def:Data_Type} to the \nameref{def:Expression}.

\begin{definition}[Well-Typed]\label{def:Well_Typed_Expression}
  An \nameref{def:Expression} is \emph{well-typed} if and only if we can show $e : \DataType$ for some type $\DataType$.
\end{definition}

We can use rules similar to operational semantics to derive the type system, much like how we derived what computation meant.
For example,
\begin{equation*}
  \dfrac{e_{1} : \SemanticType{Nat} \;\; e_{2} : \SemanticType{Nat}}{e_{1} + e_{2} : \SemanticType{Nat}} \SemanticRuleName{type-add}
\end{equation*}

and now the typing of values needs to be defined, because otherwise the \SemanticRuleName{type-add} doesn't know what types it is taking in, only what they are supposed to be for the addition to be defined.
\begin{equation*}
  \dfrac{\SemanticValue{v} \in \NaturalNumbers}{\SemanticValue{v} : \SemanticType{Nat}}
\end{equation*}

So far, we have seen semantic type rules for operations that take in one type and return the same type.
The same general formulation holds true for operations that take in one type and return another.
\begin{equation*}
  \dfrac{\SemanticInput{e_{1}} : \SemanticType{Nat} \;\; \SemanticInput{e_{2}} : \SemanticType{Nat}}{\SemanticInput{e_{1}} = \SemanticInput{e_{2}} : \SemanticType{Bool}}
\end{equation*}

The above equality operator works for the natural numbers, but it won't work with booleans, so we need to make another semantic type rule about that relationship.
\begin{equation*}
  \dfrac{\SemanticInput{e_{1}} : \SemanticType{Bool} \;\; \SemanticInput{e_{2}} : \SemanticType{Bool}}{\SemanticInput{e_{1}} = \SemanticInput{e_{2}} : \SemanticType{Bool}}
\end{equation*}

\subsubsection{Dynamic Type Checking}\label{subsubsec:Dynamic_Type_Checking}
Whenever a typing rule depends on the evaluation relation $\EvaluatesTo$, we must defer type checking to runtime.
Depending on the language, there might be a lot or a little amount of checking.
Some languages actually defer all type checking to runtime, these are \emph{dynamically typed} languages, like Python and JavaScript.

\subsubsection{Overloading}\label{subsubsec:Overloading}
\nameref{def:Overload_Operator}s are a form of \nameref{def:Ad_Hoc_Polymorphism}.
They rely on the typing rules of a language to determine what type of operation to perform.
For example,
\begin{equation*}
  \dfrac{e_{1} \EvaluatesTo \SemanticValue{v_{1}} \;\; e_{2} \EvaluatesTo \SemanticValue{v_{2}} \;\; \SemanticValue{v_{1}} : \SemanticType{Nat} \;\; \SemanticValue{v_{2}} : \SemanticType{Nat} \;\; \SemanticValue{v} = \SemanticValue{v_{1}} + \SemanticValue{v_{2}}}{e_{1} \SemanticInput{+} e_{2} \EvaluatesTo \SemanticValue{v}}\SemanticRuleName{add}
\end{equation*}
\begin{equation*}
  \dfrac{e_{1} \EvaluatesTo \SemanticValue{v_{1}} \;\; e_{2} \EvaluatesTo \SemanticValue{v_{2}} \;\; \SemanticValue{v_{1}} : \SemanticType{String} \;\; \SemanticValue{v_{2}} : \SemanticType{String} \;\; \SemanticValue{v} = \SemanticValue{v_{1}} ++ \SemanticValue{v_{2}}}{e_{1} \SemanticInput{+} e_{2} \EvaluatesTo \SemanticValue{v}}\SemanticRuleName{concat}
\end{equation*}

%%% Local Variables:
%%% mode: latex
%%% TeX-master: "../../EDAP05-Concepts_Programming_Languages-Reference_Sheet"
%%% End:
