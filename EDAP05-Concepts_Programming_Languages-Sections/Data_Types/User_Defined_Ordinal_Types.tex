\subsection{User-Defined Ordinal Types}\label{subsec:User_Defined_Ordinal_Types}
\begin{definition}[Ordinal Type]\label{def:Ordinal_Type}
  An \emph{ordinal type} is a \nameref{def:Variable_Type} in which the range of possible values can be associated with the set of positive integers.

  For example, in Java, the primitive ordinal types are: \texttt{int}, \texttt{char}, and \texttt{boolean}.

  \begin{remark}
    There are 2 \nameref*{subsec:User_Defined_Ordinal_Types} that are supported by most programming languages:
    \begin{itemize}[noitemsep]
    \item \nameref{subsubsec:Enumeration_Types}
    \item \nameref{subsubsec:Subrange_Types}
    \end{itemize}
  \end{remark}
\end{definition}

\subsubsection{Enumeration Types}\label{subsubsec:Enumeration_Types}
\begin{definition}[Enumeration Type]\label{def:Enumeration_Type}
  An \emph{enumeration type} is one in which all of the possible values, which are named constants, are provided (enumerated) in the definition.
  Enumeration types provide a way of defining and grouping collections of named constants, called \emph{enumeration constants}.

  This is an example of an enumeration type in C\#:
\begin{minted}[frame=lines,linenos]{csharp}
enum days {Mon, Tue, Wed, Thu, Fri, Sat, Sun};
\end{minted}

  \begin{remark}
    Typically, each of the numeration constants is implicitly assigned an integer literal, though they can be given integer literals explicitly too.
  \end{remark}
\end{definition}

The design issues for \nameref{def:Enumeration_Type}s are:
\begin{itemize}[noitemsep]
\item Is an enumeration constant allowed to appear in more than one \nameref{def:Enumeration_Type} definition, and if so, how is the type of an occurance of that enumeration constant in the program checked?
\item Are enumeration constants coerced to integers?
\item Are any other types coerced to an \nameref{def:Enumeration_Type}?
\end{itemize}

\paragraph{Designs}\label{par:Enumeration_Types_Designs}
In languages without native support of \nameref{def:Enumeration_Type}s, they are simulated with integer values.
For example,
\begin{minted}[frame=lines,linenos]{c}
int red = 0, blue = 1;
\end{minted}

However, this can lead to unexpected behavior.
For example, the variables \texttt{red} and \texttt{blue} can be added together.
In essence, there would be no \nameref{def:Type_Checking}.
The value for those variables could be overwritten somewhere.
Though, that issue would be solved by making the variable a constant instead.

C and Pascal introduced the use of \nameref{def:Enumeration_Type}s.
These implicitly use default values, integers, as the enumeration constants.
However, the values can be set explicitly, by the programmer.
With these \nameref{def:Enumeration_Type}s, we have and avoid these issues:
\begin{minted}[frame=lines,linenos]{c}
enum colors {red, blue, green, yellow, black};
colors myColor = blue, yourColor = red;
myColor++; // Valid code, sets myColor from blue to green
myColor = 4; // Illegal
myColor = (colors) 4; // Legal because 4 is being typecast
\end{minted}

These help prevent some issues, but not all.

The next iteration was in Ada.
They allowed for \emph{overloaded literals} in their \nameref{def:Enumeration_Type}s.
This means there were enumeration constants shared between 2 \nameref{def:Enumeration_Type}s in the same referencing environment.
In this case, the value must be determinable from the context of the \nameref{def:Enumeration_Type}.
Sometimes, a mroe explicit specification must be used.
Additionally, because the enumerations constants were \textbf{not} coerced to integers, nor were the \textbf{enumeration variables}, the range of operations and range of values for the enumeration constants was limited.
This allowed the compiler to pick up many more errors.

\begin{remark*}
  None of the relatively recent scripting kinds of languages include \nameref{def:Enumeration_Type}s.
  These include Perl, JavaScript, PHP, Python, Ruby, and Lua.
\end{remark*}

\paragraph{Evaluation}\label{par:Enumeration_Types_Evaluation}
Enhancements to both \nameref{subsec:Readability} and \nameref{subsec:Reliability}.
\begin{itemize}[noitemsep]
\item \nameref{subsec:Readability} is enhanced by better named values
\item \nameref{subsec:Reliability} is enhanced by being able to perform \nameref{def:Type_Checking} on the \nameref{def:Enumeration_Type}s.
  \begin{itemize}[noitemsep]
  \item No arithmetic operations allowed on \nameref{def:Enumeration_Type}s.
  \item No enumeration variable can be assigned a value outside the \nameref{def:Enumeration_Type}'s assigned range.
  \end{itemize}
\end{itemize}

\subsubsection{Subrange Types}\label{subsubsec:Subrange_Types}
\begin{definition}[Subrange Type]\label{def:Subrange_Type}
  A \emph{subrange type} is a contiguous sequence of an \nameref{def:Ordinal_Type}.
  For example, this is a subrange: \texttt{12..14}.
\end{definition}

\paragraph{Ada's Design}\label{par:Adas_Subrange_Types_Design}
Ada included \nameref{def:Subrange_Type}s in \nameref{def:Ada_Subtype}s.

\begin{definition}[Subtype]\label{def:Ada_Subtype}
  A \emph{subtype} in Ada is an extension, usually constrained, version of existing types.
  For example,
\begin{minted}[frame=lines,linenos]{ada}
type Days is (Mon, Tue, Wed, Thu, Fri, Sat, Sun);
subtype Weekdays is Days range Mon..Fri;
Day1 : Days;
Day2 : Weekdays;
...
Day2 := Day1; -- Will only work if Day1 has Mon-Fri, fails if Day1 = Sat or Sun
\end{minted}
\end{definition}

The compiler generates range-checking code for every assignment to the subrange type.
Subranges require run-time range checking.

\paragraph{Evaluation}\label{par:Subrange_Types_Evaluation}
\nameref{def:Subrange_Type}s improve \nameref{subsec:Readability} by making it clear that \nameref{def:Variable}s of \nameref{def:Ada_Subtype}s can only store a certain range of values.
\nameref{subsec:Reliability} is increased with \nameref{def:Subrange_Type}s because assigning a value to a subrange variable outside its range is detected as an error.

\subsubsection{Implementation of \nameref*{subsec:User_Defined_Ordinal_Types}}\label{subsubsec:Implementation_User_Defined_Ordinal_Types}
\nameref{def:Enumeration_Type}s are usually implemented on integers.
However, without restrictions on ranges of values and possible operations, this does \textbf{not} improve \nameref{subsec:Reliability}.

\nameref{def:Subrange_Type}s are implemented the same way as their parent types, except range checks are implicity included by the compiler in every assignment of a variable or expression to a subrange variable.

%%% Local Variables:
%%% mode: latex
%%% TeX-master: "../../EDAP05-Concepts_Programming_Languages-Reference_Sheet"
%%% End:
