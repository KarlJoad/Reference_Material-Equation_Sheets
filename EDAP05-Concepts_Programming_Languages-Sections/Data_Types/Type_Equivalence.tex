\subsection{Type Equivalence}\label{subsec:Type_Equivalence}
This section does not deal with \nameref{def:Type_Compatibility}, which works for scalar \nameref{def:Data_Type}s, but rather \nameref{def:Type_Equivalence}.
\begin{definition}[Type Compatibility]\label{def:Type_Compatibility}
  \emph{Type compatibility} dictates the \nameref{def:Data_Type} of operands that are acceptable for eacch of the operations of the language.
  This is called compatibility because there are cases when the \nameref{def:Data_Type} of the operand can be implicitly converted by the compiler or run-time system to make the operand acceptable to the operator.

  An example of this is the addition of an integer number and a real number.
  The integer number is typecast to a real number, then the addition is performed.
\end{definition}

\nameref{def:Type_Compatibility} rules are strict for predefined scalar types.
However, structured \nameref{def:Data_Type}s such as \nameref{def:Array}s, \nameref{def:Record_Data_Type}, and others require more complex rules.
Since \nameref{def:Data_Type} coercion is unlikely, the question is if the 2 \nameref{def:Data_Type}s are equivalent.

\begin{definition}[Type Equivalence]\label{def:Type_Equivalence}
  \emph{Type equivalence} is a strict form of \nameref{def:Type_Compatibility}.
  It is when an operand of one \nameref{def:Data_Type} can be substituted for another operand of the same type, without \nameref{def:Data_Type} coercion.

  The design of type equivalence rules influence the design of \nameref{def:Data_Type}s and operations provided for values of those types.

  There are 2 approaches to determining \nameref{def:Type_Equivalence}:
  \begin{enumerate}[noitemsep]
  \item \nameref{def:Type_Equivalence-Name}
  \item \nameref{def:Type_Equivalence-Structure}
  \end{enumerate}

  There is a general algorithm for determining if two \nameref{def:Data_Type}s are equivalent.
  ``$\ldots$ when all constant expressions are replaced by their values and all type names are replaced by their definitions.
  In the case of recursive types, the expansion is the infinite limit of the partial expansions $\ldots$''.
  Meaning you replace everything you can, and if there is an infinite recursion somewhere, you cut it off eventually, usually when you pass the same point again.

  There are 4 types of equivalence that can be established.
  \begin{enumerate}[noitemsep]
  \item \nameref{def:Type_Equivalence-Primitive}
  \item \nameref{def:Type_Equivalence-Structure}
  \item \nameref{def:Type_Equivalence-Reference}
  \item \nameref{def:Type_Equivalence-User_Defined}
  \end{enumerate}
\end{definition}

Different languages use different approaches and combinations of these types of \nameref{def:Type_Equivalence}.
You would have to look at the language's specification to find out exactly what is used where.
Additionally, object-oriented languages present their own, unique, type of \nameref{def:Type_Compatibility} with object compatibility and its relationship to the inheritance hierarchy.

\begin{definition}[Name Type Equivalence]\label{def:Type_Equivalence-Name}
  \emph{Name type equivalence} means that 2 \nameref{def:Variable}s have \nameref{def:Type_Equivalence} if they are defined in:
  \begin{itemize}[noitemsep]
  \item The same declaration
  \item In Declarations that use the same \nameref{def:Data_Type} name
  \end{itemize}

  This is easier to implement, but more restrictive.
  In a strict interpretation, a \nameref{def:Variable} whose \nameref{def:Variable_Type} is a subrange of the integers would \textbf{not} be equivalent to an integer \nameref{def:Variable_Type} \nameref{def:Variable}.
  For example,
\begin{minted}[frame=lines,linenos]{ada}
type Indextype is 1..100;
count : Integer;
index : Indextype;
\end{minted}
  \texttt{count} and \texttt{index} could \textbf{not} be substituted for each other.

  \begin{remark}
    To use \nameref{def:Type_Equivalence-Name}, all \nameref{def:Variable_Type}s must have names.
    If the language supports anonymous \nameref{def:Data_Type}s, then they must be given internal names by the compiler/interpreter.
  \end{remark}
\end{definition}

\begin{definition}[Primitive Equivalence]\label{def:Type_Equivalence-Primitive}
  \emph{Primitive equivalence} directly compares two values by comparing their bit patterns in memory.
\end{definition}

\begin{definition}[Structure Type Equivalence]\label{def:Type_Equivalence-Structure}
  \emph{Structure type equivalence} means 2 \nameref{def:Variable}s have \nameref{def:Type_Equivalence} if their types have identical structures.
  The entire structure's \nameref{def:Variable_Type}s must be compared to determine equivalence.
  This may potentially invole recursing through the structure or iterating over the structure.
  However, this is more flexible than \nameref{def:Type_Equivalence-Name}, but more difficult to implement.

  Ada, with its hyper-strict \nameref{def:Type_Equivalence} has defined 2 ways to make new types.
  \begin{enumerate}[noitemsep]
  \item \nameref{def:Derived_Type}
  \item \nameref{def:Subtype}
  \end{enumerate}
\end{definition}

\begin{definition}[Reference Equivalence]\label{def:Type_Equivalence-Reference}
  \emph{Reference Equivalence} checks whether two pointers/references point to the same address in \nameref{def:Memory}.
\end{definition}

\begin{definition}[User-Defined Equivalence]\label{def:Type_Equivalence-User_Defined}
  \emph{User-defined equivalence} performs arbitrary equality checking but isn't provided by the language itself, and must be specified by the programmer.
\end{definition}

\begin{definition}[Derived Type]\label{def:Derived_Type}
  A \emph{derived type} is a new \nameref{def:Data_Type} that is based on some previously defined \nameref{def:Data_Type}, that \textbf{is not equivalent, but may have an identical structure}.
  Derived types inherit all the properties of their parent types.

  Take the following Ada code snippet as an example.
\begin{minted}[frame=lines,linenos]{ada}
type Celsius is new Float;
type Fahrenheit is new Float;
\end{minted}
  These are not equivalent, though they have identical structures.
  They are also not type equivalent to any other \texttt{Float} type.
\end{definition}

\begin{definition}[Subtype]\label{def:Ada_Subtype}
  A \emph{subtype} is a possibly range-constrained version of an existing type.
  A subtype \textbf{is equivalent with its parent type}.

  Take the following Ada code snippet as an example.
\begin{minted}[frame=lines,linenos]{ada}
subtype Small_type is Integer range 0..99;
\end{minted}
  The \texttt{Small\textunderscore{}type} is equivalent to the \texttt{Integer} type.
\end{definition}
%%% Local Variables:
%%% mode: latex
%%% TeX-master: "../../EDAP05-Concepts_Programming_Languages-Reference_Sheet"
%%% End:
