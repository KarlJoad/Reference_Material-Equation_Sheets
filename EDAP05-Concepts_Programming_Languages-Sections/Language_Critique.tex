\section{Language Critique}\label{sec:Language_Critique}
There are several very open-ended questions that need to be asked when categorizing and critiqueing languages:
\begin{enumerate}[noitemsep]
\item What programming language is best for \emph{what task}?
\item \emph{What criteria} do we measure?
  \begin{itemize}[noitemsep]
  \item Most criteria do not have good measurement tools.
  \end{itemize}

\item \emph{How} do we obtain measurements for these criteria?
\end{enumerate}

These are all qualities of:
\begin{itemize}[noitemsep]
\item The language
\item The language implementation(s)
\item The available tooling for the language and that particular implementation
\item The available libraries for the language and that particular implementation
\item Other infrastructure
  \begin{itemize}[noitemsep]
  \item User groups
  \item Books
  \item etc.
  \end{itemize}
\end{itemize}

\begin{table}[h!]
  \centering
  \begin{tabular}{cccc}
    \toprule
    & \multicolumn{3}{c}{Criteria} \\
    \midrule
    Characteristic & \nameref{subsec:Readability} & \nameref{subsec:Writability} & \nameref{subsec:Reliability} \\
    \midrule
    \nameref{subsubsec:Simplicity} & \checkmark{} & \checkmark{} &  \checkmark{} \\
    \nameref{subsubsec:Orthogonality} & \checkmark{} & \checkmark{} & \checkmark{} \\
    \nameref{subsubsec:Data_Types} & \checkmark{} & \checkmark{} & \checkmark{} \\
    \nameref{subsubsec:Syntax_Design} & \checkmark{} & \checkmark{} & \checkmark{} \\
    \midrule
    \nameref{subsubsec:Abstraction_Support} & & \checkmark{} & \checkmark{} \\
    \nameref{subsubsec:Expressivity} & & \checkmark{} & \checkmark{} \\
    \midrule
    \nameref{subsubsec:Type_Checking} & & & \checkmark{} \\
    \nameref{subsubsec:Exception_Handling} & & & \checkmark{} \\
    Restricted \nameref{subsubsec:Aliasing} & & & \checkmark{} \\
    \bottomrule
  \end{tabular}
  \caption[Language Evaluation Criteria]{Language Evaluation Criteran and the Characteristics that Affect Them}
  \label{tab:Language_Evaluation_Criteria}
\end{table}

Some additional criteria that could be used to evaluate programming languages are:
\begin{itemize}[noitemsep]
\item Portability: Ease with with programs can be moved from one implementation to another
\item Generality: The applicability to a wide range of applications
\item Well-Definedness: The completeness and precision of the language's official defining document
\end{itemize}

Some of criteria are given different weightings/importance by different people, thus making each slightly subjective.
Additionally, many of these criteria are not precisely defined, nor are they exactly measurable.

\subsection{Readability}\label{subsec:Readability}
\begin{definition}[Readability]\label{def:Readability}
  \emph{Readability} is how easily a program can be read and understood \emph{by a human}.
  Some languages do not support certain functions, but programmers try to make the language do what it is not designed to do.
  This will lead to complicated and difficult-to-read programs.
  
  The idea of program readability was first presented as the software life-cycle concept~\parencite{Booch1987}.
  The initial coding was downplayed compared to earlier, and the maintenance and improvement of the code was brought to the forefront.
\end{definition}

\subsubsection{Simplicity}\label{subsubsec:Simplicity}
There are 2 main factors and 1 minor factor for a language's simplicity:
\begin{enumerate}[noitemsep]
\item The number of features present in the language.
\item The \nameref{def:Feature_Multiplicity} of a language.
\item The ability to \nameref{def:Overload_Operator}s.
\end{enumerate}

Assembly language is on the most-simple end of the simplicity spectrum.
In assembly, the form and meaning of most statements are incredibly simple, but without more complex control statements, the program's structure is less obvious.

\begin{definition}[Feature Multiplicity]\label{def:Feature_Multiplicity}
  \emph{Feature multiplicity} is when there is more than one way to accomplish a particular operation with language built-in features.
  For example, in Java these are all equivalent when evaluated as standalone expressions:
  \begin{enumerate}[noitemsep]
  \item \texttt{count = count + 1}
  \item \texttt{count += 1}
  \item \texttt{count++}
  \item \texttt{++count}
  \end{enumerate}
\end{definition}

\begin{minted}[frame=lines,linenos]{python3}
def d(x):
    r = x[::-1]
    return x == r
\end{minted}

\begin{definition}[Overload Operator]\label{def:Overload_Operator}
  An \emph{overloaded operator} is one where a single symbol has more than one meaning.
  For example the \texttt{+} operator can be overloaded to add 2 integers, 2 floating-point numbers, or an integer and a floating-point number.
  This overloading helps \emph{improve} the \nameref{subsubsec:Simplicity} of a language.
\end{definition}

\begin{minted}[frame=lines,linenos]{python3}
x = 3 + 4  # Evaluates to 7
y = 3.0 + 4.0  # Evaluates to 7.0
z = 3 + 4.0  # Evaluates to 7.0
\end{minted}

\subsubsection{Orthogonality}\label{subsubsec:Orthogonality}
\begin{definition}[Orthogonality]\label{def:Orthogonality}
  \emph{Orthogonality} in a programming language means that a relatively small set of primitive constructs can be combined in a relatively small number of ways to build the control and data structures of a language.
  Additionally, every possible combination of primitives is legal and meaningful.

  The more orthogonal a language, the fewer exceptions to the language rules there can be.
  These fewer exceptions means there is a higher degree of regularity in the language design, making it easier to learn, read, and understand.

  \begin{remark}[Over-Orthogonality]\label{rmk:Over_Orthogonality}
    Too much orthogonality can cause problems.
    Having too much combinational freedom with primitive constructs and their combinations can make for an extremely complex compound construct.
    This leads to unnecessary complexity.
  \end{remark}
\end{definition}

\begin{minted}[frame=lines,linenos]{c}
// global variable section

float f1 = 2.0f * 2.0f;
float f2 = sqrt(2.0f); // error
\end{minted}

\subsubsection{Data Types}\label{subsubsec:Data_Types}
The use of data types conveys intent when reading and writing the program.
For example, a boolean data type conveys a true/false value better than an integer that is 1/0 for true/false respectively.
\begin{itemize}[noitemsep]
\item \texttt{timeOut = 1}
\item \texttt{timeOut = true}
\end{itemize}

\begin{minted}[frame=lines,linenos]{java}
enum Color {
  Red, Green, Blue
};

Color c = readColorFromUser();
\end{minted}

\subsubsection{Syntax Design}\label{subsubsec:Syntax_Design}
There are 2 main syntactic design choices that affect \nameref{subsec:Readability}:
\begin{enumerate}[noitemsep]
\item \nameref{par:Reserved_Words}
\item \nameref{par:Syntax_Form_Meaning}
\end{enumerate}

\paragraph{Reserved/Special Words}\label{par:Reserved_Words}
\begin{definition}[Reserved Word]\label{def:Reserved_Word}
  \emph{Reserved word}s are words that are reserved by the language constructors because those particular words have a meaning in the language.
  For example:
  \begin{itemize}[noitemsep]
  \item \texttt{while}
  \item \texttt{class}
  \item \texttt{for}
  \end{itemize}
\end{definition}

There are also special words and matching characters that can denote groups of instructions.
\begin{itemize}[noitemsep]
\item C and its decendants
  \begin{itemize}[noitemsep]
  \item Matching brances
  \item \texttt{\{} and \texttt{\}}
  \end{itemize}
\item Ada/Fortran 95 and their decendants:
  \begin{itemize}[noitemsep]
  \item Distinct closing syntax for each statement group
  \item \texttt{end if} to end an if statement
  \end{itemize}
\end{itemize}

Also, can these \nameref{def:Reserved_Word}s be used as names for program variables?
If so, this will increase overall complexity of a program.
The code block below, from Fortran 95, illustrates this point.
\begin{minted}[frame=lines,linenos]{fortran}
program hello
  implicit none
  integer end, do
  do = 0
  end = 10
  do do=do, end
    print *,do
  end do
end program hello
\end{minted}

\paragraph{Form and Meaning}\label{par:Syntax_Form_Meaning}
Statements should be designed such that their appearance partially indicates what their purpose is.
For example, the UNIX command \texttt{grep} gives no hint at what it is supposed to do, unless you know the text editor \texttt{ed}.

Semantics, or meaning, should follow directly from syntax or form.
In some cases, this principle is violated by 2 language constructs that are identical or similar in appearance, but different in meaning, depending on the context.
For example, C's \texttt{static} \nameref{def:Reserved_Word}s.

\subsection{Writability}\label{subsec:Writability}
Writability is a measure of how easy it is to write a program in a language for a given problem domain.
This is closely related to the language characteristics presented in \Cref{subsec:Readability}, \nameref{subsec:Readability}.
The definition of the problem domain is incredibly important, because C would not be used to make a GUI, and Visual BASIC would not be used to make an operating system.

\subsubsection{\nameref{subsubsec:Simplicity} and \nameref{subsubsec:Orthogonality}}\label{subsubsec:Written_Simplicity_Orthogonality}
Programmers might not know all the features for a language.
Or, the might know about them, but use them incorrectly.
This means there should be a smaller number of primitive constructs and a consistent set of rules for combining them.
This reduces the number of primitive constructs in a language, and allows a programmer to design a complex solution by only using a simple set of primitive constructs.
By reducing the orthogonality of a program, the total possible combinations of constructs is reduced, simplifying the process of reading and writing the program.

\subsubsection{Support for Abstraction}\label{subsubsec:Abstraction_Support}
\begin{definition}[Abstraction]\label{def:Abstraction}
  \emph{Abstraction} is the ability to define and then use complicated structures or operations in ways that allow many of the details to be ignored.
  This is a key concept in modern programming language design.

  There are 2 categories of abstraction:
  \begin{enumerate}[noitemsep]
  \item \nameref{par:Process_Abstraction}
  \item \nameref{par:Data_Abstraction}
  \end{enumerate}
\end{definition}

\paragraph{Process Abstraction}\label{par:Process_Abstraction}
Process abstraction is the use of a subprogram to implement some block of code used multiple times.
For example, a sorting algorithm.
If the code for the algorithm could not be factored out into a separate piece of code, the algorithm would need to be copied everywhere it was used.
This would lead to additional complexity and reduce the \nameref{subsec:Readability} of the code.

\paragraph{Data Abstraction}\label{par:Data_Abstraction}
For example, representing a binary tree in C++/Java is done by making a tree node class that has 2 pointers and an integer.
This abstraction is more natural to think about than what would need to be done in Fortran 77.
In Fortran 77, there would need to be 3 parallel integer arrays, where 2 of the integers in each array would be used as subscripts to find their children.

\subsubsection{Expressivity}\label{subsubsec:Expressivity}
Expressivity has several characteristics.
\begin{enumerate}[noitemsep]
\item Verbosity of the language
  \begin{itemize}[noitemsep]
  \item The amount of code needed to describe some computation to the computer.
  \end{itemize}
\item Powerful/Convenient way to specify computations.
  \begin{itemize}[noitemsep]
  \item \texttt{count++} vs. \texttt{count = count + 1} to increment a value in Java
  \end{itemize}
\end{enumerate}

\subsection{Reliability}\label{subsec:Reliability}
Reliability is a measure of the program performing to its specifications reliably under all conditions.

\subsubsection{Type Checking}\label{subsubsec:Type_Checking}
\begin{definition}[Type Checking]\label{def:Type_Checking}
  \emph{Type checking} is a process for testing for type errors in a given program, by the compiler or the interpreter, depending on its implementation (\nameref{subsec:Interpretation} vs. \nameref{def:Compilation}).
  Runtime type checking is expensive, so compile-time checking is preferred.
\end{definition}

The earlier that type checking can occur reduces the potential errors, and corrective actions can be taken.

\subsubsection{Exception Handling}\label{subsubsec:Exception_Handling}
The programming language should have the ability to intercept runtime errors, along with other unusual conditions, take corrective actions, the continue normally is traditionally called \emph{exception handling}.

\subsubsection{Aliasing}\label{subsubsec:Aliasing}
\begin{definition}[Aliasing]\label{def:Aliasing}
  \emph{Aliasing} is having 2 or more distinct names that can be used to access the same memory location.
  Most languages allow for 2 pointers to point to the same thing in memory, but others prevent this completely.
\end{definition}

Sometimes, \nameref{def:Aliasing} is used to overcome deficiencies in the language's \nameref{subsubsec:Abstraction_Support}.
Others greatly restrict possible \nameref{def:Aliasing} to increase their \nameref{subsec:Reliability}.

\subsubsection{\nameref{subsec:Readability} and \nameref{subsec:Writability}}\label{subsubsec:Reliable_Readability_and_Writability}
The \nameref{subsec:Readability} and \nameref{subsec:Writability} greatly influence a program's \nameref{subsec:Reliability}.
A program written in a language that exceeds the languages original problem domain will use unnatural approaches to solve the problem.
These unnatural approaches are less likely to be correct for all possible situations.
Thus, the easier a program is to write, the more likely it is to be correct for all possible situations.

Programs that are difficult to read will affect the writing and maintenance phases of the software's life cycle.

\subsection{Cost}\label{subsec:Cost}
There are several parts that increase the cost of a programming language.
\begin{enumerate}[noitemsep]
\item Training programmers in a new language
  \begin{itemize}[noitemsep]
  \item Function of \nameref{subsubsec:Simplicity} and \nameref{subsubsec:Orthogonality}
  \item Function of programmer experience
  \end{itemize}
  
\item Writing software
  \begin{itemize}[noitemsep]
  \item Function of \nameref{subsec:Writability} of the language
  \end{itemize}
  
\item Compilation time
  \begin{itemize}[noitemsep]
  \item Time to compile a program
  \item Resources required to compile a program in a language
  \end{itemize}

\item Run time
  \begin{itemize}[noitemsep]
  \item Performance during runtime
  \item Dependent on the effort made to optimize the input source code
  \end{itemize}

\item Financial cost of special software
  \begin{itemize}[noitemsep]
  \item The cost of using the \nameref{def:Compiler} for a language for instance.
  \item Languages with free \nameref{def:Compiler}s or interpreters tend to be accepted more quickly than languages with a financial cost
  \end{itemize}

\item Cost of limited reliability
  \begin{itemize}[noitemsep]
  \item Maintenance time
    \begin{itemize}[noitemsep]
    \item Corrections made to errors in the program
    \item Modifications to add new functionality
    \end{itemize}

  \item Insurance cost, in special cases
    \begin{itemize}[noitemsep]
    \item Airplanes
    \item Nuclear power plants
    \item X-Ray machines
    \end{itemize}
  \end{itemize}
\end{enumerate}

%%% Local Variables:
%%% mode: latex
%%% TeX-master: "../EDAP05-Concepts_Programming_Languages-Reference_Sheet"
%%% End:
