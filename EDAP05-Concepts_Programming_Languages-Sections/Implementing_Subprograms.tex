\section{Implementing \nameref*{sec:Subprograms}}\label{sec:Implementing_Subprograms}
\subsection{General Semantics of Calls and Returns}\label{subsec:Semantics_of_Calls_and_Returns}
\begin{definition}[Subprogram Linkage]\label{def:Subprogram_Linkage}
  The \texttt{call} and \texttt{return} operations are together called \emph{subprogram linkage}.
\end{definition}

There are several steps that must occur for a subprogram's \texttt{call} to work.
\begin{enumerate}[noitemsep]
\item Include the implementation of that language's parameter-passing method.
\item If local variables are not static, there must be space allocation for the \nameref{def:Local_Variable}s declared in the subprogram, and bind them to storage.
\item Must save the execution status of the CPU, everything required to jump back to the point where the \texttt{all} occurs. This includes:
  \begin{itemize}[noitemsep]
  \item CPU status bits
  \item The Environment Pointer (\nameref{def:Dynamic_Link})
  \item Register values
  \end{itemize}
\item Arrange the transfer of control to the code of the subprogram, and ensure control can be returned to the proper place when execution is done.
\item If the language supports nested subprograms, there needs to be a mechanism to provide access to the parent subprogram's variables. (\nameref{def:Static_Link})
\end{enumerate}

The required actions for a subprogram to return its execution to the parent subprogram are:
\begin{enumerate}[noitemsep]
\item If the parameters are \nameref{def:Parameter_Passing-Out_Mode} or \nameref{def:Parameter_Passing-Inout_Mode}, the local values of the associated \nameref{def:Formal_Parameter}s must be copied to the \nameref{def:Actual_Parameter}s.
\item All storage used for \nameref{def:Local_Variable}s must be deallocated.
\item Restore the execution status of the calling parent subprogram.
\item Return control to the calling parent subprogram.
\end{enumerate}

\subsection{Implementing ``Simple'' Subprograms}\label{subsec:Implementing_Simple_Subprograms}
In this case, ``simple'' means subprograms that cannot be nested and all local variables are static.

There must be storage for:
\begin{itemize}[noitemsep]
\item Status information for the caller program.
\item Parameters
\item The return \nameref{def:Memory} address
\item Return values for functions
\item Temporaries used by the code of the subprogram(s)
\end{itemize}

Some of the semantic actions in the next 2 sections (\Crefrange{subsubsec:Implementing_Simple_Subprogram-Call}{subsubsec:Implementing_Simple_Subprogram-Return}) can be occur at 2 different times during \nameref{def:Subprogram_Linkage}.
These are called the \emph{prologue} and \emph{epilogue} of the \nameref{def:Subprogram_Linkage}.

\subsubsection{Semantics of the Subprogram \texttt{Call}}\label{subsubsec:Implementing_Simple_Subprogram-Call}
The following steps must be followed:
\begin{enumerate}[noitemsep]
\item Save the execution status of the current program unit.
\item Compute and pass the parameters.
\item Pass the return address to be called at the end of subprogram execution.
\item Transfer control to the called subprogram.
\end{enumerate}

The last 3 actions must be done by the caller program.
The first action could be done by the caller program or the called subprogram.

\subsubsection{Semantics of the Subprogram \texttt{Return}}\label{subsubsec:Implementing_Simple_Subprogram-Return}
These are the steps that must be followed to \texttt{return} from a subprogram:
\begin{enumerate}[noitemsep]
\item If parameters are \nameref{def:Pass_By_Value_Result} or \nameref{def:Parameter_Passing-Inout_Mode}, the current values of those \nameref{def:Formal_Parameter}s are moved/made available to the corresponding \nameref{def:Actual_Parameter}s.
\item If the subprogram is a function, the functional value is moved to a place accessible to the caller.
\item The execution status of the caller program is restored.
\item Control is transferred back to the caller program.
\end{enumerate}

The first, third, and fourth steps could be handled by the called subprogram.

\subsubsection{Activation Records for ``Simple'' Subprograms}\label{subsubsec:Implementing_Simple_Subprogram-Activation_Record}
\begin{definition}[Activation Record]\label{def:Activation_Record-Simple_Subprograms}
  An \emph{activation record} for a ``simple'' subprogram is the noncode portion, because the data it describes is only relevant during the activation/execution of the subprogram.
  A concrete example of an activation record is called an \emph{activation record instance}.

  In this ``simple'' subprogram setup, activation records are of fixed size.
  There can also only be one active version of a given subprogram at a time (since recursion isn't supported with just \nameref{def:Static_Variable_Binding_Lifetime}s).
\end{definition}

%%% Local Variables:
%%% mode: latex
%%% TeX-master: "../EDAP05-Concepts_Programming_Languages-Reference_Sheet"
%%% End:
