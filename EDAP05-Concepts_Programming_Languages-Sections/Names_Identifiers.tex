\section{Names}\label{sec:Names}
\emph{Names} or \emph{identifiers} are, obviously, names given to things.
They can identify:
\begin{itemize}[noitemsep]
\item Variables
\item Subprograms
\item Formal Parameters
\item Other program constructs
\end{itemize}

\subsection{Issues}\label{subsec:Names_Issues}
There are 2 questions that need to be asked when designing the names or identifiers possible in a language.
\begin{enumerate}[noitemsep]
\item Are names case-sensitive? For example, are these identifiers different?
  \begin{itemize}[noitemsep]
  \item \texttt{myvariable}
  \item \texttt{MYVARIABLE}
  \item \texttt{MyVariable}
  \item \texttt{myVariable}
  \end{itemize}
\item Are the special words of the language \nameref{def:Reserved_Word}s or are they \nameref{def:Keyword}s?
\end{enumerate}

\subsection{Name Forms}\label{subsec:Name_Forms}
How is a name/identifier defined?
\begin{itemize}[noitemsep]
\item Is there a character limit on the identifier/name?
\item Are all characters in the identifier/name significant?
\item What characters are allowed in the identifier/name?
\item Are there special characters required by a language?
  \begin{itemize}[noitemsep]
  \item \texttt{\$} being required in front of identifiers in PHP
  \item \texttt{\$}, \texttt{@}, \texttt{\%} specifying a ``type''  in Perl
  \item \texttt{@} and \texttt{@@} to denote an instance or class variable in Ruby, respectively
  \end{itemize}
\end{itemize}

Some languages are \emph{case-sensitive}.
C, Java, C++, etc.\ would all treat \texttt{rose}, \texttt{ROSE}, and \texttt{Rose} differently.
This could be a detriment to readability, because names that \textit{look} similar are actually not.
In terms of writability, the programmer must remember the exact typecasing of the identifier/name.

\subsection{Special Names}\label{subsec:Special_Names}
There are \nameref{def:Reserved_Word}s and \nameref{def:Keyword}s.
They are similar in that the programming language specification defines that these words have special meanings when constructing programs.
However, the 2 differ in how these words can potentially be reused.

\begin{definition}[Keyword]\label{def:Keyword}
  \emph{Keyword}s are words that are defined by the language constructors to have some special meaning.
  However, it only has these special meanings in \textbf{\emph{certain contexts}}.
  This means you can define a keyword as a variable and use it together with the keyword.
  For example, this is a perfectly valid piece of Fortran code:
\begin{minted}[frame=lines,linenos]{fortran}
Integer Apple
Integer = 4
\end{minted}
\end{definition}

\begin{definition}[Reserved Word]\label{def:Reserved_Word}
  \emph{Reserved word}s are words that are reserved by the language constructors because those particular words have a meaning in the language.
  These words cannot be used as identifiers for \textbf{ANYTHING} else.
  For example:
  \begin{itemize}[noitemsep]
  \item \texttt{while}
  \item \texttt{class}
  \item \texttt{for}
  \end{itemize}

  \begin{remark}[Too Many \nameref{def:Reserved_Word}s]\label{rmk:Too_Many_Reserved_Words}
    The potential problem with \nameref{def:Reserved_Word}s is that if a language has a large number of reserved words, the user might have a hard time creating names for themselves.
    Unfortunately, the most commonly chosen words by programs are usually \nameref{def:Reserved_Word}s.
    For example,
    \begin{itemize}[noitemsep]
    \item \texttt{LENGTH}
    \item \texttt{BOTTOM}
    \item \texttt{DESTINATION}
    \item \texttt{COUNT}
    \end{itemize}
  \end{remark}
\end{definition}

\subsection{Variables}\label{subsec:Variables}
\begin{definition}[Variable]\label{def:Variable}
  A program \emph{variable} is an abstraction of a computer \nameref{def:Memory} cell or a collection of \nameref{def:Memory} cells.
  A variable can be characterized by a sextuple of attributes:
  \begin{enumerate}[noitemsep]
  \item \nameref{subsubsec:Variable_Name}
  \item \nameref{subsubsec:Variable_Address}
  \item \nameref{subsubsec:Variable_Value}
  \item \nameref{subsubsec:Variable_Type}
  \item Lifetime
  \item Scope
  \end{enumerate}
\end{definition}

\subsubsection{Name}\label{subsubsec:Variable_Name}
Most \nameref{def:Variable}s have names.
These are symbolic references to the value that is actually stored.
There are various issues that may arise with the name of a variable, which were discussed earlier.

\subsubsection{Address}\label{subsubsec:Variable_Address}
\begin{definition}[Address]\label{def:Variable_Address}
  The \emph{address} of a \nameref{def:Variable} is the machine's memory address with which the \nameref{def:Variable} is associated.

  The address of a variable is sometimes called its \emph{L-Value}.
  This is because the address is required when the name of a variable appears on the left-hand side of an assignment statement.

  \begin{remark}[Alias]
    An \emph{alias} is having another \nameref{def:Variable} have the same \nameref{def:Variable_Address}, so the 2 \nameref{def:Variable}s point to the same value in \nameref{def:Memory}.
  \end{remark}
\end{definition}

For some languages, it is possible for the same \nameref{def:Variable} to be associated with different addresses at different times during the \nameref{def:Variable}'s lifetime.

\subsubsection{Type}\label{subsubsec:Variable_Type}
\begin{definition}[Type]\label{def:Variable_Type}
  The \emph{type} of a \nameref{def:Variable} determines the range of values that \nameref{def:Variable} can store.
  For example, the \texttt{int} type in Java specifies a value range of $-2147483648$ to $2147483647$.
  It is a 32-bit signed integer.
\end{definition}

\subsubsection{Value}\label{subsubsec:Variable_Value}
\begin{definition}[Value]\label{def:Variable_Value}
  The \emph{value} of a \nameref{def:Variable} is the contents of the \nameref{def:Memory} cell or cells associated with the \nameref{def:Variable}.
  The value of a variable is sometimes called it \emph{R-Value}.
  This is because the value of the \nameref{def:Variable} is required on the right-hand side of an assignment statement.
  To access the \textit{r}-value, the \textit{l}-value must be determined first.
  
  \begin{remark}[Abstract Memory Cells]\label{rmk:Abstract_Memory_Cells}
    While in hardware, the individual sizes of \nameref{def:Memory} are fixed, we can think of \nameref{def:Memory} as having \emph{abstract memory cells}, that can accomodate anything we attempt to put into \nameref{def:Memory}.
    This means that a single-precision floating point number technically takes up 4 bytes, 32 bits, of \nameref{def:Memory} cells, that number only takes one abstract memory cell.
  \end{remark}
\end{definition}

\subsection{Binding}\label{subsec:Binding}
\begin{definition}[Binding]\label{def:Binding}
  A \emph{binding} is an association between an attribute and an entity.
  This association can be between:
  \begin{itemize}[noitemsep]
  \item A variable
    \begin{itemize}[noitemsep]
    \item Its type
    \item Its value
    \end{itemize}
  \item An operation
    \begin{itemize}[noitemsep]
    \item Its symbol
    \end{itemize}
  \end{itemize}

  The time at which a binding occurs is called the \nameref{def:Binding_Time}.
\end{definition}

\begin{definition}[Binding Time]\label{def:Binding_Time}
  The time at which \nameref{def:Binding} occurs is called the \emph{binding time}.
  These include:
  \begin{itemize}[noitemsep]
  \item Language Design Time
    \begin{itemize}[noitemsep]
    \item Defining \texttt{*} to represent multiplication
    \end{itemize}
  \item Language Implementation Time
    \begin{itemize}[noitemsep]
    \item Having an \texttt{int} in C be a 32-bit signed integer
    \end{itemize}
  \item Compiler Time
    \begin{itemize}[noitemsep]
    \item The type of a variable in a Java program
    \end{itemize}
  \item Link Time
    \begin{itemize}[noitemsep]
    \item A call to a library subprogram is bound to the subprogram code
    \end{itemize}
  \item Load Time
    \begin{itemize}[noitemsep]
    \item \nameref{def:Variable} bound when loaded into \nameref{def:Memory}
    \item Could happen at run time too
    \end{itemize}
  \item Run Time
    \begin{itemize}[noitemsep]
    \item \nameref{def:Variable} bound when loaded into \nameref{def:Memory}
    \item Could happen at compile time too
    \end{itemize}
  \end{itemize}
\end{definition}

We need to know the \nameref{def:Binding_Time}s for the attributes of a program to understand the semantics of the programming language.

\subsubsection{Binding of Attributes to \nameref*{def:Variable}s}\label{subsubsec:Binding_Attributes_Variables}
\begin{definition}[Static]\label{def:Static_Variable_Binding}
  A \nameref{def:Binding} is \emph{static} if the \nameref{def:Binding} first occurs before run time begins and remains unchanged throughout program execution.
  An example of this is declaring a \nameref{def:Variable} as an \texttt{int} in C.
  Throughout the whole C program, that \nameref{def:Variable} can only hold signed 32-big integers.
\begin{minted}[frame=lines,linenos]{c}
int x = 4;
float x = 4.0; // Error here, x already declared
x = 4.0 // Error here, x is of int type
\end{minted}
\end{definition}

\begin{definition}[Dynamic]\label{def:Dynamic_Variable_Binding}
  A \nameref{def:Binding} is \emph{dynamic} if the \nameref{def:Binding} occurs during run time, or can change in the course of program execution.
  An example of this is declaring a \nameref{def:Variable} in Python.
\begin{minted}[frame=lines,linenos]{python3}
x = 4
x = [1, 2, 3]
x = 'dynamically bound string'
\end{minted}
  All three lines have a variable declaration, where the \nameref{def:Binding} occur during the program's execution and changed during it.
\end{definition}

We are only concerned with the distinction between \nameref{def:Static_Variable_Binding} and \nameref{def:Dynamic_Variable_Binding} \nameref{def:Variable} \nameref{def:Binding}.
Meaning, we will ignore how hardware may bind and unbind things repeatedly when it is switching and moving things around.

\subsubsection{Type \nameref*{subsec:Binding}s}\label{subsubsec:Type_Bindings}
Before a \nameref{def:Variable} can be used or referenced in a program, its \nameref{def:Variable_Type} must be declared.
A \nameref{def:Variable}'s \emph{type} determines the range of values that can be stored in the \nameref{def:Variable}.
In a more abstract sense, it also determines what kind of operations make sense and are possible to use on these \nameref{def:Variable}s.
There are 2 important aspects of this \nameref{def:Binding}:
\begin{enumerate}[noitemsep]
\item How the \nameref{def:Variable} \nameref{def:Variable_Type} is specified
\item When the \nameref{def:Binding} takes place
\end{enumerate}

There are 2 ways to bind \nameref{def:Variable_Type}s to \nameref{def:Variable}s:
\begin{enumerate}[noitemsep]
\item \nameref{par:Static_Variable_Type_Binding}
  \begin{itemize}[noitemsep]
  \item \nameref{def:Explicit_Static_Variable_Type_Binding}
  \item \nameref{def:Implicit_Static_Variable_Type_Binding}
  \end{itemize}
\item \nameref{par:Dynamic_Variable_Type_Binding}
\end{enumerate}

\paragraph{Static \nameref*{def:Variable_Type} \nameref*{subsec:Binding}}\label{par:Static_Variable_Type_Binding}
\begin{definition}[Static]\label{def:Static_Variable_Type_Binding}
  \emph{Static} \nameref{def:Binding} of \nameref{def:Variable}s means that the \nameref{def:Variable_Type} of a \nameref{def:Variable} is given to the program, either \nameref{def:Explicit_Static_Variable_Type_Binding}ly or \nameref{def:Explicit_Static_Variable_Type_Binding}ly.
  Once the \nameref{def:Variable_Type} is declared, it cannot be changed throughout the entire program's execution.

  There are 2 ways to \nameref{def:Static_Variable_Type_Binding}ly bind a \nameref{def:Variable_Type} to a \nameref{def:Variable}:
  \begin{enumerate}[noitemsep]
  \item \nameref{def:Explicit_Static_Variable_Type_Binding}
  \item \nameref{def:Implicit_Static_Variable_Type_Binding}
  \end{enumerate}
\end{definition}

\begin{definition}[Explicit]\label{def:Explicit_Static_Variable_Type_Binding}
  An \emph{explicit} declaration is a statement that explicitly sets each \nameref{def:Variable} to its respective \nameref{def:Variable_Type}.
  For example,
\begin{minted}[frame=lines,linenos]{c}
int x = 0;
float x = 0.0;
char x = 'x';
\end{minted}
\end{definition}

\begin{definition}[Implicit]\label{def:Implicit_Static_Variable_Type_Binding}
  An \emph{implicit} declaration associates \nameref{def:Variable}s with \nameref{def:Variable_Type}s through default conventions.
  The first appearance of a \nameref{def:Variable} name is its implicit declaration.
\end{definition}

Implicit declarations are handled by the language processor (\nameref{def:Compiler} or Interpreter).
There are several ways to have implicit declarations work, some of which are:
\begin{itemize}[noitemsep]
\item Naming conventions
  \begin{itemize}[noitemsep]
  \item In \textbf{Fortran}, if an identifier starts with
    \begin{itemize}[noitemsep]
    \item \texttt{I}, \texttt{J}, \texttt{K}, \texttt{L}, \texttt{M}, or \texttt{N}, or their lowercase versions, it is \nameref{def:Implicit_Static_Variable_Type_Binding}ly declared to be an \texttt{Integer} type.
    \item Otherwise, it is \nameref{def:Implicit_Static_Variable_Type_Binding}ly declared to be a \texttt{Real} type.
    \end{itemize}
  \item In \textbf{Perl}, an identifier must be preceded by a special character denoting the \nameref{def:Variable_Type}. This method forms separate namespaces for each \nameref{def:Variable} \nameref{def:Variable_Type}.
    \begin{itemize}[noitemsep]
    \item \texttt{\$}, is a scalar. This holds numbers and strings
    \item \texttt{@}, is an array.
    \item \texttt{\%}, is a hash structure.
    \item The separate namespaces means that all 3 of these variables are considered unique, and potentially unrelated.
      \begin{itemize}[noitemsep]
      \item \texttt{\$apple}
      \item \texttt{@apple}
      \item \texttt{\%apple}
      \end{itemize}
    \end{itemize}
  \end{itemize}
  
\item Context or type inference
  \begin{itemize}[noitemsep]
  \item In \textbf{C\#}, a \texttt{var} declaration for a \nameref{def:Variable} must include an initial value, which determines the \nameref{def:Variable_Type} of the \nameref{def:Variable}.
\begin{minted}[frame=lines,linenos]{javascript}
var sum = 0;
var total = 0.0;
var name = "Fred";
\end{minted}
  \item \texttt{sum}, \texttt{total}, and \texttt{name} are an \texttt{int}, \texttt{float}, and \texttt{string}, respectively.
  \end{itemize}
\end{itemize}

\begin{remark*}
  Both \nameref{def:Explicit_Static_Variable_Type_Binding} and \nameref{def:Implicit_Static_Variable_Type_Binding} declarations create \nameref{def:Static_Variable_Binding} \nameref{def:Binding}s to \nameref{def:Variable_Type}s.
\end{remark*}

\paragraph{Dynamic \nameref*{def:Variable_Type} \nameref*{subsec:Binding}}\label{par:Dynamic_Variable_Type_Binding}
With \nameref*{par:Dynamic_Variable_Type_Binding}, the \nameref{def:Variable_Type} of a \nameref{def:Variable}:
\begin{itemize}[noitemsep]
\item Is not specified by a declaration statement
\item Cannot be determined by the spelling of the \nameref{def:Variable}'s name
\end{itemize}

\begin{definition}[Dynamic]\label{def:Dynamic_Variable_Type_Binding}
  A \emph{dynamic} \nameref{def:Binding} happens when a \nameref{def:Variable} is bound to a \nameref{def:Variable_Type} \textbf{when it is assigned a \nameref{def:Variable_Value}.}
  Such an assignment might also bind the \nameref{def:Variable} to an \nameref{def:Variable_Address}.

  Any \nameref{def:Variable} can be assigned any \nameref{def:Variable_Type}.
  A \nameref{def:Variable}'s \nameref{def:Variable_Type} can be changed any number of times during program execution.

  The name of the \nameref{def:Variable} is bound to the \nameref{def:Variable}, then the \nameref{def:Variable} is bound to a \nameref{def:Variable_Type} and given its \nameref{def:Variable_Value}.
\end{definition}

The primary benefit of having \nameref{def:Dynamic_Variable_Type_Binding} \nameref{def:Binding} is the programming flexibility it provides.
The 2 major disadvantages are:
\begin{enumerate}[noitemsep]
\item Programs are less reliable, because error-detection of the compiler/interpreter is diminished relative to a compiler/interpreter for a language with \nameref{def:Static_Variable_Type_Binding} \nameref{def:Variable_Type} \nameref{def:Binding}s.
\item The \nameref{subsec:Cost} is quite high because of the \nameref{def:Type_Checking} that must occur at run time. Also, every variable must have a run-time descriptor to describe the \nameref{def:Variable}'s current \nameref{def:Variable_Type}.
\end{enumerate}

\begin{remark*}
  \nameref{def:Dynamic_Variable_Type_Binding} \nameref{def:Variable_Type} \nameref{def:Binding} is usually implemented with \nameref{subsec:Interpretation}.
  This is because:
  \begin{itemize}[noitemsep]
  \item The overall \nameref{subsec:Cost} of \nameref{def:Variable_Type} handing is hidden by the \nameref{subsec:Cost} of the interpreter.
  \item The \nameref{def:Variable_Type} of an operation's operands must be known to translate the instruction to the correct machine code instruction, which isn't possible with \nameref{def:Dynamic_Variable_Type_Binding} \nameref{def:Variable_Type} \nameref{def:Binding}.
  \end{itemize}
\end{remark*}

\subsubsection{Storage \nameref*{subsec:Binding}s and Lifetime}\label{subsubsec:Storage_Bindings_and_Lifetime}
The \nameref{def:Memory} cell to which a \nameref{def:Variable} is bound must be pulled from the pool of available \nameref{def:Memory}.
The act of binding the \nameref{def:Variable_Value} to a \nameref{def:Variable} is called \nameref{def:Variable_Memory_Allocation}.
The act of unbinding is called \nameref{def:Variable_Memory_Deallocation}.

\begin{definition}[Allocation]\label{def:Variable_Memory_Allocation}
  \emph{Allocation} is the act of binding a \nameref{def:Variable_Value} to a \nameref{def:Memory} cell for a \nameref{def:Variable}.
\end{definition}

\begin{definition}[Deallocation]\label{def:Variable_Memory_Deallocation}
  \emph{Deallocation} is the process of placing a \nameref{def:Memory} cell that has been unbound from a variable back into the pool of available \nameref{def:Memory}.
\end{definition}

\begin{definition}[Lifetime]\label{def:Variable_Lifetime}
  The \emph{lifetime} of a \nameref{def:Variable} is the time in which the \nameref{def:Variable} is bound to a \nameref{def:Memory} cell.
  The lifetime of a \nameref{def:Variable} starts when it is bound to a cell and ends when it has been unbound from that cell.

  We will split the discussion of scalar \nameref{def:Variable}s into 4 categories, according to their lifetimes.
  \begin{itemize}[noitemsep]
  \item \nameref{par:Static_Variable_Binding_Lifetime}
  \item \nameref{par:Stack-Dynamic_Variable_Binding_Lifetime}
  \item \nameref{par:Explicit_Heap-Dynamic_Variable_Binding_Lifetime}
  \item \nameref{par:Implicit_Heap-Dynamic_Variable_Binding_Lifetime}
  \end{itemize}
\end{definition}

\paragraph{Static Variables}\label{par:Static_Variable_Binding_Lifetime}
\begin{definition}[Static Variable]\label{def:Static_Variable_Binding_Lifetime}
  \emph{Static variable}s are those that are bound to \nameref{def:Memory} cells before program execution begins and remain bound until the program terminates.
\end{definition}

\nameref{def:Static_Variable_Binding_Lifetime}s can be used as globally accessible \nameref{def:Variable}s, or ensure that subprograms are history-sensitive.

The pros and cons of \nameref{def:Static_Variable_Binding_Lifetime}s are:
\begin{itemize}[noitemsep]
\item Pros
  \begin{itemize}[noitemsep]
  \item Efficiency. All \nameref{def:Memory} addressing is done with absolute addresses, making things very fast.
  \end{itemize}
\item Cons
  \begin{itemize}[noitemsep]
  \item Reduced flexibility. If there is a language that only has \nameref{def:Static_Variable_Binding_Lifetime}s, then recursive subprograms are impossible.
  \end{itemize}
\end{itemize}

\begin{remark*}
  In C and C++, \texttt{static} can be set on functions, making the \nameref{def:Variable}s declared in the function \nameref{def:Static_Variable_Binding_Lifetime}.
\end{remark*}

\begin{remark*}
  In Java, C++, and C\#, \texttt{static} can appear on classes, meaning class \nameref{def:Variable}s are created statically some time before the class is first instantiated.
\end{remark*}

\paragraph{Stack-Dynamic Variables}\label{par:Stack-Dynamic_Variable_Binding_Lifetime}
\begin{definition}[Stack-Dynamic Variable]\label{def:Stack-Dynamic_Variable_Binding_Lifetime}
  \emph{Stack-dynamic variable}s are those whose storage \nameref{def:Binding}s are created when their declaration statements are elaborated.
  These are allocated from the run-time stack.

  \begin{remark}
    These are the variables that are most commonly used.
  \end{remark}

  \begin{remark}
    In languages that allow for variable declaration anywhere in the function, like Java and C++, the \nameref{def:Stack-Dynamic_Variable_Binding_Lifetime}s may be bound to storage at the beginning of the block.
    In these cases, the \nameref{def:Variable} becomes visible at the declaration, but the storage \nameref{def:Binding} occurs when the block begins execution.
  \end{remark}
\end{definition}

\begin{definition}[Elaboration]\label{def:Variable_Storage_Binding_Elaboration}
  \emph{Elaboration} of a \nameref{def:Variable} declaration refers to the storage \nameref{def:Variable_Memory_Allocation} and \nameref{def:Binding} process indicated by the declaration, which takes place when execution reaches that code.

  This occurs at run time.
\end{definition}

The advantages and disadvantages of \nameref{def:Stack-Dynamic_Variable_Binding_Lifetime}s, compared to \nameref{def:Static_Variable_Binding_Lifetime}s, are:
\begin{itemize}[noitemsep]
\item Advantages
  \begin{itemize}[noitemsep]
  \item Allows for recursive subprograms that have local variables
  \item All subprograms can share the same memory space for their locals, allowing for a smaller memory footprint, by only having some variables bound to storage at once.
  \end{itemize}
\item Disadvantages
  \begin{itemize}[noitemsep]
  \item Runtime overhead of \nameref{def:Variable_Memory_Allocation} and \nameref{def:Variable_Memory_Deallocation}.
  \item Slower accessing of \nameref{def:Stack-Dynamic_Variable_Binding_Lifetime}s because of indirect addressing.
  \item Subprograms cannot be history-sensitive with just \nameref{def:Stack-Dynamic_Variable_Binding_Lifetime}s.
  \end{itemize}
\end{itemize}

\paragraph{Explicit Heap-Dynamic Variables}\label{par:Explicit_Heap-Dynamic_Variable_Binding_Lifetime}
\begin{definition}[Explicit Heap-Dynamic Variable]\label{def:Explicit_Heap-Dynamic_Variable_Binding_Lifetime}
  \emph{Explicit heap-dynamic variables} are namess (abstract) memory cells that are allocated and deallocated by explicit run-time instructions written by the programmer.
  These \nameref{def:Variable}s are allocated to and deallocated from the \nameref{def:Heap}.
  They can only be referenced through pointers or reference variables.

  The pointer/reference can only be created and returned by:
  \begin{itemize}[noitemsep]
  \item An operator (in C++), \texttt{new}
  \item A subprogram (in C), \texttt{malloc}
  \end{itemize}

  Some languages include ways to destroy these pointers/references:
  \begin{itemize}[noitemsep]
  \item An operator (in C++), \texttt{delete}
  \item A subprogram (in C), \texttt{free}
  \end{itemize}
\end{definition}

An example of an \nameref{def:Explicit_Heap-Dynamic_Variable_Binding_Lifetime} is shown below.
\begin{minted}[frame=lines,linenos]{c++}
int *intnode; // Create a pointer
intnode = new int; // Create the heap-dynamic variable
...
delete intnode; // Deallocate the heap-dynamic variable to which intnode points
\end{minted}

The \nameref{def:Heap} is highly disorganized because of the unpredictability of its use.
Garbage collection can help organize it.

There are 2 ways to manage the \nameref{def:Heap}.
\begin{enumerate}[noitemsep]
\item Explicit \nameref{def:Variable_Memory_Deallocation}
  \begin{itemize}[noitemsep]
  \item The programmer must explicitly free the \nameref{def:Memory} themselves.
  \end{itemize}
\item Implicit \nameref{def:Variable_Memory_Deallocation}
  \begin{itemize}[noitemsep]
  \item The programming language has facilities, called \emph{garbage collection} that automatically manages the \nameref{def:Heap}.
  \end{itemize}
\end{enumerate}

The advantages and disadvantages of these types of \nameref{def:Variable}s are:
\begin{itemize}[noitemsep]
\item Advantages
  \begin{itemize}[noitemsep]
  \item \nameref{def:Explicit_Heap-Dynamic_Variable_Binding_Lifetime} are often used to construct dynamic data structures, like linked lists and trees. These are built conveniently using pointers and data.
  \end{itemize}
\item Disadvantages
  \begin{itemize}[noitemsep]
  \item The difficulty of using pointer/reference variables correctly
  \item The \nameref{subsec:Cost} of using these pointers/reference variables
  \item Complexity of the required storage management implementation (Although, this is a question of \nameref{def:Heap} management, which is costly and complicated).
  \end{itemize}
\end{itemize}

\paragraph{Implicit Heap-Dynamic Variables}\label{par:Implicit_Heap-Dynamic_Variable_Binding_Lifetime}
\begin{definition}[Implicit Heap-Dynamic Variable]\label{def:Implicit_Heap-Dynamic_Variable_Binding_Lifetime}
  \emph{Implicit heap-dynamic variable}s are bound to \nameref{def:Heap} storage onnly when they are assigned values.
\end{definition}

The advantages and disadvantages of these types of \nameref{def:Variable}s are:
\begin{itemize}[noitemsep]
\item Advantages
  \begin{itemize}[noitemsep]
  \item Highest degree of flexibility, allowing for highly generic code
  \end{itemize}
\item Disadvantages
  \begin{itemize}[noitemsep]
  \item Run time overhead of maintaining all the dynamic attributes, which could include subscript types and ranges
  \item Loss of some error dectection by the compiler/interpreter
  \end{itemize}
\end{itemize}
%%% Local Variables:
%%% mode: latex
%%% TeX-master: "../EDAP05-Concepts_Programming_Languages-Reference_Sheet"
%%% End:
