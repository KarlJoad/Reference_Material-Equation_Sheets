\subsection{Design Issues for Abstract Data Types}\label{subsec:ADT_Design_Issues}
\begin{itemize}[noitemsep]
\item The \nameref{def:Abstract_Data_Type} name must be externally visible, to allow for object creation
\item The \nameref{def:Abstract_Data_Type} representation must be hidden.
\item There are few built-in operations by a language for \nameref{def:Abstract_Data_Type} operations
  \begin{itemize}[noitemsep]
  \item If there are some defined, they are the most basic ones: assignment, comparison.
  \item Overloading of subprograms should be allowed
  \end{itemize}
\item The form of the container for the interface to the \nameref{def:Abstract_Data_Type}.
\item Whether \nameref{def:Abstract_Data_Type} can be parameterized.
\item What access controls are provided, and how are such controls specified?
\item Is the specification of the \nameref{def:Abstract_Data_Type} physically separate from its implementation?
\end{itemize}

\subsubsection{Specification of Abstract Datatypes}\label{subsubsec:Specification_Abstract_Datatype}
\begin{definition}[Specification]\label{def:ADT_Specification}
  We need a \emph{specification} of the \nameref{def:Abstract_Data_Type} to ensure that the it is implemented with the correct functionality and that it operates as expected.
  We also need to specify the actions that are required for the \nameref{def:Abstract_Data_Type} in the \nameref{def:ADT_Interface}.
\end{definition}

For example, if we have an \nameref{def:ADT_Interface} of an \texttt{IntVector}, that supports the following operations:
\begin{itemize}[noitemsep]
\item create : \texttt{\SemanticType{int}} $\rightarrow$ \texttt{\SemanticType{IntVector}}
\item length : \texttt{\SemanticType{IntVector}} $\rightarrow$ \texttt{\SemanticType{int}}
\item append : \texttt{(\SemanticType{IntVector}, \SemanticType{int})} $\rightarrow$ \texttt{()}
\item get : \texttt{(\SemanticType{IntVector}, \SemanticType{int})} $\rightarrow$ \texttt{\SemanticType{int}}
\end{itemize}

In a code example, we can write our \nameref{def:ADT_Interface} as a Rust Trait.
\inputminted[frame=lines,linenos]{rust}{./EDAP05-Concepts_Programming_Languages-Sections/Abstract_Data_Types/Code/IntVector_Interface.rs}

If we have no specification, we could implement \texttt{IntVector} in Python like so;
\inputminted[frame=lines,linenos]{python3}{./EDAP05-Concepts_Programming_Languages-Sections/Abstract_Data_Types/Code/IntVector_No_Spec.py}

However, this is a completely useless \nameref{def:ADT_Implementation}.
We need a \nameref{def:ADT_Specification} to ensure that the \nameref{def:ADT_Implementation} makes sense.
\begin{itemize}[noitemsep]
\item If $\ell = \mathtt{length}(v)$, and we call $\mathtt{append}(v, x)$ once, then afterwards $\mathtt{length}(v) = \ell + 1$.
\item $\mathtt{get}(\mathtt{create}(\ell), i) = 0$ if $i \in \lbrace 0, 1, \ldots, \ell-1 \rbrace$, otherwise, there is an error.
\end{itemize}

Using this \nameref{def:ADT_Specification}, we can make an \nameref{def:ADT_Implementation} that behaves the way we expect it to, which will allow the programmer to use the \nameref{def:ADT_Interface} safely, without knowing the actual implementation details.

%%% Local Variables:
%%% mode: latex
%%% TeX-master: "../../EDAP05-Concepts_Programming_Languages-Reference_Sheet"
%%% End:
