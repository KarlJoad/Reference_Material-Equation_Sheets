\subsection{Variables}\label{subsec:Variables}
\begin{definition}[Variable]\label{def:Variable}
  A program \emph{variable} is an abstraction of a computer \nameref{def:Memory} cell or a collection of \nameref{def:Memory} cells.
  A variable can be characterized by a sextuple of attributes:
  \begin{enumerate}[noitemsep]
  \item \nameref{subsubsec:Variable_Name}
  \item \nameref{subsubsec:Variable_Address}
  \item \nameref{subsubsec:Variable_Value}
  \item \nameref{subsubsec:Variable_Type}
  \item \nameref{subsubsec:Storage_Bindings_and_Lifetime}
  \item \nameref{subsec:Variable_Scope}
  \end{enumerate}
\end{definition}

\subsubsection{Name}\label{subsubsec:Variable_Name}
Most \nameref{def:Variable}s have names.
These are symbolic references to the value that is actually stored.
There are various issues that may arise with the name of a variable, which were discussed earlier.

\subsubsection{Address}\label{subsubsec:Variable_Address}
\begin{definition}[Address]\label{def:Variable_Address}
  The \emph{address} of a \nameref{def:Variable} is the machine's memory address with which the \nameref{def:Variable} is associated.

  The address of a variable is sometimes called its \emph{L-Value}.
  This is because the address is required when the name of a variable appears on the left-hand side of an assignment statement.

  \begin{remark}[Alias]
    An \emph{alias} is having another \nameref{def:Variable} have the same \nameref{def:Variable_Address}, so the 2 \nameref{def:Variable}s point to the same value in \nameref{def:Memory}.
  \end{remark}
\end{definition}

For some languages, it is possible for the same \nameref{def:Variable} to be associated with different addresses at different times during the \nameref{def:Variable}'s lifetime.

\subsubsection{Type}\label{subsubsec:Variable_Type}
\begin{definition}[Type]\label{def:Variable_Type}
  The \emph{type} of a \nameref{def:Variable} determines the range of values that \nameref{def:Variable} can store.
  For example, the \texttt{int} type in Java specifies a value range of $-2147483648$ to $2147483647$.
  It is a 32-bit signed integer.
\end{definition}

\subsubsection{Value}\label{subsubsec:Variable_Value}
\begin{definition}[Value]\label{def:Variable_Value}
  The \emph{value} of a \nameref{def:Variable} is the contents of the \nameref{def:Memory} cell or cells associated with the \nameref{def:Variable}.
  The value of a variable is sometimes called it \emph{R-Value}.
  This is because the value of the \nameref{def:Variable} is required on the right-hand side of an assignment statement.
  To access the \textit{r}-value, the \textit{l}-value must be determined first.
  
  \begin{remark}[Abstract Memory Cells]\label{rmk:Abstract_Memory_Cells}
    While in hardware, the individual sizes of \nameref{def:Memory} are fixed, we can think of \nameref{def:Memory} as having \emph{abstract memory cells}, that can accomodate anything we attempt to put into \nameref{def:Memory}.
    This means that a single-precision floating point number technically takes up 4 bytes, 32 bits, of \nameref{def:Memory} cells, that number only takes one abstract memory cell.
  \end{remark}
\end{definition}

%%% Local Variables:
%%% mode: latex
%%% TeX-master: "../../EDAP05-Concepts_Programming_Languages-Reference_Sheet"
%%% End:
