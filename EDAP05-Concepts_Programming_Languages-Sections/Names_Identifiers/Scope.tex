\subsection{Scope}\label{subsec:Variable_Scope}
\begin{definition}[Scope]\label{def:Variable_Scope}
  The \emph{scope} of a \nameref{def:Variable} is the range of statements in which the \nameref{def:Variable} is \nameref{def:Visible_Variable}.
\end{definition}

\nameref{def:Variable_Scope} might seem similar to \nameref{def:Variable_Lifetime}, but they are different.
Here are 2 examples that illustrate this point:
\begin{enumerate}[noitemsep]
\item At the second \texttt{print(x)}, \texttt{x} is \emph{in scope} (visible), but is \emph{dead} (deallocated).
\inputminted[frame=lines,linenos]{c}{./EDAP05-Concepts_Programming_Languages-Sections/Names_Identifiers/Code/Variable_InScope_Dead.c}
  
\item When executing inside of \texttt{g(y)}, the \texttt{x} in function \texttt{f} is \emph{alive} (allocated), but is \emph{out of scope} (not visible).
\inputminted[frame=lines,linenos]{python3}{./EDAP05-Concepts_Programming_Languages-Sections/Names_Identifiers/Code/Variable_OutScope_Alive.py}
\end{enumerate}

\begin{definition}[Visible]\label{def:Visible_Variable}
  A \nameref{def:Variable} is \emph{visible} in a statements if it can be referenced in that statement.

  \begin{remark}[In-Scope]\label{rmk:In_Scope_Variable}
    Sometimes, a \nameref{def:Variable} that is \nameref{def:Visible_Variable} is called \emph{in-\nameref{def:Variable_Scope}}.
  \end{remark}
\end{definition}

\begin{definition}[Local Variable]\label{def:Local_Variable}
  A \nameref{def:Variable} is a \emph{local variable} in a program unit or block if it is declared there.
  \nameref{def:Variable}s defined within subprograms are also local variables.
  The \nameref{def:Variable_Scope} is usually the body of the subprogram in which they are defined.

  \begin{remark}[Storage Binding]
    If the \nameref{def:Local_Variable} is a \nameref{def:Stack-Dynamic_Variable_Binding_Lifetime}, it is bound to storage when the subprogram begins and unbound when that execution terminates.
  \end{remark}
\end{definition}

\begin{definition}[Nonlocal Variable]\label{def:Nonlocal_Variable}
  The nonlocal variables of a program are visible with that particular program unit or block, but are not declared there.

  \begin{remark}[Global Variables]
    Global variables are a special case of \nameref{def:Nonlocal_Variable}s.
    These are discussed in \Cref{subsubsec:Variable_Global_Scope}.
  \end{remark}
\end{definition}

\subsubsection{Static Scope}\label{subsubsec:Static_Scope}
\begin{definition}[Static Scoping]\label{def:Static_Scoping}
  \emph{Static scoping} is a way to statically determine the scope of a \nameref{def:Variable}.
  When there is a reference to a \nameref{def:Variable}, the attributes of the \nameref{def:Variable} can be determined by finding the statement in which it is declared (either explicitly or implicitly).
  This makes it easy for a human reader and compiler/interpreter to figure out the \nameref{def:Variable_Type} of every \nameref{def:Variable} in the program.

  There are 2 types of statically-scoped languages:
  \begin{enumerate}[noitemsep]
  \item Subprograms can be nested inside programs (Python)
  \item Subprograms cannot be nested inside programs (Java)
  \end{enumerate}

  In a brace ($\lbrace \rbrace$)-delimited programming language, like C, if 2 things with the same name are declared at the same nested scoping level, then they are siblings.
  In the code example below, \texttt{sub1()} and \texttt{sub2()} are functions that are static siblings.

  \begin{remark}[Lexical Scoping]\label{rmk:Variable_Lexical_Scoping}
    \nameref{def:Static_Scoping} is sometimes called \emph{lexical scoping}.
  \end{remark}
\end{definition}

\nameref{def:Static_Scoping} creates a tree-like structure, where each \nameref{def:Variable} declared in a program unit/block has a \nameref{def:Variable_Static_Parent}.
Then, each \nameref{def:Variable_Static_Parent} has a list of \nameref{def:Variable_Static_Ancestor}s.

\begin{definition}[Static Parent]\label{def:Variable_Static_Parent}
  If the \nameref{def:Variable} referenced is not present as a \nameref{def:Local_Variable}, then we have to go to the next outer program unit or block, the \emph{static parent}.
\end{definition}

\begin{definition}[Static Ancestor]\label{def:Variable_Static_Ancestor}
  The \emph{static ancestor}s are all the \nameref{def:Variable_Static_Parent}s to that particular program unit/block.
  The static ancestor for functions would be the \nameref{def:Variable_Scope} that contains both functions.
\end{definition}

This is illustrated by finding \texttt{x} in \texttt{sub2()} in this JavaScript function.
\inputminted[frame=lines,linenos]{javascript}{./EDAP05-Concepts_Programming_Languages-Sections/Names_Identifiers/Code/Static_Scope.js}

The \texttt{x} in \texttt{sub2()} refers to the \texttt{x=3} in \texttt{big()}, because \texttt{sub1()} is not a \nameref{def:Variable_Static_Ancestor} of \texttt{sub2()}.
Even though \texttt{sub1()} calls \texttt{sub2()}, because they share the same nested \nameref{def:Variable_Scope}, the \texttt{x} in \texttt{sub2()} refers to the \mintinline{javascript}{var x = 3;}.
This is because both \texttt{sub1()} and \texttt{sub2()} are in the same \nameref{def:Variable_Scope}, namely the function \texttt{big()}'s scope.
However, inside of \texttt{sub1()}, the use of the variable \texttt{x} would refer to the \texttt{x=7} value, and never the \texttt{x=3} value.

In some languages, if a \nameref{def:Variable} is declared in a sub-program unit/block, like in \texttt{sub1()}, then preceding the \nameref{def:Variable} name with the outer program unit/block will give the outer \nameref{def:Variable} \nameref{def:Variable_Value}.

\subsubsection{Dynamic Scope}\label{subsubsec:Dynamic_Scope}
\begin{definition}[Dynamic Scoping]\label{def:Dynamic_Scoping}
  \emph{Dynamic scoping} is based on the calling sequence of subprograms, and not their spatial relationship to each other.
  Thus, the scope can only be determined at run time.

  Some languages that implement this are:
  \begin{itemize}[noitemsep]
  \item APL
  \item SNOBOL4
  \item Early LISP
  \item Perl (Allowed, but must be said explicitly)
  \item Common LISP (Allowed, but must be said explicitly)
  \end{itemize}
\end{definition}

To illustrate \nameref{def:Dynamic_Scoping}, look at the code block below, and assume it is in a language that uses \nameref{def:Dynamic_Scoping}.
\inputminted[frame=lines,linenos]{javascript}{./EDAP05-Concepts_Programming_Languages-Sections/Names_Identifiers/Code/Dynamic_Scope.js}

The call to \texttt{sub1()} by \texttt{big()}, which then calls \texttt{sub2()}.
In that running of \texttt{sub2()}, the use of \texttt{x} cannot be determined locally (within that function).
Thus, it goes to its dynamic parent, \texttt{sub1()} and finds a declaration for \texttt{x}.
Thus, the \texttt{y} in \texttt{sub2()} evaluates to \texttt{y=7}.

The next instruction, the call to \texttt{sub2()} by \texttt{big()}, produces a different result.
In that running of \texttt{sub2()}, the use of \texttt{x} cannot be determined locally (within that function).
Thus, it goes to its dynamic parent, \texttt{big()} and finds a declaration for \texttt{x}.
Thus, the \texttt{y} in \texttt{sub2()} evaluates to \texttt{y=3}.

\subsubsection{Blocks}\label{subsubsec:Variable_Blocks}
New \nameref{subsubsec:Static_Scope}s can be defined in the middle of executing code.
This allows a small section to have its own \nameref{def:Local_Variable}s.

\begin{definition}[Block]\label{def:Block_Scope}
  A \emph{block} is a section of code that has its own \nameref{def:Local_Variable}s.
  These \nameref{def:Local_Variable}s are \textbf{not} shared with any \nameref{def:Variable_Static_Ancestor}s.
\end{definition}

The use of \nameref{def:Block_Scope}s create a \nameref{def:Block_Structured_Language}.

\begin{definition}[Block-Structured Language]\label{def:Block_Structured_Language}
  The use of \nameref{def:Block_Scope}s to create the \nameref{subsubsec:Static_Scope}s creates a \emph{block-structured language}.
\end{definition}

Consider the following C function:
\begin{minted}[frame=lines,linenos]{c}
void sub() {
  int count;
  ...
  while (...) {
    int count;
    count++;
  }
  ...
}
\end{minted}
The \texttt{count} inside the \texttt{while} loop is that loop's local count, and does not reference the \texttt{count} in \texttt{sub}.

\paragraph{Blocks in Functional Languages}\label{par:Variable_Block_in_Functional_Languages}
Since \nameref{def:Variable}s in \nameref{def:Functional_Programming_Language}s actually evaluate and store expressions, they behave differently.
Each functional language handles this differently, so you will have to look at the language specification to find out exactly how \nameref{def:Variable}s are scoped.

\subsubsection{Declaration Order}\label{subsubsec:Variable_Declaration_Order}
Some languages require that all \nameref{def:Variable} declarations occur at the beginning of a function (C89).

In some languages, \nameref{def:Variable}s cannot used before they have been declared.
\begin{itemize}[noitemsep]
\item Some of these languages allow for \nameref{def:Variable}s to be declared anywhere in the function, but can only be referenced \textbf{after} their declaration until the end of their scope.
\item Some of these languages allow for \nameref{def:Variable}s to be declared anywhere in the function, but if used before their declaration, they use a value like \texttt{undefined} (JavaScript).
\end{itemize}

This is a highly language-dependent thing, and one must consult with the language specification to figure out exactly how it works.

\subsubsection{Global Scope}\label{subsubsec:Variable_Global_Scope}
\begin{definition}[Global Variable]\label{def:Global_Variable}
  These are usually \nameref{def:Variable}s that sit outside of all functions.
  They can be accessed from anywhere in the program.
  They can also be defined and/or declared in other files in the program's project.
\end{definition}

It important to note the difference between a \nameref{def:Global_Variable}'s declaration and definition.
\begin{itemize}[noitemsep]
\item Declaration: The \nameref{def:Variable_Type}s and attributes are bound, but the \nameref{def:Memory} space required is \textbf{not} allocated.
\item Definition: The \nameref{def:Variable_Type}s and attributes are bound, but the \nameref{def:Memory} space required \textbf{is} allocated.
\end{itemize}

\begin{remark*}
  This is a highly language-dependent thing, and one must consult with the language specification to figure out exactly how it works.
\end{remark*}

%%% Local Variables:
%%% mode: latex
%%% TeX-master: "../../EDAP05-Concepts_Programming_Languages-Reference_Sheet"
%%% End:
