\section{Functional Programming}\label{sec:Functional_Programming}
Recall the definition of a \nameref{def:Functional_Programming_Language}.
\defFunctionalProgrammingLanguage*{}

But, in short, these types of languages rely on the \emph{evaluation of expressions}, rather than the execution of commands.

\begin{large}
  \textbf{In this section, the functional language used is Standard ML.}
\end{large}

\subsection{\nameref*{def:Function_Side_Effect}s in Functional Languages}\label{subsec:Functional_Function_Side_Effects}
In a strict definition of \nameref{def:Functional_Programming_Language}s, \nameref{def:Function_Side_Effect}s are not allowed.
However, \nameref{def:Function_Side_Effect}s include:
\begin{itemize}[noitemsep]
\item Writing a file to disk
\item Sending packets over the network
\item Writing to standard output
\end{itemize}

However, if a functional language did not include these, the language is severely hampered in its usability.\footnote{Some functional languages are like this, but they are not widely used, so are not discussed.}
To get ``around'' this issue, there are 3 ways functional languages can handle these non-pure actions.
\begin{enumerate}[noitemsep]
\item The language was \emph{imperative first}.
  These are languages that:
  \begin{itemize}[noitemsep]
  \item Were originally \nameref{def:Imperative_Programming_Language}s, like C, C++, Java
  \item Can be adapted to behave like functional languages if they are written correctly.
  \end{itemize}
\item The language was \emph{functional first}.
  These are languages that:
  \begin{itemize}[noitemsep]
  \item Are by default true \nameref{def:Functional_Programming_Language}s, like Haskell, F\#, the ML-family
  \item Allow \nameref{def:Function_Side_Effect}s to be contained within the functions themselves, and not outside them.
  \item Has the compiler track these functions with \nameref{def:Function_Side_Effect}s so they are handled correctly.
  \end{itemize}
\item The language is \emph{monadic}.
  There is only a single language that supports this model right now, Haskell.
  A monad is:
  \begin{itemize}[noitemsep]
  \item An abstraction or generalization over ordered things
  \item These things may have side-effects, which are tagged.
  \item Monads allow for an imperative-like state to be introduced to a functional environment. However, the state exists \emph{ONLY} for the monad's function(s) and its/their children.
  \item Monads may also allow for state to be present throughout a program, but permission to access the state must be passed to functions, like an \nameref{def:Actual_Parameter}.
  \end{itemize}
\end{enumerate}

\subsection{Standard ML}\label{subsec:Functional-Standard_ML}
\begin{definition}[Standard ML]\label{def:Standard_ML}
  \emph{Standard ML}, \emph{SML}, \emph{Standard Meta Language} is a general-purpose, modular, \nameref{def:Functional_Programming_Language} with compile-time type checking and type inference.
  It uses \nameref{def:Pass_By_Value} for its evaluation of \nameref{def:Actual_Parameter}s to functions.
\end{definition}

Computation in Standard ML consists of sequential evaluation of expressions.
Each expression has 3 characteristics:
\begin{enumerate}[noitemsep]
\item It may or may not have a \emph{\nameref{def:Data_Type}}.
\item It may or may not have a \emph{\nameref{def:Variable_Value}}.
\item It may or may not have an \emph{effect}. These include:
  \begin{itemize}[noitemsep]
  \item Raising an exception
  \item Modifying memory
  \item Performing I/O
  \item Sending a message on a network
  \end{itemize}
\end{enumerate}

Every expression is \textbf{required} to have at least one \nameref{def:Data_Type}, making them \emph{well-typed}.
If an expression does \textbf{not} have a \nameref{def:Data_Type}, it is called \emph{ill-typed}, and cannot be evaluated.
The \emph{type checker} is in charge of checking the types of all expressions.

Well-typed expressions can be evaluated to determine their value.
This evaluation can be interrupted by an exception and/or a run-time fault.

\begin{definition}[Soundness]\label{def:Functional_Lang-Soundness}
  The \emph{soundness} of the \nameref{def:Data_Type} system ensures the accuracy of predictions made by the type checker.
\end{definition}

\subsubsection{Type Checking}\label{subsubsec:Functional-SML-Type_Checking}
A type in \nameref{def:Standard_ML} requires 3 things:
\begin{enumerate}[noitemsep]
\item A \emph{name} for the type, for later uses.
\item The \emph{values} that the type can have.
\item The \emph{operations} that may be performed with this type.
\end{enumerate}

There are several basic \nameref{def:Data_Type}s in \nameref{def:Standard_ML}.
\begin{enumerate}[noitemsep]
\item \texttt{int}, corresponding to integer numbers
\item \texttt{real}, corresponding to floating-point numbers
\item \texttt{char}, denoted \texttt{\#``c''}
\item \texttt{string}, denoted \texttt{``string''}
\item \texttt{bool}, either \texttt{true} or \texttt{false}
\end{enumerate}

\subsubsection{\texttt{If}-Statements in Standard ML}\label{subsubsec:Functional-SML-If_Statement}
The \texttt{if}-statement syntax is shown below.
\begin{minted}[frame=lines,linenos]{sml}
if conditionalExp then exp1 else exp2
\end{minted}

There are a few things to note.
The \texttt{conditionalExp} must evaluate to an expression of a \texttt{bool} type.
The expressions \texttt{exp1} and \texttt{exp2} \textbf{MUST} have the same type.
However, even if both branches have the same type, one branch can still raise raise an exception during runtime.

\subsubsection{Declarations in Standard ML}\label{subsubsec:Functional-SML-Declarations}
By default, both variables and types have global scope.
However, \nameref{def:Standard_ML} has introduced methods to limit the scope of things.
These are described more fully in \Cref{par:Functional-SML-Limit_Scope}.

\paragraph{Variables}\label{par:Functional-SML-Variable_Declarations}
Variables in \nameref{def:Standard_ML} are slightly different than \nameref{def:Variable}s in \nameref{def:Imperative_Programming_Language}s.
In \nameref{def:Standard_ML}, variables maintain the same value throughout their life.
If a variable is rebound, then a whole new memory cell is created for the new value and the old discarded.
For example,
\begin{minted}[frame=lines,linenos]{sml}
val a : int = 5;
val a = a + 1; (* This creates a whole NEW "a" to use *)
(* Create 2 new values at once *)
val a : int = 6 and b : real = 3.14;
\end{minted}

It is important to note that variables are \emph{bound-by-value} in \nameref{def:Standard_ML}.
This means that the right-hand side is evaluated first, meaning that the rebinding of \texttt{a} in the previous example is expanded to \mintinline{sml}{val a : int = 5 + 1}.

\paragraph{Types}\label{par:Functional-SML-Type_Declarations}
Aliased \nameref{def:Data_Type}s can be created in \nameref{def:Standard_ML}.
For example,
\begin{minted}[frame=lines,linenos]{sml}
type float = real; (* Creates a new type "float" that has a range the same as the base "real" type *)
type count = int and average = real; (* Introduces 2 new types at once *)

(* There is one case that does not work, because 2 new types are being introduced AT ONCE. *)
type float = real and average = float;
\end{minted}

\paragraph{Limiting Scope}\label{par:Functional-SML-Limit_Scope}

\subsubsection{Functions in Standard ML}\label{subsubsec:Functional-SML-Functions}

\subsubsection{Pattern Matching in Standard ML}\label{subsubsec:Functional-SML-Pattern_Matching}
\begin{definition}[Pattern Matching]\label{def:Pattern_Matching}
  
\end{definition}

\subsubsection{\nameref*{def:Data_Type} Inferencing}\label{subsubsec:Functional-SML-Data_Type_Inferencing}
\begin{definition}[\nameref*{def:Data_Type} Inferencing]\label{def:Functional-SML-Data_Type_Inferencing}
  
\end{definition}

%%% Local Variables:
%%% mode: latex
%%% TeX-master: "../EDAP05-Concepts_Programming_Languages-Reference_Sheet"
%%% End:
