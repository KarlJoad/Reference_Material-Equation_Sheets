\subsection{Introduction}\label{subsec:OOP_Intro}
\begin{definition}[Object-Oriented Programming]\label{def:Object_Oriented_Programming}
  \emph{Object-oriented programming} (\emph{OOP}) is a programming paradigm designed around the manipulation of \nameref{def:Object}s that accurately simulate real-life situations.
  These languages must provide 3 key language features:
  \begin{enumerate}[noitemsep]
  \item \nameref{def:Abstract_Data_Type}
  \item \nameref{def:OOP_Inheritance}
  \item Dynamic binding of method calls to methods.
  \end{enumerate}

  The use of an object-oriented language allows for \nameref{def:Abstract_Data_Type}s to be reused.
  This allows for programmers to reuse large portions of code and only having to change the parts that are necessary.
  This increases productivity and increases \nameref{subsec:Reliability}.
\end{definition}

%%% Local Variables:
%%% mode: latex
%%% TeX-master: "../../EDAP05-Concepts_Programming_Languages-Reference_Sheet"
%%% End:
