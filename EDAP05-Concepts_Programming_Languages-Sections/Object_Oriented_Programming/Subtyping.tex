\subsection{Subtyping in Object-Oriented Languages}\label{subsec:Subtyping_Object_Oriented_Languages}
This is a continuation from \Cref{subsec:Subtype_Polymorphism}.

From earlier, the 3 things that \nameref{def:Object_Oriented_Programming} must support are:
\begin{enumerate}[noitemsep]
\item Dynamic method binding or \emph{\nameref{def:OOP_Dynamic_Dispatch}}
\item \nameref{def:OOP_Inheritance}
\item Support for \nameref{def:Abstract_Data_Type}s
\end{enumerate}

The subtyping rules for methods follow the subtyping rules for subroutines, so the following overriding between classes C and D is perfectly safe:
\inputminted[frame=lines,linenos]{java}{./EDAP05-Concepts_Programming_Languages-Sections/Object_Oriented_Programming/Code/MethodSubtyping.java}

One complication here is the implicit \texttt{self} or \texttt{this} reference (or pointer, in C++).
This self-reference allows methods to access their own state, so its type is always fixed to be a subtype of the type of the declaring class.

This self-reference is technically a ‘hidden parameter’ to each method, but since a method can only be called on an object whose dynamic type is a subtype of the class in which that method was defined, it is safe for us to use this more precise type information.

%%% Local Variables:
%%% mode: latex
%%% TeX-master: "../../EDAP05-Concepts_Programming_Languages-Reference_Sheet"
%%% End:
