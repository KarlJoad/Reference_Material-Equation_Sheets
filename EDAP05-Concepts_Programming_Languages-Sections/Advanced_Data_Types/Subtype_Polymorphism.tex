\subsection{Subtype Polymorphism}\label{subsec:Subtype_Polymorphism}
\begin{definition}[Subtype Polymorphism]\label{def:Subtype_Polymorphism}
  \emph{Subtype polymorphism} can be summarized with the following phrase; ``This value has a \nameref{def:Data_Type}, and while you don't know what exact type it is, this \nameref{def:Data_Type} is a more restricted (or general) version of another \nameref{def:Data_Type} that you \textit{do} know, so therefore there are some things that you can do with it''.
\end{definition}

\begin{definition}[Subtype]\label{def:Subtype}
  A \nameref{def:Data_Type} $\SemanticType{T}$ is a \emph{subtype} of type $\SemanticType{U}$, denoted $\SemanticType{T} <: \SemanticType{U}$ or $\SemanticType{U} :> \SemanticType{T}$, if any value $v : \SemanticType{T}$ can be used in any context that requires a value of type $\SemanticType{U}$.

  Some examples of subtypes are:
  \begin{itemize}[noitemsep]
  \item $\mathtt{\SemanticType{[2 TO 5]}} <: \mathtt{\SemanticType{[2 TO 6]}}$
  \item $\mathtt{\SemanticType{[2 TO 5]}} <: \mathtt{\SemanticType{[1 TO 5]}}$
  \item $\mathtt{\SemanticType{[2 TO 5]}} <: \mathtt{\SemanticType{[1 TO 6]}}$
  \item $\mathtt{\SemanticType{[2 TO 5]}} <: \mathtt{INTEGER}$
  \end{itemize}

  This is because the supertype contains all possible values that the subtype can take, and more.
  These are all based on subset relations, meaning that typing inherits a few properties.
  \begin{propertylist}
  \item Subtyping is \emph{reflexive}. Each type \SemanticType{T} is a subtype and supertype of itself, i.e.\ $\mathtt{\SemanticType{T}} <: \mathtt{\SemanticType{T}}$.\label{prop:Subtype_Reflexive}
  \item Subtyping is \emph{transitive}. Meaning if we know $\mathtt{\SemanticType{T}} <: \mathtt{\SemanticType{U}}$ and $\mathtt{\SemanticType{U}} <: \mathtt{\SemanticType{V}}$, then $\mathtt{\SemanticType{T}} <: \mathtt{\SemanticType{V}}$ is true.\label{prop:Subtype_Transitive}
  \end{propertylist}
\end{definition}

\subsubsection{Typing Conversions}\label{subsubsec:Typing_Conversions}
\begin{definition}[Widening Conversion]\label{def:Widening_Conversion}
  Whenever we use a value of a \nameref{def:Subtype} in a place that expects a supertype, the language must perform a \emph{widening conversion}.

  \begin{remark}[Implicit]\label{rmk:Widening_Conversion_Implicit}
    Most languages make this an implicit operation, because no information is lost, making it a type-safe operation.
  \end{remark}
\end{definition}

\begin{definition}[Narrowing Conversion]\label{def:Narrowing_Conversion}
  Some languages also allow us to translate a supertype to a subtype, using a \emph{narrowing conversion}.

  \begin{remark}[Explicit]\label{rmk:Narrowing_Conversion_Explicit}
    The languages that support \nameref{def:Narrowing_Conversion}s make it an explicit command.
    This instruction may fail, or may lead to information loss, meaning it is not type-safe.
    So the language designers tend to make it an explicit operation.
  \end{remark}
\end{definition}

\subsubsection{Variance of Types}\label{subsubsec:Type_Variance}
\textbf{TODO!! Give Example of each, with reasoning!}
Using a \texttt{\SemanticType{Box}} Scala trait as the basis of our discussion, the code is shown below.
\inputminted[frame=lines,linenos]{scala}{./EDAP05-Concepts_Programming_Languages-Sections/Advanced_Data_Types/Code/Box.scala}

\begin{definition}[Covariance]\label{def:Type_Covariance}
  Let $\DataType$ be a type constructor with formal type parameters $\DataType_{1}, \ldots , \DataType_{k}$, such that $\SemanticType{T} = \DataType [\DataType_{1}, \ldots, \DataType_{k}]$ is a type.
  Let $i \in \lbrace 1,\ldots, k \rbrace$.

  If for all $\DataType_{i}' <: \DataType_{i}$ we can always substitute a value of type $\DataType [\DataType_{1}',\ldots, \DataType_{i}', \ldots, \DataType_{k}']$ in a context that expects a value of type $\DataType [\DataType_{1}, \ldots, \DataType_{i}, \ldots, \DataType_{k}]$ without violating type preservation then $\DataType_{i}$ is \emph{covariant} in $\SemanticType{T}$.

  Using the \SemanticType{Box} example from above, we can create an \emph{covariant} Scala trait by writing something like this,
  \inputminted[frame=lines,linenos]{scala}{./EDAP05-Concepts_Programming_Languages-Sections/Advanced_Data_Types/Code/CovariantBox.scala}
  When we vary the type parameter \SemanticType{T} towards a subtype, the type of \SemanticType{ReadBox[T]} also varies towards that of a subtype; we say that \SemanticType{T} is \textbf{covariant}.
\end{definition}

\begin{definition}[Contravariance]\label{def:Type_Contravariance}
  Let $\DataType$ be a type constructor with formal type parameters $\DataType_{1}, \ldots , \DataType_{k}$, such that $\SemanticType{T} = \DataType [\DataType_{1}, \ldots, \DataType_{k}]$ is a type.
  Let $i \in \lbrace 1,\ldots, k \rbrace$.

  If for all $\DataType_{i}' :> \DataType_{i}$ we can always substitute a value of type $\DataType [\DataType_{1}',\ldots, \DataType_{i}', \ldots, \DataType_{k}']$ in a context that expects a value of type $\DataType [\DataType_{1}, \ldots, \DataType_{i}, \ldots, \DataType_{k}]$ without violating type preservation then $\DataType_{i}$ is \emph{contravariant} in $\SemanticType{T}$.

  Using the \SemanticType{Box} example from above, we can create an \emph{contravariant} Scala trait by writing something like this,
  \inputminted[frame=lines,linenos]{scala}{./EDAP05-Concepts_Programming_Languages-Sections/Advanced_Data_Types/Code/ContravariantBox.scala}
  When we vary the type parameter \SemanticType{T} towards a subtype, the type of \SemanticType{WriteBox[T]} varies in the opposite direction, towards that of a supertype; we say that \SemanticType{T} is \textbf{contravariant}.
\end{definition}

\begin{definition}[Invariance]\label{def:Type_Invariance}
  Let $\DataType$ be a type constructor with formal type parameters $\DataType_{1}, \ldots , \DataType_{k}$, such that $\SemanticType{T} = \DataType [\DataType_{1}, \ldots, \DataType_{k}]$ is a type.
  Let $i \in \lbrace 1,\ldots, k \rbrace$.

  If $\DataType_{i}$ is neither covariant nor contravariant in $\SemanticType{T}$, then $\DataType_{i}$ is \emph{invariant} in $\SemanticType{T}$.

  Using the \SemanticType{Box} example from above, we can create an \emph{invariant} method by writing something like this,
  \inputminted[frame=lines,linenos]{scala}{./EDAP05-Concepts_Programming_Languages-Sections/Advanced_Data_Types/Code/InvariantBox.scala}
  \begin{itemize}[noitemsep]
  \item $\mathtt{\SemanticType{A}} :> \mathtt{\SemanticType{B}}$: For example, consider \texttt{A = INTEGER} and \texttt{B = [1 TO 10]}.
    In this case, \texttt{box.put(b)} is safe, as \texttt{Box[A]} can store any number. However, \texttt{box.get()} might now return the number 99, which does not fit into the variable \texttt{b}.
    \textbf{Thus, this option is not statically type-safe.}
  \item $\mathtt{\SemanticType{A}} <: \mathtt{\SemanticType{B}}$: For example, consider \texttt{A = [1 TO 10]} and \texttt{B = INTEGER}.
    In this case, \texttt{box.get()} works, but \texttt{box.put(b)} does not: \texttt{b} might be 99, which we cannot pass to an operation that only accepts \texttt{A} as parameter.
    \textbf{Thus, this option is not statically type-safe either.}
  \end{itemize}
\end{definition}

\begin{definition}[Bivariance]\label{def:Type_Bivariance}
  Let $\DataType$ be a type constructor with formal type parameters $\DataType_{1}, \ldots , \DataType_{k}$, such that $\SemanticType{T} = \DataType [\DataType_{1}, \ldots, \DataType_{k}]$ is a type.
  Let $i \in \lbrace 1,\ldots, k \rbrace$.

  If for all $\DataType_{i}' <: \DataType_{i}$ we can always substitute a value of type $\DataType [\DataType_{1}',\ldots, \DataType_{i}', \ldots, \DataType_{k}']$ in a context that expects a value of type $\DataType [\DataType_{1}, \ldots, \DataType_{i}, \ldots, \DataType_{k}]$ without violating type preservation \textbf{AND}, if for all $\DataType_{i}' :> \DataType_{i}$ we can always substitute a value of type $\DataType' [\DataType_{1}', \ldots, \DataType_{i}', \ldots, \DataType_{k}']$ in a context that expects a value of type $\DataType [\DataType_{1}, \ldots, \DataType_{i}, \ldots, \DataType_{k}]$ without violating type preservation then $\DataType_{i}$ is \emph{bivariant} in $\SemanticType{T}$.

  \begin{remark}
    If both the input and output allow for the type to be both broadened and narrowed, it is not terribly interesting to study.
    Thus, we will not be studying them in much detail in this class.
  \end{remark}
\end{definition}

\paragraph{Definition-Site Variance}\label{par:Definition_Site_Variance}
\begin{definition}[Definition-Site Variance]\label{def:Definition_Site_Variance}
  \emph{Declaration-site variance} means that we decide about the type variable's variance when we define our \nameref{def:Abstract_Data_Type}s and \nameref{def:OOP_Class}es.
\end{definition}

\paragraph{Use-Site Variance}\label{par:Use_Site_Variance}
\begin{definition}[Use-Site Variance]\label{def:Use_Site_Variance}
  \emph{Use-site variance} means that we decide about the type variable's variance when we declare an instance of our \nameref{def:Abstract_Data_Type}s and \nameref{def:OOP_Class}es.
\end{definition}

%%% Local Variables:
%%% mode: latex
%%% TeX-master: "../../EDAP05-Concepts_Programming_Languages-Reference_Sheet"
%%% End:


%%% Local Variables:
%%% mode: latex
%%% TeX-master: "../../EDAP05-Concepts_Programming_Languages-Reference_Sheet"
%%% End:
