\subsubsection{Variance of Types}\label{subsubsec:Type_Variance}
Using a \texttt{\SemanticType{Box}} Scala trait as the basis of our discussion, the code is shown below.
\inputminted[frame=lines,linenos]{scala}{./EDAP05-Concepts_Programming_Languages-Sections/Advanced_Data_Types/Code/Box.scala}

\begin{definition}[Covariance]\label{def:Type_Covariance}
  Let $\DataType$ be a type constructor with formal type parameters $\DataType_{1}, \ldots , \DataType_{k}$, such that $\SemanticType{T} = \DataType [\DataType_{1}, \ldots, \DataType_{k}]$ is a type.
  Let $i \in \lbrace 1,\ldots, k \rbrace$.

  If for all $\DataType_{i}' <: \DataType_{i}$ we can always substitute a value of type $\DataType [\DataType_{1}',\ldots, \DataType_{i}', \ldots, \DataType_{k}']$ in a context that expects a value of type $\DataType [\DataType_{1}, \ldots, \DataType_{i}, \ldots, \DataType_{k}]$ without violating type preservation then $\DataType_{i}$ is \emph{covariant} in $\SemanticType{T}$.

  Using the \SemanticType{Box} example from above, we can create an \emph{covariant} Scala trait by writing something like this,
  \inputminted[frame=lines,linenos]{scala}{./EDAP05-Concepts_Programming_Languages-Sections/Advanced_Data_Types/Code/CovariantBox.scala}
  When we vary the type parameter \SemanticType{T} towards a subtype, the type of \SemanticType{ReadBox[T]} also varies towards that of a subtype; we say that \SemanticType{T} is \textbf{covariant}.

  Namely, if $\SemanticType{A} <: \SemanticType{B}$ and $C[\SemanticType{\alpha}]$, and $C[A] <: C[B]$, then $\SemanticType{\alpha}$ is covariant.
\end{definition}

\begin{definition}[Contravariance]\label{def:Type_Contravariance}
  Let $\DataType$ be a type constructor with formal type parameters $\DataType_{1}, \ldots , \DataType_{k}$, such that $\SemanticType{T} = \DataType [\DataType_{1}, \ldots, \DataType_{k}]$ is a type.
  Let $i \in \lbrace 1,\ldots, k \rbrace$.

  If for all $\DataType_{i}' :> \DataType_{i}$ we can always substitute a value of type $\DataType [\DataType_{1}',\ldots, \DataType_{i}', \ldots, \DataType_{k}']$ in a context that expects a value of type $\DataType [\DataType_{1}, \ldots, \DataType_{i}, \ldots, \DataType_{k}]$ without violating type preservation then $\DataType_{i}$ is \emph{contravariant} in $\SemanticType{T}$.

  Using the \SemanticType{Box} example from above, we can create an \emph{contravariant} Scala trait by writing something like this,
  \inputminted[frame=lines,linenos]{scala}{./EDAP05-Concepts_Programming_Languages-Sections/Advanced_Data_Types/Code/ContravariantBox.scala}
  When we vary the type parameter \SemanticType{T} towards a subtype, the type of \SemanticType{WriteBox[T]} varies in the opposite direction, towards that of a supertype; we say that \SemanticType{T} is \textbf{contravariant}.

  Namely, if $\SemanticType{A} <: \SemanticType{B}$ and $C[\SemanticType{\alpha}]$, and $C[A] :> C[B]$, then $\SemanticType{\alpha}$ is contravariant.
\end{definition}

\begin{definition}[Invariance]\label{def:Type_Invariance}
  Let $\DataType$ be a type constructor with formal type parameters $\DataType_{1}, \ldots , \DataType_{k}$, such that $\SemanticType{T} = \DataType [\DataType_{1}, \ldots, \DataType_{k}]$ is a type.
  Let $i \in \lbrace 1,\ldots, k \rbrace$.

  If $\DataType_{i}$ is neither covariant nor contravariant in $\SemanticType{T}$, then $\DataType_{i}$ is \emph{invariant} in $\SemanticType{T}$.

  Using the \SemanticType{Box} example from above, we can create an \emph{invariant} method by writing something like this,
  \inputminted[frame=lines,linenos]{scala}{./EDAP05-Concepts_Programming_Languages-Sections/Advanced_Data_Types/Code/InvariantBox.scala}
  \begin{itemize}[noitemsep]
  \item $\mathtt{\SemanticType{A}} :> \mathtt{\SemanticType{B}}$: For example, consider \texttt{A = INTEGER} and \texttt{B = [1 TO 10]}.
    In this case, \texttt{box.put(b)} is safe, as \texttt{Box[A]} can store any number. However, \texttt{box.get()} might now return the number 99, which does not fit into the variable \texttt{b}.
    \textbf{Thus, this option is not statically type-safe.}
  \item $\mathtt{\SemanticType{A}} <: \mathtt{\SemanticType{B}}$: For example, consider \texttt{A = [1 TO 10]} and \texttt{B = INTEGER}.
    In this case, \texttt{box.get()} works, but \texttt{box.put(b)} does not: \texttt{b} might be 99, which we cannot pass to an operation that only accepts \texttt{A} as parameter.
    \textbf{Thus, this option is not statically type-safe either.}
  \end{itemize}

  Namely, if $\SemanticType{A} <: \SemanticType{B}$ and $C[\SemanticType{\alpha}]$, and $C[A] \neq C[B]$, then $\SemanticType{\alpha}$ is invariant.
\end{definition}

\begin{definition}[Bivariance]\label{def:Type_Bivariance}
  Let $\DataType$ be a type constructor with formal type parameters $\DataType_{1}, \ldots , \DataType_{k}$, such that $\SemanticType{T} = \DataType [\DataType_{1}, \ldots, \DataType_{k}]$ is a type.
  Let $i \in \lbrace 1,\ldots, k \rbrace$.

  If for all $\DataType_{i}' <: \DataType_{i}$ we can always substitute a value of type $\DataType [\DataType_{1}',\ldots, \DataType_{i}', \ldots, \DataType_{k}']$ in a context that expects a value of type $\DataType [\DataType_{1}, \ldots, \DataType_{i}, \ldots, \DataType_{k}]$ without violating type preservation \textbf{AND}, if for all $\DataType_{i}' :> \DataType_{i}$ we can always substitute a value of type $\DataType' [\DataType_{1}', \ldots, \DataType_{i}', \ldots, \DataType_{k}']$ in a context that expects a value of type $\DataType [\DataType_{1}, \ldots, \DataType_{i}, \ldots, \DataType_{k}]$ without violating type preservation then $\DataType_{i}$ is \emph{bivariant} in $\SemanticType{T}$.

  \begin{remark}
    If both the input and output allow for the type to be both broadened and narrowed, it is not terribly interesting to study.
    Thus, we will not be studying them in much detail in this class.
  \end{remark}

  Namely, if $\SemanticType{A} <: \SemanticType{B}$ and $C[\SemanticType{\alpha}]$, and $C[A] = C[B]$, then $\SemanticType{\alpha}$ is bivariant.
\end{definition}

\paragraph{Definition-Site Variance}\label{par:Definition_Site_Variance}
\begin{definition}[Definition-Site Variance]\label{def:Definition_Site_Variance}
  \emph{Declaration-site variance} means that we decide about the type variable's variance when we define our \nameref{def:Abstract_Data_Type}s and \nameref{def:OOP_Class}es.
\end{definition}

In Scala and C\#, type parameters are always invariant unless they are declared to be covariant or contravariant.
The \texttt{+} and \texttt{-} operators are needed to specify \nameref{def:Type_Covariance} and \nameref{def:Type_Contravariance}.
\inputminted[frame=lines,linenos]{scala}{./EDAP05-Concepts_Programming_Languages-Sections/Advanced_Data_Types/Code/Definition_Site_Variance.scala}

\paragraph{Use-Site Variance}\label{par:Use_Site_Variance}
\begin{definition}[Use-Site Variance]\label{def:Use_Site_Variance}
  \emph{Use-site variance} means that we decide about the type variable's variance when we declare an instance of our \nameref{def:Abstract_Data_Type}s and \nameref{def:OOP_Class}es.
\end{definition}

In Java, this is written with either an \texttt{extends} or \texttt{super} \nameref{def:Keyword}.
Assuming that $\mathtt{\SemanticType{B}} <: \mathtt{\SemanticType{A}}$:
\inputminted[frame=lines,linenos]{scala}{./EDAP05-Concepts_Programming_Languages-Sections/Advanced_Data_Types/Code/Use_Site_Variance.java}

Java will now prevent us from calling any method in \texttt{\SemanticType{covariantBox}} that has the type parameter in a contravariant position, so we can only call \texttt{covariantBox.get()}.
Analogously, for \texttt{\SemanticType{contravariantBox}}, we can only call the \texttt{put()} operation.
The \texttt{\SemanticType{invariantBox}} prohibits calls to either \texttt{put()} or \texttt{get()}; but if we had an operation that does not contain the type parameter of the \texttt{Box} type anywhere (e.g., a to-string operation of $type () \rightarrow \SemanticType{String})$, then we could still call that method.

\nameref{def:Use_Site_Variance} requires us to write more complex types, but allows us to re-use the same type definition (and the same implementations) for covariant, contravariant, bivariant, and invariant uses.

This is shown in the code below.
\inputminted[frame=lines,linenos]{java}{./EDAP05-Concepts_Programming_Languages-Sections/Advanced_Data_Types/Code/Variant_Java_Class.java}

%%% Local Variables:
%%% mode: latex
%%% TeX-master: "../../EDAP05-Concepts_Programming_Languages-Reference_Sheet"
%%% End:
