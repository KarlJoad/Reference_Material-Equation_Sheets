\subsection{Ad-Hoc Polymorphism}\label{subsec:Ad_Hoc_Polymorphism}
\begin{definition}[Ad-Hoc Polymorphism]\label{def:Ad_Hoc_Polymorphism}
  \emph{Ad-Hoc polymorphism} can be summarized by the following phrase; ``This value has a \nameref{def:Data_Type}, and you don't need to know what it is, but here are a few things you can do with it''.

  Depending on the language, this can be achieved with \nameref{def:Typeclass}es.
\end{definition}

To motivate this discussion, we need to refer back to the previous section on \nameref{subsec:Parametric_Polymorphism} and the code block below.
\inputminted[frame=lines,linenos]{rust}{./EDAP05-Concepts_Programming_Languages-Sections/Advanced_Data_Types/Code/Parametric_Polymorphic_Max.rs}
However, Rust will complain about this code, and suggest that we must add a \nameref{def:Type_Constraint} to bound the \nameref{def:Type_Parameter} \SemanticType{\texttt{T}} by a Rust trait.
This is because the \nameref{def:Data_Type} \SemanticType{\texttt{T}} could be \textbf{ANY} type present in the programming language/project.
For example, finding the maximum of a string is an ambiguous operation.
What does it mean to find the ``maximum string''?

The trait that is used to bound \SemanticType{\texttt{T}} is \SemanticType{\texttt{std::cmp::PartialOrd}}.
This means that \textbf{ANY} type \SemanticType{\texttt{T}} that we feed into the \texttt{max2} function \textbf{MUST} be bounded by \SemanticType{\texttt{std::cmp::PartialOrd}}, meaning the definition of relational operators (\texttt{<}, \texttt{>}, etc.) is defined.
Knowing this, we rewrite our \texttt{max2} code as shown below, and attempt to use it on something that does not have its type bounded.
\inputminted[frame=lines,linenos]{rust}{./EDAP05-Concepts_Programming_Languages-Sections/Advanced_Data_Types/Code/Parametric_Polymorphic_Max-Start_Typeclass.rs}

\begin{definition}[Type Constraint]\label{def:Type_Constraint}
  A \emph{type constraint} is a method of constraining a \nameref{def:Data_Type} to \textbf{ONLY} \nameref{def:Data_Type}s that have certain characteristics about them, functions defined, etc.
  This is used to bound the possible types that can be used by certain things to ensure that all the operations specifed in a portion of a program are defined for any possible \nameref{def:Data_Type} that can be fed in.
\end{definition}

\begin{definition}[Typeclass]\label{def:Typeclass}
  A \emph{typeclass} is a form of a \nameref{def:Type_Constraint}.
  It was first introduced by the Haskell programming language and has since been introduced to other programming languages as well.
  The name typeclass is language-agnostic, meaning other languages might call these something else.
  
  \begin{remark}[Rust Traits]\label{rmk:Rust_Trait}
    The Rust programming language calls \nameref{def:Typeclass}es \emph{trait}s.
    The traits used in Rust \textbf{are not the same} as the ones in Scala.
    They just so happen to have the same name.
  \end{remark}
  
  \begin{remark}[Typeclass/Class Confusion]
    \textbf{BE CAREFUL!}
    \nameref{def:Typeclass}es and \nameref{def:OOP_Class}es are \textbf{NOT} related to each other and are fundamentally different.
  \end{remark}
\end{definition}

\subsubsection{User-Defined Typeclasses}\label{subsubsec:User_Defined_Typeclasses}
Most programming languages that support \nameref{def:Typeclass}es allow for the programmer to specify their own \nameref{def:Typeclass}es for use in their programs to bound their \nameref{def:Data_Type}s.
In this section, we are going to construct another \texttt{max} function, but this one will use a greater-than operator that we define.
By creating a new operator for ourselves, we need to bound the possible inputs to the operator to ensure that it works for all the \nameref{def:Data_Type}s we expect to put into it.

%%% Local Variables:
%%% mode: latex
%%% TeX-master: "../../EDAP05-Concepts_Programming_Languages-Reference_Sheet"
%%% End:
