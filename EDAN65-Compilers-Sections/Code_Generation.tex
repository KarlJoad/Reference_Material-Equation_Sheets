\section{Code Generation}\label{sec:Code_Generation}
Code generation is the last step of the compilation process.
The assembly code is also optimized in this step, before the code is generated, in the \nameref{subsec:Intermediate_Code} portion.

\subsection{Intermediate Code}\label{subsec:Intermediate_Code}
There are 2 common forms of intermediate code generation:
\begin{enumerate}[noitemsep]
\item \nameref{subsubsec:Three_Address_Code}
\item \nameref{subsubsec:Stack_Code}
\end{enumerate}

\subsubsection{Three-Address Code}\label{subsubsec:Three_Address_Code}
The \nameref{subsubsec:Three_Address_Code} is the type of \nameref{subsec:Intermediate_Code} used for static, compiled languages.
An example of \nameref{subsubsec:Three_Address_Code} is shown below.
They have these elements on common:
\begin{itemize}[noitemsep]
\item Each instruction has 3 operands, in the form of \mint{nasm}{op src1 src2 dest}
\item \nameref{def:Temporary_Variable}s are frequently used
\item Very related to a register-based machine
\item \textbf{\emph{VERY GOOD FOR OPTIMIZATION}}
  \begin{itemize}[noitemsep]
  \item Removing unnecessary instructions for operations
  \item Keep as many values in registers rather than memory
  \end{itemize}
\end{itemize}

\begin{assemblysource}
    ADD v1 v2 t1
    JEQ t1 v3 L1
    SUB v3 1 t2
    MOV t2 c1
  L1:
    MOV v2 v4
\end{assemblysource}

\subsubsection{Stack Code}\label{subsubsec:Stack_Code}
The other type of \nameref{subsec:Intermediate_Code} is more commonly used for dynamic languages.
These have these elements in common:
\begin{itemize}[noitemsep]
\item The use of a \emph{value stack} instead of \nameref{def:Temporary_Variable}s
\item Each instruction needs the necessary operands popped, performs the operation, then pushes the result
\item Commonly used for interpreters and virtual machines
\item Similar to \nameref{subsec:LRParsing}
  \begin{itemize}[noitemsep]
  \item Push the operands/arguments onto the value stack
  \item Pop the appropriate number of arguments off the top of the value stack when performing an operation
  \item Perform the operation and push the result
  \end{itemize}
\end{itemize}

\begin{assemblysource}
  PUSH v1
  PUSH v2
  ADD
  PUSH v3
  JEQ L1
  PUSH v3
  PUSH 1
  SUB
  POP v1
  L1:
  PUSH v2
  POP v4
\end{assemblysource}

\subsection{Target Code Generation}\label{subsec:Target_Code_Generation}
The assembly code that we generate in this course is completely \textbf{\emph{UNOPTIMIZED}}.
Thus, the assembly code that we generated was nearly identical to the \nameref{subsubsec:Three_Address_Code}.

%%% Local Variables:
%%% mode: latex
%%% TeX-master: "../EDAN65-Compilers-Reference_Sheet"
%%% End:
