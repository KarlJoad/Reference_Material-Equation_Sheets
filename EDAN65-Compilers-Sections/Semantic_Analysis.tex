\section{Semantic Analysis}\label{sec:Semantic_Analysis}
This is the time where we can start attaching some meaning to the tokens that we have read.
We can attach types to the tokens to note that they are number literals, variables, identifiers, expressions, etc.
This will begin the portion of the compiler where we have read in our program and now we need to start making sure that it makes sense.

To improve modularity, it is better to seaparate issues of syntax (parsing) from issues of semantics (type-checking and translation to machine code).

\begin{table}[h!]
  \centering
  \begin{tabular}{p{3cm}p{6cm}p{6cm}}
    \toprule
    & \multicolumn{1}{c}{\textbf{\nameref{def:Concrete_Syntax}}} & \multicolumn{1}{c}{\textbf{\nameref{def:Abstract_Syntax}}} \\
    \midrule
    What does it Describe? & The concrete representation of the programs & The abstract structure of the programs \\ \midrule
    Main Use & Parsing text to trees & Model representing program inside compiler \\ \midrule
    Underlying formalism & Context-Free Grammar & Recursive Data Types \\ \midrule
    What is Named? & Only non-terminals. (Productions usually anonymous) & Nonterminals and Productions \\ \midrule
    What tokens occur in the grammar? & All tokens corresponding to ``words'' in the text & Usually tokens with values (identifiers, literals) \\ \midrule
    & Independent of abstract structure & Independent of parser and parser algorithm \\
    \bottomrule
  \end{tabular}
  \caption{Concrete Syntax vs. Abstract Syntax}
  \label{tab:Concrete_vs_Abstract_Syntax}
\end{table}

\begin{definition}[Concrete Syntax]\label{def:Concrete_Syntax}
  The \emph{concrete syntax} is more verbose than that of an \nameref{def:Abstract_Syntax}.
  It contains the necessary information, grammar transformations, and elimination of ambiguity required to parse the program.
  However, they can be unwieldy to use in the later stages of compilation.
\end{definition}

\begin{definition}[Abstract Syntax]\label{def:Abstract_Syntax}
  The \emph{abstract syntax} is less verbose than that of the \nameref{def:Concrete_Syntax}, but it contains all the of the same information.
  This makes a clean interface between the parser and later stages of a compiler (or other kinds of program-analysis tools).
  
  The \emph{abstract syntax} conveys the phrase structure of the source program, with all the parsing issues resolved, but no semantic interpretation yet.

  Early compilers did not use these because of memory issues.
\end{definition}

\subsection{Parse Trees}\label{subsec:Parse_Trees}
One method of doing this is for the parser to produce a \emph{parse tree}.
\begin{definition}[Parse Tree]\label{def:Parse_Tree}
  A \emph{parse tree} is a data structure for later phases of the compiler to traverse.
  It describes the entirety of the program within a tree structure.
  Here, there is exactly one leaf for each token of the input and one internal node for each grammar rule reduced during the parse.
  This is called a \nameref{def:Concrete_Parse_Tree}
\end{definition}

\begin{definition}[Concrete Parse Tree]\label{def:Concrete_Parse_Tree}
  A \emph{concrete parse tree} is one that has exactly one leaf for each token of the input and one internal node for each grammar rule reduced during the parse.
  This represents the \emph{\nameref{def:Concrete_Syntax}} of the source language, which may be difficult to use internally.
  Much of the punctuation that is borken up from the input string conveys no information to the compiler, but are useful for the programmer.
  However, once the \nameref{def:Parse_Tree} is built, the structure of the tree conveys the structuring information in a more convienent way.

  Additionally, the structure of the \nameref{def:Concrete_Parse_Tree} may depend too much on the grammar and \nameref{def:Concrete_Syntax}.
  Grammar transformations and the elimination of ambiguity should only take place during parsing, and no later.
\end{definition}

\begin{definition}[Abstract Parse Tree]\label{def:Abstract_Parse_Tree}
  An \emph{abstract parse tree} is also called an \emph{Abstract Syntax Tree}.
  Here, the \nameref{def:Concrete_Parse_Tree} is represented only by the operations present in the program.
  The precedence and formatting of the tree is handled by the \nameref{def:Concrete_Syntax} and \nameref{def:Concrete_Parse_Tree}.

  \emph{Abstract Parse/Syntax Trees} are data structures within the compiler program, and are not going to be used outside of it.

  \begin{remark}[Abstract Syntax Tree in Java]\label{rmk:Abstract_Syntax_Tree_in_Java}
    In Java, because of its Object-Oriented nature, the \nameref{def:Abstract_Parse_Tree} is organized in the following way:
    \begin{itemize}[noitemsep]
    \item An abstract class for each nonterminal
    \item A subclass for each production
    \item etc.
    \end{itemize}
  \end{remark}
\end{definition}

\begin{table}[h!]
  \centering
  \begin{tabular}{ccc}
    \toprule
    \textbf{Abstract Grammar} & \textbf{Object-Oriented Model} & \textbf{Other Model} (Algebraic Data Types) \\
    \midrule
    Programming Language & Java & Haskell \\
    Nonterminal & Superclass & Type, Sort \\
    Production & Subclass & Constructor, Operator \\
    \bottomrule
  \end{tabular}
  \caption{Abstract Syntax Tree Hierarchy in Programming Paradigms}
  \label{tab:Abstract_Syntax_Tree_Programming_Paradigms}
\end{table}
\subsubsection{Abstract Parse/Syntax Trees}\label{subsubsec:Abstract_Parse_Trees}


%%% Local Variables:
%%% mode: latex
%%% TeX-master: "../EDAN65-Compilers-Reference_Sheet"
%%% End:
