\section{Reference Attribute Grammars}\label{sec:Reference_Attribute_Grammars}
First, for this course, \theauthor{} used the \href{http://jastadd.org/web/documentation/reference-manual.php}{JastAdd} program to use Reference Attribute Grammars.
This means that if you use something different, then the syntax of these might change, but the concept stays the same.
Most \nameref{sec:Reference_Attribute_Grammars} support all of the elements (\Crefrange{subsec:Synthesized_Attributes}{subsec:Circular_Attributes}) that are shown in this section.

\begin{definition}[Reference Attribute Grammar]\label{def:Reference_Attribute_Grammar}
  A \emph{reference attribute grammar} or \emph{RAG} is a formal way to define \nameref{def:Attribute}s for nodes in the \nameref{def:Abstract_Parse_Tree} and associating them with values.
  In the case of JastAdd, these are written in a \nameref{def:Declarative_Programming} style.
\end{definition}

\begin{definition}[Attribute]\label{def:Attribute}
  An \emph{attribute} is similar to a key-value pair from a Hash-based data structure, except the value is an equation that returns some value.
  These are selectively attached to the various nodes in the \nameref{def:Abstract_Parse_Tree}.
  For example, there may be an attribute that attaches a reference from a variable use to the variable's declaration.
  There are many types of attributes, each of which is explained in this chapter.
\end{definition}

\begin{definition}[Declarative Programming]\label{def:Declarative_Programming}
  \emph{Declarative programming} is a programming paradigm that expresses the logic of a computation without describing its control flow.
  In short, this means programs describe their desired results without explicitly listing commands or steps that must be performed.

  There are a few more points of interest when it comes to \nameref{def:Declarative_Programming}.
  \begin{itemize}[noitemsep]
  \item \nameref{def:Declarative_Programming} describes what a computation should perform, not necessarily how it does it.
  \item \nameref{def:Declarative_Programming} lacks side effects. Functions cannot change data that is visible outside of the function itself. For example:
    \begin{itemize}[noitemsep]
    \item Array elements cannot be updated. This can be skirted around by making a copy of the array.
    \item Strings cannot be changed. Similar way to skirt the issue as with arrays.
    \item Counters cannot be updated which is visible outside of the function.
    \end{itemize}
    This means that every time a function is run, if the input is the same, the output will be the same \textbf{\emph{EVERY TIME}}.
  \item \nameref{def:Declarative_Programming} resembles mathematical logic.
  \end{itemize}
\end{definition}

\subsection{Synthesized Attributes}\label{subsec:Synthesized_Attributes}
\begin{definition}[Synthesized Attribute]\label{def:Synthesized_Attribute}
  A \emph{synthesized attribute} is an \nameref{def:Attribute} where the equation is defined in the same node as the \nameref{def:Attribute}.
\end{definition}

\subsubsection{Defining Equations for \nameref{subsec:Synthesized_Attributes}}\label{subsubsec:Define_Synthesized_Equations}

\subsection{Collection Attributes}\label{subsec:Collection_Attributes}
\begin{definition}[Collection Attribute]\label{def:Collection_Attribute}
  A \emph{collection attribute} is an \nameref{def:Attribute} that is defined by a set of \nameref{def:Contribution}s instead of an equation.
  These are special attributes that are attached to specific nodes in the \nameref{def:Abstract_Parse_Tree} that can have specific things added to them with the \nameref{def:Contribution} equations.
\begin{javasource}
  coll T A.c() [fresh]
    [with m]
    [root R];
\end{javasource}
  \begin{itemize}[noitemsep]
  \item \texttt{T}: type of the attribute. Usually T is a subtype of \texttt{java.lang.Collection}.
  \item \texttt{A}: \nameref{def:Abstract_Parse_Tree} class on which the attribute is evaluated (put on).
  \item \texttt{.c()}: Declares the attribute name, in this case c.
  \item \texttt{fresh} (Optional): How the \nameref{def:Collection_Attribute} is initialized.
    The Java expression \texttt{fresh} creates an empty instance of the result type, \texttt{T}.
    This part is optional if \texttt{T} is a concrete type with a default constructor, if it is omitted the default constructor of the type \texttt{T} is used, i.e. \texttt{new T()}.
  \item \texttt{with m} (Optional): Specifies the name of a method to be used for updating the \nameref{def:Collection_Attribute} object.
    This part is optional and the default method \texttt{add} is used if no method \texttt{m} is specified.
    The update method must fulfill these requirements:
    \begin{itemize}[noitemsep]
    \item The method \texttt{m}, should be a one-argument method of \texttt{T}. It only takes one argument, the thing being updated.
    \item The method \texttt{m} should mutate the \texttt{T} object by adding \emph{one} object to it.
    \item The method \texttt{m} should be commutative, in the sense that the order of calling \texttt{m} for different contributions should yield the same resulting \texttt{T} value.
    \end{itemize}
  \item \texttt{root R} (Optional): Declares the \nameref{def:Collection_Attribute} root type, \texttt{R}.
    The collection mechanism starts by finding the nearest ancestor node of type \texttt{R} for the \texttt{A} node which the collection attribute is evaluated on.
    The subtree rooted at that nearest \texttt{R} ancestor is searched for contributions to \texttt{A.c()}, this means that the collection is scoped to the subtree of \texttt{R}, and contributions outside that tree are not visible.
    \begin{itemize}[noitemsep]
    \item This allows you to have multiple \nameref{def:Collection_Attribute}s in a single \nameref{def:Abstract_Parse_Tree} by placing the attribute on an arbitrary root type.
    \item If you do this, only the tree beneath that node type will be able to access the \nameref{def:Collection_Attribute}.
    \end{itemize}
  \end{itemize}
  \begin{remark}
    These are the main way to introduce mutable objects to the \nameref{def:Reference_Attribute_Grammar}.
    However, because the contributions cannot have side-effects, this technically fits into the \nameref{def:Declarative_Programming} paradigm.
  \end{remark}
\end{definition}

\begin{definition}[Contribution]\label{def:Contribution}
  A \emph{contribution} is a special equation that adds or mutates a value inside of the \nameref{def:Collection_Attribute}.
  There are 3 main ways to contribute to a \nameref{def:Collection_Attribute}:
  \begin{enumerate}[noitemsep]
  \item Contribute a single value to a single \nameref{def:Collection_Attribute}. This is the most commonly used one.
\begin{javasource}
  N1 contributes value-expression
    [when conditional-expression]
    to A.c()
    [for A-reference-expression];
\end{javasource}
    \begin{itemize}[noitemsep]
    \item \texttt{N1}: The type of \nameref{def:Abstract_Parse_Tree} node that is providing the \nameref{def:Contribution}.
    \item \texttt{value-expression}: Java expression that evaluates to an object to be added to the intermediate collection of the target collection attribute.
    \item \texttt{when conditional-expression} (Optional): the contribution is only added to the target collection attribute if the Java expression cond-exp evaluates to \texttt{true}.
    \item \texttt{A}: Node type where the target collection attribute is declared. Matches the node type declared for the \nameref{def:Collection_Attribute}.
    \item \texttt{.c()}: The name of the target collection attribute.
    \item \texttt{for A-reference-expression} (Optional): Java expression which evaluates to a reference to the AST node that owns the collection attribute this contribution is contributing to.
      This is the target expression, and it can be omitted if the target node is identical to the collection root node.
      \begin{itemize}[noitemsep]
      \item Essentially, if the \nameref{def:Collection_Attribute} is not located on the \texttt{N1} node type, then we have to give a pointer to what node the \nameref{def:Collection_Attribute} is actually contained.
      \item This means there must be an \nameref{def:Attribute} that has a pointer to the node type (\nameref{def:Reference_Attribute}) that does contain the \nameref{def:Collection_Attribute}.
      \item It is likely that the Pointer will have to be an \nameref{def:Inherited_Attribute}.
      \end{itemize}
    \end{itemize}

  \item Contribute a single value to multiple target \nameref{def:Collection_Attribute}s.
\begin{javasource}
  N1 contributes value-expression
    [when cond-expression]
    to A.c()
    [for each A-reference-set];
\end{javasource}
    \begin{itemize}[noitemsep]
    \item \texttt{N1}: The type of \nameref{def:Abstract_Parse_Tree} node that is providing the \nameref{def:Contribution}.
    \item \texttt{value-expression}: Java expression that evaluates to an object to be added to the intermediate collection of the target collection attribute.
    \item \texttt{when conditional-expression} (Optional): the contribution is only added to the target collection attribute if the Java expression cond-exp evaluates to \texttt{true}.
    \item \texttt{A}: Node type where the target collection attribute is declared. Matches the node type declared for the \nameref{def:Collection_Attribute}.
    \item \texttt{.c()}: The name of the target collection attribute.
    \item \texttt{for each A-reference-set} (Optional): Java expression that evaluates to an \texttt{Iterable<A>} containing references for the set of contribution target nodes.
      \begin{itemize}[noitemsep]
      \item This is just an extension of the \texttt{A-reference-expression}.
      \item Instead of a single \texttt{A-reference-expression}, there is an \texttt{Iterable} collection of \texttt{A-reference-expression}s.
      \end{itemize}
    \end{itemize}

  \item Contribute multiple values to multiple target \nameref{def:Collection_Attribute}s.
\begin{javasource}
  N1 contributes each value-iterable
    [when conditional-expression]
    to A.c()
    [for each A-reference-set];
\end{javasource}
    \begin{itemize}[noitemsep]
    \item \texttt{N1}: The type of \nameref{def:Abstract_Parse_Tree} node that is providing the \nameref{def:Contribution}.
    \item \texttt{each value-expression}: This syntax works if \texttt{value-expression} has the type \texttt{Iterable<E>} where \texttt{E} is the element type of the \nameref{def:Collection_Attribute}.
      \begin{itemize}[noitemsep]
      \item For example, if the collection attribute is declared as \texttt{coll LinkedList<String> $\ldots$} then \texttt{value-iterable} should have the type \texttt{Iterable<String>}.
      \item Meaning the \texttt{value-iterable} should be an \texttt{Iterable} collection.
      \end{itemize}
    \item \texttt{when conditional-expression} (Optional): the contribution is only added to the target collection attribute if the Java expression cond-exp evaluates to \texttt{true}.
    \item \texttt{A}: Node type where the target collection attribute is declared. Matches the node type declared for the \nameref{def:Collection_Attribute}.
    \item \texttt{.c()}: The name of the target collection attribute.
    \item \texttt{for each A-reference-set} (Optional): Java expression that evaluates to an \texttt{Iterable<A>} containing references for the set of contribution target nodes.
      \begin{itemize}[noitemsep]
      \item This is just an extension of the \texttt{A-reference-expression}.
      \item Instead of a single \texttt{A-reference-expression}, there is an \texttt{Iterable} collection of \texttt{A-reference-expression}s.
      \end{itemize}
    \end{itemize}
  \end{enumerate}
\end{definition}

%%% Local Variables:
%%% mode: latex
%%% TeX-master: "../EDAN65-Compilers-Reference_Sheet"
%%% End:
