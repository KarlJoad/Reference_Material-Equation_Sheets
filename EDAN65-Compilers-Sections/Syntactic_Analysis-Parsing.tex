\section{Syntactic Analysis/Parsing}\label{sec:SyntacticAnalysis}
There are 2 kinds of parsing techniques:
\begin{enumerate}[noitemsep]
\item \nameref{subsec:LLParsing}
\item \nameref{subsec:LRParsing}
\end{enumerate}

The table below will help characterize the differences between them.
\begin{table}[h!]
  \centering
  \begin{tabular}{p{4cm}cc}
    \toprule
    & $LL(k)$ & $LR(k)$ \\
    \midrule
    Parses Input & \multicolumn{2}{c}{\emph{L}eft-to-Right} \\ \midrule
    Derivation & \emph{L}eftmost & \emph{R}ightmost \\ \midrule
    Lookahead & \multicolumn{2}{c}{$k$ Symbols} \\ \midrule
    Build the Tree & Top Down & Bottom Up \\ \midrule
    Select Rule & After seeing its first $k$ tokens & After seeing all its tokens, and an addition $k$ tokens \\ \midrule
    Left Recursion & No & Yes \\ \midrule
    Unlimited Common Prefix & No & Yes \\ \midrule
    Resolve Ambiguities Through Rule Priority & Dangling Else & Dangling Else, Associativity, Priority \\ \midrule
    Error Recovery & Trial-and-Error & Good Algorithms Exist \\ \midrule
    Implement by Hand? & Possible & Too complicated. Use a generator. \\
    \bottomrule
  \end{tabular}
  \caption{LL vs. LR Parsing}
  \label{tab:LLvsLRParsing}
\end{table}

The block below will show the difference in derivation between \nameref{subsec:LLParsing} and \nameref{subsec:LRParsing}.
\begin{blackbox}
  Say you have a set of productions as follows:
  \begin{align*}
    p1&: X \rightarrow YZV \\
    p2&: Y \rightarrow ab \\
    p2&: Z \rightarrow c \\
    p3&: V \rightarrow de
  \end{align*}

  The LL derivation will be as follows:
  \begin{equation*}
    \underline{X} \Rightarrow \underline{Y}ZV \Rightarrow ab\underline{Z}V \Rightarrow abc\underline{V} \Rightarrow abcde
  \end{equation*}

  The LR derivation has 2 options, both of which achieve the same thing.
  \begin{enumerate}[noitemsep]
  \item The way it works in practice, you have the terminals and have to go ``up'' the production rules.
    \begin{equation*}
      \underline{ab}cde \Rightarrow Y\underline{c}de \Rightarrow YZ\underline{de} \Rightarrow \underline{YZV} \Rightarrow X
    \end{equation*}
  \item The way it works in theory, you have a starting nonterminal and work your way down by deriving the right-most side of the input string.
    \begin{equation*}
      \underline{X} \Rightarrow YZ\underline{V} \Rightarrow Y\underline{Z}de \Rightarrow \underline{Y}cde \Rightarrow abcde
    \end{equation*}
  \end{enumerate}
\end{blackbox}

\subsection{LL Parsing}\label{subsec:LLParsing}
There are 5 basic steps in constructing an $\LLParse(1)$ parser.
\begin{enumerate}[noitemsep]
\item Write the grammar in canonical form.
\item Compute \nameref{subsubsec:Nullable}, \nameref{subsubsec:FIRST}, and \nameref{subsubsec:FOLLOW}.
\item Use them to contruct a table. It shows what production to select given the current lookahead token.
\item Check for conflicts.
  \begin{enumerate}[noitemsep]
  \item If there \emph{are} conflicts, then the grammar is not $\LLParse(1)$.
  \item If there are \emph{no} conflicts, then there is a straight-forward implementation using table-driven parser or recursive descent.
  \end{enumerate}
\end{enumerate}

Many times, you will encounter \nameref{def:Fixed-Point_Problems}.
\begin{definition}[Fixed-Point Problems]\label{def:Fixed-Point_Problems}
  \emph{Fixed-point problems} have the form:
  \begin{equation}\label{eq:Fixed-Point_Problems}
    x == f(x)
  \end{equation}

  Can we find a value $x$ for which the equation holds (i.e., a solution)?
  $x$ is then called the \emph{fixed point} of the function $f(x)$.

  \nameref{def:Fixed-Point_Problems} can (sometimes) be solved using iteration.
  The steps involved in \emph{fixed-point iteration} are:
  \begin{enumerate}[noitemsep]
  \item Guess an initial value $x_{0}$
  \item Apply the function iteratively
  \item Iterate until the fixed point is reached.
  \end{enumerate}
  \begin{align*}
    x_{1} &= f(x_{0}) \\
    x_{2} &= f(x_{1}) \\
    & \vdots \\
    x_{n} &= f(x_{n-1})
  \end{align*}
  You continue this iteration until $x_{n} = x_{n-1}$, and $x_{n}$ is called the \emph{fixed point}.
\end{definition}

\subsubsection{Nullable}\label{subsubsec:Nullable}
\begin{definition}[Nullable]\label{def:Nullable}
  For the production $p$, where $p: X \rightarrow \gamma$, and $X$ and $\gamma$ are nonterminals; $p$ is \emph{\nameref{def:Nullable}} if we can derive $\epsilon$ from $\gamma$.

  More formally, this can be defined as $\Nullable(\gamma)$ is true iff the empty sequence can be derived from $\gamma$:
  \begin{equation*}
    \Nullable(\gamma) =
    \begin{cases}
      \True, &  \exists(\gamma \Rightarrow \!\! * \, \epsilon) \\
      \False, & \text{Otherwise}
    \end{cases}
  \end{equation*}

  You can define an equation system for \nameref{def:Nullable} given that $G=(N,T,P,S)$.
  \begin{subequations}
    \begin{equation}\label{eq:NullableEpsilon}
      \Nullable(\epsilon)== \True
    \end{equation}
    \begin{equation}\label{eq:NullableTerminals}
      \Nullable(t) = \False
    \end{equation}
    where $t \in T$, i.e., $t$ is a terminal symbol
    
    \begin{equation}\label{eq:NullableNonterminals}
      \Nullable(X) = \Nullable(\gamma_{1}) \OR \ldots \OR \Nullable(\gamma_{n})
    \end{equation}
    where $X \rightarrow \gamma_{1}, \ldots, X \rightarrow \gamma_{n}$ are all the productions for $X$ in $P$.
    
    \begin{equation}\label{eq:NullableNonterminalorTerminal}
      \Nullable(s\gamma) = \Nullable(s) \AND \Nullable(\gamma)
    \end{equation}
    where $s \in N \cup T$, i.e., $s$ is a nonterminal or a terminal
  \end{subequations}

  \begin{remark}
    The equations (\Crefrange{eq:NullableEpsilon}{eq:NullableNonterminalorTerminal}) for \nameref{def:Nullable} are recursive.
    Therefore, you can't just calculate these recursively, because you might never terminate if the empty sequence is never reached.
    This is one set of \nameref{def:Fixed-Point_Problems}.
  \end{remark}
\end{definition}

\begin{blackbox}
  One way to think about solving a problem asking about \nameref{def:Nullable} is shown below. I will use this grammar to demonstrate:
  \begin{align*}
    p1&: X \rightarrow Y \vert Z \\
    p2&: Y \rightarrow a \vert b \vert V \\
    p3&: Z \rightarrow c \vert \epsilon \\
    p4&: V \rightarrow d \vert Y
  \end{align*}

  \begin{enumerate}[noitemsep]
  \item Determine the nonterminal you are interested in, let's say $X$.
  \item Find all the productions with $X$ on the \textbf{\emph{LEFT-HAND SIDE}}.
    \begin{itemize}[noitemsep]
    \item $X$ is present on the left-hand side of $p1$.
    \end{itemize}
  \item Follow these productions to their right-hand side.
    \begin{itemize}[noitemsep]
    \item So we are considering the right-hand side of $p1$, which is $Y \vert Z$.
    \end{itemize}
  \item You will evaluate each of the tokens on the \textbf{\emph{RIGHT-HAND SIDE}} of the production(s) we are interested in.
  \item If the token we are looking at on the \textbf{\emph{RIGHT-HAND SIDE}} is a terminal, the nonterminal $\gamma$ is \textbf{\emph{NOT}} nullable.
    \begin{itemize}[noitemsep]
    \item This is the case for $Y$.
      Since either $a$ or $b$ could present, and $V$ can either produce $d$ or recurse back to $Y$, $Y$ can NEVER yield $\epsilon$.
    \item In our case, both $X \rightarrow Y$ and $X \rightarrow Z$; $Y$ and $Z$ are nonterminals, so this step doesn't apply.
    \end{itemize}
  \item If the token that we are looking at on the \textbf{\emph{RIGHT-HAND SIDE}} is a nonterminal, then follow them.
    \begin{itemize}[noitemsep]
    \item In both $X \rightarrow Y$ and $X \rightarrow Z$, $Y$ and $Z$ are nonterminals, so we follow both.
      \begin{itemize}[noitemsep]
      \item Since we already calculated $Y$ in the previous step, we know that $Y$ is not nullable.
        However, we can add the values that $Y$ can produce to a set to make sure we are correct.
        So, $\Nullable(X) = \lbrace a, b, d \rbrace$.
        Now we move onto $Z$.
      \item The production for $Z$ is $Z \rightarrow c \vert \epsilon$.
        In this case, $Z$ may be $\epsilon$.
        We can add these values to our $\Nullable(X)$ set: $\Nullable(X) = \lbrace a, b, c, d, \epsilon \rbrace$
      \end{itemize}
    \end{itemize}
  \item Once we have computed all possible $\Nullable(X)$ occurrences, we are done.
  \end{enumerate}

  This leaves us with our $\Nullable(X)$ set: $\lbrace a, b, c, d, \epsilon \rbrace$.
  Since there is an option for $X$ to be $\epsilon$, $X$ \textbf{is} Nullable.
\end{blackbox}

\subsubsection{FIRST}\label{subsubsec:FIRST}
\begin{definition}[FIRST]\label{def:FIRST}
  For the production $p$, where $p: X \rightarrow \gamma$, and $X$ and $\gamma$ are nonterminals.
  The $\text{FIRST}(\gamma)$ are the \textbf{tokens} that occur \emph{\nameref{def:FIRST}} in a sentence derived from $\gamma$.

  More formally, this can be defined as: $\FIRST(\gamma)$ is the set of tokens that can occur \emph{first} in the sentences derived from $\gamma$.
  \begin{equation*}
    \FIRST(\gamma) = \lbrace t \in T \vert \gamma \Rightarrow \!\! * \, t \delta \rbrace
  \end{equation*}

  You can define an equation system for \nameref{def:FIRST} given that $G=(N,T,P,S)$.
  \begin{subequations}
    \begin{equation}\label{eq:FIRST_Epsilon}
      \FIRST(\epsilon) = \emptyset
    \end{equation}
    \begin{equation}\label{eq:FIRST_Terminal}
      \FIRST(t) = \lbrace t \rbrace
    \end{equation}
    where $t \in T$, i.e., $t$ is a terminal symbol

    \begin{equation}\label{eq:FIRST_Nonterminal}
      \FIRST(X) = \FIRST(\gamma_{1}) \cup \ldots \cup \FIRST(\gamma_{n})
    \end{equation}
    where $X \rightarrow \gamma_{1}, \ldots, X \rightarrow \gamma_{n}$ are all the productions for $X$ in $P$.

    \begin{equation}\label{eq:FIRST_Nonterminal_or_Terminal}
      \FIRST(s\gamma) = \FIRST(s) \cup (\text{if } \Nullable(s) \text{ then } \FIRST(\gamma) \text{ else } \emptyset \text{ fi})
    \end{equation}
    where $s \in N \cup T$, i.e., $s$ is a nonterminal or a terminal
  \end{subequations}

  \begin{remark}
    The equations (\Crefrange{eq:FIRST_Epsilon}{eq:FIRST_Nonterminal_or_Terminal}) for \nameref{def:FIRST} are recursive.
    Therefore, they might not terminate, so you must calculate this as another set of \nameref{def:Fixed-Point_Problems}.
  \end{remark}
\end{definition}

\begin{blackbox}
  One way to think about solving a problem asking about \nameref{def:FIRST} is shown below. I will use this grammar to demonstrate:
  \begin{align*}
    p1&: X \rightarrow YZa \\
    p2&: Y \rightarrow b \vert Z \vert V \\
    p3&: Z \rightarrow c \vert \epsilon \\
    p4&: V \rightarrow \epsilon
  \end{align*}

  \begin{enumerate}[noitemsep]
  \item Determine the nonterminal you are interested in, let's say $Y$.
  \item Find all productions with $Y$ on the left-hand side.
    \begin{itemize}[noitemsep]
    \item $Y$ is present on the left-hand side of $p2$.
    \end{itemize}
  \item Follow each of these productions to their right-hand side.
    \begin{itemize}[noitemsep]
    \item So we are considering the right-hand side of $p2$, which is $b \vert Z \vert V$.
    \end{itemize}
  \item If the first token on the \textbf{\emph{RIGHT-HAND SIDE}} is a terminal, add it to the $\FIRST$ set.
    \begin{itemize}[noitemsep]
    \item $\FIRST(Y) = \lbrace b \rbrace$
    \end{itemize}
  \item If the first token on the  \textbf{\emph{RIGHT-HAND SIDE}} is a nonterminal, go to that production and compute $\FIRST$ on that.
    \begin{itemize}[noitemsep]
    \item Since $Z$ is an option in $p2$, we compute $\FIRST(Z)$, which yields $\lbrace c, \epsilon \rbrace$.
      We add both to the $\FIRST(Y)$ list.
      Our list is now $\FIRST(Y) = \lbrace b, c, \epsilon \rbrace$
    \item Since $B$ is an option in $p2$, we compute $\FIRST(V)$, which yields $\lbrace \epsilon \rbrace$.
      We add this to the $\FIRST(Y)$ list.
      Our list is now $\FIRST(Y) = \lbrace b, c, \epsilon \rbrace$
    \end{itemize}
  \item Since $\epsilon$ is an empty string, we can remove it from the list, or just ignore it.
  \item Once we have computed all possible $\FIRST(Y)$ occurrences, we are done.
  \end{enumerate}

  This leaves us with our $\FIRST(Y)$ set: $\lbrace b, c \rbrace$, which is the solution.
\end{blackbox}

\subsubsection{FOLLOW}\label{subsubsec:FOLLOW}
\begin{definition}[FOLLOW]\label{def:FOLLOW}
  For the production $p$, where $p: X \rightarrow \gamma$, and $X$ and $\gamma$ are nonterminals.
  The $\text{FOLLOW}(X)$ are the \textbf{tokens} that \emph{\nameref{def:FOLLOW}} immediately after an $X$-sentence.

  More formally, this can be defined as: $\FOLLOW(X)$ is the set of tokens that can occur as the \emph{first} token \emph{following} $X$, in any \nameref{rmk:Sentential_Form} derived from the start symbol $S$:
  \begin{equation*}
    \FOLLOW(X) = \lbrace t \in T \vert S \Rightarrow \!\! * \, \alpha X t \beta \rbrace
  \end{equation*}
  
  The nonterminal $X$ occurs on the right-hand side of a number of productions.

  Let $Y \rightarrow \gamma X \delta$ denote such an occurrence, where $\gamma$ and $\delta$ are arbitrary sequences of terminals and nonterminals.
  You can define an equation system for \nameref{def:FOLLOW} given that $G=(N,T,P,S)$.
  \begin{subequations}
    \begin{equation}\label{eq:FOLLOW_Nonterminals}
      \FOLLOW(X) = \bigcup \FOLLOW(Y \rightarrow \gamma \underline{X} \delta)
    \end{equation}
    over all occurrences $Y \rightarrow \gamma X \delta$, and where
    \begin{equation}\label{eq:FOLLOW_Sentential_Form}
      \FOLLOW(Y \rightarrow \gamma \underline{X} \delta) = \FIRST(\delta) \cup (\text{if } \Nullable(\delta) \text{ then } \FOLLOW(Y) \text{ else } \emptyset \text{ fi})
    \end{equation}
  \end{subequations}

  \begin{remark}
    Again, the equations (\Crefrange{eq:FOLLOW_Nonterminals}{eq:FOLLOW_Sentential_Form}) are recursive.
    Therefore, they might not terminate, so you must calculate this as another set of \nameref{def:Fixed-Point_Problems}.
  \end{remark}

  \begin{remark}[Sentential Form]\label{rmk:Sentential_Form}
    Sequence of terminal and nonterminal symbols.
  \end{remark}
\end{definition}

\begin{blackbox}
  One way to think about solving a problem asking about \nameref{def:FOLLOW} is shown below. I will use this grammar to demonstrate:
  \begin{align*}
    p1&: S \rightarrow Xa \\
    p2&: X \rightarrow Y \vert Yb \\
    p3&: Y \rightarrow YZc \vert \epsilon \\
    p4&: Z \rightarrow d \vert \epsilon
  \end{align*}
  \begin{enumerate}[noitemsep]
  \item Determine the nonterminal you are interested in, let's say $Y$.
  \item Find all occurrences of that nonterminal on the \textbf{\emph{RIGHT-HAND SIDE}} of the productions.
    If the nonterminal is \textbf{\emph{NOT}} present anywhere on the right-hand side, it yields the empty set, $\lbrace \emptyset \rbrace$.
    In this case our nonterminal occurs in:
    \begin{itemize}[noitemsep]
    \item $p2$
    \item $p3$
    \end{itemize}
  \item Find the terminals that can directly follow your nonterminal \textit{in the same production}.
    \begin{itemize}[noitemsep]
    \item $b$, from $p2$
    \item $c$, from $p3$
    \end{itemize}
  \item If there are nonterminals after the nonterminal you are interested in, follow them.
    In this case, we follow $Z$.
    Then, find the nonterminals that can directly follow that nonterminal and add them to you're list.
    \begin{itemize}[noitemsep]
    \item $b$, from $p2$
    \item $c$, from $p3$
    \item $d$, from $p4$
    \end{itemize}
  \item If nothing (no nonterminal AND no terminal) follows the nonterminal you want, then go ``backwards'' through the production.
    \begin{enumerate}[noitemsep]
    \item Since $p2$ has $X \rightarrow Y$ as an option and nothing follows this $Y$
    \item Go backwards, up to the production(s) that produces $X$ on the right-hand side
    \item Compute Follow on the right-hand side of that production.
      This produces:
      \begin{itemize}[noitemsep]
      \item $a$, from $p1$
      \end{itemize}
    \end{enumerate}
  \end{enumerate}

  This leaves us with our $\FOLLOW(Y)$ set: $\lbrace a, b, c, d \rbrace$, which is the solution.
\end{blackbox}

\subsubsection{Constructing an \texorpdfstring{$\LLParse(1)$}{LL(1)} Table}\label{subsubsec:Construct_LL1_Table}
Using the information that was gathered from the \nameref{subsubsec:Nullable}, \nameref{subsubsec:FIRST}, and \nameref{subsubsec:FOLLOW} calculations, you can construct an $\LLParse(1)$ parse table with the following steps.
\begin{enumerate}[noitemsep]
\item Look at each production $p: X \rightarrow \gamma$.
\item Compute the token set $\FIRST(\gamma)$. Add $p$ to each corresponding entry for $X$.
\item Check if $\gamma$ is \nameref{def:Nullable}.
  \begin{enumerate}[noitemsep]
  \item If so, compute the token set $\FOLLOW(X)$, and add $p$ to each corresponding entry for $X$.
  \end{enumerate}
\end{enumerate}

\begin{example}[]{LL1 Parse Table}
  An example of an $\LLParse(1)$ table is shown in
  \Cref{tab:LL1_Parse_Table_Example}. It uses the grammar below.
  \begin{align*}
    p0&: S \rightarrow \text{varDecl } \$ \\
    p1&: \text{varDecl} \rightarrow \text{type } \text{ID } \text{optInit} \\
    p2&: \text{type} \rightarrow \text{``Integer''} \\
    p3&: \text{type} \rightarrow \text{``Boolean''} \\
    p4&: \text{optInit} \rightarrow \text{``=''} \text{INT} \\
    p5&: \text{optInit} \rightarrow \epsilon
  \end{align*}

  The steps to constructing an $\LLParse(1)$ table are:
  \begin{enumerate}[noitemsep]
  \item Look at each production $p: X \rightarrow \gamma$
  \item Compute $\FIRST(\gamma)$, and add the corresponding $p$ to the table for $X$
  \item If $\gamma$ is Nullable, then compute $\FOLLOW(\gamma)$ and add the $p$ with the Nullable production to each corresponding value for $X$
  \end{enumerate}
\end{example}

\begin{table}[h!]
  \centering
  \begin{tabular}{c||cccccc}
    \toprule
    & ID & Integer & Boolean & ``='' & INT & \$ \\
    \midrule
    S & & $p0$ & $p0$ & & & \\
    varDecl & & $p1$ & $p1$ & & & \\
    type & & $p2$ & $p3$ & & & \\
    optInit & & & & $p4$ & $p5$ \\
    \bottomrule
  \end{tabular}
  \caption{$\LLParse(1)$ Table Example}
  \label{tab:LL1_Parse_Table_Example}
\end{table}
  
\paragraph{\texorpdfstring{$\LLParse(1)$}{LL(1)} Parse Table Conflicts}\label{par:LL1_Table_Conflicts}
For every case where an $\LLParse(1)$ grammar fails, it can be illustrated by the parse table for the grammer.
The table for any of these grammars with have a \emph{conflict} in one of its cells, as shows when there are multiple productions in the cell.
Some of the cases where this may happen are listed below:
\begin{enumerate}[noitemsep]
\item Ambiguous Grammar
\item Left-Recursive
\item Common Prefix
\end{enumerate}

\subsubsection{Eliminating Left Recursion}\label{subsubsec:Eliminate_Left_Recursion}
$\LLParse(1)$ cannot support left recursion, however, they can support right recursion.
So, if we have a grammar with left recursion, and want to $\LLParse(1)$ parse it, we need to rewrite the grammar.
We can introduce a new nonterminal, which allows us to recurse on the right side of the production, but not the left.

\subsection{LR Parsing}\label{subsec:LRParsing}
$\LRParse(k)$ parsing stands for Left-to-right parse, rightmost-derivation, $k$-token lookahead.
This parsing technique postpones the decision of which production to use until it sees the entire right-hand side of the production in question (and $k$ more tokens beyond).

An LR parser has a \emph{stack} and and \emph{input}.
The first $k$ tokens of the input are the \emph{lookahead}.
Based on the contents of the stack and the lookahead, the parser performs 1 of 2 actions.
\begin{enumerate}[noitemsep]
\item \nameref{def:LR_Shift}
\item \nameref{def:LR_Reduce}
\end{enumerate}

\begin{definition}[Shift]\label{def:LR_Shift}
  A \emph{shift} operation corresponds to moving the first input token onto the top of the stack.
  This is equivalent to reading the token and moving it onto the stack, and advancing forward through the sentence.
  \begin{remark}[Accepting]\label{rmk:LR_Accept}
    \nameref{def:LR_Shift}ing over the EOF (End of File) marker, typically denoted \$, is called \emph{accepting} and causes the parser to stop successfully.
  \end{remark}
\end{definition}

\begin{definition}[Reduce]\label{def:LR_Reduce}
  A \emph{reduce} operation corresponds to choosing a grammar rule $X \rightarrow ABC$; pop $C, B, A$ off the top of the stack, and push $X$ onto the stack.
\end{definition}

Initially, the stack is empty and the parser is sitting at the beginning of the input.

In practice, $k>1$ is not used.
These would generate incredibly large tables that would be hard to used and hard to make.
Most reasonable programming languages can be described by $\LRParse(1)$ grammars.

\subsubsection{LR Finite State Automata}\label{subsubsec:LRFiniteStateAutomata}
These automata are \nameref{def:DeterministicFiniteAutomataDFA}.
They are used in the parser to decide when to \nameref{def:LR_Shift} and when to \nameref{def:LR_Reduce}, and are applied to the stack.

They can be used to generate an \nameref{subsubsec:LRParseTable}.

Each of the states consists of one ore more \nameref{def:LRItem}s.

\begin{definition}[LR Item]\label{def:LRItem}
  The parser uses a \nameref{def:DeterministicFiniteAutomataDFA} to decide whether to shift or reduce the expression that is has seen so far.
  The \emph{states} in the DFA are sets of \emph{LR Items}.

  An example of an \nameref{def:LRItem} is shown in \Cref{eq:LRItem}.
  \begin{equation}\label{eq:LRItem}
    X \rightarrow \alpha \bullet \beta \:\:\: t,s
  \end{equation}

  An $\LRParse(1)$ item is a production extended with:
  \begin{itemize}[noitemsep]
  \item A \emph{dot}  ($\bullet$), corresponding to the position in the input sentence.
  \item One or more possible \emph{lookahead} terminal symbols: $t,s$.
    \begin{itemize}[noitemsep]
    \item We will use ? when the lookahead doesn't matter.
    \end{itemize}
  \end{itemize}

  \begin{remark}
    The stack that is referenced in this section is left side of the dot that is present in \nameref{def:LRItem}s.
  \end{remark}
  
  The $\LRParse(1)$ item corresponds to a state where (using \Cref{eq:LRItem}):
  \begin{itemize}[noitemsep]
  \item The topmost part of the stack is $\alpha$
  \item The first part of the remaining input is expected to match either, in no particular order:
    \begin{itemize}[noitemsep]
    \item $\beta t$
    \item $\beta s$
    \end{itemize}
  \end{itemize}
\end{definition}

\subsubsection{LR Parse Table}\label{subsubsec:LRParseTable}
This table is generated after making an \nameref{subsubsec:LRFiniteStateAutomata}.
To make one, there are 4 actions to note in the table.
\begin{enumerate}[noitemsep]
\item \nameref{par:ShiftActions}
\item \nameref{par:ReduceActions}
\item \nameref{par:GotoActions}
\item \nameref{par:AcceptAction}
\item Errors are denoted by blank entries in the table.
\end{enumerate}

\paragraph{Shift Actions}\label{par:ShiftActions}
These are found by reading \textbf{\emph{TOKENS}}.
For each \textbf{token edge, $t$}, from state $j$ to state $k$, add a \textbf{\emph{shift action}} $\mathbf{s} k$ to $\text{table}[j,t]$.
(This corresponds to reading a token and pushing it onto the stack.)

\paragraph{Reduce Actions}\label{par:ReduceActions}
These are found when the \textbf{dot is at the end}.
Add a \textbf{\emph{reduce action}} $\mathbf{r} p$ (reduce $p$) to $\text{table}[j,t]$, where $p$ is the production and $t$ is the lookahead token.
(This corresponds to popping the right-hand side of a production off the stack and going backwards through the state machine.)

\paragraph{Goto Actions}\label{par:GotoActions}
These are found by reading a \textbf{\emph{NONTERMINAL}}.
For each \emph{nonterminal edge, $X$}, from state $j$ to state $k$, add a \textbf{\emph{Goto Action}} $ g\mathbf{} k$ (goto state $k$) to $\text{table}[j,X]$.
(This corresponds to pushing the left-hand side of a nonterminal production onto the stack.)

\paragraph{Accept Action}\label{par:AcceptAction}
These are found when in a state containing an LR item with your \textbf{dot on the left of \$}.
If this is so, then add an \textbf{\emph{accept action}} $a$ to the table with indices $\text{table}[j,\$]$.
(If we are about to perform a shift action over the EOF (End Of File) token, \$, then parsing has succeeded.)

\subsubsection{\texorpdfstring{$\LALRParse(1)$ Parsing Tables}{LALR(1) Parsing Tables}}\label{subsubsec:LALR1_Parsing Tables}
Since $\LRParse(1)$ tables can get quite large, a smaller table can be made by merging any 2 states whose items are identical except for lookahead sets.
The resulting parser is called a \emph{Lookahead $\LRParse(1)$} parser.
For example, compare the same states, but different parser types as shown in \Cref{subtab:LR1_Parser_State} and \Cref{subtab:LALR1_Parser_State}.
\begin{table}[h!]
  \centering
  \begin{subtable}[t!]{0.30\linewidth}
    \begin{tabular}[t!]{lc}
      \toprule
      Productions & Lookahead Token \\
      \midrule
      $S' \rightarrow \bullet S \$$ & ? \\
      $S \rightarrow \bullet V = E$ & \$ \\
      $S \rightarrow \bullet E$ & \$ \\
      $E \rightarrow \bullet V$ & \$ \\
      $V \rightarrow \bullet x$ & \$ \\
      $V \rightarrow \bullet * E$ & \$ \\
      $V \rightarrow \bullet x$ & = \\
      $V \rightarrow \bullet * E$ & = \\
      \bottomrule
    \end{tabular}
    \caption{$\LRParse(1)$ Table State}
    \label{subtab:LR1_Parser_State}
  \end{subtable}
  \begin{subtable}[t!]{0.30\linewidth}
    \begin{tabular}[t!]{lc}
      \toprule
      Productions & Lookahead Token \\
      \midrule
      $S' \rightarrow \bullet S \$$ & ? \\
      $S \rightarrow \bullet V = E$ & \$ \\
      $S \rightarrow \bullet E$ & \$ \\
      $E \rightarrow \bullet V$ & \$ \\
      $V \rightarrow \bullet x$ & \$,= \\
      $V \rightarrow \bullet * E$ & \$,= \\
      \bottomrule
    \end{tabular}
    \caption{ $\LALRParse(1)$ Table State}
    \label{subtab:LALR1_Parser_State}
  \end{subtable}
  \caption{$\LRParse(1)$ Table State vs $\LALRParse(1)$ Table State}
  \label{tab:LR1_Parser_vs_LALR1_Parser}
\end{table}

\subsubsection{Syntax Versus Semantics}\label{subsubsec:LR_Syntax_vs_Semantics}
There are some things that context-free grammars cannot describe, and thus cannot be parsed correctly.
For instance, the expression
\begin{equation*}
  a + 5 \AND b
\end{equation*}

The precedence here has the mathematical addition in greater priority than the logical AND operator.
But logical and mathematical operators are not allowed in the same expression, because the types of the variables and operations don't make sense together.
However, the context-free grammar and parser have no knowledge of the \emph{types} of the variables and oeprators in play here.
So, the solution is to let this expression pass through the parser, but it should be caught later, during the \nameref{sec:Semantic_Analysis}.
%%% Local Variables:
%%% mode: latex
%%% TeX-master: "../EDAN65-Compilers-Reference_Sheet"
%%% End:
