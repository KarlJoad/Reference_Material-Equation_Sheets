\section{Syntax}\label{sec:Syntax}
The C language helped define a whole class of syntax that is used by many programming languages today.
C's syntactic decisions can be seen in Java, Rust, C++, C\#, and many others.

\subsection{Primitive Types}\label{subsec:Primitive_Types}
C has just 4 primitive types:
\begin{description}[noitemsep]
\item[\cinline{char}:] \textbf{One byte} integers (0--255), meant to represent ASCII characters.
\item[\cinline{int}:] Integers, which is defined to be \textit{at least} 16 bits.
  \begin{itemize}[noitemsep]
  \item Additional prefixes can be used to increase or decrease the range of the integer.
  \item These are shown in \Cref{subsubsec:Integer_Type_Prefixes}.
  \end{itemize}
\item[\cinline{float}:] Single precision IEEE floating point number.
\item[\cinline{double}:] Double precision IEEE floating point number.
\end{description}

\subsubsection{Integer Type Prefixes}\label{subsubsec:Integer_Type_Prefixes}
\begin{description}[noitemsep]
\item[\cinline{signed}:] The default for integers, meaning you \textbf{do not} have to specify this.
  Can represent both negative and positive integers.
  The range for this is $-2^{\text{\# bits} - 1}$--$2^{\text{\# bits} - 1}-1$
\item[\cinline{unsigned}:], Can only represent 0 and positive integers.
  Its range is $0$--$2^{\text{\# bits}}$
\item[\cinline{short}:] Tells the compiler that the integer must be at least 16 bits.
\item[\cinline{long}:] Tells the compiler that the integer must be at least 32 bits.
\item[\cinline{long long}:] Tells the compiler that the integer must be at least 64 bits.
\end{description}

\subsection{Basic Operators}\label{subsec:Basic_Operators}
Operators perform some operation on expressions.
This could be an arithmetic, a relational, logical, etc.\ operation.

\subsubsection{Arithmetic Operators}\label{subsubsec:Arithmetic_Operators}
These operators are for performing mathematical operations.
These are well-defined for integers and floating-point numbers.

\begin{description}[noitemsep]
\item[\cinline{+}] The addition operator.
  Works similarly for integers and floating-point numbers.

\item[\cinline{-}] The subtraction operator.
  Works similarly for integers and floating-point numbers.

\item[\cinline{*}] The multiplication operator.
  Works similarly for integers and floating-point numbers.

\item[\cinline{/}] The division operator.
  This returns the quotient of a division.
  This has a different result for integers and floating-point numbers.
  \begin{itemize}[noitemsep]
  \item Integers return the quotient of the division, but no fractional part.
  \item Floating-points return the entire remainder of the division.
  \end{itemize}

\item[\texttt{\%}] The modulo operator.
  Returns the remainder of a division operation when dividing integers.
  Note that this is only defined for integers.
\end{description}

\subsubsection{Logical}\label{subsubsec:Logical_Operators}
Logical operators work with boolean values.
\begin{description}[noitemsep]
\item[\cinline{!}] The logical NOT operator.
\item[\cinline{&&}] The logical AND operator.
  Returns \texttt{1} if and only if the left and right expressions are \textbf{BOTH} \texttt{1} at the same time.
  Otherwise, \texttt{0} is returned.
\item[\cinline{||}] The logical OR operator.
  Returns \texttt{0} if and only if both the left and right expressions are \texttt{0} at the same time.
  Otherwise, \texttt{1} is returned.
\end{description}

\subsubsection{Relational}\label{subsubsec:Relational_Operators}
These relational operators are used to define a value in relation to another.
Typically, these are used for boolean comparisons.

\begin{description}[noitemsep]
\item[\cinline{==}] The equality operator.
  Returns \texttt{1} if and only if the two expressions are equal, \texttt{0} otherwise.

\item[\cinline{!=}] The inequality operator.
  Returns \texttt{0} if and only if the two expressions are not equal, \texttt{1} otherwise.

\item[\cinline{>}] The greater-than operator.
  Returns \texttt{1} if and only if the left expression has a greater value than the right one.

\item[\cinline{>=}] The greater-than-or-equal-to operator.
  Returns \texttt{1} is and only if the left expression has a greater value or equal value than the right one.

\item[\cinline{<}] The less-than operator.
  Returns \texttt{1} if and only if the left expression has a lesser value than the right one.

\item[\cinline{<=}] The less-than-or-equal-to operator.
  Returns \texttt{1} is and only if the left expression has a lesser value or equal value than the right one.
\end{description}

\subsubsection{Assignment}\label{subsubsec:Assignment_Operators}
Unlike many other languages, in C, the assignment operator \texttt{=} is also an expression.
This means that when an assignment is performed, it also returns a value, in this case, it returns the value that was assigned to that particular name.

\begin{description}
\item[\cinline{=}] The assignment operator.
  Assigns a value to a given name.
  Returns the value of the assignment.

\item[\cinline{+=}] The add-and-assign operator.
  Takes the name on the left, adds the value on the right to the value on the left, and stores the result in the value on the left.

\item[\cinline{*=}] The multiply-and-assign operator.
  Takes the name on the left, multiplies the value on the left by to the value on the right, and stores the result in the value on the left.
\end{description}

Only \texttt{+=} and \texttt{*=} are shown, but there are similar ones defined for other \nameref{subsubsec:Arithmetic_Operators} too.

\subsubsection{Conditional Operator}\label{subsubsec:Conditional_Operator}
There is only one conditional operator, sometimes called the ternary operation or conditional expression.
It is defined like so: \cinline{bool ? true_exp : false_exp}.

\subsubsection{Bitwise Operators}\label{subsubsec:Bitwise_Operators}
Bitwise operators work on the component bits of a number.
This means they behave slightly differently than any other operator, but are not typically used in day-to-day calculations.
Usually, they are used to efficiently work with memory.

\begin{description}
\item[\cinline{&}] Bitwise AND
\item[\cinline{|}] Bitwise OR
\item[\cinline{^}] Bitwise exclusive OR (XOR)
\item[\cinline{~}] Bitwise negation, one's complement.
\item[\cinline{>>}] Bitwise SHIFT right
\item[\cinline{<<}] Bitwise SHIFT left
\end{description}

\subsection{Boolean Expressions}\label{subsec:Boolean_Expressions}
Because C is so low-level, the concept of \texttt{true} and \texttt{false} are defined as integers.

\begin{description}
\item[\cinline{0}] \texttt{false}.
\item[\cinline{1}] \texttt{true}.
  Technically, any non-zero value is considered \texttt{true}.
\end{description}

\begin{listing}[h!tbp]
\csourcefile{./C_Primer-Sections/Syntax/Code/logical-operators.c}
\caption{Logical Operators}
\label{lst:Logical_Operators}
\begin{minted}[frame=lines,linenos]{console}
$ ./a.out
1
1
0
0
1
\end{minted}
\end{listing}

%%% Local Variables:
%%% mode: latex
%%% TeX-master: "../../C_Primer"
%%% End:


\subsection{Control Flow}\label{subsec:Control_Flow}
Control flow is the idea of changing the direction a program executes based on some predicate.
Whether this change in direction is to a new direction is because of a change in state, or because something must be repeated is irrelevant.

\subsubsection{Branching}\label{subsubsec:Branching}
Branching has to deal with the changing of a program's control flow based on a predicate to perform some separate actions based on the state of the predicate.
There are 4 types for this:
\begin{description}[noitemsep]
\item[\nameref{par:if_Statement}] The most basic change in flow control.
  If the predicate provided is \texttt{true}-thy, performs an action, then returns to normal.

\item[\nameref{par:if_else_Statement}] If the predicate is \texttt{true}-thy, then perform some action, if the predicate is \texttt{false}, then perform some other action.

\item[\nameref{par:if_elseif_else_Statement}] If the predicate in the first \texttt{if} is \texttt{true}-thy, then that branch is taken.
  If the first predicate is \texttt{false}, then the \texttt{else if}'s predicate is checked for truth.
  The the \texttt{else if}'s predicate is \texttt{false}, then execution continues through any other \texttt{else if}s that may be present.
  If none of the \texttt{if} or \texttt{else if}s' predicates were \texttt{true}-thy, then the \texttt{else} is taken.
\item[\nameref{par:switch_case_Statement}] The \texttt{switch}-\texttt{case} statement allows you to choose from many different paths based on the \textbf{VALUE} of some expression.
\end{description}


%%% Local Variables:
%%% mode: latex
%%% TeX-master: "../../C_Primer"
%%% End:



%%% Local Variables:
%%% mode: latex
%%% TeX-master: "../C_Primer"
%%% End:
