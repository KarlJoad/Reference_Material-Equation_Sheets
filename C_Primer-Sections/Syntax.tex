\section{Syntax}\label{sec:Syntax}
The C language helped define a whole class of syntax that is used by many programming languages today.
C's syntactic decisions can be seen in Java, Rust, C++, C\#, and many others.

\subsection{Primitive Types}\label{subsec:Primitive_Types}
C has just 4 primitive types:
\begin{description}[noitemsep]
\item[\cinline{char}:] \textbf{One byte} integers (0--255), meant to represent ASCII characters.
\item[\cinline{int}:] Integers, which is defined to be \textit{at least} 16 bits.
  \begin{itemize}[noitemsep]
  \item Additional prefixes can be used to increase or decrease the range of the integer.
  \item These are shown in \Cref{subsubsec:Integer_Type_Prefixes}.
  \end{itemize}
\item[\cinline{float}:] Single precision IEEE floating point number.
\item[\cinline{double}:] Double precision IEEE floating point number.
\end{description}

\subsubsection{Integer Type Prefixes}\label{subsubsec:Integer_Type_Prefixes}
\begin{description}[noitemsep]
\item[\cinline{signed}:] The default for integers, meaning you \textbf{do not} have to specify this.
  Can represent both negative and positive integers.
  The range for this is $-2^{\text{\# bits} - 1}$--$2^{\text{\# bits} - 1}-1$
\item[\cinline{unsigned}:], Can only represent 0 and positive integers.
  Its range is $0$--$2^{\text{\# bits}}$
\item[\cinline{short}:] Tells the compiler that the integer must be at least 16 bits.
\item[\cinline{long}:] Tells the compiler that the integer must be at least 32 bits.
\item[\cinline{long long}:] Tells the compiler that the integer must be at least 64 bits.
\end{description}


%%% Local Variables:
%%% mode: latex
%%% TeX-master: "../C_Primer"
%%% End:
