\section{Arrays}\label{sec:Arrays}
An \nameref{def:Array} is the only way to store multiple items under a single name.

\begin{definition}[Array]\label{def:Array}
  A \emph{array} is a contiguous block of memory.
  Arrays are indexed items, meaning they use a number to identify each item.
\end{definition}

\nameref{def:Array}s are declared with the following syntax \cinline{type arr_name[size]}.
This declaration also implicitly allocates space for its storage.
This allocates the space from the function \textbf{stack} \textit{NOT} the heap.

There is \textbf{NO} metadata about the array, such as its length.
This means:
\begin{itemize}[noitemsep]
\item \textbf{NO} implicit size
\item \textbf{NO} bounds checking
\end{itemize}

All of the ways to statically declare, allocate, and define an array are shown in \Cref{lst:Arrays}.

\begin{listing}[h!tbp]
\csourcefile{./C_Primer-Sections/Arrays/Code/arrays.c}
\caption{Arrays, their Declaration and Definition}
\label{lst:Arrays}
\end{listing}

The syntax of an \nameref{def:Array} actually is syntactic sugar for \nameref{def:Pointer} arithmetic.
The name we give to the array is actually the name of the \textbf{\nameref{def:Pointer}} that points to the beginning of the array, the zeroth element in the array.
This is shown in \Cref{lst:Array_Pointer_Similarity}.

\begin{listing}[h!tbp]
\csourcefile{./C_Primer-Sections/Arrays/Code/array-pointer-similarity.c}
\caption{Array-Pointer Similarity}
\label{lst:Array_Pointer_Similarity}
\begin{minted}[frame=lines,linenos]{console}
$ ./a.out
3rd element in x, using array syntax: 3
3rd element in x, using pointer syntax: 3
\end{minted}
\end{listing}

Because arrays are fancy pointers, you can also typecast arrays.

It is important to note that arrays that are present on the stack of a program \textbf{MUST} be of fixed size.
This is because the compiler needs to know how much space to allocate for each procedure in the program.
However, if we want a dynamically sized array, we will need to turn to the heap.
Using the heap is discussed in \Cref{sec:Memory_Management}.
To access the heap from our program, we need a \nameref{def:Pointer} to point to the array on the heap.
By doing this, we still have a fixed-size stack, because pointers are of fixed size, but we allow use of dynamically sized data structures.

%%% Local Variables:
%%% mode: latex
%%% TeX-master: "../C_Primer"
%%% End:
