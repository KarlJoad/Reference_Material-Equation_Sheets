\section{Memory Allocation}\label{sec:Memory_Allocation}
Because C is a language that does not provide many abstractions, it also requires the programmer to remember and manage their memory usage.
So, \textbf{YOU} must be the one to manage the memory, there is \textbf{NO} built-in garbage collector for you to use.

Memory allocation is done on the heap of the program's execution space in memory.
When you allocate memory in your program, you are actually requesting the operating system to give you the memory you want.

\subsection{\texttt{malloc}}\label{subsec:malloc}
This is the simplest function of all possible memory allocation functions.
\texttt{malloc}:
\begin{itemize}
\item Takes one argument:
  \begin{enumerate}
  \item The number of bytes to allocate.
  \end{enumerate}
\item Returns a \textbf{POINTER} to the front of the allocated memory.
\end{itemize}

\texttt{malloc} {\large{\textbf{\emph{DOES NOT}}}} initialize memory, so it will be garbage.

\subsection{\texttt{calloc}}\label{subsec:calloc}
This is quite similar to malloc.
\texttt{calloc}:
\begin{itemize}
\item Takes 2 arguments:
  \begin{enumerate}
  \item The number of spaces to allocate, for example the number of elements in an array.
  \item The number of bytes to allocate, for the type being stored.
  \end{enumerate}
\item Returns a \textbf{POINTER} to the front of the allocated memory.
\end{itemize}

\texttt{calloc} {\large{\textbf{\emph{ZEROS}}}} memory, so this does have a slight performance penalty.

\subsection{\texttt{realloc}}\label{subsec:realloc}
\texttt{realloc} is used to \textbf{REALLOCATE} an existing memory location.
\begin{itemize}
\item Takes 2 arguments:
  \begin{enumerate}
  \item The pointer to the memory location previously allocated with either \texttt{malloc} or \texttt{calloc}.
  \item The amount of memory to reallocate, in bytes.
  \end{enumerate}
\item If the \texttt{NULL} pointer is passed to \texttt{realloc}, it will behave exactly like \texttt{malloc}.
\item Returns a \textbf{POINTER} to the front of the reallocated memory
\end{itemize}

\subsection{\texttt{free}}\label{subsec:free}
\texttt{free} is used to free memory that was previously allocated, removing from the programming space entirely.
\begin{itemize}
\item Takes 1 argument:
  \begin{enumerate}
  \item A pointer to the memory to be deallocated.
  \end{enumerate}
\item Returns \texttt{void}.
\end{itemize}

%%% Local Variables:
%%% mode: latex
%%% TeX-master: "../C_Primer"
%%% End: