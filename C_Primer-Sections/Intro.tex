\section{Introduction}\label{sec:Intro}
This document is intended for programmers that are newer to the C language and its facilities.
This is meant as a quick, supplementary, reference document for these programmers.
Many of the code examples in this document are either from IIT's CS 351 course, or Kernighan and Ritchie's~\cite{KernighanRitchieCProg} manual, \citetitle{KernighanRitchieCProg}.

\subsection{Properties}\label{subsec:Properties}
C is referred to as ``low-level'' today.
That means there are relatively few abstractions and few ``syntactic sugars'' for expressing computations.
This means when you write C, you can typically guess what the assembly would look like, which also lends itself to C's execution speed.
However, this also means that you have \textbf{VERY FEW} built-in language protections for your computations.
So, you open yourself up to a whole new class of problems when developing and writing your programs.
The language will not protect you, but C compilers will typically attempt to throw warnings or errors about the most egregious errors you may write.

Here are some properties of C that you should know about.

\subsubsection{Imperative}\label{subsubsec:Imperative}
C is an imperative language, meaning you express computation as a series of steps.
This is based off a finite-state based understanding of computation.

This stands in stark contrast to functional languages, which express computation as function applications to expressions.

\subsubsection{Procedural}\label{subsubsec:Procedural}
A procedural language allows you to organize repeated computations into logical blocks, typically referred to as functions or procedures.

\subsubsection{Lexically Scoped}\label{subsubsec:Lexically_Scoped}
C is lexically scoped because variables inside of procedures cannot exist outside of their logical blocks.

This stands in contrast to Dynamically Scoped languages, where variables are sometimes available outside of the written scope for that variable.
Bash is an example of a dynamically scoped language.

\subsubsection{Statically Typed}\label{subsubsec:Statically_Typed}
C is statically typed, because the types of \textbf{ALL} expressions \textbf{MUST} be specified \textbf{BEFORE compilation}.
This also means that when something is declared to be a certain type, it stays that way, unlike Python.
This means that type checking happens during compilation, ensuring that all expressions have well-formed types.

To ensure program flexibility, we also introduce type-casting, where we either widen the type or narrow it.
For example, taking an \texttt{int} and turning it into a \texttt{long} is a widening type cast, which are usually safe.
This means that sometimes we must deal with type polymorphism, although C's handling of this class of problems is iminal and quite basic.

\subsubsection{Weakly Type Checked}\label{subsubsec:Weakly_Type_Checked}
C is technically strongly typed on its operations, however, it becomes weakly typed because compilers cannot always ensure well-formedness of expressions when \nameref{sec:Pointers} are used.
Because C is weakly typed, you are not always guaranteed that when you access data that you are interpreting the bytes the right way.

C becomes weakly-typed because you can typecast pointers, pull values out of unions with different types, and in general do weird things with anything in memory.
This also means that the compiler might not throw type-checking errors during the compilation phase of program development.
Some of these pointer issues will only arise after running the program and ensuring that it is tested properly.

Some of these issues are illustrated in \Cref{lst:Weak_Type_Checking}.
\begin{listing}[h!tbp]
\csourcefile{./C_Primer-Sections/Intro/Code/weak-type-checking.c}
\caption{Illustration of C's Weak Type Checking}
\label{lst:Weak_Type_Checking}
\begin{minted}[frame=lines,linenos]{console}
$ ./a.out
c = 'D', i = 1, u = 4294967295, f =10.00000
r1 = 4, r2 = 5
\end{minted}
\end{listing}

%%% Local Variables:
%%% mode: latex
%%% TeX-master: "../C_Primer"
%%% End:
