\section{Introduction}\label{sec:Intro}
This document is intended for programmers that are newer to the C language and its facilities.
This is meant as a quick, supplementary, reference document for these programmers.
Many of the code examples in this document are either from IIT's CS 351 course, or Kernighan and Ritchie's~\cite{KernighanRitchieCProg} manual, \citetitle{KernighanRitchieCProg}.

\subsection{Properties}\label{subsec:Properties}
C is referred to as ``low-level'' today.
That means there are relatively few abstractions and few ``syntactic sugars'' for expressing computations.
This means when you write C, you can typically guess what the assembly would look like, which also lends itself to C's execution speed.
However, this also means that you have \textbf{VERY FEW} built-in language protections for your computations.
So, you open yourself up to a whole new class of problems when developing and writing your programs.
The language will not protect you, but C compilers will typically attempt to throw warnings or errors about the most egregious errors you may write.

Here are some properties of C that you should know about.


%%% Local Variables:
%%% mode: latex
%%% TeX-master: "../C_Primer"
%%% End:
