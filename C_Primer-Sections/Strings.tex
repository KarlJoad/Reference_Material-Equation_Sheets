\section{Strings}\label{sec:Strings}
In C, strings are actually character arrays.
These arrays are terminated with the null character, \cinline{\0}; it's numerical value is 0.
This is shown in \Cref{lst:String_Character_Array}.

\begin{listing}[h!tbp]
\csourcefile{./C_Primer-Sections/Strings/Code/char-array.c}
\caption{String/Character Array}
\label{lst:String_Character_Array}
\begin{minted}[frame=lines,linenos]{console}
$ ./a.out
hello world!
\end{minted}
\end{listing}

\texttt{printf} treats strings as a character array terminated by a null character.

\subsection{String Utilities}\label{subsec:String_Utilities}
All the functions shown below string are from \texttt{<string.h>}.

\begin{description}[noitemsep]
\item[\cinline{char *strcpy(char *dest, const char *src)}] Copy characters from source to destination array, including the \cinline{\0}.
\item[\cinline{char *strcat(char *dest, const char *src)}] Concatenates 2 strings, removing null bytes and adding one to the very end of the destination string.
\item[\cinline{int strcmp(const char *s1, const char *s2)}] Compares strings byte-by-byte, returning whichever string is greater than the other, as determined by the component characters' ASCII code.
\item[\cinline{size_t strlen(const char *s)}] Finds length of string by incrementing a counter until it finds the null character.
\item[\cinline{void *memcpy(void *dest, const void *src, size_t n)}] Copies the contents of one memory location to another.
\item[\cinline{void* memmove(void *dest, const void *src, size_t n)}] Moves the contents of one memory location to another.
\end{description}

%%% Local Variables:
%%% mode: latex
%%% TeX-master: "../C_Primer"
%%% End:
