\section{Compilation}\label{sec:Compilation}
C, along with C++ can be quite painful to compile for larger projects.
You could manually compile every \texttt{.c} file with \texttt{gcc}.
Makefiles help us manage this.

\subsection{Stages}\label{subsec:Compilation_Stages}
\begin{enumerate}[noitemsep]
\item Preprocessing
  \begin{itemize}[noitemsep]
  \item Preprocessor directives starting with \cinline{#}
  \item Text substitution
  \item Macros
  \item Conditional compilation
  \item Performs complete textual substitution behind the scenes
  \end{itemize}
\item Compile
  \begin{itemize}[noitemsep]
  \item From source language to object code/binary
  \end{itemize}
\item Link
  \begin{itemize}[noitemsep]
  \item Put inter-related object codes together
  \item Resolve calls/references and definitions
  \item Put absolute/relative addresses into the binary for the =call= instruction
  \item Want to support /selective/ public APIs
  \item Don't always want to allow linking a call to a definition
  \end{itemize}
\end{enumerate}

\subsection{Makefiles}\label{subsec:Makefiles}
Makefiles allow for:
\begin{itemize}[noitemsep]
\item Incremental compilation
\item Automated compilation
\end{itemize}

These are written as a list of targets, prerequisites, and directives.
In addition, they can include variables to simplify typing things out.
They also support arbitrarily long lists, file globbing (like regular expressions), and a couple of other things.
An example \texttt{Makefile} is shown in \Cref{lst:Example_Makefile}.

\begin{listing}[h!tbp]
\inputminted[frame=lines,linenos]{makefile}{./C_Primer-Sections/Compilation/Code/Example_Makefile}
\caption{Example \texttt{Makefile}}
\label{lst:Example_Makefile}
\end{listing}

%%% Local Variables:
%%% mode: latex
%%% TeX-master: "../C_Primer"
%%% End:
