\section{Pointers}\label{sec:Pointers}
This should technically go in \Cref{sec:Syntax}, but pointers deserve their own section.

\begin{definition}[Pointer]\label{def:Pointer}
A \emph{pointer} is a variable declared to store a memory address.
With this memory address, we can refer to data in-memory.
The size of the pointer is determined by the architecture of the CPU.\@
\end{definition}

A pointer is designated by its \textbf{DECLARED} type, \textbf{NOT} its contents.
This allows the data the pointer points to to be re-interpreted based on the declared type of the pointer.
This is shown in \Cref{lst:Pointers_Reinterpret_Data}.

\begin{listing}[h!tbp]
\csourcefile{./C_Primer-Sections/Pointers/Code/pointers-reinterpret-data.c}
\caption{Pointers Reinterpret Data}
\label{lst:Pointers_Reinterpret_Data}

\begin{minted}[frame=lines,linenos]{console}
$ ./a.out
ip: 63
cp: ?
fp: 0.000000
\end{minted}
\end{listing}

\subsection{Pointer Syntax}\label{subsec:Pointer_Syntax}.
The syntax that \nameref{def:Pointer}s use can sometimes be confusing for new programmers.
So, we will break down each portion of a pointer and its usage to more fully understand them.


%%% Local Variables:
%%% mode: latex
%%% TeX-master: "../C_Primer"
%%% End:
