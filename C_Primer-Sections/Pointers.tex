\section{Pointers}\label{sec:Pointers}
This should technically go in \Cref{sec:Syntax}, but pointers deserve their own section.

\begin{definition}[Pointer]\label{def:Pointer}
A \emph{pointer} is a variable declared to store a memory address.
With this memory address, we can refer to data in-memory.
The size of the pointer is determined by the architecture of the CPU.\@
\end{definition}

A pointer is designated by its \textbf{DECLARED} type, \textbf{NOT} its contents.
This allows the data the pointer points to to be re-interpreted based on the declared type of the pointer.
This is shown in \Cref{lst:Pointers_Reinterpret_Data}.

\begin{listing}[h!tbp]
\csourcefile{./C_Primer-Sections/Pointers/Code/pointers-reinterpret-data.c}
\caption{Pointers Reinterpret Data}
\label{lst:Pointers_Reinterpret_Data}

\begin{minted}[frame=lines,linenos]{console}
$ ./a.out
ip: 63
cp: ?
fp: 0.000000
\end{minted}
\end{listing}

\subsection{Pointer Syntax}\label{subsec:Pointer_Syntax}.
The syntax that \nameref{def:Pointer}s use can sometimes be confusing for new programmers.
So, we will break down each portion of a pointer and its usage to more fully understand them.

\subsubsection{Declaration}\label{subsubsec:Pointer_Declaration}
\nameref{def:Pointer}s are declared using the same type name as the data type they will store.
In addition, you \textbf{MUST} add a \cinline{*} to the declaration.
The placement of this symbol does not matter, but for clarity, most programmers put it on the variable name.
However, it can be attached to the type as well.

\Cref{lst:Pointer_Declaration} shows how to do this.
\begin{listing}[h!tbp]
\csourcefile{./C_Primer-Sections/Pointers/Code/pointer-declaration.c}
\caption{Pointer Declaration}
\label{lst:Pointer_Declaration}
\end{listing}

\subsubsection{Getting an Address}\label{subsubsec:Getting_an_Address}
If you already have a name that points to the actual value, and you want the address of that value, you can use the \cinline{&} unary operator.
Its usage is shown in \Cref{lst:Address_Operator}.
\begin{listing}[h!tbp]
\csourcefile{./C_Primer-Sections/Pointers/Code/address-operator.c}
\caption{Address \cinline{&} Operator}
\label{lst:Address_Operator}
\begin{minted}[frame=lines,linenos]{console}
$ ./a.out
x: 15
*xp: 15
xp: 0x7ffc4868a204
\end{minted}
\end{listing}

\subsubsection{Dereferencing}\label{subsubsec:Dereferencing_Pointers}
In a major point of confusion for the C language, the \cinline{*} operator is \textbf{also} used to \textbf{dereference} the pointer!
An example of declaring, getting the address of a variable, and the dereferencing of pointers is shown in \Cref{lst:Dereferencing_Pointers}.
\begin{listing}[h!tbp]
\csourcefile{./C_Primer-Sections/Pointers/Code/dereferencing-pointers.c}
\caption{Dereferencing Pointers}
\label{lst:Dereferencing_Pointers}

\begin{minted}[frame=lines,linenos]{console}
$ ./a.out
i=10, j=20, *p=20, *q=20
\end{minted}
\end{listing}

\subsection{\emph{Why} have Pointers?}\label{subsec:Why_Have_Pointers}
\nameref{def:Pointer}s are partly a cross-over from assembly, as C is really just a thin wrapper around that.
However, it does also allow direct access to memory and allow \emph{us} to make many optimizations we could not make otherwise.
This means you must \textbf{manage memory yourself}.

\begin{itemize}
\item In C, everything is \emph{ALWAYS} \nameref{par:Pass_by_Value}.
\item This even happens on composite data structures, like \cinline{struct}s.
\item \nameref{def:Pointer}s enable us to perform \emph{actions at a distance}, essentially allow us to get \nameref{par:Pass_by_Reference} without explicitly allowing that.
\item This also allows functions very deep in the call stack to affect variables earlier in the stack.
\end{itemize}

One good reason is illustrated in \Cref{lst:int_Pointer_Swap}.

\begin{listing}[h!tbp]
\csourcefile{./C_Primer-Sections/Pointers/Code/int-pointer-swap.c}
\caption{Swap with Pointers}
\label{lst:int_Pointer_Swap}
\end{listing}

\subsection{Uninitialized Pointers}\label{subsec:Uninitialized_Pointers}
Even though a \nameref{def:Pointer} can only store memory addresses, it behaves exactly like a regular variable for most puposes.

\begin{itemize}
\item When a \nameref{def:Pointer} is declared, they are given their underlying storage, which \textbf{will contain garbage}!
\item If you dereference this pointer, you dereference garbage, which is undefined behavior.
  \begin{itemize}
  \item If you're lucky, this is will result in a \texttt{SEGFAULT}, terminating program execution.
  \item If you're unlucky, unknown results may happen, and you might never know about it.
  \item You are assured that your program will not exceed its own memory space though, which means your program cannot crash another.
  \end{itemize}
\end{itemize}

\Cref{lst:Uninitialized_Pointer} gives an example of what could happen if you don't initialize your \nameref{def:Pointer}s.
\begin{listing}[h!tbp]
\csourcefile{./C_Primer-Sections/Pointers/Code/uninitialized-pointers.c}
\caption{Uninitialized Pointers}
\label{lst:Uninitialized_Pointer}

\begin{minted}[frame=lines,linenos]{console}
$ ./a.out
[1]    29456 segmentation fault (core dumped)  ./a.out
\end{minted}
\end{listing}

\subsection{\texorpdfstring{\cinline{NULL}}{\texttt{NULL}} Pointers}\label{subsec:NULL_Pointers}
The \cinline{NULL} \nameref{def:Pointer} is \textbf{NEVER} returned by the \cinline{&} operator.
It is usually a smart idea to define pointers when you declare them, and if you don't immediately define them with a usable value, to define them with \cinline{NULL}.
This helps prevent many runtime issues that the compiler cannot check for.
\Cref{lst:NULL_Pointer-Checker} gives an example of using the \cinline{NULL} pointer this way.

\begin{itemize}[noitemsep]
\item The \cinline{NULL} pointers is safe to use as a predicate, in \texttt{if} statements, for example.
  This usage is shown in \Cref{lst:NULL_Pointer-Sentinel}.
\item Written as \texttt{0} or \texttt{NULL} in \emph{pointer context}
  \begin{itemize}
  \item Typically \cinline{#define NULL 0}
  \end{itemize}
\end{itemize}

\begin{listing}[h!tbp]
\csourcefile{./C_Primer-Sections/Pointers/Code/null-pointers-checker.c}
\caption{\texorpdfstring{\cinline{NULL}}{\texttt{NULL}} Pointer as Checker}
\label{lst:NULL_Pointer-Checker}

\begin{minted}[frame=lines,linenos]{console}
$ ./a.out
[1]    29456 segmentation fault (core dumped)  ./a.out
\end{minted}
\end{listing}

\begin{listing}[h!tbp]
\csourcefile{./C_Primer-Sections/Pointers/Code/null-pointers-sentinel.c}
\caption{\texorpdfstring{\cinline{NULL}}{\texttt{NULL}} Pointer as Sentinel}
\label{lst:NULL_Pointer-Sentinel}
\end{listing}

\subsection{\texorpdfstring{\cinline{void}}{\texttt{void}} Pointer}\label{subsec:void_Pointer}
This is a pointer type that says there is no type.
It is compatible with \textbf{EVERY} other \nameref{def:Pointer} type, but it \textbf{CANNOT} be used directly.
The \cinline{void} pointer \textbf{MUST} be typecast before being used in \textbf{ANY} way.

%%% Local Variables:
%%% mode: latex
%%% TeX-master: "../C_Primer"
%%% End:
