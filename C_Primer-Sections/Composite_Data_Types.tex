\section{Composite Data Types}\label{sec:Composite_Data_Types}
This is \textit{similar} to objects in Object-Oriented Programming, or the \mintinline{haskell}{datatype} in Haskell.

\subsection{\texorpdfstring{\cinline{struct}}{\texttt{struct}}}\label{subsec:struct}
The \cinline{struct} keyword allows us to put multiple separate data types together and refer to them by their field name.
You access the fields by using the \cinline{.} operator.
An example of this is shown in \Cref{lst:struct_Usage}.

\begin{listing}[h!tbp]
\csourcefile{./C_Primer-Sections/Composite_Data_Types/Code/struct.c}
\caption{\texorpdfstring{\cinline{struct}}{\texttt{struct}} Usage}
\label{lst:struct_Usage}
\end{listing}


%%% Local Variables:
%%% mode: latex
%%% TeX-master: "../C_Primer"
%%% End:
