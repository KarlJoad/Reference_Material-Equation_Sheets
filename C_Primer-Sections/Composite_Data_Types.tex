\section{Composite Data Types}\label{sec:Composite_Data_Types}
This is \textit{similar} to objects in Object-Oriented Programming, or the \mintinline{haskell}{datatype} in Haskell.

\subsection{\texorpdfstring{\cinline{struct}}{\texttt{struct}}}\label{subsec:struct}
The \cinline{struct} keyword allows us to put multiple separate data types together and refer to them by their field name.
You access the fields by using the \cinline{.} operator.
An example of this is shown in \Cref{lst:struct_Usage}.

\begin{listing}[h!tbp]
\csourcefile{./C_Primer-Sections/Composite_Data_Types/Code/struct.c}
\caption{\texorpdfstring{\cinline{struct}}{\texttt{struct}} Usage}
\label{lst:struct_Usage}
\end{listing}

\subsection{\texorpdfstring{\cinline{union}}{\texttt{union}}}\label{subsec:union}
The \cinline{union} keyword allows us to define a single type that can be of one type from many.
However, these are \textbf{NOT} like Haskell's \mintinline{haskell}{datatype} or Rust's union, in that we do not have type protections about accessing this data.
\Cref{lst:union_Usage} gives an example of this.

\begin{listing}[h!tbp]
\csourcefile{./C_Primer-Sections/Composite_Data_Types/Code/union.c}
\caption{\texorpdfstring{\cinline{union}}{\texttt{union}} Usage}
\label{lst:union_Usage}
\end{listing}


%%% Local Variables:
%%% mode: latex
%%% TeX-master: "../C_Primer"
%%% End:
