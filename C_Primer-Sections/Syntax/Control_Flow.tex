\subsection{Control Flow}\label{subsec:Control_Flow}
Control flow is the idea of changing the direction a program executes based on some predicate.
Whether this change in direction is to a new direction is because of a change in state, or because something must be repeated is irrelevant.

\subsubsection{Branching}\label{subsubsec:Branching}
Branching has to deal with the changing of a program's control flow based on a predicate to perform some separate actions based on the state of the predicate.
There are 4 types for this:
\begin{description}[noitemsep]
\item[\nameref{par:if_Statement}] The most basic change in flow control.
  If the predicate provided is \texttt{true}-thy, performs an action, then returns to normal.

\item[\nameref{par:if_else_Statement}] If the predicate is \texttt{true}-thy, then perform some action, if the predicate is \texttt{false}, then perform some other action.

\item[\nameref{par:if_elseif_else_Statement}] If the predicate in the first \texttt{if} is \texttt{true}-thy, then that branch is taken.
  If the first predicate is \texttt{false}, then the \texttt{else if}'s predicate is checked for truth.
  The the \texttt{else if}'s predicate is \texttt{false}, then execution continues through any other \texttt{else if}s that may be present.
  If none of the \texttt{if} or \texttt{else if}s' predicates were \texttt{true}-thy, then the \texttt{else} is taken.
\item[\nameref{par:switch_case_Statement}] The \texttt{switch}-\texttt{case} statement allows you to choose from many different paths based on the \textbf{VALUE} of some expression.
\end{description}

\paragraph{\texorpdfstring{\cinline{if}}{\texttt{if}}}\label{par:if_Statement}
The \texttt{if} statement has a very basic syntax, shown below in \Cref{lst:if_Statement}.
Typically, this is only used when there needs to be a small change in the state of the program based off the predicate's value.

\begin{listing}[h!tbp]
\csourcefile{./C_Primer-Sections/Syntax/Code/if-statement.c}
\caption{\texorpdfstring{\cinline{if}}{\texttt{if}} Statement Syntax}
\label{lst:if_Statement}
\end{listing}

\paragraph{\texorpdfstring{\cinline{if} \cinline{else}}{\texttt{if}-\texttt{else}}}\label{par:if_else_Statement}
The \cinline{if else} statement is used to perform two distinct actions based on the predicate's value.
The syntax of this is shown in \Cref{lst:if_else_Statement}.

\begin{listing}[h!tbp]
\csourcefile{./C_Primer-Sections/Syntax/Code/if-else-statement.c}
\caption{\texorpdfstring{\cinline{if else}}{\texttt{if else}} Statement Syntax}
\label{lst:if_else_Statement}
\end{listing}

\paragraph{\texorpdfstring{\cinline{if} \cinline{else if} \cinline{else}}{\texttt{if}-\texttt{else if}-\texttt{case}}}\label{par:if_elseif_else_Statement}
The \cinline{if else if else} is used to choose between $n$ options.
However, this comes with the downside that \textbf{EACH} potential path's predicate \textbf{MUST} be evaluated before any action can occur.
The syntax for this is shown in \Cref{lst:if_elseif_else_Statement}.

\begin{listing}[h!tbp]
\csourcefile{./C_Primer-Sections/Syntax/Code/if-elseif-else-statement.c}
\caption{\texorpdfstring{\cinline{if else if else}}{\texttt{if else if else}} Statement Syntax}
\label{lst:if_elseif_else_Statement}
\end{listing}

\paragraph{\texorpdfstring{\cinline{switch} \cinline{case}}{\texttt{switch}-\texttt{case}}}\label{par:switch_case_Statement}
The \cinline{switch case} statement is used to choose between $n$ different options.
However, unlike the \nameref{par:if_elseif_else_Statement}, the expression is only evaluated the once, and the branch is then jumped to in constant time, making this a better option for making one decision of many.
The syntax for this is shown in \Cref{lst:switch_case_Statement}.

\begin{listing}[h!tbp]
\csourcefile{./C_Primer-Sections/Syntax/Code/switch-case-statement.c}
\caption{\texorpdfstring{\cinline{switch case}}{\texttt{switch case}} Statement Syntax}
\label{lst:switch_case_Statement}
\end{listing}

\subsubsection{Repetition}\label{subsubsec:Repetition}
Here, we want to perform some action a number of times.
There are 3 structures for performing a repeating action:
\begin{description}[noitemsep]
\item[\nameref{par:while_Loop} Loop] This is generally used when the number of repetitions is unknown or uncountable at any given point in time.
  It continues to repeat until the predicate ceases to be \texttt{true}-thy or an explicit \texttt{break} occurs.

\item[\nameref{par:for_Loop} Loop] The \texttt{for} loop is used when the number of repetitions is both known and countable.

\item[\nameref{par:dowhile_Loop} Loop] The \texttt{do while} loop is the same as the \texttt{while} loop, but the difference is that the body of the loop is executed \textbf{ONCE before} evaluating the predicate.
\end{description}

\paragraph{\texorpdfstring{\cinline{while}}{\texttt{while}}}\label{par:while_Loop}
The \cinline{while} loop is typically used when the number of reptitions is unknown or uncountable.
This allows for easy definition of infinite loops, allowing for a program to continue execution until some condition has been met.
The syntax for this is shown in \Cref{lst:while_Loop}.

\begin{listing}[h!tbp]
\csourcefile{./C_Primer-Sections/Syntax/Code/while-loop.c}
\caption{\texorpdfstring{\cinline{while}}{\texttt{while}} Loop Syntax}
\label{lst:while_Loop}
\end{listing}

\paragraph{\texorpdfstring{\cinline{for}}{\texttt{for}}}\label{par:for_Loop}
The \cinline{for} loop is used when the number of repetitions is both countable.
This type of loop can be modelled by the \nameref{par:while_Loop} as well, but this helps ensure that all the components of the loop are present, helping prevent infinite loops.
The syntax of this is shown in \Cref{lst:for_Loop}.

\begin{listing}[h!tbp]
\csourcefile{./C_Primer-Sections/Syntax/Code/for-loop.c}
\caption{\texorpdfstring{\cinline{for}}{\texttt{for}} Loop Syntax}
\label{lst:for_Loop}
\end{listing}


%%% Local Variables:
%%% mode: latex
%%% TeX-master: "../../C_Primer"
%%% End:
