\subsection{Control Flow}\label{subsec:Control_Flow}
Control flow is the idea of changing the direction a program executes based on some predicate.
Whether this change in direction is to a new direction is because of a change in state, or because something must be repeated is irrelevant.

\subsubsection{Branching}\label{subsubsec:Branching}
Branching has to deal with the changing of a program's control flow based on a predicate to perform some separate actions based on the state of the predicate.
There are 4 types for this:
\begin{description}[noitemsep]
\item[\nameref{par:if_Statement}] The most basic change in flow control.
  If the predicate provided is \texttt{true}-thy, performs an action, then returns to normal.

\item[\nameref{par:if_else_Statement}] If the predicate is \texttt{true}-thy, then perform some action, if the predicate is \texttt{false}, then perform some other action.

\item[\nameref{par:if_elseif_else_Statement}] If the predicate in the first \texttt{if} is \texttt{true}-thy, then that branch is taken.
  If the first predicate is \texttt{false}, then the \texttt{else if}'s predicate is checked for truth.
  The the \texttt{else if}'s predicate is \texttt{false}, then execution continues through any other \texttt{else if}s that may be present.
  If none of the \texttt{if} or \texttt{else if}s' predicates were \texttt{true}-thy, then the \texttt{else} is taken.
\item[\nameref{par:switch_case_Statement}] The \texttt{switch}-\texttt{case} statement allows you to choose from many different paths based on the \textbf{VALUE} of some expression.
\end{description}


%%% Local Variables:
%%% mode: latex
%%% TeX-master: "../../C_Primer"
%%% End:
