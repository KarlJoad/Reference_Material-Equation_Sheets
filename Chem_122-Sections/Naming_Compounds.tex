\section{Naming Compounds} \label{sec:Naming Compounds}
There is a specific way to name compounds, however, it depends on the type of elements in the compound.

\subsection{Naming Inorganic Compounds} \label{subsec:Naming Inorganic Compounds}
Inorganic compounds are non-carbon-based compounds.
There are a 2 rules for naming these compounds:
\begin{enumerate}[noitemsep, nolistsep]
\item If there are only 2 elements in the compound and the first element only has 1 atom, then \emph{DO NOT USE A PREFIX}
\item Otherwise, use the numerical prefix for the number of atoms present
  \begin{enumerate}[noitemsep, nolistsep]
  \item 1 atom $\rightarrow$ Mono-
  \item 2 atoms $\rightarrow$ Di-
  \item 3 atoms $\rightarrow$ Tri-
  \item 4 atoms $\rightarrow$ Tetra-
  \item so on and so forth
  \end{enumerate}
\end{enumerate}

\subsection{Naming Metallic Compounds} \label{subsec:Naming Metallic Compounds}
Metallic compounds are ones that have \nameref{def:Metal}s in them.
The rules for naming metallic compounds are similar to the inorganic ones.
The only difference is naming the metal itself.
You use the roman numerals when writing the compound and say the number out loud.
For example, rust is \ch{ Fe_{2}O_{3}}, is written Iron (\RomanNumeral{3}) Oxide.