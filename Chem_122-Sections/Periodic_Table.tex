\section{The Periodic Table} \label{sec:Periodic Table}
The Periodic Table was developed as a way to categorize the many chemical elements in the world.
Most of the elements on the table are naturally occurring, but some of the heaviest elements have been synthesized in laboratories.
\begin{definition}[The Periodic Table] \label{def:Periodic Table}
  \emph{The Periodic Table} is an arrangement of elements by their atomic number, Z, or the number of protons in the nucleus.
  \begin{remark}
    The number of protons is the \emph{ONLY} thing that determines what an element is.
    If an element has a different number of neutrons, that is an \nameref{def:Isotope}.
    If an element has a different number of electrons, that is an \nameref{def:Ion}.
  \end{remark}
  \begin{remark}
    On \nameref{def:Periodic Table}, the elements are always in their electroneutral form.
    This means they have the same number of protons and electrons.
  \end{remark}
\end{definition}

A single element drawn out of \nameref{def:Periodic Table} will look like \figurename.
\begin{figure}[h!]
  %\begin{tikzpicture}
\end{tikzpicture}
  \caption{Example Element from Periodic Table}
  \label{fig:Single Element Periodic Table}
\end{figure}