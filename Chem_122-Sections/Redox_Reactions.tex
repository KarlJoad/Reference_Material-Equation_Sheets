\section{Reduction-Oxidation Reactions} \label{sec:Redox Reactions}
These reactions are somewhat unique because the electrons \emph{ONLY} move around but do not change the output of the reaction.

\begin{definition}[Reduction-Oxidation Reactions] \label{def:Redox Reactions}
  \emph{Reduction-Oxidation Reactions} or \emph{Redox Reactions} are a type of chemical reaction.
  These reactions take place due to differences in electronegativity and each element's \nameref{def:Oxidation State}.
  The compounds that undergo a Redox Reaction are either \nameref{def:Reduced}/\nameref{def:Oxidizing Agent} or \nameref{def:Oxidized}/\nameref{def:Reducing Agent}.
\end{definition}

\begin{definition}[Oxidation State] \label{def:Oxidation State}
  An \emph{oxidation state} describes the degree of loss of electrons in a chemical compound.
\end{definition}

There are some rules for how elements in a compound are assigned their \nameref{def:Oxidation State}.

\begin{enumerate}[noitemsep, nolistsep]
  \item Elemental form of element $\rightarrow$ $0$
  \item Monoatomic ion $\rightarrow$ \nameref{def:Oxidation State} = Charge
  \begin{itemize}
    \item Peroxides \ch{O2^{2-}} are always -2
    \item Hidrides \ch{H^{-}} are always -1
  \end{itemize}
  \item Fluorine is always $-1$
  \item Groups 1 and 2 are always $+1$ and $+2$, respectively
  \item Hydrogen, \ch{H}, will usually be $+1$
  \item Oxygen, \ch{O}, will usually be $-2$
  \item The sum of all \nameref{def:Oxidation State}s \textbf{\emph{MUST}} = Charge
\end{enumerate}

\begin{definition}[Reduced] \label{def:Reduced}
  A compound that is \emph{reduced} is one that \textbf{\emph{GAINS}} electrons.
  A \nameref{def:Reduced} compound is the one that is \nameref{def:Oxidizing Agent}.
\end{definition}

\begin{definition}[Oxidized] \label{def:Oxidized}
  A compound that is \emph{oxidized} is one that \textbf{\emph{LOSES}} electrons.
  A \nameref{def:Oxidized} compound is the \nameref{def:Reducing Agent}.
\end{definition}

\begin{definition}[Oxidizing Agent] \label{def:Oxidizing Agent}
  An \emph{oxidizing agent} or \emph{oxidant} is one that \textbf{\emph{GAINS}} electrons.
  A \nameref{def:Oxidizing Agent} is one that is \nameref{def:Reduced}.
\end{definition}

\begin{definition}[Reducing Agent] \label{def:Reducing Agent}
  A \emph{reducing agent} or \emph{reductant} is one that \textbf{\emph{LOSES}} electrons.
  A \nameref{def:Reducing Agent} is one that is \nameref{def:Oxidized}.
\end{definition}

\begin{example}[]{Redox Reaction 1}
  Given the chemical equation below, which compounds are \nameref{def:Reduced} and \nameref{def:Oxidized}, and which are the \nameref{def:Oxidizing Agent} and \nameref{def:Reducing Agent}?
  
  \ch{SO2 (g) + I2 (aq) -> SO3 (g) + I^{-}}
  
  \tcblower
  
  Solution from Redox and More.
\end{example}