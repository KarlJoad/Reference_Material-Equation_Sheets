\section{Eletrolytes} \label{sec:Eletrolytes}
\begin{definition}[Eletrolyte] \label{def:Electrolyte}
  An \emph{electrolyte} is an ion that dissociates into its constituent ions when dissolved in water.
  For example, salt.
\end{definition}

\begin{definition}[Strong Electrolyte] \label{def:Strong Electrolyte}
  A \emph{strong electrolyte} is an \nameref{def:Electrolyte} that \emph{completely} dissociates into its constituent ions when dissolved in water.
\end{definition}

\begin{definition}[Weak Electrolyte] \label{def:Weak Electrolyte}
  A \emph{weak electrolyte} is an \nameref{def:Electrolyte} that \emph{somewhat} dissociates into its constituent ions when dissolved in water.
\end{definition}

\begin{definition}[Non-Electrolyte] \label{def:Non-Electrolyte}
  A \emph{non-electrolyte} is a molecule that \textbf{\emph{DOES NOT}} dissolve into its constituent ions when dissolved in water.
  For example, sugar.
\end{definition}

There are some easy and general rules for determining if something is an electrolyte.
\begin{enumerate}[noitemsep, nolistsep]
\item \emph{Ionic} compound means it is a \nameref{def:Strong Electrolyte}
\item \emph{Molecular} compounds have 3 potential electrolytic states
  \begin{enumerate}
  \item Strong Acids/Bases are \nameref{def:Strong Electrolyte}s
  \item Weak Acids/Bases are \nameref{def:Weak Electrolyte}s
  \item All others are \nameref{def:Non-Electrolyte}s
  \end{enumerate}
\end{enumerate}

Make sure you refer to your solubility chart when referring to the strong and weak electrolytes above.
Your solubility chart/table is in your textbook or online.
These will give you which molecules are soluble in water.

\subsection{Acids} \label{subsec:Acids}
\begin{definition}[Acid] \label{def:Acid}
  An \emph{acid} is an \nameref{def:Electrolyte} that dissociates in water.
  It is a compound that when dissociated makes the solution end up with additional \ch{H^{+}} ions.
  These ions will bond with any free hydroxide molecule (\ch{OH^{-}}).
  \begin{remark}
    You may hear these solutions being called ``corrosive''.
  \end{remark}
  \begin{remark}
    Acids can be dangerous chemicals!
    Depending on the concentration of the acid, they can eat through skin, concrete, and even glass!!
    \textbf{Handle these with extreme caution!!}
  \end{remark}
\end{definition}

There are 2 types of acids:
\begin{enumerate}[noitemsep, nolistsep]
  \item \nameref{def:Bronstad-Lowry Acid}
  \begin{itemize}[noitemsep, nolistsep]
    \item \nameref{def:Arrhenius Acid}s are a special form of \nameref{def:Bronstad-Lowry Acid}s
  \end{itemize}
  \item \nameref{def:Lewis Acid}
\end{enumerate}

\begin{definition}[Br\o nstad-Lowry Acid] \label{def:Bronstad-Lowry Acid}
  A \emph{Br\o nstad-Lowry acid} is a general name for compounds that are proton (\ch{H^{+}}) donors.
  
  \begin{remark}
    In general, \nameref{def:Acid}s refer to \nameref{def:Bronstad-Lowry Acid}s.
  \end{remark}
\end{definition}

\begin{definition}[Arrhenius Acid] \label{def:Arrhenius Acid}
  An \emph{Arrhenius acid} are aqueous \nameref{def:Bronstad-Lowry Acid}s that are proton donors.
  Usually when they donate their proton, they form a \nameref{def:Hydronium} ion \ch{H3O^{+}}.
\end{definition}

\begin{definition}[Lewis Acid] \label{def:Lewis Acid}
  A \emph{Lewis acid} is a molecule that forms a covalent bond with an electron pair.
  This generalizes such that an acid is a chemical species that accepts electron pairs directly or by releasing protons.
  
  \begin{remark}
    \nameref{def:Lewis Acid}s will usually be referred to as Lewis acids.
  \end{remark}
\end{definition}

\begin{definition}[Hydronium] \label{def:Hydronium}
  \emph{Hydronium} is a water molecule with an extra proton (\ch{H3O^{+}}).
  In \nameref{def:Redox Reactions} that utilize \nameref{def:Arrhenius Acid}s the proton donor gives the water solution the extra proton.
  The water becomes an \nameref{def:Hydronium} ion.
\end{definition}

\subsection{Bases} \label{subsec:Bases}
\begin{definition}[Base] \label{def:Base}
  A \emph{base} is an \nameref{def:Electrolyte} that dissociates in water.
  It is a compound that when dissociated makes the solution end up with additional \ch{OH^{-}} ions.
  These ions will bond with any free hydrogen ions (\ch{H^{+}}).
  
  \begin{remark}
    You may hear these solutions being called ``caustic''.
  \end{remark}

  \begin{remark}
    Bases can be dangerous chemicals!
    Depending on the concentration of the base, they can eat through plastic, skin, and many other items!!
    \textbf{Handle these with extreme caution!!}
  \end{remark}
\end{definition}

\subsection{Dilution} \label{subsec:Dilution}
There is a simple formula to follow when diluting acids and bases.

\begin{equation} \label{eq:Dilution}
  M_{1} V_{1} = M_{2} V_{2}
\end{equation}

\begin{itemize}[noitemsep, nolistsep]
  \item $M$ is the molarity of the solution
  \item $V$ is the volume of the solution
\end{itemize}
%%% Local Variables:
%%% mode: latex
%%% TeX-master: "../Chem_122-Reference_Sheet"
%%% End:
