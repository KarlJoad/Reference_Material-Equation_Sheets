\section{Introduction} \label{sec:Introduction}
\subsection{Basic Chemistry Things} \label{subsec:Basic Chemistry Things}
\begin{definition}[Chemistry] \label{def:Chemistry}
  \emph{Chemistry} is the study of \nameref{def:Matter} and its changes.
  \begin{remark}
    We tend to use the macroscopic world to visualize the microscopic world.
  \end{remark}
\end{definition}

\begin{definition}[Matter] \label{def:Matter}
  \emph{Matter} is ``stuff'' that has both mass and volume.
\end{definition}

\begin{definition}[Scientific Method] \label{def:Scientific Method}
  The \emph{scientific method} is a systematic approach to research that utilizes qualitative or quantitative measurements.
\end{definition}

\begin{definition}[Hypothesis] \label{def:Hypothesis}
  A \emph{hypothesis} is a tentative explanation that will be tested using the \nameref{def:Scientific Method}.
\end{definition}

\begin{definition}[Law] \label{def:Law}
  A \emph{law} is a statement of a relationship between phenomena that is alwasy the same, under the same conditions.
  \begin{remark}
    These tend to be drawn from large amounts of data.
  \end{remark}
\end{definition}

\begin{example}[]{Law 1}
  Chlorine (Cl) is a highly reactive gas.
\end{example}

\begin{example}[]{Law 2}
  Matter is neither created nor destroyed.
\end{example}

\begin{definition}[Theory] \label{def:Theory}
  A \emph{theory} is a unifying principle that explains a body of facts based on facts and laws.
  These are constantly tested for validity.
  \begin{remark}
    A \nameref{def:Hypothesis} can turn into a \nameref{def:Theory} with enough experimentation and acceptance.
  \end{remark}
\end{definition}

\begin{example}[]{Theory 1}
  Reactivity of elements depends on the element's electron $\left( e^{-} \right)$ configuration.
\end{example}

\begin{example}[]{Theory 2}
  All matter is made up of tiny, indestructible particles, called atoms.
\end{example}

\subsection{Matter} \label{subsec:Matter}
As \Cref{def:Matter} said, matter must have both volume and mass.
Matter can have several \nameref{def:Matter State}s.

\begin{definition}[Matter State] \label{def:Matter State}
  A \emph{matter state} or \emph{state of matter} is just the configuration of atoms in a particular material.
  There are 3 common states:
  \begin{enumerate}[noitemsep, nolistsep]
  \item Solid
  \item Liquid
  \item Gas
  \end{enumerate}
\end{definition}

But, \nameref{def:Matter} can be categorized in different ways as well.
The Matter Tree is one way to categorize them.
\begin{figure}[h!]
  \begin{tikzpicture}
\end{tikzpicture}
  \caption{Matter Tree}
  \label{fig:Matter Tree}
\end{figure}

Others include:
\begin{itemize}[noitemsep, nolistsep]
  \item Atomic Weight
  \item Chemical Properties
  \item Physical Properties
  \item \nameref{sec:Periodic Table}
  \item And many others
\end{itemize}

\subsection{Significant Figures} \label{subsec:Sig Figs}
\begin{definition}[Significant Figures] \label{def:Sig Figs}
  \emph{Significant Figures} or \emph{Sig Figs} are ways to handle uncertainty in our measurements.
  In general, we treat the data that we receive as inexact numbers, thus we must confirm our suspicions several times.
  Additionally, \nameref{def:Precision} and \nameref{def:Accuracy} are used interchangably when they shouldn't.
\end{definition}

\begin{definition}[Precision] \label{def:Precision}
  \emph{Precision} is defined as the closeness of data points to each other.
  If you think about a dartboard, this would be all the darts landing right next to each other.
\end{definition}

\begin{definition}[Accuracy] \label{def:Accuracy}
  \emph{Accuracy} is defined as how close your data is to the predicted true real value.
  \begin{remark}
    Generally, this must be done with a minimum of 3 trials, but more will yield more accurate data.
  \end{remark}
\end{definition}

\subsubsection{Rules for Significant Figures} \label{subsec:Rules for Sig Figs}
\begin{enumerate}[noitemsep, nolistsep]
\item 0s between any non-zero digit is significant ($100$, both 0s are significant)
\item 0s at the beginning of an integer are not significant ($010 = 10$)
\item 0s at the end are significant is the number is a decimal (0.003050 has 4 sig figs)
\item 0s at the end, if there is no decimal/fractional portion, are not significant (16000 has 2 sig figs)
\end{enumerate}

\begin{example}[]{Addition and Subtraction of Significant Figures}
  Add $20.3056$, $1.34$, and $54.2$ and keeping in mind significant figures.

  \tcblower

  You want to find the least precise number first, in this case it is $54.2$ because it only has one decimal place.
  This also determines how many decimal places to go past on the solution.
  Adding these 3 together gives $75.8456$, but because of $54.2$, it becomes $75.8$.
\end{example}

\begin{example}[]{Multiplication and Division of Significant Figures}
  Multiply $3.4456$ and $2.15$ keeping in mind significant figures.

  \tcblower

  You find the number with the least number of significant figures and use that.
  So, $2.15$ has 3 sig figs, that's the same amount your answer must have.
  \[ 3.4456 \times 2.15 = 7.40804 \]
  But because we can only have 3 sig figs in our answer, $7.41$ is our solution.
\end{example}
