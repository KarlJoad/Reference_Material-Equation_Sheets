\section{Chemical Equations} \label{sec:Chemical Equations}
\nameref{def:Chemical Equations} form one of the key pillars of chemistry.
\begin{definition}[Chemical Equations] \label{def:Chemical Equations}
  Just like in mathematical equations, \emph{chemical equations} define the way certain chemical reactions take place.
  Just like in mathematics, there are some rules that are followed.
  \begin{enumerate}[noitemsep, nolistsep]
    \item You need to have the same elements on each side of the reaction
    \item There must be reactants (input) and products (output)
    \item There must be a \emph{\nameref{def:Species}} on each element/compound
    \item There must be an arrow, $\rightarrow$ or $\leftrightarrow$ showing the direction of reaction
    \begin{itemize}[noitemsep, nolistsep]
      \item $\leftrightarrow$ is for reactions that can occur in both directions. Nearly all do not follow this.
    \end{itemize}
    \item Elements \textbf{\emph{MUST}} follow the \nameref{def:Law of Conservation of Matter}. Refer to \Cref{sec:Stoichiometry}, \nameref{sec:Stoichiometry} for further information.
  \end{enumerate}
\end{definition}

\begin{definition}[Species] \label{def:Species}
  An element's or compound's \emph{species} is the current state of matter.
  It can only be one of the following:
  \begin{itemize}[noitemsep, nolistsep]
    \item Solid - (s)
    \item Liquid - ($\ell$)
    \item Gas - (g)
    \item Aqueous - (aq)
  \end{itemize}
\end{definition}

\begin{example}[]{Chemical Equation 1}
  What are the coefficients ($a, b, c, d$) for this chemical equation? \\
  \ch{$a$~CH4 (g) + $b$~Cl2 (g) -> $c$~C Cl4 ($\ell$) + $d$~HCl (g)}
  
  \tcblower
  
  \ch{1 CH4 (g) + 4 Cl2 (g) -> 1 C Cl4 ($\ell$) + 4 HCl (g)}
\end{example}

\begin{example}[]{Chemical Equation 2}
  What are the coefficients ($a, b, c, d$) for this chemical equation? \\
  \ch{$a$~Al4C3 (s) + $b$~H2O ($\ell$) -> $c$~Al(OH)3 (s) + $d$~CH4 (g)}
  
  \tcblower
  
  \ch{1 Al4C3 (s) + 12 H2O ($\ell$) -> 4 Al(OH)3 (s) + 3 CH4 (g)}
\end{example}

\begin{example}[]{Chemical Equation 3}
  What are the coefficients ($a, b, c, d$) for this chemical equation? \\
  \ch{$a$~AgNO3 (aq) + $b$~Na2SO4 (aq) -> $c$~Ag2SO4 (s) + $d$~NaNO3 (aq)}
  
  \tcblower
  
  \ch{2 AgNO3 (aq) + 1 Na2SO4 (aq) -> 1 Ag2SO4 (s) + 2 NaNO3 (aq)}
\end{example}

\begin{example}[]{Chemical Equation 4}
  What are the coefficients ($a, b, c, d$) for this chemical equation? \\
  \ch{$a$~BF3 (g) + $b$~H2O ($\ell$) -> $c$~HF ($\ell$) + $d$~H3BO3 (aq)}
  
  \tcblower
  
  \ch{1 BF3 (g) + 3 H2O ($\ell$) -> 3 HF (aq) + 1 H3BO3 (aq)}
\end{example}

\begin{example}[]{Chemical Equation 5}
  What are the coefficients ($a, b, c, d$) for this chemical equation? \\
  \ch{$a$~C3H8 (g) + $b$~SO2 (g) -> $c$~H2O (g) + $d$~CO2 (g)}
  
  \tcblower
  
  \ch{1 C3H8 (g) + 5 SO2 (g) -> 4 H2O (g) + 3 CO2 (g)}
\end{example}

\begin{example}[]{Chemical Equation 6}
  What are the coefficients ($a, b, c, d$) for this chemical equation? \\
  \ch{$a$~C6H14 ($\ell$) + $b$~O2 (g) -> $c$~CO2 (g) + $d$~H2O (g)}
  
  \tcblower
  
  \ch{2 C6H14 ($\ell$) + 19 O2 (g) -> 14 CO2 (g) + 12 H2O (g)}
\end{example}

\subsection{Limiting Reactants} \label{subsec:Limiting Reactants}
\begin{definition}[Limiting Reactants] \label{def:Limiting Reactants}
  The \emph{limiting reactant(s)} is the reactant(s) that limits a reaction's total output.
  This means that there is some amount of excess reagent from the other, no limited reactant.
\end{definition}

\begin{example}[]{Limiting Reactants 1}
  Given the chemical equation \ch{2 Al (s) + 3 Cl2 (g) -> 2 AlCl3 (s)} and 1.50 \si{\mole} of \ch{Al} and 3.0 \si{\mole} of \ch{Cl2}.
  Which reactant limits the reaction?

  \tcblower
  
  \begin{align*}
    1.50 \si{\mole} \ch{Al} \left( \frac{3 \si{\mole} \ch{Cl2}}{2 \si{\mole} \ch{Al}} \right) &= 2.25 \si{\mole} \ch{Cl2} \\
    3.00 \si{\mole} \ch{Cl2} \left( \frac{2 \si{\mole} \ch{Cl2}}{3 \si{\mole} \ch{Cl2}} \right) &= 2.00 \si{\mole} \ch{Al}
  \end{align*}
  Since 2.25 \si{\mole} of \ch{Cl2} would be consumed when we have 1.50 \si{\mole} of \ch{Al}, and 2.00 \si{\mole} \ch{Al} would be consumed when we have 3.00 \si{\mole} \ch{Cl2}, the \ch{Al} is the limiting reactant.

  This means that 1.50 \si{\mole} of \ch{AlCl3} will be created from this reaction.
  \begin{equation*}
    1.50 \si{\mole} \ch{Al} \left( \frac{2 \si{\mole} \ch{AlCl3}}{2 \si{\mole} \ch{Al}} \right) = 1.50 \si{\mole} \ch{AlCl3}
  \end{equation*}

  We will have .75 \si{\mole} \ch{Cl2} left after the reaction takes place.
  \begin{equation*}
    3.00 \si{\mole} \ch{Cl2} - 2.25 \si{\mole} \ch{Cl2} = .75 \si{\mole} \ch{Cl2}
  \end{equation*}
\end{example}

\begin{example}[]{Limiting Reactants 2}
  Given the chemical equation \ch{Zn (s) + 2 AgNO3 (aq) -> 2 Ag (s) + Zn(NO3)2 (aq)} and 2.00 \si{\gram} of \ch{Zn} and 2.50 \si{\gram} of \ch{AgNO3}.
  Which reactant limits the reaction?

  \tcblower
\end{example}
%%% Local Variables:
%%% mode: latex
%%% TeX-master: "../Chem_122-Reference_Sheet"
%%% End:
