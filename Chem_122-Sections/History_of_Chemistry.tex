\section{History of Chemistry} \label{sec:History of Chemistry}
\subsection{Dalton} \label{subsec:Dalton}
Dalton created the first meaningful definition of an atom.
He made several claims:
\begin{enumerate}[noitemsep, nolistsep]
\item Atoms are very small
\item The same element's atoms are identical, but different elements have different atoms.
\item Atoms are neither created, nor destroyed (\nameref{def:Law of Conservation of Matter})
\item Compounds are 2 or more elements together.
\end{enumerate}

\begin{definition}[Law of Conservation of Matter] \label{def:Law of Conservation of Matter}
  Matter is neither created, nor destroyed.
  It can \emph{only} change forms.
\end{definition}

\subsection{Thomson} \label{subsec:Thomson}
Thomson made several discoveries about atoms and their constituent particles.
For his experimentation, he used a cathode ray (a beam of positively charged ions) and magnets.
He discovered the \nameref{def:Electron}.

\begin{definition}[Electron] \label{def:Electron}
  The \emph{electron} is one of 3 particles that make up an atom.
  Electrons are negatively charged particles that are contained \emph{outside} of the nucleus.
  An electron's position and velocity can not be known simultaneously.
  This is known as the \nameref{def:Heisenbergs Uncertainty Principle}
\end{definition}
The cathode ray deflected from the ``negative'' magnetic plate to the ``positive''.
From this he calculated the \nameref{def:Magnetic Deflection}.

\begin{definition}[Magnetic Deflection] \label{def:Magnetic Deflection}
  When Thomson deflected his cathode ray with magnets, he measured how far it deviated from the starting line.
  \begin{equation} \label{eq:Magnetic Deflection}
    1.76 \times 10^{8} \si{\coulomb / \gram}
  \end{equation}
\end{definition}

\subsection{Millikan} \label{subsec:Millikan}
Millikan made 2 significant contributions to the model of the atom.
He discovered the charge of a single \nameref{def:Electron} and the mass of a single \nameref{def:Electron}.
\begin{equation} \label{eq:Electron Charge}
  1.602 \times 10^{-19} \si{\coulomb}
\end{equation}
\begin{equation} \label{eq:Electron Mass}
  9.10938 \times 10^{-28} \si{\gram}
\end{equation}

Both the \nameref{eq:Electron Charge} and \nameref{eq:Electron Mass} were drawn from \nameref{subsec:Thomson}'s work with \nameref{def:Magnetic Deflection}.

\subsection{Becquerel} \label{subsec:Becquerel}
Becquerel did his work with high energy radiation caused by radioactivity.
He found that there were 3 types of particles released by radioactive decay.
\begin{enumerate}[noitemsep, nolistsep]
  \item Alpha Particles ($\alpha$) - Positively charged particles that are charged helium atoms.
  \item Beta Particles ($\beta$) - Negatively charged particles that are essentially high speed electrons.
  \item Gamma particles ($\gamma$) - Uncharged particles that have next to no mass and are quite energetic.
\end{enumerate}

\subsection{Johnson} \label{subsec:Johnson}
Johnson developed one of the first models for single atoms.
This was called the \nameref{def:Plum Pudding Model}.

\begin{definition}[Plum Pudding Model] \label{def:Plum Pudding Model}
  The \emph{Plum Pudding Model} is a visualization of an atom.
  It is based off the plum pudding desert, which was one of Johnson's favorites.
  The positive charges were held together in a ``soft'' shell.
  The electrons were evenly distributed on the outer surface of the positively charged ``plum.''
  One of the hallmarks of this model was that the entire atom was \emph{not} empty space.
\end{definition}

\subsection{Rutherford} \label{subsec:Rutherford}
Rutherford performed experiments with $\alpha$-particles.
He ``shot'' these particles at a piece of gold foil and observed what happened with after the particles passed through.

\begin{definition}[Rutherford Model] \label{def:Rutherford Model}
  This is, more or less, the next model of the atom.
  Rutherford challenged \nameref{subsec:Johnson}'s \nameref{def:Plum Pudding Model} of the atom.
  When Rutherford sent the beam of alpha particles through the gold foil, he found most of them didn't deflect, i.e. hit any thing.
  However some did, and were scattered in all directions.
  Rutherford proved that atoms are mostly empty space, with the positive charges being held in a small dense area he called the \emph{nucleus}.
  \begin{remark}
    One thing to note about the \emph{nucleus} in the \nameref{def:Rutherford Model} is that there was no concept of the neutron yet.
    Since neutrons are uncharged particles, they were not discovered until much later.
  \end{remark}
\end{definition}

\subsection{Atomic Mass Units, \si{\atomicmassunit}} \label{subsec:AMU}
Eventually the neutron was discovered and the current understanding of the fundamental particles in atoms was completed.
These include the:
\begin{itemize}[noitemsep, nolistsep]
  \item Proton ($p^{+}$)
  \item Neutron ($N^{0}$)
  \item Electron ($e^{-}$)
\end{itemize}

The mass of each of these particles was found and the Atomic Mass Unit was developed to make calculations easier.
\begin{equation} \label{eq:AMU Equivalency}
  1 \si{\atomicmassunit} = 1.66054 \times 10^{-24} \si{\gram}
\end{equation}
Thus, the atomic mass for each particles is as follows:
\begin{itemize}[noitemsep, nolistsep]
  \item Proton ($p^{+}$) - $1.672623 \times 10^{-24} \si{\gram} = 1.0074 \si{\atomicmassunit}$
  \item Neutron ($N^{0}$) - $1.674927 \times 10^{-24} \si{\gram} = 1.0087 \si{\atomicmassunit}$
  \item Electron ($e^{-}$) - $9.109383 \times 10^{-28} \si{\gram} = 5.486 \times 10^{-4} \si{\atomicmassunit}$
\end{itemize}