\section{History of Chemistry} \label{sec:History of Chemistry}
\subsection{Dalton} \label{subsec:Dalton}
Dalton created the first meaningful definition of an atom.
He made several claims:
\begin{enumerate}[noitemsep, nolistsep]
\item Atoms are very small
\item The same element's atoms are identical, but different elements have different atoms.
\item Atoms are neither created, nor destroyed (\nameref{def:Law of Conservation of Matter})
\item Compounds are 2 or more elements together.
\end{enumerate}

\begin{definition}[Law of Conservation of Matter] \label{def:Law of Conservation of Matter}
  Matter is neither created, nor destroyed.
  It can \emph{only} change forms.
\end{definition}

\subsection{Thomson} \label{subsec:Thomson}
Thomson made several discoveries about atoms and their constiuent particles.
For his experimentation, he used a cathode ray (a beam of postively charged ions) and magnets.
He discovered the \nameref{def:Electron}.

\begin{definition}[Electron] \label{def:Electron}
  The \emph{electron} is one of 3 particles that make up an atom.
  Electrons are negatively charged particles that are contained \emph{outside} of the nucleus.
  An electron's position and velocity can not be known simultaneously.
  This is known as the \nameref{def:Heisenbergs Uncertainty Principle}
\end{definition}
The cathode ray deflected from the ``negative'' magnetic plate to the ``postive''.
From this he calculated the \nameref{def:Magnetic Deflection}.

\begin{definition}[Magnetic Deflection] \label{def:Magnetic Deflection}
  When Thomson deflected his cathode ray with magnets, he measured how far it deviated from the starting line.
  \begin{equation} \label{eq:Magnetic Deflection}
    1.76 \times 10^{8} \si{\coulomb / \gram}
  \end{equation}
\end{definition}
