\section{Exam 1 Equations} \label{sec:Exam 1}
	\subsection{Ch. 2 - Interatomic Forces}
		\begin{equation}
			E_{N} = E_{A} + E_{R} = -\frac{A}{r} + \frac{B}{r^{n}}
		\end{equation}
		\begin{equation}
			A = \frac{1}{4 \pi \epsilon_{0}} \left( Z_{1} e \right) \left( Z_{2} e \right)
		\end{equation}
		\begin{itemize}[noitemsep]
			\item This equation works for both $A$ and $B$
			\item $\epsilon_{0}$ = $8.85 \times 10^{-12} \si{\farad / \meter}$
			\item $e$ = $1.602 \times 10^{-19} \si{\coulomb}$
			\item $r$ - Radius in \si{\meter}
		\end{itemize}

		\begin{equation}
			F_{A} = \frac{\left( 1.602 \times 10^{-19} \right)^{2}}{4 \pi \left( 8.85 \times 10^{-12} \right) r^{2}} \left( \lVert Z_{1} \rVert \right) \left( \lVert Z_{2} \rVert \right)
		\end{equation}
		\begin{itemize}[noitemsep]
			\item $F_{A}$ - Force of Attraction
			\item $r$ - Distance in \si{\meter}
			\item $Z$ - Number of Valence Electrons
			\item $F_{A}$ - Interatomic Force in \si{\newton}
			\item $-F_{A} = F_{R}$ - Attractive and Repulsive Force Equal and Opposite
		\end{itemize}

		\begin{align}
			\text{Force} &= \frac{dE}{dr} \\
			\text{Elastic Modulus} &= \frac{dF}{dr}
		\end{align}
	
		\begin{equation}
			\text{\%IC} = \left( 1 - e^{\frac{\left( x_{A}-x_{B} \right)^{2}}{4}} \right) \times 100\%
		\end{equation}
		\begin{itemize}[noitemsep]
			\item $\text{\%IC}$ - \% Ionic Character
			\item $x$ - Electronegativities
		\end{itemize}
	
	\subsection{Ch. 3 - Structures of Metals/Ceramics}
		\subsubsection{Lattice Parameters}
			\begin{align}
				a_{\text{BCC}} &= \frac{4r}{\sqrt{3}} \\
				a_{\text{FCC}} &= \frac{4r}{\sqrt{2}} \\
				a_{\text{HCP}} &= \frac{c}{1.633} 
			\end{align}
			\begin{itemize}[noitemsep]
				\item $a$ - Lattice Parameter
				\item $r$ - Radius of atom
			\end{itemize}

		\subsubsection{Volume of Hexagonal Prism}
			\begin{equation}
				V_{H} = \frac{3 \sqrt{3}}{2} a^{2} h
			\end{equation}

		\subsubsection{Densities}
			\begin{equation}
				\rho = \frac{nA}{V_{C} N_{A}}
			\end{equation}
			\begin{itemize}[noitemsep]
				\item $n$ - Number of atoms/unit cell
				\item $A$ - Molar Mass of Material
				\item $V_{C}$ - Volume of Unit Cell in \si{\centi \meter \cubic}
				\item $N_{A}$ - Avogadro's Number ($6.022 \times 10^{23}$)
			\end{itemize}

			\begin{equation}
				\text{Planar Density} = \frac{\frac{\text{Atoms}}{\text{2D Unit Area}}}{\frac{\text{Area}}{\text{2D Repeat Unit}}}
			\end{equation}
			\begin{equation}
				\text{Linear Density} = \frac{\text{\# of Atoms in a Direction}}{\text{Magnitude of Linear Vector}}
			\end{equation}
			\begin{itemize}[noitemsep]
				\item The repeat units/vector magnitude are in terms of atomic radii
			\end{itemize}

			\begin{equation}
				\text{APF} = \frac{\frac{\text{Atoms}}{\text{Unit Cell}} \left( \frac{4}{3} \pi \left( \text{atom radius} \right)^{3} \right)}{\text{Unit Cell Volume}}
			\end{equation}	

		\subsubsection{Thermal Expansion}
			\begin{equation}
				\frac{\Delta L}{L_{0}} = \alpha \left( T_{2} - T_{1} \right)
			\end{equation}
			\begin{itemize}[noitemsep]
				\item $E \uparrow$, $T_{m} \uparrow$
				\item $E \uparrow$, $\alpha \downarrow$
			\end{itemize}

		\subsubsection{Convert between Coordinates}
			\begin{equation}
				\left[ XYZ \right] = \left[ a_{1} a_{2} a_{3} c \right]
				\begin{aligned}
					a_{1} &= \frac{1}{3} \left( 2X - Y \right) \\
					a_{2} &= \frac{1}{3} \left( 2Y - X \right) \\
					a_{3} &= - \left( a_{1} + a_{2} \right) \\
					c &= Z \\
					a_{1} + a_{2} + a_{3} &= 0 \\
				\end{aligned}
			\end{equation}
	
		\subsubsection{Planes}
			\begin{enumerate}[noitemsep]
				\item Given $x$, $y$, $z$ as intersects
				\item Convert to $\frac{1}{x}$, $\frac{1}{y}$, $\frac{1}{z}$
				\item Reduce to smallest common denominator
				\item Leave as $\left( \frac{1}{x} \frac{1}{y} \frac{1}{z} \right)$
			\end{enumerate}
	
		\subsubsection{Light Refraction}
			\begin{equation}
				D = \frac{n \lambda}{2 \sin \theta}
			\end{equation}
			\begin{itemize}[noitemsep]
				\item $n$ = 1
				\item $\lambda$ - Wavelength in \si{\nano \meter}
				\item $\theta$ - Angle of Incidence
				\item $\theta$ is usually given as $2 \theta$. Be careful
			\end{itemize}
	
		\subsubsection{Randoms}
			\begin{equation}
				D_{HKL} = \frac{a}{\sqrt{h^{2}+k^{2}+l^{2}}}
			\end{equation}
			\begin{itemize}[noitemsep]
				\item ONLY for cubic structures
			\end{itemize}