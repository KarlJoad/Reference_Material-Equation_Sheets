\section{Exam 5 Equations}
	\subsection{Ch. 12 - Mechanical Properties of Ceramics} \label{subsec:Ch12 Mechanical Properties of Ceramics}
	Overall, ceramics are categorized by:
		\begin{enumerate}[noitemsep]
			\item Ionic Bonds
			\item Few Slip Systems
			\item Dislocations \emph{\textbf{\textsc{cannot}}} move
		\end{enumerate}
	
		\subsubsection{Strength of Ceramics} \label{subsubsec:Ceramics Strength}
			\begin{equation} \label{eq:Ceramics Max Strength}
				\sigma_{\text{Max}} = 2 \sigma_{0} \sqrt{\frac{a}{\rho}}
			\end{equation}
			\begin{itemize}[noitemsep]
				\item $\sigma_{\text{Max}}$ - Maximum Strength
				\item $a$ - Half major axis length
				\item $\rho$ - Radius of Crack tip
			\end{itemize}
		
			\begin{equation} \label{eq:Ceramics Fracture Strength}
				\sigma_{\text{FS}} = A \sqrt{\frac{E \gamma}{a}}
			\end{equation}
			\begin{itemize}[noitemsep]
				\item $\sigma_{\text{FS}}$ - Fracture Strength
				\item $\gamma$ - Surface Energy
				\item $A$ - Constant
			\end{itemize}
			
			\begin{equation} \label{eq:Flexural Strength}
				\sigma_{\text{FXS}} = \frac{3 F_{f} L}{2 b d^{2}}
			\end{equation}
			\begin{equation}
				\sigma_{\text{FXS}} = \frac{F_{f} L}{\pi R^{3}}
			\end{equation}
			\begin{itemize}[noitemsep]
				\item $\sigma_{\text{FXS}}$ - Flexural Strength
			\end{itemize}
	
		\subsubsection{Elasticity of Ceramics} \label{subsubsec:Ceramics Elasticity}
			\begin{equation}
				E = \frac{F}{\delta} \cdot \frac{L^{3}}{4bd^{2}}
			\end{equation}
			\begin{equation}
				E = \frac{F}{\delta} \cdot \frac{L^{3}}{12 \pi R^{4}}
			\end{equation}
			\begin{itemize}[noitemsep]
				\item $\delta$ - Midpoint Deflection
			\end{itemize}
		
		
	\subsection{Ch. 15 - Mechanical Properties of Polymers} \label{subsec:Ch15 Mechanical Properties of Polymers}
		\begin{equation}
			\sigma_{\text{TS}} = \sigma_{\text{TS}\infty} - \frac{A}{\bar{M}_{n}}
		\end{equation}
		\subsubsection{Molecular Weight} \label{subsubsec:Polymer Molecular Weight}
			\begin{itemize}[noitemsep]
				\item $E$ - No relationship
				\item $\sigma_{\text{TS}}$ - Increases
				\item More carbon chain entanglement
			\end{itemize}
		
		\subsubsection{Degree of Crystallinity} \label{subsubsec:Polymer Degree of Crystallinity}
			\begin{itemize}[noitemsep]
				\item $E$ - Increases
				\item $\sigma_{\text{TS}}$ - Increases
				\item Stronger secondary bonds between carbon chains
			\end{itemize}
		
		\subsubsection{Deformation by Drawing} \label{subsubsec:Polymer Deformation by Drawing}
			\begin{itemize}[noitemsep]
				\item $E$ - Increases
				\item $\sigma_{\text{TS}}$ - Increases
				\item Carbon chains are aligning, effectively testing the Carbon-Carbon bonds
			\end{itemize}
		
			\paragraph{Annealing/Pre-Drawing} \label{par:Polymer Annealing/Pre-Drawing}
				\begin{itemize}[noitemsep]
					\item $E$ - Increases
					\item $\sigma_{\text{TS}}$ - Increases
				\end{itemize}
			
			\paragraph{Post-Drawing} \label{par:Polymer Post-Drawing}
				\begin{itemize}[noitemsep]
					\item $E$ - Decreases
					\item $\sigma_{\text{TS}}$ - Decreases
				\end{itemize}
	
	\subsection{Ch. 18 - Electrical Properties}
		\subsubsection{Resistivity/Conductance} \label{subsubsec:Resistivity/Conductance}
			\begin{equation} \label{eq:Resistivity}
				\rho = \frac{RA}{\ell}
			\end{equation}
			\begin{equation} \label{eq:Conductance}
				\sigma = \frac{1}{\rho} = \frac{\ell}{RA}
			\end{equation}
			\begin{itemize}[noitemsep]
				\item $\rho$ - Resistivity
				\item $\sigma$ - Conductance
				\item $R$ - Resistance
				\item $A$ - Cross-Sectional Area
				\item $\ell$ - Length
			\end{itemize}
			\begin{equation} \label{eq:Total Resistivity}
				\rho = \rho_{\text{Thermal}} + \rho_{\text{Impurity}} + \rho_{\text{Deform}}
			\end{equation}
		
		\subsubsection{Conductance with Doping} \label{subsubsec:Conductance with Doping}
			\begin{equation} \label{eq:Conductance with Doping}
				\sigma = n \lvert e \rvert \mu_{e} + p \lvert e \rvert \mu_{h}
			\end{equation}
			\begin{itemize}[noitemsep]
				\item $p=0$ In Metals
				\item If $p=n$ it is intrinsic
				\item $e$ = $1.602 \times 10^{-19} \si{\coulomb}$
			\end{itemize}
		
		\subsubsection{Electron Drift Velocity} \label{subsubsec:Electron Drift Velocity}
			\begin{equation} \label{eq:Electron Drift Velocity}
				v_{d} = \frac{\mu_{e}}{\mu_{h}} \xi
			\end{equation}
			\begin{equation} \label{eq:Time for Electron to Travel}
				t = \frac{\ell}{v_{d}}
			\end{equation}