\documentclass[10pt,letterpaper,final,twoside,notitlepage]{article}
\usepackage[margin=.5in]{geometry} % 1/2 inch margins on all pages
\usepackage[utf8]{inputenc} % Define the input encoding
\usepackage[USenglish]{babel} % Define language used
\usepackage{amsmath,amsfonts,amssymb}
\usepackage{amsthm} % Gives us plain, definition, and remark to use in \theoremstyle{style}
\usepackage{mathtools} % Allow for text and math in align* environment.
\usepackage{thmtools}
\usepackage{thm-restate}
\usepackage{graphicx}

\usepackage[
backend=biber,
style=alphabetic,
citestyle=authoryear]{biblatex} % Must include citation somewhere in document to print bibliography
\usepackage{hyperref} % Generate hyperlinks to referenced items
\usepackage[nottoc]{tocbibind} % Prints the Reference/Bibliography in TOC as well
\usepackage[noabbrev,nameinlink]{cleveref} % Fancy cross-references in the document everywhere
\usepackage{nameref} % Can make references by name to places
\usepackage{caption} % Allows for greater control over captions in figure, algorithm, table, etc. environments
\usepackage{subcaption} % Allows for multiple figures in one Figure environment
\usepackage[binary-units=true]{siunitx} % Gives us ways to typeset units for stuff
\usepackage{csquotes} % Context-sensitive quotation facilities
\usepackage{enumitem} % Provides [noitemsep, nolistsep] for more compact lists
\usepackage{chngcntr} % Allows us to tamper with the counter a little more
\usepackage{empheq} % Allow boxing of equations in special math environments
\usepackage[x11names]{xcolor} % Gives access to coloring text in environments or just text, MUST be before tikz
\usepackage{tcolorbox} % Allows us to create boxes of various types for examples
\usepackage{tikz} % Allows us to create TikZ and PGF Pictures
\usepackage{ctable} % Greater control over tables and how they look
\usepackage{diagbox} % Allow us to have shared diagonal cells in tables
\usepackage{multirow} % Allow us to have a single cell in a table span multiple rows
\usepackage{titling} % Put document information throughout the document programmatically
\usepackage[linesnumbered,ruled,vlined]{algorithm2e} % Allows us to write algorithms in a nice style.

\counterwithin{figure}{section}
\counterwithin{table}{section}
\counterwithin{equation}{section}
\counterwithin{algocf}{section}
\crefname{algocf}{algorithm}{algorithms}
\Crefname{algocf}{Algorithm}{Algorithms}
\setcounter{secnumdepth}{4}
\setcounter{tocdepth}{4} % Include \paragraph in toc
\crefname{paragraph}{paragraph}{paragraphs}
\Crefname{paragraph}{Paragraph}{Paragraphs}

% Create a theorem environment
\theoremstyle{plain}
\newtheorem{theorem}{Theorem}[section]
% Create a numbered theorem-like environment for lemmas
\newtheorem{lemma}{Lemma}[theorem]

% Create a definition environment
\theoremstyle{definition}
\newtheorem{definition}{Defn}
\newtheorem{corollary}{Corollary}[section]
% \begin{definition}[Term] \label{def:}
%   Make sure the term is emphasized with \emph{term}.
%   This ensures that if \emph is changed, it shows up everywhere
% \end{definition}

% Create a numbered remark environment numbered based on definition
% NOTE: This version of remark MUST go inside a definition environment
\theoremstyle{remark}
\newtheorem{remark}{Remark}[definition]
%\counterwithin{definition}{subsection} % Uncomment to have definitions use section.subsection numbering

% Create an unnumbered remark environment for general use
% NOTE: This version of remark has NO restrictions on placement
\newtheorem*{remark*}{Remark}

% Create a special list that handles properties. It can be broken and restarted
\newlist{propertylist}{enumerate}{1} % {Name}{Template}{Max Depth}
% [newlistname, LevelsToApplyTo]{formatting options}
\setlist[propertylist, 1]{label=\textbf{(\roman*)}, ref=\textbf{(\roman*)}, noitemsep, nolistsep}
\crefname{propertylisti}{property}{properties}
\Crefname{propertylisti}{Property}{Properties}

% Create a special list that handles enumerate starting with lower letters. Breakable/Restartable.
\newlist{boldalphlist}{enumerate}{1} % {Name}{Template}{Max Depth}
% [newlistname, LevelsToApplyTo]{formatting options}
\setlist[boldalphlist, 1]{label=\textbf{(\alph*)}, ref=\alph*, noitemsep, nolistsep} % Set options

\newlist{nocrefenumerate}{enumerate}{1} % {Name}{Template}{Max Depth}
% [newlistname, LevelsToApplyTo]{formatting options}
\setlist[nocrefenumerate, 1]{label=(\arabic*), ref=(\arabic*), noitemsep, nolistsep}

% Create a list that allows for deeper nesting of numbers. By default enumerate only allows depth=4.
\newlist{nestednums}{enumerate}{6}
% [newlistname, LevelsToApplyTo]{formatting options}
\setlist[nestednums]{noitemsep, label*=\arabic*.}

\tcbuselibrary{breakable} % Allow tcolorboxes to be broken across pages
% Create a tcolorbox for examples
% /begin{example}[extra name]{NAME}
% Create a tcolorbox for examples
% Argument #1 is optional, given by [], that is the textbook's problem number
% Argument #2 is mandatory, given by {}, that is the title for the example
% Avoid putting special characters, (), [], {}, ",", etc. in the title.
\newtcolorbox[auto counter,
number within=section,
number format=\arabic,
crefname={example}{examples}, % Define reference format for cref (No Capitals)
Crefname={Example}{Examples}, % Reference format for cleveref (With Capitals)
]{example}[2][]{ % The [2][] Means the first argument is optional
  width=\textwidth,
  title={Example \thetcbcounter: #2. #1}, % Parentheses and commas are not well supported
  fonttitle=\bfseries,
  label={ex:#2},
  nameref=#2,
  colbacktitle=white!100!black,
  coltitle=black!100!white,
  colback=white!100!black,
  upperbox=visible,
  lowerbox=visible,
  sharp corners=all,
  breakable
}

% Create a tcolorbox for general use
\newtcolorbox[% auto counter,
% number within=section,
% number format=\arabic,
% crefname={example}{examples}, % Define reference format for cref (No Capitals)
% Crefname={Example}{Examples}, % Reference format for cleveref (With Capitals)
]{blackbox}{
  width=\textwidth,
  % title={},
  fonttitle=\bfseries,
  % label={},
  % nameref=,
  colbacktitle=white!100!black,
  coltitle=black!100!white,
  colback=white!100!black,
  upperbox=visible,
  lowerbox=visible,
  sharp corners=all
}

% Redefine the 'end of proof' symbol to be a black square, not blank
\renewcommand{\qedsymbol}{$\blacksquare$} % Change proofs to have black square at end

% Common Mathematical Stuff
\newcommand{\Abs}[1]{\ensuremath{\lvert #1 \rvert}}
\newcommand{\DNE}{\ensuremath{\mathrm{DNE}}} % Used when limit of function Does Not Exist

% Complex Numbers functions
\renewcommand{\Re}{\operatorname{Re}} % Redefine to use the command, but not the fraktur version
\renewcommand{\Im}{\operatorname{Im}} % Redefine to use the command, but not the fraktur version
\newcommand{\Real}[1]{\ensuremath{\Re \lbrace #1 \rbrace}}
\newcommand{\Imag}[1]{\ensuremath{\Im \lbrace #1 \rbrace}}
\newcommand{\Conjugate}[1]{\ensuremath{\overline{#1}}}
\newcommand{\Modulus}[1]{\ensuremath{\lvert #1 \rvert}}
\DeclareMathOperator{\PrincipalArg}{\ensuremath{Arg}}

% Math Operators that are useful to abstract the written math away to one spot
% Number Sets
\DeclareMathOperator{\RealNumbers}{\ensuremath{\mathbb{R}}}
\DeclareMathOperator{\AllIntegers}{\ensuremath{\mathbb{Z}}}
\DeclareMathOperator{\PositiveInts}{\ensuremath{\mathbb{Z}^{+}}}
\DeclareMathOperator{\NegativeInts}{\ensuremath{\mathbb{Z}^{-}}}
\DeclareMathOperator{\NaturalNumbers}{\ensuremath{\mathbb{N}}}
\DeclareMathOperator{\ComplexNumbers}{\ensuremath{\mathbb{C}}}
\DeclareMathOperator{\RationalNumbers}{\ensuremath{\mathbb{Q}}}

% Calculus operators
\DeclareMathOperator*{\argmax}{argmax} % Thin Space and subscripts are UNDER in display

% Signal and System Functions
\DeclareMathOperator{\UnitStep}{\mathcal{U}}
\DeclareMathOperator{\sinc}{sinc} % sinc(x) = (sin(pi x)/(pi x))

% Transformations
\DeclareMathOperator{\Lapl}{\mathcal{L}} % Declare a Laplace symbol to be used

% Logical Operators
\DeclareMathOperator{\XOR}{\oplus}

% x86 CPU Registers
\newcommand{\rbpRegister}{\texttt{\%rbp}}
\newcommand{\rspRegister}{\texttt{\%rsp}}
\newcommand{\ripRegister}{\texttt{\%rip}}
\newcommand{\raxRegister}{\texttt{\%rax}}
\newcommand{\rbxRegister}{\texttt{\%rbx}}

%%% Local Variables:
%%% mode: latex
%%% TeX-master: shared
%%% End:


% These packages are more specific to certain documents, but will be availabe in the template
% \usepackage{esint} % Provides us with more types of integral symbols to use
% \usepackage[outputdir=./TeX_Output]{minted} % Allow us to nicely typeset 300+ programming languages

% \graphicspath{{./Drawings/Course}} % Uncomment this to use pictures in this document
% \addbibresource{./Bibliographies/CourseNum-Name.bib}

% Math Operators that are useful to abstract the written math away to one spot
% These are supposed to be document-specific mathematical operators that will make life easier
% Many fundamental operators are defined in Reference_Sheet_Preamble.tex

\DeclareMathOperator{\TimeDelay}{\mathrm{TD}_{k}}
\DeclareMathOperator{\FoldTime}{\mathrm{FD}}
\DeclareMathOperator{\SignalOperator}{\mathcal{T}}
\DeclareMathOperator{\ZTransformRelation}{\overset{z}{\longleftrightarrow}}
\DeclareMathOperator{\ZTransform}{\mathcal{Z}}
\DeclareMathOperator{\OneSideZTransformRelation}{\overset{z^{+}}{\longleftrightarrow}}
\DeclareMathOperator{\OneSideZTransform}{\mathcal{Z^{+}}}
\DeclareMathOperator{\UnitImpulse}{\delta}
\DeclareMathOperator{\ROC}{\mathrm{ROC}}
\DeclareMathOperator{\FourierTransform}{F}
\DeclareMathOperator{\FourierTransformRelation}{\overset{F}{\longleftrightarrow}}
\DeclareMathOperator{\CircularConvolution}{\circledast}
\DeclareMathOperator{\DFTRelation}{\overset{\text{DFT}}{\underset{N}{\longleftrightarrow}}}

\begin{titlepage}
  \title{EITF75: Systems and Signals - Equation Sheet}
  \author{Karl Hallsby}
  \date{Last Edited: \today} % We want to inform people when this document was last edited
\end{titlepage}

\begin{document}
\section{Convolution}
\begin{equation}\label{eq:Cross_Correlation-Convolution}
  r_{y,x}(k) = y(n) * x(-n)
\end{equation}

\section{$\ZTransform$-Transform}
\begin{table}[h!]
  \centering
  \renewcommand{\arraystretch}{1.4}
  \begin{tabular}{ccc}
    \toprule
    Signal, $x(n)$ & $z$-Transform, $X(z)$ & $\ROC$ \\
    \midrule
    $\UnitImpulse(n)$ & 1 & All $z$ \\
    $\UnitStep(n)$ & $\frac{1}{1-z^{-1}}$ & $\lvert z \rvert > 1$ \\
    $a^{n}\UnitStep(n)$ & $\frac{1}{1-az^{-1}}$ & $\lvert z \rvert > \lvert a \rvert$ \\
    $na^{n}\UnitStep(n)$ & $\frac{az^{-1}}{{\left( 1-az^{-1} \right)}^{2}}$ & $\lvert z \rvert > \lvert a \rvert$ \\
    $-a^{n}\UnitStep(-n-1)$ & $\frac{1}{1-az^{-1}}$ & $\lvert z \rvert < \lvert a \rvert$ \\
    $-na^{n}\UnitStep(-n-1)$ & $\frac{az^{-z}}{{\left( 1-az^{-1} \right)}^{2}}$ & $\lvert z \rvert < \lvert a \rvert$ \\
    $\left( \cos \omega_{0}n \right) \UnitStep(n)$ & $\frac{1-z^{-1}\cos \omega_{0}}{1-2z^{-1}\cos \omega_{0} + z^{-2}}$ & $\lvert z \rvert > 1$ \\
    $\left( \sin \omega_{0}n \right) \UnitStep(n)$ & $\frac{z^{-1}\sin \omega_{0}}{1-2z^{-1}\cos \omega_{0} + z^{-2}}$ & $\lvert z \rvert > 1$ \\
    $\left( a^{n} \cos \omega_{0}n \right) \UnitStep(n)$ & $\frac{1-az^{-1}\cos \omega_{0}}{1-2az^{-1} \cos \omega_{0} + a^{2}z^{-2}}$ & $\lvert z \rvert > \lvert a \rvert$ \\
        $\left( a^{n} \sin \omega_{0}n \right) \UnitStep(n)$ & $\frac{az^{-1}\sin \omega_{0}}{1-2az^{-1} \cos \omega_{0} + a^{2}z^{-2}}$ & $\lvert z \rvert > \lvert a \rvert$ \\
    \bottomrule
  \end{tabular}
  \caption{Common $\ZTransform$-Transforms}
  \label{tab:Common Z-Transforms}
\end{table}

 \subsection{Properties of the \texorpdfstring{$\ZTransform$-Transform}{Z-Transform}}\label{subsec:Z-Transform Properties}
\begin{table}[h!]
  \centering
  \begin{tabular}{p{4cm}ccp{5cm}}
    \toprule
    Property & Time Domain & $z$-Domain & $\ROC$ \\
    \midrule
    \multirow{3}{*}{Notation} & $x(n)$ & $X(z)$ & $\ROC: r_{2} < \lvert z \rvert < r_{1}$ \\
             & $x_{1}(n)$ & $X_{1}(z)$ & $\ROC_{1}$ \\
             & $x_{2}(n)$ & $X_{2}(z)$ & $\ROC_{2}$ \\
    Linearity & $a_{1}x_{1}(n) + a_{2}x_{2}(n)$ & $a_{1}X_{1}(z) + a_{2}X_{2}(z)$ & At least the intersection of $\ROC_{1}$ and $\ROC_{2}$ \\
    Time Shifting & $x(n-k)$ & $z^{-k}X(z)$ & That of $X(z)$, except $z=0$ if $k>0$ and $z=\infty$ if $k<0$ \\
    Scaling & $a^{n}x(n)$ & $X(a^{-1}z)$ & $\lvert a \rvert r_{2} < \lvert z \rvert < \lvert a \rvert r_{1}$ \\
    Time Reversal & $x(-n)$ & $X(z^{-1})$ & $\frac{1}{r_{1}} < \lvert z \rvert < \frac{1}{r_{2}}$ \\
    Conjugation & $x^{*}(n)$ & $X^{*}(z^{*})$ & $\ROC$ \\
    Real Part & $\Re \lbrace x(n) \rbrace$ & $\frac{1}{2} \left[ X(z) + X^{*}(z^{*}) \right]$ & Includes $\ROC$ \\
    Imaginary Part & $\Im \lbrace x(n) \rbrace$ & $\frac{1}{2} j \left[ X(z) - X^{*}(z^{*}) \right]$ & Includes $\ROC$ \\
    Differentiation & $nx(n)$ & $-z \frac{dX(z)}{dz}$ & $r_{2} < \lvert z \rvert r_{1}$ \\
    Convolution & $x_{1} * x_{2}$ & $X_{1}(z)X_{2}(z)$ & At least, the intersection of $\ROC_{1}$ and $\ROC_{2}$ \\
    2 Sequence Correlation & $r_{x_{1}x_{2}}(l) = x_{1}(l) * x_{2}(-l)$ & $R_{x_{1}x_{2}}(z) = X_{1}(z)x_{2}(z^{-1})$ & At least, the intersection of $\ROC$ of $X_{1}(z)$ and $X_{2}(z^{-1})$ \\
    Initial Value Theorem & If $x(n)$ causal & $x(0) = \lim\limits_{z \rightarrow \infty} X(z)$ & \\
    2 Sequence Multiplication & $x_{1}(n)x_{2}(n)$ & $\frac{1}{2 \pi j} \oint_{C} X_{1}(v)X_{2}(\frac{z}{v}) v^{-1} dv$ & At least, $r_{1l}r_{2l} < \lvert a \rvert < r_{1u}r_{2u}$ \\
    Parsevals Relation & $\sum\limits_{n=-\infty}^{\infty} x_{1}(n)x_{2}^{*}(n)$ &= $\frac{1}{2 \pi j} \oint_{C} X_{1}(v)X_{2}^{*}(\frac{1}{v^{*}})v^{-1} dv$ & \\
    \bottomrule
  \end{tabular}
  \caption{Z-Transform Properties}
  \label{tab:Z-Transform Properties}
\end{table}

\subsection{One-Sided \texorpdfstring{$\ZTransform$-Transform}{Z-Transform}}
The \emph{one-sided $z$-transform} is the same as the \nameref{def:Z-Transform}, but is only defined at $n$ values greater than or equal to 0.
\begin{equation}\label{eq:One-Sided Z-Transform}
  X(z) \equiv \sum_{n=0}^{\infty} x(n)z^{-n}
\end{equation}

\subsubsection{Time Delay}\label{subsubsec:One-Sided_Z-Transform_Properties-Time_Delay}
If
\begin{equation*}
  x(n) \OneSideZTransformRelation X^{+}(z)
\end{equation*}
then
\begin{equation}\label{eq:One-Sided_Z-Transform_Properties-Time_Delay}
  x(n-k) \OneSideZTransformRelation z^{-k} \left[ X^{+}(z) + \sum\limits_{n=1}^{k}x(-n)z^{n} \right], \:\:\: k>0
\end{equation}


\section{DTFT}
\begin{equation}\label{eq:1}
  \begin{aligned}
    z &= e^{j 2\pi f} \\
    z &= e^{j \omega} \\
  \end{aligned}
\end{equation}

\begin{equation}\label{eq:2}
  \begin{aligned}
    X(f) &= \sum\limits_{n=-\infty}^{\infty} x(n) e^{-j 2\pi f n} \\
    X(\omega) &= \sum\limits_{n=-\infty}^{\infty} x(n) e^{-j \omega n} \\
  \end{aligned}
\end{equation}

\section{DFT}
The $N$-point DFT is shown as:
\begin{equation}\label{eq:DFT}
  X_{DFT}(k) = \sum\limits_{n=0}^{N-1} x(n) e^{-j 2\pi \frac{k}{N} n} \:\: \text{for } k=0, 1, 2, \ldots, N-1
\end{equation}

If $N$ is specified, then replace all occurrences of $N$ in \Cref{eq:DFT} with that value.

\begin{remark}
  If the length, $N$ of the DFT is not specified, it is assumed that $N = \text{length of the signal}$.
  If the length of the DFT $N$ is greater than the length of the signal, you are sampling the DTFT of the signal.
\end{remark}
\begin{equation}\label{eq:DFT_of_Cosine}
  \begin{aligned}
    x(n) &= A \cos \left( 2\pi \frac{k_{0}}{N} n \right),\; 0 < k_{0} \leq N-1 \\
    &= \frac{A}{2} \left( e^{j \frac{2\pi k_{0}}{N} n} + e^{-j \frac{2\pi k_{0}}{N} n} \right) \\
    X(k) &= \frac{AN}{2} \biggl[ (\delta(k-k_{0}) \bmod N) + (\delta(k+k_{0}) \bmod N) \biggr]
  \end{aligned}
\end{equation}


\begin{equation}\label{eq:IDFT}
  x_{IDFT}(n) = \frac{1}{N} \sum\limits_{k=0}^{N-1} X(k) e^{j 2\pi \frac{k}{N} n} \:\: \text{for } n = 0, 1, \ldots, N-1
\end{equation}

\begin{table}[h!]
  \centering
  \begin{tabular}{ccc}
    \toprule
    Property & Time Domain $x(n)$ & DFT Domain $X(k)$ \\
    \midrule
    Notation & $x(n)$, $y(n)$ & $X(k)$, $Y(k)$ \\
    Periodicity & $x(n) = x(n+N)$ & $X(k) = X(k+N)$ \\
    Linearity & $a_{1}x_{1}(n) + a_{2}x_{2}(n)$ & $a_{1}X_{1}(k) + a_{2}X_{2}(k)$ \\
    Time Reversal & $x(N-n)$ & $X(N-k)$ \\
    Circular Time Shifting & $x(n - n_{0} \bmod N)$ & $X(k) e^{-j 2\pi \frac{k}{N} n_{0}} $ \\
    Circular Frequency Shift & $x(n)e^{j2\pi l n/N}$ & $X(k-l \bmod N)$ \\
    Complex Conjugate & $X^{*}(n)$ & $X^{*}(N-k)$ \\
    Circular Convolution & $x(n) \CircularConvolution y(n)$ & $X(k)Y(k)$ \\
    Circular Correlation & $x(n) \CircularConvolution y^{*}(-n)$ & $X(k)Y^{*}(k)$ \\
    2 Sequence Multiplication & $x_{1}(n)x_{2}(n)$ & $\frac{1}{N} X_{1}(k) \CircularConvolution X_{2}(k)$ \\
    Parsevals Theorem & $\sum\limits_{n=0}^{N-1} x(n) y^{*}(n)$ & $\frac{1}{N} \sum\limits_{k=0}^{N-1} X(k)Y^{*}(k)$ \\
    \bottomrule
  \end{tabular}
  \caption{Properties of the DFT}
  \label{tab:DFT_Properties}
\end{table}

\begin{equation}\label{eq:Circular_Convolution-Simplified}
  x_{1}(n) \CircularConvolution x_{2}(n) = \sum\limits_{k=0}^{N-1} x_{1}(k) x_{2}(n-k \bmod N)
\end{equation}
It is important to remember that the modulus (mod) operator yields 0 when the input is a multiple of the divisor.

\begin{definition}[Decimation]\label{def:Decimation}
  \emph{Decimation} takes an input signal and compresses it.
  Decimation uses the symbol $D \in \PositiveInts$.
  \begin{equation*}
    y(m) = x(mD)
  \end{equation*}

  If decimation occurs later in the system, then if just the input and output are compared, $y(m)$ appears it was sampled at
  \begin{equation}\label{eq:Input_Output_Decimation}
    f = \frac{F_{S}}{D}
  \end{equation}
  
  Thus, when we perform sampling on the input signal, then there is folding at
  \begin{equation}\label{eq:Decimation_Sampling_Frequency_Change}
    f = \frac{F_{S}}{2D}
  \end{equation}
\end{definition}
\end{document}
%%% Local Variables:
%%% mode: latex
%%% TeX-master: t
%%% End:
