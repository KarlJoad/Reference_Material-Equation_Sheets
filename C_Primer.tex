\documentclass[10pt,letterpaper,final,twoside,notitlepage]{article}
\usepackage[margin=.5in]{geometry} % 1/2 inch margins on all pages
\usepackage[utf8]{inputenc} % Define the input encoding
\usepackage[USenglish]{babel} % Define language used
\usepackage{amsmath,amsfonts,amssymb}
\usepackage{amsthm} % Gives us plain, definition, and remark to use in \theoremstyle{style}
\usepackage{mathtools} % Allow for text and math in align* environment.
\usepackage{thmtools}
\usepackage{thm-restate}
\usepackage{graphicx}

\usepackage[
backend=biber,
style=alphabetic,
citestyle=authoryear]{biblatex} % Must include citation somewhere in document to print bibliography
\usepackage{hyperref} % Generate hyperlinks to referenced items
\usepackage[nottoc]{tocbibind} % Prints the Reference/Bibliography in TOC as well
\usepackage[noabbrev,nameinlink]{cleveref} % Fancy cross-references in the document everywhere
\usepackage{nameref} % Can make references by name to places
\usepackage{caption} % Allows for greater control over captions in figure, algorithm, table, etc. environments
\usepackage{subcaption} % Allows for multiple figures in one Figure environment
\usepackage[binary-units=true]{siunitx} % Gives us ways to typeset units for stuff
\usepackage{csquotes} % Context-sensitive quotation facilities
\usepackage{enumitem} % Provides [noitemsep, nolistsep] for more compact lists
\usepackage{chngcntr} % Allows us to tamper with the counter a little more
\usepackage{empheq} % Allow boxing of equations in special math environments
\usepackage[x11names]{xcolor} % Gives access to coloring text in environments or just text, MUST be before tikz
\usepackage{tcolorbox} % Allows us to create boxes of various types for examples
\usepackage{tikz} % Allows us to create TikZ and PGF Pictures
\usepackage{ctable} % Greater control over tables and how they look
\usepackage{diagbox} % Allow us to have shared diagonal cells in tables
\usepackage{multirow} % Allow us to have a single cell in a table span multiple rows
\usepackage{titling} % Put document information throughout the document programmatically
\usepackage[linesnumbered,ruled,vlined]{algorithm2e} % Allows us to write algorithms in a nice style.

\counterwithin{figure}{section}
\counterwithin{table}{section}
\counterwithin{equation}{section}
\counterwithin{algocf}{section}
\crefname{algocf}{algorithm}{algorithms}
\Crefname{algocf}{Algorithm}{Algorithms}
\setcounter{secnumdepth}{4}
\setcounter{tocdepth}{4} % Include \paragraph in toc
\crefname{paragraph}{paragraph}{paragraphs}
\Crefname{paragraph}{Paragraph}{Paragraphs}

% Create a theorem environment
\theoremstyle{plain}
\newtheorem{theorem}{Theorem}[section]
% Create a numbered theorem-like environment for lemmas
\newtheorem{lemma}{Lemma}[theorem]

% Create a definition environment
\theoremstyle{definition}
\newtheorem{definition}{Defn}
\newtheorem{corollary}{Corollary}[section]
% \begin{definition}[Term] \label{def:}
%   Make sure the term is emphasized with \emph{term}.
%   This ensures that if \emph is changed, it shows up everywhere
% \end{definition}

% Create a numbered remark environment numbered based on definition
% NOTE: This version of remark MUST go inside a definition environment
\theoremstyle{remark}
\newtheorem{remark}{Remark}[definition]
%\counterwithin{definition}{subsection} % Uncomment to have definitions use section.subsection numbering

% Create an unnumbered remark environment for general use
% NOTE: This version of remark has NO restrictions on placement
\newtheorem*{remark*}{Remark}

% Create a special list that handles properties. It can be broken and restarted
\newlist{propertylist}{enumerate}{1} % {Name}{Template}{Max Depth}
% [newlistname, LevelsToApplyTo]{formatting options}
\setlist[propertylist, 1]{label=\textbf{(\roman*)}, ref=\textbf{(\roman*)}, noitemsep, nolistsep}
\crefname{propertylisti}{property}{properties}
\Crefname{propertylisti}{Property}{Properties}

% Create a special list that handles enumerate starting with lower letters. Breakable/Restartable.
\newlist{boldalphlist}{enumerate}{1} % {Name}{Template}{Max Depth}
% [newlistname, LevelsToApplyTo]{formatting options}
\setlist[boldalphlist, 1]{label=\textbf{(\alph*)}, ref=\alph*, noitemsep, nolistsep} % Set options

\newlist{nocrefenumerate}{enumerate}{1} % {Name}{Template}{Max Depth}
% [newlistname, LevelsToApplyTo]{formatting options}
\setlist[nocrefenumerate, 1]{label=(\arabic*), ref=(\arabic*), noitemsep, nolistsep}

% Create a list that allows for deeper nesting of numbers. By default enumerate only allows depth=4.
\newlist{nestednums}{enumerate}{6}
% [newlistname, LevelsToApplyTo]{formatting options}
\setlist[nestednums]{noitemsep, label*=\arabic*.}

\tcbuselibrary{breakable} % Allow tcolorboxes to be broken across pages
% Create a tcolorbox for examples
% /begin{example}[extra name]{NAME}
% Create a tcolorbox for examples
% Argument #1 is optional, given by [], that is the textbook's problem number
% Argument #2 is mandatory, given by {}, that is the title for the example
% Avoid putting special characters, (), [], {}, ",", etc. in the title.
\newtcolorbox[auto counter,
number within=section,
number format=\arabic,
crefname={example}{examples}, % Define reference format for cref (No Capitals)
Crefname={Example}{Examples}, % Reference format for cleveref (With Capitals)
]{example}[2][]{ % The [2][] Means the first argument is optional
  width=\textwidth,
  title={Example \thetcbcounter: #2. #1}, % Parentheses and commas are not well supported
  fonttitle=\bfseries,
  label={ex:#2},
  nameref=#2,
  colbacktitle=white!100!black,
  coltitle=black!100!white,
  colback=white!100!black,
  upperbox=visible,
  lowerbox=visible,
  sharp corners=all,
  breakable
}

% Create a tcolorbox for general use
\newtcolorbox[% auto counter,
% number within=section,
% number format=\arabic,
% crefname={example}{examples}, % Define reference format for cref (No Capitals)
% Crefname={Example}{Examples}, % Reference format for cleveref (With Capitals)
]{blackbox}{
  width=\textwidth,
  % title={},
  fonttitle=\bfseries,
  % label={},
  % nameref=,
  colbacktitle=white!100!black,
  coltitle=black!100!white,
  colback=white!100!black,
  upperbox=visible,
  lowerbox=visible,
  sharp corners=all
}

% Redefine the 'end of proof' symbol to be a black square, not blank
\renewcommand{\qedsymbol}{$\blacksquare$} % Change proofs to have black square at end

% Common Mathematical Stuff
\newcommand{\Abs}[1]{\ensuremath{\lvert #1 \rvert}}
\newcommand{\DNE}{\ensuremath{\mathrm{DNE}}} % Used when limit of function Does Not Exist

% Complex Numbers functions
\renewcommand{\Re}{\operatorname{Re}} % Redefine to use the command, but not the fraktur version
\renewcommand{\Im}{\operatorname{Im}} % Redefine to use the command, but not the fraktur version
\newcommand{\Real}[1]{\ensuremath{\Re \lbrace #1 \rbrace}}
\newcommand{\Imag}[1]{\ensuremath{\Im \lbrace #1 \rbrace}}
\newcommand{\Conjugate}[1]{\ensuremath{\overline{#1}}}
\newcommand{\Modulus}[1]{\ensuremath{\lvert #1 \rvert}}
\DeclareMathOperator{\PrincipalArg}{\ensuremath{Arg}}

% Math Operators that are useful to abstract the written math away to one spot
% Number Sets
\DeclareMathOperator{\RealNumbers}{\ensuremath{\mathbb{R}}}
\DeclareMathOperator{\AllIntegers}{\ensuremath{\mathbb{Z}}}
\DeclareMathOperator{\PositiveInts}{\ensuremath{\mathbb{Z}^{+}}}
\DeclareMathOperator{\NegativeInts}{\ensuremath{\mathbb{Z}^{-}}}
\DeclareMathOperator{\NaturalNumbers}{\ensuremath{\mathbb{N}}}
\DeclareMathOperator{\ComplexNumbers}{\ensuremath{\mathbb{C}}}
\DeclareMathOperator{\RationalNumbers}{\ensuremath{\mathbb{Q}}}

% Calculus operators
\DeclareMathOperator*{\argmax}{argmax} % Thin Space and subscripts are UNDER in display

% Signal and System Functions
\DeclareMathOperator{\UnitStep}{\mathcal{U}}
\DeclareMathOperator{\sinc}{sinc} % sinc(x) = (sin(pi x)/(pi x))

% Transformations
\DeclareMathOperator{\Lapl}{\mathcal{L}} % Declare a Laplace symbol to be used

% Logical Operators
\DeclareMathOperator{\XOR}{\oplus}

% x86 CPU Registers
\newcommand{\rbpRegister}{\texttt{\%rbp}}
\newcommand{\rspRegister}{\texttt{\%rsp}}
\newcommand{\ripRegister}{\texttt{\%rip}}
\newcommand{\raxRegister}{\texttt{\%rax}}
\newcommand{\rbxRegister}{\texttt{\%rbx}}

%%% Local Variables:
%%% mode: latex
%%% TeX-master: shared
%%% End:


% These packages are more specific to certain documents, but will be availabe in the template
% \usepackage{esint} % Provides us with more types of integral symbols to use
\usepackage[outputdir=./TeX_Output]{minted} % Allow us to nicely typeset 300+ programming languages
\crefname{lstlisting}{listing}{listings}
\Crefname{lstlisting}{Listing}{Listings}
% This document must be compiled with the -shell-escape flag if the packages above are uncommented

% \graphicspath{{./Drawings/Course/}} % Uncomment this to use pictures in this document
\addbibresource{./Bibliographies/C_Primer.bib}
% % Define English Imperial units.
% Length
\DeclareSIUnit\inch{in}
\DeclareSIUnit\in{in}

\DeclareSIUnit\feet{ft}
\DeclareSIUnit\ft{ft}

\DeclareSIUnit\yard{yd}
\DeclareSIUnit\yd{yd}

\DeclareSIUnit\mile{mi}
\DeclareSIUnit\mi{mi}

% Volume
\DeclareSIUnit\fluidOunce{fl oz}
\DeclareSIUnit\floz{fl oz}

\DeclareSIUnit\pint{pt}
\DeclareSIUnit\pt{pt}

\DeclareSIUnit\quart{qt}
\DeclareSIUnit\qt{qt}

\DeclareSIUnit\gallon{gal}
\DeclareSIUnit\gal{gal}

% Mass
\DeclareSIUnit\grain{gr}
\DeclareSIUnit\gr{gr}

\DeclareSIUnit\ounce{oz}
\DeclareSIUnit\oz{oz}

\DeclareSIUnit\pound{lb}
\DeclareSIUnit\lb{lb}

\DeclareSIUnit\poundMass{lbm}
\DeclareSIUnit\lbm{lbm}

\DeclareSIUnit\ton{t}
\DeclareSIUnit\slug{slug}

% Temperature
\DeclareSIUnit\rankine{R}
\DeclareSIUnit[number-unit-product={}]\degreeF{\degree{}F}
\DeclareSIUnit[number-unit-product={}]\dF{\degree{}F}
\DeclareSIUnit[number-unit-product={}]\degF{\degree{}F}

\DeclareSIUnit[number-unit-product={}]\degreeR{\degree{}R}
\DeclareSIUnit[number-unit-product={}]\dR{\degree{}R}
\DeclareSIUnit[number-unit-product={}]\degR{\degree{}R}
% \DeclareSIUnit[number-unit-product={}]\Rankine{\degree{}R}
% \DeclareSIUnit[number-unit-product={}]\rankine{\degree{}R}
\DeclareSIUnit[number-unit-product={}]\degreeRankine{\degree{}R}

% Pressure
\DeclareSIUnit\bar{bar}
\DeclareSIUnit\atm{atm}
\DeclareSIUnit\psia{psia}
\DeclareSIUnit\psig{psig}
\DeclareSIUnit\psi{psi}

% Energy
\DeclareSIUnit\btu{btu}
\DeclareSIUnit\BTU{BTU}

%Force
\DeclareSIUnit\poundForce{lbf}
\DeclareSIUnit\lbf{lbf}

% volumetric flow
\DeclareSIUnit\cfm{cfm}
\DeclareSIUnit\CFM{CFM}

% Moles
\DeclareSIUnit\lbmol{lbmol}

%%% Local Variables:
%%% mode: latex
%%% TeX-master: shared
%%% End:


% Math Operators that are useful to abstract the written math away to one spot
% These are supposed to be document-specific mathematical operators that will make life easier
% Many fundamental operators are defined in Reference_Sheet_Preamble.tex

\usemintedstyle{emacs} % Best for use on white backgrounds. Use for inline-code
% This macro creates the 2 minted environments for kernel source code with common options
\def\mintedkernelargs{
  frame=lines, % Surround the source code with lines on top and bottom
  linenos, % We want to show line numbers for each line in the margin
  % Colors used here are xcolor X11 colors.
  % style=fruity, % Use the fruity color scheme. Best for use on black backgrounds. Use for code blocks.
  % bgcolor=black, % Set the background used
  style=emacs,
  bgcolor=white,
  autogobble=true, % Automatically remove shared indentation from files
  breaklines=true, % Break lines that are too long at convenient locations
}
\newcommand{\makenewmintedfiles}[1]{
  \newminted[csource]{c}{#1} % Use with \begin{csource} code \end{csource}

  \newmintedfile[csourcefile]{c}{#1} % Use with \csourcefile[additional-options]{Filename}
}
\expandafter\makenewmintedfiles\expandafter{\mintedkernelargs}
\newmintinline[cinline]{c}{% Use with \cinline{code}
  style=emacs,
  bgcolor=white,
}

\begin{titlepage}
  \title{C Primer}
  \author{Karl Hallsby}
  \date{Last Edited: \today} % We want to inform people when this document was last edited
\end{titlepage}

\begin{document}
\pagenumbering{gobble}
\maketitle
\pagenumbering{roman} % i, ii, iii on beginning pages, that don't have content
\tableofcontents
\clearpage
\pagenumbering{arabic} % 1,2,3 on content pages

\section{Introduction}\label{sec:Intro}
This document is intended for programmers that are newer to the C language and its facilities.
This is meant as a quick, supplementary, reference document for these programmers.
Many of the code examples in this document are either from IIT's CS 351 course, or Kernighan and Ritchie's~\cite{KernighanRitchieCProg} manual, \citetitle{KernighanRitchieCProg}.

\subsection{Properties}\label{subsec:Properties}
C is referred to as ``low-level'' today.
That means there are relatively few abstractions and few ``syntactic sugars'' for expressing computations.
This means when you write C, you can typically guess what the assembly would look like, which also lends itself to C's execution speed.
However, this also means that you have \textbf{VERY FEW} built-in language protections for your computations.
So, you open yourself up to a whole new class of problems when developing and writing your programs.
The language will not protect you, but C compilers will typically attempt to throw warnings or errors about the most egregious errors you may write.

Here are some properties of C that you should know about.

\subsubsection{Imperative}\label{subsubsec:Imperative}
C is an imperative language, meaning you express computation as a series of steps.
This is based off a finite-state based understanding of computation.

This stands in stark contrast to functional languages, which express computation as function applications to expressions.

\subsubsection{Procedural}\label{subsubsec:Procedural}
A procedural language allows you to organize repeated computations into logical blocks, typically referred to as functions or procedures.

\subsubsection{Lexically Scoped}\label{subsubsec:Lexically_Scoped}
C is lexically scoped because variables inside of procedures cannot exist outside of their logical blocks.

This stands in contrast to Dynamically Scoped languages, where variables are sometimes available outside of the written scope for that variable.
Bash is an example of a dynamically scoped language.


%%% Local Variables:
%%% mode: latex
%%% TeX-master: "../C_Primer"
%%% End:


\section{Syntax}\label{sec:Syntax}
The C language helped define a whole class of syntax that is used by many programming languages today.
C's syntactic decisions can be seen in Java, Rust, C++, C\#, and many others.

\subsection{Primitive Types}\label{subsec:Primitive_Types}
C has just 4 primitive types:
\begin{description}[noitemsep]
\item[\cinline{char}:] \textbf{One byte} integers (0--255), meant to represent ASCII characters.
\item[\cinline{int}:] Integers, which is defined to be \textit{at least} 16 bits.
  \begin{itemize}[noitemsep]
  \item Additional prefixes can be used to increase or decrease the range of the integer.
  \item These are shown in \Cref{subsubsec:Integer_Type_Prefixes}.
  \end{itemize}
\item[\cinline{float}:] Single precision IEEE floating point number.
\item[\cinline{double}:] Double precision IEEE floating point number.
\end{description}

\subsubsection{Integer Type Prefixes}\label{subsubsec:Integer_Type_Prefixes}
\begin{description}[noitemsep]
\item[\cinline{signed}:] The default for integers, meaning you \textbf{do not} have to specify this.
  Can represent both negative and positive integers.
  The range for this is $-2^{\text{\# bits} - 1}$--$2^{\text{\# bits} - 1}-1$
\item[\cinline{unsigned}:], Can only represent 0 and positive integers.
  Its range is $0$--$2^{\text{\# bits}}$
\item[\cinline{short}:] Tells the compiler that the integer must be at least 16 bits.
\item[\cinline{long}:] Tells the compiler that the integer must be at least 32 bits.
\item[\cinline{long long}:] Tells the compiler that the integer must be at least 64 bits.
\end{description}

\subsection{Basic Operators}\label{subsec:Basic_Operators}
Operators perform some operation on expressions.
This could be an arithmetic, a relational, logical, etc.\ operation.

\subsubsection{Arithmetic Operators}\label{subsubsec:Arithmetic_Operators}
These operators are for performing mathematical operations.
These are well-defined for integers and floating-point numbers.

\begin{description}[noitemsep]
\item[\cinline{+}] The addition operator.
  Works similarly for integers and floating-point numbers.

\item[\cinline{-}] The subtraction operator.
  Works similarly for integers and floating-point numbers.

\item[\cinline{*}] The multiplication operator.
  Works similarly for integers and floating-point numbers.

\item[\cinline{/}] The division operator.
  This returns the quotient of a division.
  This has a different result for integers and floating-point numbers.
  \begin{itemize}[noitemsep]
  \item Integers return the quotient of the division, but no fractional part.
  \item Floating-points return the entire remainder of the division.
  \end{itemize}

\item[\texttt{\%}] The modulo operator.
  Returns the remainder of a division operation when dividing integers.
  Note that this is only defined for integers.
\end{description}

\subsubsection{Logical}\label{subsubsec:Logical_Operators}
Logical operators work with boolean values.
\begin{description}[noitemsep]
\item[\cinline{!}] The logical NOT operator.
\item[\cinline{&&}] The logical AND operator.
  Returns \texttt{1} if and only if the left and right expressions are \textbf{BOTH} \texttt{1} at the same time.
  Otherwise, \texttt{0} is returned.
\item[\cinline{||}] The logical OR operator.
  Returns \texttt{0} if and only if both the left and right expressions are \texttt{0} at the same time.
  Otherwise, \texttt{1} is returned.
\end{description}

\subsubsection{Relational}\label{subsubsec:Relational_Operators}
These relational operators are used to define a value in relation to another.
Typically, these are used for boolean comparisons.

\begin{description}[noitemsep]
\item[\cinline{==}] The equality operator.
  Returns \texttt{1} if and only if the two expressions are equal, \texttt{0} otherwise.

\item[\cinline{!=}] The inequality operator.
  Returns \texttt{0} if and only if the two expressions are not equal, \texttt{1} otherwise.

\item[\cinline{>}] The greater-than operator.
  Returns \texttt{1} if and only if the left expression has a greater value than the right one.

\item[\cinline{>=}] The greater-than-or-equal-to operator.
  Returns \texttt{1} is and only if the left expression has a greater value or equal value than the right one.

\item[\cinline{<}] The less-than operator.
  Returns \texttt{1} if and only if the left expression has a lesser value than the right one.

\item[\cinline{<=}] The less-than-or-equal-to operator.
  Returns \texttt{1} is and only if the left expression has a lesser value or equal value than the right one.
\end{description}

\subsubsection{Assignment}\label{subsubsec:Assignment_Operators}
Unlike many other languages, in C, the assignment operator \texttt{=} is also an expression.
This means that when an assignment is performed, it also returns a value, in this case, it returns the value that was assigned to that particular name.

\begin{description}
\item[\cinline{=}] The assignment operator.
  Assigns a value to a given name.
  Returns the value of the assignment.

\item[\cinline{+=}] The add-and-assign operator.
  Takes the name on the left, adds the value on the right to the value on the left, and stores the result in the value on the left.

\item[\cinline{*=}] The multiply-and-assign operator.
  Takes the name on the left, multiplies the value on the left by to the value on the right, and stores the result in the value on the left.
\end{description}

Only \texttt{+=} and \texttt{*=} are shown, but there are similar ones defined for other \nameref{subsubsec:Arithmetic_Operators} too.

\subsubsection{Conditional Operator}\label{subsubsec:Conditional_Operator}
There is only one conditional operator, sometimes called the ternary operation or conditional expression.
It is defined like so: \cinline{bool ? true_exp : false_exp}.

\subsubsection{Bitwise Operators}\label{subsubsec:Bitwise_Operators}
Bitwise operators work on the component bits of a number.
This means they behave slightly differently than any other operator, but are not typically used in day-to-day calculations.
Usually, they are used to efficiently work with memory.

\begin{description}
\item[\cinline{&}] Bitwise AND
\item[\cinline{|}] Bitwise OR
\item[\cinline{^}] Bitwise exclusive OR (XOR)
\item[\cinline{~}] Bitwise negation, one's complement.
\item[\cinline{>>}] Bitwise SHIFT right
\item[\cinline{<<}] Bitwise SHIFT left
\end{description}

\subsection{Boolean Expressions}\label{subsec:Boolean_Expressions}
Because C is so low-level, the concept of \texttt{true} and \texttt{false} are defined as integers.

\begin{description}
\item[\cinline{0}] \texttt{false}.
\item[\cinline{1}] \texttt{true}.
  Technically, any non-zero value is considered \texttt{true}.
\end{description}

\begin{listing}[h!tbp]
\csourcefile{./C_Primer-Sections/Syntax/Code/logical-operators.c}
\caption{Logical Operators}
\label{lst:Logical_Operators}
\begin{minted}[frame=lines,linenos]{console}
$ ./a.out
1
1
0
0
1
\end{minted}
\end{listing}

%%% Local Variables:
%%% mode: latex
%%% TeX-master: "../../C_Primer"
%%% End:


\subsection{Control Flow}\label{subsec:Control_Flow}
Control flow is the idea of changing the direction a program executes based on some predicate.
Whether this change in direction is to a new direction is because of a change in state, or because something must be repeated is irrelevant.

\subsubsection{Branching}\label{subsubsec:Branching}
Branching has to deal with the changing of a program's control flow based on a predicate to perform some separate actions based on the state of the predicate.
There are 4 types for this:
\begin{description}[noitemsep]
\item[\nameref{par:if_Statement}] The most basic change in flow control.
  If the predicate provided is \texttt{true}-thy, performs an action, then returns to normal.

\item[\nameref{par:if_else_Statement}] If the predicate is \texttt{true}-thy, then perform some action, if the predicate is \texttt{false}, then perform some other action.

\item[\nameref{par:if_elseif_else_Statement}] If the predicate in the first \texttt{if} is \texttt{true}-thy, then that branch is taken.
  If the first predicate is \texttt{false}, then the \texttt{else if}'s predicate is checked for truth.
  The the \texttt{else if}'s predicate is \texttt{false}, then execution continues through any other \texttt{else if}s that may be present.
  If none of the \texttt{if} or \texttt{else if}s' predicates were \texttt{true}-thy, then the \texttt{else} is taken.
\item[\nameref{par:switch_case_Statement}] The \texttt{switch}-\texttt{case} statement allows you to choose from many different paths based on the \textbf{VALUE} of some expression.
\end{description}


%%% Local Variables:
%%% mode: latex
%%% TeX-master: "../../C_Primer"
%%% End:


\subsection{Variables}\label{subsec:Variables}
Because C is a statically typed language, the type of every expression \textbf{MUST} be known before or during compilation.
In addition, C compilers do not have any type inferencing, so you \textbf{MUST} explicitly tell the compiler the type of your variables.
This means that you \textbf{MUST} declare before use.
It is important to note that the declaration implicitly allocates storage for the data that will be stored.

One thing that will come up throughout this section is the concept of aliveness and scope.
It is import to note that variables do \textbf{not} have to be in-scope to be alive, and vice versa.

\subsubsection{Visibility}\label{subsubsec:Variable_Visibility}
Visibility or \emph{scope} is where a symbol can be seen from.
If the symbol cannot be seen, then it cannot be used in any way

In addition, we need to ask \textit{how} we can refer to the symbol.
This includes what kid of identifiers/modifiers/namespacing is needed to identify the symbol in question.

\paragraph{Global Variables}\label{par:Global_Variables}
They \textbf{MUST} be declared outside any function.
These are not deallocated \textbf{AT ANY TIME} during a program's execution, as they are always in-scope, however the variable may not be alive.
Additionally, these are \textbf{ALWAYS} available, until the program terminates.

Using global variables is typically bad practice as this can introduce weird and hard-to-debug errors into a program.
So, most variables that you will use will be local variables.
Local variables are limited to the scope they were created within.
Typically, the scope is a function, but can be an \texttt{if}, a \texttt{while}, etc.

\paragraph{\texorpdfstring{\cinline{extern}}{\texttt{extern}} Variables}\label{par:extern_Variables}
\cinline{extern} is used to denote a variable that is external to this program.
This means the variable in question is a global variable in another file

The opposite of the \cinline{extern} keyword is the \cinline{static} keyword, which limits the scope of a symbol to the file it is declared in.
In addition, when the variable is declared to be \cinline{static}, the value lasts throughout the program's execution, ensuring the variable is always available.


%%% Local Variables:
%%% mode: latex
%%% TeX-master: "../C_Primer"
%%% End:


\section{Pointers}\label{sec:Pointers}
This should technically go in \Cref{sec:Syntax}, but pointers deserve their own section.

\begin{definition}[Pointer]\label{def:Pointer}
A \emph{pointer} is a variable declared to store a memory address.
With this memory address, we can refer to data in-memory.
The size of the pointer is determined by the architecture of the CPU.\@
\end{definition}

A pointer is designated by its \textbf{DECLARED} type, \textbf{NOT} its contents.
This allows the data the pointer points to to be re-interpreted based on the declared type of the pointer.
This is shown in \Cref{lst:Pointers_Reinterpret_Data}.

\begin{listing}[h!tbp]
\csourcefile{./C_Primer-Sections/Pointers/Code/pointers-reinterpret-data.c}
\caption{Pointers Reinterpret Data}
\label{lst:Pointers_Reinterpret_Data}

\begin{minted}[frame=lines,linenos]{console}
$ ./a.out
ip: 63
cp: ?
fp: 0.000000
\end{minted}
\end{listing}

\subsection{Pointer Syntax}\label{subsec:Pointer_Syntax}.
The syntax that \nameref{def:Pointer}s use can sometimes be confusing for new programmers.
So, we will break down each portion of a pointer and its usage to more fully understand them.

\subsubsection{Declaration}\label{subsubsec:Pointer_Declaration}
\nameref{def:Pointer}s are declared using the same type name as the data type they will store.
In addition, you \textbf{MUST} add a \cinline{*} to the declaration.
The placement of this symbol does not matter, but for clarity, most programmers put it on the variable name.
However, it can be attached to the type as well.

\Cref{lst:Pointer_Declaration} shows how to do this.
\begin{listing}[h!tbp]
\csourcefile{./C_Primer-Sections/Pointers/Code/pointer-declaration.c}
\caption{Pointer Declaration}
\label{lst:Pointer_Declaration}
\end{listing}

\subsubsection{Getting an Address}\label{subsubsec:Getting_an_Address}
If you already have a name that points to the actual value, and you want the address of that value, you can use the \cinline{&} unary operator.
Its usage is shown in \Cref{lst:Address_Operator}.
\begin{listing}[h!tbp]
\csourcefile{./C_Primer-Sections/Pointers/Code/address-operator.c}
\caption{Address \cinline{&} Operator}
\label{lst:Address_Operator}
\begin{minted}[frame=lines,linenos]{console}
$ ./a.out
x: 15
*xp: 15
xp: 0x7ffc4868a204
\end{minted}
\end{listing}

\subsubsection{Dereferencing}\label{subsubsec:Dereferencing_Pointers}
In a major point of confusion for the C language, the \cinline{*} operator is \textbf{also} used to \textbf{dereference} the pointer!
An example of declaring, getting the address of a variable, and the dereferencing of pointers is shown in \Cref{lst:Dereferencing_Pointers}.
\begin{listing}[h!tbp]
\csourcefile{./C_Primer-Sections/Pointers/Code/dereferencing-pointers.c}
\caption{Dereferencing Pointers}
\label{lst:Dereferencing_Pointers}

\begin{minted}[frame=lines,linenos]{console}
$ ./a.out
i=10, j=20, *p=20, *q=20
\end{minted}
\end{listing}

\subsection{\emph{Why} have Pointers?}\label{subsec:Why_Have_Pointers}
\nameref{def:Pointer}s are partly a cross-over from assembly, as C is really just a thin wrapper around that.
However, it does also allow direct access to memory and allow \emph{us} to make many optimizations we could not make otherwise.
This means you must \textbf{manage memory yourself}.

\begin{itemize}
\item In C, everything is \emph{ALWAYS} \nameref{par:Pass_by_Value}.
\item This even happens on composite data structures, like \cinline{struct}s.
\item \nameref{def:Pointer}s enable us to perform \emph{actions at a distance}, essentially allow us to get \nameref{par:Pass_by_Reference} without explicitly allowing that.
\item This also allows functions very deep in the call stack to affect variables earlier in the stack.
\end{itemize}

One good reason is illustrated in \Cref{lst:int_Pointer_Swap}.

\begin{listing}[h!tbp]
\csourcefile{./C_Primer-Sections/Pointers/Code/int-pointer-swap.c}
\caption{Swap with Pointers}
\label{lst:int_Pointer_Swap}
\end{listing}

\subsection{Uninitialized Pointers}\label{subsec:Uninitialized_Pointers}
Even though a \nameref{def:Pointer} can only store memory addresses, it behaves exactly like a regular variable for most puposes.

\begin{itemize}
\item When a \nameref{def:Pointer} is declared, they are given their underlying storage, which \textbf{will contain garbage}!
\item If you dereference this pointer, you dereference garbage, which is undefined behavior.
  \begin{itemize}
  \item If you're lucky, this is will result in a \texttt{SEGFAULT}, terminating program execution.
  \item If you're unlucky, unknown results may happen, and you might never know about it.
  \item You are assured that your program will not exceed its own memory space though, which means your program cannot crash another.
  \end{itemize}
\end{itemize}

\Cref{lst:Uninitialized_Pointer} gives an example of what could happen if you don't initialize your \nameref{def:Pointer}s.
\begin{listing}[h!tbp]
\csourcefile{./C_Primer-Sections/Pointers/Code/uninitialized-pointers.c}
\caption{Uninitialized Pointers}
\label{lst:Uninitialized_Pointer}

\begin{minted}[frame=lines,linenos]{console}
$ ./a.out
[1]    29456 segmentation fault (core dumped)  ./a.out
\end{minted}
\end{listing}

\subsection{\texorpdfstring{\cinline{NULL}}{\texttt{NULL}} Pointers}\label{subsec:NULL_Pointers}
The \cinline{NULL} \nameref{def:Pointer} is \textbf{NEVER} returned by the \cinline{&} operator.
It is usually a smart idea to define pointers when you declare them, and if you don't immediately define them with a usable value, to define them with \cinline{NULL}.
This helps prevent many runtime issues that the compiler cannot check for.
\Cref{lst:NULL_Pointer-Checker} gives an example of using the \cinline{NULL} pointer this way.

\begin{itemize}[noitemsep]
\item The \cinline{NULL} pointers is safe to use as a predicate, in \texttt{if} statements, for example.
  This usage is shown in \Cref{lst:NULL_Pointer-Sentinel}.
\item Written as \texttt{0} or \texttt{NULL} in \emph{pointer context}
  \begin{itemize}
  \item Typically \cinline{#define NULL 0}
  \end{itemize}
\end{itemize}

\begin{listing}[h!tbp]
\csourcefile{./C_Primer-Sections/Pointers/Code/null-pointers-checker.c}
\caption{\texorpdfstring{\cinline{NULL}}{\texttt{NULL}} Pointer as Checker}
\label{lst:NULL_Pointer-Checker}

\begin{minted}[frame=lines,linenos]{console}
$ ./a.out
[1]    29456 segmentation fault (core dumped)  ./a.out
\end{minted}
\end{listing}


%%% Local Variables:
%%% mode: latex
%%% TeX-master: "../C_Primer"
%%% End:


\section{Arrays}\label{sec:Arrays}
An \nameref{def:Array} is the only way to store multiple items under a single name.

\begin{definition}[Array]\label{def:Array}
  A \emph{array} is a contiguous block of memory.
  Arrays are indexed items, meaning they use a number to identify each item.
\end{definition}


%%% Local Variables:
%%% mode: latex
%%% TeX-master: "../C_Primer"
%%% End:


\section{Strings}\label{sec:Strings}
In C, strings are actually character arrays.
These arrays are terminated with the null character, \cinline{\0}; it's numerical value is 0.
This is shown in \Cref{lst:String_Character_Array}.

\begin{listing}[h!tbp]
\csourcefile{./C_Primer-Sections/Strings/Code/char-array.c}
\caption{String/Character Array}
\label{lst:String_Character_Array}
\begin{minted}[frame=lines,linenos]{console}
$ ./a.out
hello world!
\end{minted}
\end{listing}

\texttt{printf} treats strings as a character array terminated by a null character.


%%% Local Variables:
%%% mode: latex
%%% TeX-master: "../C_Primer"
%%% End:


\section{Memory Management}\label{sec:Memory_Management}
Because C is a language that does not provide many abstractions, it also requires the programmer to remember and manage their memory usage.
So, \textbf{YOU} must be the one to manage the memory, there is \textbf{NO} built-in garbage collector for you to use.

Memory allocation is done on the heap of the program's execution space in memory.
When you allocate memory in your program, you are actually requesting the operating system to give you the memory you want.

\subsection{\texttt{malloc}}\label{subsec:malloc}
This is the simplest function of all possible memory allocation functions.
\texttt{malloc}:
\begin{itemize}
\item Takes one argument:
  \begin{enumerate}
  \item The number of bytes to allocate.
  \end{enumerate}
\item Returns a \textbf{POINTER} to the front of the allocated memory.
\end{itemize}

\texttt{malloc} {\large{\textbf{\emph{DOES NOT}}}} initialize memory, so it will be garbage.

\subsection{\texttt{calloc}}\label{subsec:calloc}
This is quite similar to malloc.
\texttt{calloc}:
\begin{itemize}
\item Takes 2 arguments:
  \begin{enumerate}
  \item The number of spaces to allocate, for example the number of elements in an array.
  \item The number of bytes to allocate, for the type being stored.
  \end{enumerate}
\item Returns a \textbf{POINTER} to the front of the allocated memory.
\end{itemize}

\texttt{calloc} {\large{\textbf{\emph{ZEROS}}}} memory, so this does have a slight performance penalty.

\subsection{\texttt{realloc}}\label{subsec:realloc}
\texttt{realloc} is used to \textbf{REALLOCATE} an existing memory location.
\begin{itemize}
\item Takes 2 arguments:
  \begin{enumerate}
  \item The pointer to the memory location previously allocated with either \texttt{malloc} or \texttt{calloc}.
  \item The amount of memory to reallocate, in bytes.
  \end{enumerate}
\item If the \texttt{NULL} pointer is passed to \texttt{realloc}, it will behave exactly like \texttt{malloc}.
\item Returns a \textbf{POINTER} to the front of the reallocated memory
\end{itemize}

\subsection{\texttt{free}}\label{subsec:free}
\texttt{free} is used to free memory that was previously allocated, removing from the programming space entirely.
\begin{itemize}
\item Takes 1 argument:
  \begin{enumerate}
  \item A pointer to the memory to be deallocated.
  \end{enumerate}
\item Returns \texttt{void}.
\end{itemize}

%%% Local Variables:
%%% mode: latex
%%% TeX-master: "../C_Primer"
%%% End:

\section{Composite Data Types}\label{sec:Composite_Data_Types}
This is \textit{similar} to objects in Object-Oriented Programming, or the \mintinline{haskell}{datatype} in Haskell.

\subsection{\texorpdfstring{\cinline{struct}}{\texttt{struct}}}\label{subsec:struct}
The \cinline{struct} keyword allows us to put multiple separate data types together and refer to them by their field name.
You access the fields by using the \cinline{.} operator.
An example of this is shown in \Cref{lst:struct_Usage}.

\begin{listing}[h!tbp]
\csourcefile{./C_Primer-Sections/Composite_Data_Types/Code/struct.c}
\caption{\texorpdfstring{\cinline{struct}}{\texttt{struct}} Usage}
\label{lst:struct_Usage}
\end{listing}

\subsection{\texorpdfstring{\cinline{union}}{\texttt{union}}}\label{subsec:union}
The \cinline{union} keyword allows us to define a single type that can be of one type from many.
However, these are \textbf{NOT} like Haskell's \mintinline{haskell}{datatype} or Rust's union, in that we do not have type protections about accessing this data.
\Cref{lst:union_Usage} gives an example of this.

\begin{listing}[h!tbp]
\csourcefile{./C_Primer-Sections/Composite_Data_Types/Code/union.c}
\caption{\texorpdfstring{\cinline{union}}{\texttt{union}} Usage}
\label{lst:union_Usage}
\end{listing}


%%% Local Variables:
%%% mode: latex
%%% TeX-master: "../C_Primer"
%%% End:


\section{Compilation}\label{sec:Compilation}
C, along with C++ can be quite painful to compile for larger projects.
You could manually compile every \texttt{.c} file with \texttt{gcc}.
Makefiles help us manage this.

\subsection{Stages}\label{subsec:Compilation_Stages}
\begin{enumerate}[noitemsep]
\item Preprocessing
  \begin{itemize}[noitemsep]
  \item Preprocessor directives starting with \cinline{#}
  \item Text substitution
  \item Macros
  \item Conditional compilation
  \item Performs complete textual substitution behind the scenes
  \end{itemize}
\item Compile
  \begin{itemize}[noitemsep]
  \item From source language to object code/binary
  \end{itemize}
\item Link
  \begin{itemize}[noitemsep]
  \item Put inter-related object codes together
  \item Resolve calls/references and definitions
  \item Put absolute/relative addresses into the binary for the =call= instruction
  \item Want to support /selective/ public APIs
  \item Don't always want to allow linking a call to a definition
  \end{itemize}
\end{enumerate}


%%% Local Variables:
%%% mode: latex
%%% TeX-master: "../C_Primer"
%%% End:



% To make this print, you must include a citation somewhere in the document
\clearpage
\printbibliography{}
\end{document}

%%% Local Variables:
%%% mode: latex
%%% TeX-master: t
%%% End:
