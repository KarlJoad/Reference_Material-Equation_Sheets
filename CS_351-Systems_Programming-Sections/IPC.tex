\section{Inter-Process Communication}\label{sec:Inter_Process_Communication}
The OS kernel is great at \emph{isolating} \nameref{def:Process}es from each other, but allowing processes to communicate with each other makes them more useful.
Allowing them to communicate enables the processes to exchange data and interact dynamically.

However, this separation is done to make programming easier.
If the OS were to \textbf{not} isolate each process
\begin{itemize}[noitemsep]
\item Any and every \nameref{def:Process} could read and/or write to any other process's memory space.
\item Thus, any process's memory integrity would not be guaranteed
\item In effect, this would make any process's control flow unpredictable.
\end{itemize}

Because the kernel enforces isolation, we need the assistance of the kernel to complete any \nameref{def:IPC}.
Two processes must explicitly request the kernel to allow them to communicate.

\begin{definition}[Inter-Process Communication]\label{def:IPC}
  \emph{Inter-Process Communication} (\emph{IPC}) is the act of two or more \nameref{def:Process}es communicating with one another.
  There are variety of mechanisms for allowing this, explored in \Cref{subsec:IPC_Mechanisms}.
\end{definition}

\subsection{Mechanisms}\label{subsec:IPC_Mechanisms}
There are a variety of mechanisms for \nameref{def:Process}es to interact and communicate with each other.
Predictably, each one of these has an intended use, has certain benefits, and has certain drawbacks.

\subsubsection{Signals}\label{subsubsec:IPC_Mechanism-Signals}
\nameref{def:Signal}s were discussed in more depth in \Cref{subsec:Signals}.
But, in the case of \nameref{def:IPC}, signals are a very limited form of communication, as the signal sends a very well-predefined message.

\subsubsection{(Regular) Files}\label{subsubsec:IPC_Mechanism-Regular_Files}
It is always possible to save the information to a regular file on the system and then have other \nameref{def:Process}es read from and write to this file to communicate.
This does have its place, but for many small reads and writes, the overhead of writing to the storage medium and using the file system will cause greater slow-downs.

\subsubsection{Shared Memory}\label{subsubsec:IPC_Mechanism-Shared_Memory}
\nameref{def:Process}es can share memory regions between each other.
This allows for very fast access, with no direct limitation on the way the information is written and read form the shared area.
However, this lack of uniformity (and potential problems with atomicity) can lead to major headaches.

\subsubsection{Pipes}\label{subsubsec:IPC_Mechanism-Pipes}
These are similar to the shell pipe \mintinline{bash}{|}.

\begin{definition}[Pipe]\label{def:Pipe}
  \emph{Pipe}s are a data structure and idea that allow us to implement \nameref{def:IPC}.
  A pipe behaves like a queue data structure, with one \nameref{def:Process} writing to the pipe, and another (not necessarily the same one) reading from it.
  A pipe uses explicit \texttt{send}, \texttt{receive}, \texttt{read}, and \texttt{write} functions to utilize the pipe, making it easier to figure out what process is doing what at what time.
  However, only 2 processes can use a pipe at a time, making it difficult to go from one process to many different processes.

  There are 2 kinds of pipes:
  \begin{enumerate}[noitemsep]
  \item \nameref{def:Named_Pipe}
  \item \nameref{def:Unnamed_Pipe}
  \end{enumerate}
\end{definition}

\subsubsection{File Locks and Semaphores}\label{subsubsec:IPC_Mechanism-File_Locks_Semaphores}
\paragraph{File Locks}\label{par:File_Locks}
\paragraph{Semaphores}\label{par:Semaphores}
\subsubsection{Sockets}\label{subsubsec:Sockets}

%%% Local Variables:
%%% mode: latex
%%% TeX-master: "../CS_351-Systems_Programming-Reference_Sheet"
%%% End:
