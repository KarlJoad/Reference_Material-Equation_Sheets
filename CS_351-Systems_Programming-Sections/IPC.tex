\section{Inter-Process Communication}\label{sec:Inter_Process_Communication}
The OS kernel is great at \emph{isolating} \nameref{def:Process}es from each other, but allowing processes to communicate with each other makes them more useful.
Allowing them to communicate enables the processes to exchange data and interact dynamically.

However, this separation is done to make programming easier.
If the OS were to \textbf{not} isolate each process
\begin{itemize}[noitemsep]
\item Any and every \nameref{def:Process} could read and/or write to any other process's memory space.
\item Thus, any process's memory integrity would not be guaranteed
\item In effect, this would make any process's control flow unpredictable.
\end{itemize}

Because the kernel enforces isolation, we need the assistance of the kernel to complete any \nameref{def:IPC}.
Two processes must explicitly request the kernel to allow them to communicate.

\begin{definition}[Inter-Process Communication]\label{def:IPC}
  \emph{Inter-Process Communication} (\emph{IPC}) is the act of two or more \nameref{def:Process}es communicating with one another.
  There are variety of mechanisms for allowing this, explored in \Cref{subsec:IPC_Mechanisms}.
\end{definition}

\subsection{Mechanisms}\label{subsec:IPC_Mechanisms}
There are a variety of mechanisms for \nameref{def:Process}es to interact and communicate with each other.
Predictably, each one of these has an intended use, has certain benefits, and has certain drawbacks.

\subsubsection{Signals}\label{subsubsec:IPC_Mechanism-Signals}
\nameref{def:Signal}s were discussed in more depth in \Cref{subsec:Signals}.
But, in the case of \nameref{def:IPC}, signals are a very limited form of communication, as the signal sends a very well-predefined message.

\subsubsection{(Regular) Files}\label{subsubsec:IPC_Mechanism-Regular_Files}
\subsubsection{Shared Memory}\label{subsubsec:IPC_Mechanism-Shared_Memory}
\subsubsection{Pipes}\label{subsubsec:IPC_Mechanism-Pipes}
\subsubsection{File Locks and Semaphores}\label{subsubsec:IPC_Mechanism-File_Locks_Semaphores}
\paragraph{File Locks}\label{par:File_Locks}
\paragraph{Semaphores}\label{par:Semaphores}
\subsubsection{Sockets}\label{subsubsec:Sockets}

%%% Local Variables:
%%% mode: latex
%%% TeX-master: "../CS_351-Systems_Programming-Reference_Sheet"
%%% End:
