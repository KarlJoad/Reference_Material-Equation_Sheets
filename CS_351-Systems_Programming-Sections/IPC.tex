\section{Inter-Process Communication}\label{sec:Inter_Process_Communication}
The OS kernel is great at \emph{isolating} \nameref{def:Process}es from each other, but allowing processes to communicate with each other makes them more useful.
Allowing them to communicate enables the processes to exchange data and interact dynamically.

However, this separation is done to make programming easier.
If the OS were to \textbf{not} isolate each process
\begin{itemize}[noitemsep]
\item Any and every \nameref{def:Process} could read and/or write to any other process's memory space.
\item Thus, any process's memory integrity would not be guaranteed
\item In effect, this would make any process's control flow unpredictable.
\end{itemize}

Because the kernel enforces isolation, we need the assistance of the kernel to complete any \nameref{def:IPC}.
Two processes must explicitly request the kernel to allow them to communicate.

\begin{definition}[Inter-Process Communication]\label{def:IPC}
  \emph{Inter-Process Communication} (\emph{IPC}) is the act of two or more \nameref{def:Process}es communicating with one another.
  There are variety of mechanisms for allowing this, explored in \Cref{subsec:IPC_Mechanisms}.
\end{definition}

\subsection{Mechanisms}\label{subsec:IPC_Mechanisms}
There are a variety of mechanisms for \nameref{def:Process}es to interact and communicate with each other.
Predictably, each one of these has an intended use, has certain benefits, and has certain drawbacks.

\subsubsection{Signals}\label{subsubsec:IPC_Mechanism-Signals}
\nameref{def:Signal}s were discussed in more depth in \Cref{subsec:Signals}.
But, in the case of \nameref{def:IPC}, signals are a very limited form of communication, as the signal sends a very well-predefined message.

\subsubsection{(Regular) Files}\label{subsubsec:IPC_Mechanism-Regular_Files}
It is always possible to save the information to a regular file on the system and then have other \nameref{def:Process}es read from and write to this file to communicate.
This does have its place, but for many small reads and writes, the overhead of writing to the storage medium and using the file system will cause greater slow-downs.

\subsubsection{Shared Memory}\label{subsubsec:IPC_Mechanism-Shared_Memory}
\nameref{def:Process}es can share memory regions between each other.
This allows for very fast access, with no direct limitation on the way the information is written and read form the shared area.
However, this lack of uniformity (and potential problems with atomicity) can lead to major headaches.

\subsubsection{Pipes}\label{subsubsec:IPC_Mechanism-Pipes}
These are similar to the shell pipe \mintinline{bash}{|}.

\begin{definition}[Pipe]\label{def:Pipe}
  \emph{Pipe}s are a data structure and idea that allow us to implement \nameref{def:IPC}.
  A pipe behaves like a queue data structure, with one \nameref{def:Process} writing to the pipe, and another (not necessarily the same one) reading from it.
  A pipe uses explicit \texttt{send}, \texttt{receive}, \texttt{read}, and \texttt{write} functions to utilize the pipe, making it easier to figure out what process is doing what at what time.
  However, only 2 processes can use a pipe at a time, making it difficult to go from one process to many different processes.

  There are 2 kinds of pipes:
  \begin{enumerate}[noitemsep]
  \item \nameref{def:Named_Pipe}
  \item \nameref{def:Unnamed_Pipe}
  \end{enumerate}
\end{definition}

Using \nameref{def:Pipe}s allows us to more easily implement correct \nameref{def:IPC} functionality.
In addition, there is no need to go to the file system, so there are no file system performance implications.
Technically \nameref{def:Pipe}s are implemented using \nameref{subsubsec:IPC_Mechanism-Shared_Memory}, so you get memory-access speeds.

\begin{definition}[Named Pipe]\label{def:Named_Pipe}
  A \emph{named pipe} has some key differences compared to a \nameref{def:Unnamed_Pipe}.
  Both of these types exist in the \textsc{unix} world.
  These differences are:
  \begin{itemize}[noitemsep]
  \item It has a specific name which can be given to it by the programmer.
    Named pipe is referred to through this name only by the reader and writer.
    All instances of a named pipe share the same pipe name.
  \item A named pipe can be used for communication between two unnamed process as well.
    \nameref{def:Process}es of different ancestry can share data through a named pipe.
  \item A named pipe exists in the file system.
    After input/output has been performed by the sharing \nameref{def:Process}es, the pipe still exists in the file system independently of the process, and can be used for communcation between some other processes.
  \item Named pipes can be used to provide communication between processes on the same computer or between processes on different computers across a network, as in case of a distributed system.
  \item A named pipe can have multiple process communicating through it, like multiple clients connected to one server.
  \end{itemize}
\end{definition}

\begin{definition}[Unnamed Pipe]\label{def:Unnamed_Pipe}
  An \emph{unnamed pipe}, sometimes called an \emph{anonymous pipe} has some key differences compared to a \nameref{def:Named_Pipe}.
  Both of these types exist in the \textsc{unix} world.
  The differences are:
  \begin{itemize}[noitemsep]
  \item Unnamed pipes are not given a name.
    It is accessible through two file descriptors that are created through the function \cinline{pipe(fd[2])}, where \texttt{fd[1]} signifies the \textbf{write} \nameref{def:File_Descriptor}, and \texttt{fd[0]} describes the \textbf{read} \nameref{def:File_Descriptor}.
  \item An unnamed pipe is only used for communication between a child and it's parent process.
  \item An unnamed pipe vanishes as soon as it is closed, or one of the \nameref{def:Process} (parent or child) completes execution.
  \item Unnamed pipes are always local; they cannot be used for communication over a network.
  \item An unnamed pipe is a one-way \nameref{def:Pipe} that typically transfers data between a parent process and a child process.
  \end{itemize}
\end{definition}

\subsubsection{File Locks and Semaphores}\label{subsubsec:IPC_Mechanism-File_Locks_Semaphores}
\paragraph{File Locks}\label{par:File_Locks}
File \nameref{def:Lock}s control concurrent access/modification of shared files.
\begin{definition}[Lock]\label{def:Lock}
  A \emph{lock} allows \textbf{only one} \nameref{def:Process} to enter the portion of code that is locked.
  While a thread holds this lock no other \nameref{def:Process} can execute on this code portion.

  \begin{remark}[Binary Semaphore]\label{rmk:Binary_Semaphore}
    Locks can be represented as \emph{binary \nameref{def:Semaphore}}s.
  \end{remark}
\end{definition}

\begin{definition}[Mutex]\label{def:Mutex}
  A \emph{mutex} (short for \emph{\textbf{mut}ual \textbf{ex}clusion}) is the same as a \nameref{def:Lock}, \textbf{but it can be system wide (shared by multiple processes)}.
  A mutex lock protects critical regions and prevents race conditions.
  That is, a process must acquire the lock before entering a critical section; it releases the lock when it exits the critical section.

  The \cinline{acquire()} function acquires the lock, preventing any other \nameref{def:Process} from using the thing the lock protects.
  Likewise, the \cinline{release()} function releases the lock, allowing another \nameref{def:Process} take acquire the lock and use the resource it protects.
  To perform its function correctly, the lock's \cinline{acquire()} and \cinline{release()} functions must be atomic.

  When a \nameref{def:Process} and/or \nameref{def:Process} attempts \texttt{acquire()} the lock, while it is already owned by someone else, it is put in a \texttt{WAITING} state.
\end{definition}

\paragraph{Semaphores}\label{par:Semaphores}
\nameref{def:Semaphore}s control shared memory's access and modification.

\begin{definition}[Semaphore]\label{def:Semaphore}
  A \emph{semaphore} regulates the number of things (\nameref{def:Process}es or \nameref{def:Process}s) performing operations on something (Shared Resource).
  Functionally, this is the same as a \nameref{def:Mutex} but allows $x$ \nameref{def:Process}es/\nameref{def:Process}s to enter or use the resource at a time.
  This allows for limits on the number of CPU, I/O or RAM intensive tasks running at the same time.

  A semaphore can only be interacted with through 2 operations \cinline{wait()} and \cinline{signal()}.
  \cinline{wait()} is similar to a \nameref{def:Mutex}'s \cinline{acquire()} function and the \cinline{signal()} function is similar to the \nameref{def:Mutex}'s \cinline{release()} function.
  Here, if a \nameref{def:Process} or \nameref{def:Process} \texttt{wait}s, if there is more of the resource, then the requester gets the resource, and the internal count of the semaphore is decremented.
  If a \nameref{def:Process} \texttt{signal}s, then it is done with the resource, and the requester loses access to the resources, and the internal count of the semaphore is incremented.
  Again, these manipulations \textbf{MUST} be atomic.

  \begin{remark}[Confusion with Mutexes]\label{rmk:Semaphore_Mutex_Confusion}
    Technically, you can create a \nameref{def:Semaphore} that acts like a \nameref{def:Mutex} by giving it a binary value.
    However, this is typically in poor programming taste, because while both function similarly, the \nameref{def:Semaphore} is for signaling the amount of a resource available and the \nameref{def:Mutex} is for signaling if code is capable of execution.
  \end{remark}

  \begin{remark}[Correct Use of Semaphores]\label{rmk:Semaphore_Correct_Usage}
    The correct use of a \nameref{def:Semaphore} is for signaling from one task to another.
    A \nameref{def:Mutex} is meant to be taken and released, always in that order, by each task that uses the shared resource it protects.
    By contrast, tasks that use \nameref{def:Semaphore}s either signal or wait, not both.
  \end{remark}
\end{definition}

\subsubsection{Sockets}\label{subsubsec:Sockets}
\nameref{def:Socket}s are mainly used for network communication.
However, they can be used on the local computer too.

\begin{definition}[Socket]\label{def:Socket}
  A \texttt{socket} is a way of connecting two nodes on a network or \nameref{def:Process}es to communicate with each other.
  One socket listens, while another socket reaches out to the other to form a connection.

  Network sockets have a high overhead due to the software-defined network stacks.
\end{definition}

Almost all modern computers use this today, except for High Performance Computing, which uses their own hardware solutions to reduce latency.

\subsection{Challenges}\label{subsec:IPC_Challenges}
\subsubsection{Link/Endpoint Creation}\label{subsubsec:Link_Endpoint_Creation}
Some common issues with creating a link and/or endpoint are:
\begin{itemize}[noitemsep]
\item Naming the endpoint
\item Looking up the endpoint
\item Need a registry to keep track of this information
\end{itemize}

\subsubsection{Data Transmission}\label{subsubsec:IPC_Challenge-Data_Transmission}
Data transmission has many questions to answer for it to be effective.
These include:
\begin{itemize}[noitemsep]
\item Unidirectional or bidirectional?
\item Single-sender or multi-sender and/or single-receiver or multi-receiver?
\item Speed of the transmission medium/link?
\item Capacity of the transmission medium?
\item Message packetizing? How does the message stream get converted to packets?
\item How is the transmission routed?
\end{itemize}

\subsubsection{Data Synchronization}\label{subsubsec:IPC_Challenge-Data_Sync}

%%% Local Variables:
%%% mode: latex
%%% TeX-master: "../CS_351-Systems_Programming-Reference_Sheet"
%%% End:
