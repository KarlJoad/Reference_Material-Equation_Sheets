\subsection{Remote Procedure Calls}\label{subsec:RPC}
The question here is: How do we make ``distributed computing look like traditional computing''?
\begin{itemize}[noitemsep]
\item The standard Client-Server protocol provides usable mechanisms for services in distributed systems
  \begin{itemize}[noitemsep]
  \item But, these require explicit communication, and require an explicit Send-Receive Paradigm.
  \end{itemize}
\item Can we use procedure calls to do this?
  \begin{itemize}[noitemsep]
  \item The goal here is to make a remote procedure call look like a local procedure call.
  \end{itemize}
\item In distributed system, the callee may be a completely different system from the one executing.
\end{itemize}

\subsubsection{Design Issues}\label{subsubsec:RPC_Design_Issues}
\begin{itemize}[noitemsep]
\item Parameter passing
  \begin{itemize}[noitemsep]
  \item Local:
    \begin{enumerate}[noitemsep]
    \item Parameters passed on the stack before jumping elsewhere
    \item The stack holds the parameters and the possible local variables until you finish the call.
    \item Parameters can be call-by-value or call-by-reference
    \end{enumerate}
  \item Remote:
    \begin{enumerate}[noitemsep]
    \item Simulate parameter passing with *Stubs and Marshaling
      \begin{itemize}[noitemsep]
      \item Client makes procedure call to client stub
      \item Server written as a standard procedure
      \item Stubs take care of packaging arguments and sending messages
      \item The packaging is called marshaling
      \item Stub compiler generates stubs automatically from specifications in an \emph{Interface Definition Language} (IDL)
      \end{itemize}
    \end{enumerate}
  \end{itemize}
\item Binding
\item Reliability
  \begin{itemize}[noitemsep]
  \item How to handle failures
  \item Message loss
  \item Client crash
  \item Server crash
  \end{itemize}
\item Performance and implementation issues
\item Exception handling
\item Interface definition
\end{itemize}

\subsubsection{Steps}\label{subsubsec:RPC_Steps}
\begin{enumerate}[noitemsep]
\item Client procedure calls client stub in normal way
\item Client stub build message and calls local OS (Marshaling)
\item Client's OS sends message to remote OS (Actual socket send happens here)
\item Remote OS gives message to server stub
\item Server stub unpacks parameters, calls server (Can also call multiple functions, need to determine the parameters passed).
\item Server does work and returns results to the server-side stub
\item Server stub marshals the message and calls the local OS
\item Server's OS sends message to client's OS
\item Client's OS gives message to client stub
\item Stub unpacks the result and return to the client
\end{enumerate}

\subsubsection{Marshaling}\label{subsubsec:RPC_Marshaling}

%%% Local Variables:
%%% mode: latex
%%% TeX-master: "../../CS_351-Systems_Programming-Reference_Sheet"
%%% End:
