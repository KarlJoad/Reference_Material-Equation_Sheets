\subsection{Sockets}\label{subsec:Sockets}
In general, networking is done with message-oriented communication.
This is typically done with packets, which is a byte oriented \textbf{stream} of data.
Packets are used because of:
\begin{itemize}[noitemsep]
\item Persistent
\item Synchronicity
\item These packets are routed through potentially multiple computers before reaching a destination
\item You buffer the packets you receive to handle the data as it comes.
\end{itemize}

Sockets are \textbf{not} limited to just inter-computer network communication.
They can also be used within the same computer.
For example:
\begin{itemize}[noitemsep]
\item Emacs (Emacs server + \texttt{emacsclient}) is written to be usable over a socket on the local computer
\item Remote communication happens for email, web browsing, etc.
\end{itemize}

Server doesn't need to \texttt{fork} unless they want to communicate with multiple clients.
Threads can also be used.
The client doesn't need to \texttt{fork} unless they want to communicate with multiple servers simultaneously.
Just like the server, threads can be used for this instead of \nameref{def:Process}es.

\subsubsection{Message-Oriented \emph{Transient} Communication}\label{subsubsec:Transient_Communication}

\subsubsection{Berkeley Socket}\label{subsubsec:Berkeley_Socket}
The typically definition of a socket follows the definition of a \emph{Berkeley socket}.

%%% Local Variables:
%%% mode: latex
%%% TeX-master: "../../CS_351-Systems_Programming-Reference_Sheet"
%%% End:
