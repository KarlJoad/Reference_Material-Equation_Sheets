\subsection{Sockets}\label{subsec:Sockets}
In general, networking is done with message-oriented communication.
This is typically done with packets, which is a byte oriented \textbf{stream} of data.
Packets are used because of:
\begin{itemize}[noitemsep]
\item Persistent
\item Synchronicity
\item These packets are routed through potentially multiple computers before reaching a destination
\item You buffer the packets you receive to handle the data as it comes.
\end{itemize}

Sockets are \textbf{not} limited to just inter-computer network communication.
They can also be used within the same computer.
For example:
\begin{itemize}[noitemsep]
\item Emacs (Emacs server + \texttt{emacsclient}) is written to be usable over a socket on the local computer
\item Remote communication happens for email, web browsing, etc.
\end{itemize}

Server doesn't need to \texttt{fork} unless they want to communicate with multiple clients.
Threads can also be used.
The client doesn't need to \texttt{fork} unless they want to communicate with multiple servers simultaneously.
Just like the server, threads can be used for this instead of \nameref{def:Process}es.

\subsubsection{Message-Oriented \emph{Transient} Communication}\label{subsubsec:Transient_Communication}

\subsubsection{Berkeley Socket}\label{subsubsec:Berkeley_Socket}
The typically definition of a socket follows the definition of a \emph{Berkeley socket}.
\begin{table}[h!tbp]
  \centering
  \begin{tabular}{lp{16cm}}
    \toprule
    Primitive & Meaning\\
    \midrule
    Sockets & Communication End point. When programming, it looks like \nameref{def:File_Descriptor}. \\
    Bind & Attach a local address to a socket. A single computer might have multiple IP addresses/host adapters. This tells the \nameref{def:Process} which network interface to listen on. \\
    Listen & Announce willingness to accept connections. Essentially a blocking call until a connection comes into the system. \\
    Accept & Block caller until a connection request arrives. Have received notification you have received a connection and you allow it (Server-side). \\
    Connection & Actively attempt to establish a connection. (Client-end). \\
    Send & Send some data over the connection. \\
    Receive & Receive some data over the connection. \\
    Close & Release this connection. \\
    \bottomrule
  \end{tabular}
  \caption{Berkeley Socket Primitives}
  \label{tab:Berkley_Socket_Primitives}
\end{table}

\paragraph{Socket Usage}\label{par:Socket_Usage}
To start using sockets, you must distinguish and denote one computer a client and another a server.
Servers behave similarly to the way we are used to running programs, but these \nameref{def:Process}es are typically very long-running.
The usual lifecycle for a socket consists of:
\begin{enumerate}[noitemsep]
\item Socket
\item Bind
\item Listen on a specific port. Now blocks.
  \begin{itemize}[noitemsep]
  \item If the parent forks, then the parent can stay here and accept new connections.
  \end{itemize}
\item Accept
\item Send/Receive
\item Close (Usually client is the one that closes the connection)
\item Wait until a new connection. (Sometimes if the connection is closed, the process ends)
\end{enumerate}

On the other hand, clients behave as shorter-running processes, compared to their server brethren.
They go through these steps:
\begin{enumerate}[noitemsep]
\item Connect
\item Send/Receive
\item Close
\end{enumerate}

\begin{listing}[h!tbp]
\csourcefile{./CS_351-Systems_Programming-Sections/Sockets/Code/server.c}
\caption{\texttt{server.c} as an Example for Server-side Socket Programming}
\label{lst:Server_Side_Socket_Programming}
\end{listing}


%%% Local Variables:
%%% mode: latex
%%% TeX-master: "../../CS_351-Systems_Programming-Reference_Sheet"
%%% End:
