\subsection{MPI (Message Passing Interface)}\label{subsec:Message_Passing_Interface}
\begin{itemize}[noitemsep]
\item Berkeley sockets are designed for general-purpose network communication
  \begin{itemize}[noitemsep]
  \item Simple send/receive primitives
  \item General-purpose protocol stacks such as TCP/IP
  \item These primitives are very heavy, because of the all the work that goes into a network stack
  \end{itemize}
\item Abstraction of these kinds of sockets are not suitable for other protocols in clusters of workstations or massively parallel systems.
  \begin{itemize}[noitemsep]
  \item New, more advanced, primitives are required to make these communications as fast as possible.
  \end{itemize}
\item However, over time, there have become a large number of incompatible proprietary libraries and protocols.
  \begin{itemize}[noitemsep]
  \item This demands a standard interface to be used, called Message Passing Interface (MPI)
  \end{itemize}
\item The MPI Reference is available online at \url{https://www.mcs.anl.gov/mpi/}
\item MPI is:
  \begin{itemize}[noitemsep]
  \item Hardware independent
  \item Primarily for highly parallel applications
  \item Transient communication
  \item Applications that require \textbf{MUCH} lower latency in comparison to sockets.
  \end{itemize}
\item Key idea here is that communication is done between groups of processes
  \begin{itemize}[noitemsep]
  \item Each endpoint is a \texttt{(groupID, processID)} pair
  \end{itemize}
\item Supports most forms of communication
\end{itemize}

\begin{table}[h!tbp]
  \centering
  \begin{tabular}{ll}
    \toprule
    Primitive & Meaning\\
    \midrule
    \cinline{MPI_bsend} & Append outgoing message\\
    \cinline{MPI_...} & \\
    \bottomrule
  \end{tabular}
\end{table}

%%% Local Variables:
%%% mode: latex
%%% TeX-master: "../../CS_351-Systems_Programming-Reference_Sheet"
%%% End:
