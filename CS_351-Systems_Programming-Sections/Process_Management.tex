\section{Process Management}\label{sec:Process_Management}
In almost all modern systems today, there are many, many \nameref{def:Process}es running ``simultaneously''.
That is in quotes because if you have multiple cores/\nameref{def:CPU}s in a single package, you actually \textit{can} run multiple processes at once.
However, we choose to limit our discussion to single core packages, to simplify our discussions and remove a whole class of issues.

By default, there is only one \nameref{def:Process} running at the start of a computer's execution.
We need ways to make more processes, change what \nameref{def:Program}s these processes are executing, and a way to wait for these processes to finish and pick up after them.

\subsection{Making Processes, \texttt{fork}}\label{subsec:Making_Processes-fork}
\texttt{fork} creates a \textbf{copy} of the current \nameref{def:Process}.
This is our \textit{only} method of creating new processes.
The child process is nearly an \textbf{exact} duplicate of the parent process, where only some process metadata in the \nameref{def:Process_Control_Block} is different.
The function prototype for \cinline{fork} is shown in \Cref{lst:PID_Definition_fork_Declaration}.


%%% Local Variables:
%%% mode: latex
%%% TeX-master: "../../CS_351-Systems_Programming-Reference_Sheet."
%%% End:


\subsection{Terminating Processes, \texttt{exit}}\label{subsec:Terminating_Processes-exit}
There are several possible ways for a \nameref{def:Process} to terminate.

\begin{itemize}[noitemsep]
\item The simplest way to terminate a \nameref{def:Process} is for the main process to \texttt{return}.
  If we are being pedantic, the compiler actually implicitly inserts an \texttt{exit} in this case, making all possible exits from a process use \texttt{exit}.

\item The \texttt{exit} \nameref{def:Syscall}.
  \begin{itemize}[noitemsep]
  \item This exits immediately
  \item This may prevent a normal \texttt{return}
  \end{itemize}
\end{itemize}

The standard \textsc{unix} \textbf{convention} is that exit status \texttt{0} is success, and any other value is some error code.

\subsubsection{\texttt{atexit}}\label{subsubsec:atexit}
\cinline{int atexit (void (*fn)())} is a unique function.
It registers a function that will be called after a \nameref{def:Program} has had \texttt{exit} called on it, but before it fully exits.
This registration is achieved by passing a pointer to the function that shoudl be run.
The registartion must happen some time before the \texttt{exit}.
There is no particular place this registration \textbf{MUST} happen though.

In addition, these handlers are inherited by child \nameref{def:Process}es.


%%% Local Variables:
%%% mode: latex
%%% TeX-master: "../../CS_351-Systems_Programming-Reference_Sheet"
%%% End:


\subsection{Getting Values from Processes, \texttt{wait}}\label{subsec:Values_from_Processes-wait}
The \cinline{wait} \nameref{def:Syscall} is the one that allows parent \nameref{def:Process}es to receive values back from their children.
It is only called by a process with $>= 1$ children.

The \cinline{wait} \nameref{def:Syscall}:
\begin{enumerate}[noitemsep]
\item Waits (if needed) for a child to terminate, and returns the \texttt{exit} status of the child.
  This informs the parent:
  \begin{itemize}[noitemsep]
  \item Termination cause
  \item Normal/abnormal termination
  \item Some macros are defined to find out the exit status of a process.
    There are \textbf{MANY} more than the ones below, I just listed a couple.
    \begin{description}[noitemsep]
    \item[\cinline{WIFEXITED(status)}:] Did the process \texttt{exit} normally?
    \item[\cinline{WEXITSTATUS(status)}:] What was the \texttt{exit} status of the child?
    \end{description}
  \end{itemize}

\item Reaps the zombified child.
  If the number of \nameref{def:Zombie_Process}es is $\geq 1$, and no specific one was given to \texttt{wait}, then \texttt{wait} picks a child.
\item Returns the reaped child's \texttt{PID} and exit status via pointer (if non-\texttt{NULL})
\end{enumerate}

If \texttt{wait} is called by a \nameref{def:Process} with no children, \texttt{wait} returns \texttt{-1} and populates \texttt{errno} with an appropriate error code.

How to use \cinline{wait} is shown in \Cref{lst:wait_Usage}.

\begin{listing}[h!tbp]
\csourcefile{./CS_351-Systems_Programming-Sections/Process_Management/Code/wait-usage.c}
\caption{\texttt{wait()} Usage}
\label{lst:wait_Usage}
\end{listing}

\subsubsection{Synchronization Mechanism}\label{subsubsec:wait-Synchronization_Mechanism}
\texttt{wait} also functions as a synchronization mechanism.
If a parent \nameref{def:Process} \texttt{wait}-s for the child to finish, this synchronizes things between the parent and the child.
An example, in code, is shown in \Cref{lst:wait_Sync}.


%%% Local Variables:
%%% mode: latex
%%% TeX-master: "../../CS_351-Systems_Programming-Reference_Sheet"
%%% End:


\subsection{Changing the Running Program, \texttt{exec}}\label{subsec:Change_Running_Program-exec}
\cinline{exec} is almost never used directly.
Instead, its family of syscalls is used, which all provide some amount of abstraction from the base \texttt{exec} call.

All of these are front-ends to \texttt{exec}.
\begin{enumerate}[noitemsep]
\item \cinline{execl}
\item \cinline{execlp}
\item \cinline{execv}
\item \cinline{execvp}
\item \cinline{execve}
\end{enumerate}

The variations in the families are denoted by the last letters in the function.
\begin{description}[noitemsep]
\item[\texttt{l}:] Arguments passed as list of strings to \cinline{main()}.
\item[\texttt{v}:] Arguments passed as array of strings to \cinline{main()}.
\item[\texttt{p}:] Path(s) to search for running program.
\item[\texttt{e}:] Environment (Environment variables and other state) specified by the caller.
\end{description}

Each of these can be mixed to some extent.
The only constant between all of these is that the first argument, the name of the file to execute.

All of these execute a \textbf{new \nameref{def:Program}} within the \textbf{current \nameref{def:Process} context}, meaning \textbf{NO} new \texttt{PID} is given.
When called, \texttt{exec} \textbf{never returns}, because it immediately starts the execution of the new program.
This is because the binary image is replaced in-place.

How to use \cinline{exec} is shown in \Cref{lst:exec_Usage}.

\begin{listing}[h!tbp]
\csourcefile{./CS_351-Systems_Programming-Sections/Process_Management/Code/exec-usage.c}
\caption{\texttt{exec()} Usage}
\label{lst:exec_Usage}

Results:
\begin{minted}[frame=lines,linenos]{console}
$ ./a.out
hello world
\end{minted}
\end{listing}

\cinline{exec} is a strong complement to \cinline{fork}, because we can make a new \nameref{def:Process} with \texttt{fork}, then change the new child to a new program with \texttt{exec}.
An example of this, in code, is shown in \Cref{lst:fork_exec}.

\begin{listing}[h!tbp]
\csourcefile{./CS_351-Systems_Programming-Sections/Process_Management/Code/fork-exec.c}
\caption{Using \texttt{fork()} and \texttt{exec()}}
\label{lst:fork_exec}
\end{listing}

Results:
\begin{minted}[frame=lines,linenos]{console}
$ ./a.out
-rwxr-xr-x 1 ... a.out
-rwxr-xr-x 1 ... demo.c
Command completed
\end{minted}

%%% Local Variables:
%%% mode: latex
%%% TeX-master: "../../CS_351-Systems_Programming-Reference_Sheet"
%%% End:



%%% Local Variables:
%%% mode: latex
%%% TeX-master: "../CS_351-Systems_Programming-Reference_Sheet"
%%% End:
