\subsection{Making Processes, \texttt{fork}}\label{subsec:Making_Processes-fork}
\texttt{fork} creates a \textbf{copy} of the current \nameref{def:Process}.
This is our \textit{only} method of creating new processes.
The child process is nearly an \textbf{exact} duplicate of the parent process, where only some process metadata in the \nameref{def:Process_Control_Block} is different.
The function prototype for \cinline{fork} is shown in \Cref{lst:PID_Definition_fork_Declaration}.

After a \cinline{fork}, the parent and child share the same:
\begin{itemize}[noitemsep]
\item \nameref{def:Register}s:
  \begin{itemize}[noitemsep]
  \item \nameref{def:Program_Counter} \texttt{PC}.
    The child starts \textbf{at the same place in the program as the parent}.
  \item \nameref{def:Stack_Pointer}, \texttt{SP}
  \item \nameref{def:Frame_Pointer}, \texttt{FP}
  \end{itemize}
\item Open Files
\end{itemize}

\begin{listing}[h!tbp]
\begin{minted}[frame=lines,linenos]{c}
#include <unistd.h>

/* typedef int pid_t */

pid_t fork();
/* Makes a system call to trap to the OS.
 * This requests the OS to create a new process.
 * This is mostly a duplicate of the original. */
\end{minted}
\caption{\texttt{pid} Definition and \texttt{fork()} Declaration}
\label{lst:PID_Definition_fork_Declaration}
\end{listing}

\texttt{fork} returns \textbf{twice}.
\begin{itemize}[noitemsep]
\item Once to the parent \nameref{def:Process}, with the \texttt{PID} of the child ($>0$).
\item Once to the child process, with the return code of \cinline{fork}.
  A returned value of \texttt{0} indicates success; it is a sentinel value.
\end{itemize}

\cinline{pid_t} is a system-wide unique Process IDentifier (\texttt{PID}).
It is \texttt{typedef}-ed from an integer, so normal integer arithmetic and rules apply.

\begin{listing}[h!tbp]
\csourcefile{./CS_351-Systems_Programming-Sections/Process_Management/Code/fork-1.c}
\caption{\texttt{fork()} Usage}
\label{lst:fork_Usage-1}
\end{listing}

\Cref{lst:fork_Usage-1} code will print \texttt{Hello World!} twice, but in no particular order.
The reason that we don't have garbage being printed out to the screen is because the \texttt{STDOUT} stream has a lock associated with it, only allowing one \nameref{def:Process} to use the screen at a time.

\begin{listing}[h!tbp]
\csourcefile{./CS_351-Systems_Programming-Sections/Process_Management/Code/fork-2.c}
\caption{\texttt{fork()} Usage}
\label{lst:fork_Usage-2}
\end{listing}

\Cref{lst:fork_Usage-2} will print \texttt{Hello World!} four times, but in no particular order.
The main parent \nameref{def:Process} has 2 children, and the parent's \textbf{first} child makes another child.

\subsubsection{Using Processes}\label{subsubsec:Using_Processes}
There is usually a split in the logical control flow between the parent and child, making them take different actions.
This is possible because the parent and child \nameref{def:Process}es receive different return values.
A simple example of this is shown in \Cref{lst:Using_fork-1}.

\begin{listing}[h!tbp]
\csourcefile{./CS_351-Systems_Programming-Sections/Process_Management/Code/use-fork-1.c}
\caption{Using \texttt{fork()}, Performing Separate Actions}
\label{lst:Using_fork-1}
\end{listing}

The results from \Cref{lst:Using_fork-1} executing is that both print statements are executed.
But, you are not guaranteed the order in which they execute.
Some orders that exist are:
\begin{itemize}[noitemsep]
\item Child prints first, seen in \Cref{lst:fork-1-Child_First}.
  \begin{listing}[h!tbp]
\begin{minted}[frame=lines,linenos]{console}
$ ./a.out
Hello from child
Hello from parent
\end{minted}
\caption{Post-\cinline{fork}, Child Finishes First}
\label{lst:fork-1-Child_First}
  \end{listing}

\item Parent prints first, seen in \Cref{lst:fork-1-Parent_First}.
  \begin{listing}[h!tbp]
\begin{minted}[frame=lines,linenos]{console}
$ ./a.out
Hello from parent
Hello from child
\end{minted}
\caption{Post-\cinline{fork}, Parent Finishes First}
\label{lst:fork-1-Parent_First}
  \end{listing}

\item Child and parent print at the same time.
  But, there is a lock for the screen, blocking multiple \nameref{def:Process}es from printing out at the same time.
  This lock is what makes the output text appear in order.
\end{itemize}

\subsubsection{\texttt{fork} Fails}\label{subsubsec:fork_Fails}
\cinline{fork}, like most other \nameref{def:Syscall}s will return \texttt{-1} on a failure.
The global variable \texttt{errno} is populated with the cause of the failure
To access \texttt{errno}, refer to \Cref{lst:errno}.

\begin{listing}[h!tbp]
\begin{csource}
#include <errno.h>

extern int errno;
\end{csource}
\caption{Using \texttt{errno} to get Error Return Codes}
\label{lst:errno}
\end{listing}

\subsubsection{\texttt{fork} Bomb}\label{subsubsec:fork_Bomb}
A fork bomb just generates new \nameref{def:Process}es as fast as possible, overloading the system.
A simplistic one is shown in \Cref{lst:fork_Bomb}.

\begin{listing}[h!tbp]
\csourcefile{./CS_351-Systems_Programming-Sections/Process_Management/Code/fork-bomb.c}
\caption{\texttt{fork()} Bomb}
\label{lst:fork_Bomb}
\end{listing}

%%% Local Variables:
%%% mode: latex
%%% TeX-master: "../../CS_351-Systems_Programming-Reference_Sheet."
%%% End:
