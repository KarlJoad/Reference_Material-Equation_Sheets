\subsection{Making Processes, \texttt{fork}}\label{subsec:Making_Processes-fork}
\texttt{fork} creates a \textbf{copy} of the current \nameref{def:Process}.
This is our \textit{only} method of creating new processes.
The child process is nearly an \textbf{exact} duplicate of the parent process, where only some process metadata in the \nameref{def:Process_Control_Block} is different.
The function prototype for \cinline{fork} is shown in \Cref{lst:PID_Definition_fork_Declaration}.

After a \cinline{fork}, the parent and child share the same:
\begin{itemize}[noitemsep]
\item \nameref{def:Register}s:
  \begin{itemize}[noitemsep]
  \item \nameref{def:Program_Counter} \texttt{PC}.
    The child starts \textbf{at the same place in the program as the parent}.
  \item \nameref{def:Stack_Pointer}, \texttt{SP}
  \item \nameref{def:Frame_Pointer}, \texttt{FP}
  \end{itemize}
\item Open Files
\end{itemize}


%%% Local Variables:
%%% mode: latex
%%% TeX-master: "../../CS_351-Systems_Programming-Reference_Sheet."
%%% End:
