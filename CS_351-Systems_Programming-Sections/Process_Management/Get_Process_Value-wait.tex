\subsection{Getting Values from Processes, \texttt{wait}}\label{subsec:Values_from_Processes-wait}
The \cinline{wait} \nameref{def:Syscall} is the one that allows parent \nameref{def:Process}es to receive values back from their children.
It is only called by a process with $>= 1$ children.

The \cinline{wait} \nameref{def:Syscall}:
\begin{enumerate}[noitemsep]
\item Waits (if needed) for a child to terminate, and returns the \texttt{exit} status of the child.
  This informs the parent:
  \begin{itemize}[noitemsep]
  \item Termination cause
  \item Normal/abnormal termination
  \item Some macros are defined to find out the exit status of a process.
    There are \textbf{MANY} more than the ones below, I just listed a couple.
    \begin{description}[noitemsep]
    \item[\cinline{WIFEXITED(status)}:] Did the process \texttt{exit} normally?
    \item[\cinline{WEXITSTATUS(status)}:] What was the \texttt{exit} status of the child?
    \end{description}
  \end{itemize}

\item Reaps the zombified child.
  If the number of \nameref{def:Zombie_Process}es is $\geq 1$, and no specific one was given to \texttt{wait}, then \texttt{wait} picks a child.
\item Returns the reaped child's \texttt{PID} and exit status via pointer (if non-\texttt{NULL})
\end{enumerate}

If \texttt{wait} is called by a \nameref{def:Process} with no children, \texttt{wait} returns \texttt{-1} and populates \texttt{errno} with an appropriate error code.

How to use \cinline{wait} is shown in \Cref{lst:wait_Usage}.

\begin{listing}[h!tbp]
\csourcefile{./CS_351-Systems_Programming-Sections/Process_Management/Code/wait-usage.c}
\caption{\texttt{wait()} Usage}
\label{lst:wait_Usage}
\end{listing}

\subsubsection{Synchronization Mechanism}\label{subsubsec:wait-Synchronization_Mechanism}
\texttt{wait} also functions as a synchronization mechanism.
If a parent \nameref{def:Process} \texttt{wait}-s for the child to finish, this synchronizes things between the parent and the child.
An example, in code, is shown in \Cref{lst:wait_Sync}.

\begin{listing}[h!tbp]
\csourcefile{./CS_351-Systems_Programming-Sections/Process_Management/Code/wait-sync.c}
\caption{Using \texttt{wait()} as a Synchronization Tool}
\label{lst:wait_Sync}
\end{listing}

%%% Local Variables:
%%% mode: latex
%%% TeX-master: "../../CS_351-Systems_Programming-Reference_Sheet"
%%% End:
