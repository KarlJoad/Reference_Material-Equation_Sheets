\subsection{Terminating Processes, \texttt{exit}}\label{subsec:Terminating_Processes-exit}
There are several possible ways for a \nameref{def:Process} to terminate.

\begin{itemize}[noitemsep]
\item The simplest way to terminate a \nameref{def:Process} is for the main process to \texttt{return}.
  If we are being pedantic, the compiler actually implicitly inserts an \texttt{exit} in this case, making all possible exits from a process use \texttt{exit}.

\item The \texttt{exit} \nameref{def:Syscall}.
  \begin{itemize}[noitemsep]
  \item This exits immediately
  \item This may prevent a normal \texttt{return}
  \end{itemize}
\end{itemize}

The standard \textsc{unix} \textbf{convention} is that exit status \texttt{0} is success, and any other value is some error code.

\subsubsection{\texttt{atexit}}\label{subsubsec:atexit}
\cinline{int atexit (void (*fn)())} is a unique function.
It registers a function that will be called after a \nameref{def:Program} has had \texttt{exit} called on it, but before it fully exits.
This registration is achieved by passing a pointer to the function that shoudl be run.
The registartion must happen some time before the \texttt{exit}.
There is no particular place this registration \textbf{MUST} happen though.

In addition, these handlers are inherited by child \nameref{def:Process}es.

\subsubsection{Zombie Processes}\label{subsubsec:Zombie_Processes}
All processes become \nameref{def:Zombie_Process} eventually, awaiting to be \nameref{def:Reap}ed.

\begin{definition}[Zombie Process]\label{def:Zombie_Process}
  A \emph{zombie process} is one that is ``dead'', because it finished its execution, but is still tracked by the OS, because the parent has not used/\nameref{def:Reap}ed/\texttt{wait}-ed them yet.\@
  This means:
  \begin{itemize}[noitemsep]
  \item The \texttt{PID} remains in-use.
  \item The child \nameref{def:Process}'s \texttt{exit} status can be queried.
   \end{itemize}

   \textbf{ALL} terminating/terminated processes turn into zombies.

   \begin{remark}[Processes Responsible for Reaping]\label{rmk:Process_Responsible_Reaping}
     All processes are responsible for \nameref{def:Reap}ing their own (immediate) children.
     If a program has 2 forks, the child of the child is \textbf{not} reaped by the original parent.
   \end{remark}

   \begin{remark}[Orphaned Process]\label{rmk:Orphaned_Process}
    If a \nameref{def:Process} is completely \emph{orphaned}, it transfers ownership to \texttt{PID = 1}, which will then \nameref{def:Reap} it.
  \end{remark}
\end{definition}

If the parent \nameref{def:Process} did \textbf{NOT} \nameref{def:Reap} its children, the only way to remove these \nameref{def:Zombie_Process}es is by \textbf{killing} the parent processes.
Note that this is \textbf{not} the same as terminating the process.

\begin{definition}[Reap]\label{def:Reap}
  To \emph{reap} a \nameref{def:Process} is to clean up after the process.
  This means closing any files, freeing any resources used, then reading the \texttt{exit} code from the child (the one being reaped), and freeing the process itself.

  Typically, this is done with the \cinline{wait} \nameref{def:Syscall}.
\end{definition}

%%% Local Variables:
%%% mode: latex
%%% TeX-master: "../../CS_351-Systems_Programming-Reference_Sheet"
%%% End:
