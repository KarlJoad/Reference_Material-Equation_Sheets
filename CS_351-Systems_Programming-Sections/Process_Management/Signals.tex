\subsection{Signals}\label{subsec:Signals}
One way to pass information between \nameref{def:Process}es that are not constantly, directly communicating is through the use of \nameref{def:Signal}s.

\begin{definition}[Signal]\label{def:Signal}
  \emph{Signal}s are messages delivered by the kernel to the user \nameref{def:Process}es.
  These can occur in the cases of:
  \begin{itemize}[noitemsep]
  \item In response to OS events (segfault)
  \item The request of another process
  \end{itemize}

  Signals are delivered by a \nameref{def:Signal_Handler} function in the \textbf{receiving process}.
\end{definition}

An example of how \nameref{def:Signal}s can be used is shown in \Cref{lst:Using_Signals}.

\begin{listing}[h!tbp]
\csourcefile{./CS_351-Systems_Programming-Sections/Process_Management/Code/use-signals.c}
\caption{Using Signals}
\label{lst:Using_Signals}
\end{listing}


%%% Local Variables:
%%% mode: latex
%%% TeX-master: "../../CS_351-Systems_Programming-Reference_Sheet"
%%% End:
