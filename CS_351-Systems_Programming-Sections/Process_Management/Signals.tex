\subsection{Signals}\label{subsec:Signals}
One way to pass information between \nameref{def:Process}es that are not constantly, directly communicating is through the use of \nameref{def:Signal}s.

\begin{definition}[Signal]\label{def:Signal}
  \emph{Signal}s are messages delivered by the kernel to the user \nameref{def:Process}es.
  These can occur in the cases of:
  \begin{itemize}[noitemsep]
  \item In response to OS events (segfault)
  \item The request of another process
  \end{itemize}

  Signals are delivered by a \nameref{def:Signal_Handler} function in the \textbf{receiving process}.
\end{definition}

An example of how \nameref{def:Signal}s can be used is shown in \Cref{lst:Using_Signals}.

\begin{listing}[h!tbp]
\csourcefile{./CS_351-Systems_Programming-Sections/Process_Management/Code/use-signals.c}
\caption{Using Signals}
\label{lst:Using_Signals}
\end{listing}

It can be useful to send a signal to multiple processes at once, in which case, the signal can be sent to a \nameref{def:Process_Group}.

\subsubsection{Signal Lifecycle}\label{subsubsec:Signal_Lifecycle}
\begin{enumerate}[noitemsep]
\item Sending a signal to a process or a \nameref{def:Process_Group}.
  \begin{itemize}[noitemsep]
  \item The \cinline{void kill(pid_t pid, int sig)} function is an example of a function that sends a \nameref{def:Signal}.
  \item Give the process with \texttt{pid} the \nameref{def:Signal} \texttt{sig}.
  \item There is a list of signals with names and values.
    The actual list is \textbf{MUCH} longer, but a few are shown below.
    \begin{description}[noitemsep]
    \item[\cinline{1} \cinline{SIGHUP}:] Terminate process, Terminal line hangup
    \item[\cinline{2} \cinline{SIGINT}:] Terminate process, Interrupt program.
    \item[\cinline{3} \cinline{SIGQUIT}:] Create core image/dump, quits program.
    \end{description}
  \end{itemize}

\item Registering a \nameref{def:Signal_Handler} for a given \nameref{def:Signal}.
  \begin{itemize}[noitemsep]
  \item Some \nameref{def:Signal}s cannot be caught by the \nameref{def:Process}.
  \item The function \cinline{sig_t signal(int sig, sig_t func)} registers a function (\texttt{func}) to a particular signal (\texttt{sig}).
  \item Children inherit their parent's signal handlers after a \texttt{fork}.
  \item Children lose their parent's signal handlers when they \texttt{exec} to another \nameref{def:Program}.
  \end{itemize}

\item Delivering a \nameref{def:Signal} to a \nameref{def:Process} (done by kernel).
  \begin{itemize}[noitemsep]
  \item 2 bitmaps per \nameref{def:Process}
    \begin{enumerate}[noitemsep]
    \item Pending
    \item Blocked
    \end{enumerate}
  \item Some \nameref{def:Signal}s cannot be delivered/blocked (\texttt{SIGKILL}, and others)
  \item Newly \texttt{fork}-ed child inherits the parent's blocked bitmap, but pending vector is empty.
  \end{itemize}

\item Designing a \nameref{def:Signal_Handler}.
\end{enumerate}

\subsubsection{Process Groups}\label{subsubsec:Process_Groups}
\begin{definition}[Process Group]\label{def:Process_Group}
  A \emph{process group} is a way to logically group several \nameref{def:Process}es together.
  Child processes \texttt{fork}-ed inherit their process group ID \texttt{pgid} from their parent.
  This leads to the following properties:
  \begin{propertylist}
  \item The founder of the group becomes the group leader.
  \item The group leader is the \nameref{def:Process} where \cinline{pid == pgid}.
  \item A \nameref{def:Process} can become a group leader by \cinline{setpgrp}.
  \item The whole \nameref{def:Process_Group} interacts with the \nameref{def:Signal}s.
  \item If \cinline{kill} is given a negative value, it will kill the corresponding process group.
  \end{propertylist}
\end{definition}

\begin{remark*}
  For most \nameref{def:Process}es started from a shell, they inherit the shell's \nameref{def:Process_Group} ID.\@
  This can be changed through the \cinline{setpgrp} function.
  If this is done, any subsequent child processes started by this one will inherit that new \texttt{pgid}.
\end{remark*}

\subsubsection{Registering Signal Handlers}\label{subsubsec:Register_Signal_Handlers}
Before we run our program, many \nameref{def:Signal_Handler}s are registered to handle certain \nameref{def:Signal}s in certain ways.
However, we can choose to override them, or define new ones, if we so choose.

\begin{definition}[Signal Handler]\label{def:Signal_Handler}
  A \emph{signal handler} is a function that has been registered to handle one particular signal.
  These can be used to override some default handlers, but there are some handler functions, like the one for \texttt{SIGTERM} that are not allowed to be changed.
\end{definition}


%%% Local Variables:
%%% mode: latex
%%% TeX-master: "../../CS_351-Systems_Programming-Reference_Sheet"
%%% End:
