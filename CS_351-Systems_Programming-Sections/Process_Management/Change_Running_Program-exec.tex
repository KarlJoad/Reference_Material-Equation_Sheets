\subsection{Changing the Running Program, \texttt{exec}}\label{subsec:Change_Running_Program-exec}
\cinline{exec} is almost never used directly.
Instead, its family of syscalls is used, which all provide some amount of abstraction from the base \texttt{exec} call.

All of these are front-ends to \texttt{exec}.
\begin{enumerate}[noitemsep]
\item \cinline{execl}
\item \cinline{execlp}
\item \cinline{execv}
\item \cinline{execvp}
\item \cinline{execve}
\end{enumerate}

The variations in the families are denoted by the last letters in the function.
\begin{description}[noitemsep]
\item[\texttt{l}:] Arguments passed as list of strings to \cinline{main()}.
\item[\texttt{v}:] Arguments passed as array of strings to \cinline{main()}.
\item[\texttt{p}:] Path(s) to search for running program.
\item[\texttt{e}:] Environment (Environment variables and other state) specified by the caller.
\end{description}


%%% Local Variables:
%%% mode: latex
%%% TeX-master: "../../CS_351-Systems_Programming-Reference_Sheet"
%%% End:
