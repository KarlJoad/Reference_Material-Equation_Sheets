\subsection{Changing the Running Program, \texttt{exec}}\label{subsec:Change_Running_Program-exec}
\cinline{exec} is almost never used directly.
Instead, its family of syscalls is used, which all provide some amount of abstraction from the base \texttt{exec} call.

All of these are front-ends to \texttt{exec}.
\begin{enumerate}[noitemsep]
\item \cinline{execl}
\item \cinline{execlp}
\item \cinline{execv}
\item \cinline{execvp}
\item \cinline{execve}
\end{enumerate}

The variations in the families are denoted by the last letters in the function.
\begin{description}[noitemsep]
\item[\texttt{l}:] Arguments passed as list of strings to \cinline{main()}.
\item[\texttt{v}:] Arguments passed as array of strings to \cinline{main()}.
\item[\texttt{p}:] Path(s) to search for running program.
\item[\texttt{e}:] Environment (Environment variables and other state) specified by the caller.
\end{description}

Each of these can be mixed to some extent.
The only constant between all of these is that the first argument, the name of the file to execute.

All of these execute a \textbf{new \nameref{def:Program}} within the \textbf{current \nameref{def:Process} context}, meaning \textbf{NO} new \texttt{PID} is given.
When called, \texttt{exec} \textbf{never returns}, because it immediately starts the execution of the new program.
This is because the binary image is replaced in-place.


%%% Local Variables:
%%% mode: latex
%%% TeX-master: "../../CS_351-Systems_Programming-Reference_Sheet"
%%% End:
