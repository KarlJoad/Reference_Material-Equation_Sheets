\section{Input/Output (I/O)}\label{sec:IO}
In \textsc{unix}, I/O devices include:
\begin{itemize}[noitemsep]
\item Disk
\item Terminal
\item Shared Memory
\item Printer
\item Network
\end{itemize}

This is mostly because \textsc{unix} made the design decision to try to view every component of a system as a \nameref{def:File}.

Due to the variety of I/O devices that need to be supported, there are a vast number of different mechanisms for using these devices.
But, there are a few common mechanisms, requirements, and activities:
\begin{itemize}[noitemsep]
\item Read/Write Ops
\item Metadata:
  \begin{itemize}[noitemsep]
  \item Name
  \item Position
  \item Directory Name
  \item Creation Date
  \item Last Access Date
  \item IP Address
  \item MAC Address
  \item TCP Packet Sequence Number
  \end{itemize}
\item Robustness
\item Thread-safety
\end{itemize}

There are few general concerns that we need to have about the idea of viewing everything as a \nameref{def:File}.
\begin{itemize}[noitemsep]
\item How are I/O endpoints represented?
  \begin{itemize}[noitemsep]
  \item \nameref{def:File_Descriptor}
  \end{itemize}
\item How do we perform I/O?\@
  \begin{itemize}[noitemsep]
  \item Byte at a time
  \item Give a chunk of memory and later check that the requested I/O completed?
  \end{itemize}
\item How do we perform I/O \emph{efficiently}?
  \begin{itemize}[noitemsep]
  \item Efficiency depends on what we define efficient to be. Essentially, what are we optimizing for?
  \end{itemize}
\end{itemize}

\subsection{I/O Devices}\label{subsec:IO_Devices}
There are 2 major types of I/O devices on \textsc{unix} systems:
\begin{enumerate}[noitemsep]
\item \nameref{def:Block_Device}s
\item \nameref{def:Character_Device}s
\end{enumerate}

\begin{definition}[Block Device]\label{def:Block_Device}
  A \emph{block device} is an I/O device that accesses and stores data in fixed-sized blocks.
  Typically, this also means they have fixed total size as well.
  This means they support seeking through their contents and random access for parts of their contents.

  Some typical devices classifed as this are:
  \begin{itemize}[noitemsep]
  \item Disk
  \item Memory
  \end{itemize}
\end{definition}

\begin{definition}[Character Device]\label{def:Character_Device}
  A \emph{character device} is an I/O device that access and receives data as a stream.
  This means it receives ``characters'' as a stream, one-by-one.
  There is no support for seeking or random access of the stream, because we are getting the data as soon as it is being given.

  Some typical devices in this category are:
  \begin{itemize}[noitemsep]
  \item Network
  \item Mouse
  \item Keyboard
  \end{itemize}
\end{definition}

\subsection{Filesystem}\label{subsec:Filesystem}
The filesystem acts as a namespace for devices, and allows for the efficient storage of data on \nameref{def:Block_Device}s.
A typical file system consists of two types of files:
\begin{enumerate}[noitemsep]
\item \textit{Regular files} consist of ASCII or binary data
  \begin{itemize}[noitemsep]
  \item Directories
  \end{itemize}
\item \textit{Special Files} may represent:
  \begin{itemize}[noitemsep]
  \item In-memory structures
  \item Sockets
  \item Raw Devices
  \end{itemize}
\end{enumerate}

\subsection{Files}\label{subsec:Files}
\begin{definition}[File]\label{def:File}
  A \emph{file} is an operating system abstraction over some other data.
  It allows us to interact with many different file systems and devices over a variety of protocols in a abstract and concise way.
  A file can be accessed by using a \textit{fully qualified path}.

  The only thing each file \textbf{MUST} have is a unique \nameref{def:inode}.
\end{definition}

\begin{definition}[\texttt{inode}]\label{def:inode}
  The \emph{inode} is a filesystem-unique number (Typically, there is one filesystem per device, so this is typically a per-device-unique number).
  The inode tracks:
  \begin{itemize}[noitemsep]
  \item Ownership
  \item Permissions
  \item Size
  \item Type
  \item Location
  \item Number of \nameref{def:Hard_Link}s
  \end{itemize}
\end{definition}

\begin{definition}[Hard Link]\label{def:Hard_Link}
  A \emph{hard link} is a link between \nameref{def:inode}s.
  Thus, they each point to the same data, and must have the same name.
  When one of the links is deleted, the total count for that inode decreases.
  Once the inode reaches \texttt{0} hard links, it is removed (deleted) from the system.
\end{definition}

\begin{definition}[Symlink]\label{def:Symlink}\label{def:Symbolic_Link}\label{def:Soft_Link}
  A \emph{symlink} (\emph{symbolic link}, \emph{soft link}) is a link between \nameref{def:File}s.
  Thus, the link points to the file, which then points to the data.
  The link is not required to have the same name as the original file.
  However, if the file that the symlink is pointing to is deleted, then we are left with a dangling pointer.
\end{definition}


%%% Local Variables:
%%% mode: latex
%%% TeX-master: "../CS_351-Systems_Programming-Reference_Sheet"
%%% End:
