\section{Processes}\label{sec:Processes}
\nameref{def:Process}es are the fundamental unit of computation within an operating system.
\begin{definition}[Process]\label{def:Process}
  A \emph{process} is a \nameref{def:Program} in execution.
  A process carries out the computation that we specify.
  A process contains:
  \begin{itemize}[noitemsep]
  \item Code (\texttt{text}) of your program.
  \item Runtime data (Global, local, dynamic variables)
  \item \nameref{def:Register}s:
    \begin{itemize}[noitemsep]
    \item \nameref{def:Program_Counter} (\texttt{PC})
    \item \nameref{def:Stack_Pointer} (\texttt{SP})
    \item \nameref{def:Frame_Pointer} (\texttt{FP})
    \end{itemize}
  \item \nameref{def:Process_Control_Block}
  \end{itemize}
\end{definition}

\begin{definition}[Program]\label{def:Program}
  A \emph{program} is the binary image stored at some file location on the storage medium.
  The program is read into memory, and then is used to start a \nameref{def:Process} that runs that program.
\end{definition}

\nameref{def:Process}es require both a \textbf{predictable} and \textbf{logical} control flow.
This means:
\begin{itemize}[noitemsep]
\item The \nameref{def:Process} must start somewhere, typically defined to be \texttt{main}.
\item Nothing can disrupt a program mid-execution.
  \begin{itemize}[noitemsep]
  \item This is further discussed in \Cref{subsec:Prevent_Process_Disruption}.
  \end{itemize}
\end{itemize}

\subsection{Prevent Process Disruption}\label{subsec:Prevent_Process_Disruption}
The easiest way to prevent a \nameref{def:Process} from having its control flow being interrupted is for the process to ``own'' the CPU for the entire duration of the process's execution.
However, this means:
\begin{itemize}[noitemsep]
\item No other process can run on this core
\item This prevents efficient multi-\nameref{def:Process}/multitasking systems
\item Malicious or poorly written program can ``take over'' the CPU
\item An idle process (for example, waiting for user input) will underutilize the CPU
\end{itemize}

For the operating system to simulate this seamless logical control flow, we use all of the information used to make a \nameref{def:Process}, and need a \nameref{def:Process_Control_Block}.
\begin{definition}[Process Control Block]\label{def:Process_Control_Block}
  The \emph{Process Control Block} (\emph{PCB}) contains additional metadata about a \nameref{def:Process}.
  This includes:
  \begin{itemize}[noitemsep]
  \item Process ID (\texttt{PID})
  \item CPU Usage
  \item Memory Usage
  \item Pending \nameref{def:Syscall}s
  \end{itemize}
\end{definition}

The \nameref{def:Process_Control_Block} is sued to allow \nameref{def:Process}es to be interrupted, saved, and moved off a core.
This allows the operating system to schedule processes according to some algorithm.

\begin{definition}[Syscall]\label{def:Syscall}
  A \emph{syscall}, short for a \emph{system call} is a way for a user-level \nameref{def:Process} to perform some computation that the operating system kernel restricts.
  Some common syscalls are:
  \begin{itemize}[noitemsep]
  \item Opening a file
  \item Reading a file
  \item Writing a file
  \item Closing a file
  \item Creating a process
  \item Changing the process's binary
  \item Reading from the network
  \item Writing to the network
  \end{itemize}
\end{definition}

\subsection{Required Hardware}\label{subsec:Required_Hardware}
Interrupting the execution of a \nameref{def:Process} requires some hardware support to be possible and efficient.
We need 3 things:
\begin{enumerate}[noitemsep]
\item A hardware mechanism to periodically interrupt the current \nameref{def:Process} to change execution to the operating system.
  \begin{itemize}[noitemsep]
  \item This is usually the \emph{periodic clock interrupt}.
  \end{itemize}
\item An operating system procedure to decide which processes to run, and in which order.
  \begin{itemize}[noitemsep]
  \item This is the operating system's \nameref{def:Process_Scheduler}.
  \end{itemize}
\item A routine for seamlessly switching between processes.
  \begin{itemize}[noitemsep]
  \item This is called a \nameref{def:Context_Switch}.
  \item Relatively speaking, these are expensive to perform.
  \item \textbf{These are external to a \nameref{def:Process}'s logical control flow}.
  \item This forms part of the process of \nameref{def:Exceptional_Control_Flow}.
  \item A \nameref{def:Context_Switch} makes no guarantee about if and/or when this \nameref{def:Process} will start running again.
  \item A \nameref{def:Context_Switch} is the only way to invoke \nameref{def:Syscall}s.
  \end{itemize}
\end{enumerate}

To schedule \nameref{def:Process}es onto one of possibly many CPUs, there are programs called \nameref{def:Process_Scheduler}s.

Every time the \nameref{def:Process_Scheduler} schedules a new \nameref{def:Process}, or when a process needs to perform a \nameref{def:Syscall}, or a kernel-level exception occurs, a \nameref{def:Context_Switch} is made.
\begin{definition}[Context Switch]\label{def:Context_Switch}
  A \emph{context switch} is the process of interrupting a \nameref{def:Process}, saving its current state, and scheduling something else to run on that CPU.\@
  This is done whenever a \nameref{def:Syscall} is made, and happens sometimes during a process's lifetime.
\end{definition}

\nameref{def:Context_Switch}es form a core part of the \nameref{def:Exceptional_Control_Flow}.

\subsection{Process Scheduling}\label{subsec:Process_Scheduling}
Process scheduling is the process by which a process is put ``scheduled'' onto a particular CPU, according to some criteria.
There are any number of scheduling algorithms, each of which optimizes for certain cases, and may yield a different order of process execution.
\begin{definition}[Process Scheduler]\label{def:Process_Scheduler}
  The \emph{process scheduler} is an operating system component that chooses which \nameref{def:Process} to run next.
  It chooses this according to some algorithm, each of which might change the order of process execution.
\end{definition}

One such \nameref{def:Process_Scheduler} is the \nameref{subsubsec:Priority_Scheduling} algorithm.

\subsubsection{Priority Scheduling}\label{subsubsec:Priority_Scheduling}
\emph{Priority scheduling} involves placing a priority on every \nameref{def:Process} that can be scheduled.
Then, the process with the highest priority is chosen first, working our way down to the lowest priority.
This is, partly, the setup most modern operating systems take today.

However, priority scheduling creates new issues that must be dealt with.
One of these is \nameref{def:Starvation}.

\begin{definition}[Starvation]\label{def:Starvation}
  \emph{Starvation} is when something that needs a resource to function does not receive this resource.
\end{definition}

In this case, a lower-priority \nameref{def:Process} can experience \nameref{def:Starvation} if only higher-priority processes are present, and continually steal CPU execution time.

\subsection{Exceptional Control Flow}\label{subsec:Exceptional_Control_Flow}
To illustrate the power of \nameref{def:Exceptional_Control_Flow}, we will use an example piece of code, \Cref{lst:Exceptional_Control_Flow}.

\begin{definition}[Exceptional Control Flow]\label{def:Exceptional_Control_Flow}
  \emph{Exceptional control flow} is designated by the fact that most of the computation involved is done in response to exceptions or special events.
  Which one depends on what ``exceptional'' means in that circumstance.
\end{definition}

\begin{listing}[h!tbp]
\csourcefile{./CS_351-Systems_Programming-Sections/Processes/Code/Exceptional_Control_Flow.c}
\caption{Exceptional Control Flow Example}
\label{lst:Exceptional_Control_Flow}
\end{listing}

If there was an exception of any kind during the execution of the \texttt{while}-loop, then the process would be interrupted by the kernel, and \nameref{def:Context_Switch}ed out.
From there, the exception would be handled (if possible), and then the operating system would continue execution (if possible).
While the execution of the \nameref{def:Process} was non-sequential, the process \textbf{is not aware} of this fact.
{\large\textbf{ALL}} of the process's state is saved \textbf{before} the context switch, so that when the process is switched back in, it continues executing from the \textbf{same} spot as before.

There are 2 kinds of exceptions in operating systems today:
\begin{enumerate}[noitemsep]
\item \nameref{subsubsec:Synchronous_Exceptions}
\item \nameref{subsubsec:Asynchronous_Exceptions}
\end{enumerate}


%%% Local Variables:
%%% mode: latex
%%% TeX-master: "../CS_351-Systems_Programming-Reference_Sheet"
%%% End:
