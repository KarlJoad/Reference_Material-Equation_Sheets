\section{Processes}\label{sec:Processes}
\nameref{def:Process}es are the fundamental unit of computation within an operating system.
\begin{definition}[Process]\label{def:Process}
  A \emph{process} is a \nameref{def:Program} in execution.
  A process carries out the computation that we specify.
  A process contains:
  \begin{itemize}[noitemsep]
  \item Code (\texttt{text}) of your program.
  \item Runtime data (Global, local, dynamic variables)
  \item \nameref{def:Register}s:
    \begin{itemize}[noitemsep]
    \item \nameref{def:Program_Counter} (\texttt{PC})
    \item \nameref{def:Stack_Pointer} (\texttt{SP})
    \item \nameref{def:Frame_Pointer} (\texttt{FP})
    \end{itemize}
  \item \nameref{def:Process_Control_Block}
  \end{itemize}
\end{definition}

\begin{definition}[Program]\label{def:Program}
  A \emph{program} is the binary image stored at some file location on the storage medium.
  The program is read into memory, and then is used to start a \nameref{def:Process} that runs that program.
\end{definition}

\nameref{def:Process}es require both a \textbf{predictable} and \textbf{logical} control flow.
This means:
\begin{itemize}[noitemsep]
\item The \nameref{def:Process} must start somewhere, typically defined to be \texttt{main}.
\item Nothing can disrupt a program mid-execution.
  \begin{itemize}[noitemsep]
  \item This is further discussed in \Cref{subsec:Prevent_Process_Disruption}.
  \end{itemize}
\end{itemize}

\subsection{Prevent Process Disruption}\label{subsec:Prevent_Process_Disruption}
The easiest way to prevent a \nameref{def:Process} from having its control flow being interrupted is for the process to ``own'' the CPU for the entire duration of the process's execution.
However, this means:
\begin{itemize}[noitemsep]
\item No other process can run on this core
\item This prevents efficient multi-\nameref{def:Process}/multitasking systems
\item Malicious or poorly written program can ``take over'' the CPU
\item An idle process (for example, waiting for user input) will underutilize the CPU
\end{itemize}

For the operating system to simulate this seamless logical control flow, we use all of the information used to make a \nameref{def:Process}, and need a \nameref{def:Process_Control_Block}.
\begin{definition}[Process Control Block]\label{def:Process_Control_Block}
  The \emph{Process Control Block} (\emph{PCB}) contains additional metadata about a \nameref{def:Process}.
  This includes:
  \begin{itemize}[noitemsep]
  \item Process ID (\texttt{PID})
  \item CPU Usage
  \item Memory Usage
  \item Pending \nameref{def:Syscall}s
  \end{itemize}
\end{definition}


%%% Local Variables:
%%% mode: latex
%%% TeX-master: "../CS_351-Systems_Programming-Reference_Sheet"
%%% End:
