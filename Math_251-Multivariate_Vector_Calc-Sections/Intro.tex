\section{Introduction}\label{sec:Intro}
\subsection{Multiple Dimensions}\label{subsec:Multiple_Dimensions}
Throughout this course, we will be working with multidimensional objects.
In this case, we will be drawing each of the terms of the multidimensional ``thing'' whatever it may be from the set of all real numbers.

In real-world terms, a dimension is just a way to measure something.
So, the outer shape of your calculator can be described by a set of three dimensional equations.

This means that the set of numbers $\RealNumbers^{2}$ is the traditional $xy$-plane for graphing one-dimensional equations.
The set of numbers $\RealNumbers^{3}$ is the traditional $xyz$ 3-dimensional graph.

\begin{remark*}
  Remember that there are \textbf{no} such things as quadrants here.
  Instead, the only way we can describe the traditional 2-D graph is by using a plane.
  Also, the individual spaces that can be identified by which direction of the origin they are lying in is called an octant.
\end{remark*}

If you take the traditional Euclidean equation for a circle, shown below, and expand it to three dimensions, you end up with a cylinder.
\begin{equation*}
  x^{2} + y^{2} = 1
\end{equation*}
This is because the $z$ term/dimension is \textbf{not} present in the equation, so $z$ is able to vary through all possible values, so long as the other constraints are satisifed.

\subsubsection{Distance in Three Dimensions}\label{subsubsec:3D_Distance}
The distance of a point any other point in three dimensions is found by a general extension to the Pythagoran Theorem.
The actual equation is shown in \Cref{eq:3D_Distance}
\begin{equation}\label{eq:3D_Distance}
  D = \sqrt{{(x_{2}-x_{1})}^{2} + {(y_{2}-y_{1})}^{2} + {(z_{2}-z_{2})}^{2}}
\end{equation}

\subsection{Vectors}\label{subsec:Vectors}
\begin{definition}[Vector]\label{def:Vector}
  A \emph{vector} in mathematics behaves identically to the way it behaves in science and engineering, i.e.\ a vector is a quantity with \textbf{both} a direction and a magnitude.

  However, we have a slightly more mathematical definition we can leverage as well, given below.
  A vector is an ordered list of $n$ numbers, typically denoted $\langle x_{1}, x_{2}, \ldots, x_{n} \rangle$.
  \begin{equation}\label{eq:Vector}
    v \in \RealNumbers^{n} = \langle x_{1}, x_{2}, \ldots, x_{n} \rangle
  \end{equation}
\end{definition}


%%% Local Variables:
%%% mode: latex
%%% TeX-master: "../../Math_251-Multivariate_Vector_Calc-Reference_Material."
%%% End:


\subsection{Dot Product}\label{subsec:Vector_Dot_Product}
\begin{definition}[Dot Product]\label{def:Dot_Product}
  The \emph{dot product} is a way to define the multiplication of two vectors that results in a single scalar value.
  The dot product is defined as shown in \Cref{eq:Dot_Product}.
  \begin{equation}\label{eq:Dot_Product}
    \begin{aligned}
      \langle a, b, c \rangle \dot \langle x, y, z \rangle &= ax + by + cz \\
      &= \magnitude{\langle a, b, c \rangle} \magnitude{\langle x, y, z \rangle} \cos(\theta), \text{where} \:\: 0 \leq \theta \leq \pi \\
    \end{aligned}
  \end{equation}
\end{definition}


%%% Local Variables:
%%% mode: latex
%%% TeX-master: "../../Math_251-Multivariate_Vector_Calc-Reference_Material"
%%% End:



%%% Local Variables:
%%% mode: latex
%%% TeX-master: "../Math_251-Multivariate_Vector_Calc-Reference_Material"
%%% End:
