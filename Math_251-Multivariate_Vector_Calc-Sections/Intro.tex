\section{Introduction}\label{sec:Intro}
\subsection{Multiple Dimensions}\label{subsec:Multiple_Dimensions}
Throughout this course, we will be working with multidimensional objects.
In this case, we will be drawing each of the terms of the multidimensional ``thing'' whatever it may be from the set of all real numbers.

In real-world terms, a dimension is just a way to measure something.
So, the outer shape of your calculator can be described by a set of three dimensional equations.

This means that the set of numbers $\RealNumbers^{2}$ is the traditional $xy$-plane for graphing one-dimensional equations.
The set of numbers $\RealNumbers^{3}$ is the traditional $xyz$ 3-dimensional graph.

\begin{remark*}
  Remember that there are \textbf{no} such things as quadrants here.
  Instead, the only way we can describe the traditional 2-D graph is by using a plane.
  Also, the individual spaces that can be identified by which direction of the origin they are lying in is called an octant.
\end{remark*}

If you take the traditional Euclidean equation for a circle, shown below, and expand it to three dimensions, you end up with a cylinder.
\begin{equation*}
  x^{2} + y^{2} = 1
\end{equation*}
This is because the $z$ term/dimension is \textbf{not} present in the equation, so $z$ is able to vary through all possible values, so long as the other constraints are satisifed.

\subsubsection{Distance in Three Dimensions}\label{subsubsec:3D_Distance}
The distance of a point any other point in three dimensions is found by a general extension to the Pythagoran Theorem.
The actual equation is shown in \Cref{eq:3D_Distance}
\begin{equation}\label{eq:3D_Distance}
  D = \sqrt{{(x_{2}-x_{1})}^{2} + {(y_{2}-y_{1})}^{2} + {(z_{2}-z_{2})}^{2}}
\end{equation}

\subsection{Vectors}\label{subsec:Vectors}
\begin{definition}[Vector]\label{def:Vector}
  A \emph{vector} in mathematics behaves identically to the way it behaves in science and engineering, i.e.\ a vector is a quantity with \textbf{both} a direction and a magnitude.

  However, we have a slightly more mathematical definition we can leverage as well, given below.
  A vector is an ordered list of $n$ numbers, typically denoted $\langle x_{1}, x_{2}, \ldots, x_{n} \rangle$.
  \begin{equation}\label{eq:Vector}
    v \in \RealNumbers^{n} = \langle x_{1}, x_{2}, \ldots, x_{n} \rangle
  \end{equation}
\end{definition}


%%% Local Variables:
%%% mode: latex
%%% TeX-master: "../../Math_251-Multivariate_Vector_Calc-Reference_Material."
%%% End:


\subsection{Dot Product}\label{subsec:Vector_Dot_Product}
\begin{definition}[Dot Product]\label{def:Dot_Product}
  The \emph{dot product} is a way to define the multiplication of two vectors that results in a single scalar value.
  The dot product is defined as shown in \Cref{eq:Dot_Product}.
  \begin{equation}\label{eq:Dot_Product}
    \begin{aligned}
      \langle a, b, c \rangle \dot \langle x, y, z \rangle &= ax + by + cz \\
      &= \magnitude{\langle a, b, c \rangle} \magnitude{\langle x, y, z \rangle} \cos(\theta), \text{where} \:\: 0 \leq \theta \leq \pi \\
    \end{aligned}
  \end{equation}
\end{definition}


%%% Local Variables:
%%% mode: latex
%%% TeX-master: "../../Math_251-Multivariate_Vector_Calc-Reference_Material"
%%% End:


\subsection{Cross Product}\label{subsec:Cross_Product}

%%% Local Variables:
%%% mode: latex
%%% TeX-master: "../../Math_251-Multivariate_Vector_Calc-Reference_Material"
%%% End:


\subsection{Cartesian and Parametric Equations}\label{subsec:Cartesian_Parametric}
\begin{definition}[Cartesian]\label{def:Cartesian}
  A \emph{Cartesian} equation is the one that most people are used to.
  For example, for a line:
  \begin{equation}\label{eq:Generic_Cartesian_Line}
    y(x) = mx + b
  \end{equation}

  In a Cartesian equation, there are the same number of variables as dimensions in the problem.
\end{definition}

\begin{definition}[Parametric]\label{def:Parametric}
  A \emph{parametric} equation is actually a set of equations that together form the resulting vector.
  For the same line as in \Cref{eq:Generic_Cartesian_Line}, we have a parametric equation as shown below.
  \begin{equation}\label{eq:Generic_Parametric_Line}
    \langle x(t), y(t) \rangle = \langle t, mt + b \rangle
  \end{equation}

  In a parametric equation, there are the same number of parameterized equations as dimensions.
\end{definition}

There are 2 standard \nameref{def:Parametric} equations that we will be using:
\begin{enumerate}[noitemsep]
\item \nameref{subsubsec:Parametric_Lines}, \Cref{eq:Parametric_Line}
\item \nameref{subsubsec:Parametric_Planes}, \Cref{eq:Parametric_Plane}
\end{enumerate}

\subsubsection{Parametric Lines}\label{subsubsec:Parametric_Lines}
The \nameref{def:Parametric} equation for a line (\Cref{eq:Parametric_Line}).
\begin{equation}\label{eq:Parametric_Line}
  \vec{r}(t) = t\vec{v} + \vec{p}
\end{equation}

You might notice that this resembles the definition of a generic \nameref{def:Cartesian} line, seen in \Cref{eq:Generic_Cartesian_Line}, but:
\begin{itemize}[noitemsep]
\item $\vec{r}(t)$ replaces $y(x)$
\item $t$ replaces $m$
\item $\vec{v}$ replaces $x$
\item $\vec{p}$ replaces $b$
\end{itemize}

Each of the terms used in a Cartesian equation for a line must be related to each other and $t$ for \Cref{eq:Parametric_Line} to work.
\begin{equation}\label{eq:Parametric_Cartesian_Line}
  t = \frac{x-p_{x}}{v_{x}} = \frac{y-p_{y}}{v_{y}} = \vec{z-p_{z}}{v_{z}}
\end{equation}
However, if $v_{z} = 0$, then the equation below holds instead:
\begin{equation*}
  \begin{aligned}
    t &= \frac{x-p_{x}}{v_{x}} = \frac{y-p_{y}}{v_{y}} \\
    z &= v_{z} \\
\end{aligned}
\end{equation*}

\begin{example}[Lecture 4]{Convert Vector to Parametric Equation}
  Given the system of equations below, convert it to a parametric equation for a line?
  \begin{equation*}
    \begin{cases}
      x(t) &= 1 + 2t \\
      y(t) &= 2 + 2t \\
      z(t) &= 1 + -1t \\
    \end{cases}
  \end{equation*}
  \tcblower{}
  Here, they ask us to convert this system of equations to a vectorized and parameterized equation for a line, \Cref{eq:Parametric_Line}.
  From just inspection, we can see the obvious solution here. \\
  First, the constants form the vector $\vec{p}$.
  \begin{equation*}
    \vec{p} = \langle 1, 2, 1 \rangle
  \end{equation*}

  Next, the other terms form the scaled vector $t\vec{v}$.
  \begin{equation*}
    t\vec{v} = \langle 2t, 2t, -1t \rangle = t \langle 2, 2, -1 \rangle
  \end{equation*}

  Thus, the parameterized equation for this line is $\vec{r}(t) = \langle 1, 2, 1 \rangle + t \langle 2, 2, -1 \rangle$.
\end{example}

\subsubsection{Parametric Planes}\label{subsubsec:Parametric_Planes}
The \nameref{def:Parametric} equation for a plane (\Cref{eq:Parametric_Plane}).
\begin{equation}\label{eq:Parametric_Plane}
  \vec{r}(s, t) = t\vec{v} + s\vec{u} + \vec{p}
\end{equation}

Here, we have defined two vectors, $\vec{v}$ and $\vec{u}$ that represent the 2 directions that the plane must be built out of.
In addition, the scalar values $t$ and $s$ scale the vectors to be of any size within the three dimentionsal space.
Lastly, $\vec{p}$ is the location where the plane is when both $\vec{v} = \vec{u} = \vec{0}$.

Because we have two vectors, $\vec{v}$ and $\vec{u}$, you can define a new, third vector called the \nameref{def:Normal_Vector}.

\begin{definition}[Normal Vector]\label{def:Normal_Vector}
  A \emph{normal vector} is a vector that is normal (orthogonal) to the surface it was created from.
  If we work with \Cref{eq:Parametric_Plane}, then the normal vector can be defined as shown below:
  \begin{equation}\label{eq:Normal_Vector}
    \vec{n} = \vec{u} \cross \vec{v}
  \end{equation}

  \begin{remark}[Component Vectors Dot Product Zero]\label{rmk:Component_Vectors_Dot_Product}
    If you take the \nameref{def:Normal_Vector} and dot it with any point on the \textbf{plane}, then the resulting value will be zero.
    This is a logical extension to \Cref{prop:Dot_Product_Magnitude_Angle}, as the normal vector $\vec{n}$ has a \SI{90}{\degree} angle to both component vectors, making $\cos(\SI{90}{\degree}) = 0$.
  \end{remark}
\end{definition}

If we wanted to convert the equation for a parametric plane back to a Cartesian form, then we can refer to the equations below:
\begin{subequations}\label{eq:Parametric_Cartesian_Plane}
  \begin{equation}\label{eq:Parametric_Cartesian_Plane_Equal_0}
    n_{x}(x-p_{x}) + n_{y}(y-p_{y}) + n_{z}(z-p_{z}) = 0
  \end{equation}
  \begin{description}[noitemsep]
  \item $n_{x}, n_{y}, n_{z}$ are the components of the plane's \nameref{def:Normal_Vector}.
  \item $x, y, z$ are the variables used in a three dimensional equation.
  \item $p_{x}, p_{y}, p_{z}$ are the components of the planes $\vec{p}$.
  \end{description}
  \begin{equation}\label{eq:Parametric_Cartesian_Plane_Equal_D}
    n_{x}x + n_{y}y + n_{z}z = D
  \end{equation}
  \begin{description}[noitemsep]
  \item $D$ is three dimensional distance, as shown in \Cref{eq:3D_Distance}.
  \item $n_{x}, n_{y}, n_{z}$ are the components of the plane's \nameref{def:Normal_Vector}.
  \item $x, y, z$ are the variables used in a three dimensional equation.
  \end{description}
\end{subequations}


%%% Local Variables:
%%% mode: latex
%%% TeX-master: "../../Math_251-Multivariate_Vector_Calc-Reference_Material"
%%% End:


%%% Local Variables:
%%% mode: latex
%%% TeX-master: "../Math_251-Multivariate_Vector_Calc-Reference_Material"
%%% End:
