\section{Introduction}\label{sec:Intro}
\subsection{Multiple Dimensions}\label{subsec:Multiple_Dimensions}
Throughout this course, we will be working with multidimensional objects.
In this case, we will be drawing each of the terms of the multidimensional ``thing'' whatever it may be from the set of all real numbers.

In real-world terms, a dimension is just a way to measure something.
So, the outer shape of your calculator can be described by a set of three dimensional equations.

This means that the set of numbers $\RealNumbers^{2}$ is the traditional $xy$-plane for graphing one-dimensional equations.
The set of numbers $\RealNumbers^{3}$ is the traditional $xyz$ 3-dimensional graph.

\begin{remark*}
  Remember that there are \textbf{no} such things as quadrants here.
  Instead, the only way we can describe the traditional 2-D graph is by using a plane.
  Also, the individual spaces that can be identified by which direction of the origin they are lying in is called an octant.
\end{remark*}

If you take the traditional Euclidean equation for a circle, shown below, and expand it to three dimensions, you end up with a cylinder.
\begin{equation*}
  x^{2} + y^{2} = 1
\end{equation*}
This is because the $z$ term/dimension is \textbf{not} present in the equation, so $z$ is able to vary through all possible values, so long as the other constraints are satisifed.

\subsubsection{Distance in Three Dimensions}\label{subsubsec:3D_Distance}
The distance of a point any other point in three dimensions is found by a general extension to the Pythagoran Theorem.
The actual equation is shown in \Cref{eq:3D_Distance}
\begin{equation}\label{eq:3D_Distance}
  D = \sqrt{{(x_{2}-x_{1})}^{2} + {(y_{2}-y_{1})}^{2} + {(z_{2}-z_{2})}^{2}}
\end{equation}

\subsection{Vectors}
\begin{definition}[Vector]\label{def:Vector}
  A \emph{vector} in mathematics behaves identically to the way it behaves in science and engineering.
  However, we have a
\end{definition}
\subsection{Vectors}\label{subsec:Vectors}
\begin{definition}[Vector]\label{def:Vector}
  A \emph{vector} in mathematics behaves identically to the way it behaves in science and engineering, i.e.\ a vector is a quantity with \textbf{both} a direction and a magnitude.

  However, we have a slightly more mathematical definition we can leverage as well, given below.
  A vector is an ordered list of $n$ numbers, typically denoted $\langle x_{1}, x_{2}, \ldots, x_{n} \rangle$.
  \begin{equation}\label{eq:Vector}
    v \in \RealNumbers^{n} = \langle x_{1}, x_{2}, \ldots, x_{n} \rangle
  \end{equation}
\end{definition}

There are several vectors that are special:
\begin{enumerate}[noitemsep]
\item $\vec{0}$, the \nameref{def:Zero_Vector}.
\item $\ihat$, one of the \nameref{def:Unit_Vector}s, with value $\langle 1, 0, 0 \rangle$.
\item $\jhat$, one of the \nameref{def:Unit_Vector}s, with value $\langle 0, 1, 0 \rangle$.
\item $\khat$, one of the \nameref{def:Unit_Vector}s, with value $\langle 0, 0, 1 \rangle$.
\end{enumerate}

\begin{definition}[Zero Vector]\label{def:Zero_Vector}
  The \emph{zero vector} is a special vector that has a magnitude of zero and is directionless.
  It is the only vector that has both of these characteristics.
  The zero vector is defined as:
  \begin{equation}\label{eq:Zero_Vector}
    \vec{0} = \langle 0, 0, 0 \rangle
  \end{equation}
\end{definition}

\begin{definition}[Unit Vector]\label{def:Unit_Vector}
  A \emph{unit vector} is a vector with a magnitude of $1$, and has a direction along one of the fundamental axes.
  These have various definitions, depending on the number of dimensions involved in the equation.
  However, the three unit vectors that are useful for this course are in \Crefrange{subeq:i_Unit_Vector}{subeq:k_Unit_Vector}.
  \begin{subequations}\label{eq:Unit_Vector}
    \begin{equation}\label{subeq:i_Unit_Vector}
      \ihat = \langle 1, 0, 0 \rangle
    \end{equation}
    \begin{equation}\label{subeq:j_Unit_Vector}
      \jhat = \langle 0, 1, 0 \rangle
    \end{equation}
    \begin{equation}\label{subeq:k_Unit_Vector}
      \khat = \langle 0, 0, 1 \rangle
    \end{equation}
  \end{subequations}
\end{definition}


%%% Local Variables:
%%% mode: latex
%%% TeX-master: "../../Math_251-Multivariate_Vector_Calc-Reference_Material."
%%% End:



%%% Local Variables:
%%% mode: latex
%%% TeX-master: "../Math_251-Multivariate_Vector_Calc-Reference_Material"
%%% End:
