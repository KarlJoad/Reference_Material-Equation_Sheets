\subsection{Cartesian and Parametric Equations}\label{subsec:Cartesian_Parametric}
\begin{definition}[Cartesian]\label{def:Cartesian}
  A \emph{Cartesian} equation is the one that most people are used to.
  For example, for a line:
  \begin{equation}\label{eq:Generic_Cartesian_Line}
    y(x) = mx + b
  \end{equation}

  In a Cartesian equation, there are the same number of variables as dimensions in the problem.
\end{definition}

\begin{definition}[Parametric]\label{def:Parametric}
  A \emph{parametric} equation is actually a set of equations that together form the resulting vector.
  For the same line as in \Cref{eq:Generic_Cartesian_Line}, we have a parametric equation as shown below.
  \begin{equation}\label{eq:Generic_Parametric_Line}
    \langle x(t), y(t) \rangle = \langle t, mt + b \rangle
  \end{equation}

  In a parametric equation, there are the same number of parameterized equations as dimensions.
\end{definition}


%%% Local Variables:
%%% mode: latex
%%% TeX-master: "../../Math_251-Multivariate_Vector_Calc-Reference_Material"
%%% End:
