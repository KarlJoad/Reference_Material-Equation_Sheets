\subsection{Dot Product}\label{subsec:Vector_Dot_Product}
\begin{definition}[Dot Product]\label{def:Dot_Product}
  The \emph{dot product} is a way to define the multiplication of two vectors that results in a single scalar value.
  The dot product is defined as shown in \Cref{eq:Dot_Product}.
  \begin{equation}\label{eq:Dot_Product}
    \begin{aligned}
      \langle a, b, c \rangle \dot \langle x, y, z \rangle &= ax + by + cz \\
      &= \magnitude{\langle a, b, c \rangle} \magnitude{\langle x, y, z \rangle} \cos(\theta), \text{where} \:\: 0 \leq \theta \leq \pi \\
    \end{aligned}
  \end{equation}
\end{definition}

\subsubsection{Properties of the Dot Product}\label{subsubsec:Dot_Product_Properties}
\begin{propertylist}
\item Another way to define the cross product.\label{prop:Dot_Product_Magnitude_Angle}
  \begin{equation*}
    \vec{u} \cdot \vec{v} = \magnitude{\vec{u}} \magnitude{\vec{v}} \cos(\theta)
  \end{equation*}
\item Dotting a vector with itself.
  \begin{equation*}
    \vec{u} \cdot \vec{u} = \magnitude{\vec{u}}^{2} \cos(0)
  \end{equation*}
\item Distributivity and Associativity of the \nameref{def:Dot_Product}.
  \begin{equation*}
    \begin{aligned}
      (\vec{u} + \vec{v}) \cdot \vec{w} &\neq \vec{u} + (\vec{v} \cdot \vec{w}) \\
      &= (\vec{u} \cdot \vec{w}) + (\vec{v} \cdot \vec{w}) \\
    \end{aligned}
  \end{equation*}
\item Associativity of scalar multiplication with the \nameref{def:Dot_Product}.
  \begin{equation*}
    (c \vec{u}) \cdot \vec{v} = c(\vec{u} \cdot \vec{v})
  \end{equation*}
\item Taking the \nameref{def:Dot_Product} with a vector and the \nameref{def:Zero_Vector}.
  \begin{equation*}
    \vec{0} \cdot \vec{u} = 0
  \end{equation*}
\end{propertylist}

One interesting thing to note about the dot product is that if $\vec{u} \cdot \vec{v} = 0$, then the vectors are orthogonal, $\vec{u} \perp \vec{v}$.


%%% Local Variables:
%%% mode: latex
%%% TeX-master: "../../Math_251-Multivariate_Vector_Calc-Reference_Material"
%%% End:
