\subsection{Cross Product}\label{subsec:Cross_Product}
\begin{definition}[Cross Product]\label{def:Cross_Product}
  The \emph{cross product} is a way to find a third, resulting, vector that is orthogonal to the two input vectors.
  To find the value of the cross product, use the equations below:
  \begin{equation}\label{eq:Cross_Product}
    \begin{aligned}
      \langle a, b, c \rangle \cross \langle x, y, z \rangle &=
      \begin{vmatrix}
        \ihat & \jhat & \khat \\
        a & b & c \\
        x & y & z
      \end{vmatrix} \\
      &=
      \ihat \begin{vmatrix}
        b & c \\
        e & f
      \end{vmatrix} -
      \jhat \begin{vmatrix}
        a & c \\
        d & f
      \end{vmatrix} +
      \khat \begin{vmatrix}
        a & b \\
        d & e
      \end{vmatrix} \\
      &= (bf - ce)\ihat - (af-cd)\jhat + (ae-bd)\khat \\
      &= \langle vf-ce, af-cd, ae-bd \rangle \\
    \end{aligned}
  \end{equation}

  The value of the cross product can also be found by \Cref{eq:Cross_Product_Magnitude}.
  \begin{equation}\label{eq:Cross_Product_Magnitude}
    \vec{u} \cross \vec{v} = \magnitude{\vec{u}} \magnitude{\vec{v}} \sin(\theta)
  \end{equation}

  \begin{remark}[Cross Product Orthogonality]\label{rmk:Cross_Product_Orthogonality}
    If $\vec{u} \cross \vec{v} = \vec{0}$, then the orthogonal \nameref{def:Vector} is parallel to some other thing.
  \end{remark}
\end{definition}

\begin{example}[Lecture 4]{Take Cross Product}
  Given the vectors $\vec{u} = \langle 3, 4, 1 \rangle$ and $\vec{v} = \langle 1, -2, -1 \rangle$, find their value when they are taken as a cross product.
  \tcblower{}
  The first step is rewrite the \nameref{def:Cross_Product} equation into the typical matrix form.
  \begin{equation*}
    \langle 3, 4, 1 \rangle \cross \langle 1, -2, -1 \rangle =
    \begin{vmatrix}
      \ihat & \jhat & \khat \\
      3 & 4 & 1 \\
      1 & -2 & -1
    \end{vmatrix}
  \end{equation*}

  Now, we take the determinant of the $3 \times 3$ matrix, and simplify that.
  \begin{align*}
    \langle 3, 4, 1 \rangle \cross \langle 1, -2, -1 \rangle &= (-4 - (-2))\ihat - (-3 - 1)\jhat + (-6 - 4) \khat \\
                                                             &= (-2)\ihat - (-4)\jhat + (-10)\khat \\
                                                             &= \langle -2, 4, -10 \rangle
  \end{align*}

  Thus, our solution is $\langle -2, 4, -10 \rangle$.
\end{example}


%%% Local Variables:
%%% mode: latex
%%% TeX-master: "../../Math_251-Multivariate_Vector_Calc-Reference_Material"
%%% End:
