\subsection{Vectors}\label{subsec:Vectors}
\begin{definition}[Vector]\label{def:Vector}
  A \emph{vector} in mathematics behaves identically to the way it behaves in science and engineering, i.e.\ a vector is a quantity with \textbf{both} a direction and a magnitude.

  However, we have a slightly more mathematical definition we can leverage as well, given below.
  A vector is an ordered list of $n$ numbers, typically denoted $\langle x_{1}, x_{2}, \ldots, x_{n} \rangle$.
  \begin{equation}\label{eq:Vector}
    v \in \RealNumbers^{n} = \langle x_{1}, x_{2}, \ldots, x_{n} \rangle
  \end{equation}
\end{definition}

There are several vectors that are special:
\begin{enumerate}[noitemsep]
\item $\vec{0}$, the \nameref{def:Zero_Vector}.
\item $\ihat$, one of the \nameref{def:Unit_Vector}s, with value $\langle 1, 0, 0 \rangle$.
\item $\jhat$, one of the \nameref{def:Unit_Vector}s, with value $\langle 0, 1, 0 \rangle$.
\item $\khat$, one of the \nameref{def:Unit_Vector}s, with value $\langle 0, 0, 1 \rangle$.
\end{enumerate}

\begin{definition}[Zero Vector]\label{def:Zero_Vector}
  The \emph{zero vector} is a special vector that has a magnitude of zero and is directionless.
  It is the only vector that has both of these characteristics.
  The zero vector is defined as:
  \begin{equation}\label{eq:Zero_Vector}
    \vec{0} = \langle 0, 0, 0 \rangle
  \end{equation}
\end{definition}

\begin{definition}[Unit Vector]\label{def:Unit_Vector}
  A \emph{unit vector} is a vector with a magnitude of $1$, and has a direction along one of the fundamental axes.
  These have various definitions, depending on the number of dimensions involved in the equation.
  However, the three unit vectors that are useful for this course are in \Crefrange{subeq:i_Unit_Vector}{subeq:k_Unit_Vector}.
  \begin{subequations}\label{eq:Unit_Vector}
    \begin{equation}\label{subeq:i_Unit_Vector}
      \ihat = \langle 1, 0, 0 \rangle
    \end{equation}
    \begin{equation}\label{subeq:j_Unit_Vector}
      \jhat = \langle 0, 1, 0 \rangle
    \end{equation}
    \begin{equation}\label{subeq:k_Unit_Vector}
      \khat = \langle 0, 0, 1 \rangle
    \end{equation}
  \end{subequations}
\end{definition}

\subsubsection{Adding Vectors}\label{subsubsec:Adding_Vectors}
\nameref{def:Vector}s are added and subtraced by performing the arithmetic operation ``from tip to tail''.
With numbers and letters, this means that each of the individual terms add together.
For example,
\begin{equation*}
  \begin{aligned}
    \langle -1, 2 \rangle + \langle 3, 4 \rangle &= \langle -1+3, 2+4 \rangle \\
    &= \langle 2, 6 \rangle \\
  \end{aligned}
\end{equation*}

Now, according to the associativity of addition, \Cref{eq:Vector_Associativity} holds true.
\begin{equation}\label{eq:Vector_Associativity}
  \vec{u} + \vec{v} = \vec{v} + \vec{u}
\end{equation}


%%% Local Variables:
%%% mode: latex
%%% TeX-master: "../../Math_251-Multivariate_Vector_Calc-Reference_Material."
%%% End:
