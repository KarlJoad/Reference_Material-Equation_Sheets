\subsection{Syscall Context}\label{subsec:Syscall_Context}
The \nameref{def:Kernel} is in \nameref{def:Process}-context during the execution of a \nameref{def:System_Call}.
The current pointer points to the current task, which is the \nameref{def:Process} that issued the \nameref{def:System_Call}.

In \nameref{def:Process}-context, the \nameref{def:Kernel} is capable of sleeping (for example, if the system call blocks on a call or explicitly calls \kernelinline{schedule()}) and is fully preemptible.
The capability to sleep means that system calls can make use of the majority of the \nameref{def:Kernel}’s functionality to simplify its own programming.
The fact that \nameref{def:Process} context is preemptible implies that, like \nameref{def:User}-space, the current task may be preempted by another task.
Because the new task may then execute the same \nameref{def:System_Call}, care must be exercised to ensure that the calls are reentrant.
Of course, this is the same concern that symmetrical multiprocessing introduces.

When the \nameref{def:System_Call} returns, control continues in \kernelinline{system_call()}, which ultimately switches to \nameref{def:User}-space and continues the execution of the \nameref{def:User} process.

%%% Local Variables:
%%% mode: latex
%%% TeX-master: "../../EDAF35-Operating_Systems-Reference_Sheet"
%%% End:
