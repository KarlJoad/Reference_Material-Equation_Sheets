\subsection{Device Manipulation}\label{subsec:Device_Manipulation}
\begin{definition}[Device]\label{def:Device}
  A \emph{device} in an \nameref{def:Operating_System} is a resource that must be controlled.
  Some of these devices are physical devices (for example, disk drives), while others can be thought of as abstract or virtual devices (for example, files).
\end{definition}

A system with multiple \nameref{def:User}s may require us to first \mintinline{c}{request()} a device, to ensure exclusive use of it.
After we are finished with the device, we \mintinline{c}{release()} it.
These functions are similar to the \mintinline{c}{open()} and \mintinline{c}{close()} \nameref{def:System_Call}s for files.
Other \nameref{def:Operating_System}s allow unmanaged access to devices.
The hazard then is the potential for device contention and perhaps \nameref{def:Deadlock}.

Once the device has been requested (and allocated to us), we can \mintinline{c}{read()}, \mintinline{c}{write()}, just as we can with files.
In fact, the similarity between I/O devices and files is so great that many \nameref{def:Operating_System}s, including UNIX, merge the two into a combined file–device structure.
In this case, a set of \nameref{def:System_Call}s can be shared between both files and \nameref{def:Device}s.
Sometimes, I/O devices are identified by special file names, directory placement, or file attributes.

%%% Local Variables:
%%% mode: latex
%%% TeX-master: "../../EDAF35-Operating_Systems-Reference_Sheet"
%%% End:
