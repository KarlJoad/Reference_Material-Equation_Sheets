\subsection{Syscall Implementation}\label{subsec:Syscall_Implementation}
The actual implementation of a system call in Linux does not need to be concerned with the behavior of the system call handler.
The hard work lies in designing and implementing the system call; registering it with the kernel is simple.

\subsubsection{Implementing Syscalls}\label{subsubsec:Implementing_Syscalls}
The first step in implementing a system call is defining its purpose.
\begin{itemize}[noitemsep]
\item What will it do?
  \begin{itemize}[noitemsep]
  \item The syscall should have exactly one purpose.
  \item Multiplexing syscalls (a single system call that does wildly different things depending on a flag argument) is discouraged in Linux.
  \end{itemize}

\item What are the new system call’s arguments, return value, and error codes?
  \begin{itemize}[noitemsep]
  \item The system call should have a clean and simple interface with the smallest number of arguments possible.
  \item The semantics and behavior of a system call are important; they must not change, because existing applications will come to rely on them.
  \end{itemize}

\item Be forward thinking; consider how the function might change over time.
\item Can new functionality be added to your system call or will any change require an entirely new function?
\item Can you easily fix bugs without breaking backward compatibility?
\item Will you need to add a flag to address forward compatibility?
  \begin{itemize}[noitemsep]
  \item Many system calls provide a flag argument to address forward compatibility.
  \item The flag is not used to multiplex different behavior across a single system call—as mentioned, but to \textbf{enable new} functionality and options without breaking backward compatibility or needing to add a new system call.
  \end{itemize}

\item Design the system call to be as general as possible.
  \begin{itemize}[noitemsep]
  \item Do not assume its use today will be the same as its use tomorrow.
  \item The purpose of the system call will remain constant but its uses may change.
  \end{itemize}

\item Is the system call portable?
  \begin{itemize}[noitemsep]
  \item Do not make assumptions about an architecture’s word size or endianness.
  \end{itemize}
\end{itemize}

\subsubsection{Parameter Verification}\label{subsubsec:Parameter_Verification}
System calls must carefully verify all their parameters to ensure that they are valid and legal.
The system call runs in kernel-space, and if the user can pass invalid input into the kernel without restraint, the system’s security and stability can suffer.

For example, file I/O syscalls must check whether the file descriptor is valid.
Process-related functions must check whether the provided PID is valid.
Every parameter must be checked to ensure it is not just valid and legal, but correct.
Processes must not ask the kernel to access resources to which the process does not have access.

One of the most important checks is the validity of any pointers that the user provides.
Imagine if a process could pass any pointer into the kernel, unchecked, even passing a pointer to which the kernel-calling process did not have read access!
Processes could then trick the kernel into copying data for which they did not have access permission, such as data belonging to another process or data mapped unreadable.
Before following a pointer into user-space, the system must ensure that:
\begin{itemize}[noitemsep]
\item The pointer points to a region of memory in user-space.
  Processes must not be able to trick the kernel into reading data in kernel-space on their behalf.
\item The pointer points to a region of memory in the process’s address space.
  The process must not be able to trick the kernel into reading someone else’s data.
\item The process must not be able to bypass memory access restrictions.
  If reading, the memory is marked readable.
  If writing, the memory is marked writable.
  If executing, the memory is marked executable.
\end{itemize}

\begin{center}
  \Large{\textbf{Kernel code must never blindly follow a pointer into user-space!}}
\end{center}

The kernel provides two methods for performing the requisite checks and the desired copy to and from user-space.
One of these two methods must always be used.
\begin{enumerate}[noitemsep]
\item For writing \textbf{to} user-space, the function \kernelinline{copy_to_user()} is provided.
  It takes three parameters.
  \begin{enumerate}[noitemsep]
  \item The first is a pointer to the destination memory address in the process’s address space.
  \item The second is a pointer to the source pointer in kernel-space.
  \item The third is the size, in bytes, of the data to copy.
\end{enumerate}

\item For reading \textbf{from} user-space, the method \kernelinline{copy_from_user()} is analogous to \kernelinline{copy_to_user()}.
  It also takes 3 parameters.
  \begin{enumerate}[noitemsep]
  \item The first is a pointer to the destination memory address in \nameref{def:Kernel}-space.
  \item The second is a pointer to the source memory address in the \nameref{def:Process}'s address space.
  \item The third is the size, in bytes, of the data to read.
\end{enumerate}

\end{enumerate}

Both of these functions return the number of bytes they failed to copy on error.
On success, they return zero.
It is standard for the syscall to return \kernelinline{-EFAULT} in the case of such an error.

Both \kernelinline{copy_to_user()} and \kernelinline{copy_from_user()} may block.
This occurs if the page containing the user data is not in physical memory but is swapped to disk.
In that case, the process sleeps until the page fault handler can bring the page from the swap file on disk into physical memory.

A final possible check is for valid permission.
In older versions of Linux, it only \texttt{root} could perform these actions.
Now, a finer-grained ``capabilities'' system is in place.

The new system enables specific access checks on specific resources.
A call to \kernelinline{capable()} with a valid capabilities flag returns nonzero if the caller holds the specified capability and zero otherwise.
See \kernelinline{<linux/capability.h>} for a list of all capabilities and what rights they entail.
For example, \kernelinline{capable(CAP_SYS_NICE)} checks whether the \textbf{caller} has the ability to modify nice values of other processes.

\begin{blackbox}
  \large{\textbf{By default, the superuser possesses all capabilities and nonroot possesses none.}}
\end{blackbox}

\subsubsection{Final Steps in Binding a Syscall}\label{subsec:Final_Steps_Binding_Syscall}
After the \nameref{def:System_Call} is written, it is trivial to register it as an official \nameref{def:System_Call}:
\begin{enumerate}[noitemsep]
\item 1. Add an entry to the end of the system call table.
  This needs to be done for each architecture that supports the \nameref{def:System_Call} (which, for most calls, is all the architectures).
  The position of the syscall in the table, starting at zero, is its \nameref{def:Syscall_Number}.
  For example, the tenth entry in the list is assigned syscall number nine.
\item For each supported architecture, define the syscall number in \kernelinline{<asm/unistd.h>}.
\item Compile the syscall into the \nameref{def:Kernel} image (as opposed to compiling as a module).
  This can be as simple as putting the \nameref{def:System_Call} in a relevant file in \kernelinline{kernel/}.
\end{enumerate}

Look at these steps in more detail with a fictional \nameref{def:System_Call}, \kernelinline{foo()}.
First, we want to append \kernelinline{sys_foo()} to the system call table, and record its \nameref{def:Syscall_Number}.
This number is the zero-indexed location of the new \kernelinline{sys_foo()} function.
For most architectures, the table is located in \texttt{entry.S}.
Although it is not explicitly specified, the \nameref{def:System_Call} is implicitly given the next subsequent syscall number.

For each architecture you want to support, the \nameref{def:System_Call} must be added to the architecture’s system call table.
Usually you would want to make the \nameref{def:System_Call} available to each architecture, so it must be placed in each architecture's system call table.
The \nameref{def:System_Call} does not need to receive the same syscall number under each architecture.
Then, the \nameref{def:Syscall_Number} is added to \kernelinline{<asm/unistd.h>}.

Because the \nameref{def:System_Call} must be compiled into the core \nameref{def:Kernel} image in all configurations, so the \kernelinline{sys_foo()} function must be placed somewhere in \kernelinline{kernel/*.c}.
You should put it wherever the function is most relevant.
For example, if the function is related to scheduling, you could define it in \kernelinline{kernel/sched.c}.

\subsubsection{Why \textbf{NOT} Implement a Syscall}\label{subsubsec:Not_Implement_Syscall}
The previous sections have shown that it is easy to implement a new \nameref{def:System_Call}, but that in no way should encourage you to do so.
Often, much more viable alternatives to providing a new \nameref{def:System_Call} are available.

Let’s look at the pros, cons, and alternatives.
The pros of implementing a new interface as a syscall are:
\begin{itemize}[noitemsep]
\item They are simple to implement and easy to use.
\item Their performance on Linux is fast.
\end{itemize}

The cons:
\begin{itemize}[noitemsep]
\item You need a syscall number, which needs to be officially assigned to you.
\item After the \nameref{def:System_Call} is in a stable series \nameref{def:Kernel}, it is written in stone.
  The interface cannot change without breaking user-space applications, which Linux explicitly disallows.
\item Each architecture needs to separately register the \nameref{def:System_Call} and support it.
\item \nameref{def:System_Call}s are not easily used from scripts and cannot be accessed directly from the filesystem.
\item Because you need an assigned syscall number, it is hard to maintain and use a \nameref{def:System_Call} outside of the master \nameref{def:Kernel} tree.
\item For simple exchanges of information, a \nameref{def:System_Call} is overkill.
\end{itemize}

The alternatives:
\begin{itemize}[noitemsep]
\item Implement a device node and \kernelinline{read()} and \kernelinline{write()} to it.
  Use \kernelinline{ioctl()} to manipulate specific settings or retrieve specific information.
\item Certain interfaces, such as semaphores, can be represented as file descriptors and manipulated as such.
\item Add the information as a file to the appropriate location in \texttt{sysfs}.
\end{itemize}

%%% Local Variables:
%%% mode: latex
%%% TeX-master: "../../EDAF35-Operating_Systems-Reference_Sheet"
%%% End:
