\subsection{Communications}\label{subsec:Communications}
Both of the models discussed are common in \nameref{def:Operating_System}s, and most systems implement both.
\nameref{subsubsec:Message_Passing} is useful for exchanging smaller amounts of data, because no conflicts need to be avoided.
It is also easier to implement than is shared memory for intercomputer communication.
\nameref{subsubsec:Shared_Memory} allows maximum speed and convenience of communication, since it can be done at memory transfer speeds when it takes place within a computer.
Problems exist, however, in the areas of protection and synchronization between the \nameref{def:Process}es sharing memory.

\subsubsection{Message Passing}\label{subsubsec:Message_Passing}
Messages can be exchanged between the \nameref{def:Process}es either directly or indirectly through a common mailbox.
Before communication can take place, a connection must be opened.
The name of the other communicator must be known.
Each \nameref{def:Process} has a \emph{process name}, and this name is translated into an identifier, PID, by which the \nameref{def:Operating_System} can refer to the \nameref{def:Process}.
The \mintinline{c}{get_processid()} \nameref{def:System_Call} does this translation.
The identifiers are then passed to general-purpose \mintinline{c}{open()} and \mintinline{c}{close()} calls provided by the file system or to specific \mintinline{c}{open_connection()} and \mintinline{c}{close_connection()} \nameref{def:System_Call}s, depending on the model of communication.
The recipient \nameref{def:Process} usually must give its permission for communication to take place with an \mintinline{c}{accept_connection()} call.

Most \nameref{def:Process}es that will be receiving connections are special-purpose \nameref{def:Daemon}s.
They execute a \mintinline{c}{wait_for_connection()} call and are awakened when a connection is made.
The source of the communication, known as the client, and the receiving \nameref{def:Daemon}, known as a server, then exchange messages by using \mintinline{c}{read_message()} and \mintinline{c}{write_message()} \nameref{def:System_Call}s.
The \mintinline{c}{close_connection()} call terminates the communication.

\subsubsection{Shared-Memory}\label{subsubsec:Shared_Memory}
In the shared-memory model, \mintinline{c}{shared_memory_create()} and \mintinline{c}{shared_memory_attach()} \nameref{def:System_Call}s are used by \nameref{def:Process}es to create and gain access to regions of memory owned by other \nameref{def:Process}es.
The \nameref{def:Operating_System} tries to prevent one \nameref{def:Process} from accessing another \nameref{def:Process}’s memory, so shared memory requires that two or more \nameref{def:Process}es agree to remove this restriction.
They can then exchange information by reading and writing data in the shared areas.
The form of the data is determined by the \nameref{def:Process}es and is not under the \nameref{def:Operating_System}’s control.
The \nameref{def:Process}es are also responsible for ensuring that they are not writing to the same location simultaneously.

%%% Local Variables:
%%% mode: latex
%%% TeX-master: "../../EDAF35-Operating_Systems-Reference_Sheet"
%%% End:
