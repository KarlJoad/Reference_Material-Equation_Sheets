\subsection{Syscall Handler}\label{subsec:Syscall_Handler}
It is not possible for \nameref{def:User}-space applications to simply execute a \nameref{def:Kernel}-function call to a function existing in \nameref{def:Kernel}-space because the \nameref{def:Kernel} exists in a protected memory space.
If applications could directly read and write to the \nameref{def:Kernel}’s address space, system security and stability would be nonexistent.

Instead, \nameref{def:User}-space applications must somehow signal to the \nameref{def:Kernel} that they want to execute a \nameref{def:System_Call} and have the system switch to \nameref{def:Kernel} mode, where the \nameref{def:System_Call} can be executed in \nameref{def:Kernel}-space by the \nameref{def:Kernel} on behalf of the application.
The mechanism to signal the \nameref{def:Kernel} is a \nameref{def:Trap}, a software interrupt.
Incur a \nameref{def:Trap}, and the system will switch to \nameref{def:Kernel}-mode and execute the exception handler.
However, in this case, the exception handler is actually the system call handler

\begin{center}
\large{\textbf{The important thing to note is that, somehow, \nameref{def:User}-space causes an exception or trap to enter the \nameref{def:Kernel}.}}
\end{center}

\subsubsection{Denoting Correct Syscall}\label{subsubsec:Denote_Correct_Syscall}
Simply entering \nameref{def:Kernel}-space alone is not sufficient because multiple \nameref{def:System_Call}s exist, all of which enter the \nameref{def:Kernel} in the same manner.
Thus, the \nameref{def:Syscall_Number} must be passed into the \nameref{def:Kernel}, usually through a \nameref{def:Register}.

\subsubsection{Parameter Passing}\label{subsubsec:Parameter_Passing}
In addition to the \nameref{def:Syscall_Number}, most \nameref{def:System_Call}s require that one or more parameters be passed to them.
Somehow, \nameref{def:User}-space must relay the parameters to the \nameref{def:Kernel} during the trap.
There are 2 main ways to do this.
\begin{enumerate}[noitemsep]
\item The easiest way is to store the parameters in registers
\item If there are not enough registers, or the parameter will not fit in a single register, one is filled with a  pointer to \nameref{def:User}-space memory where all the parameters are stored.
\end{enumerate}

The return value is sent back to \nameref{def:User}-space also by a register.

%%% Local Variables:
%%% mode: latex
%%% TeX-master: "../../EDAF35-Operating_Systems-Reference_Sheet"
%%% End:
