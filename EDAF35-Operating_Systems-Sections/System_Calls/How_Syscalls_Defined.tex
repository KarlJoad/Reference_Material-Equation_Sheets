\subsection{How are Syscalls Defined?}\label{subsec:How_Syscalls_Defined}
In this section, we will be analyzing the \mintinline{c}{getpid()} \nameref{def:System_Call}.
It is defined to return an \textbf{integer} (to the \nameref{def:User}-space) that represents the current \nameref{def:Process}'s PID.\@
The implementation of \mintinline{c}{getpid()} is shown below.
\inputminted[frame=lines,linenos]{c}{./EDAF35-Operating_Systems-Sections/System_Calls/Code/getpid_Implementation.c}

\mintinline{c}{SYSCALL_DEFINE0} is a macro that defines a system call with no parameters (hence the 0).
The expanded code looks like this:
\inputminted[frame=lines,linenos]{c}{./EDAF35-Operating_Systems-Sections/System_Calls/Code/getpid_Expanded.c}

\nameref{def:System_Call}s have a strict definition.
\begin{enumerate}[noitemsep]
\item The asmlinkage modifier on the function definition is a directive to tell the compiler to look only on the stack for this function’s arguments.
  \textbf{This is a required modifier for all system calls.}
\item The function returns a \mintinline{c}{long}.
  For compatibility between 32- and 64-bit systems, system calls defined to return an \mintinline{c}{int} in \nameref{def:User}-space return a \mintinline{c}{long} in the \nameref{def:Kernel}.
\item The \mintinline{c}{getpid()} \nameref{def:System_Call} is defined as \mintinline{c}{sys_getpid()} in the \nameref{def:Kernel}.
  \begin{itemize}[noitemsep]
  \item This is the naming convention taken with all \nameref{def:System_Call}s in Linux.
  \item \nameref{def:System_Call} \mintinline{c}{bar()} is implemented in the \nameref{def:Kernel} as function \mintinline{c}{sys_bar()}.
  \end{itemize}
\end{enumerate}


%%% Local Variables:
%%% mode: latex
%%% TeX-master: "../../EDAF35-Operating_Systems-Reference_Sheet"
%%% End:
