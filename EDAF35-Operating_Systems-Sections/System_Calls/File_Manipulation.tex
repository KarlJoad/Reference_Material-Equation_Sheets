\subsection{File Manipulation}\label{subsec:File_Manipulation}
We first need to be able to \mintinline{c}{create()} and \mintinline{c}{delete()} files.
Either \nameref{def:System_Call} requires the name of the file and perhaps some of the file’s attributes.
Once the file is created, we need to \mintinline{c}{open()} it and to use it.
We may then \mintinline{c}{read()}, \mintinline{c}{write()}, or perform any other \nameref{def:API}-defined action(s).
Finally, we need to \mintinline{c}{close()} the file, indicating that we are no longer using it.

We may need these same sets of operations for directories if we have a directory structure for organizing files in the file system.
In addition, for either files or directories, we need to be able to determine the values of various attributes and perhaps to reset them if necessary.

\begin{definition}[File Attribute]\label{def:File_Attribute}
  A \emph{file attribute} contains metadata about the file.
  This includes the file's name, type, protection codes, accounting information, and so on.
\end{definition}

\begin{remark*}
  If the system programs are callable by other programs, then each can be considered an \nameref{def:API} by other system programs.
\end{remark*}

%%% Local Variables:
%%% mode: latex
%%% TeX-master: "../../EDAF35-Operating_Systems-Reference_Sheet"
%%% End:
