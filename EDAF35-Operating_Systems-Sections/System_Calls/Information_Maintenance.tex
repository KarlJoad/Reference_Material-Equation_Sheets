\subsection{Information Maintenance}\label{subsec:Information_Maintenance}
Many \nameref{def:System_Call}s exist simply for the purpose of transferring information between the \nameref{def:User} program and the \nameref{def:Operating_System}.
For example, most systems have a \nameref{def:System_Call} to return the current \mintinline{c}{time()} and \mintinline{c}{date()}.
Other \nameref{def:System_Call}s may return information about the system, such as the number of current \nameref{def:User}s, the version number of the \nameref{def:Operating_System}, the amount of free memory or disk space, and so on.

Another set of \nameref{def:System_Call}s is helpful in debugging a program.
Many systems provide \nameref{def:System_Call}s to \mintinline{c}{dump()} memory.
A program \texttt{trace} lists each \nameref{def:System_Call} as it is executed.
In addition, the \nameref{def:Operating_System} keeps information about all its \nameref{def:Process}es, and \nameref{def:System_Call}s are used to access this information.

%%% Local Variables:
%%% mode: latex
%%% TeX-master: "../../EDAF35-Operating_Systems-Reference_Sheet"
%%% End:
