\section{Virtual Machines}\label{sec:Virtual_Machines}
With a \nameref{def:Virtual_Machine}, guest \nameref{def:Operating_System}s and applications run on a system utilizing \nameref{def:Virtualization}.

\begin{definition}[Virtualization]\label{def:Virtualization}
  \emph{Virtualization} is the act of abstracting the \nameref{def:Hardware} of a single computer into several different execution environments.
  Each of these execution environments operates under the illusion that they are the only environment running on its own private computer.
  These ``computer'' for these environments appears to be native hardware and things in these environments behave as native hardware would but that also protects, manages, and limits them.
\end{definition}

This concept is similar to the \nameref{subsubsec:Layered_Approach} of \nameref{def:Operating_System} implementation (see \Cref{subsubsec:Layered_Approach}).
In the case of \nameref{def:Virtualization}, there is a layer that creates a virtual system on which operating systems or applications can run.


%%% Local Variables:
%%% mode: latex
%%% TeX-master: "../EDAF35-Operating_Systems-Reference_Sheet"
%%% End:
