\section{Virtual Machines}\label{sec:Virtual_Machines}
With a \nameref{def:Virtual_Machine}, guest \nameref{def:Operating_System}s and applications run on a system utilizing \nameref{def:Virtualization}.

\begin{definition}[Virtualization]\label{def:Virtualization}
  \emph{Virtualization} is the act of abstracting the \nameref{def:Hardware} of a single computer into several different execution environments.
  Each of these execution environments operates under the illusion that they are the only environment running on its own private computer.
  These ``computer'' for these environments appears to be native hardware and things in these environments behave as native hardware would but that also protects, manages, and limits them.
\end{definition}

This concept is similar to the \nameref{subsubsec:Layered_Approach} of \nameref{def:Operating_System} implementation (see \Cref{subsubsec:Layered_Approach}).
In the case of \nameref{def:Virtualization}, there is a layer that creates a virtual system on which operating systems or applications can run.

\begin{definition}[Virtual Machine]\label{def:Virtual_Machine}
  A \emph{Virtual Machine} (\emph{VM}, not to be confused with \nameref{def:Virtual_Memory}'s VM) is an \nameref{def:Operating_System} that is running inside an execution environment provided by \nameref{def:Virtualization}.
  The operating system running on this virtualized hardware believes it is running on real hardware, with all of its quirks and problems.
  This operating system also believes it is the \textbf{only} one running, when there could be several virtual machines running on the same hardware.
\end{definition}

\nameref{def:Virtual_Machine} implementations involve several components.
At the base is the \nameref{def:VM_Host}.

\begin{definition}[Host]\label{def:VM_Host}
  The \emph{host} is the the underlying physical \nameref{def:Hardware} system that runs the virtual machines.
\end{definition}

The \nameref{def:VM_Host} runs the \nameref{def:Virtual_Machine_Manager}.

\begin{definition}[Virtual Machine Manager]\label{def:Virtual_Machine_Manager}\label{def:Hypervisor}\label{def:VMM}
  The \emph{Virtual Machine Manager} (\emph{VMM}), also known as a \emph{hypervisor} creates and runs virtual machines by providing an interface that is identical to the host to every guest..
  \begin{remark}
    This is true except in the case of \nameref{subsubsec:Paravirtualization}.
  \end{remark}
\end{definition}

Each guest process is provided with a virtual copy of the host.
Usually, the guest process is an \nameref{def:Operating_System}.
A single physical machine can run multiple operating systems concurrently, each in its own virtual machine.


%%% Local Variables:
%%% mode: latex
%%% TeX-master: "../EDAF35-Operating_Systems-Reference_Sheet"
%%% End:
