\section{Virtual Machines}\label{sec:Virtual_Machines}
With a \nameref{def:Virtual_Machine}, guest \nameref{def:Operating_System}s and applications run on a system utilizing \nameref{def:Virtualization}.

\begin{definition}[Virtualization]\label{def:Virtualization}
  \emph{Virtualization} is the act of abstracting the \nameref{def:Hardware} of a single computer into several different execution environments.
  Each of these execution environments operates under the illusion that they are the only environment running on its own private computer.
  These ``computer'' for these environments appears to be native hardware and things in these environments behave as native hardware would but that also protects, manages, and limits them.
\end{definition}

This concept is similar to the \nameref{subsubsec:Layered_Approach} of \nameref{def:Operating_System} implementation (see \Cref{subsubsec:Layered_Approach}).
In the case of \nameref{def:Virtualization}, there is a layer that creates a virtual system on which operating systems or applications can run.

\begin{definition}[Virtual Machine]\label{def:Virtual_Machine}
  A \emph{Virtual Machine} (\emph{VM}, not to be confused with \nameref{def:Virtual_Memory}'s VM) is an \nameref{def:Operating_System} that is running inside an execution environment provided by \nameref{def:Virtualization}.
  The operating system running on this virtualized hardware believes it is running on real hardware, with all of its quirks and problems.
  This operating system also believes it is the \textbf{only} one running, when there could be several virtual machines running on the same hardware.
\end{definition}

\nameref{def:Virtual_Machine} implementations involve several components.
At the base is the \nameref{def:VM_Host}.

\begin{definition}[Host]\label{def:VM_Host}
  The \emph{host} is the the underlying physical \nameref{def:Hardware} system that runs the virtual machines.
\end{definition}

The \nameref{def:VM_Host} runs the \nameref{def:Virtual_Machine_Manager}.

\begin{definition}[Virtual Machine Manager]\label{def:Virtual_Machine_Manager}\label{def:Hypervisor}\label{def:VMM}
  The \emph{Virtual Machine Manager} (\emph{VMM}), also known as a \emph{hypervisor} creates and runs virtual machines by providing an interface that is identical to the host to every guest..
  \begin{remark}
    This is true except in the case of \nameref{subsubsec:Paravirtualization}.
  \end{remark}
\end{definition}

Each guest process is provided with a virtual copy of the host.
Usually, the guest process is an \nameref{def:Operating_System}.
A single physical machine can run multiple operating systems concurrently, each in its own virtual machine.

With \nameref{def:Virtualization}, the definition of ``\nameref{def:Operating_System}'' once again blurs.
For example, consider \nameref{def:VMM} software such as VMware ESX.\@
This virtualization software:
\begin{itemize}[noitemsep]
\item Is installed on the hardware
\item Runs when the hardware boots
\item Provides services to applications
  \begin{itemize}[noitemsep]
  \item CPU scheduling
  \item Memory Management
  \item Migration of VMs between systems.
  \end{itemize}
\end{itemize}

Furthermore, the ``applications'' are guest \nameref{def:Operating_System}s.
Is the VMware ESX VMM an operating system that, in turn, runs other operating systems?
It certainly acts like an operating system.

The implementation of VMMs varies greatly. Options include the following:
\begin{itemize}[noitemsep]
\item Hardware-based solutions that provide support for virtual machine creation and management via \nameref{def:Firmware}.
  These VMMs, which are only found in mainframes, are known as \nameref{def:Type0_Hypervisor}.
  IBM LPARs and Oracle LDOMs are examples.
\item \nameref{def:Operating_System}-like software built to provide \nameref{def:Virtualization}, including VMware ESX, Joyent SmartOS, and Citrix XenServer.
  These VMMs are known as \nameref{def:Type1_Hypervisor}s.
\item General-purpose \nameref{def:Operating_System}s that provide standard functions \textbf{as well as} \nameref{def:VMM} functions, including Microsoft Windows Server with HyperV and Linux with the KVM feature.
  Because such systems have a feature set similar to \nameref{def:Type1_Hypervisor}s, they are also considered type 1.
\item Regular applications that run on standard \nameref{def:Operating_System}s but provide \nameref{def:VMM} features to guest operating systems.
  These applications, which include VMware, Parallels Desktop, and Oracle VirtualBox, are \nameref{def:Type2_Hypervisor}s.
\item \nameref{subsubsec:Paravirtualization}, a technique in which the guest \nameref{def:Operating_System} is modified to work in cooperation with the \nameref{def:VMM} to optimize performance.
\item Programming-environment virtualization, in which \nameref{def:VMM}s do not virtualize real hardware but instead create an optimized virtual system.
  This technique is used by Oracle Java and Microsoft.Net.
\item Emulators that allow applications written for one hardware environment to run on a very different hardware environment, such as a different type of CPU.\@
\item Application containment, which is not virtualization at all but rather provides virtualization-like features by segregating applications from the \nameref{def:Operating_System}.
  Solaris Zones, BSD Jails, and Docker ``contain'' applications, making them more secure and manageable.
\end{itemize}

A formal definition of \nameref{def:Virtualization} helped to establish system requirements and a target for functionality.
The virtualization requirements stated that:
\begin{enumerate}[noitemsep]
\item A VMM provides an environment for programs that is essentially identical to the original machine.
\item Programs running within that environment show only minor performance decreases.
\item The VMM is in complete control of physical system resources.
\end{enumerate}

These requirements of fidelity, performance, and safety still guide \nameref{def:Virtualization} efforts today.

\subsection{Benefits and Features}\label{subsubsec:VM_Benefits_Features}
Several advantages make \nameref{def:Virtualization} attractive.
Most of them are related to the ability for several different execution environments (\nameref{def:Operating_System}s) to concurrently share the same \nameref{def:Hardware}.

One important advantage of \nameref{def:Virtualization} is that the host system is protected from the \nameref{def:Virtual_Machine}s, just as the virtual machines are protected from each other.
A virus inside a guest \nameref{def:Operating_System} might damage that operating system but is unlikely to affect the host or the other guests.
Because each virtual machine is almost completely isolated from all other virtual machines, there are almost no protection problems.
This isolation can also be a problem, because it is difficult to share legitimate resources.
However, there are 2 approaches handling this problem:
\begin{enumerate}[noitemsep]
\item Share a \nameref{def:File_System} \nameref{def:Volume} and thus to share files.
\item Define a virtual network of \nameref{def:Virtual_Machine}s, each of which can send information over the virtual communications network.
  The network is modeled after physical communication networks but is implemented in software.
\end{enumerate}

One feature common to most \nameref{def:Virtualization} implementations is the ability to \textbf{suspend}, a running \nameref{def:Virtual_Machine}.
Most \nameref{def:Operating_System}s provide this basic feature for \nameref{def:Process}es, but in a virtualized system, the guest operating system \textbf{is} a ``process'', which means it can be suspended.
\nameref{def:VMM}s can go one step further and allow copies and \textbf{snapshots} to be made of the guest.
The copy can be used to create a new VM or to move a VM from one machine to another with its current state intact.
The guest can then resume where it was, as if on its original machine, creating a clone.
The snapshot records a point in time, and the guest can be reset to that point if necessary.
Often, VMMs allow any number snapshots to be taken, so long as storage allows.
These abilities are used to good advantage in virtual environments.

A \nameref{def:Virtual_Machine} system is a perfect vehicle for low-level development.
For example, changing an \nameref{def:Operating_System} is a difficult task.
Operating systems are large and complex programs, and a change in one part may cause obscure bugs to appear in some other part.
The power of the operating system makes changing it particularly dangerous.
Because the operating system executes in kernel mode, a wrong change in a pointer could cause an error that would destroy the entire file system.

Another benefits is that system programmers can be given their own \nameref{def:Virtual_Machine}, and development is done on the virtual machine instead of on a physical machine.
When a deployment happens, normal system operation is disrupted only when a completed and tested change is ready to be put into production.
Another advantage of virtual machines for developers is that multiple \nameref{def:Operating_System}s can run concurrently on the developer’s workstation.
This virtualized workstation allows for rapid porting and testing of programs in varying environments.
In addition, multiple versions of a program can run, each in its own isolated operating system, within one system.

A major advantage of \nameref{def:Virtual_Machine}s in production data-center use is system \nameref{def:Consolidation}.

\begin{definition}[Consolidation]\label{def:Consolidation}
  \emph{Consolidation} involves taking physically separate systems and running them in \nameref{def:Virtual_Machine}s on one system.
  These physical-to-virtual conversions typically result in a maximization of resource usage, since many separate lightly used systems can be combined to create one more heavily used system.
\end{definition}

One of the tools that make this possible is templating, where one standard \nameref{def:Virtual_Machine} image, including an
installed and configured guest \nameref{def:Operating_System} and applications, is saved and used as a source for multiple running VMs.
Other features include managing the patching of all guests, backing up and restoring the guests, and monitoring their resource use.
This batch management allows a single system administrator to handle many times the regular amount of physical servers.

Some VMMs include a live migration feature that moves a running guest from one physical server to another without interrupting its operation or active network connections.
If a server is overloaded, live migration can thus free resources on the source host while not disrupting the guest.
Similarly, when \nameref{def:VM_Host} \nameref{def:Hardware} must be repaired or upgraded, guests can be migrated to other servers, the evacuated host can be maintained, and then the guests can be migrated back.
This operation occurs without downtime and without interruption to users.

%%% Local Variables:
%%% mode: latex
%%% TeX-master: "../../EDAF35-Operating_Systems-Reference_Sheet"
%%% End:


\subsection{Building Blocks}\label{subsec:Building_Blocks}
Although the virtual machine concept is useful, it is difficult to implement.
Much work is required to provide an exact duplicate of the underlying machine.
This is especially a challenge on dual-mode systems, where the underlying machine has only user mode and kernel mode.

\begin{remark*}
  Note that these building blocks are not required by \nameref{def:Type0_Hypervisor}s.
\end{remark*}

The ability to virtualize heavily depends on the hardware features provided by the CPU.\@
If the features are sufficient, then it is possible to write a \nameref{def:VMM} that provides a guest environment.
Otherwise, \nameref{def:Virtualization} is impossible.

If it is possible to virtualize a system, the \nameref{def:VMM}s use several techniques to implement it, including \nameref{subsubsec:Trap_and_Emulate} and \nameref{subsubsec:Binary_Translation}.


%%% Local Variables:
%%% mode: latex
%%% TeX-master: "../../EDAF35-Operating_Systems-Reference_Sheet"
%%% End:



%%% Local Variables:
%%% mode: latex
%%% TeX-master: "../EDAF35-Operating_Systems-Reference_Sheet"
%%% End:
