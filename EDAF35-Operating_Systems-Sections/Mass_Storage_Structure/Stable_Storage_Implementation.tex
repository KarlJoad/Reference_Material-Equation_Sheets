\subsection{Stable-Storage Implementation}\label{subsec:Stable_Storage_Implementation}
\begin{definition}[Stable-Storage]\label{def:Stable_Storage}
  \emph{Stable-Storage} is a secondary storage medium where the data it contains is never lost.
\end{definition}

To implement such storage, we need:
\begin{itemize}[noitemsep]
\item To replicate the required information on multiple storage devices with independent failure modes.
\item To coordinate the writing of updates in a way that guarantees that a failure during an update will not leave all the copies in a damaged state.
\item When we are recovering from a failure, we can force all copies to a consistent and correct value, even if another failure occurs during the recovery.
\end{itemize}

When writing to disk in these circumstances, there are 3 possible outcomes:
\begin{enumerate}[noitemsep]
\item \textbf{Successful Completion}.
  The data was written correctly on disk.
\item \textbf{Partial Failure}.
  A failure occurred during transfer, so only some of the sectors were written with the new data.
  The sector being written during the failure may have been corrupted.
\item \textbf{Total Failure}.
  The failure occurred before the disk write started, so the previous data values on the disk remain intact.
\end{enumerate}

If a failure occurs \textbf{during} the writing of a block, the following steps must be followed:
\begin{enumerate}[noitemsep]
\item Write the information onto the first physical block.
\item When the first write completes successfully, write the same information onto the second physical block.
\item Declare the operation complete only after the second write completes successfully.
\item During recovery from a failure, each pair of physical blocks is examined.
  \begin{itemize}[noitemsep]
  \item If both are the same and no detectable error exists, then no further action is necessary.
  \item If one block contains a detectable error then we replace its contents with the value of the other block.
  \item If neither block contains a detectable error, but the blocks differ in content, then we replace the content of the first block with that of the second.
  \end{itemize}
\end{enumerate}

Because waiting for disk writes to complete (synchronous I/O) is time consuming, many storage arrays add NVRAM as a cache.
Since the storage is nonvolatile, it can be trusted to store the data en route to the disks.
It is considered part of the stable storage.
Writes to it are much faster than to magnetic disk, so performance is greatly improved.

%%% Local Variables:
%%% mode: latex
%%% TeX-master: "../../EDAF35-Operating_Systems-Reference_Sheet"
%%% End:
