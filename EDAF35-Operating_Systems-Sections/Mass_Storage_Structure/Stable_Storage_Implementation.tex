\subsection{Stable-Storage Implementation}\label{subsec:Stable_Storage_Implementation}
\begin{definition}[Stable-Storage]\label{def:Stable_Storage}
  \emph{Stable-Storage} is a secondary storage medium where the data it contains is never lost.
\end{definition}

To implement such storage, we need:
\begin{itemize}[noitemsep]
\item To replicate the required information on multiple storage devices with independent failure modes.
\item To coordinate the writing of updates in a way that guarantees that a failure during an update will not leave all the copies in a damaged state.
\item When we are recovering from a failure, we can force all copies to a consistent and correct value, even if another failure occurs during the recovery.
\end{itemize}

When writing to disk in these circumstances, there are 3 possible outcomes:
\begin{enumerate}[noitemsep]
\item \textbf{Successful Completion}.
  The data was written correctly on disk.
\item \textbf{Partial Failure}.
  A failure occurred during transfer, so only some of the sectors were written with the new data.
  The sector being written during the failure may have been corrupted.
\item \textbf{Total Failure}.
  The failure occurred before the disk write started, so the previous data values on the disk remain intact.
\end{enumerate}


%%% Local Variables:
%%% mode: latex
%%% TeX-master: "../../EDAF35-Operating_Systems-Reference_Sheet"
%%% End:
