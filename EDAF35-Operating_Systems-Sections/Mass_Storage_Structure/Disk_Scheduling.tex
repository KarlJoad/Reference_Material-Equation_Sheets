\subsection{Disk Scheduling}\label{subsec:Disk_Scheduling}
Just like with CPU~\nameref{subsec:Scheduling} and \nameref{sec:Virtual_Memory}, the \nameref{def:Operating_System} attempts to maximize the efficient use of hardware resources.
For disk drives, this means minimizing access time and maximizing bandwidth.

\begin{definition}[Disk Bandwidth]\label{def:Disk_Bandwidth}
  The \emph{disk bandwidth} is the total number of bytes transferred, divided by the total time between the first request for service and the completion of the last transfer.

  The service time will include the seek time (time for disk arm to move heads to cylinder) and the rotational latency (time for platter to spin to correct sector).
\end{definition}

When performing disk I/O, we need to know a few things:
\begin{itemize}[noitemsep]
\item Whether this operation is input or output.
\item What the disk address for the transfer is.
\item What the memory address for the transfer is.
\item What the number of sectors to be transferred is.
\end{itemize}

If the desired disk drive \textbf{and} controller are available, the request can be serviced immediately.
If the drive or controller is busy, any new requests for service will be placed in the queue of pending requests for that drive.

Like before, the choice of how to handle pending requests in the queue is handled by the choice of disk-scheduling algorithm.
Choosing the best one is difficult because performance depends heavily on the number and types of requests.
\nameref{subsubsec:SSTF_Disk_Scheduling} is common and has a natural appeal because it increases performance over \nameref{subsubsec:FCFS_Disk_Scheduling}.
\nameref{subsubsec:SCAN_Disk_Scheduling} and \nameref{par:Circular_SCAN_Disk_Scheduling} perform better for systems that place a heavy load on the disk, because they are less likely to cause a starvation problem.
For any particular list of requests, we \textbf{can} define an optimal order of retrieval, but the computation needed to find an optimal schedule may not justify the savings over SSTF or SCAN.\@
Because of the complexity involved with this topic, and the potential cost by choosing the wrong algorithm, the disk-scheduling algorithm should be a separate module of the OS, so it can be replaced if need be.

Requests for disk service can be greatly influenced by the file-allocation method.
A program reading a contiguously allocated file will generate several requests that are close together on the disk, resulting in limited head movement.
A linked or indexed file, in contrast, may include blocks that are widely scattered on the disk, resulting in greater head movement.
The location of directories and index blocks is also important.
Since every file must be opened to be used, and opening a file requires searching the directory structure, the directories will be accessed frequently.

\subsubsection{First-Come First-Serve Scheduling}\label{subsubsec:FCFS_Disk_Scheduling}
This version of \emph{First-Come First-Serve Scheduling} is like all the others discussed earlier.
The first disk seek request that comes into the queue is handled first, without regard to the distance the head must travel to find the requested data.

\subsubsection{Shortest-Seek-Time-First Scheduling}\label{subsubsec:SSTF_Disk_Scheduling}
\emph{Shortest-Seek-Time-First Scheduling} services all the requests close to the current head position before moving the head far away to service other requests.
This method selects the request with the least seek time from the current head position.

This is a variant of \nameref{par:SJF_Scheduling}, and like that algorithm, this one can also experience \nameref{def:Starvation}.

\subsubsection{SCAN Scheduling}\label{subsubsec:SCAN_Disk_Scheduling}
\emph{SCAN Scheduling} starts the disk arm at one end of the disk and moves toward the other end, servicing requests as it reaches each cylinder, until it gets to the other end of the disk.
At the other end, the direction of head movement is reversed, and servicing continues on the way back.
The head continuously scans back and forth across the disk.

\paragraph{Circular SCAN Scheduling}\label{par:Circular_SCAN_Disk_Scheduling}
\emph{Circular SCAN} (\emph{C-SCAN}) \emph{Scheduling} is a variant of \nameref{subsubsec:SCAN_Disk_Scheduling} designed to provide a more uniform wait time.
Like SCAN, C-SCAN moves the head from one end of the disk to the other, servicing requests along the way.
When the head reaches the other end, it immediately returns to the beginning of the disk without servicing any requests on the return trip.
Essentially, this treats the cylinders as a circular list that wraps around from the final cylinder to the first.


%%% Local Variables:
%%% mode: latex
%%% TeX-master: "../../EDAF35-Operating_Systems-Reference_Sheet"
%%% End:
