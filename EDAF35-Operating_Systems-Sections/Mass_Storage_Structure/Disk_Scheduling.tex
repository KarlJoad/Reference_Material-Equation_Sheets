\subsection{Disk Scheduling}\label{subsec:Disk_Scheduling}
Just like with CPU~\nameref{subsec:Scheduling} and \nameref{sec:Virtual_Memory}, the \nameref{def:Operating_System} attempts to maximize the efficient use of hardware resources.
For disk drives, this means minimizing access time and maximizing bandwidth.

\begin{definition}[Disk Bandwidth]\label{def:Disk_Bandwidth}
  The \emph{disk bandwidth} is the total number of bytes transferred, divided by the total time between the first request for service and the completion of the last transfer.

  The service time will include the seek time (time for disk arm to move heads to cylinder) and the rotational latency (time for platter to spin to correct sector).
\end{definition}

When performing disk I/O, we need to know a few things:
\begin{itemize}[noitemsep]
\item Whether this operation is input or output.
\item What the disk address for the transfer is.
\item What the memory address for the transfer is.
\item What the number of sectors to be transferred is.
\end{itemize}

If the desired disk drive \textbf{and} controller are available, the request can be serviced immediately.
If the drive or controller is busy, any new requests for service will be placed in the queue of pending requests for that drive.


%%% Local Variables:
%%% mode: latex
%%% TeX-master: "../../EDAF35-Operating_Systems-Reference_Sheet"
%%% End:
