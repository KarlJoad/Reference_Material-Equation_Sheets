\subsection{Disk Management}\label{subsec:Disk_Management}
The OS handles many other aspects of the disks and their usage as well.

\subsubsection{Formatting}\label{subsubsec:Formatting}
There are 2 kinds of formatting:
\begin{enumerate}[noitemsep]
\item \nameref{def:Low_Level_Formatting}
\item \nameref{def:High_Level_Formatting}
\end{enumerate}

Most hard disks are low-level-formatted at the factory as a part of the manufacturing process.
This enables the manufacturer to test the disk and to initialize the mapping from logical block numbers to defect-free sectors on the disk.

\begin{definition}[Low-Level Formatting]\label{def:Low_Level_Formatting}
  \emph{Low-Level Formatting} or \emph{Physical Formatting} is the process of dividing the sectors in the tracks on the platters.
  This fills each sector with a special data structure typically consisting of a header, a data area (usually 512 bytes), and a trailer.
  The header and trailer contain information used by the disk controller: a sector number and an error-correcting code.
\end{definition}

The ECC contains enough information, if only a few bits of data have been corrupted, to enable the controller to identify which bits have changed and calculate what their correct values should be.
It then reports a recoverable soft error.
The controller automatically does the ECC processing whenever a sector is read or written.

It is usually possible to choose among a few data sizes, such as 256, 512, and 1,024 bytes.
Formatting a disk with a larger sector size means that fewer sectors can fit on each track; but it also means that fewer headers and trailers are written on each track and more space is available for user data.
\nameref{def:Operating_System}s may not support sizes other than a sector size of 512 bytes.

\begin{definition}[High-Level Formatting]\label{def:High_Level_Formatting}
  \emph{High-Level Formatting} or \emph{Logical Formatting} is the process an operating system goes through to partition the disk into groups of cylinders and put a file system onto these partitions.
  Each partition can be logically viewed as a separate disk, which can have its own file system.
  This stores the initial file system data structures, which can include maps of free and allocated storage and an initial empty directory.
\end{definition}

To increase efficiency, most \nameref{def:File_System}s group blocks together into larger chunks, frequently called clusters.
Disk I/O is done via blocks, but file system I/O is done via clusters.

Sometimes we want to use a disk partition just as a large sequential array of logical blocks, without any structure.
This array is called a raw disk, and I/O to this array is termed raw I/O.
Because there are no structures, and hence no file syste, raw I/O bypasses all file-system services, such as the buffer cache, file locking, prefetching, space allocation, file names, and directories.


%%% Local Variables:
%%% mode: latex
%%% TeX-master: "../../EDAF35-Operating_Systems-Reference_Sheet"
%%% End:
