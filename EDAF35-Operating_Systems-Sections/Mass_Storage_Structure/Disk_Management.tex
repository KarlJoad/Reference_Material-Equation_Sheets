\subsection{Disk Management}\label{subsec:Disk_Management}
The OS handles many other aspects of the disks and their usage as well.

\subsubsection{Formatting}\label{subsubsec:Formatting}
There are 2 kinds of formatting:
\begin{enumerate}[noitemsep]
\item \nameref{def:Low_Level_Formatting}
\item \nameref{def:High_Level_Formatting}
\end{enumerate}

\begin{definition}[Low-Level Formatting]\label{def:Low_Level_Formatting}
  \emph{Low-Level Formatting} or \emph{Physical Formatting} is the process of dividing the sectors in the tracks on the platters.
  This fills each sector with a special data structure typically consisting of a header, a data area (usually 512 bytes), and a trailer.
  The header and trailer contain information used by the disk controller: a sector number and an error-correcting code.
\end{definition}

\begin{definition}[High-Level Formatting]\label{def:High_Level_Formatting}
  \emph{High-Level Formatting} or \emph{Logical Formatting} is the process an operating system goes through to partition the disk into groups of cylinders and put a file system onto these partitions.
  Each partition can be logically viewed as a separate disk, which can have its own file system.
  This stores the initial file system data structures, which can include maps of free and allocated storage and an initial empty directory.
\end{definition}


%%% Local Variables:
%%% mode: latex
%%% TeX-master: "../../EDAF35-Operating_Systems-Reference_Sheet"
%%% End:
