\subsection{Disk Structure}\label{subsec:Disk_Structure}
Here, we view all potential storage media (\Cref{subsec:Structure_Storage_Media}) as being the same.

Drives are addressed as large one-dimensional arrays of \nameref{def:Logical_Block}s.

\begin{definition}[Logical Block]\label{def:Logical_Block}
  The \emph{logical block} is the smallest unit of transfer.
  The size of a logical block is usually 512 bytes, although some disks can use a different \nameref{def:Low_Level_Formatting} to have a different logical block size, such as 1,024 bytes.
\end{definition}

This one-dimensional array of logical blocks is mapped onto the sectors of the disk sequentially.
Sector 0 is the \emph{first sector} of the \emph{first track} on the \emph{outermost cylinder}.
The mapping proceeds in order through that track, then through the rest of the tracks in that cylinder, and then through the rest of the cylinders from outermost to innermost.

In theory, we can convert a \nameref{def:Logical_Block} number into a disk address that consists of a cylinder number, a track number within that cylinder, and a sector number within that track.
However, in practice, it is difficult to perform this translation.
\begin{enumerate}[noitemsep]
\item Most disks have some defective sectors after manufacturing.
  \begin{itemize}[noitemsep]
  \item The mapping hides this by substituting spare sectors from elsewhere on the disk.
\end{itemize}

\item The number of sectors per track is not a constant on some drives, because of the distance from the center.
\end{enumerate}

Handling the issue of the number of sectors is done through one of 2 methods.
\begin{enumerate}[noitemsep]
\item \textbf{Constant Linear Velocity} (CLV).
  The density of bits per track is uniform.
  The farther a track is from the center of the disk, the greater its length, so the more sectors it can hold.
  The number of sectors per track decreases from outer tracks to inner ones.
  The drive increases its rotation speed as the head moves from the outer to the inner tracks to keep the same rate of data moving under the head.
  This method is used in CD and DVD drives.
\item \textbf{Constant Angular Velocity} (CAV).
  The disk rotation speed can stays constant
  The density of bits decreases from inner tracks to outer tracks to keep the data rate constant.
  This method is used in traditional hard disks.
\end{enumerate}

%%% Local Variables:
%%% mode: latex
%%% TeX-master: "../../EDAF35-Operating_Systems-Reference_Sheet"
%%% End:
