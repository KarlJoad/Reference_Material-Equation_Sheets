\subsection{Disk Attachment}\label{subsec:Disk_Attachment}
This section is concerned with how these physical disks or tapes are attached to the computer for use.
There are 2 ways to access disk storage:
\begin{enumerate}[noitemsep]
\item \nameref{def:Host_Attached_Storage}
\item \nameref{def:Network_Attached_Storage}
\end{enumerate}

\subsubsection{Host-Attached Storage}\label{subsubsec:Host_Attached_Storage}
In \nameref{def:Host_Attached_Storage} any combination of I/O bus architectures and protocols can be used.
The I/O commands that initiate data transfers to a host-attached storage device are reads and writes of logical data blocks directed to specifically identified storage units (such as bus ID or target logical unit).

\begin{definition}[Host-Attached Storage]\label{def:Host_Attached_Storage}
  \emph{Host-Attached Storage} is disk storage that is attached to the computer through the local I/O bus.
\end{definition}

\subsubsection{Network-Attached Storage}\label{subsubsec:Network_Attached_Storage}
\nameref{def:Network_Attached_Storage} provides a convenient way for all the computers on a LAN to share a pool of storage with the same ease of naming and access enjoyed with local \nameref{def:Host_Attached_Storage}.
However, it tends to be less efficient and have lower performance than some direct-attached storage options.

\begin{definition}[Network-Attached Storage]\label{def:Network_Attached_Storage}
  \emph{Network-Attached Storage} (\emph{NAS}) is disk storage that is attached to the local computer by making a connection to a remote host that serves a distributed file system to connectees.
  The remote procedure calls (RPCs) are carried via TCP or UDP over an IP network.
  Typically, the client and server lie on the same local-area network (LAN) that carries all data traffic to the clients.
\end{definition}

\subsubsection{Storage-Area Network}\label{subsubsec:Storage_Area_Network}
\nameref{def:Storage_Area_Network}s require an interconnect between the storage arrays, one that allows for very high-speed communication.

\begin{definition}[Storage-Area Network]\label{def:Storage_Area_Network}
  \emph{Storage-Area Network}s, (\emph{SAN}s) behave somewhat opposite a \nameref{def:Network_Attached_Storage} system.
  A SAN is a private network (using storage protocols rather than networking protocols) connecting servers and storage units.
  The power of a SAN lies in its flexibility.
  Multiple hosts and multiple storage arrays can attach to the same SAN, and storage can be dynamically allocated to hosts.
  Here, the data is stored in physically separate machines and all brought together with a network backbone.

  A SAN switch allows or prohibits access between the hosts and the storage.
  Instead, any service can connect to the SAN and access the data within.
\end{definition}

%%% Local Variables:
%%% mode: latex
%%% TeX-master: "../../EDAF35-Operating_Systems-Reference_Sheet"
%%% End:
