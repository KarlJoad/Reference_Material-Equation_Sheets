\subsection{Disk Attachment}\label{subsec:Disk_Attachment}
This section is concerned with how these physical disks or tapes are attached to the computer for use.
There are 2 ways to access disk storage:
\begin{enumerate}[noitemsep]
\item \nameref{def:Host_Attached_Storage}
\item \nameref{def:Network_Attached_Storage}
\end{enumerate}

\subsubsection{Host-Attached Storage}\label{subsubsec:Host_Attached_Storage}
In \nameref{def:Host_Attached_Storage} any combination of I/O bus architectures and protocols can be used.
The I/O commands that initiate data transfers to a host-attached storage device are reads and writes of logical data blocks directed to specifically identified storage units (such as bus ID or target logical unit).

\begin{definition}[Host-Attached Storage]\label{def:Host_Attached_Storage}
  \emph{Host-Attached Storage} is disk storage that is attached to the computer through the local I/O bus.
\end{definition}

\subsubsection{Network-Attached Storage}\label{subsubsec:Network_Attached_Storage}
\nameref{def:Network_Attached_Storage} provides a convenient way for all the computers on a LAN to share a pool of storage with the same ease of naming and access enjoyed with local \nameref{def:Host_Attached_Storage}.
However, it tends to be less efficient and have lower performance than some direct-attached storage options.


%%% Local Variables:
%%% mode: latex
%%% TeX-master: "../../EDAF35-Operating_Systems-Reference_Sheet"
%%% End:
