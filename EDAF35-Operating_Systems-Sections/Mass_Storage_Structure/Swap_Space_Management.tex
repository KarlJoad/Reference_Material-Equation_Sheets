\subsection{Swap-Space Management}\label{subsec:Swap_Space_Management}
Swap-space management is another low-level task of the \nameref{def:Operating_System}.
\nameref{def:Swapping} occurs when the amount of physical memory reaches a critically low point and \nameref{def:Process}es are moved from memory to swap space to free available \nameref{def:Physical_Memory}.
\nameref{def:Virtual_Memory} uses disk space as an extension of main memory.

Most systems combine swapping with virtual memory techniques and swap pages, not necessarily entire processes.
In addition, most swapping systems only swap \nameref{def:Anonymous_Memory}.
``\nameref{def:Swapping}'' and ``\nameref{def:Paging}'' are used interchangeably now.

Since disk access is much slower than memory access, using swap space significantly decreases system performance.
The main goal for the design and implementation of swap space is to provide the best throughput for the virtual memory system.

Each swap area consists of a series of page-sized slots, which are used to hold swapped pages.
Associated with each swap area is a \emph{swap map}, which is an array of integer counters, each corresponding to a page slot in the swap area.
\begin{itemize}[noitemsep]
\item If the value of a counter is 0, the corresponding page slot is available.
\item Values greater than 0 indicate that the page slot is occupied by a swapped page, where the value of the counter indicates the number of mappings to the swapped page.
\end{itemize}

\subsubsection{Swap-Space Usage}\label{subsubsec:Swap_Space_Usage}
The amount of swap space needed on a system can vary depending on the amount of \nameref{def:Physical_Memory}, the amount of \nameref{def:Virtual_Memory} it is backing, and the way in which the virtual memory is used.

It is safer to overestimate than to underestimate the amount of swap space required, because if a system runs out of swap space it may be forced to abort \nameref{def:Process}es or may crash entirely.
Overestimation wastes disk space that could otherwise be used for \nameref{def:File}s, but it does no other harm.

Some \nameref{def:Operating_System}s allow the use of multiple swap spaces, including both swap files and swap partitions.
These are usually placed on separate disks so that the load placed on the I/O system by \nameref{def:Paging} and \nameref{def:Swapping} can be spread over the system’s I/O bandwidth.


%%% Local Variables:
%%% mode: latex
%%% TeX-master: "../../EDAF35-Operating_Systems-Reference_Sheet"
%%% End:
