\subsection{System and Network Threats}\label{subsec:System_Network_Threats}
Program threats typically use a breakdown in the protection mechanisms of a system to attack programs.
In contrast, system and network threats involve the abuse of services and network connections.
System and network threats create a situation in which \nameref{def:Operating_System} resources and \nameref{def:User} files are misused.

Sometimes, a system and network attack is used to launch a program attack, and vice versa.
The more open an operating system is, the more services it has enabled and the more functions it allows, the more likely it is that a bug is available to exploit.
Such changes reduce the system’s \nameref{def:Attack_Surface}.

\begin{definition}[Attack Surface]\label{def:Attack_Surface}
  The \emph{attack surface} of a system is the set of ways in which an attacker can try to break into the system.
\end{definition}

It is important to note that masquerading and replay attacks are also commonly launched over networks between systems.
In general, sharing secrets (to prove identity and as keys to encryption) is required for authentication and encryption, and sharing secrets is easier in environments in which secure sharing methods exist.

\subsubsection{Worms}\label{subsubsec:Worms}
\begin{definition}[Worm]\label{def:Worm}
  A \emph{worm} is a process that uses a spawn mechanism to duplicate itself.
  The spawned copies use up system resources and may lock out all other \nameref{def:Process}es.
  The copies can also execute arbitrary code, which means they can cause other damage too.
\end{definition}

On computer networks, worms are particularly potent, since they may reproduce themselves among systems and thus shut down an entire network.


%%% Local Variables:
%%% mode: latex
%%% TeX-master: "../../EDAF35-Operating_Systems-Reference_Sheet"
%%% End:
