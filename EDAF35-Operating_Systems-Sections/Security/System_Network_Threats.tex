\subsection{System and Network Threats}\label{subsec:System_Network_Threats}
Program threats typically use a breakdown in the protection mechanisms of a system to attack programs.
In contrast, system and network threats involve the abuse of services and network connections.
System and network threats create a situation in which \nameref{def:Operating_System} resources and \nameref{def:User} files are misused.

Sometimes, a system and network attack is used to launch a program attack, and vice versa.
The more open an operating system is, the more services it has enabled and the more functions it allows, the more likely it is that a bug is available to exploit.
Such changes reduce the system’s \nameref{def:Attack_Surface}.

\begin{definition}[Attack Surface]\label{def:Attack_Surface}
  The \emph{attack surface} of a system is the set of ways in which an attacker can try to break into the system.
\end{definition}

It is important to note that masquerading and replay attacks are also commonly launched over networks between systems.
In general, sharing secrets (to prove identity and as keys to encryption) is required for authentication and encryption, and sharing secrets is easier in environments in which secure sharing methods exist.

\subsubsection{Worms}\label{subsubsec:Worms}
\begin{definition}[Worm]\label{def:Worm}
  A \emph{worm} is a process that uses a spawn mechanism to duplicate itself.
  The spawned copies use up system resources and may lock out all other \nameref{def:Process}es.
  The copies can also execute arbitrary code, which means they can cause other damage too.
\end{definition}

On computer networks, worms are particularly potent, since they may reproduce themselves among systems and thus shut down an entire network.

\subsubsection{Port Scanning}\label{subsubsec:Port_Scanning}
Port scanning is not an attack but rather a means for a cracker to detect a system’s vulnerabilities to attack.
Port scanning typically is automated, involving a tool that attempts to create a TCP/IP connection to a specific port or a range of ports.

One such program is \href{https://www.insecure.org/nmap/}{\texttt{nmap}} is a very versatile open-source utility for network exploration and security auditing.
When pointed at a target, it will determine what services are running, including application names and versions.
It can identify the host operating system.
It can also provide information about defenses, such as what firewalls are defending the target.
It does not exploit any known bugs.

Because port scans are detectable, they frequently are launched from zombie systems.
Such systems are previously compromised, independent systems that are serving their original owners while also being used for nefarious purposes.
Zombies make attackers particularly difficult to prosecute because determining the source of the attack and the person that launched it is challenging.
This is one of many reasons for securing all systems, including ``inconsequential'' systems, not just systems containing ``valuable'' information or services.

\subsubsection{Denial of Service}\label{subsubsec:Denial_of_Service}
Denial-of-service attacks are aimed not at gaining information or stealing resources but rather at disrupting legitimate use of a system or facility.
Most such attacks involve systems that the attacker has not penetrated.
Launching an attack that prevents legitimate use is frequently easier than breaking into a machine or facility.
Generally, it is impossible to prevent denial-of-service attacks.
The attacks use the same mechanisms as normal operation.

Denial-of-service attacks are generally network-based.
They fall into two categories:
\begin{enumerate}[noitemsep]
\item The attack uses so many facility resources that no useful work can be done.
\item Disrupting the network of the facility.
\end{enumerate}

Even more difficult to prevent and resolve are distributed denial-of-service (DDOS) attacks.
These attacks are launched from multiple sites at once, toward a common target, typically by zombies.

%%% Local Variables:
%%% mode: latex
%%% TeX-master: "../../EDAF35-Operating_Systems-Reference_Sheet"
%%% End:
