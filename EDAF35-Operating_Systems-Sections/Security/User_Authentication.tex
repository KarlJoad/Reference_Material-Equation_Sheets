\subsection{User Authentication}\label{subsec:User_Authentication}
If a system cannot authenticate a user, then authenticating that a message came from that user is pointless.
Thus, a major security problem for operating systems is user authentication.
The protection system depends on the ability to identify the programs and processes currently executing, which in turn depends on the ability to identify each user of the system.

Users normally identify themselves.
So how do we determine whether a user’s identity is authentic?
Generally, user authentication is based on one or more of three things:
\begin{enumerate}[noitemsep]
\item The user’s possession of something (a key or card).
\item The user’s knowledge of something (a user identifier and password).
\item An attribute of the user (fingerprint, retina pattern, or signature).
\end{enumerate}


%%% Local Variables:
%%% mode: latex
%%% TeX-master: "../../EDAF35-Operating_Systems-Reference_Sheet"
%%% End:
