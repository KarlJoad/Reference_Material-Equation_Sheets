\subsection{User Authentication}\label{subsec:User_Authentication}
If a system cannot authenticate a user, then authenticating that a message came from that user is pointless.
Thus, a major security problem for operating systems is user authentication.
The protection system depends on the ability to identify the programs and processes currently executing, which in turn depends on the ability to identify each user of the system.

Users normally identify themselves.
So how do we determine whether a user’s identity is authentic?
Generally, user authentication is based on one or more of three things:
\begin{enumerate}[noitemsep]
\item The user’s possession of something (a key or card).
\item The user’s knowledge of something (a user identifier and password).
\item An attribute of the user (fingerprint, retina pattern, or signature).
\end{enumerate}

\subsubsection{Passwords}\label{subsubsec:User_Authentication_Passwords}
The most common approach to authenticating a user identity is the use of passwords.
When the user identifies themself by user ID or account name, they are asked for a password.
If the user-supplied password matches the password stored in the system, the system assumes that the account is being accessed by the owner of that account.
Passwords are often used to protect objects in the computer system, in the absence of more complete protection schemes.
They can be treated as a special case of keys or capabilities.
Different passwords may be associated with different access rights.
For example, different passwords may be used for reading files, appending files, and updating files.

In practice, most systems require only one password for a user to gain full rights.
Although more passwords theoretically would be more secure, such systems tend not to be implemented due to the classic trade-off between security and convenience.
If security makes something inconvenient, then the security is frequently bypassed or otherwise circumvented.


%%% Local Variables:
%%% mode: latex
%%% TeX-master: "../../EDAF35-Operating_Systems-Reference_Sheet"
%%% End:
