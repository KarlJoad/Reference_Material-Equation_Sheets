\subsection{The Security Problem}\label{subsec:Security_Problem}
In \Cref{sec:Protection}, we discussed mechanisms that the operating system can provide that allow users to protect their resources.
These mechanisms work well \textbf{only} as long as the users conform to the intended use of and access to these resources.

A system is secure if its resources are used and accessed as intended under \textbf{all} circumstances.
Unfortunately, total security cannot be achieved.
Nonetheless, there must have mechanisms to make security breaches a rare occurrence, rather than the norm.

Security violations (or misuse) of the system can be categorized as intentional (malicious) or accidental.
It is easier to protect against accidental misuse than against malicious misuse.
For the most part, protection mechanisms are the core of protection from accidents.

We should note that in our discussion of security, the term:
\begin{itemize}[noitemsep]
\item Intruder is for those attempting to breach security.
\item A threat is the \textbf{potential} for a security violation, such as the discovery of a vulnerability.
\item An attack is an attempt to break security.
\end{itemize}

%%% Local Variables:
%%% mode: latex
%%% TeX-master: "../../EDAF35-Operating_Systems-Reference_Sheet"
%%% End:
