\subsection{The Security Problem}\label{subsec:Security_Problem}
In \Cref{sec:Protection}, we discussed mechanisms that the operating system can provide that allow users to protect their resources.
These mechanisms work well \textbf{only} as long as the users conform to the intended use of and access to these resources.

A system is secure if its resources are used and accessed as intended under \textbf{all} circumstances.
Unfortunately, total security cannot be achieved.
Nonetheless, there must have mechanisms to make security breaches a rare occurrence, rather than the norm.

Security violations (or misuse) of the system can be categorized as intentional (malicious) or accidental.
It is easier to protect against accidental misuse than against malicious misuse.
For the most part, protection mechanisms are the core of protection from accidents.

We should note that in our discussion of security, the term:
\begin{itemize}[noitemsep]
\item Intruder is for those attempting to breach security.
\item A threat is the \textbf{potential} for a security violation, such as the discovery of a vulnerability.
\item An attack is an attempt to break security.
\end{itemize}

The most common forms of security violations include:
\begin{itemize}[noitemsep]
\item \textbf{Breach of confidentiality}.
  This type of violation involves unauthorized reading of data (or theft of information).
  Typically, a breach of confidentiality is the goal of an intruder.
  Capturing secret data from a system or a data stream, such as credit-card information or identity information for identity theft, can result directly in money for the intruder.
\item \textbf{Breach of integrity}.
  This violation involves unauthorized modification of data.
  Such attacks can result in passing of liability to an innocent party or modification of the source code of an important commercial application.
\item \textbf{Breach of availability}.
  This violation involves unauthorized destruction of data.
  Some intruders would rather wreak havoc and gain status or bragging rights than gain financially.
  Website defacement is a common example of this type of security breach.
\item \textbf{Theft of service}.
  This violation involves unauthorized use of resources.
  For example, an intruder (or intrusion program) may install a \nameref{def:Daemon} on a system that acts as a file server.
\item \textbf{Denial of service}.
  This violation involves preventing legitimate use of the system.
  Denial-of-service (DOS) attacks are sometimes accidental.
  The original Internet worm turned into a DOS attack when a bug failed to delay its rapid spread.
\end{itemize}

%%% Local Variables:
%%% mode: latex
%%% TeX-master: "../../EDAF35-Operating_Systems-Reference_Sheet"
%%% End:
