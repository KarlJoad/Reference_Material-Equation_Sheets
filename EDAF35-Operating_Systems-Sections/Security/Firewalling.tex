\subsection{Firewalling}\label{subsec:Firewalling}
Now, how can a trusted computer can be connected safely to an untrustworthy network?
The most common solution is the use of a \nameref{def:Firewall} to separate trusted and untrusted systems.

\begin{definition}[Firewall]\label{def:Firewall}
  A \emph{firewall} is a computer, appliance, or router that sits between the trusted and the untrusted.
\end{definition}

A network \nameref{def:Firewall} limits network access between the two security domains and monitors and logs all connections.
It can also limit connections based on source or destination address, source or destination port, or direction of the connection.
In fact, a network firewall can separate a network into multiple domains.
A common implementation has the Internet as the untrusted domain; a semitrusted/semisecure network, called the demilitarized zone (DMZ); and the trusted computers as a third domain.

Of course, a \nameref{def:Firewall} itself must be secure and attack-proof.
Otherwise, its ability to secure connections can be compromised.
Furthermore, firewalls do not prevent attacks that tunnel, or travel within protocols or connections that the firewall allows.
A buffer-overflow attack to a web server will not be stopped by the firewall, for example, because the HTTP connection is allowed; it is the contents of the HTTP connection that house the attack.
Likewise, denial-of-service attacks can affect firewalls as much as any other machines.
Another vulnerability of firewalls is spoofing, in which an unauthorized host pretends to be an authorized host by meeting some authorization criterion.

%%% Local Variables:
%%% mode: latex
%%% TeX-master: "../../EDAF35-Operating_Systems-Reference_Sheet"
%%% End:
