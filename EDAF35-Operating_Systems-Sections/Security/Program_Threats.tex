\subsection{Program Threats}\label{subsec:Program_Threats}
\nameref{def:Process}es, along with the \nameref{def:Kernel}, are the only means of accomplishing work on a computer.
Therefore, writing a program that creates a breach of security, or causing a normal process to change its behavior and create a breach, is a common goal of intruders.
Most nonprogram security events have causing a program threat as a goal.

\subsubsection{Trojan Horse}\label{subsubsec:Trojan_Horse}
Many systems have mechanisms for allowing programs written by users to be executed by other \nameref{def:User}s.
If these programs are executed in a domain that provides the \nameref{def:Access_Right}s of the executing user, the other users may misuse these rights.
For example, a text editing program may include code to search the file to be edited for certain keywords.
If any are found, the entire file is copied to a special area accessible to the \textbf{creator} of the text editor.
A code segment that misuses its environment is called a \nameref{def:Trojan_Horse}.

\begin{definition}[Trojan Horse]\label{def:Trojan_Horse}
  A \emph{trojan horse} or \emph{trojan} is any malware which misleads users of its true intent.
  Its code segment misuses the execution environment it is given to achieve something other than what is intended.
\end{definition}

\nameref{def:Trojan_Horse}s also include malware that mimics genuine software, to ensure the user is complacent and inputs secure information without realizing it is stolen/compromised.

Another variation on the \nameref{def:Trojan_Horse} is \nameref{def:Spyware}.
Spyware sometimes accompanies a program that the user has chosen to install.
\begin{definition}[Spyware]\label{def:Spyware}
  \emph{Spyware} is malware that intentionally monitors the \nameref{def:User}'s actions.
  The usual goal is to download ads to display on the user’s system, create pop-up browser windows when certain sites are visited, or use \nameref{def:Covert_Channel}s.
\end{definition}

Most frequently, \nameref{def:Spyware} comes with freeware or shareware programs, but sometimes is included with commercial software.

\begin{definition}[Covert Channel]\label{def:Covert_Channel}
  \emph{Covert channel}s are intended to capture information from the user’s system and return it to a central site.
\end{definition}

\subsubsection{Trap Door}\label{subsubsec:Trap_Door}
Another common, and very hard to detect type of malware is a \nameref{def:Trap_Door}.
\begin{definition}[Trap Door]\label{def:Trap_Door}
  A \emph{trap door} is a piece of malware where the designer of the program or system leaves a security hole in the software that only they are capable of using.
\end{definition}

For instance, the code might check for a specific user ID or password, and it might circumvent normal security procedures.

Trap doors pose a difficult problem because, to detect them, we have to analyze all the source code for all components of a system.
Given that software systems may consist of millions of lines of code, this analysis is not done frequently.

\subsubsection{Logic Bomb}\label{subsubsec:Logic_Bomb}
\nameref{def:Logic_Bomb}s is not malware, per-se; rather it is code that is executed \textbf{only} when certain conditions have been met.
What that code does determines how it is classified.

\begin{definition}[Logic Bomb]\label{def:Logic_Bomb}
  \emph{Logic bomb}s are programs that creates a security incident \textbf{only} under certain conditions.
  It would be hard to detect because under normal operations, there would be no security hole.
  However, when a predefined set of parameters was met, the security hole would be created.
\end{definition}

\subsubsection{Stack and Buffer Overflow}\label{subsubsec:Stack_Buffer_Overflow}
The stack- or buffer-overflow attack is the most common way for an attacker outside the system to gain unauthorized access to the target system.
An authorized user of the system may also use this exploit for privilege escalation.

Essentially, the attack exploits a bug in a program.
The bug can be a simple case of poor programming, in which the programmer neglected to code bounds checking on an input field.
In this case, the attacker sends more data than the program was expecting.
By using trial and error, or by examining the source code of the attacked program if it is available, the attacker determines the vulnerability and writes a program to do the following:
\begin{enumerate}[noitemsep]
\item Overflow an input field, command-line argument, or input buffer until it writes into the stack.
\item Overwrite the current return address on the stack with the address of the exploit code loaded in step 3.
\item Write a simple set of code for the next space in the stack that includes the commands that the attacker wishes to execute.
\end{enumerate}

The result of this attack program’s execution will be a privileged command execution.

When a function is invoked in a typical computer architecture, the local variables defined in the function, the parameters passed to the function, and the address to which control returns once the function exits are stored in a stack frame.
At the end of a stack frame, there is the return address, which specifies where to return control once the function exits.
The frame pointer must be fixed on the stack, as the value of the stack pointer can vary during the function call.
The saved frame pointer allows relative access to parameters and automatic variables.

Given this standard memory layout, an intruder could execute a buffer-overflow attack.
The goal being to replace the return address in the stack frame so that it points to a code segment containing the attacking program.

One solution to this problem is for the CPU to have a feature that disallows execution of code in a stack section of memory.
Recent versions x86 chips include the \texttt{NX} feature to prevent this type of attack.
The hardware implementation involves the use of a new bit in the page tables of the CPUs.
This bit marks the associated page as nonexecutable, so that instructions cannot be read from it and executed.


%%% Local Variables:
%%% mode: latex
%%% TeX-master: "../../EDAF35-Operating_Systems-Reference_Sheet"
%%% End:
