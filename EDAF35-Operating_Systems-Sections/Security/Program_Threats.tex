\subsection{Program Threats}\label{subsec:Program_Threats}
\nameref{def:Process}es, along with the \nameref{def:Kernel}, are the only means of accomplishing work on a computer.
Therefore, writing a program that creates a breach of security, or causing a normal process to change its behavior and create a breach, is a common goal of intruders.
Most nonprogram security events have causing a program threat as a goal.

\subsubsection{Trojan Horse}\label{subsubsec:Trojan_Horse}
Many systems have mechanisms for allowing programs written by users to be executed by other \nameref{def:User}s.
If these programs are executed in a domain that provides the \nameref{def:Access_Right}s of the executing user, the other users may misuse these rights.
For example, a text editing program may include code to search the file to be edited for certain keywords.
If any are found, the entire file is copied to a special area accessible to the \textbf{creator} of the text editor.
A code segment that misuses its environment is called a \nameref{def:Trojan_Horse}.

\begin{definition}[Trojan Horse]\label{def:Trojan_Horse}
  A \emph{trojan horse} or \emph{trojan} is any malware which misleads users of its true intent.
  Its code segment misuses the execution environment it is given to achieve something other than what is intended.
\end{definition}

\nameref{def:Trojan_Horse}s also include malware that mimics genuine software, to ensure the user is complacent and inputs secure information without realizing it is stolen/compromised.

Another variation on the \nameref{def:Trojan_Horse} is \nameref{def:Spyware}.
Spyware sometimes accompanies a program that the user has chosen to install.
\begin{definition}[Spyware]\label{def:Spyware}
  \emph{Spyware} is malware that intentionally monitors the \nameref{def:User}'s actions.
  The usual goal is to download ads to display on the user’s system, create pop-up browser windows when certain sites are visited, or use \nameref{def:Covert_Channel}s.
\end{definition}

Most frequently, \nameref{def:Spyware} comes with freeware or shareware programs, but sometimes is included with commercial software.

\begin{definition}[Covert Channel]\label{def:Covert_Channel}
  \emph{Covert channel}s are intended to capture information from the user’s system and return it to a central site.
\end{definition}


%%% Local Variables:
%%% mode: latex
%%% TeX-master: "../../EDAF35-Operating_Systems-Reference_Sheet"
%%% End:
