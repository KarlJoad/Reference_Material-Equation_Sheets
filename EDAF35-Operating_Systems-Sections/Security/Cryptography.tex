\subsection{Cryptography}\label{subsec:Cryptography}
In an isolated computer, the \nameref{def:Operating_System} can reliably determine the sender and recipient of all interprocess communication, since it controls all communication channels in the computer.
In a network of computers, the situation is quite different.
A networked computer receives bits ``from the wire'' with no immediate and reliable way of determining what machine or application sent those bits.
Similarly, the computer sends bits onto the network with no way of knowing who might eventually receive them.
Additionally, there is no way of knowing if an eavesdropper listened to the communication.

It is generally considered infeasible to build a network of any scale in which the source and destination addresses of packets can be trusted in this sense.
Therefore, the only alternative is somehow to eliminate the need to trust the network.
This is the job of cryptography.

Cryptography enables a recipient of a message to verify that the message was created by some computer possessing a certain key.
Similarly, a sender can encode its message so that only a computer with a certain key can decode the message.
Keys are designed so that it is not computationally feasible to derive them from the messages they were used to generate or from any other public information.
Thus, they provide a much more trustworthy means of constraining senders and receivers of messages.

\begin{blackbox}
  For a greater in-depth discussion of cryptography and the mathematics, algorithms, and theory behind what is in use, please refer to \href{file:./EDIN01-Cryptography-Reference_Sheet.pdf}{EDIN01, Cryptography}.
\end{blackbox}


%%% Local Variables:
%%% mode: latex
%%% TeX-master: "../../EDAF35-Operating_Systems-Reference_Sheet"
%%% End:
