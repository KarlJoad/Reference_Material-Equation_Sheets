\subsection{Cryptography}\label{subsec:Cryptography}
In an isolated computer, the \nameref{def:Operating_System} can reliably determine the sender and recipient of all interprocess communication, since it controls all communication channels in the computer.
In a network of computers, the situation is quite different.
A networked computer receives bits ``from the wire'' with no immediate and reliable way of determining what machine or application sent those bits.
Similarly, the computer sends bits onto the network with no way of knowing who might eventually receive them.
Additionally, there is no way of knowing if an eavesdropper listened to the communication.

It is generally considered infeasible to build a network of any scale in which the source and destination addresses of packets can be trusted in this sense.
Therefore, the only alternative is somehow to eliminate the need to trust the network.
This is the job of cryptography.

Cryptography enables a recipient of a message to verify that the message was created by some computer possessing a certain key.
Similarly, a sender can encode its message so that only a computer with a certain key can decode the message.
Keys are designed so that it is not computationally feasible to derive them from the messages they were used to generate or from any other public information.
Thus, they provide a much more trustworthy means of constraining senders and receivers of messages.

\begin{blackbox}
  For a greater in-depth discussion of cryptography and the mathematics, algorithms, and theory behind what is in use, please refer to \href{file:./EDIN01-Cryptography-Reference_Sheet.pdf}{EDIN01, Cryptography}.
\end{blackbox}

\subsubsection{Encryption}\label{subsubsec:Encryption}
Encryption is used frequently in many aspects of modern computing. It is used to send
messages securely across across a network, as well as to protect database data,
files, and even entire disks from having their contents read by unauthorized
entities. An encryption algorithm enables the sender of a message to ensure that
only a computer possessing a certain key can read the message, or ensure that
the writer of data is the only reader of that data.

An encryption algorithm consists of the following components:
\begin{itemize}[noitemsep]
\item A set $\Keyspace$ of keys.
\item A set $\Messages$ of messages.
\item A set $\Ciphertexts$ of ciphertexts.
\item An encrypting function $E : \Keyspace \rightarrow (\Messages \rightarrow \Ciphertexts)$.
  Meaning that for each $k \in \Keyspace$, $E_{k}$ is a function for generating ciphertexts from plaintext messages.
  Both $E$ and $E_{k}$ for any $k$ should be efficiently computable functions.
\item A decrypting function $D : \Keyspace \rightarrow (\Ciphertexts \rightarrow \Messages)$.
  That is, for each $k \in \Keyspace$, $D_{k}$ is a function for generating plaintext messages from ciphertexts.
  Both $D$ and $D_{k}$ for any $k$ should be efficiently computable functions.
\end{itemize}

An encryption algorithm must provide this essential property:
\begin{blackbox}
Given a ciphertext $c \in \Ciphertexts$, a computer can compute $m$ such that $E_{k}(m) = c$ only if it possesses $k$.
\end{blackbox}

Thus, a computer holding $k$ can decrypt ciphertexts to the plaintexts used to produce them, but a computer not holding $k$ cannot decrypt ciphertexts.
Since ciphertexts are generally exposed (for example, sent on a network), it must be infeasible to derive $k$ from the ciphertexts.


%%% Local Variables:
%%% mode: latex
%%% TeX-master: "../../EDAF35-Operating_Systems-Reference_Sheet"
%%% End:
