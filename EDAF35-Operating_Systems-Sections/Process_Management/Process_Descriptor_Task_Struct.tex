\subsection{The Process Descriptor and Task \texttt{struct}}\label{subsec:Process_Descriptor_Task_Struct}
\begin{definition}[Task List]\label{def:Task_List}
  The \nameref{def:Kernel} stores the list of \nameref{def:Process}es in a circular doubly-linked list called the \emph{task list}.
  Each element in the task list is a \nameref{def:Process_Descriptor} of the type \mintinline{c}{struct task_struct}.
\end{definition}

\begin{definition}[Process Descriptor]\label{def:Process_Descriptor}
  The \emph{process descriptor} contains all the information about a specific \nameref{def:Process}, including:
  \begin{itemize}[noitemsep]
  \item Open files
  \item The \nameref{def:Process}’s address space,
  \item Pending signals,
  \item The \nameref{def:Process}’s state,
  \item The \nameref{def:Process}'s priority
  \item The \nameref{def:Process}'s policy
  \item The \nameref{def:Process}'s parent
  \item The \nameref{def:Process}'s id (PID)
  \end{itemize}

  In Linux, the process descriptor is of type \mintinline{c}{struct task_struct}, which is defined in \texttt{<linux/sched.h>}.
\end{definition}

\subsubsection{Allocating the Process Descriptor}\label{subsubsec:Allocate_Process_Descriptor}
Like in any other programming language, the \mintinline{c}{task_struct} record must be initialized somehow.
This is done with the \emph{slab allocator} to provide object reuse and \nameref{def:Cache_Coloring}.

\begin{definition}[Cache Coloring]\label{def:Cache_Coloring}
  \emph{Cache coloring} (also known as page coloring) is the process of attempting to allocate free pages that are contiguous from the CPU cache's point of view, in order to maximize the total number of pages cached by the processor.
  Cache coloring is typically employed by low-level dynamic memory allocation code in the operating system, when mapping virtual memory to physical memory.
  A virtual memory subsystem that lacks cache coloring is less deterministic with regards to cache performance, as differences in page allocation from one program run to the next can lead to large differences in program performance.
\end{definition}

\subsubsection{Storing the Process Descriptor}\label{subsubsec:Storing_Process_Descriptor}
The system identifies \nameref{def:Process}es by a unique Process Identification Value or PID.\@
The PID is a numerical value represented by the opaque type\footnote{``An opaque type is a data type whose physical representation is unknown or irrelevant''~\cite[pg.~26]{LKD3}.} \mintinline{c}{pid_t}, which is typically an \mintinline{c}{int}.
Because of backward compatibility with earlier UNIX and Linux versions, the default maximum value is only 32,768 (that of a \mintinline{c}{(short int)}), although the value can be increased as high as four million (this is controlled in \mintinline{c}{<linux/threads.h>}).
The \nameref{def:Kernel} stores this value as PID inside each \nameref{def:Process_Descriptor}.
This maximum value is important because it is essentially the maximum number of \nameref{def:Process}es that may exist concurrently on the system.

Inside the \nameref{def:Kernel}, tasks are typically referenced directly by a pointer to their \mintinline{c}{task_struct} structure.
In fact, most \nameref{def:Kernel} code that deals with \nameref{def:Process}es works directly with \mintinline{c}{struct task_struct}.
Consequently, it is useful to be able to quickly look up the \nameref{def:Process_Descriptor} of the currently executing task, which is done via the current macro.
This macro must be independently implemented by each architecture.
Some architectures save a pointer to the \mintinline{c}{task_struct} structure of the currently running \nameref{def:Process} in a register, enabling for efficient access.
Other architectures make use of the fact that \mintinline{c}{struct thread_info} is stored on the \nameref{def:Kernel} stack to calculate the location of \mintinline{c}{thread_info} and subsequently the \mintinline{c}{task_struct}.

\subsubsection{Process State}\label{subsubsec:Process_State}
The state field of the process descriptor describes the current condition of the process.
Each process on the system is in exactly one of five different states.
This value is represented by one of five flags:
\begin{enumerate}[noitemsep]
\item \mintinline{c}{TASK_RUNNING}: The process is runnable; it is either currently running or on a runqueue waiting to run.
  This is the only possible state for a \nameref{def:Process} executing in \nameref{def:User}-space; it can also apply to a process in \nameref{def:Kernel}-space that is actively running.
\item \mintinline{c}{TASK_INTERRUPTIBLE}: The \nameref{def:Process} is sleeping (that is, it is blocked), waiting for some condition to exist.
  When this condition exists, the \nameref{def:Kernel} sets the \nameref{def:Process}’s state to \mintinline{c}{TASK_RUNNING}.
  The \nameref{def:Process} also awakes prematurely and becomes runnable if it receives a signal.
\item \mintinline{c}{TASK_UNINTERRUPTIBLE}: This state is identical to \mintinline{c}{TASK_INTERRUPTIBLE} \textbf{except that it does not wake up and become runnable if it receives a signal}.
  This is used in situations where the \nameref{def:Process} must wait without interruption or when the event is expected to occur quite quickly.
  Because the task does not respond to signals in this state, \mintinline{c}{TASK_UNINTERRUPTIBLE} is less often used than \mintinline{c}{TASK_INTERRUPTIBLE}.
\item \mintinline{c}{__TASK_TRACED}: The \nameref{def:Process} is being traced by another \nameref{def:Process}, such as a debugger, via \texttt{ptrace}.
\item \mintinline{c}{__TASK_STOPPED}: \nameref{def:Process} execution has stopped; the task is not running nor is it eligible to run.
  This occurs if the task receives the \texttt{SIGSTOP}, \texttt{SIGTSTP}, \texttt{SIGTTIN}, or \texttt{SIGTTOU} signal or if it receives any signal while it is being debugged.
\end{enumerate}

\subsubsection{Manipulating the Current Process's State}\label{subsubsec:Manipulate_Current_Process_State}
Kernel code often needs to change a process’s state.
The preferred mechanism is using
\begin{minted}[frame=lines,linenos]{c}
set_task_state(task, state); /* set task ‘task’ to state ‘state’ */
\end{minted}

This function sets the given \texttt{task} to the given \texttt{state}.
If applicable, it also provides a memory barrier to force ordering on other processors.
This is only needed on SMP systems.

\subsubsection{Process Context}\label{subsubsec:Process_Context}
Normal program execution occurs in \nameref{def:User}-space.
When a program executes a system call or triggers an exception, it enters \nameref{def:Kernel}-space.
At this point, the \nameref{def:Kernel} is said to be ``executing on behalf of the process'' and is in \nameref{def:Process}-context.
When in process context, the \texttt{current} macro is valid.

\begin{remark*}
  Other than process context there is \nameref{def:Interrupt}-context.
  In interrupt context, the system is not running on behalf of a process but is executing an interrupt handler.
  No \nameref{def:Process} is tied to interrupt handlers.
\end{remark*}

Upon exiting the kernel, the \nameref{def:Process} resumes execution in \nameref{def:User}-space, unless a higher-priority process has become runnable in the interim.
If that happens, the scheduler is invoked to select the higher priority process.
\nameref{def:System_Call}s and exception handlers are well-defined interfaces into the kernel.
A \nameref{def:Process} can begin executing in kernel-space only through one of these interfaces.
All access to the \nameref{def:Kernel} is through these interfaces.

\subsubsection{Process Family Tree}\label{subsubsec:Process_Family_Tree}
All processes are descendants of the init \nameref{def:Process}, whose PID is one.
The kernel starts \texttt{init} in the last step of the boot process.
The \texttt{init} process, in turn, reads the system initscripts and executes more programs, eventually completing the boot process.

Every process on the system has exactly one parent.
Likewise, every process has zero or more children.
Processes that are all direct children of the same parent are called siblings.
The relationship between processes is stored in the process descriptor.
Each \mintinline{c}{task_struct} has a pointer to the parent’s \mintinline{c}{task_struct}, named \texttt{parent}, and a list of children, named \texttt{children}.

%%% Local Variables:
%%% mode: latex
%%% TeX-master: "../../EDAF35-Operating_Systems-Reference_Sheet"
%%% End:
