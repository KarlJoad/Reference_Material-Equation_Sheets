\subsection{Process Termination}\label{subsec:Process_Termination}
When a process terminates, the \nameref{def:Kernel} releases the resources owned by the process and notifies the child’s parent of its demise.
Usually, process destruction is self-induced.
It occurs when the process calls the \mintinline{c}{exit()} \nameref{def:System_Call}.
This can be done either explicitly when it is ready to terminate or implicitly on return from the main subroutine of any program (The C compiler places a call to \mintinline{c}{exit()} after \mintinline{c}{main()} returns).

A process can also terminate involuntarily.
This occurs when the process receives a signal or exception it cannot handle or ignore.


%%% Local Variables:
%%% mode: latex
%%% TeX-master: "../../EDAF35-Operating_Systems-Reference_Sheet"
%%% End:
