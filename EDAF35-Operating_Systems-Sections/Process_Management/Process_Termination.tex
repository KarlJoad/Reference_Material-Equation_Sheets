\subsection{Process Termination}\label{subsec:Process_Termination}
When a process terminates, the \nameref{def:Kernel} releases the resources owned by the process and notifies the child’s parent of its demise.
Usually, process destruction is self-induced.
It occurs when the process calls the \mintinline{c}{exit()} \nameref{def:System_Call}.
This can be done either explicitly when it is ready to terminate or implicitly on return from the main subroutine of any program (The C compiler places a call to \mintinline{c}{exit()} after \mintinline{c}{main()} returns).

A process can also terminate involuntarily.
This occurs when the process receives a signal or exception it cannot handle or ignore.

Regardless of how a process terminates, the bulk of the work is handled by \mintinline{c}{do_exit()}, defined in \mintinline{c}{kernel/exit.c}, which completes a number of chores:
\begin{enumerate}
\item It sets the \mintinline{c}{PF_EXITING} flag in the flags member of the \mintinline{c}{task_struct}.
\item It calls \mintinline{c}{del_timer_sync()} to remove any kernel timers.
  Upon return, it is guaranteed that no timer is queued and that no timer handler is running.
\item If BSD process accounting is enabled, \mintinline{c}{do_exit()} calls \mintinline{c}{acct_update_integrals()} to write out accounting information.
\item It calls \mintinline{c}{exit_mm()} to release the \mintinline{c}{mm_struct} held by this process.
  If no other process is using this address space, i.e.\ the address space is not shared, the \nameref{def:Kernel} then destroys it.
\item It calls \mintinline{c}{exit_sem()}.
  If the process is queued waiting for an IPC semaphore, it is dequeued here.
\item It then calls \mintinline{c}{exit_files()} and \mintinline{c}{exit_fs()} to decrement the usage count of objects related to file descriptors and filesystem data, respectively.
  If either usage counts reach zero, the object is no longer in use by any process, and it is destroyed.
\item It sets the \nameref{def:Process}’s exit code, stored in the \mintinline{c}{exit_code} member of the \mintinline{c}{task_struct}, to the code provided by \mintinline{c}{exit()} or whatever \nameref{def:Kernel} mechanism forced the termination.
  The exit code is stored here for optional retrieval by the parent.
\item It calls \mintinline{c}{exit_notify()} to send signals to the \nameref{def:Process}’s parent, reparents any of the \nameref{def:Process}’s children to another thread in their thread group or the init process, and sets the \nameref{def:Process}’s exit state, stored in \mintinline{c}{exit_state} in the \mintinline{c}{task_struct} structure, to \mintinline{c}{EXIT_ZOMBIE}.
\item \mintinline{c}{do_exit()} calls \mintinline{c}{schedule()} to switch to a new process.
  Because the process is now not schedulable, this is the last code the \nameref{def:Process} will ever execute.
  \mintinline{c}{do_exit()} never returns.
\end{enumerate}

At this point, we have a \nameref{def:Zombie_Process}.

\begin{definition}[Zombie Process]\label{def:Zombie_Process}
  A \emph{zombie process} in Linux is a process that has been completed, but its entry still remains in the process table due to lack of correspondence between the parent and child \nameref{def:Process}es.
  This happens when the \nameref{def:Process} has been terminated, but the \nameref{def:Process_Descriptor} has \textbf{not} be deallocated yet.

  \begin{itemize}[noitemsep]
  \item All objects associated with the \nameref{def:Process} are freed.
    \begin{itemize}[noitemsep]
    \item This assumes that this \nameref{def:Process} was the only one using these objects, i.e.\ no other \nameref{def:Thread}s/\nameref{def:Process}es were using them.
    \end{itemize}
  \item The \nameref{def:Process} is not runnable and no longer has an address space in which to run.
  \item The process is in the \mintinline{c}{EXIT_ZOMBIE} exit state.
  \item The only memory it occupies is its \nameref{def:Kernel} stack, the \mintinline{c}{thread_info} structure, and the \mintinline{c}{task_struct} structure.
  \item The \nameref{def:Process} exists solely to provide information to its parent.
    After the parent retrieves the information, or notifies the \nameref{def:Kernel} that it is uninterested, the remaining memory held by the process is freed and returned to the system for use.
  \end{itemize}
\end{definition}


%%% Local Variables:
%%% mode: latex
%%% TeX-master: "../../EDAF35-Operating_Systems-Reference_Sheet"
%%% End:
