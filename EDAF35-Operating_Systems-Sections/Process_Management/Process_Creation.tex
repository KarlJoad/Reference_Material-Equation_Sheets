\subsection{Process Creation}\label{subsec:Process_Creation}
Most operating systems implement a \texttt{spawn} mechanism to create a new process in a new address space, read in an executable, and begin executing it.
UNIX takes the unusual approach of separating these steps into two distinct functions: \kernelinline{fork()} and \kernelinline{exec()}.
The first, \kernelinline{fork()}, creates a child process that is a copy of the current task.
It differs from the parent only in:
\begin{itemize}[noitemsep]
\item Its PID (which is unique)
\item Its PPID (parent’s PID, which is set to the original process)
\item Certain resources and statistics, such as pending signals, which are not inherited
\end{itemize}

The second function, \kernelinline{exec()}, loads a new executable into the address space and begins executing it.

\subsubsection{Copy-on-Write}\label{subsubsec:Process_Copy_on_Write}
In Linux, \kernelinline{fork()} is implemented through the use of copy-on-write pages.
Copy-on-write (or CoW) is a technique to delay or altogether prevent copying of the data.
Rather than duplicate the process address space, the parent and the child can share a single read-only copy.

The data, however, is marked in such a way that if it is written to, a duplicate is made and each \nameref{def:Process} receives their own unique copy.
Consequently, the duplication of resources occurs only when they are written; until then, they are shared read-only.
This technique delays the copying of each page in the address space until it is actually written to.
In the case that the pages are never written—for example, if \kernelinline{exec()} is called immediately after \kernelinline{fork()}—they never need to be copied.

The only overhead incurred by \kernelinline{fork()} is the duplication of the parent’s page tables and the creation of a unique \nameref{def:Process_Descriptor} for the child.
In the common case that a \nameref{def:Process} executes a new executable image immediately after forking, this optimization prevents the wasted copying of large amounts of data.

\subsubsection{Forking}\label{subsubsec:Forking}
The bulk of the work in forking is handled by \kernelinline{do_fork()}, which is defined in \kernelinline{kernel/fork.c}, by calling \kernelinline{copy_process()} and then starting the process.
The interesting work is done by \kernelinline{copy_process()}:
\begin{enumerate}
\item It calls \kernelinline{dup_task_struct()}, which creates a new kernel stack, \kernelinline{thread_info} structure, and \kernelinline{task_struct} for the new process.
  The new values are identical to those of the current task.
  At this point, the child and parent \nameref{def:Process_Descriptor}s are identical.
\item It then checks that the new child will not exceed the resource limits on the number of processes for the current user.
\item The child needs to differentiate itself from its parent.
  Various members of the \nameref{def:Process_Descriptor} are cleared or set to initial values.
  Members of the \nameref{def:Process_Descriptor} not inherited are primarily statistical information.
  The bulk of the values in \kernelinline{task_struct} remain unchanged.
\item The child’s state is set to \kernelinline{TASK_UNINTERRUPTIBLE} to ensure that it does not yet run.
\item \kernelinline{copy_process()} calls \kernelinline{copy_flags()} to update the flags member of the \kernelinline{task_struct}.
  The \kernelinline{PF_SUPERPRIV} flag, which denotes whether a task used superuser privileges, is cleared.
  The \kernelinline{PF_FORKNOEXEC} flag, which denotes a process that has not called \kernelinline{exec()}, is set.
\item It calls \kernelinline{alloc_pid()} to assign an available PID to the new task.
\item Depending on the flags passed to \kernelinline{clone()}, \kernelinline{copy_process()} either duplicates or shares open \nameref{def:File}s, filesystem information, signal handlers, process address space, and namespace.
  These resources are typically shared between \nameref{def:Thread}s in a given \nameref{def:Process}; otherwise they are unique and copied here.
\item Finally, \kernelinline{copy_process()} cleans up and returns to the caller a pointer to the new child.
\end{enumerate}

Back in \kernelinline{do_fork()}, if \kernelinline{copy_process()} returns successfully, the new child is woken up and run.
Deliberately, the kernel runs the child process first.

%%% Local Variables:
%%% mode: latex
%%% TeX-master: "../../EDAF35-Operating_Systems-Reference_Sheet"
%%% End:
