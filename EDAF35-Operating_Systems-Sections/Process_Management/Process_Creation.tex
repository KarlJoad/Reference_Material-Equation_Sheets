\subsection{Process Creation}\label{subsec:Process_Creation}
Most operating systems implement a \texttt{spawn} mechanism to create a new process in a new address space, read in an executable, and begin executing it.
UNIX takes the unusual approach of separating these steps into two distinct functions: \mintinline{c}{fork()} and \mintinline{c}{exec()}.
The first, \mintinline{c}{fork()}, creates a child process that is a copy of the current task.
It differs from the parent only in:
\begin{itemize}[noitemsep]
\item Its PID (which is unique)
\item Its PPID (parent’s PID, which is set to the original process)
\item Certain resources and statistics, such as pending signals, which are not inherited
\end{itemize}

The second function, \mintinline{c}{exec()}, loads a new executable into the address space and begins executing it.


%%% Local Variables:
%%% mode: latex
%%% TeX-master: "../../EDAF35-Operating_Systems-Reference_Sheet"
%%% End:
