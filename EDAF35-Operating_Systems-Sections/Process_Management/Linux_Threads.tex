\subsection{Linux Implementation of Threads}\label{subsec:Linux_Implementation_Threads}
\nameref{def:Thread}s are a popular modern programming abstraction.
They provide multiple executors within the same program in a shared memory address space.
They can also share open files and other resources.
\nameref{def:Thread}s enable concurrent programming and, on multiple processor systems, true parallelism.

The Linux kernel is unique in that there is no concept of a \nameref{def:Thread}.
Instead, Linux implements all \nameref{def:Thread}s as standard \nameref{def:Process}es.
The Linux kernel does not provide any special scheduling semantics or data structures to represent \nameref{def:Thread}s.
Instead, a \nameref{def:Thread} is merely a \nameref{def:Process} that shares certain resources with other \nameref{def:Process}es.
Each \nameref{def:Thread} has a unique \mintinline{c}{task_struct} and appears to the kernel as a normal \nameref{def:Process} which just happen to share resources, such as an address space, with other \nameref{def:Process}es.

For example, assume you have a \nameref{def:Process} that consists of four \nameref{def:Thread}s.
In Linux, there are simply four \nameref{def:Process}es and thus four normal \mintinline{c}{task_struct} structures.
The four \nameref{def:Process}es are set up to share certain resources.
The result is quite elegant.
However, on systems with explicit \nameref{def:Thread} support, one \nameref{def:Process_Descriptor} might exist that points to the four different \nameref{def:Thread}s.
The \nameref{def:Process_Descriptor} describes the shared resources, such as an address space or open files.
The \nameref{def:Thread}s then describe the resources they alone possess.


%%% Local Variables:
%%% mode: latex
%%% TeX-master: "../../EDAF35-Operating_Systems-Reference_Sheet"
%%% End:
