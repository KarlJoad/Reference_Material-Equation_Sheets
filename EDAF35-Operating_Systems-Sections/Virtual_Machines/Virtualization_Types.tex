\subsection{Types of Virtualization}\label{subsec:Virtualization_Types}
Here, I discuss the ways that virtual environments are created using \nameref{def:Virtualization} and the \nameref{subsec:Building_Blocks} discussed earlier.
Of course, the hardware on which the virtual machines are running can cause great variation in implementation methods.
Throughout this discussion, it is safe to assume that \nameref{def:VMM}s take advantage of hardware assistance when possible.

\subsubsection{Virtual Machine Life Cycle}\label{subsubsec:VM_Life_Cycle}
Whatever the hypervisor type, at the time a virtual machine is created, its creator gives the \nameref{def:VMM} certain parameters.
These parameters usually include:
\begin{itemize}[noitemsep]
\item Number of CPUs.
\item Amount of memory.
\item Networking details.
\item Storage details.
\end{itemize}

For example, a user might want to create a new guest with:
\begin{itemize}[noitemsep]
\item Two \nameref{def:Virtual_CPU}s
\item \SI{4}{\gibi{} \byte{}} of memory
\item \SI{10}{\gibi{} \byte{}} of disk space
\item One network interface that gets its IP address via DHCP
\item Access to the DVD drive
\end{itemize}

In the case of a \nameref{def:Type0_Hypervisor}, the resources are usually dedicated in hardware.
In this situation, if there are not two virtual CPUs available and unallocated, the creation request will fail.
For other hypervisor types, the resources are dedicated or virtualized, depending on the type.
Obviously, an IP address cannot be shared, but the \nameref{def:Virtual_CPU}s are usually multiplexed on the physical CPUs.
Similarly, memory management usually involves allocating more memory to guests than actually exists in physical memory.
This is more complicated than multiplexing memory like we do the CPU.\@

When the \nameref{def:Virtual_Machine} is no longer needed, it can be deleted.
When this happens, the \nameref{def:VMM} first frees up any used disk space and then removes the configuration associated with the virtual machine, essentially ``forgetting'' the virtual machine.
These steps are quite simple compared with building, configuring, running, and removing physical machines.

\subsubsection{Type 0 Hypervisor}\label{subsubsec:Type0_Hypervisor}
\nameref{def:Type0_Hypervisor}s have existed for many years under many names, including ``partitions'' and ``domains''.
They are a hardware feature, and that brings its own positives and negatives.
The feature set of a \nameref{def:Type0_Hypervisor} tends to be smaller than those of the other types because it is implemented in \nameref{def:Hardware}.

\begin{definition}[Type 0 Hypervisor]\label{def:Type0_Hypervisor}
  A \emph{Type 0 Hypervisor} provides the \nameref{def:Virtualization} required to run \nameref{def:Virtual_Machine}s in \nameref{def:Firmware}.
  This means that there is \textbf{NO} abstraction required and no abstraction overhead incurred in this method.
  However, this type of virtualization is also relatively inflexible, as the number of virtual domains that can be created for execution is limited by firmware.
\end{definition}

\nameref{def:Operating_System}s need do nothing special to take advantage of their features.
The \nameref{def:VMM} itself is encoded in the firmware and loaded at boot time.
In turn, it loads the guest images to run in each partition.
Each guest believes that it has dedicated hardware, because the hardware assigned to that \nameref{def:Virtual_Machine} \textbf{is} dedicated, simplifying many implementation details.

I/O presents some difficulty, because it is not easy to dedicate I/O devices to guests if there are not enough.
Either all guests must get their own I/O devices, or the system must provided I/O device sharing.
In these cases, the \nameref{def:Hypervisor} manages shared access or grants all devices to a control partition.
In the \textbf{control partition}, a guest operating system provides services (such as networking) via \nameref{def:Daemon}s to other guests, and the hypervisor routes I/O requests appropriately.


%%% Local Variables:
%%% mode: latex
%%% TeX-master: "../../EDAF35-Operating_Systems-Reference_Sheet"
%%% End:
