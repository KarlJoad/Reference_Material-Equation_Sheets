\subsection{Benefits and Features}\label{subsubsec:VM_Benefits_Features}
Several advantages make \nameref{def:Virtualization} attractive.
Most of them are related to the ability for several different execution environments (\nameref{def:Operating_System}s) to concurrently share the same \nameref{def:Hardware}.

One important advantage of \nameref{def:Virtualization} is that the host system is protected from the \nameref{def:Virtual_Machine}s, just as the virtual machines are protected from each other.
A virus inside a guest \nameref{def:Operating_System} might damage that operating system but is unlikely to affect the host or the other guests.
Because each virtual machine is almost completely isolated from all other virtual machines, there are almost no protection problems.
This isolation can also be a problem, because it is difficult to share legitimate resources.
However, there are 2 approaches handling this problem:
\begin{enumerate}[noitemsep]
\item Share a \nameref{def:File_System} \nameref{def:Volume} and thus to share files.
\item Define a virtual network of \nameref{def:Virtual_Machine}s, each of which can send information over the virtual communications network.
  The network is modeled after physical communication networks but is implemented in software.
\end{enumerate}

One feature common to most \nameref{def:Virtualization} implementations is the ability to \textbf{suspend}, a running \nameref{def:Virtual_Machine}.
Most \nameref{def:Operating_System}s provide this basic feature for \nameref{def:Process}es, but in a virtualized system, the guest operating system \textbf{is} a ``process'', which means it can be suspended.
\nameref{def:VMM}s can go one step further and allow copies and \textbf{snapshots} to be made of the guest.
The copy can be used to create a new VM or to move a VM from one machine to another with its current state intact.
The guest can then resume where it was, as if on its original machine, creating a clone.
The snapshot records a point in time, and the guest can be reset to that point if necessary.
Often, VMMs allow any number snapshots to be taken, so long as storage allows.
These abilities are used to good advantage in virtual environments.


%%% Local Variables:
%%% mode: latex
%%% TeX-master: "../../EDAF35-Operating_Systems-Reference_Sheet"
%%% End:
