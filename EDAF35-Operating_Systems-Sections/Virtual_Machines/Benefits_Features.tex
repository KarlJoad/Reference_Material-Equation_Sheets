\subsection{Benefits and Features}\label{subsubsec:VM_Benefits_Features}
Several advantages make \nameref{def:Virtualization} attractive.
Most of them are related to the ability for several different execution environments (\nameref{def:Operating_System}s) to concurrently share the same \nameref{def:Hardware}.

One important advantage of \nameref{def:Virtualization} is that the host system is protected from the \nameref{def:Virtual_Machine}s, just as the virtual machines are protected from each other.
A virus inside a guest \nameref{def:Operating_System} might damage that operating system but is unlikely to affect the host or the other guests.
Because each virtual machine is almost completely isolated from all other virtual machines, there are almost no protection problems.
This isolation can also be a problem, because it is difficult to share legitimate resources.
However, there are 2 approaches handling this problem:
\begin{enumerate}[noitemsep]
\item Share a \nameref{def:File_System} \nameref{def:Volume} and thus to share files.
\item Define a virtual network of \nameref{def:Virtual_Machine}s, each of which can send information over the virtual communications network.
  The network is modeled after physical communication networks but is implemented in software.
\end{enumerate}

One feature common to most \nameref{def:Virtualization} implementations is the ability to \textbf{suspend}, a running \nameref{def:Virtual_Machine}.
Most \nameref{def:Operating_System}s provide this basic feature for \nameref{def:Process}es, but in a virtualized system, the guest operating system \textbf{is} a ``process'', which means it can be suspended.
\nameref{def:VMM}s can go one step further and allow copies and \textbf{snapshots} to be made of the guest.
The copy can be used to create a new VM or to move a VM from one machine to another with its current state intact.
The guest can then resume where it was, as if on its original machine, creating a clone.
The snapshot records a point in time, and the guest can be reset to that point if necessary.
Often, VMMs allow any number snapshots to be taken, so long as storage allows.
These abilities are used to good advantage in virtual environments.

A \nameref{def:Virtual_Machine} system is a perfect vehicle for low-level development.
For example, changing an \nameref{def:Operating_System} is a difficult task.
Operating systems are large and complex programs, and a change in one part may cause obscure bugs to appear in some other part.
The power of the operating system makes changing it particularly dangerous.
Because the operating system executes in kernel mode, a wrong change in a pointer could cause an error that would destroy the entire file system.

Another benefits is that system programmers can be given their own \nameref{def:Virtual_Machine}, and development is done on the virtual machine instead of on a physical machine.
When a deployment happens, normal system operation is disrupted only when a completed and tested change is ready to be put into production.
Another advantage of virtual machines for developers is that multiple \nameref{def:Operating_System}s can run concurrently on the developer’s workstation.
This virtualized workstation allows for rapid porting and testing of programs in varying environments.
In addition, multiple versions of a program can run, each in its own isolated operating system, within one system.

A major advantage of \nameref{def:Virtual_Machine}s in production data-center use is system \nameref{def:Consolidation}.

\begin{definition}[Consolidation]\label{def:Consolidation}
  \emph{Consolidation} involves taking physically separate systems and running them in \nameref{def:Virtual_Machine}s on one system.
  These physical-to-virtual conversions typically result in a maximization of resource usage, since many separate lightly used systems can be combined to create one more heavily used system.
\end{definition}

One of the tools that make this possible is templating, where one standard \nameref{def:Virtual_Machine} image, including an
installed and configured guest \nameref{def:Operating_System} and applications, is saved and used as a source for multiple running VMs.
Other features include managing the patching of all guests, backing up and restoring the guests, and monitoring their resource use.
This batch management allows a single system administrator to handle many times the regular amount of physical servers.

Some VMMs include a live migration feature that moves a running guest from one physical server to another without interrupting its operation or active network connections.
If a server is overloaded, live migration can thus free resources on the source host while not disrupting the guest.
Similarly, when \nameref{def:VM_Host} \nameref{def:Hardware} must be repaired or upgraded, guests can be migrated to other servers, the evacuated host can be maintained, and then the guests can be migrated back.
This operation occurs without downtime and without interruption to users.

%%% Local Variables:
%%% mode: latex
%%% TeX-master: "../../EDAF35-Operating_Systems-Reference_Sheet"
%%% End:
