\subsection{Building Blocks}\label{subsec:Building_Blocks}
Although the virtual machine concept is useful, it is difficult to implement.
Much work is required to provide an exact duplicate of the underlying machine.
This is especially a challenge on dual-mode systems, where the underlying machine has only user mode and kernel mode.

\begin{remark*}
  Note that these building blocks are not required by \nameref{def:Type0_Hypervisor}s.
\end{remark*}

The ability to virtualize heavily depends on the hardware features provided by the CPU.\@
If the features are sufficient, then it is possible to write a \nameref{def:VMM} that provides a guest environment.
Otherwise, \nameref{def:Virtualization} is impossible.

If it is possible to virtualize a system, the \nameref{def:VMM}s use several techniques to implement it, including \nameref{subsubsec:Trap_and_Emulate} and \nameref{subsubsec:Binary_Translation}.

One important concept found in most virtualization options is the implementation of a \nameref{def:Virtual_CPU}.

\begin{definition}[Virtual CPU]\label{def:Virtual_CPU}\label{def:VCPU}
  The \emph{Virtual CPU} (\emph{VCPU}) does not execute code.
  Rather, it represents the state of the guest's CPU as the guest machine believes it to be.
  For each guest, the VMM maintains a VCPU representing that guest's current CPU state, much like a \nameref{def:Process_Control_Block}.
  When the guest is \nameref{def:Context_Switch}ed onto a physical CPU by the \nameref{def:VMM}, information from the VCPU is used to load the right context, much as a general-purpose \nameref{def:Operating_System} would use the PCB.\@
\end{definition}


%%% Local Variables:
%%% mode: latex
%%% TeX-master: "../../EDAF35-Operating_Systems-Reference_Sheet"
%%% End:
