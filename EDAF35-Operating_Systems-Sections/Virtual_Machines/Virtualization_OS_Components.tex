\subsection{Virtualization and OS Components}\label{subsec:Virtualization_OS_Components}
Here, I discuss how the \nameref{def:VMM} provides core \nameref{def:Operating_System} functions like \nameref{subsec:Scheduling}, I/O, and memory management.

\subsubsection{CPU Scheduling}\label{subsubsec:VM_CPU_Scheduling}
A system with \nameref{def:Virtualization}, frequently acts like a multiprocessor system.
The virtualization software presents one or more \nameref{def:Virtual_CPU}s to each of the \nameref{def:Virtual_Machine}s running on the system and then schedules the use of the physical CPUs among the virtual machines.
The significant variations among virtualization technologies make it difficult to generalize the effect of virtualization on scheduling.


%%% Local Variables:
%%% mode: latex
%%% TeX-master: "../../EDAF35-Operating_Systems-Reference_Sheet"
%%% End:
