\section{Process Management}\label{sec:Process_Management}
\begin{definition}[Process]\label{def:Process}
  A \emph{process} is a \nameref{def:Program} in the midst of execution, and all its related resources.
  In fact, two or more processes can exist that are executing the same program.
  Processes are, however, more than just the executing program code (often called the text section in Unix).
  They also include a set of resources such as open files and pending signals, internal \nameref{def:Kernel} data, processor state, a memory address space with one or more memory mappings, one or more \nameref{def:Thread}s of execution, and a data section containing global variables.

  Processes, in effect, are the living result of running program code.
\end{definition}


%%% Local Variables:
%%% mode: latex
%%% TeX-master: "../EDAF35-Operating_Systems-Reference_Sheet"
%%% End:
