\section{Process Management}\label{sec:Process_Management}
\begin{definition}[Process]\label{def:Process}
  A \emph{process} is a \nameref{def:Program} in the midst of execution, and all its related resources.
  In fact, two or more processes can exist that are executing the same program.
  Processes are, however, more than just the executing program code (often called the text section in Unix).
  They also include a set of resources such as open files and pending signals, internal \nameref{def:Kernel} data, processor state, a memory address space with one or more memory mappings, one or more \nameref{def:Thread}s of execution, and a data section containing global variables.

  Processes, in effect, are the living result of running program code.
\end{definition}

\begin{definition}[Program]\label{def:Program}
  A \emph{program} is object code stored on some media, typically as a \nameref{def:File}.
  These contain the instructions that the processor will execute when the program is running as a \nameref{def:Process}.
  These instructions are stored in what is called the \emph{text section} of the program.
  It also contains statically allocated information, such as \kernelinline{static} variables.
\end{definition}

Occassionally, \nameref{def:Thread}s can be subject to \nameref{def:Preemption}.

\begin{definition}[Preemption]\label{def:Preemption}
  \emph{Preemption} is the act of temporarily interrupting a task being carried out by a computer system, without requiring its cooperation, and with the intention of resuming the task at a later time.
  Such changes of the executed task are known as context switches.
  It is normally carried out by a privileged task or part of the system known as a preemptive scheduler, which has the power to preempt, or interrupt, and later resume, other tasks in the system.

  In any given system design, some operations performed by the system may not be preemptible.
  This usually applies to \nameref{def:Kernel} functions and service interrupts which, if not permitted to run to completion, would tend to produce race conditions resulting in deadlock.
\end{definition}

On modern \nameref{def:Operating_System}s, \nameref{def:Process}es provide two virtualizations:
\begin{enumerate}[noitemsep]
\item a virtualized processor
  \begin{itemize}[noitemsep]
  \item The virtual processor gives \emph{this} \nameref{def:Process} the illusion that it alone monopolizes the system, despite possibly sharing the processor among hundreds of other processes.
\end{itemize}
\item Virtual memory lets the process allocate and manage memory as if it alone owned all the memory in the system.
  \begin{itemize}[noitemsep]
  \item \nameref{def:Thread}s share the virtual memory abstraction, whereas each receives its own virtualized processor.
  \end{itemize}
\end{enumerate}

\subsection{The Process Life Cycle}\label{subsec:Process_Life_Cycle}
A \nameref{def:Process} begins its life when, the \kernelinline{fork()} \nameref{def:System_Call} is called.
This creates a new \nameref{def:Process} by duplicating an existing one.
The \nameref{def:Process} that calls \kernelinline{fork()} is the \textbf{parent}, whereas the new \nameref{def:Process} is the \textbf{child}.
The \kernelinline{fork()} \nameref{def:System_Call} returns from the kernel twice: once in the \textbf{parent} and once in the newborn \textbf{child}.
The parent resumes execution and the child starts execution at the same place, where the call to \kernelinline{fork()} returns.

Often, immediately after a \texttt{fork} it is desirable to execute a new, different \nameref{def:Program}.
The \kernelinline{exec()} family of function calls creates a new address space and loads a new program into it.

Finally, a program exits via the \kernelinline{exit()} \nameref{def:System_Call}.
This function terminates the \nameref{def:Process} and frees all its resources.
A parent \nameref{def:Process} can inquire about the status of a terminated child via the \kernelinline{wait4()} \nameref{def:System_Call}, which enables a \nameref{def:Process} to wait for the termination of a specific \nameref{def:Process}.
When a \nameref{def:Process} exits, it is placed into a special zombie state that represents terminated \nameref{def:Process}es until the parent calls \kernelinline{wait()} or \kernelinline{waitpid()}.

\subsection{The Process Descriptor and Task \texttt{struct}}\label{subsec:Process_Descriptor_Task_Struct}
\begin{definition}[Task List]\label{def:Task_List}
  The \nameref{def:Kernel} stores the list of \nameref{def:Process}es in a circular doubly-linked list called the \emph{task list}.
  Each element in the task list is a \nameref{def:Process_Descriptor} of the type \mintinline{c}{struct task_struct}.
\end{definition}

\begin{definition}[Process Descriptor]\label{def:Process_Descriptor}
  The \emph{process descriptor} contains all the information about a specific \nameref{def:Process}, including:
  \begin{itemize}[noitemsep]
  \item Open files
  \item The \nameref{def:Process}’s address space,
  \item Pending signals,
  \item The \nameref{def:Process}’s state,
  \item The \nameref{def:Process}'s priority
  \item The \nameref{def:Process}'s policy
  \item The \nameref{def:Process}'s parent
  \item The \nameref{def:Process}'s id (PID)
  \end{itemize}

  In Linux, the process descriptor is of type \mintinline{c}{struct task_struct}, which is defined in \texttt{<linux/sched.h>}.
\end{definition}

\subsubsection{Allocating the Process Descriptor}\label{subsubsec:Allocate_Process_Descriptor}
Like in any other programming language, the \mintinline{c}{task_struct} record must be initialized somehow.
This is done with the \emph{slab allocator} to provide object reuse and \nameref{def:Cache_Coloring}.

\begin{definition}[Cache Coloring]\label{def:Cache_Coloring}
  \emph{Cache coloring} (also known as page coloring) is the process of attempting to allocate free pages that are contiguous from the CPU cache's point of view, in order to maximize the total number of pages cached by the processor.
  Cache coloring is typically employed by low-level dynamic memory allocation code in the operating system, when mapping virtual memory to physical memory.
  A virtual memory subsystem that lacks cache coloring is less deterministic with regards to cache performance, as differences in page allocation from one program run to the next can lead to large differences in program performance.
\end{definition}

\subsubsection{Storing the Process Descriptor}\label{subsubsec:Storing_Process_Descriptor}
The system identifies \nameref{def:Process}es by a unique Process Identification Value or PID.\@
The PID is a numerical value represented by the opaque type\footnote{``An opaque type is a data type whose physical representation is unknown or irrelevant''~\cite[pg.~26]{LKD3}.} \mintinline{c}{pid_t}, which is typically an \mintinline{c}{int}.
Because of backward compatibility with earlier UNIX and Linux versions, the default maximum value is only 32,768 (that of a \mintinline{c}{(short int)}), although the value can be increased as high as four million (this is controlled in \mintinline{c}{<linux/threads.h>}).
The \nameref{def:Kernel} stores this value as PID inside each \nameref{def:Process_Descriptor}.
This maximum value is important because it is essentially the maximum number of \nameref{def:Process}es that may exist concurrently on the system.

Inside the \nameref{def:Kernel}, tasks are typically referenced directly by a pointer to their \mintinline{c}{task_struct} structure.
In fact, most \nameref{def:Kernel} code that deals with \nameref{def:Process}es works directly with \mintinline{c}{struct task_struct}.
Consequently, it is useful to be able to quickly look up the \nameref{def:Process_Descriptor} of the currently executing task, which is done via the current macro.
This macro must be independently implemented by each architecture.
Some architectures save a pointer to the \mintinline{c}{task_struct} structure of the currently running \nameref{def:Process} in a register, enabling for efficient access.
Other architectures make use of the fact that \mintinline{c}{struct thread_info} is stored on the \nameref{def:Kernel} stack to calculate the location of \mintinline{c}{thread_info} and subsequently the \mintinline{c}{task_struct}.

\subsubsection{Process State}\label{subsubsec:Process_State}
The state field of the process descriptor describes the current condition of the process.
Each process on the system is in exactly one of five different states.
This value is represented by one of five flags:
\begin{enumerate}[noitemsep]
\item \mintinline{c}{TASK_RUNNING}: The process is runnable; it is either currently running or on a runqueue waiting to run.
  This is the only possible state for a \nameref{def:Process} executing in \nameref{def:User}-space; it can also apply to a process in \nameref{def:Kernel}-space that is actively running.
\item \mintinline{c}{TASK_INTERRUPTIBLE}: The \nameref{def:Process} is sleeping (that is, it is blocked), waiting for some condition to exist.
  When this condition exists, the \nameref{def:Kernel} sets the \nameref{def:Process}’s state to \mintinline{c}{TASK_RUNNING}.
  The \nameref{def:Process} also awakes prematurely and becomes runnable if it receives a signal.
\item \mintinline{c}{TASK_UNINTERRUPTIBLE}: This state is identical to \mintinline{c}{TASK_INTERRUPTIBLE} \textbf{except that it does not wake up and become runnable if it receives a signal}.
  This is used in situations where the \nameref{def:Process} must wait without interruption or when the event is expected to occur quite quickly.
  Because the task does not respond to signals in this state, \mintinline{c}{TASK_UNINTERRUPTIBLE} is less often used than \mintinline{c}{TASK_INTERRUPTIBLE}.
\item \mintinline{c}{__TASK_TRACED}: The \nameref{def:Process} is being traced by another \nameref{def:Process}, such as a debugger, via \texttt{ptrace}.
\item \mintinline{c}{__TASK_STOPPED}: \nameref{def:Process} execution has stopped; the task is not running nor is it eligible to run.
  This occurs if the task receives the \texttt{SIGSTOP}, \texttt{SIGTSTP}, \texttt{SIGTTIN}, or \texttt{SIGTTOU} signal or if it receives any signal while it is being debugged.
\end{enumerate}

\subsubsection{Manipulating the Current Process's State}\label{subsubsec:Manipulate_Current_Process_State}
Kernel code often needs to change a process’s state.
The preferred mechanism is using
\begin{minted}[frame=lines,linenos]{c}
set_task_state(task, state); /* set task ‘task’ to state ‘state’ */
\end{minted}

This function sets the given \texttt{task} to the given \texttt{state}.
If applicable, it also provides a memory barrier to force ordering on other processors.
This is only needed on SMP systems.

\subsubsection{Process Context}\label{subsubsec:Process_Context}
Normal program execution occurs in \nameref{def:User}-space.
When a program executes a system call or triggers an exception, it enters \nameref{def:Kernel}-space.
At this point, the \nameref{def:Kernel} is said to be ``executing on behalf of the process'' and is in \nameref{def:Process}-context.
When in process context, the \texttt{current} macro is valid.

\begin{remark*}
  Other than process context there is \nameref{def:Interrupt}-context.
  In interrupt context, the system is not running on behalf of a process but is executing an interrupt handler.
  No \nameref{def:Process} is tied to interrupt handlers.
\end{remark*}

Upon exiting the kernel, the \nameref{def:Process} resumes execution in \nameref{def:User}-space, unless a higher-priority process has become runnable in the interim.
If that happens, the scheduler is invoked to select the higher priority process.
\nameref{def:System_Call}s and exception handlers are well-defined interfaces into the kernel.
A \nameref{def:Process} can begin executing in kernel-space only through one of these interfaces.
All access to the \nameref{def:Kernel} is through these interfaces.

\subsubsection{Process Family Tree}\label{subsubsec:Process_Family_Tree}
All processes are descendants of the init \nameref{def:Process}, whose PID is one.
The kernel starts \texttt{init} in the last step of the boot process.
The \texttt{init} process, in turn, reads the system initscripts and executes more programs, eventually completing the boot process.

Every process on the system has exactly one parent.
Likewise, every process has zero or more children.
Processes that are all direct children of the same parent are called siblings.
The relationship between processes is stored in the process descriptor.
Each \mintinline{c}{task_struct} has a pointer to the parent’s \mintinline{c}{task_struct}, named \texttt{parent}, and a list of children, named \texttt{children}.

%%% Local Variables:
%%% mode: latex
%%% TeX-master: "../../EDAF35-Operating_Systems-Reference_Sheet"
%%% End:


\subsection{Process Creation}\label{subsec:Process_Creation}
Most operating systems implement a \texttt{spawn} mechanism to create a new process in a new address space, read in an executable, and begin executing it.
UNIX takes the unusual approach of separating these steps into two distinct functions: \mintinline{c}{fork()} and \mintinline{c}{exec()}.
The first, \mintinline{c}{fork()}, creates a child process that is a copy of the current task.
It differs from the parent only in:
\begin{itemize}[noitemsep]
\item Its PID (which is unique)
\item Its PPID (parent’s PID, which is set to the original process)
\item Certain resources and statistics, such as pending signals, which are not inherited
\end{itemize}

The second function, \mintinline{c}{exec()}, loads a new executable into the address space and begins executing it.

\subsubsection{Copy-on-Write}\label{subsubsec:Process_Copy_on_Write}
In Linux, \mintinline{c}{fork()} is implemented through the use of copy-on-write pages.
Copy-on-write (or CoW) is a technique to delay or altogether prevent copying of the data.
Rather than duplicate the process address space, the parent and the child can share a single read-only copy.

The data, however, is marked in such a way that if it is written to, a duplicate is made and each \nameref{def:Process} receives their own unique copy.
Consequently, the duplication of resources occurs only when they are written; until then, they are shared read-only.
This technique delays the copying of each page in the address space until it is actually written to.
In the case that the pages are never written—for example, if \mintinline{c}{exec()} is called immediately after \mintinline{c}{fork()}—they never need to be copied.

The only overhead incurred by \mintinline{c}{fork()} is the duplication of the parent’s page tables and the creation of a unique \nameref{def:Process_Descriptor} for the child.
In the common case that a \nameref{def:Process} executes a new executable image immediately after forking, this optimization prevents the wasted copying of large amounts of data.


%%% Local Variables:
%%% mode: latex
%%% TeX-master: "../../EDAF35-Operating_Systems-Reference_Sheet"
%%% End:


\subsection{Process Termination}\label{subsec:Process_Termination}
When a process terminates, the \nameref{def:Kernel} releases the resources owned by the process and notifies the child’s parent of its demise.
Usually, process destruction is self-induced.
It occurs when the process calls the \mintinline{c}{exit()} \nameref{def:System_Call}.
This can be done either explicitly when it is ready to terminate or implicitly on return from the main subroutine of any program (The C compiler places a call to \mintinline{c}{exit()} after \mintinline{c}{main()} returns).

A process can also terminate involuntarily.
This occurs when the process receives a signal or exception it cannot handle or ignore.

Regardless of how a process terminates, the bulk of the work is handled by \mintinline{c}{do_exit()}, defined in \mintinline{c}{kernel/exit.c}, which completes a number of chores:
\begin{enumerate}
\item It sets the \mintinline{c}{PF_EXITING} flag in the flags member of the \mintinline{c}{task_struct}.
\item It calls \mintinline{c}{del_timer_sync()} to remove any kernel timers.
  Upon return, it is guaranteed that no timer is queued and that no timer handler is running.
\item If BSD process accounting is enabled, \mintinline{c}{do_exit()} calls \mintinline{c}{acct_update_integrals()} to write out accounting information.
\item It calls \mintinline{c}{exit_mm()} to release the \mintinline{c}{mm_struct} held by this process.
  If no other process is using this address space, i.e.\ the address space is not shared, the \nameref{def:Kernel} then destroys it.
\item It calls \mintinline{c}{exit_sem()}.
  If the process is queued waiting for an IPC semaphore, it is dequeued here.
\item It then calls \mintinline{c}{exit_files()} and \mintinline{c}{exit_fs()} to decrement the usage count of objects related to file descriptors and filesystem data, respectively.
  If either usage counts reach zero, the object is no longer in use by any process, and it is destroyed.
\item It sets the \nameref{def:Process}’s exit code, stored in the \mintinline{c}{exit_code} member of the \mintinline{c}{task_struct}, to the code provided by \mintinline{c}{exit()} or whatever \nameref{def:Kernel} mechanism forced the termination.
  The exit code is stored here for optional retrieval by the parent.
\item It calls \mintinline{c}{exit_notify()} to send signals to the \nameref{def:Process}’s parent, reparents any of the \nameref{def:Process}’s children to another thread in their thread group or the init process, and sets the \nameref{def:Process}’s exit state, stored in \mintinline{c}{exit_state} in the \mintinline{c}{task_struct} structure, to \mintinline{c}{EXIT_ZOMBIE}.
\item \mintinline{c}{do_exit()} calls \mintinline{c}{schedule()} to switch to a new process.
  Because the process is now not schedulable, this is the last code the \nameref{def:Process} will ever execute.
  \mintinline{c}{do_exit()} never returns.
\end{enumerate}

At this point, we have a \nameref{def:Zombie_Process}.

\begin{definition}[Zombie Process]\label{def:Zombie_Process}
  A \emph{zombie process} in Linux is a process that has been completed, but its entry still remains in the process table due to lack of correspondence between the parent and child \nameref{def:Process}es.
  This happens when the \nameref{def:Process} has been terminated, but the \nameref{def:Process_Descriptor} has \textbf{not} be deallocated yet.

  \begin{itemize}[noitemsep]
  \item All objects associated with the \nameref{def:Process} are freed.
    \begin{itemize}[noitemsep]
    \item This assumes that this \nameref{def:Process} was the only one using these objects, i.e.\ no other \nameref{def:Thread}s/\nameref{def:Process}es were using them.
    \end{itemize}
  \item The \nameref{def:Process} is not runnable and no longer has an address space in which to run.
  \item The process is in the \mintinline{c}{EXIT_ZOMBIE} exit state.
  \item The only memory it occupies is its \nameref{def:Kernel} stack, the \mintinline{c}{thread_info} structure, and the \mintinline{c}{task_struct} structure.
  \item The \nameref{def:Process} exists solely to provide information to its parent.
    After the parent retrieves the information, or notifies the \nameref{def:Kernel} that it is uninterested, the remaining memory held by the process is freed and returned to the system for use.
  \end{itemize}
\end{definition}


%%% Local Variables:
%%% mode: latex
%%% TeX-master: "../../EDAF35-Operating_Systems-Reference_Sheet"
%%% End:


%%% Local Variables:
%%% mode: latex
%%% TeX-master: "../EDAF35-Operating_Systems-Reference_Sheet"
%%% End:
