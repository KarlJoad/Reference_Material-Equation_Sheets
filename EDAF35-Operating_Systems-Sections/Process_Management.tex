\section{Process Management}\label{sec:Process_Management}
\begin{definition}[Process]\label{def:Process}
  A \emph{process} is a \nameref{def:Program} in the midst of execution, and all its related resources.
  In fact, two or more processes can exist that are executing the same program.
  Processes are, however, more than just the executing program code (often called the text section in Unix).
  They also include a set of resources such as open files and pending signals, internal \nameref{def:Kernel} data, processor state, a memory address space with one or more memory mappings, one or more \nameref{def:Thread}s of execution, and a data section containing global variables.

  Processes, in effect, are the living result of running program code.
\end{definition}

\begin{definition}[Program]\label{def:Program}
  A \emph{program} is object code stored on some media, typically as a \nameref{def:File}.
  These contain the instructions that the processor will execute when the program is running as a \nameref{def:Process}.
  These instructions are stored in what is called the \emph{text section} of the program.
  It also contains statically allocated information, such as \mintinline{c}{static} variables.
\end{definition}

\begin{definition}[Thread]\label{def:Thread}
  \emph{Threads of execution}, often shortened to \emph{threads}, are the objects of activity within the process.
  Each thread includes a unique program counter, process stack, and set of processor \nameref{def:Register}s.
  The \nameref{def:Kernel} schedules individual threads, not processes.

  In traditional UNIX systems, each process consists of one thread.
  In modern systems, however, multithreaded programs are common.

  \begin{remark}[Threads in Linux]\label{rmk:Linux_Threads}
    Linux has a unique implementation of threads; it does not differentiate between threads and processes.
    To Linux, a thread is just a special kind of process.
  \end{remark}
\end{definition}


%%% Local Variables:
%%% mode: latex
%%% TeX-master: "../EDAF35-Operating_Systems-Reference_Sheet"
%%% End:
