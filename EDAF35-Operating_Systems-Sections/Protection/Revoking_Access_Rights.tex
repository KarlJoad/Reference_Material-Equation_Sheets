\subsection{Revoking Access Rights}\label{subsec:Revoking_Access_Rights}
In a dynamic protection system, we may sometimes need to revoke access rights to objects shared by different users.
This raises various questions about revocation that must be answered.
These include:
\begin{itemize}[noitemsep]
\item \textbf{Immediate versus delayed}.
  \begin{itemize}[noitemsep]
  \item Does revocation occur immediately, or is it delayed?
  \item If revocation is delayed, can we find out when it will take place?
  \end{itemize}
\item \textbf{Selective versus general}.
  \begin{itemize}[noitemsep]
  \item When an access right to an object is revoked, does it affect all the users who have an access right to that object, or can we specify a select group of users whose access rights should be revoked?
  \end{itemize}
\item \textbf{Partial versus total}.
  \begin{itemize}[noitemsep]
  \item Can a subset of the rights associated with an object be revoked, or must we revoke all access rights for this object?
  \end{itemize}
\item \textbf{Temporary versus permanent}.
  \begin{itemize}[noitemsep]
  \item Can access be revoked permanently (that is, the revoked access right will \textbf{never} again be available), or can access be revoked and later be obtained again?
  \end{itemize}
\end{itemize}

With an \nameref{def:Access_List} scheme, revocation is easy.
The access list is searched for any \nameref{def:Access_Right}s to be revoked, and they are deleted from the list.
Revocation can be: immediate or delayed, general or selective, total or partial, and permanent or temporary.


%%% Local Variables:
%%% mode: latex
%%% TeX-master: "../../EDAF35-Operating_Systems-Reference_Sheet"
%%% End:
