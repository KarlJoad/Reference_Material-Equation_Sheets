\subsection{Revoking Access Rights}\label{subsec:Revoking_Access_Rights}
In a dynamic protection system, we may sometimes need to revoke access rights to objects shared by different users.
This raises various questions about revocation that must be answered.
These include:
\begin{itemize}[noitemsep]
\item \textbf{Immediate versus delayed}.
  \begin{itemize}[noitemsep]
  \item Does revocation occur immediately, or is it delayed?
  \item If revocation is delayed, can we find out when it will take place?
  \end{itemize}
\item \textbf{Selective versus general}.
  \begin{itemize}[noitemsep]
  \item When an access right to an object is revoked, does it affect all the users who have an access right to that object, or can we specify a select group of users whose access rights should be revoked?
  \end{itemize}
\item \textbf{Partial versus total}.
  \begin{itemize}[noitemsep]
  \item Can a subset of the rights associated with an object be revoked, or must we revoke all access rights for this object?
  \end{itemize}
\item \textbf{Temporary versus permanent}.
  \begin{itemize}[noitemsep]
  \item Can access be revoked permanently (that is, the revoked access right will \textbf{never} again be available), or can access be revoked and later be obtained again?
  \end{itemize}
\end{itemize}

With an \nameref{def:Access_List} scheme, revocation is easy.
The access list is searched for any \nameref{def:Access_Right}s to be revoked, and they are deleted from the list.
Revocation can be: immediate or delayed, general or selective, total or partial, and permanent or temporary.

Capabilities present a much more difficult revocation problem.
Since the capabilities are distributed throughout the system, we must find them before we can revoke them.
Schemes that implement revocation for capabilities must include the following:
\begin{itemize}[noitemsep]
\item \nameref{subsubsec:Capability_Revokation_Reacquisition}
\item \nameref{subsubsec:Capability_Revokation_Back_Pointers}
\item \nameref{subsubsec:Capability_Revokation_Indirection}
\item \nameref{subsubsec:Capability_Revokation_Keys}
\end{itemize}

\subsubsection{Reacquisition}\label{subsubsec:Capability_Revokation_Reacquisition}
Periodically, capabilities are deleted from each domain.
If a \nameref{def:Process} wants to use a capability, it may find that that capability has been deleted.
The process may then attempt to reacquire the capability.
If access has been revoked, the process will not be able to reacquire the capability.


%%% Local Variables:
%%% mode: latex
%%% TeX-master: "../../EDAF35-Operating_Systems-Reference_Sheet"
%%% End:
