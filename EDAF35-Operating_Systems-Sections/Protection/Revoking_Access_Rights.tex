\subsection{Revoking Access Rights}\label{subsec:Revoking_Access_Rights}
In a dynamic protection system, we may sometimes need to revoke access rights to objects shared by different users.
This raises various questions about revocation that must be answered.
These include:
\begin{itemize}[noitemsep]
\item \textbf{Immediate versus delayed}.
  \begin{itemize}[noitemsep]
  \item Does revocation occur immediately, or is it delayed?
  \item If revocation is delayed, can we find out when it will take place?
  \end{itemize}
\item \textbf{Selective versus general}.
  \begin{itemize}[noitemsep]
  \item When an access right to an object is revoked, does it affect all the users who have an access right to that object, or can we specify a select group of users whose access rights should be revoked?
  \end{itemize}
\item \textbf{Partial versus total}.
  \begin{itemize}[noitemsep]
  \item Can a subset of the rights associated with an object be revoked, or must we revoke all access rights for this object?
  \end{itemize}
\item \textbf{Temporary versus permanent}.
  \begin{itemize}[noitemsep]
  \item Can access be revoked permanently (that is, the revoked access right will \textbf{never} again be available), or can access be revoked and later be obtained again?
  \end{itemize}
\end{itemize}

With an \nameref{def:Access_List} scheme, revocation is easy.
The access list is searched for any \nameref{def:Access_Right}s to be revoked, and they are deleted from the list.
Revocation can be: immediate or delayed, general or selective, total or partial, and permanent or temporary.

Capabilities present a much more difficult revocation problem.
Since the capabilities are distributed throughout the system, we must find them before we can revoke them.
Schemes that implement revocation for capabilities must include the following:
\begin{itemize}[noitemsep]
\item \nameref{subsubsec:Capability_Revokation_Reacquisition}
\item \nameref{subsubsec:Capability_Revokation_Back_Pointers}
\item \nameref{subsubsec:Capability_Revokation_Indirection}
\item \nameref{subsubsec:Capability_Revokation_Keys}
\end{itemize}

\subsubsection{Reacquisition}\label{subsubsec:Capability_Revokation_Reacquisition}
Periodically, capabilities are deleted from each domain.
If a \nameref{def:Process} wants to use a capability, it may find that that capability has been deleted.
The process may then attempt to reacquire the capability.
If access has been revoked, the process will not be able to reacquire the capability.

\subsubsection{Back-pointers}\label{subsubsec:Capability_Revokation_Back_Pointers}
A list of pointers is maintained with each object, pointing to all capabilities associated with that object.
When revocation is required, we can follow these pointers, changing the capabilities as necessary.
This is quite general, but its implementation is costly because of all the pointer traversal.

\subsubsection{Indirection}\label{subsubsec:Capability_Revokation_Indirection}
Each capability points \textbf{indirectly} through its unique entry in the global table, and that entry points to the object.
Revocation is done by searching the global table for the desired entry and deleting it.
Then, when an access is attempted, the capability is found to point to an illegal table entry.
Table entries can then be reused for other capabilities, since both the capability and the table entry contain the unique name of the object.
The object for a capability and its table entry must match.
It does not allow selective revocation.

\subsubsection{Keys}\label{subsubsec:Capability_Revokation_Keys}
A key is a unique bit pattern that can be associated with a capability.
This key is defined when the capability is created, and it can be neither modified nor inspected by the \nameref{def:Process} that owns the capability.
A master key is associated with each object; it can be defined or replaced with the \texttt{set-key} operation.
When a capability is created, the current value of the master key is associated with the capability.
When the capability is exercised, its key is compared with the master key.
If the keys match, the operation is allowed to continue; otherwise, an exception condition is raised.

Revocation replaces the master key with a new value using the \texttt{set-key} operation again.
This invalidates all previous capabilities for this object.
This scheme does not allow selective revocation, since only one master key is associated with each object.
If we associate a list of keys with each object, then selective revocation can be implemented.

Finally, we can group all keys into one global table of keys.
A capability is valid only if its key matches some key in the global table.
We implement revocation by removing the matching key from the table.
With this scheme, a key can be associated with several objects, and several keys can be associated with each object, providing maximum flexibility.

In key-based schemes, the operations of defining keys, inserting them into lists, and deleting them from lists should not be available to all users.
In particular, it would be reasonable to allow only the owner of an object to set the keys for that object.
This choice, however, is a policy decision that the protection system can implement but should not define.

%%% Local Variables:
%%% mode: latex
%%% TeX-master: "../../EDAF35-Operating_Systems-Reference_Sheet"
%%% End:
