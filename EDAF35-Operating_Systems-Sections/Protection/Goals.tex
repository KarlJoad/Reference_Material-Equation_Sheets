\subsection{Goals of Protection}\label{subsec:Goals_of_Protection}
System protection was originally conceived in tandem with multiprogramming \nameref{def:Operating_System}s, so that untrustworthy \nameref{def:User}s might safely:
\begin{itemize}[noitemsep]
\item Share a common logical name space
  \begin{itemize}[noitemsep]
  \item Share a directory of files
  \end{itemize}
\item Share a common physical name space
  \begin{itemize}[noitemsep]
  \item Such as memory.
  \end{itemize}
\end{itemize}

Protection can improve reliability by detecting latent errors at the interfaces between component subsystems.
Early detection of interface errors can often prevent contamination of a healthy subsystem by a malfunctioning subsystem.
Also, an unprotected resource cannot defend against use (or misuse) by an unauthorized or incompetent user.
A protection-oriented system provides means to distinguish between authorized and unauthorized usage.

The role of protection in a computer system is to provide a mechanism for the enforcement of the policies governing resource use.


%%% Local Variables:
%%% mode: latex
%%% TeX-master: "../../EDAF35-Operating_Systems-Reference_Sheet"
%%% End:
