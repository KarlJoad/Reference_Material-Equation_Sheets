\subsection{Domains of Protection}\label{subsec:Domains_of_Protection}
A computer system is a collection of \nameref{def:Process}es and objects.
Objects refer to:
\begin{itemize}[noitemsep]
\item Hardware Objects
  \begin{itemize}[noitemsep]
  \item CPU
  \item Memory Segments
  \item Printers
  \item Disks
  \item Tape Drives
  \end{itemize}
\item Software Objects
  \begin{itemize}[noitemsep]
  \item Files
  \item Programs
  \item Semaphores
  \end{itemize}
\end{itemize}

Each object has a unique name that differentiates it from all other objects in the system, and each can be accessed only through well-defined and meaningful operations.
The operations that are possible depends on the object in question.
For example, on a CPU, we can perform executions, memory can be read from and written to, etc.

A \nameref{def:Process} should be allowed to access only those resources for which it has authorization.
Furthermore, at any time, a process should be able to access only those resources that it currently requires to complete its task.
This second requirement, is commonly referred to as the \nameref{def:Need_To_Know_Principle}.

\begin{definition}[Need-to-Know Principle]\label{def:Need_To_Know_Principle}
  The \emph{Need-to-Know principle} states that a \nameref{def:Process} should be able to access \textbf{only} those resources that it currently requires to complete its \textbf{current} task.

  This is useful in limiting the amount of damage a faulty process can cause in the system.
\end{definition}


%%% Local Variables:
%%% mode: latex
%%% TeX-master: "../../EDAF35-Operating_Systems-Reference_Sheet"
%%% End:
