\subsection{Domains of Protection}\label{subsec:Domains_of_Protection}
A computer system is a collection of \nameref{def:Process}es and objects.
Objects refer to:
\begin{itemize}[noitemsep]
\item Hardware Objects
  \begin{itemize}[noitemsep]
  \item CPU
  \item Memory Segments
  \item Printers
  \item Disks
  \item Tape Drives
  \end{itemize}
\item Software Objects
  \begin{itemize}[noitemsep]
  \item Files
  \item Programs
  \item Semaphores
  \end{itemize}
\end{itemize}

Each object has a unique name that differentiates it from all other objects in the system, and each can be accessed only through well-defined and meaningful operations.
The operations that are possible depends on the object in question.
For example, on a CPU, we can perform executions, memory can be read from and written to, etc.

A \nameref{def:Process} should be allowed to access only those resources for which it has authorization.
Furthermore, at any time, a process should be able to access only those resources that it currently requires to complete its task.
This second requirement, is commonly referred to as the \nameref{def:Need_To_Know_Principle}.

\begin{definition}[Need-to-Know Principle]\label{def:Need_To_Know_Principle}
  The \emph{Need-to-Know principle} states that a \nameref{def:Process} should be able to access \textbf{only} those resources that it currently requires to complete its \textbf{current} task.

  This is useful in limiting the amount of damage a faulty process can cause in the system.
\end{definition}

\subsection{Domain Structure}\label{subsubsec:Domain_Structure}
A \nameref{def:Process} operates within a \nameref{def:Protection_Domain}.

\begin{definition}[Protection Domain]\label{def:Protection_Domain}
  A \emph{protection domain} specifies the resources that a \nameref{def:Process} may access.
  Each domain defines a set of objects and \nameref{def:Access_Right}s.
  A domain is a collection of access rights, each of which is an ordered pair

  \begin{equation}\label{eq:Protection_Domain}
    \langle \text{object-name}, \text{rights-set} \rangle
  \end{equation}
\end{definition}

A \nameref{def:Protection_Domain} can be defined in many ways.
\begin{itemize}[noitemsep]
\item Each user may be a domain. In this case, the set of objects that
  can be accessed depends on the identity of the user. Domain
  switching occurs when the user is changed —generally when one user
  logs out and another user logs in.
\item Each process may be a domain.
  In this case, the set of objects that can be accessed depends on the
  identity of the process. Domain switching occurs when one process
  sends a message to another process and then waits for a response.
\item Each procedure may be a domain. In this case, the set of objects
  that can be accessed corresponds to the local variables defined
  within the procedure. Domain switching occurs when a procedure call
  is made.
\end{itemize}

\begin{definition}[Access Right]\label{def:Access_Right}
  The ability to execute an operation on an object is an \emph{access right}.
\end{definition}

\nameref{def:Protection_Domain}s may share \nameref{def:Access_Right}s.
For example, if we have three domains: $D_{1}$, $D_{2}$, and $D_{3}$ an an access right $\langle O_{4}, \text{print} \rangle$ that is shared by $D_{2}$ and $D_{3}$, a process executing in either of these domains can print object $O_{4}$.

The association between a \nameref{def:Process} and a \nameref{def:Protection_Domain} may be either \textbf{static}, if the set of resources available to the process is fixed throughout the process’s lifetime, or \textbf{dynamic}.

\subsubsection{Static Domain Structure}\label{subsubsec:Static_Domain_Structure}
A static association between \nameref{def:Process}es and \nameref{def:Protection_Domain}s is possible, but it is not a very flexible system, and makes it difficult to adhere to the need-to-know principle.
To counter this, a mechanism must be available to change the content of a domain.

This mechanism is needed because a \nameref{def:Process} may execute in two different phases and may, for example, need read access in one phase and write access in another.
If the \nameref{def:Protection_Domain} is static, the domain must be defined to include \textbf{both} read and write access throughout the process's execution.

However, this arrangement provides more rights than are needed in each of the two phases, since we have read access in the phase where we need only write access, and vice versa.

\subsubsection{Dynamic Domain Structure}\label{subsubsec:Dynamic_Domain_Structure}
A dynamic association between \nameref{def:Process}es and \nameref{def:Protection_Domain}s relies on a mechanism to allow \nameref{def:Domain_Switching}.

\begin{definition}[Domain Switching]\label{def:Domain_Switching}
  \emph{Domain switching} is a tool that enables a \nameref{def:Process} to switch from one \nameref{def:Protection_Domain} to another.
  We may also allow the content of a domain to be changed.
\end{definition}

If we cannot change the content of a domain, we can provide the same effect by creating a new domain with the changed content and switching to that new domain when we want to change the domain content.

%%% Local Variables:
%%% mode: latex
%%% TeX-master: "../../EDAF35-Operating_Systems-Reference_Sheet"
%%% End:
