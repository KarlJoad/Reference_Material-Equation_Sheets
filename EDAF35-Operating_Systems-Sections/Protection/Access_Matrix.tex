\subsection{Access Matrix}\label{subsec:Access_Matrix}
A general model of protection can be viewed abstractly as a matrix, called an \nameref{def:Access_Matrix}.
\begin{definition}[Access Matrix]\label{def:Access_Matrix}
  An \emph{access matrix} is a way to represent the access rights between \nameref{def:Protection_Domain}s and objects.
  The rows of the access matrix are the domains, and the columns represent objects.
  Each entry in the matrix consists of a set of \nameref{def:Access_Right}s.
  The entry \texttt{access($i$,$j$)} defines the set of operations that a process executing in domain $D_{i}$ can invoke on object $O_{j}$.

  \begin{remark}
    Because the column defines objects explicitly, we can omit the object name from the \nameref{def:Access_Right}.
  \end{remark}
\end{definition}

% Not using \toprule \midrule and \bottomrule was deliberate.
% This is not a table, it is a lookup matrix, and it is easier to read with the extra lines.
\begin{table}[h!tbp]
  \centering
  \begin{tabular}{|c|c|c|c|c|}
    \hline
    \diagbox{Domain}{Object} & $F_{1}$ & $F_{2}$ & $F_{3}$ & Printer \\
    \hline
    $D_{1}$ & Read & & Read & \\
    \hline
    $D_{2}$ & & & & Print \\
    \hline
    $D_{3}$ & Read & Execute & & \\
    \hline
    \multirow{2}{*}{$D_{4}$} & Read & & Read & \\
                             & Write & & Write & \\
    \hline
  \end{tabular}
  \caption{Basic Access Matrix}
  \label{tab:Basic_Access_Matrix}
\end{table}


%%% Local Variables:
%%% mode: latex
%%% TeX-master: "../../EDAF35-Operating_Systems-Reference_Sheet"
%%% End:
