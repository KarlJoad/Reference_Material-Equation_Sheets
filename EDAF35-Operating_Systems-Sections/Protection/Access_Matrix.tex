\subsection{Access Matrix}\label{subsec:Access_Matrix}
A general model of protection can be viewed abstractly as a matrix, called an \nameref{def:Access_Matrix}.
\begin{definition}[Access Matrix]\label{def:Access_Matrix}
  An \emph{access matrix} is a way to represent the access rights between \nameref{def:Protection_Domain}s and objects.
  The rows of the access matrix are the domains, and the columns represent objects.
  Each entry in the matrix consists of a set of \nameref{def:Access_Right}s.
  The entry \texttt{access($i$,$j$)} defines the set of operations that a process executing in domain $D_{i}$ can invoke on object $O_{j}$.

  \begin{remark}
    Because the column defines objects explicitly, we can omit the object name from the \nameref{def:Access_Right}.
  \end{remark}
\end{definition}

% Not using \toprule \midrule and \bottomrule was deliberate.
% This is not a table, it is a lookup matrix, and it is easier to read with the extra lines.
\begin{table}[h!tbp]
  \centering
  \begin{tabular}{|c|c|c|c|c|}
    \hline
    \diagbox{Domain}{Object} & $F_{1}$ & $F_{2}$ & $F_{3}$ & Printer \\
    \hline
    $D_{1}$ & Read & & Read & \\
    \hline
    $D_{2}$ & & & & Print \\
    \hline
    $D_{3}$ & Read & Execute & & \\
    \hline
    \multirow{2}{*}{$D_{4}$} & Read & & Read & \\
                             & Write & & Write & \\
    \hline
  \end{tabular}
  \caption{Basic Access Matrix}
  \label{tab:Basic_Access_Matrix}
\end{table}

Using \Cref{tab:Basic_Access_Matrix}, we can determine the \nameref{def:Access_Right}s for various domains on various objects.
A process executing in domain $D_{1}$ can read files $F_{1}$ and $F_{3}$.
A process executing in domain $D_{4}$ has the same privileges as one executing in domain $D_{1}$; but in addition, it can also write onto files $F_{1}$ and $F_{3}$.

Access matrices can provide a variety of \textbf{both} mechanisms and policies for system protection.
The mechanism consists of implementing the access matrix and ensuring that the semantic properties we have outlined hold.
This means we must ensure that a process executing in domain $D_{i}$ can access only those objects specified in row $i$ as allowed by the access-matrix entries.
The policy decisions involve which rights should be included in the $(i, j)$th entry.
We must also decide the domain in which each process executes, which is usually decided by the operating system.

When we switch a process from one domain to another, we are executing a \texttt{switch} operation on the domain (which becomes an object).
We can control \nameref{def:Domain_Switching} by including domains as objects of the \nameref{def:Access_Matrix} (\Cref{tab:Access_Matrix_Domain_Objects}).

\begin{table}[h!tbp]
  \centering
  \begin{tabular}{|c|c|c|c|c|c|c|c|c|}
    \hline
    \diagbox{Domain}{Object} & $F_{1}$ & $F_{2}$ & $F_{3}$ & Printer & $D_{1}$ & $D_{2}$ & $D_{3}$ & $D_{4}$ \\
    \hline
    $D_{1}$ & Read & & Read & & Switch & & & \\
    \hline
    $D_{2}$ & & & Print & & & Switch & Switch & \\
    \hline
    $D_{3}$ & & Read & Execute & & & & & \\
    \hline
    \multirow{2}{*}{$D_{4}$} & Read & & Read & & Switch & & & \\
                             & Write & & Write & & Switch & & & \\
    \hline
  \end{tabular}
  \caption{\nameref*{def:Access_Matrix} with Domains as Objects}
  \label{tab:Access_Matrix_Domain_Objects}
\end{table}

Similarly, when we change the content of the access matrix, we are performing an operation on an object: the access matrix.
Again, we can control these changes by including the access matrix itself as an object.
If we want to be pedantic, we must protect each entry in the \nameref{def:Access_Matrix} because each entry can be modified individually.
This requires three additional operations: \texttt{copy}, \texttt{owner}, and \texttt{control}.

The \texttt{copy} right has 3 variants.
\begin{enumerate}[noitemsep]
\item Allows the access rights to be copied only within the column.
\item A right is ``copied'' from access(i, j) to access(k, j); it is then removed from access(i, j).
  This action is a \texttt{transfer} of a right, rather than a copy.
\item Propagation of the copy right may be limited.
  That is, when the set of rights $R^{*}$ is copied from \texttt{access($i$, $j$)} to \texttt{access($k$, $j$)},  the rights $R$ is created.
  A process executing in domain $D_{k}$ cannot further copy the copy of the set of rights, $R$.
\end{enumerate}

The \texttt{owner} right controls the addition and removal of rights.
If \texttt{access($i$, $j$)} includes the \texttt{owner} right, then a process executing in domain $D_{i}$ can add and remove any right in any entry in column $j$.
This means that there are ways to change the entries in a column.

The \texttt{control} right allows for changing the entries in a row.
The \texttt{control} right is applicable only to domain objects.
If \texttt{access($i$, $j$)} includes the \texttt{control} right, then a process executing in domain $D_{i}$ can remove any access right from row $j$.


%%% Local Variables:
%%% mode: latex
%%% TeX-master: "../../EDAF35-Operating_Systems-Reference_Sheet"
%%% End:
