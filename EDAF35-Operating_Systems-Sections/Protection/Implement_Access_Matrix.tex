\subsection{Implementing Access Matrices}\label{subsec:Implement_Access_Matrices}
In general, an \nameref{def:Access_Matrix} will be sparse, i.e.\ most of the entries in the table will be empty.
Along with this, traditional sparse data structure techniques are not terribly useful for this.
These are because of the way the protection facilities are used.

Most systems use a combination of \nameref{def:Access_List}s and \nameref{def:Capability_List}s.
When a \nameref{def:Process} first tries to access an object, the access list is searched.
If access is denied, an exception condition occurs.
Otherwise, a capability is created and attached to the process.
Subsequent references use the capability to quickly determine access.
After the last access, the capability is destroyed.

\subsubsection{Global Table}\label{subsubsec:Global_Access_Matrix}
The simplest implementation of the access matrix is a global table consisting of a set of ordered tuples $\langle \text{domain}, \text{object}, \text{rights-set} \rangle$.
Whenever an operation $M$ is executed on an object $O_{j}$ within domain $D_{i}$, the global table is searched for a tuple $\langle D_{i}, O_{j}, R_{k} \rangle$, where $M \in R_{k}$.
If this tuple is found, the operation is allowed to continue; otherwise, an exception (or error) is raised.

This implementation suffers from several drawbacks.
The table is usually large and thus cannot be kept in main memory, so additional I/O is needed.
Virtual memory techniques are often used for managing this table (\nameref{def:Memory_Mapping}).
In addition, it is difficult to take advantage of special groupings of objects or domains.

\subsubsection{Access Lists}\label{subsubsec:Access_Lists}
\nameref{def:Access_List}s are created by only working with the columns of a \nameref{def:Access_Matrix}.

\begin{definition}[Access List]\label{def:Access_List}
  An \emph{access list} is a way for \textbf{objects} to track what domains can perform what actions on that object.
   The resulting list for each object consists of ordered pairs $\langle \text{domain}, \text{rights-set} \rangle$, which define all domains with a nonempty set of access rights for that object.
\end{definition}

This can be extended to define an \nameref{def:Access_List} and a default set of \nameref{def:Access_Right}s.
When an operation $M$ on an object $O_{j}$ is attempted in domain $D_{i}$, we search the access list for object $O_{j}$, looking for an entry $\langle D_{i}, R_{k} \rangle$ with $M \in R_{k}$.
If the entry is found, we allow the operation; if it is not, we check the default set.
If $M$ is in the default set, we allow the access.
Otherwise, access is denied, and an exception condition occurs.

Access lists correspond directly to the needs of users.
When a user creates an object, they can specify which domains can access the object, as well as what operations are allowed.
However, access-right information for a particular domain is not localized, making determining the set of access rights for each domain difficult.
In addition, every access to an object must be checked, requiring a search of the access list.
In a large system with long access lists, this search can be time consuming.

\subsubsection{Capability Lists}\label{subsubsec:Capability_Lists}
\nameref{def:Capability_List}s are created by only working with the rows of a \nameref{def:Access_Matrix}.

\begin{definition}[Capability List]\label{def:Capability_List}
  A \emph{capability list} is a way for \textbf{domains} to track what objects they can access and the actions the domain can perform on that object.
\end{definition}

To execute operation $M$ on object $O_{j}$, the process executes the operation $M$, specifying the capability (or pointer) for object $O_{j}$ as a parameter.
Just having the possession of a capability means that access to that object is allowed.

The \nameref{def:Capability_List} is associated with a \nameref{def:Protection_Domain}, but it is never directly accessible to a \nameref{def:Process} executing in that domain.
The list itself is a protected object, maintained by the \nameref{def:Operating_System} and only indirectly accessed by the user.
Capability-based protection relies on the fact that the capabilities are never allowed to migrate into any space directly accessible by a user process.
If all capabilities are secure, the object they protect is also secure against unauthorized access.

To provide protection, we must distinguish capabilities from other kinds of objects, and they must be interpreted for higher-level programs run.
Capabilities are usually distinguished from other data in one of two ways:
\begin{enumerate}[noitemsep]
\item Each object has a tag to denote whether it is a capability or accessible data.
  The tags themselves must not be directly accessible by an application program.
  Hardware or firmware support may be used to enforce this restriction.
  Although only one bit is necessary to distinguish between capabilities and other objects, more bits are often used.
  This extension allows all objects to be tagged with their types by the hardware.
  Thus, the hardware can distinguish integers, floating-point numbers, pointers, Booleans, characters, instructions, capabilities, and uninitialized values by their tags.
\item Alternatively, the address space associated with a \nameref{def:Process} can be split into two parts.
  One part is accessible to the program and contains the program’s normal data and instructions.
  The other part, containing the capability list, is accessible only by the operating system.
  \nameref{subsec:Segmentation} (\Cref{subsec:Segmentation}) is useful to support this approach.
\end{enumerate}

\nameref{def:Capability_List}s do not correspond directly to the needs of \nameref{def:User}s, but they are useful for localizing information for a given \nameref{def:Process}.
The process attempting access must present a capability for that access.
Then, the protection system needs only to verify that the capability is valid.
Revocation of capabilities, however, may be inefficient.

\subsubsection{Lock-Key}\label{subsubsec:Lock_Key_Access}
The lock–key scheme is a compromise between \nameref{def:Access_List}s and \nameref{def:Capability_List}s.
Each object has a list of unique bit patterns, called locks.
Similarly, each domain has a list of unique bit patterns, called keys.
A \nameref{def:Process} executing in a domain can access an object only if that domain has a key that matches one of the locks of the object.

As with capability lists, the list of keys for a domain must be managed by the operating system on behalf of the domain.
\nameref{def:User}s are not allowed to examine or modify the list of keys (or locks) directly.

The lock–key mechanism, as mentioned, is a compromise between \nameref{def:Access_List}s and \nameref{def:Capability_List}s.
The mechanism can be both effective and flexible, depending on the length of the keys.
The keys can be passed freely from domain to domain.
In addition, access privileges can be effectively revoked by the simple technique of changing some of the locks associated with the object.

%%% Local Variables:
%%% mode: latex
%%% TeX-master: "../../EDAF35-Operating_Systems-Reference_Sheet"
%%% End:
