\subsection{Implementing Access Matrices}\label{subsec:Implement_Access_Matrices}
In general, an \nameref{def:Access_Matrix} will be sparse, i.e.\ most of the entries in the table will be empty.
Along with this, traditional sparse data structure techniques are not terribly useful for this.
These are because of the way the protection facilities are used.

Most systems use a combination of \nameref{def:Access_List}s and \nameref{def:Capability_List}s.
When a \nameref{def:Process} first tries to access an object, the access list is searched.
If access is denied, an exception condition occurs.
Otherwise, a capability is created and attached to the process.
Subsequent references use the capability to quickly determine access.
After the last access, the capability is destroyed.

\subsubsection{Global Table}\label{subsubsec:Global_Access_Matrix}
The simplest implementation of the access matrix is a global table consisting of a set of ordered tuples $\langle \text{domain}, \text{object}, \text{rights-set} \rangle$.
Whenever an operation $M$ is executed on an object $O_{j}$ within domain $D_{i}$, the global table is searched for a tuple $\langle D_{i}, O_{j}, R_{k} \rangle$, where $M \in R_{k}$.
If this tuple is found, the operation is allowed to continue; otherwise, an exception (or error) is raised.

This implementation suffers from several drawbacks.
The table is usually large and thus cannot be kept in main memory, so additional I/O is needed.
Virtual memory techniques are often used for managing this table (\nameref{def:Memory_Mapping}).
In addition, it is difficult to take advantage of special groupings of objects or domains.

\subsubsection{Access Lists}\label{subsubsec:Access_Lists}
\nameref{def:Access_List}s are created by only working with the columns of a \nameref{def:Access_Matrix}.

\begin{definition}[Access List]\label{def:Access_List}
  An \emph{access list} is a way for \textbf{objects} to track what domains can perform what actions on that object.
   The resulting list for each object consists of ordered pairs $\langle \text{domain}, \text{rights-set} \rangle$, which define all domains with a nonempty set of access rights for that object.
\end{definition}

This can be extended to define an \nameref{def:Access_List} and a default set of \nameref{def:Access_Right}s.
When an operation $M$ on an object $O_{j}$ is attempted in domain $D_{i}$, we search the access list for object $O_{j}$, looking for an entry $\langle D_{i}, R_{k} \rangle$ with $M \in R_{k}$.
If the entry is found, we allow the operation; if it is not, we check the default set.
If $M$ is in the default set, we allow the access.
Otherwise, access is denied, and an exception condition occurs.

Access lists correspond directly to the needs of users.
When a user creates an object, they can specify which domains can access the object, as well as what operations are allowed.
However, access-right information for a particular domain is not localized, making determining the set of access rights for each domain difficult.
In addition, every access to an object must be checked, requiring a search of the access list.
In a large system with long access lists, this search can be time consuming.


%%% Local Variables:
%%% mode: latex
%%% TeX-master: "../../EDAF35-Operating_Systems-Reference_Sheet"
%%% End:
