\subsection{Principles of Protection}\label{subsec:Protection_Principles}
Principles that are used for protection allow us to make decisions about what policies to set on a system and what mechanisms enforce those policies.
A key, time-tested principle for protection is the \nameref{def:Principle_Least_Privilege}.

\begin{definition}[Principle of Least Privilege]\label{def:Principle_Least_Privilege}
  The \emph{principle of least privilege} states that programs, users, and even systems be given just enough privileges to perform their tasks.
\end{definition}

An \nameref{def:Operating_System} following this principle implements its features, programs, system calls, and data structures so that failure or compromise of a component does the minimum damage.
The overflow of a buffer in a system \nameref{def:Daemon} might cause the daemon process to fail, but should not allow the execution of code from the daemon process’s stack.


%%% Local Variables:
%%% mode: latex
%%% TeX-master: "../../EDAF35-Operating_Systems-Reference_Sheet"
%%% End:
