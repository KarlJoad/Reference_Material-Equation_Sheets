\section{Protection}\label{sec:Protection}
Protection refers to a mechanism for controlling the access of \nameref{def:Program}s, \nameref{def:Process}es, or \nameref{def:User}s to the resources defined by a computer system.
This must provide a means for specifying the controls to be imposed, together with a means of enforcement.
Protection and security mean different things here.
Security is a measure of confidence that the integrity of a system and its data will be preserved.

If system protection interferes with the ease of use of the system or significantly decreases system performance, then its use must be weighed carefully against the purpose of the system.
For instance, we would want to have a complex protection system on a computer used by a university to process students’ grades and also used by students for classwork.
A similar protection system would not be suited to a computer being used for number crunching, in which performance is of utmost importance.


%%% Local Variables:
%%% mode: latex
%%% TeX-master: "../EDAF35-Operating_Systems-Reference_Sheet"
%%% End:
