\section{Protection}\label{sec:Protection}
Protection refers to a mechanism for controlling the access of \nameref{def:Program}s, \nameref{def:Process}es, or \nameref{def:User}s to the resources defined by a computer system.
This must provide a means for specifying the controls to be imposed, together with a means of enforcement.
Protection and security mean different things here.
Security is a measure of confidence that the integrity of a system and its data will be preserved.

If system protection interferes with the ease of use of the system or significantly decreases system performance, then its use must be weighed carefully against the purpose of the system.
For instance, we would want to have a complex protection system on a computer used by a university to process students’ grades and also used by students for classwork.
A similar protection system would not be suited to a computer being used for number crunching, in which performance is of utmost importance.

\subsection{Goals of Protection}\label{subsec:Goals_of_Protection}
System protection was originally conceived in tandem with multiprogramming \nameref{def:Operating_System}s, so that untrustworthy \nameref{def:User}s might safely:
\begin{itemize}[noitemsep]
\item Share a common logical name space
  \begin{itemize}[noitemsep]
  \item Share a directory of files
  \end{itemize}
\item Share a common physical name space
  \begin{itemize}[noitemsep]
  \item Such as memory.
  \end{itemize}
\end{itemize}

Protection can improve reliability by detecting latent errors at the interfaces between component subsystems.
Early detection of interface errors can often prevent contamination of a healthy subsystem by a malfunctioning subsystem.
Also, an unprotected resource cannot defend against use (or misuse) by an unauthorized or incompetent user.
A protection-oriented system provides means to distinguish between authorized and unauthorized usage.

The role of protection in a computer system is to provide a mechanism for the enforcement of the policies governing resource use.

Policies for resource use may vary by application, and they may change over time.
The application programmer needs to use protection mechanisms as well, to guard resources created and supported by an application subsystem against misuse.
The \nameref{def:Operating_System} should provide protection mechanisms, but application designers can use them as well in designing their own protection software.
Note that mechanisms are distinct from policies.


%%% Local Variables:
%%% mode: latex
%%% TeX-master: "../../EDAF35-Operating_Systems-Reference_Sheet"
%%% End:


\subsection{Principles of Protection}\label{subsec:Protection_Principles}

%%% Local Variables:
%%% mode: latex
%%% TeX-master: "../../EDAF35-Operating_Systems-Reference_Sheet"
%%% End:



%%% Local Variables:
%%% mode: latex
%%% TeX-master: "../EDAF35-Operating_Systems-Reference_Sheet"
%%% End:
