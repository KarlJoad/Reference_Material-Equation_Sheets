\section{Protection}\label{sec:Protection}
Protection refers to a mechanism for controlling the access of \nameref{def:Program}s, \nameref{def:Process}es, or \nameref{def:User}s to the resources defined by a computer system.
This must provide a means for specifying the controls to be imposed, together with a means of enforcement.
Protection and security mean different things here.
Security is a measure of confidence that the integrity of a system and its data will be preserved.

If system protection interferes with the ease of use of the system or significantly decreases system performance, then its use must be weighed carefully against the purpose of the system.
For instance, we would want to have a complex protection system on a computer used by a university to process students’ grades and also used by students for classwork.
A similar protection system would not be suited to a computer being used for number crunching, in which performance is of utmost importance.

\subsection{Goals of Protection}\label{subsec:Goals_of_Protection}
System protection was originally conceived in tandem with multiprogramming \nameref{def:Operating_System}s, so that untrustworthy \nameref{def:User}s might safely:
\begin{itemize}[noitemsep]
\item Share a common logical name space
  \begin{itemize}[noitemsep]
  \item Share a directory of files
  \end{itemize}
\item Share a common physical name space
  \begin{itemize}[noitemsep]
  \item Such as memory.
  \end{itemize}
\end{itemize}


%%% Local Variables:
%%% mode: latex
%%% TeX-master: "../../EDAF35-Operating_Systems-Reference_Sheet"
%%% End:


\subsection{Principles of Protection}\label{subsec:Protection_Principles}
Principles that are used for protection allow us to make decisions about what policies to set on a system and what mechanisms enforce those policies.
A key, time-tested principle for protection is the \nameref{def:Principle_Least_Privilege}.

\begin{definition}[Principle of Least Privilege]\label{def:Principle_Least_Privilege}
  The \emph{principle of least privilege} states that programs, users, and even systems be given just enough privileges to perform their tasks.
\end{definition}

An \nameref{def:Operating_System} following this principle implements its features, programs, system calls, and data structures so that failure or compromise of a component does the minimum damage.
The overflow of a buffer in a system \nameref{def:Daemon} might cause the daemon process to fail, but should not allow the execution of code from the daemon process’s stack.

This \nameref{def:Operating_System} also provides \nameref{def:System_Call}s and services that allow applications to be written with fine-grained access controls.
It provides mechanisms to enable privileges \emph{when} they are needed and to disable them when they are not needed.
Also, audit trails are created for all privileged function access.
This allows for tracing all protection and security activities on the system.

Managing users with the \nameref{def:Principle_Least_Privilege} entails creating a separate account for each user, each with just the privileges that the user needs.

%%% Local Variables:
%%% mode: latex
%%% TeX-master: "../../EDAF35-Operating_Systems-Reference_Sheet"
%%% End:



%%% Local Variables:
%%% mode: latex
%%% TeX-master: "../EDAF35-Operating_Systems-Reference_Sheet"
%%% End:
