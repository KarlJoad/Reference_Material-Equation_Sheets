\section{Protection}\label{sec:Protection}
Protection refers to a mechanism for controlling the access of \nameref{def:Program}s, \nameref{def:Process}es, or \nameref{def:User}s to the resources defined by a computer system.
This must provide a means for specifying the controls to be imposed, together with a means of enforcement.
Protection and security mean different things here.
Security is a measure of confidence that the integrity of a system and its data will be preserved.

If system protection interferes with the ease of use of the system or significantly decreases system performance, then its use must be weighed carefully against the purpose of the system.
For instance, we would want to have a complex protection system on a computer used by a university to process students’ grades and also used by students for classwork.
A similar protection system would not be suited to a computer being used for number crunching, in which performance is of utmost importance.

\subsection{Goals of Protection}\label{subsec:Goals_of_Protection}
System protection was originally conceived in tandem with multiprogramming \nameref{def:Operating_System}s, so that untrustworthy \nameref{def:User}s might safely:
\begin{itemize}[noitemsep]
\item Share a common logical name space
  \begin{itemize}[noitemsep]
  \item Share a directory of files
  \end{itemize}
\item Share a common physical name space
  \begin{itemize}[noitemsep]
  \item Such as memory.
  \end{itemize}
\end{itemize}

Protection can improve reliability by detecting latent errors at the interfaces between component subsystems.
Early detection of interface errors can often prevent contamination of a healthy subsystem by a malfunctioning subsystem.
Also, an unprotected resource cannot defend against use (or misuse) by an unauthorized or incompetent user.
A protection-oriented system provides means to distinguish between authorized and unauthorized usage.

The role of protection in a computer system is to provide a mechanism for the enforcement of the policies governing resource use.

Policies for resource use may vary by application, and they may change over time.
The application programmer needs to use protection mechanisms as well, to guard resources created and supported by an application subsystem against misuse.
The \nameref{def:Operating_System} should provide protection mechanisms, but application designers can use them as well in designing their own protection software.
Note that mechanisms are distinct from policies.


%%% Local Variables:
%%% mode: latex
%%% TeX-master: "../../EDAF35-Operating_Systems-Reference_Sheet"
%%% End:


\subsection{Principles of Protection}\label{subsec:Protection_Principles}

%%% Local Variables:
%%% mode: latex
%%% TeX-master: "../../EDAF35-Operating_Systems-Reference_Sheet"
%%% End:


\subsection{Domains of Protection}\label{subsec:Domains_of_Protection}
A computer system is a collection of \nameref{def:Process}es and objects.
Objects refer to:
\begin{itemize}[noitemsep]
\item Hardware Objects
  \begin{itemize}[noitemsep]
  \item CPU
  \item Memory Segments
  \item Printers
  \item Disks
  \item Tape Drives
  \end{itemize}
\item Software Objects
  \begin{itemize}[noitemsep]
  \item Files
  \item Programs
  \item Semaphores
  \end{itemize}
\end{itemize}

Each object has a unique name that differentiates it from all other objects in the system, and each can be accessed only through well-defined and meaningful operations.
The operations that are possible depends on the object in question.
For example, on a CPU, we can perform executions, memory can be read from and written to, etc.

A \nameref{def:Process} should be allowed to access only those resources for which it has authorization.
Furthermore, at any time, a process should be able to access only those resources that it currently requires to complete its task.
This second requirement, is commonly referred to as the \nameref{def:Need_To_Know_Principle}.

\begin{definition}[Need-to-Know Principle]\label{def:Need_To_Know_Principle}
  The \emph{Need-to-Know principle} states that a \nameref{def:Process} should be able to access \textbf{only} those resources that it currently requires to complete its \textbf{current} task.

  This is useful in limiting the amount of damage a faulty process can cause in the system.
\end{definition}

\subsection{Domain Structure}\label{subsubsec:Domain_Structure}
A \nameref{def:Process} operates within a \nameref{def:Protection_Domain}.

\begin{definition}[Protection Domain]\label{def:Protection_Domain}
  A \emph{protection domain} specifies the resources that a \nameref{def:Process} may access.
  Each domain defines a set of objects and \nameref{def:Access_Right}s.
  A domain is a collection of access rights, each of which is an ordered pair

  \begin{equation}\label{eq:Protection_Domain}
    \langle \text{object-name}, \text{rights-set} \rangle
  \end{equation}
\end{definition}

A \nameref{def:Protection_Domain} can be defined in many ways.
\begin{itemize}[noitemsep]
\item Each user may be a domain. In this case, the set of objects that
  can be accessed depends on the identity of the user. Domain
  switching occurs when the user is changed —generally when one user
  logs out and another user logs in.
\item Each process may be a domain.
  In this case, the set of objects that can be accessed depends on the
  identity of the process. Domain switching occurs when one process
  sends a message to another process and then waits for a response.
\item Each procedure may be a domain. In this case, the set of objects
  that can be accessed corresponds to the local variables defined
  within the procedure. Domain switching occurs when a procedure call
  is made.
\end{itemize}

\begin{definition}[Access Right]\label{def:Access_Right}
  The ability to execute an operation on an object is an \emph{access right}.
\end{definition}

\nameref{def:Protection_Domain}s may share \nameref{def:Access_Right}s.
For example, if we have three domains: $D_{1}$, $D_{2}$, and $D_{3}$ an an access right $\langle O_{4}, \text{print} \rangle$ that is shared by $D_{2}$ and $D_{3}$, a process executing in either of these domains can print object $O_{4}$.

The association between a \nameref{def:Process} and a \nameref{def:Protection_Domain} may be either \textbf{static}, if the set of resources available to the process is fixed throughout the process’s lifetime, or \textbf{dynamic}.

\subsubsection{Static Domain Structure}\label{subsubsec:Static_Domain_Structure}
A static association between \nameref{def:Process}es and \nameref{def:Protection_Domain}s is possible, but it is not a very flexible system, and makes it difficult to adhere to the need-to-know principle.
To counter this, a mechanism must be available to change the content of a domain.

This mechanism is needed because a \nameref{def:Process} may execute in two different phases and may, for example, need read access in one phase and write access in another.
If the \nameref{def:Protection_Domain} is static, the domain must be defined to include \textbf{both} read and write access throughout the process's execution.

However, this arrangement provides more rights than are needed in each of the two phases, since we have read access in the phase where we need only write access, and vice versa.

\subsubsection{Dynamic Domain Structure}\label{subsubsec:Dynamic_Domain_Structure}
A dynamic association between \nameref{def:Process}es and \nameref{def:Protection_Domain}s relies on a mechanism to allow \nameref{def:Domain_Switching}.

\begin{definition}[Domain Switching]\label{def:Domain_Switching}
  \emph{Domain switching} is a tool that enables a \nameref{def:Process} to switch from one \nameref{def:Protection_Domain} to another.
  We may also allow the content of a domain to be changed.
\end{definition}

If we cannot change the content of a domain, we can provide the same effect by creating a new domain with the changed content and switching to that new domain when we want to change the domain content.

\subsubsection{\textsc{unix}-Style Protection Domains and Schemes}\label{subsubsec:UNIX_Protections}
In the \textsc{unix} operating system, a domain is associated with the \nameref{def:User}.
Switching the domain corresponds to changing the user identification.

This change is accomplished through the \nameref{def:File_System}.
Each file contains a lot of metadata regarding the attributes of the file, including who owns the file and to which domain (\nameref{def:User}) that files belongs to.
The first one is the file's owner identification (\texttt{userID}); the second is the \texttt{setuid} bit (the domain bit).

When the \texttt{setuid} bit is on, and a \nameref{def:User} executes that file, the process that is created has its \texttt{userID} set to the owner of the file.
When the \texttt{setuid} bit is off, the \texttt{userID} does not change.
For example, when a user $A$ (that is, a user with \texttt{userID} = $A$) starts executing a file owned by $B$, whose associated domain bit is off, the \texttt{userID} of the process is set to $A$.
When the \texttt{setuid} bit is on, the \texttt{userID} is set to that of the owner of the file: $B$.
When the process exits, this temporary \texttt{userID} change ends.

Other methods are used to change domains in operating systems in which \texttt{userID}s are used for domain definition.
This mechanism is used when an otherwise privileged facility needs to be made available to the general user population.
For instance, it might be desirable to allow users to access a network without letting them write their own networking programs.
On a \textsc{unix} system, the \texttt{setuid} bit on a networking program would be set, causing the \texttt{userID} to change to another \texttt{userID} that is capable of using the network when the program is run.
The \texttt{userID} would change to that of a user with network access privilege (such as \texttt{root}).
One problem with this method is that if a user manages to create a file with \texttt{userID} root and with its \texttt{setuid} bit on, that user can become root and do anything and everything on the system.

An alternative to this method used in some other \nameref{def:Operating_System}s is to place privileged programs in a special \nameref{def:Directory}.
The operating system is designed to change the \texttt{userID} of any program run from this directory.
This eliminates the security problem of when intruders create programs to manipulate the \texttt{setuid} feature and hide the programs in the system for later use.
This method is less flexible than that used in \textsc{unix}, however.

Even more restrictive, and thus more protective, are systems that simply do not allow a change of \texttt{userID}.
In these \nameref{def:Operating_System}s, special techniques must be used to allow users access to privileged facilities.
One method, for instance, is to use a \nameref{def:Daemon} process that is started at boot and runs as a special \texttt{userID}.
Regular \nameref{def:User}s then run a separate program, which sends requests to this process whenever they need to use the facility.

In any of these systems, great care must be taken in writing privileged programs.
Any oversight can result in a total lack of protection on the system.

%%% Local Variables:
%%% mode: latex
%%% TeX-master: "../../EDAF35-Operating_Systems-Reference_Sheet"
%%% End:


\subsection{Access Matrix}\label{subsec:Access_Matrix}
A general model of protection can be viewed abstractly as a matrix, called an \nameref{def:Access_Matrix}.
\begin{definition}[Access Matrix]\label{def:Access_Matrix}
  An \emph{access matrix} is a way to represent the access rights between \nameref{def:Protection_Domain}s and objects.
  The rows of the access matrix are the domains, and the columns represent objects.
  Each entry in the matrix consists of a set of \nameref{def:Access_Right}s.
  The entry \texttt{access($i$,$j$)} defines the set of operations that a process executing in domain $D_{i}$ can invoke on object $O_{j}$.

  \begin{remark}
    Because the column defines objects explicitly, we can omit the object name from the \nameref{def:Access_Right}.
  \end{remark}
\end{definition}

% Not using \toprule \midrule and \bottomrule was deliberate.
% This is not a table, it is a lookup matrix, and it is easier to read with the extra lines.
\begin{table}[h!tbp]
  \centering
  \begin{tabular}{|c|c|c|c|c|}
    \hline
    \diagbox{Domain}{Object} & $F_{1}$ & $F_{2}$ & $F_{3}$ & Printer \\
    \hline
    $D_{1}$ & Read & & Read & \\
    \hline
    $D_{2}$ & & & & Print \\
    \hline
    $D_{3}$ & Read & Execute & & \\
    \hline
    \multirow{2}{*}{$D_{4}$} & Read & & Read & \\
                             & Write & & Write & \\
    \hline
  \end{tabular}
  \caption{Basic Access Matrix}
  \label{tab:Basic_Access_Matrix}
\end{table}

Using \Cref{tab:Basic_Access_Matrix}, we can determine the \nameref{def:Access_Right}s for various domains on various objects.
A process executing in domain $D_{1}$ can read files $F_{1}$ and $F_{3}$.
A process executing in domain $D_{4}$ has the same privileges as one executing in domain $D_{1}$; but in addition, it can also write onto files $F_{1}$ and $F_{3}$.

Access matrices can provide a variety of \textbf{both} mechanisms and policies for system protection.
The mechanism consists of implementing the access matrix and ensuring that the semantic properties we have outlined hold.
This means we must ensure that a process executing in domain $D_{i}$ can access only those objects specified in row $i$ as allowed by the access-matrix entries.
The policy decisions involve which rights should be included in the $(i, j)$th entry.
We must also decide the domain in which each process executes, which is usually decided by the operating system.

When we switch a process from one domain to another, we are executing a \texttt{switch} operation on the domain (which becomes an object).
We can control \nameref{def:Domain_Switching} by including domains as objects of the \nameref{def:Access_Matrix} (\Cref{tab:Access_Matrix_Domain_Objects}).

\begin{table}[h!tbp]
  \centering
  \begin{tabular}{|c|c|c|c|c|c|c|c|c|}
    \hline
    \diagbox{Domain}{Object} & $F_{1}$ & $F_{2}$ & $F_{3}$ & Printer & $D_{1}$ & $D_{2}$ & $D_{3}$ & $D_{4}$ \\
    \hline
    $D_{1}$ & Read & & Read & & Switch & & & \\
    \hline
    $D_{2}$ & & & Print & & & Switch & Switch & \\
    \hline
    $D_{3}$ & & Read & Execute & & & & & \\
    \hline
    \multirow{2}{*}{$D_{4}$} & Read & & Read & & Switch & & & \\
                             & Write & & Write & & Switch & & & \\
    \hline
  \end{tabular}
  \caption{\nameref*{def:Access_Matrix} with Domains as Objects}
  \label{tab:Access_Matrix_Domain_Objects}
\end{table}


%%% Local Variables:
%%% mode: latex
%%% TeX-master: "../../EDAF35-Operating_Systems-Reference_Sheet"
%%% End:


\subsection{Implementing Access Matrices}\label{subsec:Implement_Access_Matrices}
In general, an \nameref{def:Access_Matrix} will be sparse, i.e.\ most of the entries in the table will be empty.
Along with this, traditional sparse data structure techniques are not terribly useful for this.
These are because of the way the protection facilities are used.

Most systems use a combination of \nameref{def:Access_List}s and \nameref{def:Capability_List}s.
When a \nameref{def:Process} first tries to access an object, the access list is searched.
If access is denied, an exception condition occurs.
Otherwise, a capability is created and attached to the process.
Subsequent references use the capability to quickly determine access.
After the last access, the capability is destroyed.

\subsubsection{Global Table}\label{subsubsec:Global_Access_Matrix}
The simplest implementation of the access matrix is a global table consisting of a set of ordered tuples $\langle \text{domain}, \text{object}, \text{rights-set} \rangle$.
Whenever an operation $M$ is executed on an object $O_{j}$ within domain $D_{i}$, the global table is searched for a tuple $\langle D_{i}, O_{j}, R_{k} \rangle$, where $M \in R_{k}$.
If this tuple is found, the operation is allowed to continue; otherwise, an exception (or error) is raised.

This implementation suffers from several drawbacks.
The table is usually large and thus cannot be kept in main memory, so additional I/O is needed.
Virtual memory techniques are often used for managing this table (\nameref{def:Memory_Mapping}).
In addition, it is difficult to take advantage of special groupings of objects or domains.

\subsubsection{Access Lists}\label{subsubsec:Access_Lists}
\nameref{def:Access_List}s are created by only working with the columns of a \nameref{def:Access_Matrix}.

\begin{definition}[Access List]\label{def:Access_List}
  An \emph{access list} is a way for \textbf{objects} to track what domains can perform what actions on that object.
   The resulting list for each object consists of ordered pairs $\langle \text{domain}, \text{rights-set} \rangle$, which define all domains with a nonempty set of access rights for that object.
\end{definition}

This can be extended to define an \nameref{def:Access_List} and a default set of \nameref{def:Access_Right}s.
When an operation $M$ on an object $O_{j}$ is attempted in domain $D_{i}$, we search the access list for object $O_{j}$, looking for an entry $\langle D_{i}, R_{k} \rangle$ with $M \in R_{k}$.
If the entry is found, we allow the operation; if it is not, we check the default set.
If $M$ is in the default set, we allow the access.
Otherwise, access is denied, and an exception condition occurs.

Access lists correspond directly to the needs of users.
When a user creates an object, they can specify which domains can access the object, as well as what operations are allowed.
However, access-right information for a particular domain is not localized, making determining the set of access rights for each domain difficult.
In addition, every access to an object must be checked, requiring a search of the access list.
In a large system with long access lists, this search can be time consuming.

\subsubsection{Capability Lists}\label{subsubsec:Capability_Lists}
\nameref{def:Capability_List}s are created by only working with the rows of a \nameref{def:Access_Matrix}.

\begin{definition}[Capability List]\label{def:Capability_List}
  A \emph{capability list} is a way for \textbf{domains} to track what objects they can access and the actions the domain can perform on that object.
\end{definition}

To execute operation $M$ on object $O_{j}$, the process executes the operation $M$, specifying the capability (or pointer) for object $O_{j}$ as a parameter.
Just having the possession of a capability means that access to that object is allowed.

The \nameref{def:Capability_List} is associated with a \nameref{def:Protection_Domain}, but it is never directly accessible to a \nameref{def:Process} executing in that domain.
The list itself is a protected object, maintained by the \nameref{def:Operating_System} and only indirectly accessed by the user.
Capability-based protection relies on the fact that the capabilities are never allowed to migrate into any space directly accessible by a user process.
If all capabilities are secure, the object they protect is also secure against unauthorized access.

To provide protection, we must distinguish capabilities from other kinds of objects, and they must be interpreted for higher-level programs run.
Capabilities are usually distinguished from other data in one of two ways:
\begin{enumerate}[noitemsep]
\item Each object has a tag to denote whether it is a capability or accessible data.
  The tags themselves must not be directly accessible by an application program.
  Hardware or firmware support may be used to enforce this restriction.
  Although only one bit is necessary to distinguish between capabilities and other objects, more bits are often used.
  This extension allows all objects to be tagged with their types by the hardware.
  Thus, the hardware can distinguish integers, floating-point numbers, pointers, Booleans, characters, instructions, capabilities, and uninitialized values by their tags.
\item Alternatively, the address space associated with a \nameref{def:Process} can be split into two parts.
  One part is accessible to the program and contains the program’s normal data and instructions.
  The other part, containing the capability list, is accessible only by the operating system.
  \nameref{subsec:Segmentation} (\Cref{subsec:Segmentation}) is useful to support this approach.
\end{enumerate}

\nameref{def:Capability_List}s do not correspond directly to the needs of \nameref{def:User}s, but they are useful for localizing information for a given \nameref{def:Process}.
The process attempting access must present a capability for that access.
Then, the protection system needs only to verify that the capability is valid.
Revocation of capabilities, however, may be inefficient.


%%% Local Variables:
%%% mode: latex
%%% TeX-master: "../../EDAF35-Operating_Systems-Reference_Sheet"
%%% End:


\subsection{Access Control}\label{subsec:Access_Control}

%%% Local Variables:
%%% mode: latex
%%% TeX-master: "../../EDAF35-Operating_Systems-Reference_Sheet"
%%% End:



%%% Local Variables:
%%% mode: latex
%%% TeX-master: "../EDAF35-Operating_Systems-Reference_Sheet"
%%% End:
