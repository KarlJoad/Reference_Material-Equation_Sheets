\section{System Calls}\label{sec:System_Calls}
\begin{definition}[System Call]\label{def:System_Call}
  Software may trigger an interrupt by executing a special operation called a \emph{system call}.
  This can also be called a monitor call.
  A system call is a messaging interface between applications and the \nameref{def:Kernel}, with the applications issuing various requests and the \nameref{def:Kernel} fulfilling them or returning an error.

  System calls provide an interface to the services made available by an \nameref{def:Operating_System}.
  These services are a set of interfaces by which \nameref{def:Process}es running in \nameref{def:User}-space can interact with the system.
  These interfaces give \nameref{def:User}-level applications:
  \begin{itemize}[noitemsep]
  \item Controlled access to hardware
  \item A mechanism with which to create new \nameref{def:Process}es
  \item A mechanism to communicate with existing ones
  \item The capability to request other \nameref{def:Operating_System} resources
  \end{itemize}

  These calls are generally available as routines written in C and C++.
  Some of the lowest-level tasks (for example, tasks where hardware must be accessed directly) may be written using assembly.

  \begin{remark}[Syscall]\label{rmk:Syscall}
    In UNIX and UNIX-like systems, \nameref{def:System_Call} is usually shortened to \emph{syscall}.
  \end{remark}
\end{definition}

There are roughly 6 different types of system calls:
\begin{enumerate}[noitemsep]
\item \nameref{subsubsec:Process_Control}
  \begin{itemize}[noitemsep]
  \item End, Abort
  \item Load, Execute
  \item Create \nameref{def:Process}, Terminate \nameref{def:Process}
  \item Get process attributes, Set process attributes
  \item Wait for time
  \item Wait event, Signal event
  \item Allocate and Free memory
  \end{itemize}
\item \nameref{subsubsec:File_Manipulation}
  \begin{itemize}[noitemsep]
  \item Create file, Delete file
  \item Open, Close
  \item Read, Write, Reposition
  \item Get file attributes, Set file attributes
  \end{itemize}
\item \nameref{subsubsec:Device_Manipulation}
  \begin{itemize}[noitemsep]
  \item Request device, Release device
  \item Read, Write, Reposition
  \item Get device attributes, Set device attributes
  \item Logically attach or detach devices
  \end{itemize}
\item \nameref{subsubsec:Information_Maintenance}
  \begin{itemize}[noitemsep]
  \item Get time or date, Set time or date
  \item Get system data, Set system data
  \item Get \nameref{def:Process}, File, or Device attributes
  \item Set \nameref{def:Process}, File, or Device attributes
  \end{itemize}
\item \nameref{subsubsec:Communications}
  \begin{itemize}[noitemsep]
  \item Create, Delete communication connection
  \item Send, Receive messages
  \item Transfer status information
  \item Attach or Detach remote devices
  \end{itemize}
\item \nameref{subsubsec:Protection}
\end{enumerate}

\nameref{def:System_Call}s provide a layer between the hardware and \nameref{def:User}-space \nameref{def:Process}es.
This layer serves three primary purposes.
\begin{enumerate}[noitemsep]
\item It provides an abstracted hardware interface for \nameref{def:User}-space programs.
  When reading or writing from a file, applications do not have to be concerned with the type of disk, media, or even the type of filesystem on which the file resides.
\item \nameref{def:System_Call}s ensure system security and stability.
  With the \nameref{def:Kernel} acting as a middle-man between system resources and \nameref{def:User}-space, the \nameref{def:Kernel} can arbitrate access based on permissions, \nameref{def:User}s, and other criteria.
  This arbitration prevents applications from incorrectly using hardware, stealing other \nameref{def:Process}es’ resources, or otherwise doing harm to the system.
\item There is a single common layer between \nameref{def:User}-space and the rest of the system allows for the virtualized system provided to \nameref{def:Process}es.
\end{enumerate}

\begin{table}[h!tbp]
  \centering
  \begin{tabular}{lll}
    \toprule
    & \textbf{Windows} & \textbf{Unix} \\
    \midrule
    \nameref{subsubsec:Process_Control} & \mintinline{cpp}{CreateProcess()} & \mintinline{c}{fork()} \\
    & \mintinline{cpp}{ExitProcess()} & \mintinline{c}{exit()} \\
    & \mintinline{cpp}{WaitForSingleObject()} & \mintinline{c}{wait()} \\
    \midrule
    \nameref{subsubsec:File_Manipulation} & \mintinline{cpp}{CreateFile()} & \mintinline{c}{open()} \\
    & \mintinline{cpp}{ReadFile()} & \mintinline{c}{read()} \\
    & \mintinline{cpp}{WriteFile()} & \mintinline{c}{write()} \\
    & \mintinline{cpp}{CloseHandle()} & \mintinline{c}{close()} \\
    \midrule
    \nameref{subsubsec:Device_Manipulation} & \mintinline{cpp}{SetConsoleMode()} & \mintinline{c}{ioctl()} \\
    & \mintinline{cpp}{ReadConsole()} & \mintinline{c}{read()} \\
    & \mintinline{cpp}{WriteConsole()} & \mintinline{c}{write()} \\
    \midrule
    \nameref{subsubsec:Information_Maintenance} & \mintinline{cpp}{GetCurrentProcessID()} & \mintinline{c}{getpid()} \\
    & \mintinline{cpp}{SetTimer()} & \mintinline{c}{alarm()} \\
    & \mintinline{cpp}{Sleep()} & \mintinline{c}{sleep()} \\
    \midrule
    \nameref{subsubsec:Communications} & \mintinline{cpp}{CreatePipe()} & \mintinline{c}{pipe()} \\
    & \mintinline{cpp}{CreateFileMapping()} & \mintinline{c}{shm_open()} \\
    & \mintinline{cpp}{MapViewOfFile()} & \mintinline{c}{mmap()} \\
    \midrule
    \nameref{subsubsec:Protection} & \mintinline{cpp}{SetFileSecurity()} & \mintinline{c}{chmod()} \\
    & \mintinline{cpp}{InitializeSecurityDescriptor()} & \mintinline{c}{umask()} \\
    & \mintinline{cpp}{SetSecurityDescriptorGroup()} & \mintinline{c}{chown()} \\
    \bottomrule
  \end{tabular}
  \caption{System Calls in Unix and Windows}
  \label{tab:System_Calls_Examples}
\end{table}

\nameref{def:System_Call}s are exposed to the programmer by an \nameref{def:API}.
\begin{definition}[Application Programming Interface]\label{def:API}
  An \emph{Application Programming Interface} (\emph{API}) specifies a set of functions that are available to an application programmer.
  They specify the parameters that are passed to each function and the return values the programmer can expect.

  Typically, API calls perform \nameref{def:System_Call}s in the background, without the programmer knowing about them.
\end{definition}

The system call interface in Linux, as with most UNIX systems, is provided in part by the C library.
The C library implements the main \nameref{def:API} on Unix systems, including the standard C library and the system call interface.
The C library is used by all C programs and is easily wrapped by other programming languages for use in their programs.
POSIX is composed of a series of standards from the IEEE that aim to provide a portable \nameref{def:Operating_System} standard roughly based on UNIX.\@

\subsection{How to Use Syscalls}\label{subsec:How_To_Use_Syscalls}
\nameref{def:System_Call}s (often called \nameref{rmk:Syscall}s in Linux) are typically accessed via functionsdefined in the standard C library.
These functions can define an arbitrary number of arguments and might\footnote{Nearly all system calls have a side effect (they result in some change of the system’s state). A few syscalls, such as \mintinline{c}{getpid()}, do not have side effects and just return data from the \nameref{def:Kernel}.} result in one or more side effects, for example writing to a file or copying some data into a provided pointer.

\nameref{def:System_Call}s also provide a return value of type \mintinline{c}{long} that signifies success or error.
Usually, though not always, a negative return value denotes an error.
A return value of zero is usually (but again, not always) a sign of success.

The C library, when a \nameref{def:System_Call} returns an error, writes a special error code into the global \mintinline{c}{errno} variable.
This variable can be translated into human-readable errors via library functions such as \mintinline{c}{perror()}.

Finally, \nameref{def:System_Call}s have well-defined behavior.
For example, the \nameref{def:System_Call} \mintinline{c}{getpid()} is defined to return an integer that is the current \nameref{def:Process}’s PID.\@
However, the definition of behavior says nothing of the implementation to achieve this behavior.
The \nameref{def:Kernel} must provide the intended behavior of the \nameref{def:System_Call} but is free to do so with whatever implementation it wants as long as the result is correct.

\subsection{How are Syscalls Defined?}\label{subsec:How_Syscalls_Defined}
In this section, we will be analyzing the \mintinline{c}{getpid()} \nameref{def:System_Call}.
It is defined to return an \textbf{integer} (to the \nameref{def:User}-space) that represents the current \nameref{def:Process}'s PID.\@
The implementation of \mintinline{c}{getpid()} is shown below.
\inputminted[frame=lines,linenos]{c}{./EDAF35-Operating_Systems-Sections/System_Calls/Code/getpid_Implementation.c}

\mintinline{c}{SYSCALL_DEFINE0} is a macro that defines a system call with no parameters (hence the 0).
The expanded code looks like this:
\inputminted[frame=lines,linenos]{c}{./EDAF35-Operating_Systems-Sections/System_Calls/Code/getpid_Expanded.c}

\nameref{def:System_Call}s have a strict definition.
\begin{enumerate}[noitemsep]
\item The asmlinkage modifier on the function definition is a directive to tell the compiler to look only on the stack for this function’s arguments.
  \textbf{This is a required modifier for all system calls.}
\item The function returns a \mintinline{c}{long}.
  For compatibility between 32- and 64-bit systems, system calls defined to return an \mintinline{c}{int} in \nameref{def:User}-space return a \mintinline{c}{long} in the \nameref{def:Kernel}.
\item The \mintinline{c}{getpid()} \nameref{def:System_Call} is defined as \mintinline{c}{sys_getpid()} in the \nameref{def:Kernel}.
  \begin{itemize}[noitemsep]
  \item This is the naming convention taken with all \nameref{def:System_Call}s in Linux.
  \item \nameref{def:System_Call} \mintinline{c}{bar()} is implemented in the \nameref{def:Kernel} as function \mintinline{c}{sys_bar()}.
  \end{itemize}
\end{enumerate}


%%% Local Variables:
%%% mode: latex
%%% TeX-master: "../../EDAF35-Operating_Systems-Reference_Sheet"
%%% End:


\subsection{Syscall Handler}\label{subsec:Syscall_Handler}
It is not possible for \nameref{def:User}-space applications to simply execute a \nameref{def:Kernel}-function call to a function existing in \nameref{def:Kernel}-space because the \nameref{def:Kernel} exists in a protected memory space.
If applications could directly read and write to the \nameref{def:Kernel}’s address space, system security and stability would be nonexistent.

Instead, \nameref{def:User}-space applications must somehow signal to the \nameref{def:Kernel} that they want to execute a \nameref{def:System_Call} and have the system switch to \nameref{def:Kernel} mode, where the \nameref{def:System_Call} can be executed in \nameref{def:Kernel}-space by the \nameref{def:Kernel} on behalf of the application.
The mechanism to signal the \nameref{def:Kernel} is a \nameref{def:Trap}, a software interrupt.
Incur a \nameref{def:Trap}, and the system will switch to \nameref{def:Kernel}-mode and execute the exception handler.
However, in this case, the exception handler is actually the system call handler

\begin{center}
\large{\textbf{The important thing to note is that, somehow, \nameref{def:User}-space causes an exception or trap to enter the \nameref{def:Kernel}.}}
\end{center}

\subsubsection{Denoting Correct Syscall}\label{subsubsec:Denote_Correct_Syscall}
Simply entering \nameref{def:Kernel}-space alone is not sufficient because multiple \nameref{def:System_Call}s exist, all of which enter the \nameref{def:Kernel} in the same manner.
Thus, the \nameref{def:Syscall_Number} must be passed into the \nameref{def:Kernel}, usually through a \nameref{def:Register}.

\subsubsection{Parameter Passing}\label{subsubsec:Parameter_Passing}
In addition to the \nameref{def:Syscall_Number}, most \nameref{def:System_Call}s require that one or more parameters be passed to them.
Somehow, \nameref{def:User}-space must relay the parameters to the \nameref{def:Kernel} during the trap.
There are 2 main ways to do this.
\begin{enumerate}[noitemsep]
\item The easiest way is to store the parameters in registers
\item If there are not enough registers, or the parameter will not fit in a single register, one is filled with a  pointer to \nameref{def:User}-space memory where all the parameters are stored.
\end{enumerate}

The return value is sent back to \nameref{def:User}-space also by a register.

%%% Local Variables:
%%% mode: latex
%%% TeX-master: "../../EDAF35-Operating_Systems-Reference_Sheet"
%%% End:


\subsection{Syscall Functions}\label{subsec:Syscall_Functions}
\subsubsection{Process Control}\label{subsubsec:Process_Control}
A running program needs to be able to halt its own execution, either normally or abnormally.
If a \nameref{def:System_Call} is made to terminate the currently running program abnormally, or if the program runs into a problem and causes an error \nameref{def:Trap}, a dump of memory is sometimes taken and an error message generated.
The dump is written to disk and may be examined by a debugger—a system program designed to aid the programmer in finding and correcting errors, or bugs—to determine the cause of the problem.

Under either normal or abnormal circumstances, the \nameref{def:Operating_System} must transfer control to the invoking command interpreter.
The command interpreter then reads the next command.

To determine how bad the execution halt was, when the program ceases execution, it will return an exit code.
By convention, and for no other reason, an exit code of \texttt{0} is considered to be the program completed execution successfully.
Otherwise, the greater the return value, the greater the severity of the error.

\subsubsection{File Manipulation}\label{subsubsec:File_Manipulation}
We first need to be able to \kernelinline{create()} and \kernelinline{delete()} files.
Either \nameref{def:System_Call} requires the name of the file and perhaps some of the file’s attributes.
Once the file is created, we need to \kernelinline{open()} it and to use it.
We may then \kernelinline{read()}, \kernelinline{write()}, or perform any other \nameref{def:API}-defined action(s).
Finally, we need to \kernelinline{close()} the file, indicating that we are no longer using it.

We may need these same sets of operations for directories if we have a directory structure for organizing files in the file system.
In addition, for either files or directories, we need to be able to determine the values of various attributes and perhaps to reset them if necessary.

\begin{definition}[File Attribute]\label{def:File_Attribute}
  A \emph{file attribute} contains metadata about the file.
  This includes the file's name, type, protection codes, accounting information, and so on.
\end{definition}

\begin{remark*}
  If the system programs are callable by other programs, then each can be considered an \nameref{def:API} by other system programs.
\end{remark*}

\subsubsection{Device Manipulation}\label{subsubsec:Device_Manipulation}
\begin{definition}[Device]\label{def:Device}
  A \emph{device} in an \nameref{def:Operating_System} is a resource that must be controlled.
  Some of these devices are physical devices (for example, disk drives), while others can be thought of as abstract or virtual devices (for example, files).
\end{definition}

A system with multiple \nameref{def:User}s may require us to first \kernelinline{request()} a device, to ensure exclusive use of it.
After we are finished with the device, we \kernelinline{release()} it.
These functions are similar to the \kernelinline{open()} and \kernelinline{close()} \nameref{def:System_Call}s for files.
Other \nameref{def:Operating_System}s allow unmanaged access to devices.
The hazard then is the potential for device contention and perhaps \nameref{def:Deadlock}.

Once the device has been requested (and allocated to us), we can \kernelinline{read()}, \kernelinline{write()}, just as we can with files.
In fact, the similarity between I/O devices and files is so great that many \nameref{def:Operating_System}s, including \textsc{unix}, merge the two into a combined file–device structure.
In this case, a set of \nameref{def:System_Call}s can be shared between both files and \nameref{def:Device}s.
Sometimes, I/O devices are identified by special file names, directory placement, or file attributes.

\subsubsection{Information Maintenance}\label{subsubsec:Information_Maintenance}
Many \nameref{def:System_Call}s exist simply for the purpose of transferring information between the \nameref{def:User} program and the \nameref{def:Operating_System}.
For example, most systems have a \nameref{def:System_Call} to return the current \kernelinline{time()} and \kernelinline{date()}.
Other \nameref{def:System_Call}s may return information about the system, such as the number of current \nameref{def:User}s, the version number of the \nameref{def:Operating_System}, the amount of free memory or disk space, and so on.

Another set of \nameref{def:System_Call}s is helpful in debugging a program.
Many systems provide \nameref{def:System_Call}s to \kernelinline{dump()} memory.
A program \texttt{trace} lists each \nameref{def:System_Call} as it is executed.
In addition, the \nameref{def:Operating_System} keeps information about all its \nameref{def:Process}es, and \nameref{def:System_Call}s are used to access this information.

\subsubsection{Communications}\label{subsubsec:Communications}
Both of the models discussed are common in \nameref{def:Operating_System}s, and most systems implement both.
\nameref{par:Message_Passing} is useful for exchanging smaller amounts of data, because no conflicts need to be avoided.
It is also easier to implement than is shared memory for intercomputer communication.
\nameref{par:Shared_Memory} allows maximum speed and convenience of communication, since it can be done at memory transfer speeds when it takes place within a computer.
Problems exist, however, in the areas of protection and synchronization between the \nameref{def:Process}es sharing memory.

\paragraph{Message Passing}\label{par:Message_Passing}
Messages can be exchanged between the \nameref{def:Process}es either directly or indirectly through a common mailbox.
Before communication can take place, a connection must be opened.
The name of the other communicator must be known.
Each \nameref{def:Process} has a \emph{process name}, and this name is translated into an identifier, PID, by which the \nameref{def:Operating_System} can refer to the \nameref{def:Process}.
The \kernelinline{get_processid()} \nameref{def:System_Call} does this translation.
The identifiers are then passed to general-purpose \kernelinline{open()} and \kernelinline{close()} calls provided by the file system or to specific \kernelinline{open_connection()} and \kernelinline{close_connection()} \nameref{def:System_Call}s, depending on the model of communication.
The recipient \nameref{def:Process} usually must give its permission for communication to take place with an \kernelinline{accept_connection()} call.

Most \nameref{def:Process}es that will be receiving connections are special-purpose \nameref{def:Daemon}s.
They execute a \kernelinline{wait_for_connection()} call and are awakened when a connection is made.
The source of the communication, known as the client, and the receiving \nameref{def:Daemon}, known as a server, then exchange messages by using \kernelinline{read_message()} and \kernelinline{write_message()} \nameref{def:System_Call}s.
The \kernelinline{close_connection()} call terminates the communication.

\paragraph{Shared-Memory}\label{par:Shared_Memory}
In the shared-memory model, \kernelinline{shared_memory_create()} and \kernelinline{shared_memory_attach()} \nameref{def:System_Call}s are used by \nameref{def:Process}es to create and gain access to regions of memory owned by other \nameref{def:Process}es.
The \nameref{def:Operating_System} tries to prevent one \nameref{def:Process} from accessing another \nameref{def:Process}’s memory, so shared memory requires that two or more \nameref{def:Process}es agree to remove this restriction.
They can then exchange information by reading and writing data in the shared areas.
The form of the data is determined by the \nameref{def:Process}es and is not under the \nameref{def:Operating_System}’s control.
The \nameref{def:Process}es are also responsible for ensuring that they are not writing to the same location simultaneously.

\subsubsection{Protection}\label{subsubsec:Protection}
Protection provides a mechanism for controlling access to the resources provided by a computer system.
Historically, protection was a concern only on multiprogrammed computer systems with several \nameref{def:User}s.
However, with the advent of networking and the Internet, all computer systems, from servers to mobile handheld devices, must be concerned with protection.

%%% Local Variables:
%%% mode: latex
%%% TeX-master: "../../EDAF35-Operating_Systems-Reference_Sheet"
%%% End:



%%% Local Variables:
%%% mode: latex
%%% TeX-master: "../EDAF35-Operating_Systems-Reference_Sheet"
%%% End:
