\section{File-System Interface}\label{sec:FS_Interface}
So that the computer system will be convenient to use, the \nameref{def:Operating_System} provides a uniform logical view of stored information, the \nameref{def:File}.

\defFile{}

Files in all \nameref{def:Operating_System}s have attributes that help describe them.
\begin{itemize}[noitemsep]
\item \textbf{Name}.
  The symbolic file name is the only information kept in human-readable form.
\item \textbf{Identifier}.
  This unique tag, usually a number, identifies the file within the file system; it is the non-human-readable name for the file.
\item \textbf{Type}.
  This information is needed for systems that support different types of files.
\item \textbf{Location}.
  This information is a pointer to a device and to the location of the file on that device.
\item \textbf{Size}.
  The current size of the file and possibly the maximum allowed size are included in this attribute.
\item \textbf{Protection}.
  Access-control information determines who can do reading, writing, executing, etc.
\item \textbf{Time, date, and user identification}.
  This information may be kept for creation, last modification, and last use.
  This data can be useful for protection, security, and usage monitoring.
\end{itemize}

The information about all files is kept in the directory structure, which is also on secondary storage.
An entry in a directory consists of the file’s name and its unique identifier.
The file's identifier in turn locates the other file attributes.
It may take more than a kilobyte to record this information for each file.
Because directories must also be nonvolatile, they must be stored on the device and brought into memory as needed.

\subsection{File Operations}\label{subsec:File_Operations}
A file is an abstract data type, as such, to define a file properly, we need to consider the operations that can be performed on files.


%%% Local Variables:
%%% mode: latex
%%% TeX-master: "../../EDAF35-Operating_Systems-Reference_Sheet"
%%% End:


\subsection{File Types}\label{subsec:File_Types}
We need to decide whether the operating system should recognize and support \nameref{def:File} types.
If an operating system recognizes the type of a file, it can then operate on the file in reasonable ways.


%%% Local Variables:
%%% mode: latex
%%% TeX-master: "../../EDAF35-Operating_Systems-Reference_Sheet"
%%% End:



%%% Local Variables:
%%% mode: latex
%%% TeX-master: "../EDAF35-Operating_Systems-Reference_Sheet"
%%% End:
