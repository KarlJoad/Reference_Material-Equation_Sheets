\section{Main Memory}\label{sec:Main_Memory}
To truly benefit from \nameref{sec:CPU_Scheduling_Synchronization}, discussed in \Cref{sec:CPU_Scheduling_Synchronization}, we must be able to keep multiple \nameref{def:Process}es in memory at the same time.

In a \nameref{def:CPU}, the only things that can be accessed directly are \nameref{def:Register}s and main memory.
The CPU can reach the registers within one clock cycle, but accessing main memory takes several clock cycles, because access is done through the memory bus.
To prevent \nameref{def:Memory_Stall}s, we can use multiple \nameref{def:Thread}s, and we can add additional \nameref{def:Cache}s to the CPU itself.
This \nameref{def:Hardware} automatically speeds up memory access, without \nameref{def:Operating_System} intervention.

\begin{definition}[Cache]\label{def:Cache}
  A \emph{cache} is a very small amount of memory, built onto the \nameref{def:CPU} itself, and acts as a buffer between the CPU and main memory.
  A cache will have a copy of a small portion of what is in main memory, and the CPU goes to find the next thing, whatever it may be, from the cache first.
  Because the cache sits on the CPU itself, and does not have to interact with the memory bus, it is significantly faster than accessing main memory, but still slightly slower than accessing a register.
\end{definition}

\subsection{Address Binding}\label{subsec:Address_Binding}
Just as in programming languages, there are different possible times when a \nameref{def:Program} stored on a disk can have \nameref{def:Memory} addresses bound to it, this is called \nameref{def:Address_Binding}.

\begin{definition}[Address Binding]\label{def:Address_Binding}
  \emph{Address binding} is the act of putting ``something'' (a variable, a function, a value, anything) at a location in \nameref{def:Memory}.
  Because these memory locations have unique addresses assigned to them, the ``something'' that was put there is bound to that memory address.
\end{definition}

For a \nameref{def:Program} to be executed, it must be brought into memory and placed within a \nameref{def:Process}.
The processes that are waiting for their binary images to be brought to memory from the disk before beginning execution form the \nameref{def:Input_Queue}.

\begin{definition}[Input Queue]\label{def:Input_Queue}
  The \emph{input queue} is the queue in which \nameref{def:Process}es are placed while they wait for their \nameref{def:Program} binary image to arrive from the disk.
\end{definition}

There are 3 major times when \nameref{def:Address_Binding} can occur:
\begin{enumerate}[noitemsep]
\item \nameref{subsubsec:Compile_Time_Address_Binding}
\item \nameref{subsubsec:Load_Time_Address_Binding}
\item \nameref{subsubsec:Execution_Time_Address_Binding}
\end{enumerate}

\subsubsection{Compile-Time Address Binding}\label{subsubsec:Compile_Time_Address_Binding}
If the addresses that the \nameref{def:Process} will use are known at compile-time, then \nameref{def:Absolute_Code} can be generated.

\begin{definition}[Absolute Code]\label{def:Absolute_Code}
  In \emph{absolute code}, the memory location of the \nameref{def:Process} is fixed during every execution, and because the \nameref{def:Program} is always the same size, anything that requires a memory can be referenced by this unchanging value.
\end{definition}

If, at some later time, the starting location changes, then the whole program must be recompiled.
In the case of assembly languages, this might mean recalculating the absolute memory locations of everything in the program.

These reasons led to the developement of relative addressing in assembly languages and the use of \nameref{subsubsec:Load_Time_Address_Binding}.

\subsubsection{Load-Time Address Binding}\label{subsubsec:Load_Time_Address_Binding}
\subsubsection{Execution-Time Address Binding}\label{subsubsec:Execution_Time_Address_Binding}

%%% Local Variables:
%%% mode: latex
%%% TeX-master: "../../EDAF35-Operating_Systems-Reference_Sheet"
%%% End:


\subsection{Logical, Physical, Virtual Address Spaces}\label{subsec:Logical_Physical_Virtual_Address_Space}
The compile-time and load-time address-binding methods generate identical logical and physical addresses.

\begin{definition}[Logical Address]\label{def:Logical_Address}
  The CPU generates virtual/\emph{logical address}es.
  How these logical addresses look, behave, and correspond to \nameref{def:Physical_Address}es depend on the memory management technique used by the \nameref{def:Memory_Management_Unit}.

  The user program deals with logical addresses.
  The memory-mapping hardware converts logical addresses into \nameref{def:Physical_Address}es.
\end{definition}

\begin{definition}[Physical Address]\label{def:Physical_Address}
  The \emph{physical address} is the actual value of a particular byte of memory's address.
  How the physical address is calculated from the \nameref{def:Logical_Address} is handled by the \nameref{def:Memory_Management_Unit}.

  The user program deals with \nameref{def:Logical_Address}es.
  The memory-mapping hardware converts logical addresses into physical addresses.
\end{definition}

However, the execution-time address-binding scheme results in differing logical and physical addresses, so we call the \nameref{def:Logical_Address} a \nameref{def:Virtual_Address} instead.

\begin{definition}[Virtual Address]\label{def:Virtual_Address}
  The CPU generates logical/\emph{virtual address}es.
  How these virtual addresses look, behave, and correspond to \nameref{def:Physical_Address}es depend on the memory management technique used by the \nameref{def:Memory_Management_Unit}.

  The user program deals with logical/virtual addresses.
  The memory-mapping hardware converts logical/virtual addresses into \nameref{def:Physical_Address}es.

  \begin{remark}[Use of Logical and Virtual Address]
    In this document, I use \nameref{def:Logical_Address} and \nameref{def:Virtual_Address} interchangeably.
  \end{remark}
\end{definition}

The set of all logical addresses/\nameref{def:Virtual_Address}es generated by a program is a \nameref{def:Logical_Address_Space}/\nameref{def:Virtual_Address_Space}.

\begin{definition}[Logical Address Space]\label{def:Logical_Address_Space}
  The \emph{logical address space} consists of all the \nameref{def:Logical_Address}es generated by a \nameref{def:Program}.

  \begin{remark}
    The \nameref{def:Logical_Address_Space} is only calculated by one \nameref{def:Program}/\nameref{def:Process} at a time.
    To find the total logical address space used, all \nameref{def:Process}es must have their logical address spaces aggregated.
  \end{remark}
\end{definition}

The set of all \nameref{def:Physical_Address}es corresponding to these logical addresses is the \nameref{def:Physical_Address_Space}.

\begin{definition}[Physical Address Space]\label{def:Physical_Address_Space}
  The \emph{physical address space} consists of all the \nameref{def:Physical_Address}es that the \nameref{def:Logical_Address}es/\nameref{def:Virtual_Address}es map to.

  \begin{remark}
    The \nameref{def:Physical_Address_Space} is only calculated by one \nameref{def:Program}/\nameref{def:Process} at a time.
    To find the total physical address space used, all \nameref{def:Process}es must have their physical address spaces aggregated.
  \end{remark}
\end{definition}

Thus, in the execution-time address-binding scheme, the logical and physical address spaces differ.
The \nameref{def:Memory_Management_Unit} handles the mapping of the \nameref{def:Virtual_Address_Space}/\nameref{def:Logical_Address_Space} to the \nameref{def:Physical_Address_Space}.
The \nameref{def:Program} generates only \nameref{def:Logical_Address}es and thinks that the \nameref{def:Process} runs in memory locations from $0$ to $max$.

For example, assume a \nameref{def:Process}'s base register, for the lowest valid memory address is $R$.
If we wanted to access the 14th byte from the beginning of the \nameref{def:Process}, the CPU would attempt to retrieve $14$, the \nameref{def:Logical_Address}.
However, on the way to memory, the \nameref{def:Memory_Management_Unit} would intercept this and redirect the access to $R+14$, the \nameref{def:Physical_Address}.

%%% Local Variables:
%%% mode: latex
%%% TeX-master: "../../EDAF35-Operating_Systems-Reference_Sheet"
%%% End:


\subsection{Dynamic Loading}\label{subsec:Dynamic_Loading}
Because of the way we have set up our memory and \nameref{def:Process}es before, a \nameref{def:Process} could only be as big as our physical memory.
However, we can skirt these issues by using \nameref{def:Dynamic_Loading}.

\begin{definition}[Dynamic Loading]\label{def:Dynamic_Loading}
  \emph{Dynamic Loading} is used to load routines into memory only when needed.
  The routines that can by dynamically loaded must be stored on disk in a \nameref{def:Relocatable_Code} format.
  If the main program that was loaded in and began execution needs a routine, it checks to see if the routine has been loaded.
  If it has not, the loader loads the desired routine into memory, then updates the \nameref{def:Process}'s address tables to point to the newly retrieved routine.
\end{definition}

The advantage of \nameref{def:Dynamic_Loading} is that a routine is loaded only when it is needed.
This is particularly useful when large amounts of code are needed to handle infrequently occurring cases, such as error routines.
This allows the total program size to be large, but the portion used (and hence loaded) be much smaller.

\nameref{def:Dynamic_Loading} does not require special support from the operating system.
It is the responsibility of the users to design their programs to take advantage of such a method.
\nameref{def:Operating_System}s may help the programmer, however, by providing library routines to implement \nameref{def:Dynamic_Loading}.

%%% Local Variables:
%%% mode: latex
%%% TeX-master: "../../EDAF35-Operating_Systems-Reference_Sheet"
%%% End:


\subsection{Dynamic Linking}\label{subsec:Dynamic_Linking}
\nameref{def:Operating_System}s provide many useful libraries to allow programmers and their work to use higher-level abstractions to make their programming life easier.
An example of this is a language library for localization, allowing a programmer to write their input and output statements in one language and it be translated to another automatically.

To achieve this, \nameref{def:Dynamic_Linking} is used, and \nameref{def:Stubs} are generated during the compilation process to inform the running \nameref{def:Process} where to find the necessary information.

\begin{definition}[Dynamic Linking]\label{def:Dynamic_Linking}
  A \nameref{def:Program} that makes use of \emph{dynamic linking} is compiled like normal.
  However, the library routines that would be provided by a \emph{dynamically linked library} (\emph{Shared Library}) are left as \nameref{def:Stub}s.
  When the program begins execution, the \nameref{def:Process} executes like normal, potentially using \nameref{def:Dynamic_Loading} in the meantime.
  However, once the \nameref{def:Process} reaches a \nameref{def:Stub}, it will: proceed like normal (if the stub is replaced by the actual code), or will fetch and load the routine, replace the stub, and then execute the routine.

  \begin{remark}[Dynamic Linking vs. Dynamic Loading]\label{rmk:Dynamic_Linking_vs_Dynamic_Loading}
    In \nameref{def:Dynamic_Linking}, a \nameref{def:Program} is compiled down to the machine code, with the \nameref{def:Stub}s in place, which are replaced \textbf{DURING} program execution.
    This allows us to share a binary library file \textbf{BETWEEN DIFFERENT} \nameref{def:Program}s.
    Whereas, in \nameref{def:Dynamic_Loading}, the code is brought in from memory from the \textbf{same} \nameref{def:Program} binary image.

    This means that if an \nameref{def:Operating_System} did not support \nameref{def:Dynamic_Linking}, each \nameref{def:Program} that required a certain library must include it.
    This leads to space inefficiencies on both the disk and in main memory.
  \end{remark}
\end{definition}

\begin{definition}[Stub]\label{def:Stub}
  A \emph{stub} is a small piece of code that indicates how to locate the appropriate memory-resident library routine or how to load the library if the routine is not already present.
  A stub is included in the \nameref{def:Program}'s binary image \textbf{FOR EACH} library-routine reference.

  When the stub is executed, it checks to see whether the needed routine is already in memory.
  If it is not, the program loads the routine into memory and replaces the stub with the address of the routine and executes the routine.
  Then, the next time that routine is reached, the routine is executed directly, incurring no cost for \nameref{def:Dynamic_Linking}.
\end{definition}

This feature can be extended to handle library updates (such as bug fixes).
If a library is replaced by a new version, all programs that use that library will automatically use the new version.
Without \nameref{def:Dynamic_Linking}, all such programs would need to be relinked to gain access to the new library.
Version information is included in the \nameref{def:Program} and the library so that programs will not accidentally execute new, incompatible versions of libraries.
Multiple versions of the same library may be loaded into memory, and each program uses its version information to decide which library to use.

Unlike \nameref{def:Dynamic_Loading}, \nameref{def:Dynamic_Linking} and dynamically linked libraries require help from the \nameref{def:Operating_System}.
If the \nameref{def:Process}es in memory are protected from one another, then the operating system is the only entity that can check all the \nameref{def:Processes}.
Only the OS can:
\begin{itemize}[noitemsep]
\item See if the needed routine is in another process’s memory space.
\item Allow multiple processes to access the same memory addresses.
\end{itemize}

\subsubsection{Static vs. Dynamic Linking}\label{subsubsec:Static_vs_Dynamic_Linking}
As programmers, we are more familiar with \nameref{def:Static_Linking}; where a library is compiled \textbf{INTO} and assembled \textbf{WITH} our code \textbf{INTO} our executable binary image.

\begin{definition}[Static Linking]\label{def:Static_Linking}
  \emph{Static linking} is the act of running a linker after the compiler has generated \nameref{def:Stub}s for library routines.
  By running the linker, the library's code is brought into our program before it is assembled.
  This means that the library (and all the code supporting the library) is compiled \textbf{INTO} our binary image as well.
\end{definition}

%%% Local Variables:
%%% mode: latex
%%% TeX-master: "../../EDAF35-Operating_Systems-Reference_Sheet"
%%% End:


\subsection{Swapping}\label{subsec:Swapping}
A \nameref{def:Process} can only be executed if it is in memory.
However, while it is \textbf{NOT} being executed, it is not required for it to be in memory.
This idea forms the basis of \nameref{def:Swapping}.

\begin{definition}[Swapping]\label{def:Swapping}
  \emph{Swapping} is the action of moving a \nameref{def:Process} out of memory to a \nameref{def:Backing_Store}, or vice versa.
\end{definition}

\begin{definition}[Backing Store]\label{def:Backing_Store}
  A \emph{backing store} is another form of storage media.
  There are typically very few requirements on the type of storage media used in a backing store, but it \textbf{MUST} be readily assessible, and be quick (not as fast as RAM, but faster than traditional file system storage).
  Typically, a fast disk is used as the backing store.

  \begin{remark}[Swap]\label{rmk:Swap}
    Sometimes the \nameref{def:Backing_Store} is called \emph{the swap}, \emph{swap-area}, \emph{swap partition}, \emph{swapfile}, etc.
    They all mean the same thing\footnote{Swap Partition and Swapfile have specify the type of the \nameref{def:Backing_Store}}.
  \end{remark}
\end{definition}

\nameref{def:Swapping} allows for the total \nameref{def:Physical_Address_Space} of \textbf{ALL} \nameref{def:Process}es to exceed the real physical memory of the system.

%%% Local Variables:
%%% mode: latex
%%% TeX-master: "../../EDAF35-Operating_Systems-Reference_Sheet"
%%% End:


%%% Local Variables:
%%% mode: latex
%%% TeX-master: "../EDAF35-Operating_Systems-Reference_Sheet"
%%% End:
