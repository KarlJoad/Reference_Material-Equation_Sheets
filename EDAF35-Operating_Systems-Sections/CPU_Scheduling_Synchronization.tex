\section{CPU Scheduling and Synchronization}\label{sec:CPU_Scheduling_Synchronization}
The growing importance of multicore systems has brought an increased emphasis on developing multithreaded applications.
In such applications, several threads, which may be sharing data, are running in parallel on different processing cores.
These \nameref{def:Process}es are called \nameref{def:Cooperating_Process}es.

\begin{definition}[Cooperating Process]\label{def:Cooperating_Process}
  A \emph{cooperating process} is one that can affect or be affected by other \nameref{def:Process}es executing on a system.
  They can share a logical address space (code and data), \textbf{\nameref{def:Thread}s}, or can share data through files and/or messages, \textbf{\nameref{subsubsec:Communications}}.
\end{definition}

Given the way that multiple \nameref{def:Thread}s can be scheduled, namely in any order (relatively speaking), as programmers, we cannot be certain about which thread will be scheduled first.
This leads to all sorts of problems because of sharing information between multiple users.
The largest, and likely the most common, error in a multi\nameref{def:Thread}ed program is the \nameref{def:Race_Condition}.

\subsection{Process/Thread Synchronization}\label{subsec:Synchronization}
The main problem that occurs in multi\nameref{def:Thread}ed programs is that there is a small portion of code that is a \nameref{def:Critical_Section}.
This leads to the development of the \nameref{subsubsec:Critical_Section_Problem}.

\begin{definition}[Critical Section]\label{def:Critical_Section}
  The \emph{critical section} of a \nameref{def:Process} is a portion where the \nameref{def:Thread} and/or \nameref{def:Process} is changing common variables, updating a table, writing a file, or other global state changes.
\end{definition}

\subsubsection{Critical Section Problem}\label{subsubsec:Critical_Section_Problem}
The \emph{Critical Section Problem} is the issue of coordinating multiple \nameref{def:Thread}s about a \nameref{def:Critical_Section} of the code.
The problem is to design a protocol that the \nameref{def:Process}es/\nameref{def:Thread}s can use to cooperate.
Each \nameref{def:Process} must request permission to enter its critical section.
The section of code implementing this request is the entry section.
The critical section may be followed by an exit section.
The remaining code is the remainder section.


%%% Local Variables:
%%% mode: latex
%%% TeX-master: "../../EDAF35-Operating_Systems-Reference_Sheet"
%%% End:

\begin{definition}[Deadlock]\label{def:Deadlock}
  \emph{Deadlock} is when 2 processes require information from each other to continue running.
  If this happens, neither process will provide the other with its required information, so they will both wait for each other, forever.
\end{definition}

%%% Local Variables:
%%% mode: latex
%%% TeX-master: "../EDAF35-Operating_Systems-Reference_Sheet"
%%% End:
