\section{CPU Scheduling and Synchronization}\label{sec:CPU_Scheduling_Synchronization}
The growing importance of multicore systems has brought an increased emphasis on developing multithreaded applications.
In such applications, several threads, which may be sharing data, are running in parallel on different processing cores.
These \nameref{def:Process}es are called \nameref{def:Cooperating_Process}es.

\begin{definition}[Cooperating Process]\label{def:Cooperating_Process}
  A \emph{cooperating process} is one that can affect or be affected by other \nameref{def:Process}es executing on a system.
  They can share a logical address space (code and data), \textbf{\nameref{def:Thread}s}, or can share data through files and/or messages, \textbf{\nameref{subsubsec:Communications}}.
\end{definition}

Given the way that multiple \nameref{def:Thread}s can be scheduled, namely in any order (relatively speaking), as programmers, we cannot be certain about which thread will be scheduled first.
This leads to all sorts of problems because of sharing information between multiple users.
The largest, and likely the most common, error in a multi\nameref{def:Thread}ed program is the \nameref{def:Race_Condition}.

\begin{definition}[Race Condition]\label{def:Race_Condition}
  A \emph{race condition} is when several processes access and manipulate the same data concurrently and the outcome of the execution depends on the particular order in which the access takes place.
  The only way to prevent a race condition is to ensure that \textbf{only one \nameref{def:Thread} can change the value at a time}.
\end{definition}

\subsection{Process/Thread Synchronization}\label{subsec:Synchronization}
The main problem that occurs in multi\nameref{def:Thread}ed programs is that there is a small portion of code that is a \nameref{def:Critical_Section}.
This leads to the development of the \nameref{subsubsec:Critical_Section_Problem}.

\begin{definition}[Critical Section]\label{def:Critical_Section}
  The \emph{critical section} of a \nameref{def:Process} is a portion where the \nameref{def:Thread} and/or \nameref{def:Process} is changing common variables, updating a table, writing a file, or other global state changes.
\end{definition}

\subsubsection{Critical Section Problem}\label{subsubsec:Critical_Section_Problem}
The \emph{Critical Section Problem} is the issue of coordinating multiple \nameref{def:Thread}s about a \nameref{def:Critical_Section} of the code.
The problem is to design a protocol that the \nameref{def:Process}es/\nameref{def:Thread}s can use to cooperate.
Each \nameref{def:Process} must request permission to enter its critical section.
The section of code implementing this request is the \nameref{def:Entry_Section}.
The critical section may be followed by an \nameref{def:Exit_Section}.
The remaining code is the \nameref{def:Remainder_Section}.

\begin{definition}[Entry Section]\label{def:Entry_Section}
  The \emph{entry section} of a \nameref{def:Process} is the portion where the request to execute the \nameref{def:Critical_Section} occurs.
  In the case of a \nameref{def:Mutex}, this is the process of aquiring the it.
  For a \nameref{def:Semaphore}, it is the process of manipulating the value it currently contains.
\end{definition}

\begin{definition}[Exit Section]\label{def:Exit_Section}
  In the \emph{exit section}, the constructs used to ensure coordination in the \nameref{def:Critical_Section} are freed.
  In the case of a \nameref{def:Mutex}, this is the process of releasing the it.
  For a \nameref{def:Semaphore}, it is the process of manipulating the value it currently contains in the opposite direction it was initially manipulated by.
\end{definition}

\begin{definition}[Remainder Section]\label{def:Remainder_Section}
  The \emph{remainder section} is the rest of the code, after this \nameref{def:Critical_Section}.
  This code may be parallelized, or not.
  It could contain further \nameref{def:Critical_Section}s.
\end{definition}

Any solution to this problem \textbf{MUST} satisfy one of the following 3 requirements:
\begin{enumerate}[noitemsep]
\item \textbf{Mutual Exclusion}.
  If \nameref{def:Process} $P_{i}$ is executing its \nameref{def:Critical_Section}, then \textbf{no other} processes can execute their critical sections.
\item \textbf{Progress}.
  If no \nameref{def:Process} is executing its \nameref{def:Critical_Section}, and some processes wish to enter their critical sections, then only those processes that \textbf{are not executing} in their \nameref{def:Remainder_Section}s can decide which will enter the \nameref{def:Critical_Section} next.
  Essentially, the only way a process gets a voice in the choice is by not having executed the critical section yet.
\item \textbf{Bounded Waiting}.
  There exists a bound on the number of times that other \nameref{def:Process}es are allowed to enter their \nameref{def:Critical_Section}s after a process has made a request to enter its critical section and before that request is granted.
\end{enumerate}

To handle the \nameref{subsubsec:Critical_Section_Problem}, there are 2 main types of \nameref{def:Kernel}s that present solutions.
\begin{enumerate}[noitemsep]
\item \nameref{def:Nonpreemptive_Kernel}s. Not used frequently today.
\item \nameref{def:Preemptive_Kernel}s. The most common type today.
\end{enumerate}

\begin{definition}[Nonpreemptive Kernel]\label{def:Nonpreemptive_Kernel}
  A \emph{nonpreemptive kernel} is a \nameref{def:Kernel} that does \textbf{NOT} use \nameref{def:Preemption} on \nameref{def:Process}es or \nameref{def:Thread}s running in kernel-mode.
\end{definition}

\begin{definition}[Preemptive Kernel]\label{def:Preemptive_Kernel}
  A \emph{preemptive kernel} is a \nameref{def:Kernel} that uses \nameref{def:Preemption} on \nameref{def:Process}es or \nameref{def:Thread}s running in kernel-mode.
  This means that we cannot say anything definitive about the state of the \nameref{def:Kernel}'s data structures at a given time, because we cannot say which process/thread is running at that time.
\end{definition}

\subsubsection{Hardware Support for Synchronization}\label{subsubsec:Hardware_Support_Synchronization}
Software-based solutions to handling multithreading and multiprocessing tends to be better than hardware-based solutions, as they are more flexible.
Many of the solutions that will be presented here are based on the idea of \textbf{\nameref{def:Lock}ing}.

\begin{definition}[Lock]\label{def:Lock}
  A \emph{lock} allows \textbf{only one} \nameref{def:Thread} to enter the portion of code that is locked.
  While a thread holds this lock no other \nameref{def:Thread} can execute on this code portion.

  \begin{remark}[Binary Semaphore]\label{rmk:Binary_Semaphore}
    Locks can be represented as \emph{binary \nameref{def:Semaphore}}s.
  \end{remark}
\end{definition}

In a single-processor system, we can solve the \nameref{subsubsec:Critical_Section_Problem} by preventing interrupts from being handled.
This would prevent the currently running instruction from being preempted in any way, and allow it to finish.
However, this does not really work on a multiprocessor system, because disabling interrupts and their handling on all processors is time consuming.

However, the idea of certain instructions being \nameref{def:Atomic} is an elegant solution to the \nameref{subsubsec:Critical_Section_Problem}.
So, most computer systems provide special hardware-level instructions that allow us to test and modify the contents of a word, or swap the contents of 2 words \nameref{def:Atomic}ally.

\begin{definition}[Atomic]\label{def:Atomic}
  An \emph{atomic} operation is one that cannot be interrupted, preempted, or altered in any way.
  As soon as an atomic operation begins, the system \textbf{MUST} finish handling it before it may do anything else.
\end{definition}

Some operations on data are possible to do at any given point in time, without affecting the potential outcome.
One example of this is \textbf{reading} from a location in memory.
However, if this location can also be written to, we need to limit the number of writers.
Additionally, if someone is waiting to write, they should get some priority over anything waiting to read.
Thus, the \nameref{def:Read_Write_Lock} was created.

\begin{definition}[Read/Write Lock]\label{def:Read_Write_Lock}
  \emph{Read/Write Lock}s allow either an unlimited number of readers \textbf{OR} 1 writer at any given time.
  Writers will be scheduled to use the lock sooner than readers, so the value is updated first, before anyone reads it again.
  But, the writer will have to wait until everyone currently reading the value is done reading, otherwise the value in memory will change underneath the readers.
\end{definition}

\subsubsection{Mutex Locks}\label{subsubsec:Mutex_Locks}
The hardware-based solutions presented in \Cref{subsubsec:Hardware_Support_Synchronization} are typically not available to application programmers.
Instead, operating system designers build software tools to handle the \nameref{subsubsec:Critical_Section_Problem}.
The simplest tool is that of a \nameref{def:Mutex}.

%%% Local Variables:
%%% mode: latex
%%% TeX-master: "../../EDAF35-Operating_Systems-Reference_Sheet"
%%% End:


\subsection{Scheduling}\label{subsec:Scheduling}
CPU scheduling is the basis of multiprogrammed operating systems.
In a single-processor system, only one process can run at a time.
Others must wait until the CPU is free and can be rescheduled.
By switching the CPU among processes, the operating system can maximize CPU utilization.

For example, a \nameref{def:Process} is executed until it must wait.
Typically the process waits for the completion of some I/O request.
In a simple computer system, the CPU then just sits idle.
All this waiting time is wasted; no useful work is accomplished.

\begin{blackbox}
  \textbf{On operating systems that support them, \nameref{def:Kernel_Thread}s, not \nameref{def:Process}es are scheduled by the operating system.}
  However, the terms ``process scheduling'' and ``thread scheduling'' are often used interchangeably.
  Process scheduling is used when discussing general scheduling concepts and thread scheduling to refer to thread-specific ideas.
\end{blackbox}

Scheduling of this kind is a fundamental operating-system function.
Almost all computer resources are scheduled before use.

\subsubsection{CPU and I/O Bursts}\label{subsubsec:CPU_IO_Bursts}
To properly schedule a \nameref{def:Process}, its \nameref{def:CPU_Burst}s and \nameref{def:I/O_Burst}s need to observed.

\begin{definition}[CPU Burst]\label{def:CPU_Burst}
  A \emph{CPU burst} is one of the states of execution for a \nameref{def:Process}.
  This is the state when the process is actively using the CPU to perform computations.
  In this state, the CPU is performing activity for \textbf{this} \nameref{def:Process}, and \textbf{IS NOT} waiting for an I/O device to perform some action or return information.
\end{definition}

\begin{definition}[I/O Burst]\label{def:I/O_Burst}
  An \emph{I/O burst} is one of the states of execution for a \nameref{def:Process}.
  This is the state when the process is waiting on the I/O device to return the requested information or perform the desired action.
  In this state, the CPU is doing no activity for \textbf{this} \nameref{def:Process}.
\end{definition}

A \nameref{def:Process} alternates between these two bursts, with the final \nameref{def:CPU_Burst} terminating this \nameref{def:Process}'s execution.
The distribution of length of CPU bursts is an exponential or hyperexponential graph.
This means:
\begin{itemize}[noitemsep]
\item There is a large number of short duration CPU bursts.
\item There is a small number of long duration CPU bursts.
\end{itemize}

We can categorize these into either \nameref{def:CPU_Bound} programs or \nameref{def:IO_Bound} programs.
\begin{itemize}[noitemsep]
\item I/O-bound programs have a small number of CPU bursts which have a relatively short duration relative to the I/O operations.
  The I/O operations take up a majority of the time the \nameref{def:Process} executes.
\item CPU-bound programs have a large number of CPU bursts, which have a relatively long duration relative to the I/O operations.
  The CPU operatiosn take up a majority of the time the \nameref{def:Process} executes.
\end{itemize}

\subsubsection{CPU Scheduler}\label{subsubsec:CPU_Scheduler}
Whenever the CPU becomes idle, i.e.\ it has finished the current CPU burst early, or there is an I/O operation, the \nameref{def:Operating_System} must select the next \nameref{def:Process} and/or \nameref{def:Thread} to schedule.
This is handled by the \nameref{def:Short_Term_Scheduler}.

\begin{definition}[Short-Term Scheduler]\label{def:Short_Term_Scheduler}
  The \emph{short-term scheduler} is responsible for scheduling either the next \nameref{def:Process} or \nameref{def:Thread} for execution on the \textbf{CPU} from all the possible ones in memory.
  This is run quite frequently, every couple hundred milliseconds, usually.

  \begin{remark}[CPU Scheduler]\label{rmk:CPU_Scheduler}
    Because the \nameref{def:Short_Term_Scheduler} only schedules tasks for the CPU, it is also called the \emph{CPU Scheduler}.
  \end{remark}
\end{definition}

\paragraph{Preemption and Scheduling}\label{par:Preemption_Scheduling}
There are 4 times when CPU scheduling occurs:
\begin{enumerate}[noitemsep]
\item When a process switches from the \texttt{RUNNING} state to the \texttt{WAITING} state.
\item When a process switches from the \texttt{RUNNING} state to the \texttt{READY} state (for example, when an interrupt occurs).
\item When a process switches from the \texttt{WAITING} state to the \texttt{READY} state (for example, at completion of I/O).
\item When a process terminates.
\end{enumerate}

%%% Local Variables:
%%% mode: latex
%%% TeX-master: "../../EDAF35-Operating_Systems-Reference_Sheet"
%%% End:


\subsection{Thread Scheduling}\label{subsec:Thread_Scheduling}

%%% Local Variables:
%%% mode: latex
%%% TeX-master: "../../EDAF35-Operating_Systems-Reference_Sheet"
%%% End:


\subsection{Real-Time Scheduling}\label{subsec:Real_Time_Scheduling}
Real-time operating systems have their own class of scheduling issues.
This depends on whether the \nameref{def:Operating_System} is a \nameref{def:Soft_Real_Time_System} or a \nameref{def:Hard_Real_Time_System}.

\begin{definition}[Soft Real-Time System]\label{def:Soft_Real_Time_System}
  \emph{Soft real-time system}s do not provide a guarantee about the scheduling of a critical real-time process.
\end{definition}

\begin{definition}[Hard Real-Time System]\label{def:Hard_Real_Time_System}
  \emph{Hard real-time system}s guarantee the execution time of a real-time process.
  These tasks will be serviced by its deadline, otherwise the process will not be executed at all.
\end{definition}

POSIX also provides support for real-time scheduling through 2 functions with 2 scheduling types.
\begin{enumerate}[noitemsep]
\item \kernelinline{pthread_attr_getsched_policy(pthread_attr_t *attr, int *policy)}
\item \kernelinline{pthread_attr_setsched_policy(pthread_attr_t *attr, int policy)}
\end{enumerate}
\begin{enumerate}[noitemsep]
\item \texttt{SCHED\_FIFO}
\item \texttt{SCHED\_RR}
\end{enumerate}

\subsubsection{Minimizing Latency}\label{subsubsec:Minimizing_Latency}
The key aspect here is the amount of time it takes for a system to respond to an event.
This is called \nameref{def:Event_Latency}.

\begin{definition}[Event Latency]\label{def:Event_Latency}
  \emph{Event latency} is the amount of time that elapses from when an event occurs to when it is serviced.
  Different events can have different event latency requirements.
\end{definition}

There are 2 factors that affect \nameref{def:Event_Latency}.
\begin{enumerate}[noitemsep]
\item Interrupt Latency.
  The amount of time from the arrival of an \nameref{def:Interrupt} to the start of the Interrupt Service Routine (ISR).
  This includes the amount of time needed to get the currently running instruction to a point where it can be switched.
  Also included is the amount of time needed to perform the switch.
\item Dispatch Latency the amount of time the scheduler needs to stop one process and start another.
  There are 2 parts that affect the value of the dispatch latency:
  \begin{enumerate}[noitemsep]
  \item \nameref{def:Preemption} of \textbf{ANY} process running tin the kernel.
  \item Release of resources used by low-priority process for higher-priority processes.
  \end{enumerate}
\end{enumerate}

\subsubsection{Scheduling}\label{subsubsec:Real_Time_Scheduling}
In this case, there are not as many choices of \nameref{def:Scheduling_Algorithm} for real-time systems as other systems.
All algorithms must be roughly based on a priority-based system all of which must support \nameref{def:Preemption}.

Most modern \nameref{def:Operating_System}s offer support for \nameref{def:Soft_Real_Time_System}s with their scheduling priorities.
Note however, that pure priority-based algorithms only guarantee soft real-time functionality, not hard.

Processes are considered periodic; they require the CPU at constant intervals.
Once a periodic process has acquired the CPU, it has a fixed processing time $t$, a deadline $d$ by which it must be serviced by the CPU, and a period $p$.
The relationship of the processing time, the deadline, and the period can be expressed as $0 \leq t \leq d \leq p$.
The rate of a periodic task is $\frac{1}{p}$.
Schedulers can take advantage of these characteristics and assign priorities according to a process’s deadline or rate requirements.
What is different about this algorithm is a process may have to announce its deadline to the scheduler.
Using an admission-control algorithm, the scheduler either admits the process, guaranteeing that the process will complete on time, or rejects the request if it cannot guarantee that the task will be serviced by its deadline.

\paragraph{Rate-Monotonic Scheduling}\label{par:Rate_Monotonic_Scheduling}
\begin{definition}[Rate-Monotonic Scheduling]\label{def:Rate_Monotonic_Scheduling}
  \emph{Rate-monotonic scheduling} is a \nameref{def:Scheduling_Algorithm} for periodic tasks that uses a static priority policy with preemption.
  If a higher-priority process arrives, and a lower priority one is running, it is immediately preempted.
  The priority is statically calculated based on the inverse of the period of the task.
  Less frequent (longer period) tasks have a lower priority, and more frequent ones have higher priority.
  Additionally, the size of the CPU burst is assumed to be constant during every period.
\end{definition}

%%% Local Variables:
%%% mode: latex
%%% TeX-master: "../../EDAF35-Operating_Systems-Reference_Sheet"
%%% End:


\subsection{Algorithm Evaluation}\label{subsec:Algorithm_Evaluation}
Now that we have selected a \nameref{def:Scheduling_Algorithm} to use, how do we know that it was the right choice?
First, we need to know what our criteria were.
Some systems might have multiple criteria at a time, such as:
\begin{itemize}[noitemsep]
\item Maximum CPU response time is 1 second.
\item Turnaround time is (on average) linearly proportional to total execution time.
\end{itemize}

To do this, there are 4 main ways to do this:
\begin{enumerate}[noitemsep]
\item \nameref{par:Deterministic_Modeling}
\item \nameref{par:Queuing_Models}
\item \nameref{par:Simulations}
\item \nameref{par:Implementation}
\end{enumerate}

\paragraph{Deterministic Modeling}\label{par:Deterministic_Modeling}
Deterministic modeling is simple and fast.
It gives us exact numbers, allowing us to compare the algorithms.
However, it requires \textbf{exact numbers for input}, and its answers apply \emph{only} to those cases.

This method takes a particular predetermined workload and defines the performance of each algorithm for that workload.

The main uses of deterministic modeling are in describing \nameref{def:Scheduling_Algorithm}s and providing examples.
In cases where we are running the same program repeatedly, we can measure the program’s processing requirements exactly, allowing us to select a \nameref{def:Scheduling_Algorithm}.
Furthermore, over a set of examples, deterministic modeling may indicate trends that can then be analyzed and proved separately.


%%% Local Variables:
%%% mode: latex
%%% TeX-master: "../../EDAF35-Operating_Systems-Reference_Sheet"
%%% End:


\begin{definition}[Deadlock]\label{def:Deadlock}
  \emph{Deadlock} is when 2 processes require information from each other to continue running.
  If this happens, neither process will provide the other with its required information, so they will both wait for each other, forever.
\end{definition}
\subsection{Deadlocks}\label{subsec:Deadlocks}
\nameref{def:Deadlock} is a serious issue in \nameref{rmk:CPU_Scheduler}s because \nameref{def:Process}es lock resources for themselves.
A good example of a deadlock is ``When two trains approach each other at a crossing, both shall come to a full stop and neither shall start up again until the other has gone.''

\begin{definition}[Deadlock]\label{def:Deadlock}
  \emph{Deadlock} is when 2 processes require information or resources from each other to continue running.
  If this happens, neither process will provide the other with its required information, so they will both wait for each other, forever.
\end{definition}

There are only 2 options for handling \nameref{def:Deadlock}s:
\begin{enumerate}[noitemsep]
\item Prevent them from happening in the first place.
\item Identify them and fix the problem that is causing them.
\item Hope they don't happen and consider them as unlikely events to occur.
  \begin{itemize}[noitemsep]
  \item This is what most desktop \nameref{def:Operating_System}s do.
  \end{itemize}
\end{enumerate}

Most \nameref{def:Operating_System}s do \textbf{NOT} provide functionality to identify \nameref{def:Deadlock}s and correct them.

%%% Local Variables:
%%% mode: latex
%%% TeX-master: "../../EDAF35-Operating_Systems-Reference_Sheet"
%%% End:



%%% Local Variables:
%%% mode: latex
%%% TeX-master: "../EDAF35-Operating_Systems-Reference_Sheet"
%%% End:
