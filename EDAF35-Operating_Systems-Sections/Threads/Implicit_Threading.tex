\subsection{Implicit Threading}\label{subsec:Implicit_Threading}
Because programs are starting to use so many \nameref{def:Thread}s that it is becoming hard to manage, the creation and management of threads is moving from developers to compilers/run-time libaries.
This way, the computer manages threads rather than the programmer.

\begin{definition}[Implicit Threading]\label{def:Implicit_Threading}
  \emph{Implicit Threading} is the transfer of \nameref{def:Thread} creation and management away from the programmer, and to the compiler and/or run-time libaries.
  This frees the programmer from having to think/worry about the issues that arise because of multithreading, while still allowing programs to take advantage of the benefits of multithreading.
\end{definition}

There exist 3 main methods for implementing implicit threading:
\begin{enumerate}[noitemsep]
\item \nameref{subsubsec:Thread_Pools}
\item \nameref{subsubsec:OpenMP}
\item \nameref{subsubsec:Grand_Central_Dispatch}
\end{enumerate}

\subsubsection{Thread Pools}\label{subsubsec:Thread_Pools}
\subsubsection{OpenMP}\label{subsubsec:OpenMP}
\subsubsection{Grand Central Dispatch}\label{subsubsec:Grand_Central_Dispatch}
%%% Local Variables:
%%% mode: latex
%%% TeX-master: "../../EDAF35-Operating_Systems-Reference_Sheet"
%%% End:
