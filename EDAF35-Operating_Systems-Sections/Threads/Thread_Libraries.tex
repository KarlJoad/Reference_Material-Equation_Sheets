\subsection{Thread Libraries}\label{subsec:Thread_Libraries}
\begin{definition}[Thread Library]\label{def:Thread_Library}
  A \emph{thread library} is an \nameref{def:API} that allows a programmer to create and manage \nameref{def:Thread}s.
\end{definition}

There are 2 main ways to implement a \nameref{def:Thread_Library}:
\begin{enumerate}[noitemsep]
\item Provide a library entirely in user space with no kernel support.
  All code and data structures for the library exist in \textbf{user-space}.
  This means that invoking a library function results in a function call in user space and not a system call.
\item Implement a kernel-level library supported directly by the operating system.
  All code and data structures for the library exist in \textbf{kernel-space}.
  Invoking a library function results in a \nameref{def:System_Call} to the \nameref{def:Kernel}.
\end{enumerate}

There are 3 different libraries that are used frequently.
\begin{enumerate}[noitemsep]
\item POSIX Pthreads.
  \begin{itemize}[noitemsep]
  \item This provides a standard interface for \textbf{how the threads should behave}.
  \item The actual implementation details are left for the implementor to decide.
  \item Global variables are shared between threads.
  \item Brought in by the standard library's \kernelinline{pthread.h} header.
  \end{itemize}
\item Windows' Threads.
  \begin{itemize}[noitemsep]
  \item Global variables are shared between threads.
  \item Must include the \kernelinline{windows.h} header.
  \end{itemize}
\item Java's Threads.
\end{enumerate}

Although there are these different libraries, all of them provide similar functionality.
So, throughout this section (unless otherwise specified), all information will be general.

\subsubsection{Synchronous/Asynchronous Threading}\label{subsubsec:Sync_Async_Threading}
There are 2 main ways to create multiple threads.
\begin{enumerate}[noitemsep]
\item Asynchronous
  \begin{itemize}[noitemsep]
  \item Once the parent thread creates a child thread, the parent resumes execution.
  \item The parent and child execute concurrently.
  \item Each thread is independent.
  \item The parent does not need to know if a child terminates.
  \item Little data sharing between threads.
  \end{itemize}
\end{enumerate}


%%% Local Variables:
%%% mode: latex
%%% TeX-master: "../../EDAF35-Operating_Systems-Reference_Sheet"
%%% End:
