\subsection{User and Kernel Threads}\label{subsec:User_Kernel_Threads}
A relationship must exist between \nameref{def:User_Thread}s and \nameref{def:Kernel_Thread}s.
There are three common ways of establishing such a relationship:
\begin{enumerate}[noitemsep]
\item The \nameref{subsubsec:Many_To_One_Model}
\item The \nameref{subsubsec:One_To_One_Model}
\item The \nameref{subsubsec:Many_To_Many_Model}
\end{enumerate}

\begin{definition}[User Thread]\label{def:User_Thread}
  \emph{User thread}s are \nameref{def:Thread}s created by a \nameref{def:User} \nameref{def:Process}.
  They are supported above the kernel and are managed without kernel support.
\end{definition}

\begin{definition}[Kernel Thread]\label{def:Kernel_Thread}
  \emph{Kernel thread}s are \nameref{def:Thread}s created by the \nameref{def:Kernel}.
  They are supported and managed directly by the operating system.
\end{definition}

\subsubsection{Many-To-One Model}\label{subsubsec:Many_To_One_Model}
\textbf{This model maps many user-level threads to a single \nameref{def:Kernel_Thread}.}
Thread management is done by the thread library in user space, so it is efficient.
However, the entire process will block if a thread makes a blocking system call.
Also, because only one thread can access the kernel at a time, multiple threads are unable to run in parallel on multicore systems.

\subsubsection{One-To-One Model}\label{subsubsec:One_To_One_Model}
\subsubsection{Many-To-Many Model}\label{subsubsec:Many_To_Many_Model}

%%% Local Variables:
%%% mode: latex
%%% TeX-master: "../../EDAF35-Operating_Systems-Reference_Sheet"
%%% End:
