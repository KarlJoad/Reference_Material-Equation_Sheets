\subsection{User and Kernel Threads}\label{subsec:User_Kernel_Threads}
A relationship must exist between \nameref{def:User_Thread}s and \nameref{def:Kernel_Thread}s.
There are three common ways of establishing such a relationship:
\begin{enumerate}[noitemsep]
\item The \nameref{subsubsec:Many_To_One_Model}
\item The \nameref{subsubsec:One_To_One_Model}
\item The \nameref{subsubsec:Many_To_Many_Model}
\end{enumerate}

\begin{definition}[User Thread]\label{def:User_Thread}
  \emph{User thread}s are \nameref{def:Thread}s created by a \nameref{def:User} \nameref{def:Process}.
  They are supported above the kernel and are managed without kernel support.
\end{definition}

\begin{definition}[Kernel Thread]\label{def:Kernel_Thread}
  \emph{Kernel thread}s are \nameref{def:Thread}s created by the \nameref{def:Kernel}.
  They are supported and managed directly by the operating system.
\end{definition}

\subsubsection{Many-To-One Model}\label{subsubsec:Many_To_One_Model}
\textbf{This model maps many user-level threads to a single \nameref{def:Kernel_Thread}.}
Thread management is done by the thread library in user space, so it is efficient.
However, the entire process will block if a thread makes a blocking system call.
Also, because only one thread can access the kernel at a time, multiple threads are unable to run in parallel on multicore systems.

\subsubsection{One-To-One Model}\label{subsubsec:One_To_One_Model}
\textbf{This model maps each user thread to an individual kernel thread.}
It provides more concurrency than the many-to-one model by allowing another thread to run when a thread makes a blocking system call.
It also allows multiple threads to run in parallel on multiprocessors.

The only drawback to this model is that creating a user thread requires creating the corresponding kernel thread, which is quite expensive.
The overhead of creating kernel threads can hurt the performance of an application.
To combat this, most implementations of this model restrict the number of threads supported by the system.
Most desktop operating systems use this model.

\subsubsection{Many-To-Many Model}\label{subsubsec:Many_To_Many_Model}
The many-to-many model multiplexes many user-level threads to a smaller or equal number of kernel threads.
The number of kernel threads may be specific to either a particular application or a particular machine (an application may be allocated more kernel threads on a multiprocessor than on a single processor).

Whereas the \nameref{subsubsec:Many_To_One_Model} allows the developer to create as many \nameref{def:User_Thread}s as they wishes, it does not result in true concurrency, because the kernel can schedule only one thread at a time.
The \nameref{subsubsec:One_To_One_Model} allows greater concurrency, but the developer must be mindful of the number of threads within an application.

The many-to-many model suffers from neither of these shortcomings:
\begin{itemize}[noitemsep]
\item Developers can create as many user threads as necessary, and the corresponding kernel threads can run in parallel on a multiprocessor.
\item When a thread performs a blocking system call, the kernel can schedule another thread for execution.
\end{itemize}

%%% Local Variables:
%%% mode: latex
%%% TeX-master: "../../EDAF35-Operating_Systems-Reference_Sheet"
%%% End:
