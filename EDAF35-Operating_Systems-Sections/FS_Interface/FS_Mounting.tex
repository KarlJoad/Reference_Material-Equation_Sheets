\subsection{File System Mounting}\label{subsec:File_System_Mounting}
Just as a \nameref{def:File} must be opened before it is used, a \nameref{def:File_System} must be mounted before it can be available to \nameref{def:Process}es on the system.
Specifically, the directory structure may be built out of multiple \nameref{def:Volume}s, which must be mounted to make them available within the file-system name space.

\begin{definition}[Mount Point]\label{def:Mount_Point}
  A \emph{mount point} is a location within the file structure where the file system will be attached.
\end{definition}

The general steps to mount a \nameref{def:File_System} are shown below.
This scheme enables the \nameref{def:Operating_System} to traverse its directory structure, switch among file systems (possibly of various types), as appropriate.
\begin{enumerate}[noitemsep]
\item The operating system is given the name of the device and the \nameref{def:Mount_Point}.
  \begin{itemize}[noitemsep]
  \item Some operating systems require that a file system type be provided when mounting; others inspect the structures of the device and determine the type of file system.
  \item Typically, a mount point is an empty directory.
  \end{itemize}

\item The operating system verifies that the device contains a valid \nameref{def:File_System}.
  \begin{itemize}[noitemsep]
  \item It does so by asking the device driver to read the device directory
and verifying that the directory has the expected format.
\end{itemize}

\item The operating system notes in its directory structure that a file system is mounted at the specified mount point.
\end{enumerate}


%%% Local Variables:
%%% mode: latex
%%% TeX-master: "../../EDAF35-Operating_Systems-Reference_Sheet"
%%% End:
