\subsection{Directory and Disk Structure}\label{subsec:Directory_Disk_Structure}
Each \nameref{def:Volume} that contains a \nameref{def:File_System} must also contain information about the \nameref{def:File}s in the system.
This information is kept in entries in a device directory (A volume's Table of Contents).
The device directory (commonly, the \nameref{def:Directory}) records information—such as name, location, size, and type—for all files on that volume.

\begin{definition}[Volume]\label{def:Volume}
  A \emph{Volume} is a logical view of the installed disks.
  Any entity that contains a \nameref{def:File_System} is considered a volume.
  In the case of a \nameref{def:RAID} setup, when members are pooled together, they are collectively addressed as a single volume.
\end{definition}

\subsubsection{Storage Structure}\label{subsubsec:Storage_Structure}
A general-purpose computer system has multiple storage devices, and those devices can be sliced up into \nameref{def:Volume}s that hold \nameref{def:File_System}s.
Computer systems may have zero or more file systems, and the file systems may be of varying types.

The file systems of computers can be extensive.
Even within a file system, it is useful to segregate files into groups and manage and act on those groups.
This organization involves the use of directories.

\subsubsection{Directory Overview}\label{subsubsec:Directory_Overview}
\begin{definition}[Directory]\label{def:Directory}
  The directory can be viewed as a symbol table that translates file names into their directory entries.
  Directories are files that are responsible for tracking and holding other files, which themselves may be directories.
\end{definition}

The directory itself can be organized in many ways.
The organization must allow us to insert entries, to delete entries, to search for a named entry, and to list all the entries in the directory.

Like a \nameref{def:File}, a \nameref{def:Directory} can be seen as an abstract data type that needs its computations defined before being concrete.
\begin{itemize}[noitemsep]
\item \textbf{Search for a file}.
  We need to be able to search a directory structure to find the entry for a particular file.
  Since files have symbolic names, and similar names may indicate a relationship among files, we may want to be able to find all files whose names match a particular pattern.

\item \textbf{Create a file}.
  New files need to be created and added to the directory.

\item \textbf{Delete a file}.
  When a file is no longer needed, we want to be able to remove it from the directory.

\item \textbf{List a directory}.
  We need to be able to list the files in a directory and the contents of the directory entry for each file in the list.

\item \textbf{Rename a file}.
  Because the name of a file represents its contents to its users, we must be able to change the name when the contents or use of the file changes.
  Renaming a file may also allow its position within the directory structure to be changed.

\item \textbf{Traverse the file system}.
  We may wish to access every directory and every file within a directory structure.
  For reliability, it is a good idea to backup the contents and structure of the entire file system at regular intervals.
\end{itemize}


%%% Local Variables:
%%% mode: latex
%%% TeX-master: "../../EDAF35-Operating_Systems-Reference_Sheet"
%%% End:
