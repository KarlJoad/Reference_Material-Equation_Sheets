\subsection{Directory and Disk Structure}\label{subsec:Directory_Disk_Structure}
Each \nameref{def:Volume} that contains a \nameref{def:File_System} must also contain information about the \nameref{def:File}s in the system.
This information is kept in entries in a device directory (A volume's Table of Contents).
The device directory (commonly, the \nameref{def:Directory}) records information—such as name, location, size, and type—for all files on that volume.

\begin{definition}[Volume]\label{def:Volume}
  A \emph{Volume} is a logical view of the installed disks.
  Any entity that contains a \nameref{def:File_System} is considered a volume.
  In the case of a \nameref{def:RAID} setup, when members are pooled together, they are collectively addressed as a single volume.
\end{definition}

\subsubsection{Storage Structure}\label{subsubsec:Storage_Structure}
A general-purpose computer system has multiple storage devices, and those devices can be sliced up into \nameref{def:Volume}s that hold \nameref{def:File_System}s.
Computer systems may have zero or more file systems, and the file systems may be of varying types.

The file systems of computers can be extensive.
Even within a file system, it is useful to segregate files into groups and manage and act on those groups.
This organization involves the use of directories.

\subsubsection{Directory Overview}\label{subsubsec:Directory_Overview}
\begin{definition}[Directory]\label{def:Directory}
  The directory can be viewed as a symbol table that translates file names into their directory entries.
  Directories are files that are responsible for tracking and holding other files, which themselves may be directories.
\end{definition}


%%% Local Variables:
%%% mode: latex
%%% TeX-master: "../../EDAF35-Operating_Systems-Reference_Sheet"
%%% End:
