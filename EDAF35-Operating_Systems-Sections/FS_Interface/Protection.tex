\subsection{Protection}\label{subsec:File_Protection}
When information is stored in a computer system, we want to keep it safe from improper access.

Most systems have evolved to use the concepts of file \nameref{def:File_Owner} and \nameref{def:File_Group}.

\subsubsection{Types of Access}\label{subsubsec:Types_File_Access}
The need to protect files is a direct result of the ability to access files.
Systems that do not permit access to the files of other users do not need protection.
Protection mechanisms provide controlled access by limiting the types of file access that can be made.
Access is permitted or denied depending on several factors, one of which is the type of access requested.
Several different types of operations may be controlled:
\begin{itemize}[noitemsep]
\item \textbf{Read}.
 Read from the file.
\item \textbf{Write}.
 Write or rewrite the file.
\item \textbf{Execute}.
 Load the file into memory and execute it.
\item \textbf{Append}.
 Write new information at the end of the file.
\item \textbf{Delete}.
 Delete the file and free its space for possible reuse.
\item \textbf{List}.
 List the name and attributes of the file.
\end{itemize}

Other, higher-level, operations, such as renaming, copying, and editing the file, may also be controlled.
For many systems, however, these functions may be implemented by a program that makes lower-level system calls.
File protection is provided at only the lower level.

\subsubsection{Access Control}\label{subsubsec:Access_Control}
The most common approach to the protection problem is to make access dependent on the identity of the user.
Different users may need different types of access to a file or directory.
The most general scheme to implement identity-dependent access is to associate with each file and directory an access-control list (ACL) specifying user names and the types of access allowed for each user.
This allows for complex access methodologies, however:
\begin{itemize}[noitemsep]
\item The ACL would get too long if we needed a rule for every possible user on the system.
\item Constructing such a list may be a tedious and unrewarding task, especially if we do not know in advance the list of users in the system.
\item The directory entry, previously of fixed size, now must be of variable size, resulting in more complicated space management.
\end{itemize}

Thus, the entries in the ACL are usually \nameref{def:File_Owner}, \nameref{def:File_Group}, and \nameref{def:File_Universe}.
\begin{definition}[Owner]\label{def:File_Owner}
  The \emph{owner} is the user who can change attributes and grant access and has the most control over the file.
  They are the creator of the file.
  The owner's User ID (UID) of a given \nameref{def:File} (or \nameref{def:Directory}) is stored with the other \nameref{def:File_Attribute}s.
\end{definition}

\begin{definition}[Group]\label{def:File_Group}
  The \emph{group} attribute defines a subset of users who can share access to the file and require similar access levels.
  The Group ID (GID) of a given \nameref{def:File} (or \nameref{def:Directory}) is stored with the other \nameref{def:File_Attribute}s.
\end{definition}

\begin{definition}[Universe]\label{def:File_Universe}
  The \emph{universe} attribute are all other possible users in the system that are not already the \nameref{def:File_Owner} or part of the \nameref{def:File_Group}.
\end{definition}

The most common recent approach is to combine access-control lists with the more general Owner/Group/Universe access-control scheme

For this scheme to work properly, permissions and access lists must be controlled tightly.
This control can be accomplished in several ways.
For example, in the \textsc{unix} system, \nameref{def:File_Group}s can be created and modified only by the manager of the facility (or by any superuser).

Another difficulty is assigning precedence when permission and ACLs conflict.
Solaris gives ACLs precedence (as they are more fine-grained and are not assigned by default).
This follows the general rule that specificity should have priority.

\subsubsection{Other Protection Approaches}\label{subsubsec:Other_File_Protection_Approaches}
Another approach to the protection problem is to associate a password with each file.
\begin{itemize}[noitemsep]
\item Advantages
  \begin{itemize}[noitemsep]
  \item If the passwords are chosen randomly and changed often, this scheme may be effective in limiting access to a file.
  \end{itemize}
\item Disadvantages
  \begin{itemize}[noitemsep]
  \item First, the number of passwords that a user needs to remember may become large, making the scheme impractical.
  \item If only one password is used for all the files, then once it is discovered, all files are accessible; protection is on an all-or-none basis.
  \item Some systems allow a user to associate a password with a directory, rather than with an individual file, to address this problem.
  \end{itemize}
\end{itemize}

\paragraph{Directory Operations}\label{par:Directory_Operation_Protections}
The directory operations that must be protected are somewhat different from the file operations.
We want to control the creation and deletion of files in a directory.
In addition, we probably want to control whether a user can determine the existence of a file in a directory.
Sometimes, knowledge of the existence and name of a file is significant in itself; thus, listing the contents of a directory must be a protected operation.
Similarly, if a path name refers to a file in a directory, the user must be allowed access to both the directory and the file.
In systems where files may have numerous path names (such as acyclic and general graphs), a given user may have different access rights to a particular file, depending on the path name used.

%%% Local Variables:
%%% mode: latex
%%% TeX-master: "../../EDAF35-Operating_Systems-Reference_Sheet"
%%% End:
