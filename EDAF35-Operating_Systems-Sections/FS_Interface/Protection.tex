\subsection{Protection}\label{subsec:File_Protection}
When information is stored in a computer system, we want to keep it safe from improper access.

Most systems have evolved to use the concepts of file \nameref{def:File_Owner} and \nameref{def:File_Group}.

\subsubsection{Types of Access}\label{subsubsec:Types_File_Access}
The need to protect files is a direct result of the ability to access files.
Systems that do not permit access to the files of other users do not need protection.
Protection mechanisms provide controlled access by limiting the types of file access that can be made.
Access is permitted or denied depending on several factors, one of which is the type of access requested.
Several different types of operations may be controlled:
\begin{itemize}[noitemsep]
\item \textbf{Read}.
 Read from the file.
\item \textbf{Write}.
 Write or rewrite the file.
\item \textbf{Execute}.
 Load the file into memory and execute it.
\item \textbf{Append}.
 Write new information at the end of the file.
\item \textbf{Delete}.
 Delete the file and free its space for possible reuse.
\item \textbf{List}.
 List the name and attributes of the file.
\end{itemize}

Other, higher-level, operations, such as renaming, copying, and editing the file, may also be controlled.
For many systems, however, these functions may be implemented by a program that makes lower-level system calls.
File protection is provided at only the lower level.

\subsubsection{Access Control}\label{subsubsec:Access_Control}
The most common approach to the protection problem is to make access dependent on the identity of the user.
Different users may need different types of access to a file or directory.
The most general scheme to implement identity-dependent access is to associate with each file and directory an access-control list (ACL) specifying user names and the types of access allowed for each user.
This allows for complex access methodologies, however:
\begin{itemize}[noitemsep]
\item The ACL would get too long if we needed a rule for every possible user on the system.
\item Constructing such a list may be a tedious and unrewarding task, especially if we do not know in advance the list of users in the system.
\item The directory entry, previously of fixed size, now must be of variable size, resulting in more complicated space management.
\end{itemize}

Thus, the entries in the ACL are usually \nameref{def:File_Owner}, \nameref{def:File_Group}, and \nameref{def:File_Universe}.
\begin{definition}[Owner]\label{def:File_Owner}
  The \emph{owner} is the user who can change attributes and grant access and has the most control over the file.
  They are the creator of the file.
  The owner's User ID (UID) of a given \nameref{def:File} (or \nameref{def:Directory}) is stored with the other \nameref{def:File_Attribute}s.
\end{definition}

\begin{definition}[Group]\label{def:File_Group}
  The \emph{group} attribute defines a subset of users who can share access to the file and require similar access levels.
  The Group ID (GID) of a given \nameref{def:File} (or \nameref{def:Directory}) is stored with the other \nameref{def:File_Attribute}s.
\end{definition}

\begin{definition}[Universe]\label{def:File_Universe}
  The \emph{universe} attribute are all other possible users in the system that are not already the \nameref{def:File_Owner} or part of the \nameref{def:File_Group}.
\end{definition}


%%% Local Variables:
%%% mode: latex
%%% TeX-master: "../../EDAF35-Operating_Systems-Reference_Sheet"
%%% End:
