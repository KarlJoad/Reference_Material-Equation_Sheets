\subsection{Access Methods}\label{subsec:Access_Methods}
\nameref{def:File}s store information.
When they are used, this information must be accessed and read into computer memory.
The 2 main ways to access the content of a file are:
\begin{enumerate}[noitemsep]
\item \nameref{subsubsec:Sequential_Access}
\item \nameref{subsubsec:Direct_Access}
\end{enumerate}

Not all operating systems support both \nameref{subsubsec:Sequential_Access} and \nameref{subsubsec:Direct_Access} for files.
Some systems allow only sequential file access; others allow only direct access.
Some systems require that a file be defined as sequential or direct when it is created.
Such a file can be accessed only in a manner consistent with its declaration.

\subsubsection{Sequential Access}\label{subsubsec:Sequential_Access}
Information in the file is processed in order, one logical record after the other.
This mode of access is by far the most common.

Reads and writes make up the bulk of the operations on a file.
A read operation reads the next portion of the file and automatically advances a file pointer, tracking the I/O location.
Similarly, the write operation, appends to the end of the file and advances to the end of the newly written material (the new end of file).

The pointers in this file can be reset to the beginning, and on some systems, a program may be able to skip forward or backward $n$ records for some integer $n$—perhaps only for $n = 1$.
Sequential access is based on a tape model of a file and works as well on sequential-access devices as it does on random-access ones.


%%% Local Variables:
%%% mode: latex
%%% TeX-master: "../../EDAF35-Operating_Systems-Reference_Sheet"
%%% End:
