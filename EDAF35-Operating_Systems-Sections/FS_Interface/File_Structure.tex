\subsection{File Structure}\label{subsec:File_Structure}
File types can also indicate the internal structure of the file.
This point brings us to one of the disadvantages of having the operating system support multiple file structures: the resulting size of the operating system is cumbersome.
If the operating system defines five different file structures, it needs to contain the code to support these file structures.

Some operating systems impose (and support) a minimal number of file structures.
\textsc{unix} considers each file to be a sequence of bytes; no interpretation of these bits is made by the \textbf{\nameref{def:Operating_System}}.
Each application program must include its own code to interpret an input file as to the appropriate structure.
This scheme provides maximum flexibility but little support.
All operating systems must support at least one structure, the executable file, so that the system is able to load and run programs.

\subsubsection{Internal File Structure}\label{subsubsec:Internal_File_Structure}
Locating an offset within a file can be complicated for the operating system.
Disk systems typically have a well-defined block size determined by the size of a sector.
All disk I/O is performed in units of one block (physical record), and all blocks are the same size.
It is unlikely that the physical record size will exactly match the length of the desired logical record.
Because \textsc{unix} defines all files to be streams of bytes, each byte can be individually addressed by its offset from the beginning (or end) of the file.
In this case, the logical record size is 1 byte.
Logical records may vary in length.
Packing a number of logical records into physical blocks is a common solution to this problem.

The logical record size, physical block size, and packing technique determine how many logical records are in each physical block.
The packing can be done either by the user’s application program or by the operating system.
In either case, the file may be considered a sequence of blocks.

Because disk space is always allocated in blocks, some portion of the last block of each file is generally wasted.
The waste incurred to keep everything in units of blocks (instead of bytes) is \nameref{def:Internal_Fragmentation}.
All \nameref{def:File_System}s suffer from internal fragmentation; the larger the block size, the greater the internal fragmentation.

%%% Local Variables:
%%% mode: latex
%%% TeX-master: "../../EDAF35-Operating_Systems-Reference_Sheet"
%%% End:
