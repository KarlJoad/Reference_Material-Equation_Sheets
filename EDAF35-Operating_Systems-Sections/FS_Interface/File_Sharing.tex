\subsection{File Sharing}\label{subsec:File_Sharing}
File sharing is very desirable for users who want to collaborate and to reduce the effort required to achieve a computing goal.
Therefore, user-oriented operating systems must accommodate the need to share files in spite of the inherent difficulties.

\subsubsection{Multiple Users}\label{subsubsec:File_Sharing-Multiple_Users}
Given a directory structure that allows files to be shared by users, the system must mediate the file sharing.
The system can either allow a user to access the files of other users by default or require that a user specifically grant access to the files.
To implement sharing and protection, the system must maintain more file and directory attributes than are needed on a single-user system.
Typically, this involves recording the \nameref{def:File_Owner} and the \nameref{def:File_Group} to determine access.

This is discussed more in \Cref{subsec:File_Protection}.

\subsubsection{Remote File Systems}\label{subsubsec:Remote_File_Systems}
With the advent of networks, communication among remote computers became possible.
Networking allows the sharing of resources spread over a distance.
One obvious resource to share is data in the form of files.

There are 2 main methods for transferring files between computing systems.
\begin{enumerate}[noitemsep]
\item Manual programs, like \texttt{ftp}. The World Wide Web falls into this category too.
\item \nameref{def:Distributed_File_System}s. There are 2 main models and 1 main concern.
  \begin{enumerate}[noitemsep]
  \item \nameref{par:Client_Server_Model}.
  \item \nameref{par:Distributed_Information_Systems}.
  \end{enumerate}
  \begin{enumerate}[noitemsep]
  \item \nameref{par:Failure_Modes}.
  \end{enumerate}
\end{enumerate}


%%% Local Variables:
%%% mode: latex
%%% TeX-master: "../../EDAF35-Operating_Systems-Reference_Sheet"
%%% End:
