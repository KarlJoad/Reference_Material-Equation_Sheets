\subsection{File Sharing}\label{subsec:File_Sharing}
File sharing is very desirable for users who want to collaborate and to reduce the effort required to achieve a computing goal.
Therefore, user-oriented operating systems must accommodate the need to share files in spite of the inherent difficulties.

\subsubsection{Multiple Users}\label{subsubsec:File_Sharing-Multiple_Users}
Given a directory structure that allows files to be shared by users, the system must mediate the file sharing.
The system can either allow a user to access the files of other users by default or require that a user specifically grant access to the files.
To implement sharing and protection, the system must maintain more file and directory attributes than are needed on a single-user system.
Typically, this involves recording the \nameref{def:File_Owner} and the \nameref{def:File_Group} to determine access.

This is discussed more in \Cref{subsec:File_Protection}.

\subsubsection{Remote File Systems}\label{subsubsec:Remote_File_Systems}
With the advent of networks, communication among remote computers became possible.
Networking allows the sharing of resources spread over a distance.
One obvious resource to share is data in the form of files.

There are 2 main methods for transferring files between computing systems.
\begin{enumerate}[noitemsep]
\item Manual programs, like \texttt{ftp}. The World Wide Web falls into this category too.
\item \nameref{def:Distributed_File_System}s. There are 2 main models and 1 main concern.
  \begin{enumerate}[noitemsep]
  \item \nameref{par:Client_Server_Model}.
  \item \nameref{par:Distributed_Information_Systems}.
  \end{enumerate}
  \begin{enumerate}[noitemsep]
  \item \nameref{par:Failure_Modes}.
  \end{enumerate}
\end{enumerate}

\begin{definition}[Distributed File System]\label{def:Distributed_File_System}
  A \emph{Distributed File System} (\emph{DFS}) is where remote directories are visible from a local machine, as if the file storage were directly attached to the computer.
\end{definition}

\paragraph{Client-Server Model}\label{par:Client_Server_Model}
The machine containing the files is the server, and the machine seeking access to the files is the client.
Generally, the server declares that a resource is available to clients and specifies exactly which resource and exactly which clients.
A server can serve multiple clients, and a client can use multiple servers, depending on the implementation details of a given client–server facility.

Client identification is more difficult.
A client can be specified by a network name or other identifier, such as an IP address, but these can be spoofed.
More secure solutions include secure authentication of the client via encrypted keys.
Unfortunately, with security come many challenges, including ensuring compatibility of the client and server and security of key exchanges.

\paragraph{Distributed Information Systems}\label{par:Distributed_Information_Systems}
Distributed information systems, also known as distributed naming services, provide unified access to the information needed for remote computing.
The Domain Name System (DNS) is one example of this type of information system.

Other distributed information systems provide user name/password/user ID/group ID space for a distributed facility.
This allows all computer-related information to be tracked and stored in a single central place and information be shared out on an as-needed basis.

\paragraph{Failure Modes}\label{par:Failure_Modes}
Local file systems can fail for a variety of reasons, including:
\begin{itemize}[noitemsep]
\item Failure of the disk containing the file system.
\item Corruption of the directory structure or other disk-management information (collectively called metadata).
\item Disk-controller failure.
\item Cable failure.
\item Host-adapter failure.
\item User failure can also cause files to be lost or entire directories or volumes to be deleted.
\item System-Administrator can cause files to be lost or entire directories or volumes to be deleted.
\end{itemize}


%%% Local Variables:
%%% mode: latex
%%% TeX-master: "../../EDAF35-Operating_Systems-Reference_Sheet"
%%% End:
