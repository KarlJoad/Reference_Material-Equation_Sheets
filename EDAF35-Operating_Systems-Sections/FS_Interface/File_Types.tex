\subsection{File Types}\label{subsec:File_Types}
We need to decide whether the operating system should recognize and support \nameref{def:File} types.
If an operating system recognizes the type of a file, it can then operate on the file in reasonable ways.

A common technique for implementing \nameref{def:File} types is to include the type as part of the file name.
The name is split into two parts: the name and the extension, which are separated by a period.
This way, the user \textbf{and} the operating system can tell from the name alone what the type of a file is.

The system uses the extension to indicate the type of the \nameref{def:File} and the type of operations that can be done on that file.
Application programs also use extensions to indicate file types in which they are interested.
These extensions are not always required, so a user may specify a file without the extension, and the application will look for a file with the given name perhaps with the extension it expects.
Because these extensions are not supported by the operating system, they should be viewed as ``hints'' to the applications that operate on them.

%%% Local Variables:
%%% mode: latex
%%% TeX-master: "../../EDAF35-Operating_Systems-Reference_Sheet"
%%% End:
