\section{Security}\label{sec:Security}
Protection is an \textbf{internal} problem: How do we provide controlled access to programs and data stored in a computer system?
Security, on the other hand, requires not only an adequate protection system but also consideration of the \textbf{external} environment within which the system operates.
A protection system is ineffective if user authentication is compromised or a program is run by an unauthorized user.
Computer resources must be guarded against unauthorized access, malicious destruction or alteration, and accidental introduction of inconsistency.

For more in-depth discussion around this topic and further consideration of closely related topics, please see \href{file:./EDIN01-Cryptography-Reference_Sheet.pdf}{EDIN01, Cryptography} and \href{file:./EITP20-Secure_Systems_Engineering-Reference_Sheet.pdf}{EITP20, Secure Systems Engineering}.

\subsection{The Security Problem}\label{subsec:Security_Problem}
In \Cref{sec:Protection}, we discussed mechanisms that the operating system can provide that allow users to protect their resources.
These mechanisms work well \textbf{only} as long as the users conform to the intended use of and access to these resources.

A system is secure if its resources are used and accessed as intended under \textbf{all} circumstances.
Unfortunately, total security cannot be achieved.
Nonetheless, there must have mechanisms to make security breaches a rare occurrence, rather than the norm.

Security violations (or misuse) of the system can be categorized as intentional (malicious) or accidental.
It is easier to protect against accidental misuse than against malicious misuse.
For the most part, protection mechanisms are the core of protection from accidents.

We should note that in our discussion of security, the term:
\begin{itemize}[noitemsep]
\item Intruder is for those attempting to breach security.
\item A threat is the \textbf{potential} for a security violation, such as the discovery of a vulnerability.
\item An attack is an attempt to break security.
\end{itemize}

The most common forms of security violations include:
\begin{itemize}[noitemsep]
\item \textbf{Breach of confidentiality}.
  This type of violation involves unauthorized reading of data (or theft of information).
  Typically, a breach of confidentiality is the goal of an intruder.
  Capturing secret data from a system or a data stream, such as credit-card information or identity information for identity theft, can result directly in money for the intruder.
\item \textbf{Breach of integrity}.
  This violation involves unauthorized modification of data.
  Such attacks can result in passing of liability to an innocent party or modification of the source code of an important commercial application.
\item \textbf{Breach of availability}.
  This violation involves unauthorized destruction of data.
  Some intruders would rather wreak havoc and gain status or bragging rights than gain financially.
  Website defacement is a common example of this type of security breach.
\item \textbf{Theft of service}.
  This violation involves unauthorized use of resources.
  For example, an intruder (or intrusion program) may install a \nameref{def:Daemon} on a system that acts as a file server.
\item \textbf{Denial of service}.
  This violation involves preventing legitimate use of the system.
  Denial-of-service (DOS) attacks are sometimes accidental.
  The original Internet worm turned into a DOS attack when a bug failed to delay its rapid spread.
\end{itemize}

Attackers use several standard methods in their attempts to breach security.
The most common is \textbf{masquerading} or \textbf{spoofing}, in which one participant in a communication pretends to be someone else (another host or another person).
By masquerading, attackers breach authentication, the correctness of identification.
They can then gain access that they would not normally be allowed or escalate their privileges—obtain privileges to which they would not normally be entitled.

Another common attack is to replay a captured exchange of data.
A replay attack consists of the malicious or fraudulent repeat of a previously valid data transmission.
Sometimes the replay comprises the entire attack.
But frequently it is done along with message modification, again to escalate privileges.
Consider the damage that could be done if a request for authentication had a legitimate user’s information replaced with an unauthorized user’s.

Yet another kind of attack is the man-in-the-middle attack, in which an attacker sits in the data flow of a communication, masquerading as the sender to the receiver, and vice versa.
In a network communication, a man-in-the-middle attack may be preceded by a session hijacking, in which an active communication session is intercepted.

Absolute protection of the system from malicious abuse is not possible, but the cost to the perpetrator can be made sufficiently high to deter most intruders.
In some cases, such as a denial-of-service attack, it is preferable to prevent the attack but sufficient to detect the attack so that countermeasures can be taken.

To protect a system, we must take security measures at four levels:
\begin{enumerate}[noitemsep]
\item \textbf{Physical}.
  The site or sites containing the computer systems must be physically secured against armed or surreptitious entry by intruders.
  Both the machine rooms and the terminals/workstations that have access to the machines must be secured.
\item \textbf{Human}.
  Authorization must be done carefully to assure that only appropriate users have access to the system.
  Even authorized users, however, may be ``encouraged'' to let others use their access.
  They may also be tricked into allowing access via social engineering, such as:
  \begin{itemize}[noitemsep]
  \item \textbf{Phishing}.
    Here, a legitimate-looking e-mail or web page
    misleads a user into entering confidential information. Another

  \item \textbf{Dumpster Diving}.
    A general term for attempting to gather information in order to gain unauthorized access to the computer (by looking through trash, finding phone books, or finding notes containing passwords, for example).
  \item These security problems are management and personnel issues, not problems pertaining to operating systems.
  \end{itemize}
\item \textbf{Operating system}.
  The system must protect itself from accidental or purposeful security breaches.
  A runaway process could constitute an accidental denial-of-service attack.
  A query to a service could reveal passwords.
  A stack overflow could allow the launching of an unauthorized process.
  The list of possible breaches is almost endless.
\item \textbf{Network}.
  Much computer data in modern systems travels over private leased lines, shared lines like the Internet, wireless connections, or dial-up lines.
  Intercepting these data could be just as harmful as breaking into a computer, and interruption of communications could constitute a remote denial-of-service attack, diminishing users’ use of and trust in the system.
\end{enumerate}

%%% Local Variables:
%%% mode: latex
%%% TeX-master: "../../EDAF35-Operating_Systems-Reference_Sheet"
%%% End:


\subsection{Program Threats}\label{subsec:Program_Threats}
\nameref{def:Process}es, along with the \nameref{def:Kernel}, are the only means of accomplishing work on a computer.
Therefore, writing a program that creates a breach of security, or causing a normal process to change its behavior and create a breach, is a common goal of intruders.
Most nonprogram security events have causing a program threat as a goal.

\subsubsection{Trojan Horse}\label{subsubsec:Trojan_Horse}
Many systems have mechanisms for allowing programs written by users to be executed by other \nameref{def:User}s.
If these programs are executed in a domain that provides the \nameref{def:Access_Right}s of the executing user, the other users may misuse these rights.
For example, a text editing program may include code to search the file to be edited for certain keywords.
If any are found, the entire file is copied to a special area accessible to the \textbf{creator} of the text editor.
A code segment that misuses its environment is called a \nameref{def:Trojan_Horse}.

\begin{definition}[Trojan Horse]\label{def:Trojan_Horse}
  A \emph{trojan horse} or \emph{trojan} is any malware which misleads users of its true intent.
  Its code segment misuses the execution environment it is given to achieve something other than what is intended.
\end{definition}

\nameref{def:Trojan_Horse}s also include malware that mimics genuine software, to ensure the user is complacent and inputs secure information without realizing it is stolen/compromised.

Another variation on the \nameref{def:Trojan_Horse} is \nameref{def:Spyware}.
Spyware sometimes accompanies a program that the user has chosen to install.
\begin{definition}[Spyware]\label{def:Spyware}
  \emph{Spyware} is malware that intentionally monitors the \nameref{def:User}'s actions.
  The usual goal is to download ads to display on the user’s system, create pop-up browser windows when certain sites are visited, or use \nameref{def:Covert_Channel}s.
\end{definition}

Most frequently, \nameref{def:Spyware} comes with freeware or shareware programs, but sometimes is included with commercial software.

\begin{definition}[Covert Channel]\label{def:Covert_Channel}
  \emph{Covert channel}s are intended to capture information from the user’s system and return it to a central site.
\end{definition}

\subsubsection{Trap Door}\label{subsubsec:Trap_Door}
Another common, and very hard to detect type of malware is a \nameref{def:Trap_Door}.
\begin{definition}[Trap Door]\label{def:Trap_Door}
  A \emph{trap door} is a piece of malware where the designer of the program or system leaves a security hole in the software that only they are capable of using.
\end{definition}

For instance, the code might check for a specific user ID or password, and it might circumvent normal security procedures.

Trap doors pose a difficult problem because, to detect them, we have to analyze all the source code for all components of a system.
Given that software systems may consist of millions of lines of code, this analysis is not done frequently.

\subsubsection{Logic Bomb}\label{subsubsec:Logic_Bomb}
\nameref{def:Logic_Bomb}s is not malware, per-se; rather it is code that is executed \textbf{only} when certain conditions have been met.
What that code does determines how it is classified.

\begin{definition}[Logic Bomb]\label{def:Logic_Bomb}
  \emph{Logic bomb}s are programs that creates a security incident \textbf{only} under certain conditions.
  It would be hard to detect because under normal operations, there would be no security hole.
  However, when a predefined set of parameters was met, the security hole would be created.
\end{definition}

\subsubsection{Stack and Buffer Overflow}\label{subsubsec:Stack_Buffer_Overflow}
The stack- or buffer-overflow attack is the most common way for an attacker outside the system to gain unauthorized access to the target system.
An authorized user of the system may also use this exploit for privilege escalation.

Essentially, the attack exploits a bug in a program.
The bug can be a simple case of poor programming, in which the programmer neglected to code bounds checking on an input field.
In this case, the attacker sends more data than the program was expecting.
By using trial and error, or by examining the source code of the attacked program if it is available, the attacker determines the vulnerability and writes a program to do the following:
\begin{enumerate}[noitemsep]
\item Overflow an input field, command-line argument, or input buffer until it writes into the stack.
\item Overwrite the current return address on the stack with the address of the exploit code loaded in step 3.
\item Write a simple set of code for the next space in the stack that includes the commands that the attacker wishes to execute.
\end{enumerate}

The result of this attack program’s execution will be a privileged command execution.

When a function is invoked in a typical computer architecture, the local variables defined in the function, the parameters passed to the function, and the address to which control returns once the function exits are stored in a stack frame.
At the end of a stack frame, there is the return address, which specifies where to return control once the function exits.
The frame pointer must be fixed on the stack, as the value of the stack pointer can vary during the function call.
The saved frame pointer allows relative access to parameters and automatic variables.

Given this standard memory layout, an intruder could execute a buffer-overflow attack.
The goal being to replace the return address in the stack frame so that it points to a code segment containing the attacking program.

One solution to this problem is for the CPU to have a feature that disallows execution of code in a stack section of memory.
Recent versions x86 chips include the \texttt{NX} feature to prevent this type of attack.
The hardware implementation involves the use of a new bit in the page tables of the CPUs.
This bit marks the associated page as nonexecutable, so that instructions cannot be read from it and executed.

\subsubsection{Viruses}\label{subsubsec:Viruses}
\begin{definition}[Virus]\label{def:Virus}
  A \emph{virus} is a fragment of code embedded in a legitimate program.
  Viruses are self-replicating and are designed to ``infect'' other programs.
\end{definition}

\nameref{def:Virus}es can wreak havoc in a system by modifying or destroying files and causing system crashes and program malfunctions.
As with most penetration attacks, viruses are very specific to architectures, \nameref{def:Operating_System}s, and applications.

Once a virus reaches a target machine, a program known as a virus dropper inserts the virus into the system.
The virus dropper is usually a Trojan horse, executed for other reasons but installing the virus as its core activity.

%%% Local Variables:
%%% mode: latex
%%% TeX-master: "../../EDAF35-Operating_Systems-Reference_Sheet"
%%% End:


\subsection{System and Network Threats}\label{subsec:System_Network_Threats}
Program threats typically use a breakdown in the protection mechanisms of a system to attack programs.
In contrast, system and network threats involve the abuse of services and network connections.
System and network threats create a situation in which \nameref{def:Operating_System} resources and \nameref{def:User} files are misused.

Sometimes, a system and network attack is used to launch a program attack, and vice versa.
The more open an operating system is, the more services it has enabled and the more functions it allows, the more likely it is that a bug is available to exploit.
Such changes reduce the system’s \nameref{def:Attack_Surface}.

\begin{definition}[Attack Surface]\label{def:Attack_Surface}
  The \emph{attack surface} of a system is the set of ways in which an attacker can try to break into the system.
\end{definition}

It is important to note that masquerading and replay attacks are also commonly launched over networks between systems.
In general, sharing secrets (to prove identity and as keys to encryption) is required for authentication and encryption, and sharing secrets is easier in environments in which secure sharing methods exist.

\subsubsection{Worms}\label{subsubsec:Worms}
\begin{definition}[Worm]\label{def:Worm}
  A \emph{worm} is a process that uses a spawn mechanism to duplicate itself.
  The spawned copies use up system resources and may lock out all other \nameref{def:Process}es.
  The copies can also execute arbitrary code, which means they can cause other damage too.
\end{definition}

On computer networks, worms are particularly potent, since they may reproduce themselves among systems and thus shut down an entire network.


%%% Local Variables:
%%% mode: latex
%%% TeX-master: "../../EDAF35-Operating_Systems-Reference_Sheet"
%%% End:


\subsection{Cryptography}\label{subsec:Cryptography}

%%% Local Variables:
%%% mode: latex
%%% TeX-master: "../../EDAF35-Operating_Systems-Reference_Sheet"
%%% End:


\subsection{User Authentication}\label{subsec:User_Authentication}
If a system cannot authenticate a user, then authenticating that a message came from that user is pointless.
Thus, a major security problem for operating systems is user authentication.
The protection system depends on the ability to identify the programs and processes currently executing, which in turn depends on the ability to identify each user of the system.

Users normally identify themselves.
So how do we determine whether a user’s identity is authentic?
Generally, user authentication is based on one or more of three things:
\begin{enumerate}[noitemsep]
\item The user’s possession of something (a key or card).
\item The user’s knowledge of something (a user identifier and password).
\item An attribute of the user (fingerprint, retina pattern, or signature).
\end{enumerate}

\subsubsection{Passwords}\label{subsubsec:User_Authentication_Passwords}
The most common approach to authenticating a user identity is the use of passwords.
When the user identifies themself by user ID or account name, they are asked for a password.
If the user-supplied password matches the password stored in the system, the system assumes that the account is being accessed by the owner of that account.
Passwords are often used to protect objects in the computer system, in the absence of more complete protection schemes.
They can be treated as a special case of keys or capabilities.
Different passwords may be associated with different access rights.
For example, different passwords may be used for reading files, appending files, and updating files.

In practice, most systems require only one password for a user to gain full rights.
Although more passwords theoretically would be more secure, such systems tend not to be implemented due to the classic trade-off between security and convenience.
If security makes something inconvenient, then the security is frequently bypassed or otherwise circumvented.

\subsubsection{Password Vulnerabilities}\label{subsubsec:Password_Vulnerabilities}
Passwords are extremely common because they are easy to understand and use.
Unfortunately, passwords can often be guessed, accidentally exposed, sniffed (read by an eavesdropper), or illegally transferred from an authorized user to an unauthorized one.
There are two common ways to guess a password.
\begin{enumerate}[noitemsep]
\item One way is for the intruder (either human or program) to know the user or to have information about the user.
  All too frequently, people use personal information as their passwords.
\item The other way is to use brute force, enumerating all possible combinations of valid password characters until the password is found.
\end{enumerate}

In addition to being guessed, passwords can be exposed as a result of
visual or electronic monitoring.

The final type of password compromise, illegal transfer, is the result of human nature.
Most computer installations have a rule that forbids users to share accounts.
This rule is sometimes implemented for accounting reasons but is often aimed at improving security.
Sometimes, users break account-sharing this behavior can result in a system’s being accessed by unauthorized users.

Passwords can be either generated by the system or selected by a user.
However, system-generated passwords may be difficult to remember.
Some systems also age passwords, forcing users to change their passwords at regular intervals.
This method is not foolproof either, because users can easily toggle between two passwords.
The solution, as implemented on some systems, is to record a password history for each user.

\subsubsection{Securing Passwords}\label{subsubsec:Securing_Passwords}
One problem with all these approaches is the difficulty of keeping the password secret within the computer.
How can the system store a password securely yet allow its use for authentication when the user presents her password?
The \textsc{unix} system uses secure hashing to avoid the necessity of keeping its password list secret.
Because the list is hashed rather than encrypted, it is impossible for the system to decrypt the stored value and determine the original password.
When a user presents a password, it is hashed and compared against the stored encoded password.
Even if the stored encoded password is seen, it cannot be decoded, so the password cannot be determined.

Systems also include a ``salt,'' or recorded random number, in the hashing algorithm.
The salt value is added to the password to ensure that if two plaintext passwords are the same, they result in different hash values.


%%% Local Variables:
%%% mode: latex
%%% TeX-master: "../../EDAF35-Operating_Systems-Reference_Sheet"
%%% End:


\subsection{Implementing Security Defenses}\label{subsec:Implementing_Security_Defenses}
\subsubsection{Security Policy}\label{subsubsec:Security_Policy}
The first step toward improving the security of any aspect of computing is to have a security policy.
Policies vary widely but generally include a statement of what is being secured.
Without a policy in place, it is impossible for users and administrators to know what is permissible, what is required, and what is not allowed.


%%% Local Variables:
%%% mode: latex
%%% TeX-master: "../../EDAF35-Operating_Systems-Reference_Sheet"
%%% End:


\subsection{Firewalling}\label{subsec:Firewalling}

%%% Local Variables:
%%% mode: latex
%%% TeX-master: "../../EDAF35-Operating_Systems-Reference_Sheet"
%%% End:


%%% Local Variables:
%%% mode: latex
%%% TeX-master: "../EDAF35-Operating_Systems-Reference_Sheet"
%%% End:
