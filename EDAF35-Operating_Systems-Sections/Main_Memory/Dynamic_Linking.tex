\subsection{Dynamic Linking}\label{subsec:Dynamic_Linking}
\nameref{def:Operating_System}s provide many useful libraries to allow programmers and their work to use higher-level abstractions to make their programming life easier.
An example of this is a language library for localization, allowing a programmer to write their input and output statements in one language and it be translated to another automatically.

To achieve this, \nameref{def:Dynamic_Linking} is used, and \nameref{def:Stub}s are generated during the compilation process to inform the running \nameref{def:Process} where to find the necessary information.

\begin{definition}[Dynamic Linking]\label{def:Dynamic_Linking}
  A \nameref{def:Program} that makes use of \emph{dynamic linking} is compiled like normal.
  However, the library routines that would be provided by a \emph{dynamically linked library} (\emph{Shared Library}) are left as \nameref{def:Stub}s.
  When the program begins execution, the \nameref{def:Process} executes like normal, potentially using \nameref{def:Dynamic_Loading} in the meantime.
  However, once the \nameref{def:Process} reaches a \nameref{def:Stub}, it will: proceed like normal (if the stub is replaced by the actual code), or will fetch and load the routine, replace the stub, and then execute the routine.

  \begin{remark}[Dynamic Linking vs. Dynamic Loading]\label{rmk:Dynamic_Linking_vs_Dynamic_Loading}
    In \nameref{def:Dynamic_Linking}, a \nameref{def:Program} is compiled down to the machine code, with the \nameref{def:Stub}s in place, which are replaced \textbf{DURING} program execution.
    This allows us to share a binary library file \textbf{BETWEEN DIFFERENT} \nameref{def:Program}s.
    Whereas, in \nameref{def:Dynamic_Loading}, the code is brought in from memory from the \textbf{same} \nameref{def:Program} binary image.

    This means that if an \nameref{def:Operating_System} did not support \nameref{def:Dynamic_Linking}, each \nameref{def:Program} that required a certain library must include it.
    This leads to space inefficiencies on both the disk and in main memory.
  \end{remark}
\end{definition}

\begin{definition}[Stub]\label{def:Stub}
  A \emph{stub} is a small piece of code that indicates how to locate the appropriate memory-resident library routine or how to load the library if the routine is not already present.
  A stub is included in the \nameref{def:Program}'s binary image \textbf{FOR EACH} library-routine reference.

  When the stub is executed, it checks to see whether the needed routine is already in memory.
  If it is not, the program loads the routine into memory and replaces the stub with the address of the routine and executes the routine.
  Then, the next time that routine is reached, the routine is executed directly, incurring no cost for \nameref{def:Dynamic_Linking}.
\end{definition}

This feature can be extended to handle library updates (such as bug fixes).
If a library is replaced by a new version, all programs that use that library will automatically use the new version.
Without \nameref{def:Dynamic_Linking}, all such programs would need to be relinked to gain access to the new library.
Version information is included in the \nameref{def:Program} and the library so that programs will not accidentally execute new, incompatible versions of libraries.
Multiple versions of the same library may be loaded into memory, and each program uses its version information to decide which library to use.

Unlike \nameref{def:Dynamic_Loading}, \nameref{def:Dynamic_Linking} and dynamically linked libraries require help from the \nameref{def:Operating_System}.
If the \nameref{def:Process}es in memory are protected from one another, then the operating system is the only entity that can check all the \nameref{def:Process}es.
Only the OS can:
\begin{itemize}[noitemsep]
\item See if the needed routine is in another process’s memory space.
\item Allow multiple processes to access the same memory addresses.
\end{itemize}

\subsubsection{Static vs. Dynamic Linking}\label{subsubsec:Static_vs_Dynamic_Linking}
As programmers, we are more familiar with \nameref{def:Static_Linking}; where a library is compiled \textbf{INTO} and assembled \textbf{WITH} our code \textbf{INTO} our executable binary image.

\begin{definition}[Static Linking]\label{def:Static_Linking}
  \emph{Static linking} is the act of running a linker after the compiler has generated \nameref{def:Stub}s for library routines.
  By running the linker, the library's code is brought into our program before it is assembled.
  This means that the library (and all the code supporting the library) is compiled \textbf{INTO} our binary image as well.
\end{definition}

%%% Local Variables:
%%% mode: latex
%%% TeX-master: "../../EDAF35-Operating_Systems-Reference_Sheet"
%%% End:
