\subsection{Page Table Structure}\label{subsec:Page_Table_Structure}
Because of the ever-increasing size of \nameref{def:Page_Table}s, we need to start organizing them more efficiently.
For example, consider a system with a 32-bit logical address space.
If the page size in such a system is \SI{4}{\kibi\byte}\footnote{\si{\kibi\byte} is a kibibyte, or \si[prefixes-as-symbols=false]{\kibi} bytes} ($2^{12}$), then a page table may consist of up to 1 million entries ($\frac{2^{32}}{2^{12}}$).
Assuming that each entry consists of 4 bytes, each process may need up to \SI{4}{\mebi\byte} of physical address space for the page table alone.
Allocating this page table contiguously is not what we want to do, because if not all the pages are in-use, then we want to be able to use the space that would otherwise be taken up by the page table.

There are 3 techniques for handling this discussed in the textbook:
\begin{enumerate}[noitemsep]
\item \nameref{subsubsec:Hierarchical_Paging}
\item \nameref{subsubsec:Hashed_Page_Tables}
\item \nameref{subsubsec:Inverted_Page_Tables}
\end{enumerate}

\subsubsection{Hierarchical Paging}\label{subsubsec:Hierarchical_Paging}
One simple solution to this problem is to divide the page table into smaller pieces.
We can accomplish this division in several ways.
One way is to use a two-level paging algorithm, in which the page table itself is also paged.

For example, consider the system with a 32-bit \nameref{def:Logical_Address_Space} and \SI{4}{\kibi\byte} again.
Fromt he perspective of the CPU, the\nameref{def:Logical_Address} is divided into a page number consisting of 20 bits and a page offset consisting of 12 bits.
However, because we page the page table, the page number is further divided into a 10-bit outer page number and a 10-bit page offset.

\subsubsection{Hashed Page Tables}\label{subsubsec:Hashed_Page_Tables}
\subsubsection{Inverted Page Tables}\label{subsubsec:Inverted_Page_Tables}

%%% Local Variables:
%%% mode: latex
%%% TeX-master: "../../EDAF35-Operating_Systems-Reference_Sheet"
%%% End:
