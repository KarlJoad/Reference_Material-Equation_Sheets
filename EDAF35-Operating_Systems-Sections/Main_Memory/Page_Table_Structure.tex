\subsection{Page Table Structure}\label{subsec:Page_Table_Structure}
Because of the ever-increasing size of \nameref{def:Page_Table}s, we need to start organizing them more efficiently.
For example, consider a system with a 32-bit logical address space.
If the page size in such a system is \SI{4}{\kibi\byte}\footnote{\si{\kibi\byte} is a kibibyte, or \si[prefixes-as-symbols=false]{\kibi} bytes} ($2^{12}$), then a page table may consist of up to 1 million entries ($\frac{2^{32}}{2^{12}}$).
Assuming that each entry consists of 4 bytes, each process may need up to \SI{4}{\mebi\byte} of physical address space for the page table alone.
Allocating this page table contiguously is not what we want to do, because if not all the pages are in-use, then we want to be able to use the space that would otherwise be taken up by the page table.

There are 3 techniques for handling this discussed in the textbook:
\begin{enumerate}[noitemsep]
\item \nameref{subsubsec:Hierarchical_Paging}
\item \nameref{subsubsec:Hashed_Page_Tables}
\item \nameref{subsubsec:Inverted_Page_Tables}
\end{enumerate}

\subsubsection{Hierarchical Paging}\label{subsubsec:Hierarchical_Paging}
One simple solution to this problem is to divide the page table into smaller pieces.
We can accomplish this division in several ways.
One way is to use a \textbf{two-level page table}, in which the page table itself is also paged.

For example, consider the system with a 32-bit \nameref{def:Logical_Address_Space} and \SI{4}{\kibi\byte} again.
Fromt he perspective of the CPU, the\nameref{def:Logical_Address} is divided into a page number consisting of 20 bits and a page offset consisting of 12 bits.
However, because we page the page table, the page number is further divided into a 10-bit outer page number and a 10-bit page offset.

If the memory capacity of the computer increased, and therefore the page count did as well, then this would start becoming problematic again.
We could divide the outer page table and page it again, giving us a three-level \nameref{def:Paging} scheme.
If we continued to do this, a machine with a 64-bit \nameref{def:Logical_Address_Space} would require 7 level of paging.
Remember that each time a \nameref{def:Page_Table} is used, we need another memory lookup and access.
This makes multiple levels of paging prohibitively slow.

\subsubsection{Hashed Page Tables}\label{subsubsec:Hashed_Page_Tables}
A hashed page table, is a \nameref{def:Page_Table} where the page number is the hash value.
Each entry in the hash table contains a linked list of elements that hash to the same location (to handle collisions).
Each element consists of three fields:
\begin{enumerate}[noitemsep]
\item Hashed page number.
\item Value of the mapped frame.
\item Pointer to the next element in the linked list (or \texttt{NULL}).
\end{enumerate}

The algorithm works as follows:
\begin{algorithm}[H]
  \DontPrintSemicolon{}
  \SetKwData{fieldOne}{Field 1}
  \SetKwData{fieldTwo}{Field 2}
  \SetKwData{fieldThree}{Field 3}
  \BlankLine{}

  The page number in the \nameref{def:Logical_Address} is hashed into the hash table. \;
  The virtual page number is compared
  with field 1 in the first element in the linked list.
  \If(There is a match){\fieldOne{} == \fieldTwo{}}{
    Corresponding page frame (\fieldTwo{}) is used to form the desired physical address.
  }\Else(There is no match){
    subsequent entries in the linked list are searched for a matching virtual page number.
  }

  \caption{Hashed Page Table Usage}
  \label{algo:Hashed_Page_Table_Usage}
\end{algorithm}

\paragraph{Clustered Page Tables}\label{par:Clustered_Page_Tables}
A variation of \nameref{subsubsec:Hashed_Page_Tables} that is useful for 64-bit address spaces, called \emph{clustered page tables}, has been proposed.
They are similar to their origin, except that each entry in the hash table refers to several pages (such as 16) rather than a single page.
Therefore, a single clustered \nameref{def:Page_Table} entry can store the mappings for multiple frames.
Clustered page tables are particularly useful for sparse address spaces, where memory references are noncontiguous \textbf{AND} scattered throughout the address space.

\subsubsection{Inverted Page Tables}\label{subsubsec:Inverted_Page_Tables}
Usually, each \nameref{def:Process} has an associated page table with one entry for each page that is in use.
However, each \nameref{def:Page_Table} may consist of millions of entries, and there may be many page tables.
These tables may consume large amounts of physical memory just to keep track of how memory is being used.

To solve this problem, we can use an inverted page table.
An inverted page table has \textbf{one entry for each frame} of memory.
Each entry consists of
\begin{itemize}[noitemsep]
\item The \nameref{def:Logical_Address} of the page stored in that physical frame
\item \nameref{par:Address_Space_Identifiers}, information about the \nameref{def:Process} that owns the page.
  \begin{itemize}[noitemsep]
  \item The table usually contains several different process address spaces that are mapped to physical memory.
  \item Ths way the logical page for a particular process is mapped to the correct corresponding physical frame.
  \end{itemize}
\end{itemize}

\begin{blackbox}
  {\large{\textbf{Thus, only one page table is in the system, and it has only one entry for each page of physical memory.}}}
\end{blackbox}

To use the table, a tuple (\Cref{eq:Inverted_Hash_Table_Tuple}) is created by the CPU as the \nameref{def:Logical_Address}.
\begin{equation}\label{eq:Inverted_Hash_Table_Tuple}
  \langle \mathtt{PID}, p, d \rangle
\end{equation}
\begin{description}[noitemsep]
\item $\mathtt{PID}$: The \nameref{def:Process}~ID.\@
\item $p$: The page number.
\item $d$: The offset of the desired location within the frame, i.e.\ in physical memory.
\end{description}

Although this scheme decreases the amount of memory needed to store each page table, it increases the amount of time needed to search the table because it is sorted by physical address, and lookups occur on virtual addresses, meaning the whole table might need to be searched before a match is found.
To skirt this issue, a hash table is used, just like in \nameref{subsubsec:Hashed_Page_Tables} to limit the search.
However, each access to the hashed inverted page table adds another memory reference.
So one virtual memory reference requires at least two real memory reads:
\begin{enumerate}[noitemsep]
\item One for the hashed inverted page table entry.
\item One for the resulting page table.
\end{enumerate}

One downside of inverted page tables is that \nameref{par:Shared_Memory} is difficult to implement.
Shared memory is usually implemented as multiple \nameref{def:Logical_Address}es (one for each \nameref{def:Process} sharing the memory) that are mapped to the same \nameref{def:Physical_Address}.
However, the usual method does not work here, because there is only one entry for every frame, one frame cannot have multiple shared \nameref{def:Logical_Address}es mapped to it.

%%% Local Variables:
%%% mode: latex
%%% TeX-master: "../../EDAF35-Operating_Systems-Reference_Sheet"
%%% End:
