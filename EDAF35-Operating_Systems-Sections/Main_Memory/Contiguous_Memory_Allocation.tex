\subsection{Contiguous Memory Allocation}\label{subsec:Contiguous_Memory_Allocation}
Main memory must contain everything for the system to run, including both the \nameref{def:Operating_System} and \nameref{def:User} \nameref{def:Process}es.
It is our jobs, as operating system engineers to make the allocation of memory for the \nameref{def:Operating_System} as efficient as possible.
The earliest and simplest method is that of \nameref{def:Contiguous_Memory_Allocation}.

Memory is usually divided into two partitions: one for the \nameref{def:Operating_System} and one for the user processes.
We can place the operating system in either low memory or high memory, depending on the location of the \nameref{def:Interrupt_Vector}.
Typically, the interrupt vector is in low addresses, so the OS is usually put there too.
Throughout this document, we assume that the OS inhabits the lowest addresses.


%%% Local Variables:
%%% mode: latex
%%% TeX-master: "../../EDAF35-Operating_Systems-Reference_Sheet"
%%% End:
