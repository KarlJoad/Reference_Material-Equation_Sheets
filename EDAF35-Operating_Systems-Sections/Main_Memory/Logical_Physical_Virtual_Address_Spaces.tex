\subsection{Logical, Physical, Virtual Address Spaces}\label{subsec:Logical_Physical_Virtual_Address_Space}
The compile-time and load-time address-binding methods generate identical logical and physical addresses.

\begin{definition}[Logical Address]\label{def:Logical_Address}
  The CPU generates virtual/\emph{logical address}es.
  How these logical addresses look, behave, and correspond to \nameref{def:Physical_Address}es depend on the memory management technique used by the \nameref{def:Memory_Management_Unit}.

  The user program deals with logical addresses.
  The memory-mapping hardware converts logical addresses into \nameref{def:Physical_Address}es.
\end{definition}

\begin{definition}[Physical Address]\label{def:Physical_Address}
  The \emph{physical address} is the actual value of a particular byte of memory's address.
  How the physical address is calculated from the \nameref{def:Logical_Address} is handled by the \nameref{def:Memory_Management_Unit}.

  The user program deals with \nameref{def:Logical_Address}es.
  The memory-mapping hardware converts logical addresses into physical addresses.
\end{definition}

However, the execution-time address-binding scheme results in differing logical and physical addresses, so we call the \nameref{def:Logical_Address} a \nameref{def:Virtual_Address} instead.

\begin{definition}[Virtual Address]\label{def:Virtual_Address}
  The CPU generates logical/\emph{virtual address}es.
  How these virtual addresses look, behave, and correspond to \nameref{def:Physical_Address}es depend on the memory management technique used by the \nameref{def:Memory_Management_Unit}.

  The user program deals with logical/virtual addresses.
  The memory-mapping hardware converts logical/virtual addresses into \nameref{def:Physical_Address}es.

  \begin{remark}[Use of Logical and Virtual Address]
    In this document, I use \nameref{def:Logical_Address} and \nameref{def:Virtual_Address} interchangeably.
  \end{remark}
\end{definition}

The set of all logical addresses/\nameref{def:Virtual_Address}es generated by a program is a \nameref{def:Logical_Address_Space}/\nameref{def:Virtual_Address_Space}.

\begin{definition}[Logical Address Space]\label{def:Logical_Address_Space}
  The \emph{logical address space} consists of all the \nameref{def:Logical_Address}es generated by a \nameref{def:Program}.

  \begin{remark}
    The \nameref{def:Logical_Address_Space} is only calculated by one \nameref{def:Program}/\nameref{def:Process} at a time.
    To find the total logical address space used, all \nameref{def:Process}es must have their logical address spaces aggregated.
  \end{remark}
\end{definition}

The set of all \nameref{def:Physical_Address}es corresponding to these logical addresses is the \nameref{def:Physical_Address_Space}.

\begin{definition}[Physical Address Space]\label{def:Physical_Address_Space}
  The \emph{physical address space} consists of all the \nameref{def:Physical_Address}es that the \nameref{def:Logical_Address}es/\nameref{def:Virtual_Address}es map to.

  \begin{remark}
    The \nameref{def:Physical_Address_Space} is only calculated by one \nameref{def:Program}/\nameref{def:Process} at a time.
    To find the total physical address space used, all \nameref{def:Process}es must have their physical address spaces aggregated.
  \end{remark}
\end{definition}


%%% Local Variables:
%%% mode: latex
%%% TeX-master: "../../EDAF35-Operating_Systems-Reference_Sheet"
%%% End:
