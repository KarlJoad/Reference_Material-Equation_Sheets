\subsection{Logical, Physical, Virtual Address Spaces}\label{subsec:Logical_Physical_Virtual_Address_Space}
The compile-time and load-time address-binding methods generate identical logical and physical addresses.

\begin{definition}[Logical Address]\label{def:Logical_Address}
  The CPU generates virtual/\emph{logical address}es.
  How these logical addresses look, behave, and correspond to \nameref{def:Physical_Address}es depend on the memory management technique used by the \nameref{def:Memory_Management_Unit}.

  The user program deals with logical addresses.
  The memory-mapping hardware converts logical addresses into \nameref{def:Physical_Address}es.
\end{definition}

\begin{definition}[Physical Address]\label{def:Physical_Address}
  The \emph{physical address} is the actual value of a particular byte of memory's address.
  How the physical address is calculated from the \nameref{def:Logical_Address} is handled by the \nameref{def:Memory_Management_Unit}.

  The user program deals with \nameref{def:Logical_Address}es.
  The memory-mapping hardware converts logical addresses into physical addresses.
\end{definition}

However, the execution-time address-binding scheme results in differing logical and physical addresses, so we call the \nameref{def:Logical_Address} a \nameref{def:Virtual_Address} instead.


%%% Local Variables:
%%% mode: latex
%%% TeX-master: "../../EDAF35-Operating_Systems-Reference_Sheet"
%%% End:
