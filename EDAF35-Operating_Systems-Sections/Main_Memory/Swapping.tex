\subsection{Swapping}\label{subsec:Swapping}
A \nameref{def:Process} can only be executed if it is in memory.
However, while it is \textbf{NOT} being executed, it is not required for it to be in memory.
This idea forms the basis of \nameref{def:Swapping}.

\begin{definition}[Swapping]\label{def:Swapping}
  \emph{Swapping} is the action of moving a \nameref{def:Process} out of memory to a \nameref{def:Backing_Store}, or vice versa.
\end{definition}

\begin{definition}[Backing Store]\label{def:Backing_Store}
  A \emph{backing store} is another form of storage media.
  There are typically very few requirements on the type of storage media used in a backing store, but it \textbf{MUST} be readily assessible, and be quick (not as fast as RAM, but faster than traditional file system storage).
  Typically, a fast disk is used as the backing store.

  \begin{remark}[Swap]\label{rmk:Swap}
    Sometimes the \nameref{def:Backing_Store} is called \emph{the swap}, \emph{swap-area}, \emph{swap partition}, \emph{swapfile}, etc.
    They all mean the same thing\footnote{Swap Partition and Swapfile have specify the type of the \nameref{def:Backing_Store}}.
  \end{remark}
\end{definition}

\nameref{def:Swapping} allows for the total \nameref{def:Physical_Address_Space} of \textbf{ALL} \nameref{def:Process}es to exceed the real physical memory of the system.
However, the \nameref{def:Context_Switch} time in a swapping system is fairly high.


%%% Local Variables:
%%% mode: latex
%%% TeX-master: "../../EDAF35-Operating_Systems-Reference_Sheet"
%%% End:
