\subsection{Swapping}\label{subsec:Swapping}
A \nameref{def:Process} can only be executed if it is in memory.
However, while it is \textbf{NOT} being executed, it is not required for it to be in memory.
This idea forms the basis of \nameref{def:Swapping}.

\begin{definition}[Swapping]\label{def:Swapping}
  \emph{Swapping} is the action of moving a \nameref{def:Process} out of memory to a \nameref{def:Backing_Store}, or vice versa.
\end{definition}

\begin{definition}[Backing Store]\label{def:Backing_Store}
  A \emph{backing store} is another form of storage media.
  There are typically very few requirements on the type of storage media used in a backing store, but it \textbf{MUST} be readily assessible, and be quick (not as fast as RAM, but faster than traditional file system storage).
  Typically, a fast disk is used as the backing store.

  \begin{remark}[Swap]\label{rmk:Swap}
    Sometimes the \nameref{def:Backing_Store} is called \emph{the swap}, \emph{swap-area}, \emph{swap partition}, \emph{swapfile}, etc.
    They all mean the same thing\footnote{Swap Partition and Swapfile have specify the type of the \nameref{def:Backing_Store}}.
  \end{remark}
\end{definition}

\nameref{def:Swapping} allows for the total \nameref{def:Physical_Address_Space} of \textbf{ALL} \nameref{def:Process}es to exceed the real physical memory of the system.
However, the \nameref{def:Context_Switch} time in a swapping system is fairly high.

The system maintains the ready queue consisting of all processes whose memory images are on the backing store or in memory and are ready to run.
Whenever the CPU scheduler decides to execute a process, it calls the \nameref{def:Dispatcher}.
The dispatcher checks to see where the next \nameref{def:Process} is.
\begin{itemize}[noitemsep]
\item If the process is in memory, the dispatcher will reload registers and transfer control, like normal.
\item If the process is in the swap, the dispatcher will swap the selected process into memory, reload registers, and transfer control.
\item If the process is in the swap, \textbf{and} if there is no free memory region, the dispatcher swaps out a process currently in memory and swaps in the desired process, reloads registers, and transfers control.
\end{itemize}


%%% Local Variables:
%%% mode: latex
%%% TeX-master: "../../EDAF35-Operating_Systems-Reference_Sheet"
%%% End:
