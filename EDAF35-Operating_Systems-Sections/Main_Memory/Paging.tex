\subsection{Paging}\label{subsec:Paging}
Most memory-management schemes used before the introduction of \nameref{def:Paging} suffered from the problem of fitting memory chunks of varying sizes on the \nameref{def:Backing_Store}
This arises because space must be found on the backing store, when main memory needs to be swapped out.
The backing store has the same \nameref{def:Fragmentation} problems discussed in connection with main memory, but access is much slower, so compaction is impossible.

\begin{definition}[Paging]\label{def:Paging}
  \emph{Paging} allows the \nameref{def:Physical_Address_Space} of a \nameref{def:Process} to be noncontiguous.
  This is achieved breaking up \nameref{def:Physical_Memory} and \nameref{def:Logical_Address_Space} into equally, fixed-size blocks.
  \begin{itemize}[noitemsep]
  \item \nameref{def:Physical_Memory} is broken up into \emph{frames}.
  \item \nameref{def:Logical_Address_Space} is broken up into \emph{pages}.
  \item The \nameref{def:Backing_Store} is also broken up, as a multiple (1 through $n$) of the frame size.
  \end{itemize}

  The list of pages and their mapping to frames for this \nameref{def:Process} is stored in the \nameref{def:Page_Table}.
  There is a page table in each process, which is used when this process \nameref{def:Context_Switch}es into the CPU.\@
  Therefore, paging also increases context switch time.

  Paging avoids the issue of \nameref{def:External_Fragmentation} and solves the problem of fitting memory chunks of varying sizes on the \nameref{def:Backing_Store}.
  However, \nameref{def:Internal_Fragmentation} is still an issue.
\end{definition}

\nameref{def:Paging} is implemented through cooperation between the operating system and the computer hardware.
Similar to how \nameref{def:Segmentation} has a segment table, \textbf{EACH \nameref{def:Process}} in a \nameref{def:Paging} system has a \nameref{def:Page_Table}.

\begin{definition}[Page Table]\label{def:Page_Table}
  A \emph{page table} is a table that maintains the mapping of logical pages to physical frames in memory.
  The page table is a simple lookup table, where the current \nameref{def:Process}'s current page number is also the value of the index.
  The value contained at that element is the location of the frame in memory.
  This is combined with the use of \nameref{def:Logical_Address}es that are generated by the CPU.\@
\end{definition}

To reach an address, the CPU generates a \nameref{def:Logical_Address} that has 2 parts, a page number $p$ and a page offset $d$.
Like before, the page number is used as an \textbf{INDEX} in the \nameref{def:Page_Table}.
The page table contains the base address of each page in physical memory.
This base address is combined with the page offset to define the physical memory address that is sent to the memory unit.

The frame size, and thus the page size, are defined by the hardware.
The size of a page is a power of 2, varying between 512 bytes and $\SI{1}{\gibi\byte}$\footnote{\si{\gibi\byte} is a gibibyte, or \si[prefixes-as-symbols=false]{\gibi} bytes} per page, depending on the computer architecture.
A power of 2 as a page size makes the translation of a \nameref{def:Logical_Address} into a page number and page offset particularly easy.

If the size of the logical address space is $2^{m}$, and a page size is $2^{n}$ bytes, then the high-order $m-n$ \textbf{bits} of a \nameref{def:Logical_Address} designate the page number, and the $n$ low-order \textbf{bits} designate the page offset.
\begin{equation}\label{eq:Page_Table_Calculations}
  \begin{aligned}
    p &= m-n \\
    d &= n \\
  \end{aligned}
\end{equation}

Thus, to translate the \nameref{def:Logical_Address} to a \nameref{def:Physical_Address}, \Cref{eq:Paging_Logical_Physical_Address_Conversion} is used.

\begin{equation}\label{eq:Paging_Logical_Physical_Address_Conversion}
  f = \bigl( i(p) \times s \bigr) + d
\end{equation}
\begin{description}[noitemsep]
\item $f$: Resulting \nameref{def:Physical_Address}.
\item $i(p)$: Mapped frame number of the given page number $p$ from the \nameref{def:Page_Table}.
\item $s$: Size of the page/frame.
\end{description}

If \nameref{def:Process} size is independent of page size, we expect \nameref{def:Internal_Fragmentation} to average one-half page per process.
This consideration suggests that small page sizes are desirable.
However, overhead is involved in maintaining the \nameref{def:Page_Table} itself, with each page-table entry increasing the overhead.
Also, disk I/O is more efficient when the amount data being transferred is larger.

Generally, page sizes have grown over time as processes, data sets, and main memory have become larger.
Today, pages typically are between $\SI{4}{\kibi{}\byte{}}$ and $\SI{8}{\kibi{}\byte{}}$ in size, and some systems support even larger page sizes.
Some CPUs and kernels now support multiple page sizes.
Variable on-the-fly page size is still being developed.

A 32-bit CPU uses 32-bit addresses, meaning that a given \nameref{def:Process}'s memory space can only be $2^{32}$ bytes ($\SI{4}{\gibi{}\byte{}}$).
However, a single 32-bit entry can point to one of $2^{32}$ different physical frames.
If frame size is 4 KB ($2^{12}$), then a system with 4-byte (32-bit) entries can address $2^{44}$ bytes (or $\SI{16}{\tebi{}\byte{}}$) of physical memory.
Therefore, paging lets us use physical memory that is larger than what can be addressed by the CPU’s address pointer length.
This means that the size of physical memory in a paged memory system is different from the maximum logical size of a process.

When a \nameref{def:Process} arrives in the system to be executed, its size, in pages, is examined.
Because pages and frames are the same size, if the process requires $n$ pages, at least $n$ frames must be available in memory.
If the required $n$ frames are available, they are allocated to this arriving process.
The first page of the process is loaded into one of the allocated frames, and the that frame's number is put in the \nameref{def:Page_Table} for this process for this page.
The next page is loaded into another frame, its frame number is put into the page table, and so on.

\nameref{def:Paging} offers a clear separation between the programmer’s view of memory and the actual hardware.
The logical addresses that the programmer uses are translated into physical addresses that the hardware uses.
This mapping is hidden from the programmer and is controlled by the operating system.
This allows the programmer to view memory as one contiguous space, containing only this one program (which helps create our definition of \nameref{def:Virtual_Memory}).
However, the user program may be scattered throughout \nameref{def:Physical_Memory}, which also holds other programs.
The difference between the programmer’s view of memory and the actual physical memory is reconciled by the address-translation hardware in the \nameref{def:Memory_Management_Unit}.

Because the \nameref{def:Operating_System} is managing \nameref{def:Physical_Memory}, it must be aware of the allocation details of physical memory, including:
\begin{itemize}[noitemsep]
\item Which frames are allocated.
\item Which frames are available.
\item The total number of frames.
\item etc.
\end{itemize}
This information is generally kept in a data structure called a \nameref{def:Frame_Table}.

\begin{definition}[Frame Table]\label{def:Frame_Table}
  The \emph{frame table} has one entry for each physical frame, indicating whether it is free or allocated and, if it is allocated, to which page of which process or processes.

  This behaves like an ownership or state table, rather than a lookup table.
  It says whether this frame is in use, and if it is, who is using it.
\end{definition}

\subsubsection{Hardware Support for Paging}\label{subsubsec:Paging_Hardware_Support}
Every access to memory must go through the paging map, so efficiency is a major consideration.
Each \nameref{def:Operating_System} has its own methods for storing \nameref{def:Page_Table}s.
Some allocate a page table for each \nameref{def:Process}, then a pointer to the page table is stored with the other register values in the \nameref{def:Process_Control_Block}.
Other operating systems provide only a few page tables, which decreases the overhead involved when processes are \nameref{def:Context_Switch}ed.

The hardware implementation of the \nameref{def:Page_Table} can be done in several ways.
In the simplest case, the page table is implemented as a set of high-speed, dedicated registers, making paging-address translation efficient.
The CPU \nameref{def:Dispatcher} reloads these registers, along with all the other registers.
Instructions to load or modify the page-table registers are privileged so that only the \nameref{def:Operating_System} can change the memory map.

%%% Local Variables:
%%% mode: latex
%%% TeX-master: "../../EDAF35-Operating_Systems-Reference_Sheet"
%%% End:
