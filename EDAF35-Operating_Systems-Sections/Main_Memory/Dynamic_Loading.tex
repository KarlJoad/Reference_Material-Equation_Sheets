\subsection{Dynamic Loading}\label{subsec:Dynamic_Loading}
Because of the way we have set up our memory and \nameref{def:Process}es before, a \nameref{def:Process} could only be as big as our physical memory.
However, we can skirt these issues by using \nameref{def:Dynamic_Loading}.

\begin{definition}[Dynamic Loading]\label{def:Dynamic_Loading}
  \emph{Dynamic Loading} is used to load routines into memory only when needed.
  The routines that can by dynamically loaded must be stored on disk in a \nameref{def:Relocatable_Code} format.
  If the main program that was loaded in and began execution needs a routine, it checks to see if the routine has been loaded.
  If it has not, the loader loads the desired routine into memory, then updates the \nameref{def:Process}'s address tables to point to the newly retrieved routine.
\end{definition}

The advantage of \nameref{def:Dynamic_Loading} is that a routine is loaded only when it is needed.
This is particularly useful when large amounts of code are needed to handle infrequently occurring cases, such as error routines.
This allows the total program size to be large, but the portion used (and hence loaded) be much smaller.

\nameref{def:Dynamic_Loading} does not require special support from the operating system.
It is the responsibility of the users to design their programs to take advantage of such a method.
\nameref{def:Operating_System}s may help the programmer, however, by providing library routines to implement \nameref{def:Dynamic_Loading}.

%%% Local Variables:
%%% mode: latex
%%% TeX-master: "../../EDAF35-Operating_Systems-Reference_Sheet"
%%% End:
