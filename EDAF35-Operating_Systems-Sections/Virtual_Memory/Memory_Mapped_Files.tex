\subsection{Memory-Mapped Files}\label{subsec:Memory_Mapped_Files}
The sequential read of a \nameref{def:File} on disk using the standard \nameref{def:System_Call}s \kernelinline{open()}, \kernelinline{read()}, and \kernelinline{write()}.
Each file access requires a system call and disk access.
To alleviate the pain of this, we could use \nameref{def:Memory_Mapping}.

\begin{definition}[Memory Mapping]\label{def:Memory_Mapping}
  \emph{Memory mapping} is the process of taking something and mapping it to a page in memory.
  This allows a page (possibly multiple) in the \nameref{def:Virtual_Address_Space} to be logically associated with a page-sized amount of something.
\end{definition}


%%% Local Variables:
%%% mode: latex
%%% TeX-master: "../../EDAF35-Operating_Systems-Reference_Sheet"
%%% End:
