\subsection{Memory-Mapped Files}\label{subsec:Memory_Mapped_Files}
The sequential read of a \nameref{def:File} on disk using the standard \nameref{def:System_Call}s \kernelinline{open()}, \kernelinline{read()}, and \kernelinline{write()}.
Each file access requires a system call and disk access.
To alleviate the pain of this, we could use \nameref{def:Memory_Mapping}.

\begin{definition}[Memory Mapping]\label{def:Memory_Mapping}
  \emph{Memory mapping} is the process of taking something and mapping it to a page in memory.
  This allows a page (possibly multiple) in the \nameref{def:Virtual_Address_Space} to be logically associated with a page-sized amount of something.
\end{definition}

\subsubsection{Basic Mechanism}\label{subsubsec:Basic_Memory_Mapping_Mechanism}
Memory mapping a file is accomplished by mapping a disk block to a page (or
pages) in memory.
The steps involved are:
\begin{enumerate}[noitemsep]
\item Initial access to the \nameref{def:File} proceeds through ordinary \nameref{def:Demand_Paging}, resulting in a \nameref{def:Page_Fault}.
\item A page-sized portion of the file is read from the file system into a physical frame.
  \begin{itemize}[noitemsep]
  \item Some systems may read in more than a page-sized chunk of memory at a time
  \end{itemize}
\item Subsequent reads and writes to the file (within the loaded paged-size amount of the file) are handled as routine memory accesses.
\end{enumerate}


%%% Local Variables:
%%% mode: latex
%%% TeX-master: "../../EDAF35-Operating_Systems-Reference_Sheet"
%%% End:
