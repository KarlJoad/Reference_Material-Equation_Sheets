\subsection{Thrashing}\label{subsec:Thrashing}
If a \nameref{def:Process} does not have the number of frames (\nameref{def:Physical_Memory}) it needs to support pages in active use, it will quickly page-fault.
At this point, it must replace some pages.
However, since all the pages are in active use, the OS will replace a page that will be needed again right away.
Consequently, the process faults repeatedly, replacing pages that it must bring back in immediately.

\begin{definition}[Thrashing]\label{def:Thrashing}
  \emph{Thrashing} is when a \nameref{def:Process} is spending more time \nameref{def:Paging} than executing.
  This happens when there are not enough frames in the system to support the number of pages the process needs.
  The \nameref{def:Page_Replacement} algorithm will replace some pages that the process needs right now, so it must page the first one back in, overwritting another page that is needed.
  Thus, the cycle continues.
\end{definition}

\subsubsection{Cause of Thrashing}\label{subsubsec:Thrashing_Cause}
\nameref{def:Thrashing} results in severe performance problems.
The \nameref{rmk:CPU_Scheduler} sees the decreasing CPU utilization and increases the degree of multiprogramming as a result.
The new \nameref{def:Process} tries to get started by taking frames from running processes, causing more \nameref{def:Page_Fault}s and a longer queue for the paging device (\nameref{def:Backing_Store}).
As a result, CPU utilization drops even further, and the CPU scheduler tries to increase the degree of multiprogramming even more.

This leads to system throughput plunging.
The \nameref{def:Page_Fault} rate increases tremendously, increasing the \nameref{def:Effective_Access_Time}.
No work is getting done, because the processes are spending all their time \nameref{def:Paging}.

\subsubsection{Limiting Thrashing}\label{subsubsec:Limiting_Thrashing}
We can limit the effects of \nameref{def:Thrashing} by using a local replacement algorithm.
With local replacement, if a \nameref{def:Process} starts thrashing, it cannot steal frames from another, causing the second one to thrash as well.
However, this does not entirely the problem.

\begin{remark*}
  The current best practice in implementing a computer system is to include enough \nameref{def:Physical_Memory} to avoid \nameref{def:Thrashing} and \nameref{def:Swapping}.
  Providing enough memory to keep all working sets in memory concurrently, except under extreme conditions, gives the best user experience.
\end{remark*}

If \nameref{def:Process}es are \nameref{def:Thrashing}, they will be in the queue for the paging device most of the time.
The average service time for a \nameref{def:Page_Fault} will still increase because of the longer average queue for the paging device.
Thus, the \nameref{def:Effective_Access_Time} will increase even for a process that is not thrashing.
To prevent thrashing, we must provide a process with as many frames as it \textbf{NEEDS}.

\subsubsection{Working-Set Model}\label{subsubsec:Working_Set_Model}
The \emph{Working-Set Model} uses the \nameref{def:Locality_Model} of \nameref{def:Process} execution.

\begin{definition}[Locality Model]\label{def:Locality_Model}
  The \emph{locality model} states, as a \nameref{def:Process} executes, it moves from locality to locality.
  A locality is a set of pages that are actively used together.
  A \nameref{def:Program} is generally composed of several different localities, which may overlap.

  For example, when a function is called, it defines a new locality.
  In this locality, memory references are made to the instructions of the function call, its local variables, and a subset of the global variables.
  When we exit the function, the process leaves this locality, since the local variables and instructions of the function are no longer in active use.
  We may return to this locality later.
\end{definition}

Localities are defined by the program structure and its data structures.
The \nameref{def:Locality_Model} states that all programs will exhibit this basic memory reference structure.

\begin{blackbox}
  \begin{remark*}
    Note that the \nameref{def:Locality_Model} is the unstated principle behind our mentions of caching so far.
    If accesses to any types of data were random rather than patterned, caching would be useless.
  \end{remark*}
\end{blackbox}

This model uses a parameter, $\Delta$, to define the \textbf{working-set window}.
The idea is to examine the most recent $\Delta$ page references.
The set of pages in the most recent $\Delta$ page references is the \textbf{working set}.
If a page is in active use, it will be in the working set.
If it is no longer being used, it will drop from the working set $\Delta$ time units after its last reference.
Thus, the working set is an approximation of the program’s locality.

The accuracy of the working set depends on the selection of $\Delta$.
\begin{itemize}[noitemsep]
\item $\Delta$ is too small, it will not encompass the entire locality.
\item $\Delta$ is too large, it may overlap several localities.
\item $\Delta$ is $\infty$, working set is the set of \textbf{ALL} pages touched during the \nameref{def:Process}'s execution.
\end{itemize}

Using this set, if we can find its size for all the \nameref{def:Process}es in the system, we can calculate the entire system's demand.
If the demand is greater than the total number of frames, then \nameref{def:Thrashing} will happen.

Once $\Delta$ is selected, use of the working-set model is simple.
The \nameref{def:Operating_System} monitors the working set of each \nameref{def:Process} and allocates to it enough frames to provide it with its working-set size.
If there are enough frames left over, another process can be initiated.
If the sum of the working-set sizes increases, making the demand of the system exceed the total number of available frames, the OS selects a process to suspend.
The process’s pages are swapped out, and its frames are reallocated to other processes.
The suspended process can be restarted later.

\subsubsection{Page-Fault Frequency}\label{subsubsec:Page_Fault_Frequency}
A strategy using the \nameref{def:Page_Fault} Frequency (PFF) can take a more direct approach.
\nameref{def:Thrashing} has a high page-fault rate.
Thus, we want to control the page-fault rate.
\begin{itemize}[noitemsep]
\item High, the process needs more frames.
\item Low, the process may have too many frames.
\end{itemize}


%%% Local Variables:
%%% mode: latex
%%% TeX-master: "../../EDAF35-Operating_Systems-Reference_Sheet"
%%% End:
