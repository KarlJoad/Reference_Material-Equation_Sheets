\subsection{Thrashing}\label{subsec:Thrashing}
If a \nameref{def:Process} does not have the number of frames (\nameref{def:Physical_Memory}) it needs to support pages in active use, it will quickly page-fault.
At this point, it must replace some pages.
However, since all the pages are in active use, the OS will replace a page that will be needed again right away.
Consequently, the process faults repeatedly, replacing pages that it must bring back in immediately.

\begin{definition}[Thrashing]\label{def:Thrashing}
  \emph{Thrashing} is when a \nameref{def:Process} is spending more time \nameref{def:Paging} than executing.
  This happens when there are not enough frames in the system to support the number of pages the process needs.
  The \nameref{def:Page_Replacement} algorithm will replace some pages that the process needs right now, so it must page the first one back in, overwritting another page that is needed.
  Thus, the cycle continues.
\end{definition}

\subsubsection{Cause of Thrashing}\label{subsubsec:Thrashing_Cause}
\nameref{def:Thrashing} results in severe performance problems.
The \nameref{rmk:CPU_Scheduler} sees the decreasing CPU utilization and increases the degree of multiprogramming as a result.
The new \nameref{def:Process} tries to get started by taking frames from running processes, causing more \nameref{def:Page_Fault}s and a longer queue for the paging device (\nameref{def:Backing_Store}).
As a result, CPU utilization drops even further, and the CPU scheduler tries to increase the degree of multiprogramming even more.

This leads to system throughput plunging.
The \nameref{def:Page_Fault} rate increases tremendously, increasing the \nameref{def:Effective_Access_Time}.
No work is getting done, because the processes are spending all their time \nameref{def:Paging}.

\subsubsection{Limiting Thrashing}\label{subsubsec:Limiting_Thrashing}
We can limit the effects of \nameref{def:Thrashing} by using a local replacement algorithm.
With local replacement, if a \nameref{def:Process} starts thrashing, it cannot steal frames from another, causing the second one to thrash as well.
However, this does not entirely the problem.

\begin{remark*}
  The current best practice in implementing a computer system is to include enough \nameref{def:Physical_Memory} to avoid \nameref{def:Thrashing} and \nameref{def:Swapping}.
  Providing enough memory to keep all working sets in memory concurrently, except under extreme conditions, gives the best user experience.
\end{remark*}


%%% Local Variables:
%%% mode: latex
%%% TeX-master: "../../EDAF35-Operating_Systems-Reference_Sheet"
%%% End:
