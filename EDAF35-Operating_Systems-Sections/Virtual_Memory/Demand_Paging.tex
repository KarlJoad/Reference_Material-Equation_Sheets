\subsection{Demand Paging}\label{subsec:Demand_Paging}
Because we can support the ability to \nameref{def:Dynamic_Loading} \nameref{def:Program}s, we don't need to load the \textbf{entire} program into memory initially.
For example, suppose a program starts with a list of available options from which the user is to select.
Loading the entire program into memory means loading the executable code for \textbf{ALL} options, no matter what option is chosen.

If we chose to load in parts of a \nameref{def:Program} only when they are needed, this is \nameref{def:Demand_Paging}.

\begin{definition}[Demand Paging]\label{def:Demand_Paging}
  \emph{Demand paging} is when individual portions of a \nameref{def:Program} are loaded into memory using \nameref{def:Dynamic_Loading} \textbf{only when they are needed}.

  With demand-paged \nameref{def:Virtual_Memory}, pages are loaded only when they are demanded during program execution.
  Thus, pages that are never accessed are never loaded into physical memory.
\end{definition}

When \nameref{def:Demand_Paging} is combined with \nameref{def:Swapping}, we do not want to swap the whole \nameref{def:Process} into memory if we don't need it.
Thus, we use a \nameref{def:Lazy_Swapper}.

\begin{definition}[Lazy Swapper]\label{def:Lazy_Swapper}
  A \emph{lazy swapper}, like a regular swapper, swaps \nameref{def:Process}es into memory from the \nameref{def:Backing_Store}.
  However, \textbf{it only swaps in pages of the process that will be needed}, and never swaps a page in that will not be used.

  A lazy swapper can be implemented as a \nameref{def:Pager}.
\end{definition}

When a \nameref{def:Process} is to be swapped in, the \nameref{def:Pager} guesses which pages will be used before the process is swapped out again.
Instead of swapping in a whole process, the pager brings only those pages into memory, avoiding reading pages into memory that will not be used anyway, decreasing the swap time and the amount of physical memory needed.

\begin{definition}[Pager]\label{def:Pager}
  In a \nameref{def:Demand_Paging} and \nameref{def:Swapping} system, we are not swapping whole \nameref{def:Process}es into and out of the \nameref{def:Backing_Store}; we are swapping individual pages of the process.
  Thus, the task of moving a process's pages into and out of the backing store is handled by the \emph{pager}, rather than the swapper.
\end{definition}

With a \nameref{def:Pager}, we need some form of hardware support to distinguish between the pages that are in memory and the pages that are on the disk.
The \nameref{par:Page_Valid_Invalid_Bit} scheme can be used for this purpose.
\begin{itemize}[noitemsep]
\item If this bit is set to ``valid'' the associated page is \textbf{both} legal and in memory.
\item If the bit is set to ``invalid'' the page either:
  \begin{itemize}[noitemsep]
  \item Is not valid, i.e.\ not in the \nameref{def:Logical_Address_Space} of the \nameref{def:Process}.
  \item Is valid but is currently on the disk.
  \end{itemize}
\end{itemize}

The page-table entry for a page that is brought into memory is set as usual, but the page-table entry for a page that is not currently in memory is either simply marked invalid or contains the address of the page on disk.
Marking a page invalid will have no effect if the process never attempts to access that page.
If we guess right and page in all pages that are actually needed and only those pages, the \nameref{def:Process} will run exactly as though we had brought in all pages.

While the \nameref{def:Process} executes and accesses pages that are \emph{memory resident}, execution proceeds normally.
Attempting to access a page marked invalid causes a \nameref{def:Page_Fault}.
The \nameref{def:Paging} hardware (\nameref{def:Memory_Management_Unit}), in translating the address through the page table, will notice the invalid bit is set, causing a \nameref{def:Trap} to the \nameref{def:Operating_System}.

\begin{definition}[Page Fault]\label{def:Page_Fault}
  A \emph{page fault} is a hardware \nameref{def:Trap} raised by the \nameref{def:Memory_Management_Unit}.
  It is caused by a \nameref{def:Process} attempting to access a page that has not been brought into memory by the \nameref{def:Pager}.

  The following sequence executes when a page fault occurs:
  \begin{enumerate}[noitemsep]
  \item \nameref{def:Trap} to the operating system.
  \item Save the user registers and process state.
  \item Determine that the interrupt was a page fault.
  \item Check that the page reference was legal and determine the location of the page on the disk.
  \item Issue a read from the disk to a free physical frame:
    \begin{enumerate}[noitemsep]
    \item Wait in a queue for this device until the read request is serviced.
    \item Wait for the device seek and/or latency time.
    \item Begin the transfer of the page to a free frame.
    \end{enumerate}
  \item While waiting, the CPU can be allocated to some other user (CPU scheduling, optional).
  \item Receive an interrupt from the disk I/O subsystem (I/O completed).
  \item Save the registers and process state for the other user (if step 6 is used).
  \item Determine that the interrupt was from the disk.
  \item Correct the \nameref{def:Page_Table} and other tables to show that the desired page is now in memory.
  \item Wait for the CPU to be allocated to this process again.
  \item Restore the user registers, process state, and new page table. Then resume the interrupted instruction.
  \end{enumerate}
\end{definition}

Handling a \nameref{def:Page_Fault} in straightforward.
\begin{enumerate}[noitemsep]
\item We check an internal table (usually kept with the \nameref{def:Process_Control_Block}) for this \nameref{def:Process} to determine whether the reference was a valid or an invalid memory access.
\item If the reference was invalid, we terminate the process (attempt to access memory that is not owned by this \nameref{def:Process}).
  If it was valid and legal, but we have not yet brought in that page, we now page it in.
\item We find a free physical frame (by taking one from the free-frame list, for example).
\item We schedule a disk operation to read the desired page into the newly allocated frame.
\item When the disk read is complete, we modify the page-frame internal table of the \nameref{def:Process} to indicate that the page is now in memory.
\item \textbf{We restart the instruction that was interrupted by the trap.}
  The process can now access the page as though it had always been in memory.
\end{enumerate}

A \nameref{def:Page_Fault} may occur at any memory reference.
A crucial requirement for \nameref{def:Demand_Paging} is the ability to restart any instruction after a page fault.
We save the state (registers, condition code, instruction counter) of the interrupted \nameref{def:Process} when the page fault occurs.
This lets us restart the process in exactly the same place and state, except that the desired page is now in memory and is accessible.

\subsubsection{Problems with Demand Paging}\label{subsubsec:Demand_Paging_Problems}
Theoretically, some programs could access several new pages of memory with each instruction execution (one page for the instruction and many for data), possibly causing multiple page faults per instruction.
This situation would result in unacceptable system performance.
Fortunately, analysis of running processes shows that this behavior is exceedingly unlikely.
Programs tend to have locality of reference giving reasonable performance from demand paging.

The major difficulty arises when one instruction may modify several different locations.
If a block (source or destination) straddles a page boundary, a page fault might occur after the move is partially done.
In addition, if the source and destination blocks overlap, the source block may have been modified, in which case we cannot simply restart the instruction.
This particular problem can be solved in two different ways.
\begin{enumerate}[noitemsep]
\item The microcode computes and attempts to access \textbf{both ends} of \textbf{both blocks}.
  If a \nameref{def:Page_Fault} is going to occur, it will happen at this step, before anything is modified.
  The instruction can then execute, knowing that no page fault can occur, since all relevant pages are in memory.
\item Use temporary registers to hold the values of overwritten locations.
  If there is a \nameref{def:Page_Fault}, all the old values are written back into memory before the \nameref{def:Trap} occurs.
  This restores memory to its state before the instruction was started, so that the instruction can be repeated.
\end{enumerate}

\subsubsection{Hardware Required}\label{subsubsec:Demand_Paging_Required_Hardware}
The hardware to support demand paging is the same as the hardware for \nameref{def:Paging} and \nameref{def:Swapping}:
\begin{enumerate}[noitemsep]
\item Page table.
  This table has the ability to mark an entry invalid through a valid–invalid bit or a special value of protection bits.
\item Secondary memory.
  This memory holds those pages that are not present in main memory.
  The secondary memory is usually a high-speed disk.
  It is known as the \emph{swap device}, and the section of disk used for this purpose is known as swap space.
\end{enumerate}

\subsubsection{Performance}\label{subsubsec:Demand_Paging_Performance}
Demand paging can significantly affect the performance of a computer system.
To illustrate, the \nameref{def:Effective_Access_Time} for demand-paged memory can be computed.

\begin{definition}[Effective Access Time]\label{def:Effective_Access_Time}
  \emph{Effective access time} is the average amount of time required to access a page in a \nameref{def:Demand_Paging} system.
  It accounts for the possibility of a page \textbf{NOT} being in memory and needing to be paged in from the swap and a page being in memory and having a quick, direct access.

  \begin{equation}\label{eq:Effective_Access_Time}
    \mathrm{EAT} = \bigl( (1-p) \times \mathrm{MAT} \bigr) + (p \times \mathrm{PFT})
  \end{equation}
  \begin{description}[noitemsep]
  \item $\mathrm{EAT}$: Effective Access Time.
  \item $p$: Probability of a \nameref{def:Page_Fault} ($0 \leq p \leq 1$).
    \begin{itemize}[noitemsep]
    \item We expect $p$ to be close to zero (we expect to have few page faults).
    \end{itemize}
  \item $\mathrm{MAT}$: Memory Access Time (Typically \SIrange{10}{200}{\nano{} \second{}}).
  \item $\mathrm{PFT}$: \nameref{def:Page_Fault} Time.
  \end{description}
\end{definition}

The \nameref{def:Page_Fault}~Time has 3 major components.
The typical times for these steps, with no waiting in queues, is shown in parentheses:
\begin{enumerate}[noitemsep]
\item Service the page-fault interrupt (\SIrange{1}{100}{\micro{} \second{}}).
\item Read in the page (\SI{8}{\milli{} \second{}}).
\item Restart the process (\SIrange{1}{100}{\micro{} \second{}}).
\end{enumerate}

These only consider the device-service time.
If a queue of processes is waiting for the device, the device-queueing time must be added further increasing the swap time.

\begin{blackbox}
  It is important to keep the page-fault rate low in a demand-paging system.
  Otherwise, the effective access time increases, slowing process execution dramatically.
\end{blackbox}

\subsubsection{Swap Usage}\label{subsubsec:Demand_Paging_Swap_Usage}

%%% Local Variables:
%%% mode: latex
%%% TeX-master: "../../EDAF35-Operating_Systems-Reference_Sheet"
%%% End:
