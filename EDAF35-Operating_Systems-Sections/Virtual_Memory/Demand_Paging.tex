\subsection{Demand Paging}\label{subsec:Demand_Paging}
Because we can support the ability to \nameref{def:Dynamic_Loading} \nameref{def:Program}s, we don't need to load the \textbf{entire} program into memory initially.
For example, suppose a program starts with a list of available options from which the user is to select.
Loading the entire program into memory means loading the executable code for \textbf{ALL} options, no matter what option is chosen.

If we chose to load in parts of a \nameref{def:Program} only when they are needed, this is \nameref{def:Demand_Paging}.

\begin{definition}[Demand Paging]\label{def:Demand_Paging}
  \emph{Demand paging} is when individual portions of a \nameref{def:Program} are loaded into memory using \nameref{def:Dynamic_Loading} \textbf{only when they are needed}.

  With demand-paged \nameref{def:Virtual_Memory}, pages are loaded only when they are demanded during program execution.
  Thus, pages that are never accessed are never loaded into physical memory.
\end{definition}

When \nameref{def:Demand_Paging} is combined with \nameref{def:Swapping}, we do not want to swap the whole \nameref{def:Process} into memory if we don't need it.
Thus, we use a \nameref{def:Lazy_Swapper}.


%%% Local Variables:
%%% mode: latex
%%% TeX-master: "../../EDAF35-Operating_Systems-Reference_Sheet"
%%% End:
