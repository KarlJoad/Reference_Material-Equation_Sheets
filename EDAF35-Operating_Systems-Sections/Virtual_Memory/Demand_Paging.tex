\subsection{Demand Paging}\label{subsec:Demand_Paging}
Because we can support the ability to \nameref{def:Dynamic_Loading} \nameref{def:Program}s, we don't need to load the \textbf{entire} program into memory initially.
For example, suppose a program starts with a list of available options from which the user is to select.
Loading the entire program into memory means loading the executable code for \textbf{ALL} options, no matter what option is chosen.

If we chose to load in parts of a \nameref{def:Program} only when they are needed, this is \nameref{def:Demand_Paging}.

\begin{definition}[Demand Paging]\label{def:Demand_Paging}
  \emph{Demand paging} is when individual portions of a \nameref{def:Program} are loaded into memory using \nameref{def:Dynamic_Loading} \textbf{only when they are needed}.

  With demand-paged \nameref{def:Virtual_Memory}, pages are loaded only when they are demanded during program execution.
  Thus, pages that are never accessed are never loaded into physical memory.
\end{definition}

When \nameref{def:Demand_Paging} is combined with \nameref{def:Swapping}, we do not want to swap the whole \nameref{def:Process} into memory if we don't need it.
Thus, we use a \nameref{def:Lazy_Swapper}.

\begin{definition}[Lazy Swapper]\label{def:Lazy_Swapper}
  A \emph{lazy swapper}, like a regular swapper, swaps \nameref{def:Process}es into memory from the \nameref{def:Backing_Store}.
  However, \textbf{it only swaps in pages of the process that will be needed}, and never swaps a page in that will not be used.

  A lazy swapper can be implemented as a \nameref{def:Pager}.
\end{definition}

When a \nameref{def:Process} is to be swapped in, the \nameref{def:Pager} guesses which pages will be used before the process is swapped out again.
Instead of swapping in a whole process, the pager brings only those pages into memory, avoiding reading pages into memory that will not be used anyway, decreasing the swap time and the amount of physical memory needed.

\begin{definition}[Pager]\label{def:Pager}
  In a \nameref{def:Demand_Paging} and \nameref{def:Swapping} system, we are not swapping whole \nameref{def:Process}es into and out of the \nameref{def:Backing_Store}; we are swapping individual pages of the process.
  Thus, the task of moving a process's pages into and out of the backing store is handled by the \emph{pager}, rather than the swapper.
\end{definition}

With a \nameref{def:Pager}, we need some form of hardware support to distinguish between the pages that are in memory and the pages that are on the disk.
The \nameref{par:Page_Valid_Invalid_Bit} scheme can be used for this purpose.
\begin{itemize}[noitemsep]
\item If this bit is set to ``valid'' the associated page is \textbf{both} legal and in memory.
\item If the bit is set to ``invalid'' the page either:
  \begin{itemize}[noitemsep]
  \item Is not valid, i.e.\ not in the \nameref{def:Logical_Address_Space} of the \nameref{def:Process}.
  \item Is valid but is currently on the disk.
  \end{itemize}
\end{itemize}


%%% Local Variables:
%%% mode: latex
%%% TeX-master: "../../EDAF35-Operating_Systems-Reference_Sheet"
%%% End:
