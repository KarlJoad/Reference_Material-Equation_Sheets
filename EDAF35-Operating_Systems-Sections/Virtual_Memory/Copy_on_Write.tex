\subsection{Copy-on-Write}\label{subsec:Memory_Copy_on_Write}
Recall that the \kernelinline{fork()} \nameref{def:System_Call} creates a child \nameref{def:Process} that is a nearly perfect duplicate of its parent.
Traditionally, this was done by creating a copy of the parent’s address space for the child, thereby \textbf{duplicating} the pages belonging to the parent.
However, considering that many child processes invoke the \kernelinline{exec()} system call immediately after creation, the copying of the parent’s address space may be unnecessary.
\nameref{def:Memory_Copy_on_Write} works by allowing the parent and child processes initially to share the same pages allowing the \kernelinline{fork()} system call to bypass the need for \nameref{def:Demand_Paging} by using a technique similar to page sharing (\Cref{subsec:Paging}).

\begin{definition}[Copy-on-Write]\label{def:Memory_Copy_on_Write}
  \emph{Copy-on-Write} allows the \kernelinline{fork()}ing parent \nameref{def:Process} to share its pages with its newly spawned child process.
  This bypasses the need for \nameref{def:Demand_Paging}, by essentially allowing pages to be shared (like in \nameref{par:Shared_Memory} schemes).
  These shared pages are marked as Copy-on-Write pages; only pages that \textbf{can be} modified need be marked as copy-on-write.
  Pages that cannot be modified (pages containing executable code) can always be shared by the parent and child.

  This allows for rapid process creation and minimizes the number of new pages that must be allocated to the newly created process.
  The pages marked as copy-on-write pages, mean if \textbf{EITHER} process writes to any of the shared pages, a copy of the shared page is created.
  The writer has its \nameref{def:Page_Table} changed to point to this one, which is a complete copy (except for the newly written changes).
  The process that did nothing to the page continues to point to the original page.
\end{definition}

To handle the allocation of pages for a \nameref{def:Memory_Copy_on_Write} page, the location that it is allocated from is important.
Many \nameref{def:Operating_System}s provide a pool of free pages for such requests.
These free pages are typically allocated when the stack or heap for a \nameref{def:Process} must expand or when there are copy-on-write pages to be managed.
Operating systems typically allocate these pages using a technique known as \nameref{def:Zero_Fill_on_Demand}.

\begin{definition}[Zero-Fill-on-Demand]\label{def:Zero_Fill_on_Demand}
  \emph{Zero-fill-on-demand} pages have been zeroed-out before being allocated, thus erasing the previous contents.
\end{definition}

\subsubsection{\texorpdfstring{\kernelinline{vfork()}}{\texttt{vfork()}}}\label{subsubsec:vfork}
Several versions of UNIX have a variation of the \kernelinline{fork()} \nameref{def:System_Call}, \kernelinline{vfork()} (virtual memory fork), that operates differently from \kernelinline{fork()} with regards to \nameref{def:Memory_Copy_on_Write}.
With \kernelinline{vfork()}, the parent \nameref{def:Process} is suspended, and the child process uses the address space of the parent.
Because \kernelinline{vfork()} does not use copy-on-write, if the child process changes any pages of the parent’s address space, the altered pages will be visible to the parent once it resumes.
Therefore, \kernelinline{vfork()} must be used with caution to ensure that the child process does not modify the address space of the parent.
\kernelinline{vfork()} is intended to be used when the child process calls \kernelinline{exec()} immediately after creation.
Because no copying of pages takes place, \kernelinline{vfork()} is an extremely efficient method of process creation.

%%% Local Variables:
%%% mode: latex
%%% TeX-master: "../../EDAF35-Operating_Systems-Reference_Sheet"
%%% End:
