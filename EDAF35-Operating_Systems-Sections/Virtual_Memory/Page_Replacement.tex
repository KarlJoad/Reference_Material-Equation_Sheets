\subsection{Page Replacement}\label{subsec:Page_Replacement}
If a \nameref{def:Process} of ten pages actually uses only half of them, then \nameref{def:Demand_Paging} saves the I/O necessary to load the five pages that are never used.
This also increases the degree of multiprogramming by running more processes, \nameref{def:Over_Allocating} memory.

\begin{definition}[Over-Allocating]\label{def:Over_Allocating}
  \emph{Over-allocating} is the process of putting more load on a component in the system than would normally be possible.

  In the context of \nameref{def:Paging} and \nameref{def:Demand_Paging}, this means that main memory has more \nameref{def:Process}es in it than would be normal in normal \nameref{def:Paging} schemes.
  This is because \textbf{ONLY} the pages that are needed to run these processes are loaded, and nothing more.
\end{definition}

If we run multiple \nameref{def:Process}es, each of which is multiple pages in size but actually only uses some of them, we can achieve higher CPU utilization and throughput, with physical frames to spare.
It is possible that each of these processes, for some reason, may suddenly try to use \textbf{all} of their pages \textbf{simulataneously}, resulting in a need for more physical frames than the system has.
The operating system has several options at this point:
\begin{enumerate}[noitemsep]
\item Terminate a user process.
  \begin{itemize}[noitemsep]
  \item \nameref{def:Demand_Paging} is the operating system’s attempt to improve the system’s utilization and throughput.
  \item Users should not be aware that their processes are running on a paged system—
  \item Paging should be logically transparent to the user.
  \item Not the best choice.
  \end{itemize}

\item Replace one page in memory with the requested one in swap (\nameref{def:Page_Replacement}).
\end{enumerate}


%%% Local Variables:
%%% mode: latex
%%% TeX-master: "../../EDAF35-Operating_Systems-Reference_Sheet"
%%% End:
