\subsection{Other Topics to Consider}\label{subsec:Other_Topics_to_Consider}
Here, we mention considerations, other than a \nameref{def:Page_Replacement} algorithm and frame allocation policy to choosing how to create our \nameref{def:Paging} system.

\subsubsection{Prepaging}\label{subsubsec:Prepaging}
An obvious property of pure \nameref{def:Demand_Paging} is the large number of \nameref{def:Page_Fault}s that occur when a \nameref{def:Process} is started or when a swapped out process is restarted.
This situation results from trying to get the initially executing locality into memory.
\nameref{def:Prepaging} helps ease this situation.

\begin{definition}[Prepaging]\label{def:Prepaging}
  \emph{Prepaging} is an attempt to prevent this high level of initial paging.
  The strategy is to bring into all the pages that will be needed into memory at one time.
\end{definition}

In a system using the \nameref{subsubsec:Working_Set_Model}, for example, we could keep with each \nameref{def:Process} a list of the pages in its working set.
If we must suspend a process, we remember the working set for that process.
When the process is to be resumed, we automatically bring back into memory its \textbf{entire working set} before restarting the process.

\nameref{def:Prepaging} may offer an advantage in some cases.
The question is whether the cost of using prepaging is less than the cost of servicing the corresponding page faults.
Prepaging becomes less effective if many of the pages brought back into memory by prepaging will not be used.


%%% Local Variables:
%%% mode: latex
%%% TeX-master: "../../EDAF35-Operating_Systems-Reference_Sheet"
%%% End:
