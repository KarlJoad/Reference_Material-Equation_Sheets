\subsection{Other Topics to Consider}\label{subsec:Other_Topics_to_Consider}
Here, we mention considerations, other than a \nameref{def:Page_Replacement} algorithm and frame allocation policy to choosing how to create our \nameref{def:Paging} system.

\subsubsection{Prepaging}\label{subsubsec:Prepaging}
An obvious property of pure \nameref{def:Demand_Paging} is the large number of \nameref{def:Page_Fault}s that occur when a \nameref{def:Process} is started or when a swapped out process is restarted.
This situation results from trying to get the initially executing locality into memory.
\nameref{def:Prepaging} helps ease this situation.

\begin{definition}[Prepaging]\label{def:Prepaging}
  \emph{Prepaging} is an attempt to prevent this high level of initial paging.
  The strategy is to bring into all the pages that will be needed into memory at one time.
\end{definition}

In a system using the \nameref{subsubsec:Working_Set_Model}, for example, we could keep with each \nameref{def:Process} a list of the pages in its working set.
If we must suspend a process, we remember the working set for that process.
When the process is to be resumed, we automatically bring back into memory its \textbf{entire working set} before restarting the process.

\nameref{def:Prepaging} may offer an advantage in some cases.
The question is whether the cost of using prepaging is less than the cost of servicing the corresponding page faults.
Prepaging becomes less effective if many of the pages brought back into memory by prepaging will not be used.

\subsubsection{Page Size}\label{subsubsec:Page_Size}
The designers of an \nameref{def:Operating_System} for an existing machine rarely have a choice concerning the page size.
However, a new machine can have different page sizes than its predecessors.
There is no single best page size.
Page sizes are almost always powers of 2, generally ranging from \SIrange{4096}{4194304}{\byte{}} ($2^{12}$ to $2^{22}$ byte, \SI{4}{\kibi{} \byte{}} to \SI{4}{\mebi{} \byte{}}).

How do we select a page size?
\begin{itemize}[noitemsep]
\item One concern is the size of the page table.
  \begin{itemize}[noitemsep]
  \item For a given virtual memory space, decreasing the page size increases the number of pages and hence the size of the page table.
  \end{itemize}

\item Because each active process must have its own copy of the \nameref{def:Page_Table}, a large page size is desirable.

\item Memory is better utilized with smaller pages.
  \begin{itemize}[noitemsep]
  \item To minimize \nameref{def:Internal_Fragmentation}, we need a small page size.
  \end{itemize}

\item Time required to read or write a page.
  Minimization of I/O time argues for a larger page size.
  \begin{itemize}[noitemsep]
  \item I/O time is composed of seek, latency, and transfer times.
    Transfer time is proportional to the amount transferred, meaning a small page size.
  \item Latency and seek time normally dwarf transfer time.
  \end{itemize}

\item With a smaller page size, total I/O should be reduced, since locality will be improved.
  \begin{itemize}[noitemsep]
  \item A smaller page size allows each page to match program locality more accurately.
\end{itemize}

\item With a smaller page size, then, we have better resolution, allowing us to isolate only the memory that is actually needed.
  \begin{itemize}[noitemsep]
  \item With a larger page size, we must allocate and transfer not only what is needed but also anything else that happens to be in the page, whether it is needed or not.
  \item Thus, a smaller page size should result in less I/O and less total allocated memory.
\end{itemize}
\item To minimize the number of page faults, we need to have a large page size.

\item What is the relationship between page size and sector size on the paging device?
\end{itemize}


%%% Local Variables:
%%% mode: latex
%%% TeX-master: "../../EDAF35-Operating_Systems-Reference_Sheet"
%%% End:
