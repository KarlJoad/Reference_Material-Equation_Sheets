\section{File-System Implementation}\label{sec:FS_Implementation}
To improve I/O efficiency, I/O transfers between memory and disk are performed in units of blocks.
Each block has one or more sectors.
Usually, the sector size is 512 bytes, but can vary from 32 bytes to 4,096 bytes.

\begin{definition}[File System]\label{def:File_System}
  \emph{File system}s provide efficient and convenient access to the disk by allowing data to be stored, located, and retrieved easily.

  The \emph{file system} (\emph{filesystem}, \emph{FS}) consists of two parts:
  \begin{enumerate}[noitemsep]
  \item A collection of \nameref{def:File}s, each storing related data.
  \item A directory structure, which organizes and provides information about all the files in the system.
  \end{enumerate}

  A file system has two quite different design problems.
  \begin{enumerate}[noitemsep]
  \item How the file system should look to the user.
    \begin{itemize}[noitemsep]
    \item This task involves defining a file and its attributes, the operations allowed on a file, and the directory structure for organizing files.
    \end{itemize}

  \item Creating algorithms and data structures to map the logical file system onto the physical secondary-storage devices.
  \end{enumerate}
\end{definition}

\subsection{File System Structure}\label{subsec:File_System_Structure}
The file system itself is generally composed of many different levels.
The levels are listed below, from highest to lowest.
\begin{itemize}[noitemsep]
\item \nameref{subsubsec:Logical_FS_Module}
\item \nameref{subsubsec:File_Organization_FS_Module}
\item \nameref{subsubsec:Basic_FS_Module}
\item \nameref{subsubsec:IO_Control_FS_Module}
\end{itemize}


%%% Local Variables:
%%% mode: latex
%%% TeX-master: "../../EDAF35-Operating_Systems-Reference_Sheet"
%%% End:



%%% Local Variables:
%%% mode: latex
%%% TeX-master: "../EDAF35-Operating_Systems-Reference_Sheet"
%%% End:
