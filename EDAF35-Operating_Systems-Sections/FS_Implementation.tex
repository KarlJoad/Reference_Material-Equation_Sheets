\section{File-System Implementation}\label{sec:FS_Implementation}
To improve I/O efficiency, I/O transfers between memory and disk are performed in units of blocks.
Each block has one or more sectors.
Usually, the sector size is 512 bytes, but can vary from 32 bytes to 4,096 bytes.

\begin{definition}[File System]\label{def:File_System}
  \emph{File system}s provide efficient and convenient access to the disk by allowing data to be stored, located, and retrieved easily.

  The \emph{file system} (\emph{filesystem}, \emph{FS}) consists of two parts:
  \begin{enumerate}[noitemsep]
  \item A collection of \nameref{def:File}s, each storing related data.
  \item A directory structure, which organizes and provides information about all the files in the system.
  \end{enumerate}

  A file system has two quite different design problems.
  \begin{enumerate}[noitemsep]
  \item How the file system should look to the user.
    \begin{itemize}[noitemsep]
    \item This task involves defining a file and its attributes, the operations allowed on a file, and the directory structure for organizing files.
    \end{itemize}

  \item Creating algorithms and data structures to map the logical file system onto the physical secondary-storage devices.
  \end{enumerate}
\end{definition}

\subsection{File System Structure}\label{subsec:File_System_Structure}
The file system itself is generally composed of many different levels.
The levels are listed below, from highest to lowest.
\begin{itemize}[noitemsep]
\item \nameref{subsubsec:Logical_FS_Module}
\item \nameref{subsubsec:File_Organization_FS_Module}
\item \nameref{subsubsec:Basic_FS_Module}
\item \nameref{subsubsec:IO_Control_FS_Module}
\end{itemize}


%%% Local Variables:
%%% mode: latex
%%% TeX-master: "../../EDAF35-Operating_Systems-Reference_Sheet"
%%% End:


\subsection{File System Implementation}\label{subsec:FS_Implementation}
Several on-disk and in-memory structures are used to implement a \nameref{def:File_System}.
These structures vary depending on the operating system and the file system, but some general principles apply.

\begin{itemize}[noitemsep]
\item Boot Control Block (per volume, if needed).
  Contains information needed by the system to boot an operating system from that volume.
  If the disk does not contain an operating system, this block can be empty.
  It is typically the first block of a volume.
  \begin{itemize}[noitemsep]
  \item In UFS, it is the boot block.
  \item In NTFS, it is the boot sector.
  \end{itemize}

\item Volume Control Block (per volume).
  contains volume (or partition) details, such as:
  \begin{itemize}[noitemsep]
  \item Number of blocks in the partition,
  \item Size of the blocks,
  \item Free-block count
  \item Free-block pointers
  \item Free-\nameref{def:File_Control_Block} count
  \item FCB pointers
  \end{itemize}
  \begin{itemize}[noitemsep]
  \item In UFS, this is called a superblock.
  \item In NTFS, it is stored in the master file table.
  \end{itemize}

\item Directory Structure (per file system) is used to organize the files.

\item Per-file \nameref{def:File_Control_Block}.
  Contains many details about the file.
  It has a unique identifier number to allow association with a directory entry.
\end{itemize}

In-memory information is used for both file-system management and performance improvement via caching.
The data is loaded at mount time, updated during file-system operations, and discarded at dismount.
\begin{itemize}[noitemsep]
\item An in-memory mount table contains information about each mounted volume.
\item An in-memory directory-structure cache holds the directory information of recently accessed directories.
  (For mount directories, it can contain a pointer to the volume table.)
\item The system-wide open-file table has a copy of the FCB of each open file, as well as other information.
\item The per-process open-file table contains a pointer to the appropriate entry in the system-wide open-file table, as well as process-dependent information.
\item Buffers hold file-system blocks when they are being read from disk or written to disk.
\end{itemize}


%%% Local Variables:
%%% mode: latex
%%% TeX-master: "../../EDAF35-Operating_Systems-Reference_Sheet"
%%% End:



%%% Local Variables:
%%% mode: latex
%%% TeX-master: "../EDAF35-Operating_Systems-Reference_Sheet"
%%% End:
