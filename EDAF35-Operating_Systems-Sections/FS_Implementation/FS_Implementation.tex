\subsection{File System Implementation}\label{subsec:FS_Implementation}
Several on-disk and in-memory structures are used to implement a \nameref{def:File_System}.
These structures vary depending on the operating system and the file system, but some general principles apply.

\begin{itemize}[noitemsep]
\item Boot Control Block (per volume, if needed).
  Contains information needed by the system to boot an operating system from that volume.
  If the disk does not contain an operating system, this block can be empty.
  It is typically the first block of a volume.
  \begin{itemize}[noitemsep]
  \item In UFS, it is the boot block.
  \item In NTFS, it is the boot sector.
  \end{itemize}

\item Volume Control Block (per volume).
  contains volume (or partition) details, such as:
  \begin{itemize}[noitemsep]
  \item Number of blocks in the partition,
  \item Size of the blocks,
  \item Free-block count
  \item Free-block pointers
  \item Free-\nameref{def:File_Control_Block} count
  \item FCB pointers
  \end{itemize}
  \begin{itemize}[noitemsep]
  \item In UFS, this is called a superblock.
  \item In NTFS, it is stored in the master file table.
  \end{itemize}

\item Directory Structure (per file system) is used to organize the files.

\item Per-file \nameref{def:File_Control_Block}.
  Contains many details about the file.
  It has a unique identifier number to allow association with a directory entry.
\end{itemize}

In-memory information is used for both file-system management and performance improvement via caching.
The data is loaded at mount time, updated during file-system operations, and discarded at dismount.
\begin{itemize}[noitemsep]
\item An in-memory mount table contains information about each mounted volume.
\item An in-memory directory-structure cache holds the directory information of recently accessed directories.
  (For mount directories, it can contain a pointer to the volume table.)
\item The system-wide open-file table has a copy of the FCB of each open file, as well as other information.
\item The per-process open-file table contains a pointer to the appropriate entry in the system-wide open-file table, as well as process-dependent information.
\item Buffers hold file-system blocks when they are being read from disk or written to disk.
\end{itemize}

\subsubsection{File Creation}\label{subsubsec:File_Creation}
To create a new file, an application program calls the logical file system.
The logical file system knows the format of the directory structures.
\begin{enumerate}[noitemsep]
\item To create a new file, it allocates a new \nameref{def:File_Control_Block}.
  \begin{itemize}[noitemsep]
  \item If the file-system implementation creates all FCBs at file-system creation time, an FCB is allocated from the set of free FCBs.
  \end{itemize}
\item The system then reads the appropriate directory into memory, updates it with the new file name and FCB
\item Writes it back to the disk.
\end{enumerate}

\subsubsection{File Opening}\label{subsubsec:File_Opening}
Now that a file has been created, it can be used for I/O.
\begin{enumerate}[noitemsep]
\item First, it must be opened. The \kernelinline{open()} call passes a file name to the logical file system.
\item The \kernelinline{open()} system call first searches the system-wide open-file table to see if the file is already in use by another process.
  \begin{itemize}[noitemsep]
  \item If it is, a per-process open-file table entry is created pointing to the existing system-wide open-file table.
    \begin{itemize}[noitemsep]
    \item This algorithm can save substantial overhead.
    \end{itemize}
  \item If the file is not already open, the directory structure is searched for the given file name.
    \begin{itemize}[noitemsep]
    \item Parts of the directory structure are usually cached in memory to speed directory operations.
    \end{itemize}
  \end{itemize}

\item Once the file is found, the FCB is copied into a system-wide open-file table in memory.
  \begin{itemize}[noitemsep]
  \item This table tracks the \nameref{def:File_Control_Block} and the number of processes that have the file open.
  \end{itemize}

\item An entry is made in the per-process open-file table, with a pointer to the entry in the system-wide open-file table and some other fields.
  \begin{itemize}[noitemsep]
  \item These other fields may include a pointer to the current location in the file (for the next \kernelinline{read()} or \kernelinline{write()} operation) and the access mode in which the file is open.
  \item The \kernelinline{open()} call returns a pointer to the appropriate entry in the per-process file-system table.
  \item All file operations are then performed via this pointer.
  \item The file name may not be part of the open-file table, as the system has no use for it once the appropriate FCB is located on disk.
  \item It could be cached, though, to save time on subsequent opens of the same file.
  \item The name given to the entry varies.
  \item \textsc{unix} systems refer to it as a file descriptor; Windows refers to it as a file handle.
  \end{itemize}

\item When a process closes the file, the per-process table entry is removed, and
  the system-wide entry’s open count is decremented.
\item When all users that have
  opened the file close it, any updated \nameref{def:File_Metadata} is copied back to the disk-based
  directory structure, and the system-wide open-file table entry is removed.
\end{enumerate}

\subsubsection{Partitions and Mounting}\label{subsubsec:Partitions_Mounting}
\paragraph{Raw Partitions}\label{par:Raw_Partitions}
Each partition can be either ``raw,'' containing no file system, or ``cooked,'' containing a file system.
A raw disk is used where no file system is appropriate.
\textsc{unix} swap space can use a raw partition, since it uses its own format on disk and does not use a file system.

\paragraph{Boot Partitions}\label{par:Boot_Partitions}
Boot information can be stored in a separate partition.
Again, it has its own format, because at boot time the system does not have the file-system code loaded and cannot interpret the file-system format.
Rather, boot information is usually a sequential series of blocks, loaded as an image into memory.
Execution of the image starts at a predefined location.

This \nameref{def:Bootloader} in turn knows enough about the \nameref{def:File_System} structure to be able to find and load the \nameref{def:Kernel} and start its execution.
The bootloader can contain more than just the instructions required to boot a specific \nameref{def:Operating_System}.

\paragraph{The Root Partition}\label{par:Root_Partition}
The root partition, which contains the \nameref{def:Operating_System} \nameref{def:Kernel} and other system files, is mounted at boot.
Other volumes can be automatically mounted at boot or manually mounted later.

As part of a successful mount operation, the operating system verifies that the device contains a valid file system.
It does so by asking the device driver to read the device directory and verifying that the directory has the expected format.
If the format is invalid, the partition must have its consistency checked and possibly corrected, either with or without user intervention.
Finally, the operating system notes in its in-memory mount table that a file system is mounted, along with the type of the file system.

\subsubsection{Virtual File Systems}\label{subsubsec:Virtual_File_Systems}
Most operating systems, use object-oriented techniques to simplify, organize, and modularize the implementation of \nameref{def:File_System}s.
The use of these methods allows very dissimilar file-system types to be implemented within the same structure, including network file systems.
Data structures and procedures are used to isolate the basic \nameref{def:System_Call} functionality from the implementation details.
Thus, the file-system implementation consists of three major layers:
\begin{enumerate}[noitemsep]
\item The \nameref{def:File_System} interface, based on the \kernelinline{open()}, \kernelinline{read()}, \kernelinline{write()}, and \kernelinline{close()} calls and on file descriptors.
\item The \nameref{def:Virtual_File_System} layer.
\item The layer implementing the actual file-system type or the remote-file-system protocol.
\end{enumerate}

\begin{definition}[Virtual File System]\label{def:Virtual_File_System}
  A \emph{Virtual File System} (\emph{VFS}) separates file-system-generic operations from their implementation.
  The VFS activates file-system-specific operations to handle local requests according to their file-system types.

  Multiple VFS interface may coexist on the same machine, allowing transparent access to different types of file systems mounted locally.
  It provides a mechanism for uniquely representing a file throughout a network, the \nameref{def:Vnode}.
\end{definition}


%%% Local Variables:
%%% mode: latex
%%% TeX-master: "../../EDAF35-Operating_Systems-Reference_Sheet"
%%% End:
