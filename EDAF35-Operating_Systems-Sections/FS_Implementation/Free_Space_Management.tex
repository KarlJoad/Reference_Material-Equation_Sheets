\subsection{Free-Space Management}\label{subsec:Free_Space_Management}
Since disk space is limited, we need to reuse the space from deleted files for new files, if possible.
To keep track of free disk space, the system maintains a free-space list (which may not be implemented as a list).
The free-space list records all free disk blocks—those not allocated to some file or directory.
To create a file, we search the free-space list for the required amount of space and allocate that space to the new file.
This space is then removed from the free-space list.
When a file is deleted, its disk space is added back to the free-space list.

\subsubsection{Bit Vector}\label{subsubsec:Free_Space_Bit_Vector}
Frequently, the free-space list is implemented as a bit map or bit vector.
Each block is represented by 1 bit.
\begin{itemize}[noitemsep]
\item If block \textbf{free}, the bit is 1.
\item If block \textbf{allocated}, the bit is 0.
\end{itemize}

\paragraph{Advantages of Bit Vector}\label{par:Free_Space_Bit_Vector_Advantages}
The main advantages are:
\begin{itemize}[noitemsep]
\item The relative \textbf{simplicity} in finding the first free block or $n$ consecutive free blocks on the disk.
\item The relative \textbf{efficiency} in finding the first free block or $n$ consecutive free blocks on the disk.
\end{itemize}

\paragraph{Disadvantages of Bit Vector}\label{par:Free_Space_Bit_Vector_Disadvantages}
Bit vectors are inefficient unless the \textbf{entire} vector is kept in main memory.
Keeping it in main memory is possible for smaller disks but not larger ones.

A \SI{1.3}{\gibi{} \byte{}} disk with 512 byte blocks would need a bit map of over \SI{332}{\kibi{} \byte{}} to track its free blocks.
\nameref{def:Disk_Cluster}ing the blocks in groups of four (making a cluster 2048 bytes) reduces the bit vector size to around \SI{83}{\kibi{} \byte{}}.


%%% Local Variables:
%%% mode: latex
%%% TeX-master: "../../EDAF35-Operating_Systems-Reference_Sheet"
%%% End:
