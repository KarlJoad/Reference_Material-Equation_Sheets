\subsection{Free-Space Management}\label{subsec:Free_Space_Management}
Since disk space is limited, we need to reuse the space from deleted files for new files, if possible.
To keep track of free disk space, the system maintains a free-space list (which may not be implemented as a list).
The free-space list records all free disk blocks—those not allocated to some file or directory.
To create a file, we search the free-space list for the required amount of space and allocate that space to the new file.
This space is then removed from the free-space list.
When a file is deleted, its disk space is added back to the free-space list.

\subsubsection{Bit Vector}\label{subsubsec:Free_Space_Bit_Vector}
Frequently, the free-space list is implemented as a bit map or bit vector.
Each block is represented by 1 bit.
\begin{itemize}[noitemsep]
\item If block \textbf{free}, the bit is 1.
\item If block \textbf{allocated}, the bit is 0.
\end{itemize}

\paragraph{Advantages of Bit Vector}\label{par:Free_Space_Bit_Vector_Advantages}
The main advantages are:
\begin{itemize}[noitemsep]
\item The relative \textbf{simplicity} in finding the first free block or $n$ consecutive free blocks on the disk.
\item The relative \textbf{efficiency} in finding the first free block or $n$ consecutive free blocks on the disk.
\end{itemize}

\paragraph{Disadvantages of Bit Vector}\label{par:Free_Space_Bit_Vector_Disadvantages}
Bit vectors are inefficient unless the \textbf{entire} vector is kept in main memory.
Keeping it in main memory is possible for smaller disks but not larger ones.

A \SI{1.3}{\gibi{} \byte{}} disk with 512 byte blocks would need a bit map of over \SI{332}{\kibi{} \byte{}} to track its free blocks.
\nameref{def:Disk_Cluster}ing the blocks in groups of four (making a cluster 2048 bytes) reduces the bit vector size to around \SI{83}{\kibi{} \byte{}}.

\subsubsection{Linked List}\label{subsubsec:Free_Space_Linked_List}
Another approach to \nameref{subsec:Free_Space_Management} is to link together all the free disk blocks, keeping a pointer to the first free block in a special location on the disk and caching it in memory.
This first block contains a pointer to the next free disk block, and so on.
The user is unable to see these free blocks.

This scheme is not efficient; to traverse the list, we must read each block, which requires substantial I/O time.
However, traversing the free list is not a frequent action.
Usually, the operating system simply needs a few free blocks so to allocate to a file.

\paragraph{File Allocation Table}\label{par:FAT_Free_Space}
The \nameref{par:File_Allocation_Table} method incorporates free-block accounting into the allocation table (which is already cached too).
No separate method is needed.

\subsubsection{Grouping}\label{subsubsec:Free_Space_Grouping}
A modification of the \nameref{subsubsec:Free_Space_Linked_List} approach mirrors the use of an \nameref{subsubsec:Indexed_File_Allocation} method.
In this case, the first free block stores the addresses of the next $n-1$ free blocks.
The last block contains the address of another free-block index.
This way, a large number of free blocks can be found quickly.

\subsubsection{Counting}\label{subsubsec:Free_Space_Counting}
Another method of tracking the free blocks mirrors the \nameref{subsubsec:Contiguous_File_Allocation} method with \nameref{def:Extent}s.
Generally, several contiguous blocks may be allocated or freed simultaneously, particularly when space is allocated with the contiguous-allocation algorithm or through \nameref{def:Disk_Cluster}ing.
Rather than keeping a list of $n$ free disk addresses, we can keep the address of the first free block and the number of contiguous blocks that are free that follow the first block.
Each entry in the free-space list then consists of a disk address and a count.
These entries can be stored in a balanced tree (or other structure), for efficient lookup, insertion, and deletion.

%%% Local Variables:
%%% mode: latex
%%% TeX-master: "../../EDAF35-Operating_Systems-Reference_Sheet"
%%% End:
