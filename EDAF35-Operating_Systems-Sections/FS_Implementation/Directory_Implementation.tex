\subsection{Directory Implementation}\label{subsec:Directory_Implementation}
The selection of directory-allocation and directory-management algorithms significantly affects the efficiency, performance, and reliability of the file system.

\subsubsection{Linear List}\label{subsubsec:Linear_List_Directory}
The simplest method of implementing a directory is to use a linear list of file names with pointers to the data blocks.
This method is simple to program but time-consuming to execute.
\begin{itemize}[noitemsep]
\item To create a new file, we must first \textbf{search} the directory to be sure that no existing file has the same name.
  Then, we add a new entry at the end of the directory.
\item To delete a file, we \textbf{search} the directory for the named file and then release the space allocated to it.
\item To reuse the directory entry, we can do one of several things:
  \begin{enumerate}[noitemsep]
  \item Mark the entry as unused (by assigning it a special name, or by including a used/unused bit).
  \item Attach it to a list of free directory entries.
  \item Copy the last entry in the directory into the freed location and to decrease the length of the directory.
  \end{enumerate}
\end{itemize}

The real disadvantage of a linear list of directory entries is that finding a file requires a linear search.
Directory information is used frequently, and users will notice if access to it is slow.


%%% Local Variables:
%%% mode: latex
%%% TeX-master: "../../EDAF35-Operating_Systems-Reference_Sheet"
%%% End:
