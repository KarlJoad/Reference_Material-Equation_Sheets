\subsection{File System Structure}\label{subsec:File_System_Structure}
The file system itself is generally composed of many different levels.
The levels are listed below, from highest to lowest.
\begin{itemize}[noitemsep]
\item \nameref{subsubsec:Logical_FS_Module}
\item \nameref{subsubsec:File_Organization_FS_Module}
\item \nameref{subsubsec:Basic_FS_Module}
\item \nameref{subsubsec:IO_Control_FS_Module}
\end{itemize}

Each level in the design uses the features of lower levels to create new features for use by higher levels.
This layered structure is useful for minimizing the duplication of code.
The I/O control and sometimes the basic file-system code can be used by multiple file systems.
Each file system can then have its own logical file-system and file-organization modules.
However, each layer introduces more overhead.

\subsubsection{Logical File System Module}\label{subsubsec:Logical_FS_Module}
The logical file system manages \nameref{def:File_Metadata} information.

\begin{definition}[Metadata]\label{def:File_Metadata}
  \emph{Metadata} includes all of the file-system structure except the actual data (or contents of the files).
\end{definition}

The logical file system manages the directory structure to provide the file-organization module with the information it needs, when given a symbolic file name.
It does this by maintaining file structure via \nameref{def:File_Control_Block}s.


%%% Local Variables:
%%% mode: latex
%%% TeX-master: "../../EDAF35-Operating_Systems-Reference_Sheet"
%%% End:
