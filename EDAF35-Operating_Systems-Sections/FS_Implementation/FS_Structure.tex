\subsection{File System Structure}\label{subsec:File_System_Structure}
The file system itself is generally composed of many different levels.
The levels are listed below, from highest to lowest.
\begin{itemize}[noitemsep]
\item \nameref{subsubsec:Logical_FS_Module}
\item \nameref{subsubsec:File_Organization_FS_Module}
\item \nameref{subsubsec:Basic_FS_Module}
\item \nameref{subsubsec:IO_Control_FS_Module}
\end{itemize}

Each level in the design uses the features of lower levels to create new features for use by higher levels.
This layered structure is useful for minimizing the duplication of code.
The I/O control and sometimes the basic file-system code can be used by multiple file systems.
Each file system can then have its own logical file-system and file-organization modules.
However, each layer introduces more overhead.

\subsubsection{Logical File System Module}\label{subsubsec:Logical_FS_Module}
The logical file system manages \nameref{def:File_Metadata} information.

\begin{definition}[Metadata]\label{def:File_Metadata}
  \emph{Metadata} includes all of the file-system structure except the actual data (or contents of the files).
\end{definition}

The logical file system manages the directory structure to provide the file-organization module with the information it needs, when given a symbolic file name.
It does this by maintaining file structure via \nameref{def:File_Control_Block}s.

\begin{definition}[File-Control Block]\label{def:File_Control_Block}
  A \emph{File-Control Block} (\emph{FCB}) (inode in \textsc{unix}) contains information about the \nameref{def:File}, including ownership, permissions, and location of the file contents.
\end{definition}

\subsubsection{File Organization Module}\label{subsubsec:File_Organization_FS_Module}
The file-organization module knows about files and their logical blocks, as well as physical blocks.
By knowing the type of file allocation used and the location of the file, the file-organization module can translate logical block addresses to physical block addresses for the basic file system to transfer.

This module also includes the free-space manager, which tracks unallocated blocks and provides these blocks to the file-organization module when requested.

\subsubsection{Basic File System Module}\label{subsubsec:Basic_FS_Module}
The basic file system only issues generic commands to the appropriate device driver to read and write physical blocks on the disk.
Each physical block is identified by its numeric disk address (Drive 1, Cylinder 73, Track 2, Sector 10).
This layer also manages the memory buffers and caches that hold various file-system, directory, and data blocks.
A block in the buffer is allocated before the transfer of a disk block can occur.
When the buffer is full, the buffer manager must find more buffer memory or free up buffer space to allow a requested I/O to complete.

Caches are used to hold frequently used file-system metadata to improve performance, so managing their contents is critical for optimum system performance.


%%% Local Variables:
%%% mode: latex
%%% TeX-master: "../../EDAF35-Operating_Systems-Reference_Sheet"
%%% End:
