\subsection{Recovery}\label{subsec:Recovery}
Files and directories are kept both in main memory and on disk, and care must be taken to ensure that a system failure does not result in loss of data or in data inconsistency.
A system crash can cause inconsistencies among on-disk file-system data structures, such as \nameref{def:Directory} structures, free-block pointers, and free \nameref{def:File_Control_Block} pointers.
Many file systems apply changes to these structures in place.
These changes can be interrupted by a crash, and inconsistencies among the structures can result.

A typical operation, such as creating a file, can involve many structural changes within the \nameref{def:File_System} on-disk.
Compounding this is the caching that operating systems do to optimize I/O performance.
Some changes may go directly to disk, while others may be cached.
If the cached changes do not reach disk before a crash occurs, more corruption is possible.


%%% Local Variables:
%%% mode: latex
%%% TeX-master: "../../EDAF35-Operating_Systems-Reference_Sheet"
%%% End:
