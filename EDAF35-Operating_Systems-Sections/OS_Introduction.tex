\section{Operating System Introduction}\label{sec:OS_Intro}
A computer system can be roughly divided into 4 parts.
\begin{itemize}[noitemsep]
\item The \nameref{def:Hardware}
\item The \nameref{def:Operating_System}
\item The \nameref{def:Application_Program}s
\item The Users
\end{itemize}

\begin{definition}[Hardware]\label{def:Hardware}
  \emph{Hardware} is the physical components of the system and provide the basic computing resources for the system..
  Hardware includes the \nameref{def:CPU}, \nameref{def:Memory}, and all I/O devices (monitor, keyboard, mouse, etc.).

  \begin{remark}[How to Differentiate]\label{rmk:Hardware_Differentiate}
    If you are finding it difficult to tell \nameref{def:Hardware}, \nameref{def:Software}, and \nameref{def:Firmware} apart, answer this simple question.
    Can you hit it with a hammer and break the thing?
    \begin{description}[noitemsep]
    \item[Yes] Then it is \nameref{def:Hardware}.
    \item[No] Then it is \nameref{def:Software}.
    \item[Yes and No] Then it is \nameref{def:Firmware}.
    \end{description}
  \end{remark}
\end{definition}

\begin{definition}[Software]\label{def:Software}
  \emph{Software} is the code that is used to build the system and make it perform operations.
  Technically, it is the electrical signals that represent \texttt{0} or \texttt{1} and makes the \nameref{def:Hardware} act in a specific, desired fashion to produce some result.

  On a higher level, this can be though of as computer code.

  \begin{remark}[How to Differentiate]\label{rmk:Software_Differentiate}
    If you are finding it difficult to tell \nameref{def:Hardware}, \nameref{def:Software}, and \nameref{def:Firmware} apart, answer this simple question.
    Can you hit it with a hammer and break the thing?
    \begin{description}[noitemsep]
    \item[Yes] Then it is \nameref{def:Hardware}.
    \item[No] Then it is \nameref{def:Software}.
    \item[Yes and No] Then it is \nameref{def:Firmware}.
    \end{description}
  \end{remark}
\end{definition}

\begin{restatable}[Operating System]{definition}{defOperatingSystem}\label{def:Operating_System}
  An \emph{operating system} is a large piece of software that controls the \nameref{def:Hardware} and coordinates the many \nameref{def:Application_Program}s various numbers of \nameref{def:User}s may use.
  It provides the means for proper use of these resources to allow the computer to run.

  By itself, an operating system does nothing useful.
  It simply provides an \textbf{environment} within which other programs can perform useful work.

  The fundamental goal of computer systems is to execute user programs and to make solving user problems easier.
  These programs require certain common operations, such as those controlling the I/O devices.

  In addition, there is no universally accepted definition of what is part of the operating system.
  A simple definition is that it includes everything a vendor ships when you order ``the operating system.''
  The features included, however, vary greatly across systems.
  Some systems take up less than a megabyte of space and lack even a full-screen editor, whereas others require gigabytes of space and are based entirely on graphical windowing systems.
  A more common definition, and the one that we usually follow, is that the operating system is the one program running at all times on the computer—usually called the \nameref{def:Kernel}.

  \begin{remark}[Kernel-Level Non-Kernal Programs]\label{rmk:Kernel_Level_Non_Kernel_Programs}
    Along with the \nameref{def:Kernel}, there are two other types of programs:
    \begin{enumerate}[noitemsep]
    \item System Programs,
      \begin{itemize}[noitemsep]
      \item Associated with the \nameref{def:Operating_System} but are not necessarily part of the \nameref{def:Kernel}.
      \end{itemize}
    \item Application Programs
      \begin{itemize}[noitemsep]
      \item Includes all programs not associated with the operation of the system
      \end{itemize}
    \end{enumerate}
  \end{remark}
\end{restatable}

\begin{definition}[Kernel]\label{def:Kernel}
  The kernel is a computer program at the core of a computer's operating system with complete control over everything in the system.
  It is the ``portion of the operating system code that is always resident in memory''.
  It facilitates interactions between hardware and software components.
  On most systems, it is one of the first programs loaded on startup (after the bootloader).
  It handles input/output requests from software, translating them into data-processing instructions for the central processing unit.
  It handles memory and its mapping, peripherals like: keyboards, monitors, printers, and speakers.
  A kernel connects the application software to the hardware of a computer.

  The critical code of the kernel is usually loaded into a separate area of memory, which is protected from access by application programs or other, less critical parts of the operating system.
  The kernel performs its tasks, such as running processes, managing hardware devices such as the hard disk, and handling interrupts, in this protected kernel space.
\end{definition}

In modern systems, we tend to have much more than a single \nameref{def:CPU} core.
In these multicore/multiprocessor systems, there are 2 ways to organize the way jobs and cores are handled.
\begin{enumerate}[noitemsep]
\item \nameref{def:Symmetric_Multiprocessor_System}
\item \nameref{def:Asymmetric_Multiprocessor_System}
\end{enumerate}

\begin{definition}[Symmetric Multiprocessor System]\label{def:Symmetric_Multiprocessor_System}
  In a \emph{symmetric multiprocessor system}, there are multiple \nameref{def:CPU}s working together.
  What makes this symmetric is that all CPUs are equal, \textbf{there is no single coordinating CPU}.
  This means that in a 4 CPU system, all 4 CPUs are peers and can work together.

  Any CPU can do anything at any time, no matter what any other core is doing.

  This is in contrast to an \nameref{def:Asymmetric_Multiprocessor_System}.
\end{definition}

\begin{definition}[Asymmetric Multiprocessor System]\label{def:Asymmetric_Multiprocessor_System}
  An \emph{asymmetric multiprocessor system} has multiple \nameref{def:CPU}s working together.
  However, to coordinate all the calculations and operations, \textbf{a single CPU is designated the master CPU}.
  Then, all the other CPUs are slave/worker CPUs.

  This is in contrast to an \nameref{def:Symmetric_Multiprocessor_System}.
\end{definition}

\begin{definition}[Application Program]\label{def:Application_Program}
  An \emph{application program} is a tool used by a \nameref{def:User} to solve some problem.
  This is the main thing a normal person will interact with.
  These pieces of software can include:
  \begin{itemize}[noitemsep]
  \item Text editors
  \item Compilers
  \item Web browsers
  \item Word Processors
  \item Spreadsheets
  \item etc.
  \end{itemize}
\end{definition}

Additionally, we can have multiple ways of working with system \nameref{def:Memory}.
The 2 main ways are:
\begin{enumerate}[noitemsep]
\item \nameref{def:Uniform_Memory_Access}
\item \nameref{def:Non_Uniform_Memory_Access}
\end{enumerate}

\begin{definition}[Uniform Memory Access]\label{def:Uniform_Memory_Access}
  In \emph{Uniform Memory Access} (\emph{UMA}), \textbf{ALL} system \nameref{def:Memory} is accessed the same way by \textbf{ALL} cores.
\end{definition}

\begin{definition}[Non-Uniform Memory Access]\label{def:Non_Uniform_Memory_Access}
  In \emph{Non-Uniform Memory Access} (\emph{NUMA}), some cores have to behave differently to access some \nameref{def:Memory}.
\end{definition}

\begin{definition}[User]\label{def:User}
  A \emph{user} is the person and/or thing that is running some \nameref{def:Application_Program}s.

  Processes that the user starts run under the user-mode or user-level permissions.
  This are significantly reduced permissions compared to the \nameref{def:Kernel}-mode permissions the \nameref{def:Operating_System} has.

  \begin{remark}[Thing Users]\label{rmk:Thing_Users}
    Not all \nameref{def:User}s are required to be people.
    The automated tasks a computer may do to provide a seamless experience for the person may be done by other users in the system.
  \end{remark}
\end{definition}

\subsection{User View}\label{subsec:User_View}
The user's view of the computer varies according to the interface they are using.

In modern times, most people are using computers with a monitor that provides a GUI, a keyboard, mouse, and the physical system itself.
These are designed for one user to use the system at a time, allowing that user to monopolize the system's resources.
The \nameref{def:Operating_System} is designed for \textbf{ease of use} in this case, with relatively little attention paid to performance and resource utilization.

More old-school, but stil in use, a \nameref{def:User} sits at a terminal connected to a mainframe or a minicomputer.
Other users are accessing the same computer through other terminals.
These users share resources and may exchange information.
The operating system in such cases is designed to maximize resource utilization, to assure that all available CPU time, memory, and I/O are used efficiently and that no individual user takes more than their fair share.

In still other cases, \nameref{def:User}s sit at workstations connected to networks of other workstations and servers.
These users have dedicated resources at their disposal, but they also share resources such as networking and servers, including file, compute, and print servers.
Therefore, their operating system is designed to compromise between individual usability and resource utilization.

Lastly, there are \nameref{def:Operating_System}s that are designed to have little to no \nameref{def:User} view.
These are typically embedded systems with very limited input/output.

\subsection{System View}\label{subsec:System_View}
From the computer’s point of view, the \nameref{def:Operating_System} is the program that interacts the most with the hardware.
A computer system has many resources that can be used to solve a problem:
\begin{itemize}[noitemsep]
\item CPU time
\item Memory space
\item File-storage space
\item I/O devices
\item etc.
\end{itemize}

The operating system acts as the manager of these resources.
Facing numerous and possibly conflicting requests for resources, the operating system must decide how to allocate them so that it can operate the computer system efficiently and fairly.
As we have seen, resource allocation is especially important where many \nameref{def:User}s access the same system.

Another, slightly different, view of an operating system emphasizes the need to control the various I/O devices and user programs.
An operating system is a control program.
A control program manages the execution of user programs to prevent errors and improper use of the computer.
It is especially concerned with the operation and control of I/O devices.

\subsection{Computer Organization}\label{subsec:Computer_Organization}
The initial program, run \textbf{\emph{RIGHT}} when the computer starts is typically kept onboard the computer \nameref{def:Hardware}, on ROMs or EEPROMs.

\begin{definition}[Firmware]\label{def:Firmware}
  \emph{Firmware} is software that is written for a specific piece of hardware in mind.
  Its characteristics fall somewhere between those of \nameref{def:Hardware} and those of software.
  It is almost always stored in the \nameref{def:Hardware}'s onboard storage.
  Typically it is stored in ROM~(Read-Only Memory) or EEPROM~(Electrically Erasable Programmable Read-Only Memory).
  It initializes all aspects of the system, from \nameref{def:CPU} \nameref{def:Register}s to device controllers, to memory contents.

  \begin{remark}[How to Differentiate]\label{rmk:Firmware_Differentiate}
    If you are finding it difficult to tell \nameref{def:Hardware}, \nameref{def:Software}, and \nameref{def:Firmware} apart, answer this simple question.
    Can you hit it with a hammer and break the thing?
    \begin{description}[noitemsep]
    \item[Yes] Then it is \nameref{def:Hardware}.
    \item[No] Then it is \nameref{def:Software}.
    \item[Yes and No] Then it is \nameref{def:Firmware}.
    \end{description}
  \end{remark}
\end{definition}

A \nameref{def:CPU} will continue its boot process, until it reaches the \texttt{init} phase, where many other system processes or \nameref{def:Daemon}s start.
Once the computer finishes going through all its \texttt{init} phases, it is ready for use, waiting for some event to occur.
These events can be a \nameref{def:Hardware} \nameref{def:Interrupt} or a software \nameref{def:System_Call}.

\begin{definition}[Daemon]\label{def:Daemon}
  In UNIX and UNIX-like \nameref{def:Operating_System}s, a \emph{daemon} is a \nameref{def:System_Program} process that runs in the ``background'', is started, stopped, and handled by the system, rather than the \nameref{def:User}.
  Daemons run constantly, from the time they are started (potentially the computer's boot) to the time they are killed (potentially when the computer shuts down).
  Typical systems are running dozens, possibly hundreds, of daemons constantly.

  Some examples of daemons are:
  \begin{itemize}[noitemsep]
  \item Network daemons to listen for network connections to connect those requests to the correct processes.
  \item Process schedulers that start processes according to a specified schedule
  \item System error monitoring services
  \item Print servers
  \end{itemize}

  \begin{remark}[Other Names]\label{rmk:Daemon_Other_Names}
    On other, non-UNIX systems, \nameref{def:Daemon}s are called other names.
    They can be called \emph{services}, \emph{subsystems}, or anything of that nature.
  \end{remark}
\end{definition}

\begin{definition}[Interrupt]\label{def:Interrupt}
  An \emph{interrupt} is a special event that the \nameref{def:CPU} \textbf{MUST} handle.
  These could be system errors, or just a button on the keyboard was pressed.
  Hardware may trigger an interrupt at any time by sending a signal to the CPU, usually by way of the system bus.

  When a CPU receives an interrupt, it immediately stops what it is doing and transfers execution to some fixed address.
  To ensure that this happens as quickly as possible, a \nameref{def:Interrupt_Vector} is created.
\end{definition}

\begin{definition}[Trap]\label{def:Trap}
  A \emph{trap} or \emph{exception} is a software-generated \nameref{def:Interrupt} caused by:
  \begin{itemize}[noitemsep]
  \item A program execution error (Division-by-zero or Invalid Memory Access).
  \item A specific request from a user program that an operating-system service be performed (Print to screen).
  \end{itemize}
\end{definition}

\begin{definition}[Interrupt Vector]\label{def:Interrupt_Vector}
  The \emph{interrupt vector} is a table/list of addresses that redirect the \nameref{def:CPU} to the location of the instructions for how to handle that particular \nameref{def:Interrupt}.
  Since only a predefined number of interrupts is possible, a table of pointers to interrupt routines is used to provide the necessary speed.
  These locations hold the addresses of the interrupt service routines for the various devices.
  This array, or interrupt vector, of addresses is then indexed by a unique device number, given with the interrupt request, to provide the address of the interrupt service routine for the interrupting device.
  The interrupt routine is called indirectly through the table, with no intermediate routine needed.
  Generally, this is stored in low memory (the first hundred or so locations).
\end{definition}

\subsection{Storage Management}\label{subsec:Storage_Management}
\begin{definition}[File]\label{def:File}
  The \nameref{def:Operating_System} abstracts from the physical properties of its storage devices to define a logical storage unit, the \emph{file}.
  The operating system maps files onto physical media and accesses these files via the storage devices.
\end{definition}

\subsection{System Programs}\label{subsec:System_Programs}
Another aspect of a modern system is its collection of system programs.
\begin{definition}[System Program]\label{def:System_Program}
  \emph{System programs}, also known as \emph{system utilities}, provide a convenient environment for program development and execution.
  Some of them are simply user interfaces to system calls.
  Others are considerably more complex.
  They can be divided into these categories:
  \begin{description}
  \item[File Management] These programs create, delete, copy, rename, print, dump, list, and generally manipulate files and directories.
  \item[Status Information] Some programs simply ask the system for the date, time, amount of available memory or disk space, number of users, or similar status information.
    Others are more complex, providing detailed performance, logging, and debugging information.
    Typically, these programs format and print the output to the terminal or other output devices or files or display it in a window of the GUI.\@
    Some systems also support a registry, which is used to store and retrieve configuration information.
  \item[File Modification] Several text editors may be available to create and modify the content of files stored on disk or other storage devices.
    There may also be special commands to search contents of files or perform transformations of the text.
  \item[Programming-Language Support] Compilers, assemblers, debuggers, and interpreters for common programming languages (such as C, C++, Java, and PERL) are often provided with the operating system or available as a separate download.
  \item[Program Loading and Execution] Once a program is assembled or compiled, it must be loaded into memory to be executed.
    Debugging systems for either higher-level languages or machine language are needed as well.
  \item[Communications] These programs provide the mechanism for creating virtual connections among processes, users, and computer systems.
    They allow users to send messages to one another’s screens, to browse Web pages, to send e-mail messages, to log in remotely, or to transfer files from one machine to another.
  \item[Background Services] All general-purpose systems have methods for launching certain \nameref{def:System_Program} processes at boot time.
    Some of these processes terminate after completing their tasks, while others continue to run until the system is halted.
    These are typically called \nameref{def:Daemon}s, and systems have dozens of them.
    In addition, operating systems that run important activities in user context rather than in kernel context may use \nameref{def:Daemon}s to run these activities.
  \end{description}
\end{definition}

%%% Local Variables:
%%% mode: latex
%%% TeX-master: "../../EDAF35-Operating_Systems-Reference_Sheet"
%%% End:


\subsection{Operating System Design and Implementation}\label{subsec:OS_Design_Implementation}
One important principle is the separation of \nameref{def:Policy} from \nameref{def:Mechanism}.
\begin{definition}[Mechanism]\label{def:Mechanism}
  A \emph{mechanism} determines how to do something.
\end{definition}

\begin{definition}[Policy]\label{def:Policy}
  A \emph{policy} determines \textbf{what} will be done given the \nameref{def:Mechanism} works correctly.
\end{definition}

The separation of \nameref{def:Policy} and \nameref{def:Mechanism} is important for system flexibility.
Policies are likely to change across places or over time.
In the worst case, each change in policy would require a change in the underlying mechanism.
A general mechanism insensitive to changes in policy would be more desirable.
A change in policy would then require redefinition of only certain parameters of the system.
For instance, consider a mechanism for giving priority to certain types of programs over others.
If the mechanism is properly separated from policy, it can be used either to support a policy decision that I/O-intensive programs should have priority over CPU-intensive ones or to support the opposite policy.

The advantages of using a higher-level language, or at least a systems-implementation language, for implementing operating systems are the same as those gained when the language is used for application programs:
\begin{itemize}[noitemsep]
\item The code can be written faster
\item Is more compact
\item Is easier to understand and debug
\end{itemize}

In addition, improvements in compiler technology will improve the generated code for the entire operating system by simple recompilation.
Finally, an \nameref{def:Operating_System} is far easier to port—to move to some other hardware —
if it is written in a higher-level language.

\begin{definition}[Port]\label{def:Software_Port}
  A \emph{port} is the process of moving a piece of software that was written for one piece of \nameref{def:Hardware} to another.
  In some cases, this only requires a recompilation of the higher-level software.
  In others, it may require completely rewriting the program.

  \begin{remark}[Port Confusion]\label{rmk:Software_Port_Confusion}
    It is important to note that the \nameref{def:Software_Port} is \textbf{\emph{NOT}} the same thing as a \nameref{def:Network_Port}.
  \end{remark}
\end{definition}

%%% Local Variables:
%%% mode: latex
%%% TeX-master: "../../EDAF35-Operating_Systems-Reference_Sheet"
%%% End:


\subsection{Operating System Structure}\label{subsec:OS_Structure}
A system as large and complex as a modern operating system must be engineered carefully if it is to function properly and be modified easily.

\subsubsection{Monolithic Approach}\label{subsubsec:Monolithic_Approach}
Many operating systems do not have well-defined structures.
Frequently, such systems started as small, simple, and limited systems and then grew beyond their original scope.
MS-DOS is an example of such a system.
It was originally designed and implemented by a few people who had no idea that it would become so popular.
It was written to provide the most functionality in the least space, so it was not carefully divided into modules.

\begin{definition}[Monolithic Kernel]\label{def:Monolithic_Kernel}
  A \emph{monolithic kernel} is an \nameref{def:Operating_System} architecture where the entire operating system is working in \nameref{def:Kernel} space, and typically uses only its own memory space to run.
  The monolithic model differs from other operating system architectures (such as the \nameref{def:Microkernel}) in that it alone defines a high-level virtual interface over computer hardware.
  A set of \nameref{def:System_Call}s implement all \nameref{def:Operating_System} services such as process management, concurrency, and memory management.

  Device drivers can be added to the \nameref{def:Kernel} as \nameref{def:Kernel_Module}s.
\end{definition}

In MS-DOS, the interfaces and levels of functionality are not well separated.
For instance, application programs are able to access the basic I/O routines to write directly to the display and disk drives.
Such freedom leaves MS-DOS vulnerable to errant (or malicious) programs, causing entire system crashes when user programs fail.

However, this was partly because MS-DOS was also limited by the hardware of its era.
Because the Intel~8088 for which it was written provides no dual mode and no hardware protection, the designers of MS-DOS had no choice but to leave the base hardware accessible.

\subsubsection{Layered Approach}\label{subsubsec:Layered_Approach}
With proper hardware support, \nameref{def:Operating_System}s can be broken into pieces that are smaller and more appropriate than those allowed by the original MS-DOS and UNIX systems.
The \nameref{def:Operating_System} can then retain much greater control over the computer and over the applications that make use of that computer.
Implementers have more freedom in changing the inner workings of the system and in creating modular \nameref{def:Operating_System}s.
Under a top-down approach, the overall functionality and features are determined and are separated into components.
Information hiding is also important, because it leaves programmers free to implement the low-level routines as they see fit, provided that the external interface of the routine stays unchanged and that the routine itself performs the advertised task.

A system can be made modular in many ways.
One method is the layered approach, in which the \nameref{def:Operating_System} is broken into a number of layers.
The bottom layer (layer 0) is the hardware; the highest (layer N) is the user interface.

A typical operating-system layer, layer $M$ consists of data structures and a set of routines that can be invoked by higher-level layers.
Layer $M$, in turn, can \textbf{\emph{ONLY}} invoke operations on lower-level layers and itself.

The main advantage of the layered approach is simplicity of construction and debugging.
The layers are selected so that each uses functions and services of only lower-level layers.
This approach simplifies debugging and system verification.
The first layer can be debugged without any concern for the rest of the system.
Once the first layer is debugged, its correct functioning can be assumed while the second layer is debugged, and so on.
If an error is found during the debugging of a particular layer, the error must be on that layer, because the layers below it are already debugged.
Thus, the design and implementation of the system are simplified.
Each layer is implemented only with operations provided by lower-level layers.
A layer does not need to know how these operations are implemented; it needs to know only what these operations do.

The major difficulty with the layered approach involves appropriately defining the various layers.
Because a layer can use only lower-level layers, careful planning is necessary.
Even with planning, there can be circular dependencies created between layers.
For example, the backing-store driver would normally be above the CPU scheduler, because the driver may need to wait for I/O and the CPU can be rescheduled during this time.
However, the CPU scheduler may have more information about all the active processes than can fit in memory.
Therefore, this information may need to be swapped in and out of memory, requiring the backing-store driver routine to be below the CPU scheduler.

A final problem with layered implementations is that they tend to be less efficient than other types.

\subsubsection{Microkernels}\label{subsubsec:Microkernels}
This method structures the \nameref{def:Operating_System} by removing all nonessential components from the \nameref{def:Kernel} and implementing them as system and user-level programs, resulting in a smaller \nameref{def:Kernel}.
There is little consensus regarding which services should remain in the kernel and which should be implemented in user space.
Typically, however, microkernels provide minimal process and memory management, in addition to a communication facility.

\begin{definition}[Microkernel]\label{def:Microkernel}
  A \emph{microkernel} (often abbreviated as $\mu$-kernel) is the near-minimum amount of software that can provide the mechanisms needed to implement an \nameref{def:Operating_System}.
  These mechanisms include:
  \begin{itemize}[noitemsep]
  \item Low-level address space management
  \item Thread management
  \item Inter-Process Communication (IPC)
  \end{itemize}

  If the hardware provides multiple rings or CPU modes, the microkernel may be the only software executing at the most privileged level, which is generally referred to as supervisor or kernel mode.
  Traditional \nameref{def:Operating_System} functions, such as device drivers, protocol stacks and file systems, are typically removed from the microkernel itself and are instead run in user space.

  In terms of the source code size, microkernels are often smaller than monolithic kernels.
\end{definition}

The main function of the \nameref{def:Microkernel} is to provide communication between the client program and the various services that are also running in user space.
Communication is provided through \nameref{par:Message_Passing}.

\subsubsection{Kernel Modules}\label{subsubsec:Kernel_Modules}
\subsubsection{Hybrid Systems}\label{subsubsec:Hybrid_Systems}

%%% Local Variables:
%%% mode: latex
%%% TeX-master: "../../EDAF35-Operating_Systems-Reference_Sheet"
%%% End:


\subsection{Operating System Debugging}\label{subsec:OS_Debugging}
\subsubsection{Failure Analysis}\label{subsubsec:Failure_Analysis}
If a process fails, most \nameref{def:Operating_System}s write the error information to a log file to alert \nameref{def:User}s that the problem occurred.
The operating system can also take a \nameref{def:Core_Dump}—— and store it in a file for later analysis.

\begin{definition}[Core Dump]\label{def:Core_Dump}
  A \emph{core dump} captures the memory of the process right as it fails and writes it to a disk.

  \begin{remark}[Why Core?]\label{rmk:Why_Core_Dump}
    The reason a \nameref{def:Core_Dump} is named the way it is is because memory was referred to as the ``core'' in the early days of computing.
  \end{remark}
\end{definition}

Running programs and \nameref{def:Core_Dump}s can be probed by a debugger, which allows a programmer to explore the code and memory of a process.
Operating-system kernel debugging is more complex than usual because of:
\begin{itemize}[noitemsep]
\item The size of the \nameref{def:Kernel}
\item The complexity of the \nameref{def:Kernel}
\item The \nameref{def:Kernel}'s control of the hardware
\item The lack of user-level debugging tools.
\end{itemize}


\begin{definition}[Crash]\label{def:Crash}
  A failure in the \nameref{def:Kernel} is called a \emph{crash}.
\end{definition}

When a \nameref{def:Crash} occurs, error information is saved to a log file, and the memory state is saved to a \nameref{def:Crash_Dump}.

\begin{definition}[Crash Dump]\label{def:Crash_Dump}
  When a \nameref{def:Crash} occurs in the \nameref{def:Kernel}, a \emph{crash dump} is generated.
  This is like a \nameref{def:Core_Dump}, in that the entire contents of that process's \nameref{def:Memory} is written to disk, except the \nameref{def:Crash}ed \nameref{def:Kernel} process is written, instead of a \nameref{def:User} program.
\end{definition}

Operating-system debugging and process debugging frequently use different tools and techniques due to the very different nature of these two tasks.

\subsubsection{Performance Tuning}\label{subsec:Performance_Tuning}
Performance tuning seeks to improve performance by removing processing bottlenecks.
To identify bottlenecks, we must be able to monitor system performance.
Thus, the \nameref{def:Operating_System} must have some means of computing and displaying measures of system behavior.
In a number of systems, the operating system does this by producing trace listings of system behavior.
All interesting events are logged with their time and important parameters and are written to a file.
Later, an analysis program can process the log file to determine system performance and to identify bottlenecks and inefficiencies.
Traces also can help people to find errors in operating-system behavior.

Another approach to performance tuning uses single-purpose, interactive tools that allow users and administrators to question the state of various system components to look for bottlenecks.
One such tool employs the UNIX command \texttt{top} to display the resources used on the system, as well as a sorted list of the ``top'' resource-using processes.

%%% Local Variables:
%%% mode: latex
%%% TeX-master: "../../EDAF35-Operating_Systems-Reference_Sheet"
%%% End:


\subsection{System Boot}\label{subsec:System_Boot}
The procedure of starting a computer by loading the \nameref{def:Kernel} is known as booting the system.
On most computer systems, a small piece of code known as the \nameref{def:Bootloader} is the first thing that runs.

\begin{definition}[Bootloader]\label{def:Bootloader}
  The \emph{bootloader} (or bootstrap loader) is a bootstrap program that:
  \begin{enumerate}[noitemsep]
  \item Locates the kernel
  \item Loads it into main memory
  \item Starts its execution
  \end{enumerate}
\end{definition}

Some computer systems, such as PCs, use a two-step process in which a simple \nameref{def:Bootloader} fetches a more complex boot program from disk, which in turn loads the \nameref{def:Kernel}.

When a CPU receives a reset event, the instruction register is loaded with a predefined memory location, and execution starts there.
At that location is the initial \nameref{def:Bootloader} program.
This program is in the form of read-only memory (ROM), because the RAM is in an unknown state at system startup.
ROM is convenient because it needs no initialization and cannot easily be infected by a computer virus.

\begin{remark*}
  A reset event on the CPU can be the computer having just booted, or it has been restarted, or the reset switched was flipped.
\end{remark*}

The \nameref{def:Bootloader} can perform a variety of tasks.
Usually, one task is to run diagnostics to determine the state of the machine.
If the diagnostics pass, the program can continue with the booting steps.
It also initializes all aspects of the system, from CPU registers to device controllers and the contents of main memory.
Sooner or later, it starts the \nameref{def:Operating_System}.
Cellular phones, tablets, and game consoles store the entire operating system in ROM.\@
Storing the operating system in ROM is suitable only for:
\begin{itemize}[noitemsep]
\item Small operating systems
\item Simple supporting hardware
\item Ensuring rugged operation
\end{itemize}

A problem with this approach is that changing the bootstrap code requires changing the ROM hardware chips.

All forms of ROM are also known as \nameref{def:Firmware}.
A problem with firmware in general is that executing code there is slower than executing code in RAM.\@
Some systems store the \nameref{def:Operating_System} in firmware and copy it to RAM for fast execution.

A final issue with \nameref{def:Firmware} is that it is relatively expensive, so usually only small amounts are available.
For large operating systems, or for systems that change frequently, the \nameref{def:Bootloader} is stored in \nameref{def:Firmware}, and the \nameref{def:Operating_System} is on disk.
In this case, the bootstrap runs diagnostics and has a bit of code that can read a single block at a fixed location (say block zero) from disk into memory and execute the code from that boot block.
The program stored in the boot block may be sophisticated enough to load the entire operating system into memory and begin its execution.

More typically, it is simple code (as it fits in a single disk block) and knows only the address on disk and length of the remainder of the bootstrap program.
GRUB is an example of an open-source \nameref{def:Bootloader} program for Linux systems.
All of the disk-bound bootstrap, and the \nameref{def:Operating_System} that is loaded, can be easily changed by writing new versions to disk.
A disk that has a boot partition is called a boot disk or system disk.
Now that the full bootstrap program has been loaded, it can traverse the file system to find the \nameref{def:Operating_System}'s \nameref{def:Kernel}, load it into \nameref{def:Memory}, and start its execution.
It is only at this point that the system is said to be running.

%%% Local Variables:
%%% mode: latex
%%% TeX-master: "../../EDAF35-Operating_Systems-Reference_Sheet"
%%% End:


%%% Local Variables:
%%% mode: latex
%%% TeX-master: "../EDAF35-Operating_Systems-Reference_Sheet"
%%% End:
