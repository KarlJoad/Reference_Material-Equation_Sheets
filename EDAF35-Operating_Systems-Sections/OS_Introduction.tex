\section{Operating System Introduction}\label{sec:OS_Intro}
A computer system can be roughly divided into 4 parts.
\begin{itemize}[noitemsep]
\item The \nameref{def:Hardware}
\item The \nameref{def:Operating_System}
\item The \nameref{def:Application_Program}s
\item The Users
\end{itemize}

\begin{definition}[Hardware]\label{def:Hardware}
  \emph{Hardware} is the physical components of the system and provide the basic computing resources for the system..
  Hardware includes the \nameref{def:CPU}, \nameref{def:Memory}, and all I/O devices (monitor, keyboard, mouse, etc.).

  \begin{remark}[How to Differentiate]\label{rmk:Hardware_Differentiate}
    If you are finding it difficult to tell \nameref{def:Hardware}, \nameref{def:Software}, and \nameref{def:Firmware} apart, answer this simple question.
    Can you hit it with a hammer and break the thing?
    \begin{description}[noitemsep]
    \item[Yes] Then it is \nameref{def:Hardware}.
    \item[No] Then it is \nameref{def:Software}.
    \item[Yes and No] Then it is \nameref{def:Firmware}.
    \end{description}
  \end{remark}
\end{definition}

\begin{definition}[Software]\label{def:Software}
  \emph{Software} is the code that is used to build the system and make it perform operations.
  Technically, it is the electrical signals that represent \texttt{0} or \texttt{1} and makes the \nameref{def:Hardware} act in a specific, desired fashion to produce some result.

  On a higher level, this can be though of as computer code.

  \begin{remark}[How to Differentiate]\label{rmk:Software_Differentiate}
    If you are finding it difficult to tell \nameref{def:Hardware}, \nameref{def:Software}, and \nameref{def:Firmware} apart, answer this simple question.
    Can you hit it with a hammer and break the thing?
    \begin{description}[noitemsep]
    \item[Yes] Then it is \nameref{def:Hardware}.
    \item[No] Then it is \nameref{def:Software}.
    \item[Yes and No] Then it is \nameref{def:Firmware}.
    \end{description}
  \end{remark}
\end{definition}

\begin{restatable}[Operating System]{definition}{defOperatingSystem}\label{def:Operating_System}
  An \emph{operating system} is a large piece of software that controls the \nameref{def:Hardware} and coordinates the many \nameref{def:Application_Program}s various numbers of \nameref{def:User}s may use.
  It provides the means for proper use of these resources to allow the computer to run.

  By itself, an operating system does nothing useful.
  It simply provides an \textbf{environment} within which other programs can perform useful work.

  The fundamental goal of computer systems is to execute user programs and to make solving user problems easier.
  These programs require certain common operations, such as those controlling the I/O devices.

  In addition, there is no universally accepted definition of what is part of the operating system.
  A simple definition is that it includes everything a vendor ships when you order ``the operating system.''
  The features included, however, vary greatly across systems.
  Some systems take up less than a megabyte of space and lack even a full-screen editor, whereas others require gigabytes of space and are based entirely on graphical windowing systems.
  A more common definition, and the one that we usually follow, is that the operating system is the one program running at all times on the computer—usually called the \nameref{def:Kernel}.

  \begin{remark}[Kernel-Level Non-Kernal Programs]\label{rmk:Kernel_Level_Non_Kernel_Programs}
    Along with the \nameref{def:Kernel}, there are two other types of programs:
    \begin{enumerate}[noitemsep]
    \item System Programs,
      \begin{itemize}[noitemsep]
      \item Associated with the \nameref{def:Operating_System} but are not necessarily part of the \nameref{def:Kernel}.
      \end{itemize}
    \item Application Programs
      \begin{itemize}[noitemsep]
      \item Includes all programs not associated with the operation of the system
      \end{itemize}
    \end{enumerate}
  \end{remark}
\end{restatable}

\begin{definition}[Kernel]\label{def:Kernel}
  The kernel is a computer program at the core of a computer's operating system with complete control over everything in the system.
  It is the ``portion of the operating system code that is always resident in memory''.
  It facilitates interactions between hardware and software components.
  On most systems, it is one of the first programs loaded on startup (after the bootloader).
  It handles input/output requests from software, translating them into data-processing instructions for the central processing unit.
  It handles memory and its mapping, peripherals like: keyboards, monitors, printers, and speakers.
  A kernel connects the application software to the hardware of a computer.

  The critical code of the kernel is usually loaded into a separate area of memory, which is protected from access by application programs or other, less critical parts of the operating system.
  The kernel performs its tasks, such as running processes, managing hardware devices such as the hard disk, and handling interrupts, in this protected kernel space.
\end{definition}

\begin{definition}[Application Program]\label{def:Application_Program}
  An \emph{application program} is a tool used by a \nameref{def:User} to solve some problem.
  This is the main thing a normal person will interact with.
  These pieces of software can include:
  \begin{itemize}[noitemsep]
  \item Text editors
  \item Compilers
  \item Web browsers
  \item Word Processors
  \item Spreadsheets
  \item etc.
  \end{itemize}
\end{definition}

\begin{definition}[User]\label{def:User}
  A \emph{user} is the person and/or thing that is running some \nameref{def:Application_Program}s.

  \begin{remark}[Thing Users]\label{rmk:Thing_Users}
    Not all \nameref{def:User}s are required to be people.
    The automated tasks a computer may do to provide a seamless experience for the person may be done by other users in the system.
  \end{remark}
\end{definition}

\subsection{User View}\label{subsec:User_View}
The user's view of the computer varies according to the interface they are using.

In modern times, most people are using computers with a monitor that provides a GUI, a keyboard, mouse, and the physical system itself.
These are designed for one user to use the system at a time, allowing that user to monopolize the system's resources.
The \nameref{def:Operating_System} is designed for \textbf{ease of use} in this case, with relatively little attention paid to performance and resource utilization.

More old-school, but stil in use, a \nameref{def:User} sits at a terminal connected to a mainframe or a minicomputer.
Other users are accessing the same computer through other terminals.
These users share resources and may exchange information.
The operating system in such cases is designed to maximize resource utilization, to assure that all available CPU time, memory, and I/O are used efficiently and that no individual user takes more than their fair share.

In still other cases, \nameref{def:User}s sit at workstations connected to networks of other workstations and servers.
These users have dedicated resources at their disposal, but they also share resources such as networking and servers, including file, compute, and print servers.
Therefore, their operating system is designed to compromise between individual usability and resource utilization.

Lastly, there are \nameref{def:Operating_System}s that are designed to have little to no \nameref{def:User} view.
These are typically embedded systems with very limited input/output.

\subsection{System View}\label{subsec:System_View}
From the computer’s point of view, the \nameref{def:Operating_System} is the program that interacts the most with the hardware.
A computer system has many resources that can be used to solve a problem:
\begin{itemize}[noitemsep]
\item CPU time
\item Memory space
\item File-storage space
\item I/O devices
\item etc.
\end{itemize}

The operating system acts as the manager of these resources.
Facing numerous and possibly conflicting requests for resources, the operating system must decide how to allocate them so that it can operate the computer system efficiently and fairly.
As we have seen, resource allocation is especially important where many \nameref{def:User}s access the same system.

Another, slightly different, view of an operating system emphasizes the need to control the various I/O devices and user programs.
An operating system is a control program.
A control program manages the execution of user programs to prevent errors and improper use of the computer.
It is especially concerned with the operation and control of I/O devices.

\subsection{Computer Organization}\label{subsec:Computer_Organization}
The initial program, run \textbf{\emph{RIGHT}} when the computer starts is typically kept onboard the computer \nameref{def:Hardware}, on ROMs or EEPROMs.

\begin{definition}[Firmware]\label{def:Firmware}
  \emph{Firmware} is software that is written for a specific piece of hardware in mind.
  Its characteristics fall somewhere between those of \nameref{def:Hardware} and those of software.
  It is almost always stored in the \nameref{def:Hardware}'s onboard storage.
  Typically it is stored in ROM~(Read-Only Memory) or EEPROM~(Electrically Erasable Programmable Read-Only Memory).
  It initializes all aspects of the system, from \nameref{def:CPU} \nameref{def:Register}s to device controllers, to memory contents.

  \begin{remark}[How to Differentiate]\label{rmk:Firmware_Differentiate}
    If you are finding it difficult to tell \nameref{def:Hardware}, \nameref{def:Software}, and \nameref{def:Firmware} apart, answer this simple question.
    Can you hit it with a hammer and break the thing?
    \begin{description}[noitemsep]
    \item[Yes] Then it is \nameref{def:Hardware}.
    \item[No] Then it is \nameref{def:Software}.
    \item[Yes and No] Then it is \nameref{def:Firmware}.
    \end{description}
  \end{remark}
\end{definition}

A \nameref{def:CPU} will continue its boot process, until it reaches the \texttt{init} phase, where many other system processes or \nameref{def:Daemon}s start.
Once the computer finishes going through all its \texttt{init} phases, it is ready for use, waiting for some event to occur.
These events can be a \nameref{def:Hardware} \nameref{def:Interrupt} or a software \nameref{def:System_Call}.

\begin{definition}[Daemon]\label{def:Daemon}
  In UNIX and UNIX-like \nameref{def:Operating_System}s, a \emph{daemon} is a \nameref{def:System_Program} process that runs in the ``background'', is started, stopped, and handled by the system, rather than the \nameref{def:User}.
  Daemons run constantly, from the time they are started (potentially the computer's boot) to the time they are killed (potentially when the computer shuts down).
  Typical systems are running dozens, possibly hundreds, of daemons constantly.

  Some examples of daemons are:
  \begin{itemize}[noitemsep]
  \item Network daemons to listen for network connections to connect those requests to the correct processes.
  \item Process schedulers that start processes according to a specified schedule
  \item System error monitoring services
  \item Print servers
  \end{itemize}

  \begin{remark}[Other Names]\label{rmk:Daemon_Other_Names}
    On other, non-UNIX systems, \nameref{def:Daemon}s are called other names.
    They can be called \emph{services}, \emph{subsystems}, or anything of that nature.
  \end{remark}
\end{definition}

\begin{definition}[Interrupt]\label{def:Interrupt}
  An \emph{interrupt} is a special event that the \nameref{def:CPU} \textbf{MUST} handle.
  These could be system errors, or just a button on the keyboard was pressed.
  Hardware may trigger an interrupt at any time by sending a signal to the CPU, usually by way of the system bus.

  When a CPU receives an interrupt, it immediately stops what it is doing and transfers execution to some fixed address.
  To ensure that this happens as quickly as possible, a \nameref{def:Interrupt_Vector} is created.
\end{definition}

\begin{definition}[Trap]\label{def:Trap}
  A \emph{trap} or \emph{exception} is a software-generated \nameref{def:Interrupt} caused by:
  \begin{itemize}[noitemsep]
  \item A program execution error (Division-by-zero or Invalid Memory Access).
  \item A specific request from a user program that an operating-system service be performed (Print to screen).
  \end{itemize}
\end{definition}

\begin{definition}[Interrupt Vector]\label{def:Interrupt_Vector}
  The \emph{interrupt vector} is a table/list of addresses that redirect the \nameref{def:CPU} to the location of the instructions for how to handle that particular \nameref{def:Interrupt}.
  Since only a predefined number of interrupts is possible, a table of pointers to interrupt routines is used to provide the necessary speed.
  These locations hold the addresses of the interrupt service routines for the various devices.
  This array, or interrupt vector, of addresses is then indexed by a unique device number, given with the interrupt request, to provide the address of the interrupt service routine for the interrupting device.
  The interrupt routine is called indirectly through the table, with no intermediate routine needed.
  Generally, this is stored in low memory (the first hundred or so locations).
\end{definition}

\subsection{Storage Management}\label{subsec:Storage_Management}
\begin{definition}[File]\label{def:File}
  The \nameref{def:Operating_System} abstracts from the physical properties of its storage devices to define a logical storage unit, the \emph{file}.
  The operating system maps files onto physical media and accesses these files via the storage devices.
\end{definition}

\subsection{System Calls}\label{subsec:System_Calls}
\begin{definition}[System Call]\label{def:System_Call}
  Software may trigger an interrupt by executing a special operation called a \emph{system call}.
  This can also be called a monitor call.

  System calls provide an interface to the services made available by an \nameref{def:Operating_System}.
  These calls are generally available as routines written in C and C++.
  Some of the lowest-level tasks (for example, tasks where hardware must be accessed directly) may have to be written using assembly instructions.

  There are roughly 6 different types of system calls:
  \begin{enumerate}[noitemsep]
  \item \nameref{subsubsec:Process_Control}
    \begin{itemize}[noitemsep]
    \item End, Abort
    \item Load, Execute
    \item Create process, Terminate process
    \item Get process attributes, Set process attributes
    \item Wait for time
    \item Wait event, Signal event
    \item Allocate and Free memory
    \end{itemize}
  \item \nameref{subsubsec:File_Manipulation}
    \begin{itemize}[noitemsep]
    \item Create file, Delete file
    \item Open, Close
    \item Read, Write, Reposition
    \item Get file attributes, Set file attributes
    \end{itemize}
  \item \nameref{subsubsec:Device_Manipulation}
    \begin{itemize}[noitemsep]
    \item Request device, Release device
    \item Read, Write, Reposition
    \item Get device attributes, Set device attributes
    \item Logically attach or detach devices
    \end{itemize}
  \item \nameref{subsubsec:Information_Maintenance}
    \begin{itemize}[noitemsep]
    \item Get time or date, Set time or date
    \item Get system data, Set system data
    \item Get Process, File, or Device attributes
    \item Set Process, File, or Device attributes
    \end{itemize}
  \item \nameref{subsubsec:Communications}
    \begin{itemize}[noitemsep]
    \item Create, Delete communication connection
    \item Send, Receive messages
    \item Transfer status information
    \item Attach or Detach remote devices
    \end{itemize}
  \item \nameref{subsubsec:Protection}
  \end{enumerate}
\end{definition}

\begin{table}[h!tbp]
  \centering
  \begin{tabular}{lll}
    \toprule
    & \textbf{Windows} & \textbf{Unix} \\
    \midrule
    \nameref{subsubsec:Process_Control} & \mintinline{cpp}{CreateProcess()} & \mintinline{c}{fork()} \\
    & \mintinline{cpp}{ExitProcess()} & \mintinline{c}{exit()} \\
    & \mintinline{cpp}{WaitForSingleObject()} & \mintinline{c}{wait()} \\
    \midrule
    \nameref{subsubsec:File_Manipulation} & \mintinline{cpp}{CreateFile()} & \mintinline{c}{open()} \\
    & \mintinline{cpp}{ReadFile()} & \mintinline{c}{read()} \\
    & \mintinline{cpp}{WriteFile()} & \mintinline{c}{write()} \\
    & \mintinline{cpp}{CloseHandle()} & \mintinline{c}{close()} \\
    \midrule
    \nameref{subsubsec:Device_Manipulation} & \mintinline{cpp}{SetConsoleMode()} & \mintinline{c}{ioctl()} \\
    & \mintinline{cpp}{ReadConsole()} & \mintinline{c}{read()} \\
    & \mintinline{cpp}{WriteConsole()} & \mintinline{c}{write()} \\
    \midrule
    \nameref{subsubsec:Information_Maintenance} & \mintinline{cpp}{GetCurrentProcessID()} & \mintinline{c}{getpid()} \\
    & \mintinline{cpp}{SetTimer()} & \mintinline{c}{alarm()} \\
    & \mintinline{cpp}{Sleep()} & \mintinline{c}{sleep()} \\
    \midrule
    \nameref{subsubsec:Communications} & \mintinline{cpp}{CreatePipe()} & \mintinline{c}{pipe()} \\
    & \mintinline{cpp}{CreateFileMapping()} & \mintinline{c}{shm_open()} \\
    & \mintinline{cpp}{MapViewOfFile()} & \mintinline{c}{mmap()} \\
    \midrule
    \nameref{subsubsec:Protection} & \mintinline{cpp}{SetFileSecurity()} & \mintinline{c}{chmod()} \\
    & \mintinline{cpp}{InitializeSecurityDescriptor()} & \mintinline{c}{umask()} \\
    & \mintinline{cpp}{SetSecurityDescriptorGroup()} & \mintinline{c}{chown()} \\
    \bottomrule
  \end{tabular}
  \caption{System Calls in Unix and Windows}
  \label{tab:System_Calls_Examples}
\end{table}

\nameref{def:System_Call}s are exposed to the programmer by an \nameref{def:API}.
\begin{definition}[Application Programming Interface]\label{def:API}
  An \emph{Application Programming Interface} (\emph{API}) specifies a set of functions that are available to an application programmer.
  They specify the parameters that are passed to each function and the return values the programmer can expect.

  Typically, API calls perform \nameref{def:System_Call}s in the background, without the programmer knowing about them.
\end{definition}

\subsubsection{Process Control}\label{subsubsec:Process_Control}
A running program needs to be able to halt its own execution, either normally or abnormally.
If a system call is made to terminate the currently running program abnormally, or if the program runs into a problem and causes an error \nameref{def:Trap}, a dump of memory is sometimes taken and an error message generated.
The dump is written to disk and may be examined by a debugger—a system program designed to aid the programmer in finding and correcting errors, or bugs—to determine the cause of the problem.

Under either normal or abnormal circumstances, the operating system must transfer control to the invoking command interpreter.
The command interpreter then reads the next command.

To determine how bad the execution halt was, when the program ceases execution, it will return an exit code.
By convention, and for no other reason, an exit code of \texttt{0} is considered to be the program completed execution successfully.
Otherwise, the greater the return value, the greater the severity of the error.

\subsubsection{File Manipulation}\label{subsubsec:File_Manipulation}
We first need to be able to \mintinline{c}{create()} and \mintinline{c}{delete()} files.
Either system call requires the name of the file and perhaps some of the file’s attributes.
Once the file is created, we need to \mintinline{c}{open()} it and to use it.
We may then \mintinline{c}{read()}, \mintinline{c}{write()}, or perform any other \nameref{def:API}-defined action(s).
Finally, we need to \mintinline{c}{close()} the file, indicating that we are no longer using it.

We may need these same sets of operations for directories if we have a directory structure for organizing files in the file system.
In addition, for either files or directories, we need to be able to determine the values of various attributes and perhaps to reset them if necessary.

\begin{definition}[File Attribute]\label{def:File_Attribute}
  A \emph{file attribute} contains metadata about the file.
  This includes the file's name, type, protection codes, accounting information, and so on.
\end{definition}

\begin{remark*}
  If the system programs are callable by other programs, then each can be considered an \nameref{def:API} by other system programs.
\end{remark*}

\subsubsection{Device Manipulation}\label{subsubsec:Device_Manipulation}
\begin{definition}[Device]\label{def:Device}
  A \emph{device} in an \nameref{def:Operating_System} is a resource that must be controlled.
  Some of these devices are physical devices (for example, disk drives), while others can be thought of as abstract or virtual devices (for example, files).
\end{definition}

A system with multiple users may require us to first \mintinline{c}{request()} a device, to ensure exclusive use of it.
After we are finished with the device, we \mintinline{c}{release()} it.
These functions are similar to the \mintinline{c}{open()} and \mintinline{c}{close()} system calls for files.
Other operating systems allow unmanaged access to devices.
The hazard then is the potential for device contention and perhaps \nameref{def:Deadlock}.

Once the device has been requested (and allocated to us), we can \mintinline{c}{read()}, \mintinline{c}{write()}, just as we can with files.
In fact, the similarity between I/O devices and files is so great that many operating systems, including UNIX, merge the two into a combined file–device structure.
In this case, a set of system calls can be shared between both files and \nameref{def:Device}s.
Sometimes, I/O devices are identified by special file names, directory placement, or file attributes.

\subsubsection{Information Maintenance}\label{subsubsec:Information_Maintenance}
Many system calls exist simply for the purpose of transferring information between the \nameref{def:User} program and the \nameref{def:Operating_System}.
For example, most systems have a system call to return the current \mintinline{c}{time()} and \mintinline{c}{date()}.
Other system calls may return information about the system, such as the number of current users, the version number of the operating system, the amount of free memory or disk space, and so on.

Another set of system calls is helpful in debugging a program.
Many systems provide system calls to \mintinline{c}{dump()} memory.
A program \texttt{trace} lists each system call as it is executed.
In addition, the \nameref{def:Operating_System} keeps information about all its processes, and \nameref{def:System_Call}s are used to access this information.

\subsubsection{Communications}\label{subsubsec:Communications}
\subsubsection{Protection}\label{subsubsec:Protection}


%%% Local Variables:
%%% mode: latex
%%% TeX-master: "../../EDAF35-Operating_Systems-Reference_Sheet"
%%% End:


%%% Local Variables:
%%% mode: latex
%%% TeX-master: "../EDAF35-Operating_Systems-Reference_Sheet"
%%% End:
