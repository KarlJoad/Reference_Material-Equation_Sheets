\section{Virtual Memory}\label{sec:Virtual_Memory}
\begin{definition}[Virtual Memory]\label{def:Virtual_Memory}
  Virtual memory involves the separation of logical memory as perceived
  by users from physical memory.
  This separation allows an extremely large virtual memory to be provided for programmers when only a smaller physical memory is available.
  Virtual memory makes the task of programming much easier, because the programmer no longer needs to worry about the amount of physical memory available; she can concentrate instead on the problem to be programmed.
\end{definition}

\begin{definition}[Physical Memory]\label{def:Physical_Memory}
  \emph{Physical memory} is the memory that is physically installed in the computer.
  There is a finite amount of this, determined by how much is installed by the system designer.
\end{definition}


%%% Local Variables:
%%% mode: latex
%%% TeX-master: "../EDAF35-Operating_Systems-Reference_Sheet"
%%% End:
