\section{Virtual Memory}\label{sec:Virtual_Memory}
As our natural definition of Von Neumann computers, the instructions being executed must be in \nameref{def:Physical_Memory}.
Thus, the first approach to meeting this requirement is to place the entire \nameref{def:Logical_Address_Space} in physical memory.
This is unfortunate, since it limits the size of a program to the size of physical memory.
\nameref{def:Dynamic_Loading} can help ease this restriction, but it generally requires special precautions and extra work by the programmer.

However, an examination of real \nameref{def:Program}s shows us that, in many cases, the entire program is not needed.
Even in those cases where the entire program is needed, it may not all be needed at the same time.
The ability to execute a program that is only partially in memory has many benefits:
\begin{itemize}[noitemsep]
\item Programs are no longer constrained by amount of \nameref{def:Physical_Memory} available.
  Users able to write programs for an extremely large \nameref{def:Virtual_Address_Space}, simplifying the programming task.
\item Each user program takes less physical memory, more programs could be run at the same time.
  This yields a corresponding increase in CPU utilization and throughput but with no increase in response time or turnaround time.
\item Less I/O needed to load or swap user programs into memory, so each user program would run faster.
\end{itemize}

\begin{definition}[Virtual Memory]\label{def:Virtual_Memory}
  Virtual memory involves the separation of logical memory as perceived by users from \nameref{def:Physical_Memory}.
  This separation allows:
  \begin{itemize}[noitemsep]
  \item Extremely large virtual memory to be provided for programmers when less physical memory is available.
  \item \nameref{def:Process}es can share files easily.
  \item \nameref{par:Shared_Memory} can be implemented.
    \begin{itemize}[noitemsep]
    \item For example, sharing a library between different processes can be handled just by mapping the virtual memory location into the process's \nameref{def:Virtual_Address_Space}.
    \item Although each process considers the library to be part of \textbf{its} virtual address space, the frames where the libraries reside in physical memory are shared by all the processes.
    \item Typically, a library is mapped read-only into the space of each process that is linked with it.
    \item This is typically implemented with \nameref{subsubsec:Shared_Pages}.
    \end{itemize}
  \item An efficient method for \nameref{def:Process} creation.
\end{itemize}

  Virtual memory makes the task of programming much easier, because the programmer no longer needs to worry about the amount of physical memory available; they can concentrate instead on the problem to be programmed.
\end{definition}

\begin{definition}[Physical Memory]\label{def:Physical_Memory}
  \emph{Physical memory} is the memory that is physically installed in the computer.
  There is a finite amount of this, determined by how much is installed by the system designer.
\end{definition}

\begin{definition}[Virtual Address Space]\label{def:Virtual_Address_Space}
  The \emph{virtual address space} of a \nameref{def:Process} consists of all the \nameref{def:Virtual_Address}es generated by a \nameref{def:Program}.
  It refers to the logical (virtual) view of \textbf{how a \nameref{def:Process} is stored in memory}.
  Typically, this view is one of perfectly continguous memory locations, when in fact, the process could be in many different, noncontiguous locations and the \nameref{def:Memory_Management_Unit} handles the \nameref{def:Paging}.

  \begin{remark}
    The \nameref{def:Virtual_Address_Space} is only calculated by one \nameref{def:Program}/\nameref{def:Process} at a time.
    To find the total virtual address space used, all \nameref{def:Process}es must have their virtual address spaces aggregated.
  \end{remark}
\end{definition}

The large blank space between the heap and the stack is part of the virtual address space but will require actual physical pages only if the heap or stack grows.
Virtual address spaces that include these holes are known as \nameref{def:Sparse_Address_Space}s.

\begin{definition}[Sparse Address Space]\label{def:Sparse_Address_Space}
  A \emph{sparse address space} is a \nameref{def:Virtual_Address_Space} that is not completely mapped to \nameref{def:Physical_Memory}.
  Typically, this is done in the case of programs for the memory space between the stack and heap, which is overwhelmingly empty.

  Sparse address spaces are beneficial because these holes can be filled on demand, when the \nameref{def:Process} needs more \nameref{def:Physical_Memory}, or to dynamically link shared objects during program execution.
\end{definition}

\subsection{Demand Paging}\label{subsec:Demand_Paging}
Because we can support the ability to \nameref{def:Dynamic_Loading} \nameref{def:Program}s, we don't need to load the \textbf{entire} program into memory initially.
For example, suppose a program starts with a list of available options from which the user is to select.
Loading the entire program into memory means loading the executable code for \textbf{ALL} options, no matter what option is chosen.

If we chose to load in parts of a \nameref{def:Program} only when they are needed, this is \nameref{def:Demand_Paging}.

\begin{definition}[Demand Paging]\label{def:Demand_Paging}
  \emph{Demand paging} is when individual portions of a \nameref{def:Program} are loaded into memory using \nameref{def:Dynamic_Loading} \textbf{only when they are needed}.

  With demand-paged \nameref{def:Virtual_Memory}, pages are loaded only when they are demanded during program execution.
  Thus, pages that are never accessed are never loaded into physical memory.
\end{definition}

When \nameref{def:Demand_Paging} is combined with \nameref{def:Swapping}, we do not want to swap the whole \nameref{def:Process} into memory if we don't need it.
Thus, we use a \nameref{def:Lazy_Swapper}.

\begin{definition}[Lazy Swapper]\label{def:Lazy_Swapper}
  A \emph{lazy swapper}, like a regular swapper, swaps \nameref{def:Process}es into memory from the \nameref{def:Backing_Store}.
  However, \textbf{it only swaps in pages of the process that will be needed}, and never swaps a page in that will not be used.

  A lazy swapper can be implemented as a \nameref{def:Pager}.
\end{definition}

When a \nameref{def:Process} is to be swapped in, the \nameref{def:Pager} guesses which pages will be used before the process is swapped out again.
Instead of swapping in a whole process, the pager brings only those pages into memory, avoiding reading pages into memory that will not be used anyway, decreasing the swap time and the amount of physical memory needed.

\begin{definition}[Pager]\label{def:Pager}
  In a \nameref{def:Demand_Paging} and \nameref{def:Swapping} system, we are not swapping whole \nameref{def:Process}es into and out of the \nameref{def:Backing_Store}; we are swapping individual pages of the process.
  Thus, the task of moving a process's pages into and out of the backing store is handled by the \emph{pager}, rather than the swapper.
\end{definition}

With a \nameref{def:Pager}, we need some form of hardware support to distinguish between the pages that are in memory and the pages that are on the disk.
The \nameref{par:Page_Valid_Invalid_Bit} scheme can be used for this purpose.
\begin{itemize}[noitemsep]
\item If this bit is set to ``valid'' the associated page is \textbf{both} legal and in memory.
\item If the bit is set to ``invalid'' the page either:
  \begin{itemize}[noitemsep]
  \item Is not valid, i.e.\ not in the \nameref{def:Logical_Address_Space} of the \nameref{def:Process}.
  \item Is valid but is currently on the disk.
  \end{itemize}
\end{itemize}

The page-table entry for a page that is brought into memory is set as usual, but the page-table entry for a page that is not currently in memory is either simply marked invalid or contains the address of the page on disk.
Marking a page invalid will have no effect if the process never attempts to access that page.
If we guess right and page in all pages that are actually needed and only those pages, the \nameref{def:Process} will run exactly as though we had brought in all pages.

While the \nameref{def:Process} executes and accesses pages that are \emph{memory resident}, execution proceeds normally.
Attempting to access a page marked invalid causes a \nameref{def:Page_Fault}.
The \nameref{def:Paging} hardware (\nameref{def:Memory_Management_Unit}), in translating the address through the page table, will notice the invalid bit is set, causing a \nameref{def:Trap} to the \nameref{def:Operating_System}.

\begin{definition}[Page Fault]\label{def:Page_Fault}
  A \emph{page fault} is a hardware \nameref{def:Trap} raised by the \nameref{def:Memory_Management_Unit}.
  It is caused by a \nameref{def:Process} attempting to access a page that has not been brought into memory by the \nameref{def:Pager}.

  The following sequence executes when a page fault occurs:
  \begin{enumerate}[noitemsep]
  \item \nameref{def:Trap} to the operating system.
  \item Save the user registers and process state.
  \item Determine that the interrupt was a page fault.
  \item Check that the page reference was legal and determine the location of the page on the disk.
  \item Issue a read from the disk to a free physical frame:
    \begin{enumerate}[noitemsep]
    \item Wait in a queue for this device until the read request is serviced.
    \item Wait for the device seek and/or latency time.
    \item Begin the transfer of the page to a free frame.
    \end{enumerate}
  \item While waiting, the CPU can be allocated to some other user (CPU scheduling, optional).
  \item Receive an interrupt from the disk I/O subsystem (I/O completed).
  \item Save the registers and process state for the other user (if step 6 is used).
  \item Determine that the interrupt was from the disk.
  \item Correct the \nameref{def:Page_Table} and other tables to show that the desired page is now in memory.
  \item Wait for the CPU to be allocated to this process again.
  \item Restore the user registers, process state, and new page table. Then resume the interrupted instruction.
  \end{enumerate}
\end{definition}

Handling a \nameref{def:Page_Fault} in straightforward.
\begin{enumerate}[noitemsep]
\item We check an internal table (usually kept with the \nameref{def:Process_Control_Block}) for this \nameref{def:Process} to determine whether the reference was a valid or an invalid memory access.
\item If the reference was invalid, we terminate the process (attempt to access memory that is not owned by this \nameref{def:Process}).
  If it was valid and legal, but we have not yet brought in that page, we now page it in.
\item We find a free physical frame (by taking one from the free-frame list, for example).
\item We schedule a disk operation to read the desired page into the newly allocated frame.
\item When the disk read is complete, we modify the page-frame internal table of the \nameref{def:Process} to indicate that the page is now in memory.
\item \textbf{We restart the instruction that was interrupted by the trap.}
  The process can now access the page as though it had always been in memory.
\end{enumerate}

A \nameref{def:Page_Fault} may occur at any memory reference.
A crucial requirement for \nameref{def:Demand_Paging} is the ability to restart any instruction after a page fault.
We save the state (registers, condition code, instruction counter) of the interrupted \nameref{def:Process} when the page fault occurs.
This lets us restart the process in exactly the same place and state, except that the desired page is now in memory and is accessible.

\subsubsection{Problems with Demand Paging}\label{subsubsec:Demand_Paging_Problems}
Theoretically, some programs could access several new pages of memory with each instruction execution (one page for the instruction and many for data), possibly causing multiple page faults per instruction.
This situation would result in unacceptable system performance.
Fortunately, analysis of running processes shows that this behavior is exceedingly unlikely.
Programs tend to have locality of reference giving reasonable performance from demand paging.

The major difficulty arises when one instruction may modify several different locations.
If a block (source or destination) straddles a page boundary, a page fault might occur after the move is partially done.
In addition, if the source and destination blocks overlap, the source block may have been modified, in which case we cannot simply restart the instruction.
This particular problem can be solved in two different ways.
\begin{enumerate}[noitemsep]
\item The microcode computes and attempts to access \textbf{both ends} of \textbf{both blocks}.
  If a \nameref{def:Page_Fault} is going to occur, it will happen at this step, before anything is modified.
  The instruction can then execute, knowing that no page fault can occur, since all relevant pages are in memory.
\item Use temporary registers to hold the values of overwritten locations.
  If there is a \nameref{def:Page_Fault}, all the old values are written back into memory before the \nameref{def:Trap} occurs.
  This restores memory to its state before the instruction was started, so that the instruction can be repeated.
\end{enumerate}

\subsubsection{Hardware Required}\label{subsubsec:Demand_Paging_Required_Hardware}
The hardware to support demand paging is the same as the hardware for \nameref{def:Paging} and \nameref{def:Swapping}:
\begin{enumerate}[noitemsep]
\item Page table.
  This table has the ability to mark an entry invalid through a valid–invalid bit or a special value of protection bits.
\item Secondary memory.
  This memory holds those pages that are not present in main memory.
  The secondary memory is usually a high-speed disk.
  It is known as the \emph{swap device}, and the section of disk used for this purpose is known as swap space.
\end{enumerate}

\subsubsection{Performance}\label{subsubsec:Demand_Paging_Performance}
Demand paging can significantly affect the performance of a computer system.
To illustrate, the \nameref{def:Effective_Access_Time} for demand-paged memory can be computed.

\subsubsection{Swap Usage}\label{subsubsec:Demand_Paging_Swap_Usage}

%%% Local Variables:
%%% mode: latex
%%% TeX-master: "../../EDAF35-Operating_Systems-Reference_Sheet"
%%% End:


\subsection{Copy-on-Write}\label{subsec:Memory_Copy_on_Write}
Recall that the \kernelinline{fork()} \nameref{def:System_Call} creates a child \nameref{def:Process} that is a nearly perfect duplicate of its parent.
Traditionally, this was done by creating a copy of the parent’s address space for the child, thereby \textbf{duplicating} the pages belonging to the parent.
However, considering that many child processes invoke the \kernelinline{exec()} system call immediately after creation, the copying of the parent’s address space may be unnecessary.
\nameref{def:Memory_Copy_on_Write} works by allowing the parent and child processes initially to share the same pages allowing the \kernelinline{fork()} system call to bypass the need for \nameref{def:Demand_Paging} by using a technique similar to page sharing (\Cref{subsec:Paging}).

\begin{definition}[Copy-on-Write]\label{def:Memory_Copy_on_Write}
  \emph{Copy-on-Write} allows the \kernelinline{fork()}ing parent \nameref{def:Process} to share its pages with its newly spawned child process.
  This bypasses the need for \nameref{def:Demand_Paging}, by essentially allowing pages to be shared (like in \nameref{par:Shared_Memory} schemes).
  These shared pages are marked as Copy-on-Write pages; only pages that \textbf{can be} modified need be marked as copy-on-write.
  Pages that cannot be modified (pages containing executable code) can always be shared by the parent and child.

  This allows for rapid process creation and minimizes the number of new pages that must be allocated to the newly created process.
  The pages marked as copy-on-write pages, mean if \textbf{EITHER} process writes to any of the shared pages, a copy of the shared page is created.
  The writer has its \nameref{def:Page_Table} changed to point to this one, which is a complete copy (except for the newly written changes).
  The process that did nothing to the page continues to point to the original page.
\end{definition}

To handle the allocation of pages for a \nameref{def:Memory_Copy_on_Write} page, the location that it is allocated from is important.
Many \nameref{def:Operating_System}s provide a pool of free pages for such requests.
These free pages are typically allocated when the stack or heap for a \nameref{def:Process} must expand or when there are copy-on-write pages to be managed.
Operating systems typically allocate these pages using a technique known as \nameref{def:Zero_Fill_on_Demand}.


%%% Local Variables:
%%% mode: latex
%%% TeX-master: "../../EDAF35-Operating_Systems-Reference_Sheet"
%%% End:


\subsection{Page Replacement}\label{subsec:Page_Replacement}
If a \nameref{def:Process} of ten pages actually uses only half of them, then \nameref{def:Demand_Paging} saves the I/O necessary to load the five pages that are never used.
This also increases the degree of multiprogramming by running more processes, \nameref{def:Over_Allocating} memory.

\begin{definition}[Over-Allocating]\label{def:Over_Allocating}
  \emph{Over-allocating} is the process of putting more load on a component in the system than would normally be possible.

  In the context of \nameref{def:Paging} and \nameref{def:Demand_Paging}, this means that main memory has more \nameref{def:Process}es in it than would be normal in normal \nameref{def:Paging} schemes.
  This is because \textbf{ONLY} the pages that are needed to run these processes are loaded, and nothing more.
\end{definition}

If we run multiple \nameref{def:Process}es, each of which is multiple pages in size but actually only uses some of them, we can achieve higher CPU utilization and throughput, with physical frames to spare.
It is possible that each of these processes, for some reason, may suddenly try to use \textbf{all} of their pages \textbf{simulataneously}, resulting in a need for more physical frames than the system has.
The operating system has several options at this point:
\begin{enumerate}[noitemsep]
\item Terminate a user process.
  \begin{itemize}[noitemsep]
  \item \nameref{def:Demand_Paging} is the operating system’s attempt to improve the system’s utilization and throughput.
  \item Users should not be aware that their processes are running on a paged system—
  \item Paging should be logically transparent to the user.
  \item Not the best choice.
  \end{itemize}

\item Replace one page in memory with the requested one in swap (\nameref{def:Page_Replacement}).
  \begin{itemize}[noitemsep]
  \item \nameref{def:Operating_System} swaps out a process, freeing \textbf{all} its frames.
  \item Reduce the level of multiprogramming.
  \item Good option in certain circumstances (\nameref{def:Thrashing}).
  \end{itemize}
\end{enumerate}

\begin{definition}[Page Replacement]\label{def:Page_Replacement}
  \emph{Page replacement} is the process of replacing a page in memory with one that is \nameref{def:Swapping} in from the swap.

  Page replacement is basic to \nameref{def:Demand_Paging}.
  It completes the separation between logical memory and \nameref{def:Physical_Memory}.
  With this mechanism, an enormous \nameref{def:Virtual_Memory} can be provided for programmers on a smaller physical memory.
  Without demand paging, user addresses are still mapped into physical addresses, and the two sets of addresses can be different, however, all the pages of a process still must be in physical memory.
  With demand paging, the size of the \nameref{def:Logical_Address_Space} is no longer constrained by physical memory.

  The procedure for performing this replacement is listed in \Cref{subsubsec:Page_Replacement_Procedure}.
\end{definition}

Buffers for I/O also consume a considerable amount of memory.
Deciding how much memory to allocate to I/O and how much to program pages is a significant challenge.
Some systems allocate a fixed percentage of memory for I/O buffers, others allow user processes and the I/O subsystem to compete for all system memory.

\subsubsection{Replacement Procedure}\label{subsubsec:Page_Replacement_Procedure}
The basic steps to perform a \nameref{def:Page_Replacement} are shown below:
\begin{enumerate}[noitemsep]
\item Find the location of the desired page on the disk.
\item Find a free frame:
  \begin{enumerate}[noitemsep]
  \item If there is a free frame, use it.
  \item If there is no free frame, use a page-replacement algorithm to select a victim frame.
  \item Write the victim frame to the disk; change the page and frame tables accordingly.
  \end{enumerate}
\item Read the desired page into the newly freed frame; change the page and frame tables.
\item Continue the user process from where the page fault occurred.
\end{enumerate}

If no frames in memory are free, two page transfers (move one page out and one in) are required, effectively doubling the \nameref{def:Page_Fault} service time and increases the \nameref{def:Effective_Access_Time} accordingly.
Reducing this overhead can be done by using a \nameref{def:Modify_Bit} (\nameref{rmk:Dirty_Bit}).

\begin{definition}[Modify Bit]\label{def:Modify_Bit}
  The \emph{modify bit} is a signalling bit associated with each page or frame has a modify bit associated with it in the hardware.
  The modify bit for a page is set by the hardware whenever any byte in the page is written, indicating that the page has been modified.

  \begin{remark}[Dirty Bit]\label{rmk:Dirty_Bit}
    The \nameref{def:Modify_Bit} is sometimes called the \emph{dirty bit}, because if it is set, then the memory can be considered ``dirty'' and must be written before any further changes are made.
  \end{remark}
\end{definition}

When we select a page for replacement, we examine its \nameref{def:Modify_Bit}.
\begin{itemize}[noitemsep]
\item Modify Bit set, we know the page has been modified since the \textbf{last} read-in from the disk.
  \begin{itemize}[noitemsep]
  \item Must write the page to the disk.
  \end{itemize}
\item Modify Bit not set, the page \textbf{has not been modified} since it was read into memory.
  \begin{itemize}[noitemsep]
  \item No need to write the memory page to the disk; an exact replica is already there.
  \end{itemize}
\end{itemize}

\begin{remark*}
  This also applies to read-only pages (for example, pages of binary code).
  Since such pages cannot be modified, they may be discarded when desired.
  This can significantly reduce the time required to service a page fault, since it reduces \textbf{I/O time} by one-half if the page has not been modified.
\end{remark*}

\subsubsection{Algorithms}\label{subsubsec:Page_Replacement_Algorithms}
In general, we select a particular replacement algorithm with the lowest \nameref{def:Page_Fault} rate.
We evaluate an algorithm by running it on a particular string of memory references, called a reference string, and computing the number of page faults.

To determine the number of \nameref{def:Page_Fault}s for a particular reference string and \nameref{def:Page_Replacement} algorithm, we also need to know the number of frames available.
Obviously, as the number of frames available increases, the number of page faults decreases.
Adding physical memory increases the number of frames.
Between the number of frames and the number of page faults, we generally expect a negative-exponential curve.

\paragraph{FIFO Page Replacement}\label{par:FIFO_Page_Replacement}
The simplest page-replacement algorithm is a first-in, first-out algorithm.
A FIFO replacement algorithm associates with each page the time when that page was brought into memory.
When a page must be replaced, the oldest page is chosen.

It is not really necessary to record the time when a page is brought in.
Instead, we can use a FIFO queue to hold all pages in memory.
When a page is brought into memory, we insert it at the tail of the queue.
When a page is replaced, we replace the page at the head of the queue.

Notice that, even if we select a page that is being actively used for replacement, everything still works correctly.
After we replace the active page with a new one, a \nameref{def:Page_Fault} will occur almost immediately to retrieve the active page from the swap.
Then, some other page must be replaced to bring the active page back into memory.
Thus, a bad replacement choice increases the page-fault rate and slows process execution, but does not cause incorrect execution.

The FIFO \nameref{def:Page_Replacement} algorithm is easy to understand and program.
However, its performance is not always good.
Sometimes, the number of faults for more frames is greater than the number of faults for fewer frames.
This unexpected result is known as \nameref{def:Beladys_Anomaly}.

\begin{definition}[Belady's Anomaly]\label{def:Beladys_Anomaly}
  \emph{Belady's Anomaly} states that if the number of available frames to hold pages increases, the number of \nameref{def:Page_Fault}s will increase.
\end{definition}

\paragraph{Optimal Page Replacement}\label{par:Optimal_Page_Replacement}
A result of the discovery of \nameref{def:Beladys_Anomaly} was the search for an optimal \nameref{def:Page_Replacement} algorithm.
This is the algorithm with the lowest \nameref{def:Page_Fault} rate of all algorithms and will never suffer from Belady's Anomaly.
It was found and is called \nameref{def:Optimal_Page_Replacement_Algorithm}.

\begin{definition}[OPT Algorithm]\label{def:Optimal_Page_Replacement_Algorithm}
  The \emph{OPT} or \emph{MIN} \nameref{def:Page_Replacement} \emph{Algorithm} is that has the lowest \nameref{def:Page_Fault} rate of any and all algorithm, and never suffers fro \nameref{def:Beladys_Anomaly}.
  It is:
  \begin{center}
    Replace the page that will not be used for the longest period of time.
  \end{center}

  Use of this \nameref{def:Page_Replacement} algorithm guarantees the lowest possible \nameref{def:Page_Fault} rate for a fixed number of frames.
\end{definition}


%%% Local Variables:
%%% mode: latex
%%% TeX-master: "../../EDAF35-Operating_Systems-Reference_Sheet"
%%% End:


\subsection{Thrashing}\label{subsec:Thrashing}
If a \nameref{def:Process} does not have the number of frames (\nameref{def:Physical_Memory}) it needs to support pages in active use, it will quickly page-fault.
At this point, it must replace some pages.
However, since all the pages are in active use, the OS will replace a page that will be needed again right away.
Consequently, the process faults repeatedly, replacing pages that it must bring back in immediately.

\begin{definition}[Thrashing]\label{def:Thrashing}
  \emph{Thrashing} is when a \nameref{def:Process} is spending more time \nameref{def:Paging} than executing.
  This happens when there are not enough frames in the system to support the number of pages the process needs.
  The \nameref{def:Page_Replacement} algorithm will replace some pages that the process needs right now, so it must page the first one back in, overwritting another page that is needed.
  Thus, the cycle continues.
\end{definition}

\subsubsection{Cause of Thrashing}\label{subsubsec:Thrashing_Cause}
\nameref{def:Thrashing} results in severe performance problems.
The \nameref{rmk:CPU_Scheduler} sees the decreasing CPU utilization and increases the degree of multiprogramming as a result.
The new \nameref{def:Process} tries to get started by taking frames from running processes, causing more \nameref{def:Page_Fault}s and a longer queue for the paging device (\nameref{def:Backing_Store}).
As a result, CPU utilization drops even further, and the CPU scheduler tries to increase the degree of multiprogramming even more.

This leads to system throughput plunging.
The \nameref{def:Page_Fault} rate increases tremendously, increasing the \nameref{def:Effective_Access_Time}.
No work is getting done, because the processes are spending all their time \nameref{def:Paging}.

\subsubsection{Limiting Thrashing}\label{subsubsec:Limiting_Thrashing}
We can limit the effects of \nameref{def:Thrashing} by using a local replacement algorithm.
With local replacement, if a \nameref{def:Process} starts thrashing, it cannot steal frames from another, causing the second one to thrash as well.
However, this does not entirely the problem.

\begin{remark*}
  The current best practice in implementing a computer system is to include enough \nameref{def:Physical_Memory} to avoid \nameref{def:Thrashing} and \nameref{def:Swapping}.
  Providing enough memory to keep all working sets in memory concurrently, except under extreme conditions, gives the best user experience.
\end{remark*}

If \nameref{def:Process}es are \nameref{def:Thrashing}, they will be in the queue for the paging device most of the time.
The average service time for a \nameref{def:Page_Fault} will still increase because of the longer average queue for the paging device.
Thus, the \nameref{def:Effective_Access_Time} will increase even for a process that is not thrashing.
To prevent thrashing, we must provide a process with as many frames as it \textbf{NEEDS}.

\subsubsection{Working-Set Model}\label{subsubsec:Working_Set_Model}
The \emph{Working-Set Model} uses the \nameref{def:Locality_Model} of \nameref{def:Process} execution.

\begin{definition}[Locality Model]\label{def:Locality_Model}
  The \emph{locality model} states, as a \nameref{def:Process} executes, it moves from locality to locality.
  A locality is a set of pages that are actively used together.
  A \nameref{def:Program} is generally composed of several different localities, which may overlap.

  For example, when a function is called, it defines a new locality.
  In this locality, memory references are made to the instructions of the function call, its local variables, and a subset of the global variables.
  When we exit the function, the process leaves this locality, since the local variables and instructions of the function are no longer in active use.
  We may return to this locality later.
\end{definition}

Localities are defined by the program structure and its data structures.
The \nameref{def:Locality_Model} states that all programs will exhibit this basic memory reference structure.

\begin{blackbox}
  \begin{remark*}
    Note that the \nameref{def:Locality_Model} is the unstated principle behind our mentions of caching so far.
    If accesses to any types of data were random rather than patterned, caching would be useless.
  \end{remark*}
\end{blackbox}

This model uses a parameter, $\Delta$, to define the \textbf{working-set window}.
The idea is to examine the most recent $\Delta$ page references.
The set of pages in the most recent $\Delta$ page references is the \textbf{working set}.
If a page is in active use, it will be in the working set.
If it is no longer being used, it will drop from the working set $\Delta$ time units after its last reference.
Thus, the working set is an approximation of the program’s locality.

The accuracy of the working set depends on the selection of $\Delta$.
\begin{itemize}[noitemsep]
\item $\Delta$ is too small, it will not encompass the entire locality.
\item $\Delta$ is too large, it may overlap several localities.
\item $\Delta$ is $\infty$, working set is the set of \textbf{ALL} pages touched during the \nameref{def:Process}'s execution.
\end{itemize}

Using this set, if we can find its size for all the \nameref{def:Process}es in the system, we can calculate the entire system's demand.
If the demand is greater than the total number of frames, then \nameref{def:Thrashing} will happen.

Once $\Delta$ is selected, use of the working-set model is simple.
The \nameref{def:Operating_System} monitors the working set of each \nameref{def:Process} and allocates to it enough frames to provide it with its working-set size.
If there are enough frames left over, another process can be initiated.
If the sum of the working-set sizes increases, making the demand of the system exceed the total number of available frames, the OS selects a process to suspend.
The process’s pages are swapped out, and its frames are reallocated to other processes.
The suspended process can be restarted later.

\subsubsection{Page-Fault Frequency}\label{subsubsec:Page_Fault_Frequency}
A strategy using the \nameref{def:Page_Fault} Frequency (PFF) can take a more direct approach.
\nameref{def:Thrashing} has a high page-fault rate.
Thus, we want to control the page-fault rate.
\begin{itemize}[noitemsep]
\item High, the process needs more frames.
\item Low, the process may have too many frames.
\end{itemize}

We can establish upper and lower bounds on the desired \nameref{def:Page_Fault} rate.
\begin{itemize}[noitemsep]
\item \nameref{def:Page_Fault} rate exceeds the upper limit, allocate the process another frame.
\item \nameref{def:Page_Fault} rate falls below the lower limit, remove a frame from the process.
\end{itemize}


%%% Local Variables:
%%% mode: latex
%%% TeX-master: "../../EDAF35-Operating_Systems-Reference_Sheet"
%%% End:


\subsection{Memory-Mapped Files}\label{subsec:Memory_Mapped_Files}
The sequential read of a \nameref{def:File} on disk using the standard \nameref{def:System_Call}s \kernelinline{open()}, \kernelinline{read()}, and \kernelinline{write()}.
Each file access requires a system call and disk access.
To alleviate the pain of this, we could use \nameref{def:Memory_Mapping}.

\begin{definition}[Memory Mapping]\label{def:Memory_Mapping}
  \emph{Memory mapping} is the process of taking something and mapping it to a page in memory.
  This allows a page (possibly multiple) in the \nameref{def:Virtual_Address_Space} to be logically associated with a page-sized amount of something.
\end{definition}

\subsubsection{Basic Mechanism}\label{subsubsec:Basic_Memory_Mapping_Mechanism}
Memory mapping a file is accomplished by mapping a disk block to a page (or
pages) in memory.
The steps involved are:
\begin{enumerate}[noitemsep]
\item Initial access to the \nameref{def:File} proceeds through ordinary \nameref{def:Demand_Paging}, resulting in a \nameref{def:Page_Fault}.
\item A page-sized portion of the file is read from the file system into a physical frame.
  \begin{itemize}[noitemsep]
  \item Some systems may read in more than a page-sized chunk of memory at a time
  \end{itemize}
\item Subsequent reads and writes to the file (within the loaded paged-size amount of the file) are handled as routine memory accesses.
\end{enumerate}


%%% Local Variables:
%%% mode: latex
%%% TeX-master: "../../EDAF35-Operating_Systems-Reference_Sheet"
%%% End:


\subsection{Allocating Kernel Memory}\label{subsec:Allocating_Kernel_Memory}
When a \nameref{def:Process} running in \nameref{def:User} mode requests additional memory, pages are allocated from the list of free page frames maintained by the \nameref{def:Kernel}.
This list is typically populated using a \nameref{def:Page_Replacement} algorithm and most likely contains free pages scattered throughout physical memory.
Additionally, \nameref{def:Internal_Fragmentation} may result, as the process will be granted an entire page frame, even if it doesn't need all of it..

\nameref{def:Kernel} memory is often allocated from a free-memory pool \textbf{different} from the list used to satisfy ordinary user-mode processes.
There are two primary reasons for this:
\begin{enumerate}[noitemsep]
\item The \nameref{def:Kernel} requests memory for data structures of varying sizes, some of which are less than a page in size.
  As a result, the kernel must use memory conservatively and attempt to minimize waste due to \nameref{def:Fragmentation}.
  Many \nameref{def:Operating_System}s do not subject kernel code or data to the paging system.
\item Pages allocated to \nameref{def:User}-mode processes do not necessarily have to be in contiguous physical memory.
  However, certain hardware devices interact directly with physical memory, not the \nameref{def:Virtual_Memory} interface.
  Consequently, the devices may require memory residing in \textbf{physically} contiguous pages.
\end{enumerate}

We discuss 2 systems for allocating memory to \nameref{def:Kernel} \nameref{def:Process}es.
\begin{enumerate}[noitemsep]
\item \nameref{subsubsec:Buddy_System}
\item \nameref{subsubsec:Slab_Allocation}
\end{enumerate}

\subsubsection{Buddy System}\label{subsubsec:Buddy_System}
\subsubsection{Slab Allocation}\label{subsubsec:Slab_Allocation}

%%% Local Variables:
%%% mode: latex
%%% TeX-master: "../../EDAF35-Operating_Systems-Reference_Sheet"
%%% End:


\subsection{Other Topics to Consider}\label{subsec:Other_Topics_to_Consider}
Here, we mention considerations, other than a \nameref{def:Page_Replacement} algorithm and frame allocation policy to choosing how to create our \nameref{def:Paging} system.

\subsubsection{Prepaging}\label{subsubsec:Prepaging}
An obvious property of pure \nameref{def:Demand_Paging} is the large number of \nameref{def:Page_Fault}s that occur when a \nameref{def:Process} is started or when a swapped out process is restarted.
This situation results from trying to get the initially executing locality into memory.
\nameref{def:Prepaging} helps ease this situation.

\begin{definition}[Prepaging]\label{def:Prepaging}
  \emph{Prepaging} is an attempt to prevent this high level of initial paging.
  The strategy is to bring into all the pages that will be needed into memory at one time.
\end{definition}

In a system using the \nameref{subsubsec:Working_Set_Model}, for example, we could keep with each \nameref{def:Process} a list of the pages in its working set.
If we must suspend a process, we remember the working set for that process.
When the process is to be resumed, we automatically bring back into memory its \textbf{entire working set} before restarting the process.

\nameref{def:Prepaging} may offer an advantage in some cases.
The question is whether the cost of using prepaging is less than the cost of servicing the corresponding page faults.
Prepaging becomes less effective if many of the pages brought back into memory by prepaging will not be used.

\subsubsection{Page Size}\label{subsubsec:Page_Size}
The designers of an \nameref{def:Operating_System} for an existing machine rarely have a choice concerning the page size.
However, a new machine can have different page sizes than its predecessors.
There is no single best page size.
Page sizes are almost always powers of 2, generally ranging from \SIrange{4096}{4194304}{\byte{}} ($2^{12}$ to $2^{22}$ byte, \SI{4}{\kibi{} \byte{}} to \SI{4}{\mebi{} \byte{}}).

How do we select a page size?
\begin{itemize}[noitemsep]
\item One concern is the size of the page table.
  \begin{itemize}[noitemsep]
  \item For a given virtual memory space, decreasing the page size increases the number of pages and hence the size of the page table.
  \end{itemize}

\item Because each active process must have its own copy of the \nameref{def:Page_Table}, a large page size is desirable.

\item Memory is better utilized with smaller pages.
  \begin{itemize}[noitemsep]
  \item To minimize \nameref{def:Internal_Fragmentation}, we need a small page size.
  \end{itemize}

\item Time required to read or write a page.
  Minimization of I/O time argues for a larger page size.
  \begin{itemize}[noitemsep]
  \item I/O time is composed of seek, latency, and transfer times.
    Transfer time is proportional to the amount transferred, meaning a small page size.
  \item Latency and seek time normally dwarf transfer time.
  \end{itemize}

\item With a smaller page size, total I/O should be reduced, since locality will be improved.
  \begin{itemize}[noitemsep]
  \item A smaller page size allows each page to match program locality more accurately.
\end{itemize}

\item With a smaller page size, then, we have better resolution, allowing us to isolate only the memory that is actually needed.
  \begin{itemize}[noitemsep]
  \item With a larger page size, we must allocate and transfer not only what is needed but also anything else that happens to be in the page, whether it is needed or not.
  \item Thus, a smaller page size should result in less I/O and less total allocated memory.
\end{itemize}
\item To minimize the number of page faults, we need to have a large page size.

\item What is the relationship between page size and sector size on the paging device?
\end{itemize}

This problem has no best answer.
As we have seen, some factors (\nameref{def:Internal_Fragmentation}, locality) argue for a small page size, whereas others (table size, I/O time) argue for a large page size.
The historical trend is toward larger page sizes.

\subsubsection{TLB Reach}\label{subsubsec:TLB_Reach}
Recall that the hit ratio for the Translation Lookaside Buffer (\texttt{TLB}) refers to the percentage of virtual address translations that are resolved in the TLB rather than the page table.
Clearly, the hit ratio is related to the number of entries in the TLB, and the way to increase the hit ratio is by increasing the number of entries in the TLB.\@
However, the associative memory used to construct the TLB is both expensive and power hungry.
Related to the hit ratio is another metric: the \nameref{def:TLB_Reach}.

\begin{definition}[TLB Reach]\label{def:TLB_Reach}
  \emph{TLB reach} refers to the amount of memory accessible from the Translation Lookaside Buffer (TLB).
  It is the number of entries multiplied by the page size.
\end{definition}

Ideally, the working set for a \nameref{def:Process} is stored in the TLB.\@
If it is not, the process will spend a considerable amount of time resolving memory references in the \nameref{def:Page_Table} rather than the TLB.\@

There are 3 methods to increase the \nameref{def:TLB_Reach}:
\begin{enumerate}[noitemsep]
\item Increase the number of entries in the TLB.\@

\item Increase the size of a page.
  \begin{itemize}[noitemsep]
  \item For example, increase the page size from \SI{8}{\kibi{} \byte{}} to \SI{32}{\kibi{} \byte{}}, quadruple the \nameref{def:TLB_Reach}.
  \item However, may to an increase in \nameref{def:Fragmentation}.
  \end{itemize}

\item Provide multiple page sizes.
  \begin{itemize}[noitemsep]
  \item Providing support for multiple page sizes requires the operating system
    —not hardware —to manage the TLB.\@
  \item One of the fields in a TLB entry must indicate the size of the page frame corresponding to the TLB entry.
  \item Managing the TLB in software and not hardware comes at a cost in performance.
  \item The increased hit ratio and TLB reach offset the performance costs.
  \item Recent trends indicate a move toward software-managed TLBs and operating-system support for multiple page sizes.
  \end{itemize}
\end{enumerate}

%%% Local Variables:
%%% mode: latex
%%% TeX-master: "../../EDAF35-Operating_Systems-Reference_Sheet"
%%% End:


%%% Local Variables:
%%% mode: latex
%%% TeX-master: "../EDAF35-Operating_Systems-Reference_Sheet"
%%% End:
