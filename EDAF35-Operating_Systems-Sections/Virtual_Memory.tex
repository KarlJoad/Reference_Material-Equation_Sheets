\section{Virtual Memory}\label{sec:Virtual_Memory}
As our natural definition of Von Neumann computers, the instructions being executed must be in \nameref{def:Physical_Memory}.
Thus, the first approach to meeting this requirement is to place the entire \nameref{def:Logical_Address_Space} in physical memory.
This is unfortunate, since it limits the size of a program to the size of physical memory.
\nameref{def:Dynamic_Loading} can help ease this restriction, but it generally requires special precautions and extra work by the programmer.

However, an examination of real \nameref{def:Program}s shows us that, in many cases, the entire program is not needed.
Even in those cases where the entire program is needed, it may not all be needed at the same time.
The ability to execute a program that is only partially in memory has many benefits:
\begin{itemize}[noitemsep]
\item Programs are no longer constrained by amount of \nameref{def:Physical_Memory} available.
  Users able to write programs for an extremely large \nameref{def:Virtual_Address_Space}, simplifying the programming task.
\item Each user program takes less physical memory, more programs could be run at the same time.
  This yields a corresponding increase in CPU utilization and throughput but with no increase in response time or turnaround time.
\item Less I/O needed to load or swap user programs into memory, so each user program would run faster.
\end{itemize}

\begin{definition}[Virtual Memory]\label{def:Virtual_Memory}
  Virtual memory involves the separation of logical memory as perceived by users from \nameref{def:Physical_Memory}.
  This separation allows:
  \begin{itemize}[noitemsep]
  \item Extremely large virtual memory to be provided for programmers when less physical memory is available.
  \item \nameref{def:Process}es can share files easily.
  \item \nameref{par:Shared_Memory} can be implemented.
    \begin{itemize}[noitemsep]
    \item For example, sharing a library between different processes can be handled just by mapping the virtual memory location into the process's \nameref{def:Virtual_Address_Space}.
    \item Although each process considers the library to be part of \textbf{its} virtual address space, the frames where the libraries reside in physical memory are shared by all the processes.
    \item Typically, a library is mapped read-only into the space of each process that is linked with it.
    \item This is typically implemented with \nameref{subsubsec:Shared_Pages}.
    \end{itemize}
  \item An efficient method for \nameref{def:Process} creation.
\end{itemize}

  Virtual memory makes the task of programming much easier, because the programmer no longer needs to worry about the amount of physical memory available; they can concentrate instead on the problem to be programmed.
\end{definition}

\begin{definition}[Physical Memory]\label{def:Physical_Memory}
  \emph{Physical memory} is the memory that is physically installed in the computer.
  There is a finite amount of this, determined by how much is installed by the system designer.
\end{definition}

\begin{definition}[Virtual Address Space]\label{def:Virtual_Address_Space}
  The \emph{virtual address space} of a \nameref{def:Process} consists of all the \nameref{def:Virtual_Address}es generated by a \nameref{def:Program}.
  It refers to the logical (virtual) view of \textbf{how a \nameref{def:Process} is stored in memory}.
  Typically, this view is one of perfectly continguous memory locations, when in fact, the process could be in many different, noncontiguous locations and the \nameref{def:Memory_Management_Unit} handles the \nameref{def:Paging}.

  \begin{remark}
    The \nameref{def:Virtual_Address_Space} is only calculated by one \nameref{def:Program}/\nameref{def:Process} at a time.
    To find the total virtual address space used, all \nameref{def:Process}es must have their virtual address spaces aggregated.
  \end{remark}
\end{definition}

The large blank space between the heap and the stack is part of the virtual address space but will require actual physical pages only if the heap or stack grows.
Virtual address spaces that include these holes are known as \nameref{def:Sparse_Address_Space}s.

\begin{definition}[Sparse Address Space]\label{def:Sparse_Address_Space}
  A \emph{sparse address space} is a \nameref{def:Virtual_Address_Space} that is not completely mapped to \nameref{def:Physical_Memory}.
  Typically, this is done in the case of programs for the memory space between the stack and heap, which is overwhelmingly empty.

  Sparse address spaces are beneficial because these holes can be filled on demand, when the \nameref{def:Process} needs more \nameref{def:Physical_Memory}, or to dynamically link shared objects during program execution.
\end{definition}

\subsection{Demand Paging}\label{subsec:Demand_Paging}
Because we can support the ability to \nameref{def:Dynamic_Loading} \nameref{def:Program}s, we don't need to load the \textbf{entire} program into memory initially.
For example, suppose a program starts with a list of available options from which the user is to select.
Loading the entire program into memory means loading the executable code for \textbf{ALL} options, no matter what option is chosen.

If we chose to load in parts of a \nameref{def:Program} only when they are needed, this is \nameref{def:Demand_Paging}.

\begin{definition}[Demand Paging]\label{def:Demand_Paging}
  \emph{Demand paging} is when individual portions of a \nameref{def:Program} are loaded into memory using \nameref{def:Dynamic_Loading} \textbf{only when they are needed}.

  With demand-paged \nameref{def:Virtual_Memory}, pages are loaded only when they are demanded during program execution.
  Thus, pages that are never accessed are never loaded into physical memory.
\end{definition}

When \nameref{def:Demand_Paging} is combined with \nameref{def:Swapping}, we do not want to swap the whole \nameref{def:Process} into memory if we don't need it.
Thus, we use a \nameref{def:Lazy_Swapper}.

\begin{definition}[Lazy Swapper]\label{def:Lazy_Swapper}
  A \emph{lazy swapper}, like a regular swapper, swaps \nameref{def:Process}es into memory from the \nameref{def:Backing_Store}.
  However, \textbf{it only swaps in pages of the process that will be needed}, and never swaps a page in that will not be used.

  A lazy swapper can be implemented as a \nameref{def:Pager}.
\end{definition}

When a \nameref{def:Process} is to be swapped in, the \nameref{def:Pager} guesses which pages will be used before the process is swapped out again.
Instead of swapping in a whole process, the pager brings only those pages into memory, avoiding reading pages into memory that will not be used anyway, decreasing the swap time and the amount of physical memory needed.

\begin{definition}[Pager]\label{def:Pager}
  In a \nameref{def:Demand_Paging} and \nameref{def:Swapping} system, we are not swapping whole \nameref{def:Process}es into and out of the \nameref{def:Backing_Store}; we are swapping individual pages of the process.
  Thus, the task of moving a process's pages into and out of the backing store is handled by the \emph{pager}, rather than the swapper.
\end{definition}

With a \nameref{def:Pager}, we need some form of hardware support to distinguish between the pages that are in memory and the pages that are on the disk.
The \nameref{par:Page_Valid_Invalid_Bit} scheme can be used for this purpose.
\begin{itemize}[noitemsep]
\item If this bit is set to ``valid'' the associated page is \textbf{both} legal and in memory.
\item If the bit is set to ``invalid'' the page either:
  \begin{itemize}[noitemsep]
  \item Is not valid, i.e.\ not in the \nameref{def:Logical_Address_Space} of the \nameref{def:Process}.
  \item Is valid but is currently on the disk.
  \end{itemize}
\end{itemize}

The page-table entry for a page that is brought into memory is set as usual, but the page-table entry for a page that is not currently in memory is either simply marked invalid or contains the address of the page on disk.
Marking a page invalid will have no effect if the process never attempts to access that page.
If we guess right and page in all pages that are actually needed and only those pages, the \nameref{def:Process} will run exactly as though we had brought in all pages.

While the \nameref{def:Process} executes and accesses pages that are \emph{memory resident}, execution proceeds normally.
Attempting to access a page marked invalid causes a \nameref{def:Page_Fault}.
The \nameref{def:Paging} hardware (\nameref{def:Memory_Management_Unit}), in translating the address through the page table, will notice the invalid bit is set, causing a \nameref{def:Trap} to the \nameref{def:Operating_System}.

\begin{definition}[Page Fault]\label{def:Page_Fault}
  A \emph{page fault} is a hardware \nameref{def:Trap} raised by the \nameref{def:Memory_Management_Unit}.
  It is caused by a \nameref{def:Process} attempting to access a page that has not been brought into memory by the \nameref{def:Pager}.

  The following sequence executes when a page fault occurs:
  \begin{enumerate}[noitemsep]
  \item \nameref{def:Trap} to the operating system.
  \item Save the user registers and process state.
  \item Determine that the interrupt was a page fault.
  \item Check that the page reference was legal and determine the location of the page on the disk.
  \item Issue a read from the disk to a free physical frame:
    \begin{enumerate}[noitemsep]
    \item Wait in a queue for this device until the read request is serviced.
    \item Wait for the device seek and/or latency time.
    \item Begin the transfer of the page to a free frame.
    \end{enumerate}
  \item While waiting, the CPU can be allocated to some other user (CPU scheduling, optional).
  \item Receive an interrupt from the disk I/O subsystem (I/O completed).
  \item Save the registers and process state for the other user (if step 6 is used).
  \item Determine that the interrupt was from the disk.
  \item Correct the \nameref{def:Page_Table} and other tables to show that the desired page is now in memory.
  \item Wait for the CPU to be allocated to this process again.
  \item Restore the user registers, process state, and new page table. Then resume the interrupted instruction.
  \end{enumerate}
\end{definition}

Handling a \nameref{def:Page_Fault} in straightforward.
\begin{enumerate}[noitemsep]
\item We check an internal table (usually kept with the \nameref{def:Process_Control_Block}) for this \nameref{def:Process} to determine whether the reference was a valid or an invalid memory access.
\item If the reference was invalid, we terminate the process (attempt to access memory that is not owned by this \nameref{def:Process}).
  If it was valid and legal, but we have not yet brought in that page, we now page it in.
\item We find a free physical frame (by taking one from the free-frame list, for example).
\item We schedule a disk operation to read the desired page into the newly allocated frame.
\item When the disk read is complete, we modify the page-frame internal table of the \nameref{def:Process} to indicate that the page is now in memory.
\item \textbf{We restart the instruction that was interrupted by the trap.}
  The process can now access the page as though it had always been in memory.
\end{enumerate}

A \nameref{def:Page_Fault} may occur at any memory reference.
A crucial requirement for \nameref{def:Demand_Paging} is the ability to restart any instruction after a page fault.
We save the state (registers, condition code, instruction counter) of the interrupted \nameref{def:Process} when the page fault occurs.
This lets us restart the process in exactly the same place and state, except that the desired page is now in memory and is accessible.

\subsubsection{Problems with Demand Paging}\label{subsubsec:Demand_Paging_Problems}
Theoretically, some programs could access several new pages of memory with each instruction execution (one page for the instruction and many for data), possibly causing multiple page faults per instruction.
This situation would result in unacceptable system performance.
Fortunately, analysis of running processes shows that this behavior is exceedingly unlikely.
Programs tend to have locality of reference giving reasonable performance from demand paging.

The major difficulty arises when one instruction may modify several different locations.
If a block (source or destination) straddles a page boundary, a page fault might occur after the move is partially done.
In addition, if the source and destination blocks overlap, the source block may have been modified, in which case we cannot simply restart the instruction.
This particular problem can be solved in two different ways.
\begin{enumerate}[noitemsep]
\item The microcode computes and attempts to access \textbf{both ends} of \textbf{both blocks}.
  If a \nameref{def:Page_Fault} is going to occur, it will happen at this step, before anything is modified.
  The instruction can then execute, knowing that no page fault can occur, since all relevant pages are in memory.
\item Use temporary registers to hold the values of overwritten locations.
  If there is a \nameref{def:Page_Fault}, all the old values are written back into memory before the \nameref{def:Trap} occurs.
  This restores memory to its state before the instruction was started, so that the instruction can be repeated.
\end{enumerate}

\subsubsection{Hardware Required}\label{subsubsec:Demand_Paging_Required_Hardware}
The hardware to support demand paging is the same as the hardware for \nameref{def:Paging} and \nameref{def:Swapping}:
\begin{enumerate}[noitemsep]
\item Page table.
  This table has the ability to mark an entry invalid through a valid–invalid bit or a special value of protection bits.
\item Secondary memory.
  This memory holds those pages that are not present in main memory.
  The secondary memory is usually a high-speed disk.
  It is known as the \emph{swap device}, and the section of disk used for this purpose is known as swap space.
\end{enumerate}

\subsubsection{Performance}\label{subsubsec:Demand_Paging_Performance}
Demand paging can significantly affect the performance of a computer system.
To illustrate, the \nameref{def:Effective_Access_Time} for demand-paged memory can be computed.

\subsubsection{Swap Usage}\label{subsubsec:Demand_Paging_Swap_Usage}

%%% Local Variables:
%%% mode: latex
%%% TeX-master: "../../EDAF35-Operating_Systems-Reference_Sheet"
%%% End:


\subsection{Copy-on-Write}\label{subsec:Memory_Copy_on_Write}
Recall that the \kernelinline{fork()} \nameref{def:System_Call} creates a child \nameref{def:Process} that is a nearly perfect duplicate of its parent.
Traditionally, this was done by creating a copy of the parent’s address space for the child, thereby \textbf{duplicating} the pages belonging to the parent.
However, considering that many child processes invoke the \kernelinline{exec()} system call immediately after creation, the copying of the parent’s address space may be unnecessary.
\nameref{def:Memory_Copy_on_Write} works by allowing the parent and child processes initially to share the same pages allowing the \kernelinline{fork()} system call to bypass the need for \nameref{def:Demand_Paging} by using a technique similar to page sharing (\Cref{subsec:Paging}).

\begin{definition}[Copy-on-Write]\label{def:Memory_Copy_on_Write}
  \emph{Copy-on-Write} allows the \kernelinline{fork()}ing parent \nameref{def:Process} to share its pages with its newly spawned child process.
  This bypasses the need for \nameref{def:Demand_Paging}, by essentially allowing pages to be shared (like in \nameref{par:Shared_Memory} schemes).
  These shared pages are marked as Copy-on-Write pages; only pages that \textbf{can be} modified need be marked as copy-on-write.
  Pages that cannot be modified (pages containing executable code) can always be shared by the parent and child.

  This allows for rapid process creation and minimizes the number of new pages that must be allocated to the newly created process.
  The pages marked as copy-on-write pages, mean if \textbf{EITHER} process writes to any of the shared pages, a copy of the shared page is created.
  The writer has its \nameref{def:Page_Table} changed to point to this one, which is a complete copy (except for the newly written changes).
  The process that did nothing to the page continues to point to the original page.
\end{definition}

To handle the allocation of pages for a \nameref{def:Memory_Copy_on_Write} page, the location that it is allocated from is important.
Many \nameref{def:Operating_System}s provide a pool of free pages for such requests.
These free pages are typically allocated when the stack or heap for a \nameref{def:Process} must expand or when there are copy-on-write pages to be managed.
Operating systems typically allocate these pages using a technique known as \nameref{def:Zero_Fill_on_Demand}.


%%% Local Variables:
%%% mode: latex
%%% TeX-master: "../../EDAF35-Operating_Systems-Reference_Sheet"
%%% End:


%%% Local Variables:
%%% mode: latex
%%% TeX-master: "../EDAF35-Operating_Systems-Reference_Sheet"
%%% End:
