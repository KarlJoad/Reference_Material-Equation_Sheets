\section{Virtual Memory}\label{sec:Virtual_Memory}
As our natural definition of Von Neumann computers, the instructions being executed must be in \nameref{def:Physical_Memory}.
Thus, the first approach to meeting this requirement is to place the entire \nameref{def:Logical_Address_Space} in physical memory.
This is unfortunate, since it limits the size of a program to the size of physical memory.
\nameref{def:Dynamic_Loading} can help ease this restriction, but it generally requires special precautions and extra work by the programmer.

However, an examination of real \nameref{def:Program}s shows us that, in many cases, the entire program is not needed.
Even in those cases where the entire program is needed, it may not all be needed at the same time.
The ability to execute a program that is only partially in memory has many benefits:
\begin{itemize}[noitemsep]
\item Programs are no longer constrained by amount of \nameref{def:Physical_Memory} available.
  Users able to write programs for an extremely large \nameref{def:Virtual_Address_Space}, simplifying the programming task.
\item Each user program takes less physical memory, more programs could be run at the same time.
  This yields a corresponding increase in CPU utilization and throughput but with no increase in response time or turnaround time.
\item Less I/O needed to load or swap user programs into memory, so each user program would run faster.
\end{itemize}

\begin{definition}[Virtual Memory]\label{def:Virtual_Memory}
  Virtual memory involves the separation of logical memory as perceived by users from \nameref{def:Physical_Memory}.
  This separation allows:
  \begin{itemize}[noitemsep]
  \item Extremely large virtual memory to be provided for programmers when less physical memory is available.
  \item \nameref{def:Process}es can share files easily.
  \item \nameref{par:Shared_Memory} can be implemented.
    \begin{itemize}[noitemsep]
    \item For example, sharing a library between different processes can be handled just by mapping the virtual memory location into the process's \nameref{def:Virtual_Address_Space}.
    \item Although each process considers the library to be part of \textbf{its} virtual address space, the frames where the libraries reside in physical memory are shared by all the processes.
    \item Typically, a library is mapped read-only into the space of each process that is linked with it.
    \item This is typically implemented with \nameref{subsubsec:Shared_Pages}.
    \end{itemize}
  \item An efficient method for \nameref{def:Process} creation.
\end{itemize}

  Virtual memory makes the task of programming much easier, because the programmer no longer needs to worry about the amount of physical memory available; they can concentrate instead on the problem to be programmed.
\end{definition}

\begin{definition}[Physical Memory]\label{def:Physical_Memory}
  \emph{Physical memory} is the memory that is physically installed in the computer.
  There is a finite amount of this, determined by how much is installed by the system designer.
\end{definition}

\begin{definition}[Virtual Address Space]\label{def:Virtual_Address_Space}
  The \emph{virtual address space} of a \nameref{def:Process} consists of all the \nameref{def:Virtual_Address}es generated by a \nameref{def:Program}.
  It refers to the logical (virtual) view of \textbf{how a \nameref{def:Process} is stored in memory}.
  Typically, this view is one of perfectly continguous memory locations, when in fact, the process could be in many different, noncontiguous locations and the \nameref{def:Memory_Management_Unit} handles the \nameref{def:Paging}.

  \begin{remark}
    The \nameref{def:Virtual_Address_Space} is only calculated by one \nameref{def:Program}/\nameref{def:Process} at a time.
    To find the total virtual address space used, all \nameref{def:Process}es must have their virtual address spaces aggregated.
  \end{remark}
\end{definition}

The large blank space between the heap and the stack is part of the virtual address space but will require actual physical pages only if the heap or stack grows.
Virtual address spaces that include these holes are known as \nameref{def:Sparse_Address_Space}s.

\begin{definition}[Sparse Address Space]\label{def:Sparse_Address_Space}
  A \emph{sparse address space} is a \nameref{def:Virtual_Address_Space} that is not completely mapped to \nameref{def:Physical_Memory}.
  Typically, this is done in the case of programs for the memory space between the stack and heap, which is overwhelmingly empty.

  Sparse address spaces are beneficial because these holes can be filled on demand, when the \nameref{def:Process} needs more \nameref{def:Physical_Memory}, or to dynamically link shared objects during program execution.
\end{definition}

\subsection{Demand Paging}\label{subsec:Demand_Paging}

%%% Local Variables:
%%% mode: latex
%%% TeX-master: "../../EDAF35-Operating_Systems-Reference_Sheet"
%%% End:


\subsection{Copy-on-Write}\label{subsec:Memory_Copy_on_Write}

%%% Local Variables:
%%% mode: latex
%%% TeX-master: "../../EDAF35-Operating_Systems-Reference_Sheet"
%%% End:


\subsection{Page Replacement}\label{subsec:Page_Replacement}
If a \nameref{def:Process} of ten pages actually uses only half of them, then \nameref{def:Demand_Paging} saves the I/O necessary to load the five pages that are never used.
This also increases the degree of multiprogramming by running more processes, \nameref{def:Over_Allocating} memory.

\begin{definition}[Over-Allocating]\label{def:Over_Allocating}
  \emph{Over-allocating} is the process of putting more load on a component in the system than would normally be possible.

  In the context of \nameref{def:Paging} and \nameref{def:Demand_Paging}, this means that main memory has more \nameref{def:Process}es in it than would be normal in normal \nameref{def:Paging} schemes.
  This is because \textbf{ONLY} the pages that are needed to run these processes are loaded, and nothing more.
\end{definition}

If we run multiple \nameref{def:Process}es, each of which is multiple pages in size but actually only uses some of them, we can achieve higher CPU utilization and throughput, with physical frames to spare.
It is possible that each of these processes, for some reason, may suddenly try to use \textbf{all} of their pages \textbf{simulataneously}, resulting in a need for more physical frames than the system has.
The operating system has several options at this point:
\begin{enumerate}[noitemsep]
\item Terminate a user process.
  \begin{itemize}[noitemsep]
  \item \nameref{def:Demand_Paging} is the operating system’s attempt to improve the system’s utilization and throughput.
  \item Users should not be aware that their processes are running on a paged system—
  \item Paging should be logically transparent to the user.
  \item Not the best choice.
  \end{itemize}

\item Replace one page in memory with the requested one in swap (\nameref{def:Page_Replacement}).
  \begin{itemize}[noitemsep]
  \item \nameref{def:Operating_System} swaps out a process, freeing \textbf{all} its frames.
  \item Reduce the level of multiprogramming.
  \item Good option in certain circumstances (\nameref{def:Thrashing}).
  \end{itemize}
\end{enumerate}

\begin{definition}[Page Replacement]\label{def:Page_Replacement}
  \emph{Page replacement} is the process of replacing a page in memory with one that is \nameref{def:Swapping} in from the swap.

  Page replacement is basic to \nameref{def:Demand_Paging}.
  It completes the separation between logical memory and \nameref{def:Physical_Memory}.
  With this mechanism, an enormous \nameref{def:Virtual_Memory} can be provided for programmers on a smaller physical memory.
  Without demand paging, user addresses are still mapped into physical addresses, and the two sets of addresses can be different, however, all the pages of a process still must be in physical memory.
  With demand paging, the size of the \nameref{def:Logical_Address_Space} is no longer constrained by physical memory.

  The procedure for performing this replacement is listed in \Cref{subsubsec:Page_Replacement_Procedure}.
\end{definition}

Buffers for I/O also consume a considerable amount of memory.
Deciding how much memory to allocate to I/O and how much to program pages is a significant challenge.
Some systems allocate a fixed percentage of memory for I/O buffers, others allow user processes and the I/O subsystem to compete for all system memory.

\subsubsection{Replacement Procedure}\label{subsubsec:Page_Replacement_Procedure}
The basic steps to perform a \nameref{def:Page_Replacement} are shown below:
\begin{enumerate}[noitemsep]
\item Find the location of the desired page on the disk.
\item Find a free frame:
  \begin{enumerate}[noitemsep]
  \item If there is a free frame, use it.
  \item If there is no free frame, use a page-replacement algorithm to select a victim frame.
  \item Write the victim frame to the disk; change the page and frame tables accordingly.
  \end{enumerate}
\item Read the desired page into the newly freed frame; change the page and frame tables.
\item Continue the user process from where the page fault occurred.
\end{enumerate}

If no frames in memory are free, two page transfers (move one page out and one in) are required, effectively doubling the \nameref{def:Page_Fault} service time and increases the \nameref{def:Effective_Access_Time} accordingly.
Reducing this overhead can be done by using a \nameref{def:Modify_Bit} (\nameref{rmk:Dirty_Bit}).

\begin{definition}[Modify Bit]\label{def:Modify_Bit}
  The \emph{modify bit} is a signalling bit associated with each page or frame has a modify bit associated with it in the hardware.
  The modify bit for a page is set by the hardware whenever any byte in the page is written, indicating that the page has been modified.

  \begin{remark}[Dirty Bit]\label{rmk:Dirty_Bit}
    The \nameref{def:Modify_Bit} is sometimes called the \emph{dirty bit}, because if it is set, then the memory can be considered ``dirty'' and must be written before any further changes are made.
  \end{remark}
\end{definition}

When we select a page for replacement, we examine its \nameref{def:Modify_Bit}.
\begin{itemize}[noitemsep]
\item Modify Bit set, we know the page has been modified since the \textbf{last} read-in from the disk.
  \begin{itemize}[noitemsep]
  \item Must write the page to the disk.
  \end{itemize}
\item Modify Bit not set, the page \textbf{has not been modified} since it was read into memory.
  \begin{itemize}[noitemsep]
  \item No need to write the memory page to the disk; an exact replica is already there.
  \end{itemize}
\end{itemize}

\begin{remark*}
  This also applies to read-only pages (for example, pages of binary code).
  Since such pages cannot be modified, they may be discarded when desired.
  This can significantly reduce the time required to service a page fault, since it reduces \textbf{I/O time} by one-half if the page has not been modified.
\end{remark*}

\subsubsection{Algorithms}\label{subsubsec:Page_Replacement_Algorithms}
In general, we select a particular replacement algorithm with the lowest \nameref{def:Page_Fault} rate.
We evaluate an algorithm by running it on a particular string of memory references, called a reference string, and computing the number of page faults.

To determine the number of \nameref{def:Page_Fault}s for a particular reference string and \nameref{def:Page_Replacement} algorithm, we also need to know the number of frames available.
Obviously, as the number of frames available increases, the number of page faults decreases.
Adding physical memory increases the number of frames.
Between the number of frames and the number of page faults, we generally expect a negative-exponential curve.

\paragraph{FIFO Page Replacement}\label{par:FIFO_Page_Replacement}
The simplest page-replacement algorithm is a first-in, first-out algorithm.
A FIFO replacement algorithm associates with each page the time when that page was brought into memory.
When a page must be replaced, the oldest page is chosen.

It is not really necessary to record the time when a page is brought in.
Instead, we can use a FIFO queue to hold all pages in memory.
When a page is brought into memory, we insert it at the tail of the queue.
When a page is replaced, we replace the page at the head of the queue.

Notice that, even if we select a page that is being actively used for replacement, everything still works correctly.
After we replace the active page with a new one, a \nameref{def:Page_Fault} will occur almost immediately to retrieve the active page from the swap.
Then, some other page must be replaced to bring the active page back into memory.
Thus, a bad replacement choice increases the page-fault rate and slows process execution, but does not cause incorrect execution.

The FIFO \nameref{def:Page_Replacement} algorithm is easy to understand and program.
However, its performance is not always good.
Sometimes, the number of faults for more frames is greater than the number of faults for fewer frames.
This unexpected result is known as \nameref{def:Beladys_Anomaly}.

\begin{definition}[Belady's Anomaly]\label{def:Beladys_Anomaly}
  \emph{Belady's Anomaly} states that if the number of available frames to hold pages increases, the number of \nameref{def:Page_Fault}s will increase.
\end{definition}

\paragraph{Optimal Page Replacement}\label{par:Optimal_Page_Replacement}
A result of the discovery of \nameref{def:Beladys_Anomaly} was the search for an optimal \nameref{def:Page_Replacement} algorithm.
This is the algorithm with the lowest \nameref{def:Page_Fault} rate of all algorithms and will never suffer from Belady's Anomaly.
It was found and is called \nameref{def:Optimal_Page_Replacement_Algorithm}.

\begin{definition}[OPT Algorithm]\label{def:Optimal_Page_Replacement_Algorithm}
  The \emph{OPT} or \emph{MIN} \nameref{def:Page_Replacement} \emph{Algorithm} is that has the lowest \nameref{def:Page_Fault} rate of any and all algorithm, and never suffers fro \nameref{def:Beladys_Anomaly}.
  It is:
  \begin{center}
    Replace the page that will not be used for the longest period of time.
  \end{center}

  Use of this \nameref{def:Page_Replacement} algorithm guarantees the lowest possible \nameref{def:Page_Fault} rate for a fixed number of frames.
\end{definition}

Unfortunately, the optimal page-replacement algorithm is difficult, if not impossible, to implement, because it requires \textbf{future} knowledge of the reference string.
This is a similar situation as with the \nameref{par:SJF_Scheduling} algorithm in \Cref{par:SJF_Scheduling}.
As a result, the optimal algorithm is used mainly for comparison studies.
For instance, a new algorithm may not be optimal, but it is within \SI{12.3}{\percent} of optimal at worst and within \SI{4.7}{\percent} on average.


%%% Local Variables:
%%% mode: latex
%%% TeX-master: "../../EDAF35-Operating_Systems-Reference_Sheet"
%%% End:


\subsection{Thrashing}\label{subsec:Thrashing}

%%% Local Variables:
%%% mode: latex
%%% TeX-master: "../../EDAF35-Operating_Systems-Reference_Sheet"
%%% End:


\subsection{Memory-Mapped Files}\label{subsec:Memory_Mapped_Files}
The sequential read of a \nameref{def:File} on disk using the standard \nameref{def:System_Call}s \kernelinline{open()}, \kernelinline{read()}, and \kernelinline{write()}.
Each file access requires a system call and disk access.
To alleviate the pain of this, we could use \nameref{def:Memory_Mapping}.


%%% Local Variables:
%%% mode: latex
%%% TeX-master: "../../EDAF35-Operating_Systems-Reference_Sheet"
%%% End:


\subsection{Allocating Kernel Memory}\label{subsec:Allocating_Kernel_Memory}
When a \nameref{def:Process} running in \nameref{def:User} mode requests additional memory, pages are allocated from the list of free page frames maintained by the \nameref{def:Kernel}.
This list is typically populated using a \nameref{def:Page_Replacement} algorithm and most likely contains free pages scattered throughout physical memory.
Additionally, \nameref{def:Internal_Fragmentation} may result, as the process will be granted an entire page frame, even if it doesn't need all of it..

\nameref{def:Kernel} memory is often allocated from a free-memory pool \textbf{different} from the list used to satisfy ordinary user-mode processes.

%%% Local Variables:
%%% mode: latex
%%% TeX-master: "../../EDAF35-Operating_Systems-Reference_Sheet"
%%% End:


\subsection{Other Topics to Consider}\label{subsec:Other_Topics_to_Consider}
Here, we mention considerations, other than a \nameref{def:Page_Replacement} algorithm and frame allocation policy to choosing how to create our \nameref{def:Paging} system.

\subsubsection{Prepaging}\label{subsubsec:Prepaging}
An obvious property of pure \nameref{def:Demand_Paging} is the large number of \nameref{def:Page_Fault}s that occur when a \nameref{def:Process} is started or when a swapped out process is restarted.
This situation results from trying to get the initially executing locality into memory.
\nameref{def:Prepaging} helps ease this situation.

\begin{definition}[Prepaging]\label{def:Prepaging}
  \emph{Prepaging} is an attempt to prevent this high level of initial paging.
  The strategy is to bring into all the pages that will be needed into memory at one time.
\end{definition}

In a system using the \nameref{subsubsec:Working_Set_Model}, for example, we could keep with each \nameref{def:Process} a list of the pages in its working set.
If we must suspend a process, we remember the working set for that process.
When the process is to be resumed, we automatically bring back into memory its \textbf{entire working set} before restarting the process.

\nameref{def:Prepaging} may offer an advantage in some cases.
The question is whether the cost of using prepaging is less than the cost of servicing the corresponding page faults.
Prepaging becomes less effective if many of the pages brought back into memory by prepaging will not be used.

\subsubsection{Page Size}\label{subsubsec:Page_Size}
The designers of an \nameref{def:Operating_System} for an existing machine rarely have a choice concerning the page size.
However, a new machine can have different page sizes than its predecessors.
There is no single best page size.
Page sizes are almost always powers of 2, generally ranging from \SIrange{4096}{4194304}{\byte{}} ($2^{12}$ to $2^{22}$ byte, \SI{4}{\kibi{} \byte{}} to \SI{4}{\mebi{} \byte{}}).

How do we select a page size?
\begin{itemize}[noitemsep]
\item One concern is the size of the page table.
  \begin{itemize}[noitemsep]
  \item For a given virtual memory space, decreasing the page size increases the number of pages and hence the size of the page table.
  \end{itemize}

\item Because each active process must have its own copy of the \nameref{def:Page_Table}, a large page size is desirable.

\item Memory is better utilized with smaller pages.
  \begin{itemize}[noitemsep]
  \item To minimize \nameref{def:Internal_Fragmentation}, we need a small page size.
  \end{itemize}

\item Time required to read or write a page.
  Minimization of I/O time argues for a larger page size.
  \begin{itemize}[noitemsep]
  \item I/O time is composed of seek, latency, and transfer times.
    Transfer time is proportional to the amount transferred, meaning a small page size.
  \item Latency and seek time normally dwarf transfer time.
  \end{itemize}

\item With a smaller page size, total I/O should be reduced, since locality will be improved.
  \begin{itemize}[noitemsep]
  \item A smaller page size allows each page to match program locality more accurately.
\end{itemize}

\item With a smaller page size, then, we have better resolution, allowing us to isolate only the memory that is actually needed.
  \begin{itemize}[noitemsep]
  \item With a larger page size, we must allocate and transfer not only what is needed but also anything else that happens to be in the page, whether it is needed or not.
  \item Thus, a smaller page size should result in less I/O and less total allocated memory.
\end{itemize}
\item To minimize the number of page faults, we need to have a large page size.

\item What is the relationship between page size and sector size on the paging device?
\end{itemize}

This problem has no best answer.
As we have seen, some factors (\nameref{def:Internal_Fragmentation}, locality) argue for a small page size, whereas others (table size, I/O time) argue for a large page size.
The historical trend is toward larger page sizes.

\subsubsection{TLB Reach}\label{subsubsec:TLB_Reach}
Recall that the hit ratio for the Translation Lookaside Buffer (\texttt{TLB}) refers to the percentage of virtual address translations that are resolved in the TLB rather than the page table.
Clearly, the hit ratio is related to the number of entries in the TLB, and the way to increase the hit ratio is by increasing the number of entries in the TLB.\@
However, the associative memory used to construct the TLB is both expensive and power hungry.
Related to the hit ratio is another metric: the \nameref{def:TLB_Reach}.

\begin{definition}[TLB Reach]\label{def:TLB_Reach}
  \emph{TLB reach} refers to the amount of memory accessible from the Translation Lookaside Buffer (TLB).
  It is the number of entries multiplied by the page size.
\end{definition}

Ideally, the working set for a \nameref{def:Process} is stored in the TLB.\@
If it is not, the process will spend a considerable amount of time resolving memory references in the \nameref{def:Page_Table} rather than the TLB.\@

There are 3 methods to increase the \nameref{def:TLB_Reach}:
\begin{enumerate}[noitemsep]
\item Increase the number of entries in the TLB.\@

\item Increase the size of a page.
  \begin{itemize}[noitemsep]
  \item For example, increase the page size from \SI{8}{\kibi{} \byte{}} to \SI{32}{\kibi{} \byte{}}, quadruple the \nameref{def:TLB_Reach}.
  \item However, may to an increase in \nameref{def:Fragmentation}.
  \end{itemize}

\item Provide multiple page sizes.
  \begin{itemize}[noitemsep]
  \item Providing support for multiple page sizes requires the operating system
    —not hardware —to manage the TLB.\@
  \item One of the fields in a TLB entry must indicate the size of the page frame corresponding to the TLB entry.
  \item Managing the TLB in software and not hardware comes at a cost in performance.
  \item The increased hit ratio and TLB reach offset the performance costs.
  \item Recent trends indicate a move toward software-managed TLBs and operating-system support for multiple page sizes.
  \end{itemize}
\end{enumerate}

\subsubsection{Inverted Page Tables}\label{subsubsec:Inverted_Page_Tables}
The purpose of this form of page management is to reduce the amount of \nameref{def:Physical_Memory} needed to track virtual-to-physical address translations, by keeping information about which \nameref{def:Virtual_Memory} page is stored in each physical frame.
We accomplish this savings by creating a table that has one entry per page of physical memory, indexed by the pair $\langle \mathtt{PID}, page-number \rangle$.
However, the inverted page table no longer contains complete information about the \nameref{def:Logical_Address_Space} of a \nameref{def:Process}, and that information is required if a referenced page is not currently in memory.

\nameref{def:Demand_Paging} requires this information to process \nameref{def:Page_Fault}s.
For the information to be available, an external page table (one per \nameref{def:Process}) must be kept.
This table looks like the traditional per-process page table and contains information on where each virtual page is located.

Since these tables are referenced only when a \nameref{def:Page_Fault} occurs, they do not need to be available quickly.
Instead, they are themselves paged in and out of memory as necessary.
But now a page fault may cause the virtual memory manager to generate another page fault as it pages in the external page table it needs to locate the virtual page on the \nameref{def:Backing_Store}.
This special case requires careful handling in the \nameref{def:Kernel} and a delay in the page-lookup processing.

\subsubsection{Program Structure}\label{subsubsec:Program_Structure}
\nameref{def:Demand_Paging} is designed to be transparent to the \nameref{def:User} \nameref{def:Program}.
In many cases, the user is completely unaware of the paged nature of memory.
However, system performance can be improved if the user (or compiler) has an awareness of the underlying demand paging.

Good selection of data structures and programming structures can increase locality and hence lower the \nameref{def:Page_Fault} rate and the number of pages in the working set.
\begin{itemize}[noitemsep]
\item A stack has good locality, since access is always made to the top.
\item A hash table is designed to scatter references, producing bad locality.
\end{itemize}

Of course, locality of reference is just one measure of the efficiency of the use of a data structure.
Other heavily weighted factors include search speed, total number of memory references, and total number of pages touched.

At a later stage, the compiler and loader can have a significant effect on \nameref{def:Paging}.
Separating code and data and generating \nameref{def:Reentrant} code means that code pages can be read-only and hence will never be modified meaning they never have to be paged out to be replaced.
The loader can avoid placing routines across page boundaries, keeping each routine completely in one page.
Routines that call each other many times can be packed into the same page.
This packaging is a variant of the bin-packing problem: try to pack the variable-sized load segments into the fixed-sized pages so that interpage references are minimized.
This is particularly useful for large page sizes.

\subsubsection{I/O Interlock and Page Locking}\label{IO:subsubsec_Interlock_Page_Locking}
When using I/O devices with \nameref{def:Demand_Paging}, we sometimes need to allow some of the pages to be locked in memory.

One solution is never to execute I/O to \nameref{def:User} memory.
Instead, data is copied between \nameref{def:Kernel} memory and user memory, then the I/O takes place only between kernel memory and the I/O device.
To write a block on tape, we first copy the block to system memory and then write it to tape.
This extra copying may result in unacceptably high overhead.

%%% Local Variables:
%%% mode: latex
%%% TeX-master: "../../EDAF35-Operating_Systems-Reference_Sheet"
%%% End:


%%% Local Variables:
%%% mode: latex
%%% TeX-master: "../EDAF35-Operating_Systems-Reference_Sheet"
%%% End:
