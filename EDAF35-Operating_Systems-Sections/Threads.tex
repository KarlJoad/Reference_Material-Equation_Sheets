\section{Threads}\label{sec:Threads}
\begin{definition}[Thread]\label{def:Thread}
  \emph{Threads of execution}, often shortened to \emph{threads}, are the objects of activity within the process.
  Each thread includes:
  \begin{itemize}[noitemsep]
  \item Thread ID
  \item A unique program counter
  \item Process stack
  \item Set of processor \nameref{def:Register}s
  \end{itemize}

  The \nameref{def:Kernel} schedules the individual threads, not \nameref{def:Process}es.

  In traditional UNIX systems, each \nameref{def:Process} consists of one thread.
  In modern systems, however, multithreaded programs/processes are common.
  In this case, this \nameref{def:Process}'s threads share:
  \begin{itemize}[noitemsep]
  \item The Code Section
  \item The Data Section
  \item Operating System resources, such as files and signals.
  \end{itemize}

  \begin{remark}[Threads in Linux]\label{rmk:Linux_Threads}
    Linux has a unique implementation of threads; it does not differentiate between \nameref{def:Thread}s and \nameref{def:Process}es.
    To Linux, a thread is just a special kind of process.
  \end{remark}
\end{definition}


%%% Local Variables:
%%% mode: latex
%%% TeX-master: "../EDAF35-Operating_Systems-Reference_Sheet"
%%% End:
