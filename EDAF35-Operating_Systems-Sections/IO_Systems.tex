\section{I/O Systems}\label{sec:IO_Systems}
Because I/O devices vary so widely in their function and speed, varied methods are needed to control them.
These methods form the I/O subsystem of the kernel, which separates the rest of the kernel from the complexities of managing I/O devices.

To encapsulate the details and oddities of different devices, the kernel of an operating system is structured to use \nameref{def:Device_Driver} modules.

\begin{definition}[Device Driver]\label{def:Device_Driver}
  \emph{Device Driver}s present a uniform device-access interface to the I/O subsystem.
  Every device requires a device driver (though certain devices can reuse another device's driver).
  The device driver hides the device-specific implementation details from the \nameref{def:Operating_System} and I/O subsystem, and only showing a standard, uniform interface.
  The type of interface exported depends on the nature of the device itself.

  This is analogous to \nameref{def:System_Call}s providing a standard interface between the application and the \nameref{def:Operating_System}.
\end{definition}

Despite the incredible variety of I/O devices, though, we need only a few concepts to understand how the devices are attached and how the software can control the hardware.


%%% Local Variables:
%%% mode: latex
%%% TeX-master: "../EDAF35-Operating_Systems-Reference_Sheet"
%%% End:
