\subsection{Algorithm Evaluation}\label{subsec:Algorithm_Evaluation}
Now that we have selected a \nameref{def:Scheduling_Algorithm} to use, how do we know that it was the right choice?
First, we need to know what our criteria were.
Some systems might have multiple criteria at a time, such as:
\begin{itemize}[noitemsep]
\item Maximum CPU response time is 1 second.
\item Turnaround time is (on average) linearly proportional to total execution time.
\end{itemize}

To do this, there are 4 main ways to do this:
\begin{enumerate}[noitemsep]
\item \nameref{par:Deterministic_Modeling}
\item \nameref{par:Queuing_Models}
\item \nameref{par:Simulations}
\item \nameref{par:Implementation}
\end{enumerate}

\paragraph{Deterministic Modeling}\label{par:Deterministic_Modeling}
Deterministic modeling is simple and fast.
It gives us exact numbers, allowing us to compare the algorithms.
However, it requires \textbf{exact numbers for input}, and its answers apply \emph{only} to those cases.

This method takes a particular predetermined workload and defines the performance of each algorithm for that workload.

The main uses of deterministic modeling are in describing \nameref{def:Scheduling_Algorithm}s and providing examples.
In cases where we are running the same program repeatedly, we can measure the program’s processing requirements exactly, allowing us to select a \nameref{def:Scheduling_Algorithm}.
Furthermore, over a set of examples, deterministic modeling may indicate trends that can then be analyzed and proved separately.


%%% Local Variables:
%%% mode: latex
%%% TeX-master: "../../EDAF35-Operating_Systems-Reference_Sheet"
%%% End:
