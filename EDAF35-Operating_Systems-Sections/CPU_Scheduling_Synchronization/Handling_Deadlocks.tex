\subsection{Handling Deadlocks}\label{subsec:Handling_Deadlocks}
There are 3 main ways to handle a deadlock:
\begin{enumerate}[noitemsep]
\item We can use a protocol to prevent (\Cref{subsubsec:Deadlock_Prevention}) or avoid (\Cref{subsubsec:Deadlock_Avoidance}) deadlocks, ensuring that the system will never enter a deadlocked state.
\item We can allow the system to enter a deadlocked state, detect it, and recover (\Cref{subsubsec:Deadlock_Detection}).
\item We can ignore the problem altogether and pretend that deadlocks never occur in the system.
\end{enumerate}

The third method of handling a \nameref{def:Deadlock} may not make sense, but from a cost perspective, it does.
Ignoring the possibility of deadlocks is cheaper.
In many systems, deadlocks occur infrequently (once per year), the extra expense of the methods may not seem worthwhile.
In addition, methods used to recover from other conditions may be put to use to recover from deadlock.
In some circumstances, a system is in a frozen state but not in a deadlocked state.

\subsubsection{Deadlock Prevention}\label{subsubsec:Deadlock_Prevention}
\subsubsection{Deadlock Avoidance}\label{subsubsec:Deadlock_Avoidance}
\subsubsection{Deadlock Detection}\label{subsubsec:Deadlock_Detection}

%%% Local Variables:
%%% mode: latex
%%% TeX-master: "../../EDAF35-Operating_Systems-Reference_Sheet"
%%% End:
