\subsection{Handling Deadlocks}\label{subsec:Handling_Deadlocks}
There are 3 main ways to handle a deadlock:
\begin{enumerate}[noitemsep]
\item We can use a protocol to prevent (\Cref{subsubsec:Deadlock_Prevention}) or avoid (\Cref{subsubsec:Deadlock_Avoidance}) deadlocks, ensuring that the system will never enter a deadlocked state.
\item We can allow the system to enter a deadlocked state, detect it, and recover (\Cref{subsubsec:Deadlock_Detection}).
\item We can ignore the problem altogether and pretend that deadlocks never occur in the system.
\end{enumerate}

The third method of handling a \nameref{def:Deadlock} may not make sense, but from a cost perspective, it does.
Ignoring the possibility of deadlocks is cheaper.
In many systems, deadlocks occur infrequently (once per year), the extra expense of the methods may not seem worthwhile.
In addition, methods used to recover from other conditions may be put to use to recover from deadlock.
In some circumstances, a system is in a frozen state but not in a deadlocked state.

\subsubsection{Deadlock Prevention}\label{subsubsec:Deadlock_Prevention}
\begin{definition}[Deadlock Prevention]\label{def:Deadlock_Prevention}
  \emph{Deadlock prevention} is done by providing a set of methods that ensure any one of the \nameref{subsubsec:Deadlock_Conditions} cannot occur.
\end{definition}

\paragraph{Mutual Exclusion}\label{par:Deadlock_Prevention-Mutual_Exclusion}
That is, at least one resource must be nonsharable.
Sharable resources, in contrast, do not require mutually exclusive access and thus cannot be involved in a deadlock.
A process never needs to wait for a sharable resource.

Read-only files are a good example of a sharable resource.
If several processes attempt to open a read-only file at the same time, they can be granted simultaneous access to the file.

\paragraph{Hold and Wait}\label{par:Deadlock_Prevention-Hold_Wait}
To ensure that the hold-and-wait condition never occurs in the system, we must guarantee that, whenever a process requests a resource, it does not hold any other resources.

We can implement this provision by requiring that system calls requesting resources for a process precede all other system calls.
An alternative protocol allows a process to request resources only when it has none.

Both these protocols have two main disadvantages.
First, resource utilization may be low, since resources may be allocated but unused for a long period.
Second, \nameref{def:Starvation} is possible.
A process that needs several popular resources may have to wait indefinitely, because at least one of the resources that it needs is always allocated to some other process.

\paragraph{No Preemption}\label{par:Deadlock_Prevention-No_Preemption}
To ensure that no \nameref{def:Preemption} of resources that have already been allocated occurs, we can use the following protocol.
If a process is holding some resources and requests another resource that cannot be immediately allocated to it (that is, the process must wait), then all resources the process is currently holding are preempted, implicitly releasing them.
The preempted resources are added to the list of resources for which the process is waiting.
The process will be restarted only when it can regain its old resources, as well as the new ones that it is requesting.

This protocol is often applied to resources whose state can be easily saved and restored later, such as CPU registers and memory space.
It cannot generally be applied to such resources as mutex locks and semaphores.

\paragraph{Circular Wait}\label{par:Deadlock_Prevention-Circular_Wait}
One way to ensure that this condition never holds is to impose a total ordering of all resource types and to require that each process requests resources in an increasing order of enumeration.
Possible side effects of preventing deadlocks are low device utilization and reduced system throughput.

Formally, we define a one-to-one function
\begin{equation*}
F: R \rightarrow \NaturalNumbers
\end{equation*}
where $R$ is the set of resources and  $\NaturalNumbers$ is the set of natural numbers.

Each process can request resources only in an increasing order of enumeration.
Meaning, a process can initially request any number of instances of a resource type, $R_{i}$.
After that, the process can request instances of resource type $R_{j}$ if and only if $F(R_{j}) > F(R_{i})$.
This could be similarly defined using a more generous comparison; a process can request instances of resource type $R_{j}$ if and only if $F(R_{j}) \geq F(R_{i})$.

\begin{blackbox}
  Keep in mind that developing an ordering, or hierarchy, does not in itself prevent deadlock.
  It is up to application developers to write programs that follow the ordering.
\end{blackbox}

\subsubsection{Deadlock Avoidance}\label{subsubsec:Deadlock_Avoidance}
\subsubsection{Deadlock Detection}\label{subsubsec:Deadlock_Detection}

%%% Local Variables:
%%% mode: latex
%%% TeX-master: "../../EDAF35-Operating_Systems-Reference_Sheet"
%%% End:
