\subsection{Handling Deadlocks}\label{subsec:Handling_Deadlocks}
There are 3 main ways to handle a deadlock:
\begin{enumerate}[noitemsep]
\item We can use a protocol to prevent (\Cref{subsubsec:Deadlock_Prevention}) or avoid (\Cref{subsubsec:Deadlock_Avoidance}) deadlocks, ensuring that the system will never enter a deadlocked state.
\item We can allow the system to enter a deadlocked state, detect it, and recover (\Cref{subsubsec:Deadlock_Detection}).
\item We can ignore the problem altogether and pretend that deadlocks never occur in the system.
\end{enumerate}

The third method of handling a \nameref{def:Deadlock} may not make sense, but from a cost perspective, it does.
Ignoring the possibility of deadlocks is cheaper.
In many systems, deadlocks occur infrequently (once per year), the extra expense of the methods may not seem worthwhile.
In addition, methods used to recover from other conditions may be put to use to recover from deadlock.
In some circumstances, a system is in a frozen state but not in a deadlocked state.

\subsubsection{Deadlock Prevention}\label{subsubsec:Deadlock_Prevention}
\begin{definition}[Deadlock Prevention]\label{def:Deadlock_Prevention}
  \emph{Deadlock prevention} is done by providing a set of methods that ensure any one of the \nameref{subsubsec:Deadlock_Conditions} cannot occur.
\end{definition}

\paragraph{Mutual Exclusion}\label{par:Deadlock_Prevention-Mutual_Exclusion}
That is, at least one resource must be nonsharable.
Sharable resources, in contrast, do not require mutually exclusive access and thus cannot be involved in a deadlock.
A process never needs to wait for a sharable resource.

Read-only files are a good example of a sharable resource.
If several processes attempt to open a read-only file at the same time, they can be granted simultaneous access to the file.

\paragraph{Hold and Wait}\label{par:Deadlock_Prevention-Hold_Wait}
To ensure that the hold-and-wait condition never occurs in the system, we must guarantee that, whenever a process requests a resource, it does not hold any other resources.

We can implement this provision by requiring that system calls requesting resources for a process precede all other system calls.
An alternative protocol allows a process to request resources only when it has none.

Both these protocols have two main disadvantages.
First, resource utilization may be low, since resources may be allocated but unused for a long period.
Second, \nameref{def:Starvation} is possible.
A process that needs several popular resources may have to wait indefinitely, because at least one of the resources that it needs is always allocated to some other process.

\subsubsection{Deadlock Avoidance}\label{subsubsec:Deadlock_Avoidance}
\subsubsection{Deadlock Detection}\label{subsubsec:Deadlock_Detection}

%%% Local Variables:
%%% mode: latex
%%% TeX-master: "../../EDAF35-Operating_Systems-Reference_Sheet"
%%% End:
