\subsection{Real-Time Scheduling}\label{subsec:Real_Time_Scheduling}
Real-time operating systems have their own class of scheduling issues.
This depends on whether the \nameref{def:Operating_System} is a \nameref{def:Soft_Real_Time_System} or a \nameref{def:Hard_Real_Time_System}.

\begin{definition}[Soft Real-Time System]\label{def:Soft_Real_Time_System}
  \emph{Soft real-time system}s do not provide a guarantee about the scheduling of a critical real-time process.
\end{definition}

\begin{definition}[Hard Real-Time System]\label{def:Hard_Real_Time_System}
  \emph{Hard real-time system}s guarantee the execution time of a real-time process.
  These tasks will be serviced by its deadline, otherwise the process will not be executed at all.
\end{definition}

POSIX also provides support for real-time scheduling through 2 functions with 2 scheduling types.
\begin{enumerate}[noitemsep]
\item \kernelinline{pthread_attr_getsched_policy(pthread_attr_t *attr, int *policy)}
\item \kernelinline{pthread_attr_setsched_policy(pthread_attr_t *attr, int policy)}
\end{enumerate}
\begin{enumerate}[noitemsep]
\item \texttt{SCHED\_FIFO}
\item \texttt{SCHED\_RR}
\end{enumerate}

\subsubsection{Minimizing Latency}\label{subsubsec:Minimizing_Latency}
The key aspect here is the amount of time it takes for a system to respond to an event.
This is called \nameref{def:Event_Latency}.

\begin{definition}[Event Latency]\label{def:Event_Latency}
  \emph{Event latency} is the amount of time that elapses from when an event occurs to when it is serviced.
  Different events can have different event latency requirements.
\end{definition}

There are 2 factors that affect \nameref{def:Event_Latency}.
\begin{enumerate}[noitemsep]
\item Interrupt Latency.
  The amount of time from the arrival of an \nameref{def:Interrupt} to the start of the Interrupt Service Routine (ISR).
  This includes the amount of time needed to get the currently running instruction to a point where it can be switched.
  Also included is the amount of time needed to perform the switch.
\item Dispatch Latency the amount of time the scheduler needs to stop one process and start another.
  There are 2 parts that affect the value of the dispatch latency:
  \begin{enumerate}[noitemsep]
  \item \nameref{def:Preemption} of \textbf{ANY} process running tin the kernel.
  \item Release of resources used by low-priority process for higher-priority processes.
  \end{enumerate}
\end{enumerate}

\subsubsection{Scheduling}\label{subsubsec:Real_Time_Scheduling}
In this case, there are not as many choices of \nameref{def:Scheduling_Algorithm} for real-time systems as other systems.
All algorithms must be roughly based on a priority-based system all of which must support \nameref{def:Preemption}.

Most modern \nameref{def:Operating_System}s offer support for \nameref{def:Soft_Real_Time_System}s with their scheduling priorities.
Note however, that pure priority-based algorithms only guarantee soft real-time functionality, not hard.

%%% Local Variables:
%%% mode: latex
%%% TeX-master: "../../EDAF35-Operating_Systems-Reference_Sheet"
%%% End:
