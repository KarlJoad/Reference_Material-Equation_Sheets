\subsection{Real-Time Scheduling}\label{subsec:Real_Time_Scheduling}
Real-time operating systems have their own class of scheduling issues.
This depends on whether the \nameref{def:Operating_System} is a \nameref{def:Soft_Real_Time_System} or a \nameref{def:Hard_Real_Time_System}.

\begin{definition}[Soft Real-Time System]\label{def:Soft_Real_Time_System}
  \emph{Soft real-time system}s do not provide a guarantee about the scheduling of a critical real-time process.
\end{definition}

\begin{definition}[Hard Real-Time System]\label{def:Hard_Real_Time_System}
  \emph{Hard real-time system}s guarantee the execution time of a real-time process.
  These tasks will be serviced by its deadline, otherwise the process will not be executed at all.
\end{definition}

POSIX also provides support for real-time scheduling through 2 functions with 2 scheduling types.
\begin{enumerate}[noitemsep]
\item \kernelinline{pthread_attr_getsched_policy(pthread_attr_t *attr, int *policy)}
\item \kernelinline{pthread_attr_setsched_policy(pthread_attr_t *attr, int policy)}
\end{enumerate}
\begin{enumerate}[noitemsep]
\item \texttt{SCHED\_FIFO}
\item \texttt{SCHED\_RR}
\end{enumerate}

\subsubsection{Minimizing Latency}\label{subsubsec:Minimizing_Latency}
The key aspect here is the amount of time it takes for a system to respond to an event.
This is called \nameref{def:Event_Latency}.


%%% Local Variables:
%%% mode: latex
%%% TeX-master: "../../EDAF35-Operating_Systems-Reference_Sheet"
%%% End:
