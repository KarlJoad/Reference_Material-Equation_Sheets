\subsection{Thread Scheduling}\label{subsec:Thread_Scheduling}
\begin{blackbox}
  \textbf{On operating systems that support them, \nameref{def:Kernel_Thread}s, not \nameref{def:Process}es are scheduled by the operating system.}
  However, the terms ``process scheduling'' and ``thread scheduling'' are often used interchangeably.
  Process scheduling is used when discussing general scheduling concepts and thread scheduling to refer to thread-specific ideas.
\end{blackbox}

\subsubsection{User- vs. Kernel-Level Thread Scheduling}\label{subsubsec:User_vs_Kernel_Thread_Scheduling}
On systems implementing the \nameref{subsubsec:Many_To_One_Model} (\Cref{subsubsec:Many_To_One_Model}) and \nameref{subsubsec:Many_To_Many_Model} (\Cref{subsubsec:Many_To_Many_Model}), the thread library schedules user-level threads to run on an available LWP.\@

\begin{definition}[Process-Contention Scope]\label{def:Process_Contention_Scope}
  \emph{Process-Contention Scope} is where each of the \nameref{def:Thread}s \textbf{WITHIN A SINGLE \nameref{def:Process}} compete with each other for execution.
\end{definition}

In \nameref{def:Process_Contention_Scope}, the \nameref{def:Thread_Library} schedules the \nameref{def:User_Thread}s onto available \nameref{def:Lightweight_Process}es, or directly onto \nameref{def:Kernel_Thread}s.
However, this does not mean that the \nameref{def:User_Thread} are being scheduled for execution.
For that to happen, the \nameref{def:Kernel} must be involved.

To decide what \nameref{def:Thread}s can execute on the CPU, we broaden our view of contention to \nameref{def:System_Contention_Scope}.

\begin{definition}[System-Contention Scope]\label{def:System_Contention_Scope}
  \emph{System-Contention Scope} is where \textbf{ALL} of the \nameref{def:Thread}s on the system, from all \nameref{def:Process}es compete with each other for execution.
\end{definition}


%%% Local Variables:
%%% mode: latex
%%% TeX-master: "../../EDAF35-Operating_Systems-Reference_Sheet"
%%% End:
