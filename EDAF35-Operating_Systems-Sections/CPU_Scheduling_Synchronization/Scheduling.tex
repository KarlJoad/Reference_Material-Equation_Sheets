\subsection{Scheduling}\label{subsec:Scheduling}
CPU scheduling is the basis of multiprogrammed operating systems.
In a single-processor system, only one process can run at a time.
Others must wait until the CPU is free and can be rescheduled.
By switching the CPU among processes, the operating system can maximize CPU utilization.

For example, a \nameref{def:Process} is executed until it must wait.
Typically the process waits for the completion of some I/O request.
In a simple computer system, the CPU then just sits idle.
All this waiting time is wasted; no useful work is accomplished.

\begin{blackbox}
  \textbf{On operating systems that support them, \nameref{def:Kernel_Thread}s, not \nameref{def:Process}es are scheduled by the operating system.}
  However, the terms ``process scheduling'' and ``thread scheduling'' are often used interchangeably.
  Process scheduling is used when discussing general scheduling concepts and thread scheduling to refer to thread-specific ideas.
\end{blackbox}

Scheduling of this kind is a fundamental operating-system function.
Almost all computer resources are scheduled before use.


%%% Local Variables:
%%% mode: latex
%%% TeX-master: "../../EDAF35-Operating_Systems-Reference_Sheet"
%%% End:
