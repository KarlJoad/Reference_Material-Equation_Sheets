\subsection{Deadlocks}\label{subsec:Deadlocks}
\nameref{def:Deadlock} is a serious issue in \nameref{rmk:CPU_Scheduler}s because \nameref{def:Process}es lock resources for themselves.
A good example of a deadlock is ``When two trains approach each other at a crossing, both shall come to a full stop and neither shall start up again until the other has gone.''

\begin{definition}[Deadlock]\label{def:Deadlock}
  \emph{Deadlock} is when 2 processes require information or resources from each other to continue running.
  If this happens, neither process will provide the other with its required information, so they will both wait for each other, forever.
\end{definition}

There are only 2 options for handling \nameref{def:Deadlock}s:
\begin{enumerate}[noitemsep]
\item Prevent them from happening in the first place.
\item Identify them and fix the problem that is causing them.
\item Hope they don't happen and consider them as unlikely events to occur.
  \begin{itemize}[noitemsep]
  \item This is what most desktop \nameref{def:Operating_System}s do.
  \end{itemize}
\end{enumerate}

%%% Local Variables:
%%% mode: latex
%%% TeX-master: "../../EDAF35-Operating_Systems-Reference_Sheet"
%%% End:
