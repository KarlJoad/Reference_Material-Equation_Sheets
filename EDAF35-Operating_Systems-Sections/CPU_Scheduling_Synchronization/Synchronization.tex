\subsection{Process/Thread Synchronization}\label{subsec:Synchronization}
The main problem that occurs in multi\nameref{def:Thread}ed programs is that there is a small portion of code that is a \nameref{def:Critical_Section}.
This leads to the development of the \nameref{subsubsec:Critical_Section_Problem}.

\begin{definition}[Critical Section]\label{def:Critical_Section}
  The \emph{critical section} of a \nameref{def:Process} is a portion where the \nameref{def:Thread} and/or \nameref{def:Process} is changing common variables, updating a table, writing a file, or other global state changes.
\end{definition}

\subsubsection{Critical Section Problem}\label{subsubsec:Critical_Section_Problem}
The \emph{Critical Section Problem} is the issue of coordinating multiple \nameref{def:Thread}s about a \nameref{def:Critical_Section} of the code.
The problem is to design a protocol that the \nameref{def:Process}es/\nameref{def:Thread}s can use to cooperate.
Each \nameref{def:Process} must request permission to enter its critical section.
The section of code implementing this request is the \nameref{def:Entry_Section}.
The critical section may be followed by an \nameref{def:Exit_Section}.
The remaining code is the \nameref{def:Remainder_Section}.

\begin{definition}[Entry Section]\label{def:Entry_Section}
  The \emph{entry section} of a \nameref{def:Process} is the portion where the request to execute the \nameref{def:Critical_Section} occurs.
  In the case of a \nameref{def:Mutex}, this is the process of aquiring the it.
  For a \nameref{def:Semaphore}, it is the process of manipulating the value it currently contains.
\end{definition}

\begin{definition}[Exit Section]\label{def:Exit_Section}
  In the \emph{exit section}, the constructs used to ensure coordination in the \nameref{def:Critical_Section} are freed.
  In the case of a \nameref{def:Mutex}, this is the process of releasing the it.
  For a \nameref{def:Semaphore}, it is the process of manipulating the value it currently contains in the opposite direction it was initially manipulated by.
\end{definition}

\begin{definition}[Remainder Section]\label{def:Remainder_Section}
  The \emph{remainder section} is the rest of the code, after this \nameref{def:Critical_Section}.
  This code may be parallelized, or not.
  It could contain further \nameref{def:Critical_Section}s.
\end{definition}

Any solution to this problem \textbf{MUST} satisfy one of the following 3 requirements:
\begin{enumerate}[noitemsep]
\item \textbf{Mutual Exclusion}.
  If \nameref{def:Process} $P_{i}$ is executing its \nameref{def:Critical_Section}, then \textbf{no other} processes can execute their critical sections.
\item \textbf{Progress}.
  If no \nameref{def:Process} is executing its \nameref{def:Critical_Section}, and some processes wish to enter their critical sections, then only those processes that \textbf{are not executing} in their \nameref{def:Remainder_Section}s can decide which will enter the \nameref{def:Critical_Section} next.
  Essentially, the only way a process gets a voice in the choice is by not having executed the critical section yet.
\item \textbf{Bounded Waiting}.
  There exists a bound on the number of times that other \nameref{def:Process}es are allowed to enter their \nameref{def:Critical_Section}s after a process has made a request to enter its critical section and before that request is granted.
\end{enumerate}

To handle the \nameref{subsubsec:Critical_Section_Problem}, there are 2 main types of \nameref{def:Kernel}s that present solutions.
\begin{enumerate}[noitemsep]
\item \nameref{def:Nonpreemptive_Kernel}s. Not used frequently today.
\item \nameref{def:Preemptive_Kernel}s. The most common type today.
\end{enumerate}

\begin{definition}[Nonpreemptive Kernel]\label{def:Nonpreemptive_Kernel}
  A \emph{nonpreemptive kernel} is a \nameref{def:Kernel} that does \textbf{NOT} use \nameref{def:Preemption} on \nameref{def:Process}es or \nameref{def:Thread}s running in kernel-mode.
\end{definition}

\begin{definition}[Preemptive Kernel]\label{def:Preemptive_Kernel}
  A \emph{preemptive kernel} is a \nameref{def:Kernel} that uses \nameref{def:Preemption} on \nameref{def:Process}es or \nameref{def:Thread}s running in kernel-mode.
  This means that we cannot say anything definitive about the state of the \nameref{def:Kernel}'s data structures at a given time, because we cannot say which process/thread is running at that time.
\end{definition}

\subsubsection{Hardware Support for Synchronization}\label{subsubsec:Hardware_Support_Synchronization}
Software-based solutions to handling multithreading and multiprocessing tends to be better than hardware-based solutions, as they are more flexible.
Many of the solutions that will be presented here are based on the idea of \textbf{\nameref{def:Lock}ing}.

\begin{definition}[Lock]\label{def:Lock}
  A \emph{lock} allows \textbf{only one} \nameref{def:Thread} to enter the portion of code that is locked.
  While a thread holds this lock no other \nameref{def:Thread} can execute on this code portion.

  \begin{remark}[Binary Semaphore]\label{rmk:Binary_Semaphore}
    Locks can be represented as \emph{binary \nameref{def:Semaphore}}s.
  \end{remark}
\end{definition}

In a single-processor system, we can solve the \nameref{subsubsec:Critical_Section_Problem} by preventing interrupts from being handled.
This would prevent the currently running instruction from being preempted in any way, and allow it to finish.
However, this does not really work on a multiprocessor system, because disabling interrupts and their handling on all processors is time consuming.

However, the idea of certain instructions being \nameref{def:Atomic} is an elegant solution to the \nameref{subsubsec:Critical_Section_Problem}.
So, most computer systems provide special hardware-level instructions that allow us to test and modify the contents of a word, or swap the contents of 2 words \nameref{def:Atomic}ally.

\begin{definition}[Atomic]\label{def:Atomic}
  An \emph{atomic} operation is one that cannot be interrupted, preempted, or altered in any way.
  As soon as an atomic operation begins, the system \textbf{MUST} finish handling it before it may do anything else.
\end{definition}

Some operations on data are possible to do at any given point in time, without affecting the potential outcome.
One example of this is \textbf{reading} from a location in memory.
However, if this location can also be written to, we need to limit the number of writers.
Additionally, if someone is waiting to write, they should get some priority over anything waiting to read.
Thus, the \nameref{def:Read_Write_Lock} was created.

\begin{definition}[Read/Write Lock]\label{def:Read_Write_Lock}
  \emph{Read/Write Lock}s allow either an unlimited number of readers \textbf{OR} 1 writer at any given time.
  Writers will be scheduled to use the lock sooner than readers, so the value is updated first, before anyone reads it again.
  But, the writer will have to wait until everyone currently reading the value is done reading, otherwise the value in memory will change underneath the readers.
\end{definition}

\subsubsection{Mutex Locks}\label{subsubsec:Mutex_Locks}
The hardware-based solutions presented in \Cref{subsubsec:Hardware_Support_Synchronization} are typically not available to application programmers.
Instead, operating system designers build software tools to handle the \nameref{subsubsec:Critical_Section_Problem}.
The simplest tool is that of a \nameref{def:Mutex}.

\begin{definition}[Mutex]\label{def:Mutex}
  A \emph{mutex} (short for \emph{\textbf{mut}ual \textbf{ex}clusion}) is the same as a \nameref{def:Lock}, \textbf{but it can be system wide (shared by multiple processes)}.
  A mutex lock protects critical regions and prevents race conditions.
  That is, a process must acquire the lock before entering a critical section; it releases the lock when it exits the critical section.

  The \kernelinline{acquire()} function acquires the lock, preventing any other \nameref{def:Thread} and/or \nameref{def:Process} from using the thing the lock protects.
  Likewise, the \kernelinline{release()} function releases the lock, allowing another \nameref{def:Thread} and/or \nameref{def:Process} take acquire the lock and use the resource it protects.
  To perform its function correctly, the lock's \kernelinline{acquire()} and \kernelinline{release()} functions must be \nameref{def:Atomic}.

  When a \nameref{def:Thread} and/or \nameref{def:Process} attempts \texttt{acquire()} the lock, while it is already owned by someone else, it is put in a \texttt{WAITING} state.
\end{definition}

%%% Local Variables:
%%% mode: latex
%%% TeX-master: "../../EDAF35-Operating_Systems-Reference_Sheet"
%%% End:
