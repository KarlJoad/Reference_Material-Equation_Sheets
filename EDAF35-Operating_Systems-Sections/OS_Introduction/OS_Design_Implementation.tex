\subsection{Operating System Design and Implementation}\label{subsec:OS_Design_Implementation}
One important principle is the separation of \nameref{def:Policy} from \nameref{def:Mechanism}.
\begin{definition}[Mechanism]\label{def:Mechanism}
  A \emph{mechanism} determines how to do something.
\end{definition}

\begin{definition}[Policy]\label{def:Policy}
  A \emph{policy} determines \textbf{what} will be done given the \nameref{def:Mechanism} works correctly.
\end{definition}

The separation of \nameref{def:Policy} and \nameref{def:Mechanism} is important for system flexibility.
Policies are likely to change across places or over time.
In the worst case, each change in policy would require a change in the underlying mechanism.
A general mechanism insensitive to changes in policy would be more desirable.
A change in policy would then require redefinition of only certain parameters of the system.
For instance, consider a mechanism for giving priority to certain types of programs over others.
If the mechanism is properly separated from policy, it can be used either to support a policy decision that I/O-intensive programs should have priority over CPU-intensive ones or to support the opposite policy.

Most \nameref{def:Operating_System}s were built with assembly.
However, in recent times (since the invention of C), they have been built with higher-level languages.
The advantages of using a higher-level language, or at least a systems-implementation language, for implementing operating systems are the same as those gained when the language is used for application programs:
\begin{itemize}[noitemsep]
\item The code can be written faster
\item Is more compact
\item Is easier to understand and debug
\end{itemize}

In addition, improvements in compiler technology will improve the generated code for the entire operating system by simple recompilation.
Finally, an \nameref{def:Operating_System} is far easier to port—to move to some other hardware —
if it is written in a higher-level language.

\begin{definition}[Port]\label{def:Software_Port}
  A \emph{port} is the process of moving a piece of software that was written for one piece of \nameref{def:Hardware} to another.
  In some cases, this only requires a recompilation of the higher-level software.
  In others, it may require completely rewriting the program.

  \begin{remark}[Port Confusion]\label{rmk:Software_Port_Confusion}
    It is important to note that the \nameref{def:Software_Port} is \textbf{\emph{NOT}} the same thing as a \nameref{def:Network_Port}.
  \end{remark}
\end{definition}

%%% Local Variables:
%%% mode: latex
%%% TeX-master: "../../EDAF35-Operating_Systems-Reference_Sheet"
%%% End:
