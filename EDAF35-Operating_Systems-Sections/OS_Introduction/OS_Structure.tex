\subsection{Operating System Structure}\label{subsec:OS_Structure}
A system as large and complex as a modern operating system must be engineered carefully if it is to function properly and be modified easily.

\subsubsection{Monolithic Approach}\label{subsubsec:Monolithic_Approach}
Many operating systems do not have well-defined structures.
Frequently, such systems started as small, simple, and limited systems and then grew beyond their original scope.
MS-DOS is an example of such a system.
It was originally designed and implemented by a few people who had no idea that it would become so popular.
It was written to provide the most functionality in the least space, so it was not carefully divided into modules.

\begin{definition}[Monolithic Kernel]\label{def:Monolithic_Kernel}
  A \emph{monolithic kernel} is an \nameref{def:Operating_System} architecture where the entire operating system is working in \nameref{def:Kernel} space, and typically uses only its own memory space to run.
  The monolithic model differs from other operating system architectures (such as the \nameref{def:Microkernel}) in that it alone defines a high-level virtual interface over computer hardware.
  A set of \nameref{def:System_Call}s implement all \nameref{def:Operating_System} services such as process management, concurrency, and memory management.

  Device drivers can be added to the \nameref{def:Kernel} as \nameref{def:Kernel_Module}s.
\end{definition}

In MS-DOS, the interfaces and levels of functionality are not well separated.
For instance, application programs are able to access the basic I/O routines to write directly to the display and disk drives.
Such freedom leaves MS-DOS vulnerable to errant (or malicious) programs, causing entire system crashes when user programs fail.

However, this was partly because MS-DOS was also limited by the hardware of its era.
Because the Intel~8088 for which it was written provides no dual mode and no hardware protection, the designers of MS-DOS had no choice but to leave the base hardware accessible.

\subsubsection{Layered Approach}\label{subsubsec:Layered_Approach}
With proper hardware support, \nameref{def:Operating_System}s can be broken into pieces that are smaller and more appropriate than those allowed by the original MS-DOS and UNIX systems.
The \nameref{def:Operating_System} can then retain much greater control over the computer and over the applications that make use of that computer.
Implementers have more freedom in changing the inner workings of the system and in creating modular \nameref{def:Operating_System}s.
Under a top-down approach, the overall functionality and features are determined and are separated into components.
Information hiding is also important, because it leaves programmers free to implement the low-level routines as they see fit, provided that the external interface of the routine stays unchanged and that the routine itself performs the advertised task.

A system can be made modular in many ways.
One method is the layered approach, in which the \nameref{def:Operating_System} is broken into a number of layers.
The bottom layer (layer 0) is the hardware; the highest (layer N) is the user interface.

A typical operating-system layer, layer $M$ consists of data structures and a set of routines that can be invoked by higher-level layers.
Layer $M$, in turn, can \textbf{\emph{ONLY}} invoke operations on lower-level layers and itself.

The main advantage of the layered approach is simplicity of construction and debugging.
The layers are selected so that each uses functions and services of only lower-level layers.
This approach simplifies debugging and system verification.
The first layer can be debugged without any concern for the rest of the system.
Once the first layer is debugged, its correct functioning can be assumed while the second layer is debugged, and so on.
If an error is found during the debugging of a particular layer, the error must be on that layer, because the layers below it are already debugged.
Thus, the design and implementation of the system are simplified.
Each layer is implemented only with operations provided by lower-level layers.
A layer does not need to know how these operations are implemented; it needs to know only what these operations do.

The major difficulty with the layered approach involves appropriately defining the various layers.
Because a layer can use only lower-level layers, careful planning is necessary.
Even with planning, there can be circular dependencies created between layers.
For example, the backing-store driver would normally be above the CPU scheduler, because the driver may need to wait for I/O and the CPU can be rescheduled during this time.
However, the CPU scheduler may have more information about all the active processes than can fit in memory.
Therefore, this information may need to be swapped in and out of memory, requiring the backing-store driver routine to be below the CPU scheduler.

A final problem with layered implementations is that they tend to be less efficient than other types.

\subsubsection{Microkernels}\label{subsubsec:Microkernels}
This method structures the \nameref{def:Operating_System} by removing all nonessential components from the \nameref{def:Kernel} and implementing them as system and user-level programs, resulting in a smaller \nameref{def:Kernel}.
There is little consensus regarding which services should remain in the kernel and which should be implemented in user space.
Typically, however, microkernels provide minimal process and memory management, in addition to a communication facility.

\begin{definition}[Microkernel]\label{def:Microkernel}
  A \emph{microkernel} (often abbreviated as $\mu$-kernel) is the near-minimum amount of software that can provide the mechanisms needed to implement an \nameref{def:Operating_System}.
  These mechanisms include:
  \begin{itemize}[noitemsep]
  \item Low-level address space management
  \item Thread management
  \item Inter-Process Communication (IPC)
  \end{itemize}

  If the hardware provides multiple rings or CPU modes, the microkernel may be the only software executing at the most privileged level, which is generally referred to as supervisor or kernel mode.
  Traditional \nameref{def:Operating_System} functions, such as device drivers, protocol stacks and file systems, are typically removed from the microkernel itself and are instead run in user space.

  In terms of the source code size, microkernels are often smaller than monolithic kernels.
\end{definition}

The main function of the \nameref{def:Microkernel} is to provide communication between the client program and the various services that are also running in user space.
Communication is provided through \nameref{par:Message_Passing}.

One benefit of the \nameref{def:Microkernel} approach is that it makes extending the \nameref{def:Operating_System} easier.
All new services are added to user space and consequently do not require modification of the \nameref{def:Kernel}.
When the \nameref{def:Kernel} does have to be modified, the changes tend to be fewer, because the \nameref{def:Microkernel} is smaller.
The resulting \nameref{def:Operating_System} is easier to port from one hardware design to another.
The \nameref{def:Microkernel} also provides more security and reliability, since most services are running as \nameref{def:User}— rather than \nameref{def:Kernel}—processes.
If a service fails, the rest of the \nameref{def:Operating_System} remains untouched.

\subsubsection{Kernel Modules}\label{subsubsec:Kernel_Modules}
\subsubsection{Hybrid Systems}\label{subsubsec:Hybrid_Systems}

%%% Local Variables:
%%% mode: latex
%%% TeX-master: "../../EDAF35-Operating_Systems-Reference_Sheet"
%%% End:
