\subsection{System Boot}\label{subsec:System_Boot}
The procedure of starting a computer by loading the \nameref{def:Kernel} is known as booting the system.
On most computer systems, a small piece of code known as the \nameref{def:Bootloader} is the first thing that runs.

\begin{definition}[Bootloader]\label{def:Bootloader}
  The \emph{bootloader} (or bootstrap loader) is a bootstrap program that:
  \begin{enumerate}[noitemsep]
  \item Locates the kernel
  \item Loads it into main memory
  \item Starts its execution
  \end{enumerate}
\end{definition}

Some computer systems, such as PCs, use a two-step process in which a simple \nameref{def:Bootloader} fetches a more complex boot program from disk, which in turn loads the \nameref{def:Kernel}.

When a CPU receives a reset event, the instruction register is loaded with a predefined memory location, and execution starts there.
At that location is the initial \nameref{def:Bootloader} program.
This program is in the form of read-only memory (ROM), because the RAM is in an unknown state at system startup.
ROM is convenient because it needs no initialization and cannot easily be infected by a computer virus.

\begin{remark*}
  A reset event on the CPU can be the computer having just booted, or it has been restarted, or the reset switched was flipped.
\end{remark*}

The \nameref{def:Bootloader} can perform a variety of tasks.
Usually, one task is to run diagnostics to determine the state of the machine.
If the diagnostics pass, the program can continue with the booting steps.
It also initializes all aspects of the system, from CPU registers to device controllers and the contents of main memory.
Sooner or later, it starts the \nameref{def:Operating_System}.
Cellular phones, tablets, and game consoles store the entire operating system in ROM.\@
Storing the operating system in ROM is suitable only for:
\begin{itemize}[noitemsep]
\item Small operating systems
\item Simple supporting hardware
\item Ensuring rugged operation
\end{itemize}

A problem with this approach is that changing the bootstrap code requires changing the ROM hardware chips.

All forms of ROM are also known as \nameref{def:Firmware}.
A problem with firmware in general is that executing code there is slower than executing code in RAM.\@
Some systems store the \nameref{def:Operating_System} in firmware and copy it to RAM for fast execution.

A final issue with \nameref{def:Firmware} is that it is relatively expensive, so usually only small amounts are available.
For large operating systems, or for systems that change frequently, the \nameref{def:Bootloader} is stored in \nameref{def:Firmware}, and the \nameref{def:Operating_System} is on disk.
In this case, the bootstrap runs diagnostics and has a bit of code that can read a single block at a fixed location (say block zero) from disk into memory and execute the code from that boot block.
The program stored in the boot block may be sophisticated enough to load the entire operating system into memory and begin its execution.

More typically, it is simple code (as it fits in a single disk block) and knows only the address on disk and length of the remainder of the bootstrap program.
GRUB is an example of an open-source \nameref{def:Bootloader} program for Linux systems.
All of the disk-bound bootstrap, and the \nameref{def:Operating_System} that is loaded, can be easily changed by writing new versions to disk.
A disk that has a boot partition is called a boot disk or system disk.
Now that the full bootstrap program has been loaded, it can traverse the file system to find the \nameref{def:Operating_System}'s \nameref{def:Kernel}, load it into \nameref{def:Memory}, and start its execution.
It is only at this point that the system is said to be running.

%%% Local Variables:
%%% mode: latex
%%% TeX-master: "../../EDAF35-Operating_Systems-Reference_Sheet"
%%% End:
