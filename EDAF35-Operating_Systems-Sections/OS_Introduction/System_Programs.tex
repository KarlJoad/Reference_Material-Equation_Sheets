\subsection{System Programs}\label{subsec:System_Programs}
Another aspect of a modern system is its collection of system programs.
\begin{definition}[System Program]\label{def:System_Program}
  \emph{System programs}, also known as \emph{system utilities}, provide a convenient environment for program development and execution.
  Some of them are simply user interfaces to system calls.
  Others are considerably more complex.
  They can be divided into these categories:
  \begin{description}
  \item[File Management] These programs create, delete, copy, rename, print, dump, list, and generally manipulate files and directories.
  \item[Status Information] Some programs simply ask the system for the date, time, amount of available memory or disk space, number of users, or similar status information.
    Others are more complex, providing detailed performance, logging, and debugging information.
    Typically, these programs format and print the output to the terminal or other output devices or files or display it in a window of the GUI.\@
    Some systems also support a registry, which is used to store and retrieve configuration information.
  \item[File Modification] Several text editors may be available to create and modify the content of files stored on disk or other storage devices.
    There may also be special commands to search contents of files or perform transformations of the text.
  \item[Programming-Language Support] Compilers, assemblers, debuggers, and interpreters for common programming languages (such as C, C++, Java, and PERL) are often provided with the operating system or available as a separate download.
  \item[Program Loading and Execution] Once a program is assembled or compiled, it must be loaded into memory to be executed.
    Debugging systems for either higher-level languages or machine language are needed as well.
  \item[Communications] These programs provide the mechanism for creating virtual connections among processes, users, and computer systems.
    They allow users to send messages to one another’s screens, to browse Web pages, to send e-mail messages, to log in remotely, or to transfer files from one machine to another.
  \item[Background Services] All general-purpose systems have methods for launching certain \nameref{def:System_Program} processes at boot time.
    Some of these processes terminate after completing their tasks, while others continue to run until the system is halted.
    These are typically called \nameref{def:Daemon}s, and systems have dozens of them.
    In addition, operating systems that run important activities in user context rather than in kernel context may use \nameref{def:Daemon}s to run these activities.
  \end{description}
\end{definition}

%%% Local Variables:
%%% mode: latex
%%% TeX-master: "../../EDAF35-Operating_Systems-Reference_Sheet"
%%% End:
