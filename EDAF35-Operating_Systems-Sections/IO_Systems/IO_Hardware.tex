\subsection{I/O Hardware}\label{subsec:IO_Hardware}
The device communicates with the machine via a connection point, a port.
If devices share a common set of wires, the connection is called a \nameref{def:Hardware_Bus}.
Buses are used widely in computer architecture and vary in their signaling methods, speed, throughput, and connection methods.

\begin{definition}[Bus]\label{def:Hardware_Bus}
  A \emph{bus} is a set of wires and a rigidly defined protocol that specifies a set of messages that can be sent on the wires.
\end{definition}

When device A plugs into device B, and device B plugs into device C, and device C plugs into a port on the computer, this arrangement is called a daisy chain.
A daisy chain usually operates as a \nameref{def:Hardware_Bus}.

All of these may be controlled by a \nameref{def:Controller}.

\begin{definition}[Controller]\label{def:Controller}
  A \emph{controller} is a collection of electronics that can operate a port, a \nameref{def:Hardware_Bus}, or a device.
\end{definition}

Because some protocols are complex, their \nameref{def:Hardware_Bus} \nameref{def:Controller} is often implemented as a separate circuit board (or a host adapter) that plugs into the computer.
It typically contains a processor, microcode, and some private memory to enable it to process the protocol messages.
Some devices have their own built-in controllers.
They can receive commands in 2 ways (some support both, some do not):
\begin{enumerate}[noitemsep]
\item The \nameref{def:Controller} has one or more registers for data and control signals.
  The processor communicates with the controller by reading and writing bit patterns in these registers.
  This communication can occur through the use of special I/O instructions that specify the transfer of a byte or word to an I/O port address.
  The I/O instruction triggers bus lines to select the proper device and to move bits into or out of a device register.
\item The device \nameref{def:Controller} can support \nameref{subsubsec:Memory_Mapped_IO}.
  In this case, the device-control registers are mapped into the address space of the processor.
  The CPU executes I/O requests using the standard data-transfer instructions to read and write the device-control registers at their mapped locations in physical memory.
\end{enumerate}

For example, a graphics controller supports both of these methods of control.
\begin{itemize}[noitemsep]
\item The graphics controller has I/O ports for basic control operations.
\item The controller has a large memory-mapped region to hold screen contents.
  The process sends output to the screen by writing data into the memory-mapped region.
  The controller generates the screen image based on the contents of this memory.
\end{itemize}

The ease of writing to a \nameref{subsubsec:Memory_Mapped_IO} \nameref{def:Controller} is offset by the disadvantage of being vulnerable to accidental modification by an incorrect pointer to an unintended region of memory.
Memory protection helps reduce this risk.

\subsubsection{I/O Ports}\label{subsubsec:IO_Ports}
An I/O port typically consists of four registers, called the status, control, data-in, and data-out registers.
\begin{enumerate}[noitemsep]
\item The \textbf{\texttt{data-in} register} is read by the host to get input.
\item The \textbf{\texttt{data-out} register} is written by the host to send output.
\item The \textbf{\texttt{status} register} contains bits that can be \textbf{read by the host}.
  These bits indicate states, such as:
  \begin{itemize}[noitemsep]
  \item Whether the current command has completed.
  \item Whether a byte is available to be read from the data-in register.
  \item Whether a device error has occurred.
  \end{itemize}

\item The \textbf{\texttt{control} register} can be \textbf{written by the host} to start a command or to change the mode of a device.
  For instance:
  \begin{itemize}[noitemsep]
  \item One bit in a register of a serial port chooses between full-duplex and half-duplex communication.
  \item Another bit enables parity checking.
  \item A third bit sets the word length to different bit lengths.
  \item Other bits select one of the speeds supported by the serial port.
  \end{itemize}
\end{enumerate}

\subsubsection{Polling}\label{subsubsec:Polling}
The controller indicates its state through the \texttt{busy} bit in the \texttt{status} register.
The controller sets the \texttt{busy} bit when it is busy working and clears it when the controller is ready to accept the next command.
The host signals its wishes by setting the \texttt{command-ready} bit in the \texttt{command} register when a command is available for the controller to execute.

The repeating loop of polling is described in the following steps:
\begin{enumerate}[noitemsep]
\item The host repeatedly reads the \texttt{busy} bit until that bit becomes clear.
\item The host sets the \texttt{write} bit in the \texttt{command} register and writes a byte into the \texttt{data-out} register.
\item The host sets the \texttt{command-ready} bit.
\item When the controller notices that the \texttt{command-ready} bit is set, it sets the busy bit.
\item The controller reads the \texttt{command} register and sees the \texttt{write} command.
  It reads the \texttt{data-out} register to get the byte and does the I/O to the device.
\item The controller clears the \texttt{command-ready} bit, clears the error bit in the status register to indicate that the device I/O succeeded, and clears the \texttt{busy} bit to indicate that it is finished.
\end{enumerate}

The host is \textbf{busy-waiting} or \textbf{polling} in the first step: it is in a loop, reading the status register over and over until the busy bit becomes clear.
If the controller and device are fast, this is reasonable.
But if the wait is long enough, the host should switch to another task.

For some devices, the host \textbf{must} service the device quickly, or data will be lost.
For instance, when data are streaming in on a serial port, the small buffer on the controller will overflow and data will be lost if the host waits too long before returning to read the bytes.

The basic polling operation is instruction-efficient.
In many computer architectures, just three CPU-instruction cycles are sufficient to poll a device:
\begin{enumerate}[noitemsep]
\item Read a device register.
\item Logical \texttt{AND} to extract a status bit.
\item Branch if not zero.
\end{enumerate}

But polling becomes inefficient when it is attempted repeatedly yet rarely finds a device ready for service, while other useful CPU processing remains undone.
In those cases, it is more efficient to have the hardware controller notify the CPU when the device becomes ready again, rather than require the CPU to poll for I/O completion.
The hardware mechanism that enables a device to notify the CPU is called an \nameref{def:Interrupt}.

\subsubsection{Interrupts}\label{subsubsec:Interrupts}
The CPU hardware has a wire called the \nameref{def:Interrupt_Request_Line} that the CPU senses after executing every instruction.

\begin{definition}[Interrupt-Request Line]\label{def:Interrupt_Request_Line}
  The \emph{interrupt-request line} is a dedicated hardware line or \nameref{def:Hardware_Bus} for sending information about exceptional circumstances to the CPU.\@
  The CPU checks this line after \textbf{every} instruction.

  If the line does not contain any information, the CPU continues normal execution.
  If it does contain some information, then the CPU performs a \nameref{def:Context_Switch}.
  However, instead of switching to another \nameref{def:Process}, the CPU jumps to the \nameref{def:Interrupt_Handler}.
\end{definition}

When the CPU detects that a controller has asserted a signal on the interrupt-request line, the CPU performs a state save and jumps to the \nameref{def:Interrupt_Handler} routine at a fixed address in memory.

\begin{definition}[Interrupt Handler]\label{def:Interrupt_Handler}
  The \emph{interrupt handler}:
  \begin{enumerate}[noitemsep]
  \item Determines the cause of the interrupt.
  \item Performs the necessary processing.
  \item Performs a state restore.
  \item Executes a return from interrupt instruction to return the CPU to the execution state prior to the interrupt.
  \end{enumerate}
\end{definition}

The basic loop of an \nameref{def:Interrupt}-driven \nameref{def:Operating_System} is:
\begin{enumerate}[noitemsep]
\item The device controller \textbf{raises} an interrupt by asserting a signal on the \nameref{def:Interrupt_Request_Line}.
\item The CPU \textbf{catches} the interrupt and \textbf{dispatches} it to the \nameref{def:Interrupt_Handler}.
\item The handler \textbf{clears} the interrupt by servicing the device.
\item The CPU returns to normal execution.
\end{enumerate}

The loop enables the CPU to respond to asynchronous events.

The interrupt mechanism is used to handle a wide variety of exceptions, such as:
\begin{itemize}[noitemsep]
\item Dividing by 0.
\item Accessing a protected or nonexistent memory address.
\item Attempting to execute a privileged instruction from user mode.
\end{itemize}

The events that trigger interrupts have a common property: they are occurrences that induce the operating system to execute an urgent, self-contained routine.

\paragraph{Interrupts on Modern Hardware}\label{par:Interrupts_Modern_Hardware}
In modern operating systems, however, we need more sophisticated interrupt-handling features, which are provided by the CPU and interrupt-controller hardware.
\begin{itemize}[noitemsep]
\item We need the ability to defer interrupt handling during critical processing.
\item We need an efficient way to dispatch to the proper interrupt handler for a device without first polling all the devices to see which one raised the interrupt.
\item We need a priority scheme, so that the operating system can distinguish between high- and low-priority interrupts and can respond with the appropriate degree of urgency.
\end{itemize}


%%% Local Variables:
%%% mode: latex
%%% TeX-master: "../../EDAF35-Operating_Systems-Reference_Sheet"
%%% End:
