\subsection{I/O Hardware}\label{subsec:IO_Hardware}
The device communicates with the machine via a connection point, a port.
If devices share a common set of wires, the connection is called a \nameref{def:Hardware_Bus}.
Buses are used widely in computer architecture and vary in their signaling methods, speed, throughput, and connection methods.

\begin{definition}[Bus]\label{def:Hardware_Bus}
  A \emph{bus} is a set of wires and a rigidly defined protocol that specifies a set of messages that can be sent on the wires.
\end{definition}

When device A plugs into device B, and device B plugs into device C, and device C plugs into a port on the computer, this arrangement is called a daisy chain.
A daisy chain usually operates as a \nameref{def:Hardware_Bus}.

All of these may be controlled by a \nameref{def:Controller}.

\begin{definition}[Controller]\label{def:Controller}
  A \emph{controller} is a collection of electronics that can operate a port, a \nameref{def:Hardware_Bus}, or a device.
\end{definition}

Because some protocols are complex, their \nameref{def:Hardware_Bus} \nameref{def:Controller} is often implemented as a separate circuit board (or a host adapter) that plugs into the computer.
It typically contains a processor, microcode, and some private memory to enable it to process the protocol messages.
Some devices have their own built-in controllers.
They can receive commands in 2 ways (some support both, some do not):
\begin{enumerate}[noitemsep]
\item The \nameref{def:Controller} has one or more registers for data and control signals.
  The processor communicates with the controller by reading and writing bit patterns in these registers.
  This communication can occur through the use of special I/O instructions that specify the transfer of a byte or word to an I/O port address.
  The I/O instruction triggers bus lines to select the proper device and to move bits into or out of a device register.
\item The device \nameref{def:Controller} can support \nameref{subsubsec:Memory_Mapped_IO}.
  In this case, the device-control registers are mapped into the address space of the processor.
  The CPU executes I/O requests using the standard data-transfer instructions to read and write the device-control registers at their mapped locations in physical memory.
\end{enumerate}

For example, a graphics controller supports both of these methods of control.
\begin{itemize}[noitemsep]
\item The graphics controller has I/O ports for basic control operations.
\item The controller has a large memory-mapped region to hold screen contents.
  The process sends output to the screen by writing data into the memory-mapped region.
  The controller generates the screen image based on the contents of this memory.
\end{itemize}

The ease of writing to a \nameref{subsubsec:Memory_Mapped_IO} \nameref{def:Controller} is offset by the disadvantage of being vulnerable to accidental modification by an incorrect pointer to an unintended region of memory.
Memory protection helps reduce this risk.


%%% Local Variables:
%%% mode: latex
%%% TeX-master: "../../EDAF35-Operating_Systems-Reference_Sheet"
%%% End:
