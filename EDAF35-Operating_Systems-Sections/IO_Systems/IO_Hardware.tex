\subsection{I/O Hardware}\label{subsec:IO_Hardware}
The device communicates with the machine via a connection point, a port.
If devices share a common set of wires, the connection is called a \nameref{def:Hardware_Bus}.
Buses are used widely in computer architecture and vary in their signaling methods, speed, throughput, and connection methods.

\begin{definition}[Bus]\label{def:Hardware_Bus}
  A \emph{bus} is a set of wires and a rigidly defined protocol that specifies a set of messages that can be sent on the wires.
\end{definition}

When device A plugs into device B, and device B plugs into device C, and device C plugs into a port on the computer, this arrangement is called a daisy chain.
A daisy chain usually operates as a \nameref{def:Hardware_Bus}.

All of these may be controlled by a \nameref{def:Controller}.

\begin{definition}[Controller]\label{def:Controller}
  A \emph{controller} is a collection of electronics that can operate a port, a \nameref{def:Hardware_Bus}, or a device.
\end{definition}

Because some protocols are complex, their \nameref{def:Hardware_Bus} \nameref{def:Controller} is often implemented as a separate circuit board (or a host adapter) that plugs into the computer.
It typically contains a processor, microcode, and some private memory to enable it to process the protocol messages.
Some devices have their own built-in controllers.

%%% Local Variables:
%%% mode: latex
%%% TeX-master: "../../EDAF35-Operating_Systems-Reference_Sheet"
%%% End:
