\subsection{STREAMS}\label{subsec:STREAMS}
\textsc{streams} enable an application to assemble pipelines of driver code dynamically.
A stream is a full-duplex connection between a device driver and a user-level process.
It consists of:
\begin{enumerate}[noitemsep]
\item a \textbf{stream head} that interfaces with the user process
\item a \textbf{driver end} that controls the device
\item zero or more \textbf{stream modules} between the stream head and the driver end
\end{enumerate}

Each of these components contains a read queue and a write queue.
Message passing is used to transfer data between queues.

\nameref{def:Kernel_Module}s provide the functionality of \textsc{streams} processing; they are pushed onto a stream by use of the \kernelinline{ioctl()} \nameref{def:System_Call}.
Because messages are exchanged between queues only in adjacent modules, a queue in one module may overflow an adjacent queue.
To prevent this from occurring, a queue may support flow control.
\begin{itemize}[noitemsep]
\item Without flow control, a queue accepts all messages and immediately sends them on to the queue in the adjacent module without buffering them.
\item A queue that supports flow control buffers messages and does not accept messages without sufficient buffer space.
  \begin{itemize}[noitemsep]
  \item This process involves exchanges of control messages between queues in adjacent modules.
  \end{itemize}
\end{itemize}

A user process writes data to a device using either the write() or putmsg() system call.
\begin{itemize}[noitemsep]
\item \kernelinline{write()} writes raw data to the stream
\item \kernelinline{putmsg()} allows the user process to specify a message.
\end{itemize}

Regardless of the system call used, the stream head copies the data into a message and delivers it to the queue for the next module in line.
This message passing continues until the message is copied to the driver end and hence the device.
Similarly, the user process reads data from the stream head using either the read() or getmsg() system call.
\begin{itemize}[noitemsep]
\item \kernelinline{read()} returns an unstructured byte stream to the stream head from its adjacent queue.
\item \kernelinline{getmsg()} is used, a message is returned to the process.
\end{itemize}

\textsc{streams} I/O is \nameref{def:Blocking_Syscall} \textbf{ONLY} for the user, and only when communicating with the stream head.
Otherwise, it is asynchronous (or \nameref{def:Nonblocking_Syscall}).
The driver end \textbf{must} respond to \nameref{def:Interrupt}s.
Unlike the stream head, which may block if it is unable to copy a message to the next queue in line, the driver end \textbf{must} handle all incoming data.
Drivers must support flow control as well.
However, if a device’s buffer is full, the device typically resorts to just dropping incoming messages.


%%% Local Variables:
%%% mode: latex
%%% TeX-master: "../../EDAF35-Operating_Systems-Reference_Sheet"
%%% End:
