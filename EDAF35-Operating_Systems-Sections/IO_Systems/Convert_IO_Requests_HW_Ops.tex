\subsection{Convert I/O Requests to Hardware Operations}\label{subsec:Convert_IO_Requests_HW_Ops}
Up to this point, no explanation has been given for how the \nameref{def:Operating_System} connects an application's request to a set of wires or hardware.

In the case of a \nameref{def:File} lookup, there are 2 methods:
\begin{enumerate}[noitemsep]
\item The \nameref{def:Path_Name} also specifies the \nameref{def:Volume} or disk that the file is on.
  Windows/MS-DOS use this.
\item The device name space is incorporated into the regular \nameref{def:File_System} namespace.
  \textsc{unix} uses this.
\end{enumerate}

If the \nameref{def:Path_Name} also specifies the \nameref{def:Volume}, then the device is easy to choose from the rest of the path.
This separation makes it easy for the operating system to associate extra functionality with each device.
For instance, it is easy to invoke spooling on any files written to the printer.


%%% Local Variables:
%%% mode: latex
%%% TeX-master: "../../EDAF35-Operating_Systems-Reference_Sheet"
%%% End:
