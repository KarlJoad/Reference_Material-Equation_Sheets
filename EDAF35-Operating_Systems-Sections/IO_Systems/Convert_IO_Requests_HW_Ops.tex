\subsection{Convert I/O Requests to Hardware Operations}\label{subsec:Convert_IO_Requests_HW_Ops}
Up to this point, no explanation has been given for how the \nameref{def:Operating_System} connects an application's request to a set of wires or hardware.

In the case of a \nameref{def:File} lookup, there are 2 methods:
\begin{enumerate}[noitemsep]
\item The \nameref{def:Path_Name} also specifies the \nameref{def:Volume} or disk that the file is on.
  Windows/MS-DOS use this.
\item The device name space is incorporated into the regular \nameref{def:File_System} namespace.
  \textsc{unix} uses this.
\end{enumerate}

If the \nameref{def:Path_Name} also specifies the \nameref{def:Volume}, then the device is easy to choose from the rest of the path.
This separation makes it easy for the operating system to associate extra functionality with each device.
For instance, it is easy to invoke spooling on any files written to the printer.

If the device name space is incorporated in the regular \nameref{def:File_System} namespace, as it is in \textsc{unix}, the normal file-system name services are provided for free.
For example, if the file system provides ownership and access control to all file names, then devices also have \nameref{def:File_Owner}s and access control.
Since files are stored on devices, such an interface provides access to the I/O system at two levels.

\begin{enumerate}[noitemsep]
\item Names can be used to access the devices themselves or to access the files stored on the devices.
  Because there is no clear   separation of the device from the path the \nameref{def:Operating_System} has a mount table that associates prefixes of \nameref{def:Path_Name}s with specific device names.
  To resolve a path name, UNIX looks up the name in the mount table with the longest matching prefix; the corresponding entry in the mount table gives the device name.

\item This device name also has the form of a name in the \nameref{def:File_System} namespace.
  When \textsc{unix} looks up this name in the file-system \nameref{def:Directory} structures, it finds not an \nameref{par:Inode} number but a $\langle \mathtt{major}, \mathtt{minor} \rangle$ \nameref{def:Device_Number}.
\end{enumerate}

\begin{definition}[Device Number]\label{def:Device_Number}
  The \emph{device number} is made up of 2 parts, a \texttt{major} and \texttt{minor} number, typically represented as
  \begin{equation*}
    \langle \mathtt{major}, \mathtt{minor} \rangle
  \end{equation*}

  The \texttt{major} device number identifies a device driver that should be called to handle I/O to this device.
  The \texttt{minor} device number  is passed to the device driver to index into a device table.
  The corresponding device-table entry gives the port address or the memory-mapped address of the device controller.
\end{definition}

Modern operating systems gain significant flexibility from the multiple stages of lookup tables in the path between a request and a physical device controller.
The mechanisms that pass requests between applications and drivers are general.
Thus, we can introduce new devices and drivers into a computer without recompiling the kernel.


%%% Local Variables:
%%% mode: latex
%%% TeX-master: "../../EDAF35-Operating_Systems-Reference_Sheet"
%%% End:
