\subsection{Application I/O Interface}\label{subsec:Application_IO_Interface}
The \nameref{def:Operating_System} wants to treat all I/O devices in a standard, uniform way.
Like other complex software-engineering problems, the approach here involves abstraction, encapsulation, and software layering.
Specifically, we can abstract away the detailed differences in I/O devices by identifying a few general kinds, where each is accessed through a standardized set of functions, its \emph{interface}.
The implementation differences are encapsulated in \nameref{def:Kernel_Module}s called \nameref{def:Device_Driver}s that internally are custom-tailored to specific devices but that export one of the standard interfaces.
The software layering used is shown in \Cref{fig:Kernel_IO_Subsystem_Structure}.

\begin{figure}[h!tbp]
  \centering
  \includegraphics[scale=1.00]{./Drawings/EDAF35-Operating_Systems/Kernel_IO_Subsystem_Structure.jpg}
  \caption{Kernel I/O Subsystem Structure}
  \label{fig:Kernel_IO_Subsystem_Structure}
\end{figure}

This layering methodology is used to hide differences among devices and their controllers from the \nameref{def:Kernel}'s I/O subsystem.
This is analogous to the I/O system calls encapsulating the behavior of devices in a few generic classes that hide hardware differences from applications.

Making the I/O subsystem independent of the hardware simplifies the job of the operating-system developer.
It also benefits the hardware manufacturers.
They either:
\begin{itemize}[noitemsep]
\item Design new devices to be compatible with an existing host controller interface.
\item Write device drivers to interface the new hardware to popular operating systems.
\end{itemize}

Thus, we (users) can attach new peripherals to a computer without waiting for the operating system to develop explicit kernel-level support code.
Unfortunately for hardware manufacturers, each operating system has its own standards for the device-driver interface.

Devices can vary in many dimensions, as seen in \Cref{tab:Characteristics_IO_Devices}.
\begin{table}[h!tbp]
  \centering
  \begin{tabular}{cll}
    \toprule
    \textbf{Aspect} & \textbf{Variation} & \textbf{Example} \\
    \midrule
    \multirow{2}{*}{Data-Transfer Mode} & Character & Terminal \\
                    & Block & Disk \\
    \multirow{2}{*}{Access Method} & Sequential & Modem/Network \\
                    & Random & Magnetic Disk \\
    \multirow{2}{*}{Transfer Schedule} & Synchronous & Magnetic Tape \\
                    & Asynchronous & Keyboard \\
    \multirow{2}{*}{Sharing} & Dedicated & Magnetic Tape \\
                    & Sharable & Keyboard \\
    \multirow{4}{*}{Device Speed} & Latency & \\
                    & Seek Time & \\
                    & Transfer Rate & \\
                    & Delay between Operations & \\
    \multirow{3}{*}{I/O Direction} & & \\
                    & Read-only & CD-ROM (After first write) \\
                    & Write-only & Graphics Controller \\
                    & Read-write & Disk \\
    \bottomrule
  \end{tabular}
  \caption{Characteristics of I/O Devices}
  \label{tab:Characteristics_IO_Devices}
\end{table}

Now, each of the entries in \Cref{tab:Characteristics_IO_Devices} are further explained:
\begin{itemize}[noitemsep]
\item Character-stream or block.
  \begin{itemize}[noitemsep]
  \item Character-stream device transfers \textbf{bytes} one-by-one.
  \item A block device transfers a block of bytes as a unit.
\end{itemize}

\item Sequential or random access.
  \begin{itemize}[noitemsep]
  \item Sequential devices transfer data in a fixed order determined by the device
  \item Random-access devices can instruct the device to seek to any of the available data storage locations.
\end{itemize}

\item Synchronous or asynchronous.
  \begin{itemize}[noitemsep]
  \item Synchronous devices performs data transfers with predictable response times, in coordination with other aspects of the system.
  \item Asynchronous devices exhibits irregular or unpredictable response times not coordinated with other computer events.
\end{itemize}

\item Sharable or dedicated.
  \begin{itemize}[noitemsep]
  \item A sharable device can be used concurrently by several \nameref{def:Process}es or \nameref{def:Thread}s.
  \item Dedicated devices cannot be shared by several concurrent \nameref{def:Process}es or \nameref{def:Thread}s.
\end{itemize}

\item Speed of operation.
  Device speeds range over a wide range of speeds, from a few bytes per second to several gigabytes per second.

\item Read–write, read only, or write only.
  Some devices perform both input and output, but others support only one data transfer direction.
\end{itemize}

%%% Local Variables:
%%% mode: latex
%%% TeX-master: "../../EDAF35-Operating_Systems-Reference_Sheet"
%%% End:
