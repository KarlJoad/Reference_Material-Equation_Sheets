\section{Historical Context}\label{sec:Historical_Context}
We are concerned with both \nameref{def:Psychological_Disorder}s and \nameref{def:Psychological_Dysfunction}s.

\begin{definition}[Psychological Disorder]\label{def:Psychological_Disorder}
  There are 3 criteria for a \emph{psychological disorder}:
  \begin{enumerate}[noitemsep]
  \item \nameref{def:Psychological_Dysfunction}
  \item Distress or Impairment.
    The \nameref{def:Psychological_Dysfunction} must cause distress to the individual, or impair their ability to function ``normally''
  \item The response to the \nameref{def:Psychological_Dysfunction} must be \emph{atypical} or \emph{not culturally expected}.
    Some people with disorders that do not affect their life terribly strongly might be considered ``eccentric'' or ``talented''.
    In addition, the more productive you are, the more abnormal you can be without society making a big fuss about it.
    \begin{itemize}[noitemsep]
    \item Note that ``normal'' does \textbf{not} refer to, for example, political dissidents.
    \end{itemize}
  \end{enumerate}
\end{definition}

\begin{definition}[Psychological Dysfunction]\label{def:Psychological_Dysfunction}
  \emph{Psychological dysfunction} refers to a breakdown in cognitive, emotional, or behavioral functioning.
  For example, if you are out on a date, it should be fun.
  But if you experience severe fear all evening and just want to go home, even though there is nothing to be afraid of, and the severe fear happens on every date, your emotions are not functioning properly.

  \begin{remark}[Harmful Dysfunction]\label{rmk:Harmful_Dysfunction}
    \emph{Harmful dysfunction} is a narrowing of the idea of a \nameref{def:Psychological_Dysfunction}.
    Namely, it is concerned with whether the behavior is out of the individual's control.
  \end{remark}
\end{definition}

But it is never easy to decide what represents a \nameref{def:Psychological_Dysfunction}, and we may never be able to satisfactorily define disease or \nameref{def:Psychological_Disorder}.
The best we may be able to do is to consider how the apparent disease or disorder matches our current understanding of its \nameref{def:Prototype}.

\begin{definition}[Prototype]\label{def:Prototype}
  A \emph{prototype} is a ``typical'' profile of a disorder.
\end{definition}

A patient may present only some features or symptoms of the disorder (there exists a minimum number) and still meet criteria for the \nameref{def:Psychological_Disorder} because their set of symptoms is close to the \nameref{def:Prototype}.

\subsection{Psychopathology}\label{subsec:Psychopathology}
\begin{definition}[Psychopathology]\label{def:Psychopathology}
  \emph{Psychopathology} is the scientific study of psychological disorders.
  This field contains many different branches, including:
  \begin{itemize}[noitemsep]
  \item \nameref{def:Clinical_Psychologist}s
  \item \nameref{def:Counseling_Psychologist}s
  \item \nameref{def:Psychiatrist}s
  \item \nameref{def:Psychiatric_Social_Worker}s
  \item \nameref{def:Psychiatric_Nurse}s
  \item \nameref{def:Marriage_Family_Therapist}s
  \item \nameref{def:Mental_Health_Counselor}s
  \end{itemize}
\end{definition}

They study patients that are \nameref{def:Present}ing a problem.

\begin{definition}[Present]\label{def:Present}
  \emph{Present} is a traditional shorthand way of indicating why the person came to the clinic.
\end{definition}

The effects that the patient \nameref{def:Present}s are used to form a \nameref{def:Clinical_Description}.
\begin{definition}[Clinical Description]\label{def:Clinical_Description}
  A \emph{clinical description} represents the unique combination of behaviors, thoughts, and feelings that make up a specific disorder.
  The clinical portion of the phrase refers both to the types of problems or disorders that you would find in a clinic or hospital and to the activities connected with assessment and treatment.
\end{definition}

Both the course of the \nameref{def:Psychological_Disorder} and the onset are used to determine the \nameref{def:Prognosis}.

\begin{definition}[Prognosis]\label{def:Prognosis}
  The anticipated course of a disorder is called the \emph{prognosis}.
\end{definition}

\subsubsection{Branches of Psychopathology}\label{subsubsec:Psychopathology_Branches}
\begin{definition}[Clinical Psychologist]\label{def:Clinical_Psychologist}
  \emph{Clinical psychologists} receive the Ph.D., doctor of philosophy, degree (or sometimes an Ed.D., doctor of education, or Psy.D., doctor of psychology) and follow a course of graduate-level study lasting approximately 5 years, which prepares them to conduct research into the causes and treatment of psychological disorders and to diagnose, assess, and treat these disorders.
  Clinical psychologists usually concentrate on more severe \nameref{def:Psychological_Disorder}s.
\end{definition}

\begin{definition}[Counseling Psychologist]\label{def:Counseling_Psychologist}
  \emph{Counseling psychologists} receive the Ph.D., doctor of philosophy, degree (or sometimes an Ed.D., doctor of education, or Psy.D., doctor of psychology) and follow a course of graduate-level study lasting approximately 5 years, which prepares them to conduct research into the causes and treatment of psychological disorders and to diagnose, assess, and treat these disorders.
  Counseling psychologists tend to study and treat adjustment and vocational issues encountered by relatively healthy individuals.
\end{definition}

\begin{definition}[Psychiatrist]\label{def:Psychiatrist}
  \emph{Psychiatrists} first earn an M.D.\ degree in medical school and then specialize in psychiatry during residency training that lasts 3 to 4 years.
  Psychiatrists also investigate the nature and causes of psychological disorders, often from a biological point of view; make diagnoses; and offer treatments.
  Many psychiatrists emphasize drugs or other biological treatments, although most use psychosocial treatments as well.
\end{definition}

\begin{definition}[Psychiatric Social Worker]\label{def:Psychiatric_Social_Worker}
  \emph{Psychiatric social workers} typically earn a master's degree in social work as they develop expertise in collecting information relevant to the social and family situation of the individual with a \nameref{def:Psychological_Disorder}.
  Social workers also treat \nameref{def:Psychological_Disorder}s, often concentrating on family problems associated with them.
\end{definition}

\begin{definition}[Psychiatric Nurse]\label{def:Psychiatric_Nurse}
  \emph{Psychiatric nurses} have advanced degrees, such as a master's or even a Ph.D., and specialize in the care and treatment of patients with \nameref{def:Psychological_Disorder}s, usually in hospitals as part of a treatment team.
\end{definition}

\begin{definition}[Marriage and Family Therapist]\label{def:Marriage_Family_Therapist}
  \emph{Marriage and family therapists} typically spend 1 to 2 years earning a master’s degree and are employed to provide clinical services by hospitals or clinics, usually under the supervision of a doctoral-level clinician.
\end{definition}

\begin{definition}[Mental Health Counselor]\label{def:Mental_Health_Counselor}
  \emph{Mental health counselors} typically spend 1 to 2 years earning a master’s degree and are employed to provide clinical services by hospitals or clin ics, usually under the supervision of a doctoral-level clinician.
\end{definition}

\subsubsection{Some Concerns of Psychopathology}\label{subsubsec:Psychopathology_Concerns}
\nameref{def:Psychopathology} works with individuals to improve their particular issue.
But, they also have some other concerns as well.

\begin{definition}[Prevalence]\label{def:Prevalence}
  \emph{Prevalence} is how many people in the population as a whole have the disorder.
\end{definition}

\begin{definition}[Incidence]\label{def:Incidence}
  \emph{Incidence} is the statistic on how many new cases occur during a given period.
\end{definition}

Some other statistics include:
\begin{itemize}[noitemsep]
\item The sex ratio, what percentage of males and females have the disorder.
\item The typical age of onset, which often differs from one disorder to another.
\item The course of the \nameref{def:Psychological_Disorder}.
\end{itemize}

\subsection{Courses of Psychological Disorders}
\begin{definition}[Chronic Course]\label{def:Chronic_Course}
  A \emph{chronic course}, means that the \nameref{def:Psychological_Disorder} tends to last a long time, sometimes a lifetime.
\end{definition}

\begin{definition}[Episodic Course]
  An \emph{episodic course}, means that the \nameref{def:Psychological_Disorder} will have a sequence of episodes in which the individual recovers, only to suffer a recurrence of the disorder.
  This may occur throughout the individual's life.
\end{definition}

\begin{definition}[Time-Limited Course]\label{def:Time_Limited_Course}
  \emph{Time-limited course}, meaning the disorder will improve without treatment in a relatively short period with little or no risk of recurrence.
\end{definition}

\subsection{Onset of Psychological Disorders}
The onset of particular \nameref{def:Psychological_Disorder}s can vary quite widely.

\begin{definition}[Acute Onset]\label{def:Acute_Onset}
  An \emph{acute onset} \nameref{def:Psychological_Disorder} is one where the disorder begins quite suddenly.
\end{definition}

\begin{definition}[Insidious Onset]\label{def:Insidious_Onset}
  An \emph{acute onset} \nameref{def:Psychological_Disorder} is one where the disorder begins quite suddenly.
\end{definition}

\subsection{Models of Abnormal Behavior}\label{subsec:Models_Abnormal_Behavior}
There are three main models that have been and continue to be used today:
\begin{enumerate}[noitemsep]
\item The \nameref{def:Supernatural_Model}
\item The \nameref{def:Biological_Model}
\item The \nameref{def:Psychological_Model}
\end{enumerate}

\subsubsection{The Supernatural Model}\label{subsubsec:Supernatural_Model}
Deviant behavior has been considered a reflection of the battle between good and evil.
When confronted with unexplainable, irrational behavior and by suffering and upheaval, people have perceived evil.

\begin{definition}[Supernatural Model]\label{def:Supernatural_Model}
  The driving factors in the \emph{supernatural model} are:
  \begin{itemize}[noitemsep]
  \item Divinities
  \item Demons
  \item Spirits
  \item Other phenomena such as:
    \begin{itemize}[noitemsep]
    \item Magnetic fields
    \item The moon
    \item The stars
    \end{itemize}
  \end{itemize}
\end{definition}

While this model is still alive today, it is limited to small religious sects and primitive cultures.
Most members of organized religions today turn to psychology and medical science for help with \nameref{def:Psychological_Disorder}s.

\subsubsection{The Biological Model}\label{subsubsec:Biological_Model}
The \nameref{def:Biological_Model} was initially developed by Hippocrates his associates left a body of work called the \textit{Hippocratic Corpus}, written between 450 and 350 \textsc{b.c.}.
In this work, they suggested that \nameref{def:Psychological_Disorder}s could be treated like any other disease.
In addition, they did not limit their search for the causes of \nameref{def:Psychopathology} to the general area of ``disease,'' because they believed that psychological disorders might also be caused by brain pathology or head trauma and could be influenced by heredity (genetics).

\begin{definition}[Biological Model]\label{def:Biological_Model}
  The \emph{biological model} states that abnormal behavior arises from the body influencing the mind.
\end{definition}

This model has gone through phases of interest and disinterest.
When it becomes the only explanation for mental health issues, it tends to lead to focusing solely on improving the life of the patient through rest, relaxation, proper diet, proper exercise, etc.
However, when taken to the extreme, this leads to a complete neglect and disinterest in the potential mental and emotional explanations for various \nameref{def:Psychological_Disorder}s.

\subsubsection{The Psychological Model}\label{subsubsec:Psychological_Model}
Plato was one of the first to explain that mental health issues were caused by poor environments and learning centers.
For example, Plato thought that the two causes of maladaptive behavior were the social and cultural influences in one’s life and the learning that took place in that environment.
If something was wrong in the environment, such as abusive parents, one’s impulses and emotions would overcome reason.
In his mind, the best treatment was to reeducate the individual through rational discussion so that the power of reason would predominate.

This formed the precursor to modern psychosocial treatment.

\begin{definition}[Psychological Model]\label{def:Psychological_Model}
  The \emph{psychological model} states that abnormal behavior arises from the mind influencing the body.

  In history, there were two movements in this model:
  \begin{enumerate}[noitemsep]
  \item Psychoanalytic, which involved analyzing the mind and the patient's history to determine their treatment.
  \item Behavioral, which involved identifying the issue and changing the patient's behavior to the issue through slow changes to the situation intended to show there is not inherent discomfort.
  \end{enumerate}
\end{definition}

Freud stated that the mind has three components:
\begin{enumerate}[noitemsep]
\item The \nameref{def:Id}.
\item The \nameref{def:Ego}.
\item The \nameref{def:Superego}.
\end{enumerate}

\begin{definition}[Id]\label{def:Id}
  The \emph{id} is the source of our strong sexual and aggressive feelings or energies.
  It is, basically, the animal within us; if totally unchecked, it would make us all rapists or killers.

  The id processes information according to the \emph{primary process}, which is emotional, irrational, illogical, fantastical, and preoccupied with sex, aggression, selfishness, and envy.
\end{definition}

\begin{definition}[Ego]\label{def:Ego}
  The \emph{ego} is the part of our mind that ensures that we act realistically.
  It  operates according to the according to the reality principle instead of the pleasure principle of the \nameref{def:Id}.

  The cognitive operations or thinking styles of the ego are characterized by logic and reason and are referred to as the \emph{secondary process}.

  The ego is responsible for ensuring the \nameref{def:Superego} and \nameref{def:Id} are in balance.
\end{definition}

\begin{definition}[Superego]\label{def:Superego}
  The \emph{superego}, represents the moral principles instilled in us by our parents and our culture.
  It is the voice within us that nags at us when we know we’re doing something wrong.

  It is fundamentally at odds with the \nameref{def:Id}.
\end{definition}

\subsection{Scientific Method and Integrative Approach}\label{subsec:Scientific_Method_Integrative_Approach}
This is the modern approach to psychopathology.
It is founded on the principle that any time a person does something, both the brain and the body are working toegether.
In addition, the person's thoughts are influencing their actions, and together they form our response.

%%% Local Variables:
%%% mode: latex
%%% TeX-master: "../PSYC_303-Intro_Psychopathology-Reference_Sheet"
%%% End:
