\section{Clinical Assessment}\label{sec:Clinical_Assessment}
\begin{definition}[Clinical Assessment]\label{def:Clinical_Assessment}
  \emph{Clinical assessment} is the systematic evaluation and measurement of psychological, biological, and social factors in an individual presenting with a possible psychological disorder.
\end{definition}

A \nameref{def:Clinical_Assessment} is used to help make a \nameref{def:Diagnosis}.

\begin{definition}[Diagnosis]\label{def:Diagnosis}
  \emph{Diagnosis} is the process of determining whether the particular problem afflicting the individual meets all criteria for a psychological disorder, as set forth in the fifth edition of the Diagnostic and Statistical Manual of Mental Disorders, or DSM-5.
\end{definition}

\subsection{Key Concepts}\label{subsec:Key_Clinical_Assessment_Concepts}
There are three major concepts that are incredibly important to know about when discussing \nameref{def:Clinical_Assessment}s and diagnoses.

\begin{enumerate}[noitemsep]
\item \nameref{def:Reliability}
\item \nameref{def:Validity}
\item \nameref{def:Standardization}
\end{enumerate}

\begin{definition}[Reliability]\label{def:Reliability}
  \emph{Reliability} is the degree to which a measurement is consistent.

  This means that running a particular test multiple times will yield similar, if not identical, diagnoses.
\end{definition}

\begin{definition}[Validity]\label{def:Validity}
  \emph{Validity} is whether something measures what it is designed to measure.

  This means that the test actual does what it says it does.
\end{definition}

\begin{definition}[Standardization]\label{def:Standardization}
  \emph{Standardization} is the process by which a certain set of standards or norms is determined for a technique to make its use consistent across different measurements.

  For example, when conducting a test, the proctor might have to give the directionsin a certain order, speak in at a certain pace, and then cannot communicate any further.
  This would be followed exactly the same way by every person every time the test is administered.
\end{definition}

\subsection{The Clinical Interview}\label{subsec:The_Clinical_Interview}
This is the opportunity for the health professional to gather:
\begin{itemize}[noitemsep]
\item The patient's personal information
\item The patient's current and past behavior, attitudes, and emotions
\item The patient's family information
\item The patient's friend group
\item Information on sexual development, religious attitudes, relevant cultural concerns, and educational history.
\end{itemize}

To achieve this, many clinicians use a \nameref{def:Mental_Status_Exam}.

\begin{definition}[Mental Status Exam]\label{def:Mental_Status_Exam}
  A \emph{mental status exam} involves the systematic observation of an individual's behavior.
  This type of observation occurs when any one person interacts with another.
  The clinician is trained to organize these observations such that they help provide information about the patient and their condition.

  The exam covers 5 categories:
  \begin{enumerate}[noitemsep]
  \item Appearance and behavior
  \item Thought processes --- Including rate of speech, speech patterns, etc.
  \item \nameref{def:Mood} and \nameref{def:Affect}
  \item Intellectual functioning
  \item Sensorium --- The general awareness of our surroundings, namely, is the patient \nameref{def:Oriented_Times_Three}.
  \end{enumerate}
\end{definition}

\begin{definition}[Oriented Times Three]\label{def:Oriented_Times_Three}
  To be \emph{oriented times three} requires that the patient is aware of who they are and who they are communicating with (person), where they are (place), and when it is (time).
\end{definition}

Such a \nameref{def:Mental_Status_Exam} can help identify various \nameref{def:Delusion}s.

\begin{definition}[Delusion]\label{def:Delusion}
  A \emph{delusion} is a distorted view of reality.

  There are several general types of delusion:
  \begin{itemize}[noitemsep]
  \item \nameref{def:Delusion_of_Persecution}
  \item \nameref{def:Delusion_of_Grandeur}
  \item \nameref{def:Ideas_of_Reference}
  \item \nameref{def:Hallucination}s
  \end{itemize}
\end{definition}

\begin{definition}[Delusion of Persection]\label{def:Delusion_of_Persecution}
  A \emph{delusion of persecution} is where the patient believes people are after them and out to get them all the time.
\end{definition}

\begin{definition}[Delusion of Grandeur]\label{def:Delusion_of_Grandeur}
  A \emph{delusion of grandeur} is where an individual thinks they are all-powerful in some way.
\end{definition}

\begin{definition}[Ideas of Reference]\label{def:Ideas_of_Reference}
   \emph{Ideas of reference} is where everything everyone else does somehow relates back to the individual.
\end{definition}

\begin{definition}[Hallucination]\label{def:Hallucination}
  \emph{Hallucinations} are things a person sees or hears when those things really aren't there.
\end{definition}

\begin{remark*}
  Patients tend to have a good general idea of the underlying issue they are facing.
\end{remark*}

\subsubsection{Interview Structure}\label{subsubsec:Interview_Structure}
Clinical interviews have three general structures:
\begin{description}[noitemsep]
\item[Structured] In a structured clinical interview, the clinician asks a set of questions from a script in an exact order and reocrds the patient's answers.
  This has the benefit of being very reliable, but not flexible.
  This can also lead to very jarring interviews, as topics that may be discussed in-depth are abruptly switched to or from.
\item[Unstructured] In an unstructured clinical interview, there is no script whatsoever.
  The clinician is free to ask questionsin any way they see fit.
  This allows for great flexibility, but next to no reliability.
  Because this is more of a conversation than an interview, both the patient and clinician may act/react differently each time.
\item[Semistructured] In a semistructured interview, the interview takes on a mix of both structured and unstructured interviews.
  The clinician has a script of questions they must ask, but they are also free to follow any information the client brings up.
  This offers a good balance of reliability and flexibility.
\end{description}


%%% Local Variables:
%%% mode: latex
%%% TeX-master: "../PSYC_303-Intro_Psychopathology-Reference_Sheet"
%%% End:
