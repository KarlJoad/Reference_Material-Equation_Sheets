\section{Clinical Assessment}\label{sec:Clinical_Assessment}
\begin{definition}[Clinical Assessment]\label{def:Clinical_Assessment}
  \emph{Clinical assessment} is the systematic evaluation and measurement of psychological, biological, and social factors in an individual presenting with a possible psychological disorder.
\end{definition}

A \nameref{def:Clinical_Assessment} is used to help make a \nameref{def:Diagnosis}.

\begin{definition}[Diagnosis]\label{def:Diagnosis}
  \emph{Diagnosis} is the process of determining whether the particular problem afflicting the individual meets all criteria for a psychological disorder, as set forth in the fifth edition of the Diagnostic and Statistical Manual of Mental Disorders, or DSM-5.
\end{definition}

\subsection{Key Concepts}\label{subsec:Key_Clinical_Assessment_Concepts}
There are three major concepts that are incredibly important to know about when discussing \nameref{def:Clinical_Assessment}s and diagnoses.

\begin{enumerate}[noitemsep]
\item \nameref{def:Reliability}
\item \nameref{def:Validity}
\item \nameref{def:Standardization}
\end{enumerate}

\begin{definition}[Reliability]\label{def:Reliability}
  \emph{Reliability} is the degree to which a measurement is consistent.

  This means that running a particular test multiple times will yield similar, if not identical, diagnoses.
\end{definition}


%%% Local Variables:
%%% mode: latex
%%% TeX-master: "../PSYC_303-Intro_Psychopathology-Reference_Sheet"
%%% End:
