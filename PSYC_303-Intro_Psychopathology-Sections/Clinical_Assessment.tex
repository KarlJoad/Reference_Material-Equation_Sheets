\section{Clinical Assessment}\label{sec:Clinical_Assessment}
\begin{definition}[Clinical Assessment]\label{def:Clinical_Assessment}
  \emph{Clinical assessment} is the systematic evaluation and measurement of psychological, biological, and social factors in an individual presenting with a possible psychological disorder.
\end{definition}

A \nameref{def:Clinical_Assessment} is used to help make a \nameref{def:Diagnosis}.

\begin{definition}[Diagnosis]\label{def:Diagnosis}
  \emph{Diagnosis} is the process of determining whether the particular problem afflicting the individual meets all criteria for a psychological disorder, as set forth in the fifth edition of the Diagnostic and Statistical Manual of Mental Disorders, or DSM-5.
\end{definition}

\subsection{Key Concepts}\label{subsec:Key_Clinical_Assessment_Concepts}
There are three major concepts that are incredibly important to know about when discussing \nameref{def:Clinical_Assessment}s and diagnoses.

\begin{enumerate}[noitemsep]
\item \nameref{def:Reliability}
\item \nameref{def:Validity}
\item \nameref{def:Standardization}
\end{enumerate}

\begin{definition}[Reliability]\label{def:Reliability}
  \emph{Reliability} is the degree to which a measurement is consistent.

  This means that running a particular test multiple times will yield similar, if not identical, diagnoses.
\end{definition}

\begin{definition}[Validity]\label{def:Validity}
  \emph{Validity} is whether something measures what it is designed to measure.

  This means that the test actual does what it says it does.
\end{definition}

\begin{definition}[Standardization]\label{def:Standardization}
  \emph{Standardization} is the process by which a certain set of standards or norms is determined for a technique to make its use consistent across different measurements.

  For example, when conducting a test, the proctor might have to give the directionsin a certain order, speak in at a certain pace, and then cannot communicate any further.
  This would be followed exactly the same way by every person every time the test is administered.
\end{definition}

\subsection{The Clinical Interview}\label{subsec:The_Clinical_Interview}
This is the opportunity for the health professional to gather:
\begin{itemize}[noitemsep]
\item The patient's personal information
\item The patient's current and past behavior, attitudes, and emotions
\item The patient's family information
\item The patient's friend group
\item Information on sexual development, religious attitudes, relevant cultural concerns, and educational history.
\end{itemize}


%%% Local Variables:
%%% mode: latex
%%% TeX-master: "../PSYC_303-Intro_Psychopathology-Reference_Sheet"
%%% End:
