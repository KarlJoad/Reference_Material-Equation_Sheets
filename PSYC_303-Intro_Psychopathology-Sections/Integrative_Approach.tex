\section{Integrative Approach to Psychopathology}\label{sec:Integrative_Approach}
We prefer to use a \emph{multidimensional integrative approach} to \nameref{def:Psychopathology}.
These dimensions include:
\begin{description}[noitemsep]
\item[Biological] Causal factors from genetics and neuroscience.
\item[Psychological] Causal factors from behavioral and cognitive processes, including learned helplessness, social learning, prepared learning, and unconscious processes.
\item[Emotional] These contribute in a variety of ways.
  Emotions play a substantial role in the development of many disorders.
\item[Developmental] These factors are always present.
\end{description}

Each of the above dimensions are \textbf{never} in isolation, each one is strongly influenced by the others.

\subsection{Genetic Contributions to Psychopathology}\label{subsec:Genetic_Contributions}
Genetics influence our physical characteristics such as hair color and eye color.
However, genetics only provide boundaries for our body to work with.
The environment influences our physical appearance too.
To some extent, our weight and even our height are affected by nutritional, social, and cultural factors.

Much of our development and, interestingly, most of our behavior, our personality, and even our intelligence are probably \nameref{def:Polygenic}.

\begin{definition}[Polygenic]\label{def:Polygenic}
  For something to be \emph{polygenic}, the end result must be reached when multiple genes are being expressed.
\end{definition}

For psychological disorders, the evidence indicates that genetic factors make some contribution to all disorders but account for less than half of the explanation.

\subsubsection{The Diathesis-Stress Model}\label{subsubsec:Diathesis-Stress_Model}
\begin{definition}[Diathesis-Stress Model]\label{def:Diathesis-Stress_Model}
  The \emph{diathesis-stress model} states that individuals inherit tendencies to express certain traits or behaviors, which may then be activated under conditions of stress.
\end{definition}

This model of gene–environment interactions has been popular, although, in view of the relationship of the environment to the structure and function of the brain, it is greatly oversimplified.

\subsubsection{The Gene-Environment Correlation Model}\label{subsubsec:Gene-Environment_Correlation_Model}
\begin{definition}[Gene-Environment Correlation Model]\label{def:Gene-Environment_Correlation_Model}
  The \emph{gene-environment correlation model} states that a person may have a tendency towards a particular psychological issue, but also have genes that help incite that disorder.
  To that end, the genetic predisposition helps cause stresses that add to the person's predisposition to the \nameref{def:Psychological_Disorder}, eventually causing issues.
\end{definition}

Although the environment cannot change our DNA, it can change the gene expression.
Genes are turned on or off by cellular material that is located just outside of the genome (``epi,'' as in the word epigenetics, means on or around) and that stress, nutrition, or other factors can affect this epigenome, which is then immediately passed down to the next generation and maybe for several generations.

\subsection{Neuroscience and its Contributions to Psychopathology}\label{subsec:Neuroscience_Contributions_Psychopathology}
The brain is made up of nerve cells called \nameref{def:Neuron}.

\begin{definition}[Neuron]\label{def:Neuron}
  \emph{Neuron}s are the nerve cells in the brain.
  They are composed of a branch called a \nameref{def:Dendrite} and an \nameref{def:Axon}.
\end{definition}

\begin{definition}[Dendrite]\label{def:Dendrite}
  \emph{Dendrites} \textbf{receive} messages from other neurons.
\end{definition}

\begin{definition}[Axon]\label{def:Axon}
  \emph{Axon} is responsible for \textbf{sending} messages to other neurons.
\end{definition}

\begin{definition}[Synapse]\label{def:Synapse}
  The \nameref{def:Dendrite}-\nameref{def:Axon} connections between neurons are \emph{synapses}.
\end{definition}

The brain creates \emph{neurotransmitters} to communicate between \nameref{def:Neuron}s.
When communicating with the rest of the body, the brain and endocrine system work together to communicate with hormones.
Many of these hormones can be related (the extent is unknown) to many \nameref{def:Psychological_Disorder}s that we have identified.
However, our current understanding is that a lack of a particular neurotransmitter will alter the way we process information and the way we exhibit certain kinds of behavior.

\subsubsection{Neurotransmitter Changes}\label{subsubsec:Neurotransmitter_Changes}
\begin{definition}[Agonist]\label{def:Agonist}
  \emph{Agonists} effectively increase the activity of a neurotransmitter by mimicking its effects.
\end{definition}

\begin{definition}[Antagonist]\label{def:Agonist}
  \emph{Antagonists} that decrease, or block, a neurotransmitter.
\end{definition}

\begin{definition}[Inverse Agonist]\label{def:Inverse_Agonist}
  \emph{Inverse agonists} that produce effects opposite to those produced by the neurotransmitter.
\end{definition}

\subsubsection{Psychosocial Influences on Brain Structure and Function}\label{subsubsec:Psychosocial_Influences_Brain_Structure_Function}
The initiating factors (reason why a problem develops) are not necessarily the same as the maintaining factors (reason why a problem persists).
In order to treat the problem effectively, it is typically more important to know and target the maintaining factors than the initiating factors.

\subsection{Behavioral and Cognitive Science}\label{subsec:Behavioral_Cognitive_Science}
\begin{definition}[Cognitive Science]\label{def:Cognitive_Science}
  \emph{Cognitive science} is concerned with how we acquire and process information and how we store and ultimately retrieve it.
\end{definition}

Complex cognitive processing of information, as well as emotional processing, is involved when conditioning occurs.

\begin{definition}[Learned Helplessness]\label{def:Learned_Helplessness}
  \emph{Learned helplessness} occurs when rats or other animals encounter conditions over which they have no control.
  The animal gives up attempting to cope and seem to develop the animal equivalent of depression.
\end{definition}

Organisms do not have to experience certain events in their environment to learn effectively.
Rather, they can learn just as much by observing what happens to someone else in a given situation.
This fairly obvious discovery came to be known as \emph{modeling} or \emph{observational learning}.

\subsection{Emotions}\label{subsec:Emotions}
The alarm reaction that activates during potentially life-threatening emergencies is called the \emph{flight or fight response}.

To define ``emotion'' is difficult, but most theorists agree that it is linked to an action tendency (a tendency to behave in a certain way), elicited by an external event (a threat) and a feeling state (terror) and accompanied by a (possibly) characteristic physiological response.
Emotions are usually short-lived, temporary states lasting from several minutes to several hours, occurring in response to an external event.
Mood is a more persistent period of affect or emotionality.

Emotion scientists now agree that emotion is composed of three related components-behavior, physiology, and cognition-but most emotion scientists tend to concentrate on one component or another.

\subsection{Cultural, Social, Interpersonal Factors}\label{subsec:Cultural_Social_Interpersonal_Factors}
In many cultures around the world, individuals may suffer from fright disorders, which are characterized by exaggerated startle responses, and other observable fear and anxiety reactions.

Fear and phobias are universal, occurring across all cultures.
But what we fear is strongly influenced by our social environment.

Our gender doesn’t cause psychopathology.
We think these substantial differences have to do with, at least in part, cultural expectations of men and women, or our gender roles.
But because gender role is a social and cultural factor that influences the form and content of a disorder.

Many studies have demonstrated that the greater the number and frequency of social relationships and contacts, the longer you are likely to live.
Interestingly, it is not just the absolute number of social contacts that is important.
It is the actual perception of loneliness.

Psychological disorders continue to carry a substantial stigma in our society.
To be anxious or depressed is to be weak and cowardly.
To be schizophrenic is to be unpredictable and crazy.

\subsection{Life-Span Development}\label{subsec:Life-Span_Development}
We tend to look at psychological disorders from a snapshot perspective: we focus on a particular point in a person’s life and assume it represents the whole person.

For example, in depressive (mood) disorders, children and adolescents do not receive the same benefit from antidepressant drugs as do adults, and for many of them these drugs pose risks that are not present in adults.

%%% Local Variables:
%%% mode: latex
%%% TeX-master: "../PSYC_303-Intro_Psychopathology-Reference_Sheet"
%%% End:
