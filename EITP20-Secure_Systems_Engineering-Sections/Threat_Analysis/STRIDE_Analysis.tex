\subsection{Microsoft STRIDE Analysis}\label{subsec:STRIDE_Analysis}
\begin{definition}[STRIDE Analysis]\label{def:STRIDE_Analysis}
  \emph{STRIDE} is an acronym standing for
  \begin{itemize}[noitemsep]
  \item [S] \nameref{def:Spoofing}
  \item [T] \nameref{def:Tampering}
  \item [R] \nameref{def:Repudiation}
  \item [I] \nameref{def:Information_Disclosure}
  \item [D] \nameref{def:Denial_of_Service}
  \item [E] \nameref{def:Elevation_of_Privilege}
  \end{itemize}

  It is a model to identify computer security threats in software systems or components of a larger software system.
  It is general enough to analyze any part of a software system.

  To perform a STRIDE analysis, it is commonly applied with these steps:
  \begin{enumerate}[noitemsep]
  \item Identify the main entities in the system
  \item Identify the interactions between/among the main entities.
  \item For each entity, perform a STRIDE analysis on the following actions:
    \begin{itemize}[noitemsep]
    \item Processes
    \item External Entities
    \item Data Flows
    \item Data Stores
    \end{itemize}
  \end{enumerate}

  \begin{remark}[Incomplete Methodology]\label{rmk:STRIDE_Incomplete_Methodology}
    Like \nameref{def:Attack_Tree}s, \nameref{def:STRIDE_Analysis} is not a complete methodology.
    It only helps identify typical attacks, but is not a ``complete package''.
  \end{remark}
\end{definition}

\begin{table}[h!]
  \centering
  \begin{tabular}{p{1.5cm}p{2.75cm}p{4cm}p{4cm}p{4cm}}
    \toprule
    \multicolumn{1}{c}{Threat} & \multicolumn{1}{c}{Property Violated} & \multicolumn{1}{c}{Definition} & \multicolumn{1}{c}{Typical Victims} & \multicolumn{1}{c}{Example} \\
    \midrule
    \nameref{def:Spoofing} & Authentication & Pretending to be something or someone other than yourself & Processes, External Entities, People & Falsely claiming to be a police officer. \\ \midrule
    \nameref{def:Tampering} & Integrity & Modifying something on disk, on a network, or in memory & Data stores, data flows, processes & Changing spreadsheet, binary contents, database contents, modifying network packets, or the program running. \\ \midrule
    \nameref{def:Repudiation} & Non-Repudiation & Claiming that you didn't do something or were not responsible. & Process & Process: ``I didn't hit the big red button''. \\ \midrule
    \nameref{def:Information_Disclosure} & Confidentiality & Providing information to someone not authorized to see it & Processes, data stores, data flows & Allow access to file contents/file metadata, network packets, contents of program memory. \\ \midrule
    \nameref{def:Denial_of_Service} & Availability & Deliberately absorbing more resources than needed to provide a service & Processes, data stores, data flows & Program uses all available memory, File fills up disk, too many network connections for real traffic. \\ \midrule
    \nameref{def:Elevation_of_Privilege} & Authorization & Allowing someone to do something they're not authorized to do. & Process & Allow normal user to execute code as administrator. Remote person can run code. \\
    \bottomrule
  \end{tabular}
  \caption{STRIDE Acronym Definitions}
  \label{tab:STRIDE_Analysis_Definitions}
\end{table}

\subsubsection{Spoofing}\label{subsubsec:Spoofing}
\begin{definition}[Spoofing]\label{def:Spoofing}
  \emph{Spoofing} is pretending to be something or someone other than yourself.
  \Cref{tab:STRIDE_Analysis_Definitions} gives one example, there are several others.

  \begin{enumerate}[noitemsep]
  \item Spoofing the identity of an entity across a network.
    There is not mediating authority that takes reponsibility for telling that something is what it claims to be.
  \item Spoofing the identity of an entity, even when there is a mediating authority.
    For example, a modified \texttt{.dll} file is mediated by the operating system to make sure it is the right one.
    If it's modified, th emediating authority might let it through, even when it has been modified.
  \item Pretending to be a specific person, for example, Barack Obama.
  \item Pretending to be in a specific role.
  \end{enumerate}
\end{definition}

\begin{table}[h!]
  \centering
  \begin{tabular}{p{5.7cm}p{5.7cm}p{5.7cm}}
    \toprule
    Threat Examples & What the Attacker Does & Notes \\
    \midrule
    \multirow{3}{5.7cm}{Spoofing a process on the same machine} & Creates a file before the real process & \\ \cline{2-3}
                    & Renaming/Linking & Creating a trojan \texttt{su} and altering the \texttt{path} \\ \cline{2-3}
                    & Renaming something & Naming your process \texttt{sshd} \\
    \midrule
    \multirow{3}{5.7cm}{Spoofing a file} & Creates a file in the local directory & This can be a library, execcutable, or config file \\ \cline{2-3}
                    & Creates a link and changes it & Change should happen between link being checked and being accessed \\ \cline{2-3}
                    & Creates many files in expected directory & Automate file creation to fill the space of files possible \\
    \midrule
    \multirow{5}{5.7cm}{Spoofing a machine} & ARP Spoofing & \\ \cline{2-3}
                    & IP Spoofing & \\ \cline{2-3}
                    & DNS Spoofing & Forward or Reverse \\ \cline{2-3}
                    & DNS Compromise & Compromise TLD, registrar, or DNS operator \\ \cline{2-3}
                    & IP Redirection & At the switch or router level \\
    \midrule
    \multirow{2}{5.7cm}{Spoofing a person} & Sets e-mail display name & \\ \cline{2-3}
                    & Takes over a real account & \\
    \midrule
    \multirow{1}{5.7cm}{Spoofing a role} & Declares themselves to be that role & Sometimes opening a special account with a relevant name \\
    \bottomrule
  \end{tabular}
  \caption{\nameref{def:Spoofing} Threats}
  \label{tab:Spoofing_Threats}
\end{table}

\paragraph{Spoofing a Process or File on the same Machine}\label{par:Spoofing_Process_File_Same_Machine}
If an attacker creates a file before the real process, the process might interpret the file as being good data, and should be trusted.
File permissions during a pipe operation, local procedure, etc.\ can create vulnerabilities.

Spoofing a process or file on a remote machine can work by creating the expected versions, or by pretending to be the expected machine.

\paragraph{Spoofing a Machine}\label{par:Spoofing_Machine}
Depending on what has been spoofed, ARP, DNS, IP, anything else in the networking stack, the attacks can vary.
Once the attackers have this setup in place, then they can continue spoofing, or perform a man-in-the-middle attack.
They could also steal cryptographic keys.

With a spoofed machine, they can perform phishing attacks to steal information using sites that appear to be valid.

\paragraph{Spoofing a Person}\label{par:Spoofing_Person}
Typically, spoofing a person involves having access to the real person's digital account(s), then pretending to be the real person through another account.

\subsubsection{Tampering}\label{subsubsec:Tampering}
\begin{definition}[Tampering]\label{def:Tampering}
  \emph{Tampering} is modifying something, typically on disk, on a network, or in memory.
  This can include changing the data in a spreadsheet, changing a binary or config file on dis, modifying a database, etc.
  On a network, packets can be added, modified, or removed.
\end{definition}

\begin{table}[h!]
  \centering
  \begin{tabular}{p{5.7cm}p{5.7cm}p{5.7cm}}
    \toprule
    Threat Examples & What the Attacker Does & Notes \\
    \midrule
    \multirow{6}{5.7cm}{\nameref{par:Tampering_File}} & Modifies a file they own and on which you rely & \\ \cline{2-3}
                                               & Modifies a file you own & \\ \cline{2-3}
                                               & Modifies a file on a file server you own & \\ \cline{2-3}
                                               & Modifies a file on their file server & Loads of fun when you include files from remote domains \\ \cline{2-3}
                                               & Modifies a file on their file server & XML schemas include remote schemas \\ \cline{2-3}
    & Modifies links or redirects & \\
    \midrule
    \multirow{6}{5.7cm}{\nameref{par:Tampering_Memory}} & Modifies your code & Hard to defend against once attacker is running code as same user. \\ \cline{2-3}
    & Modifes data they've supplied to your API & Pass by value, not by reference when crossing a trust boundary \\
    \midrule
    \multirow{6}{5.7cm}{\nameref{par:Tampering_Network}} & Redirect flow of data to their machine & Often stage 1 of tampering \\ \cline{2-3}
                                               & Modifies data flowing over entwork & Even easier when the network is wireless (WiFi, 3G, LTE, etc.) \\ \cline{2-3}
    & Enhances spoofing attacks & \\
    \bottomrule
  \end{tabular}
  \caption{\nameref{def:Tampering} Threats}
  \label{tab:Tampering_Threats}
\end{table}

\paragraph{Tampering with a File}\label{par:Tampering_File}
Attackers can create/modify files if they have write permission.
If the code you wrote relies on files other things (people or files) wrote, there's a possibility it was written maliciously.

This can commonly happen when a website has been compromised and is serving malicious JavaScript.
They can also modify links that the website gives you to be compromised as well.
This same link manipulation can happen on your local disk as well.

\paragraph{Tampering with Memory}\label{par:Tampering_Memory}
Attacker can modify your code, if they are running at the same or higher privilege level.
This is difficult to find and handle, since the bad code is running at the same privilege and under the same user as the good code, and it is near impossible to tell them apart.

If you handle data in a pass-by-reference fashion, the attacker can modify it after security checks have been performed.

\paragraph{Tampering with a Network}\label{par:Tampering_Network}
There are a variety of tricks for this.
The attacker can forward some data that is intact and some that is modified.

Other tricks use the radio communication used for wireless data transmission.
Software-Defined Radio has made this type of attack trivial.

\subsubsection{Repudiation}\label{subsubsec:Repudiation}
\begin{definition}[Repudiation]\label{def:Repudiation}
  \emph{Repudiation} is claiming you didn't do something, or were not responsible for what happened.
  This can be done honestly or deceptively.

  Typically, this is done above any technical layer, and is instead in the business logic of buying products, for example.

  Repudiation often deals with logs and logging, which tracks everything that happened in order to find who/what was/is responsible.
\end{definition}

\begin{table}[h!]
  \centering
  \begin{tabular}{p{5.7cm}p{5.7cm}p{5.7cm}}
    \toprule
    Threat Examples & What the Attacker Does & Notes \\
    \midrule
    \multirow{5}{5.7cm}{Repudiating an action} & Claims to have not clicked & Maybe they really did \\ \cline{2-3}
                                               & Claims to have received & Maybe they did, but didn't read. Maybe cached. Maybe pre-fetched. \\ \cline{2-3}
                                               & Claims to have been a fraud victim & \\ \cline{2-3}
                                               & Uses someone else's account & \\ \cline{2-3}
                                               & Uses someone else's payment instrument without authorization & \\
    \midrule
    \multirow{2}{5.7cm}{Attacking the logs} & Notices you have no logs & \\ \cline{2-3}
                                               & Puts attacks in logs to confuse logs, log-reading code, or a person & \\
    \bottomrule
  \end{tabular}
  \caption{\nameref{def:Repudiation} Threats}
  \label{tab:Repudiation_Threats}
\end{table}

\paragraph{Attacking the Logs}\label{par:Attack_Logs}
If there is no logging, we don't retain logs, cannot analyze them, repudiation actions are difficult to dispute.
There needs to be log centralization and analysis capabilities.
There needs to be a clear definition of what is logged.

\paragraph{Repudiating an Action}\label{par:Repudiating_Action}
More useful to talk about ``someone'' than ``an attacker'' in this context.
Since many people who perform some action that may need to have repudiation are usually people that have been failed by technology or a process.

\subsubsection{Information Disclosure}\label{subsubsec:Information_Disclosure}
\begin{definition}[Information Disclosure]\label{def:Information_Disclosure}
  \emph{Information disclosure} is about allowing people to see information they are not authorized to see.
\end{definition}

\begin{table}[h!]
  \centering
  \begin{tabular}{p{5.7cm}p{5.7cm}p{5.7cm}}
    \toprule
    Threat Examples & What the Attacker Does & Notes \\
    \midrule
    \multirow{4}{5.7cm}{Information disclosure against a process} & Extract secrets from error messages & \\ \cline{2-3}
                    & Reads error messages from username/passwords to entire database tables & \\ \cline{2-3}
                    & Extract machine secrets from error messages & Makes defense against memory corruption less useful \\ \cline{2-3}
                    & Extract business/personal secrets from error cases & \\
    \midrule
    \multirow{9}{5.7cm}{Information disclosure against data stores} & Takes advantage of inappropriate or missing ACLs & \\ \cline{2-3}
                    & Takes advantage of bad database permissions & \\ \cline{2-3}
                    & Finds files protected by obscurity & \\ \cline{2-3}
                    & Finds cryptographic keys on disk or in memory & \\ \cline{2-3}
                    & Sees interesting information in metadata & \\ \cline{2-3}
                    & Reads files as they traverse the network & \\ \cline{2-3}
                    & Gets data from logs or temp files & \\ \cline{2-3}
                    & Gets data from swap or other temp storage & \\ \cline{2-3}
                    & Extracts data by obtaining device, changing OS & \\
    \midrule
    \multirow{5}{5.7cm}{Information disclosure against data flows} & Reads data on network & \\ \cline{2-3}
                    & Redirects traffic to enable reading data on network & \\ \cline{2-3}
                    & Learn secrets by analyzing traffic & \\ \cline{2-3}
                    & Learn who's talking to whom by watching DNS & \\ \cline{2-3}
                    & Learn who's talking to whom by social network info disclosure & \\
    \bottomrule
  \end{tabular}
  \caption{\nameref{def:Information_Disclosure} Threats}
  \label{tab:Information_Disclosure_Threats}
\end{table}

\paragraph{Information Disclosure from a Process}\label{par:Information_Disclosure_Process}
A process will disclose information that helps further attacks.
This can be done by leaking memory addresses, extracting secrets from error messages, or extracting design details from errors messages.

\paragraph{Information Disclosure from a Data Store}\label{par:Information_Disclosure_Data_Store}
Since data stores store data, they can leak it multiple ways:
\begin{itemize}[noitemsep]
\item Failure to properly use security mechanisms
\item Not setting permisions
\item Cryptographic keys being found/leaked
\item Files read over the network are readable over the network
\item Metadata (filenames, author, etc.) is also important
\item OS-level leaks when swapping things out
\item Data extracted from things using OS under attacker control.
\end{itemize}

\paragraph{Information Disclosure from a Data Flow}\label{par:Information_Disclosure_Data_Flow}
Data flows are particularly susceptible.
Data flows on a single machine that is shared among multiple mutually distrustful uses of a system is a main susceptibility.
Attackers can redirect information to themselves.
Attackers can gain information even when the traffic is encrypted.

\subsubsection{Denial of Service}\label{subsubsec:Denial_of_Service}
\begin{definition}[Denial of Service]\label{def:Denial_of_Service}
  \emph{Denial-of-service} attacks consume a resource that is needed to provide a service.
\end{definition}

\begin{table}[h!]
  \centering
  \begin{tabular}{p{5.7cm}p{5.7cm}p{5.7cm}}
    \toprule
    Threat Examples & What the Attacker Does & Notes \\
    \midrule
    \multirow{4}{5.7cm}{Denial of service against a process} & Absorbs memory & \\ \cline{2-3}
                    & Absorbs storage & \\ \cline{2-3}
                    & Absorbs CPU cycles & \\ \cline{2-3}
                    & Uses process as an amplifier & \\
    \midrule
    \multirow{2}{5.7cm}{Denial of service against a data store} & Fills data store up & \\ \cline{2-3}
                    & Makes enough requests to slow down the system & \\
    \midrule
    \multirow{1}{5.7cm}{Denial of service against a data flow} & Consumes network resources & \\
    \bottomrule
  \end{tabular}
  \caption{\nameref{def:Denial_of_Service} Threats}
  \label{tab:Denial_of_Service_Threats}
\end{table}

These attacks can be split into attacks that work while the attacker is actively attacking, and ones that occur persistently.
These can also be amplified or unamplified.
Amplified attacks are relatively small attacks that have big effects.

\subsubsection{Elevation of Privilege}\label{subsubsec:Elevation_of_Privilege}
\begin{definition}[Elevation of Privilege]\label{def:Elevation_of_Privilege}
  \emph{Elevation of privilege} is allowing someone to do something they are not authorized to to.
  For example, allowing a remote user to execute code as an administrator, or allowing a remote person without privileges to run code.
\end{definition}

\begin{table}[h!]
  \centering
  \begin{tabular}{p{5.7cm}p{5.7cm}p{5.7cm}}
    \toprule
    Threat Examples & What the Attacker Does & Notes \\
    \midrule
    \multirow{2}{5.7cm}{Elevation of privilege against a process by corrupting the process} & Send inputs that the code doesn't handle properly & These are very common and are usually high impact \\ \cline{2-3}
                    & Gains access to read/write memory inappropriately & Writing memory is bad, but reading memory allows further attacks. \\
    \midrule
    \multirow{1}{5.7cm}{Elevation through missed authorization checks} & & \\
    \midrule
    \multirow{1}{5.7cm}{Elevation through buggy authorization checks} & & Centralizing such checks makes bugs easier to manage \\
    \midrule
    \multirow{1}{5.7cm}{Elevation through data tampering} & Modifies bits on disk to do things other than what the authorized user intends & \\
    \bottomrule
  \end{tabular}
  \caption{\nameref{def:Elevation_of_Privilege} Threats}
  \label{tab:Elevation_of_Privilege_Threats}
\end{table}

\paragraph{Elevate Privileges by Corrupting a Process}\label{par:Elevation_of_Privilege_Corrupt_Process}
Corrupting a process can involve:
\begin{itemize}[noitemsep]
\item Smashing the stack
\item Exploiting information on the heap
\item Other techniques
\end{itemize}

These attacks allow the attacker to gain influence or control over a program's control flow.

\paragraph{Elevate Privileges through Authorization Failures}\label{par:Elevation_of_Privilege_Authorization_Failure}
If we don't check authorization on every path through a program, this can be a problem.
If there are buggy checks, the attacker can take advantage of them.
If a program relies on other programs, configuration files, or datasets being trustworthy, those are possible targets as well.

\subsubsection{STRIDE-per-Interaction}\label{subsubsec:STRIDE_per_Interaction}
This variant of \nameref{def:STRIDE_Analysis} involves applying the traditional \nameref{def:STRIDE_Analysis} to interations between elements in a system, rather than considering the system as a whole.

%%% Local Variables:
%%% mode: latex
%%% TeX-master: "../../EITP20-Secure_Systems_Engineering-Reference_Sheet"
%%% End:
