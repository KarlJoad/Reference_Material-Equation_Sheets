\subsection{Microsoft STRIDE Analysis}\label{subsec:STRIDE_Analysis}
\begin{definition}[STRIDE Analysis]\label{def:STRIDE_Analysis}
  \emph{STRIDE} is an acronym standing for
  \begin{itemize}[noitemsep]
  \item [S] \nameref{def:Spoofing}
  \item [T] \nameref{def:Tampering}
  \item [R] \nameref{def:Repudiation}
  \item [I] \nameref{def:Information_Disclosure}
  \item [D] \nameref{def:Denial_of_Service}
  \item [E] \nameref{def:Elevation_of_Privilege}
  \end{itemize}

  It is a model to identify computer security threats in software systems or components of a larger software system.
  It is general enough to analyze any part of a software system.

  To perform a STRIDE analysis, it is commonly applied with these steps:
  \begin{enumerate}[noitemsep]
  \item Identify the main entities in the system
  \item Identify the interactions between/among the main entities.
  \item For each entity, perform a STRIDE analysis on the following actions:
    \begin{itemize}[noitemsep]
    \item Processes
    \item External Entities
    \item Data Flows
    \item Data Stores
    \end{itemize}
  \end{enumerate}

  \begin{remark}[Incomplete Methodology]\label{rmk:STRIDE_Incomplete_Methodology}
    Like \nameref{def:Attack_Tree}s, \nameref{def:STRIDE_Analysis} is not a complete methodology.
    It only helps identify typical attacks, but is not a ``complete package''.
  \end{remark}
\end{definition}

\begin{table}[h!]
  \centering
  \begin{tabular}{p{1.5cm}p{2.75cm}p{4cm}p{4cm}p{4cm}}
    \toprule
    \multicolumn{1}{c}{Threat} & \multicolumn{1}{c}{Property Violated} & \multicolumn{1}{c}{Definition} & \multicolumn{1}{c}{Typical Victims} & \multicolumn{1}{c}{Example} \\
    \midrule
    \nameref{def:Spoofing} & Authentication & Pretending to be something or someone other than yourself & Processes, External Entities, People & Falsely claiming to be a police officer. \\ \midrule
    \nameref{def:Tampering} & Integrity & Modifying something on disk, on a network, or in memory & Data stores, data flows, processes & Changing spreadsheet, binary contents, database contents, modifying network packets, or the program running. \\ \midrule
    \nameref{def:Repudiation} & Non-Repudiation & Claiming that you didn't do something or were not responsible. & Process & Process: ``I didn't hit the big red button''. \\ \midrule
    \nameref{def:Information_Disclosure} & Confidentiality & Providing information to someone not authorized to see it & Processes, data stores, data flows & Allow access to file contents/file metadata, network packets, contents of program memory. \\ \midrule
    \nameref{def:Denial_of_Service} & Availability & Deliberately absorbing more resources than needed to provide a service & Processes, data stores, data flows & Program uses all available memory, File fills up disk, too many network connections for real traffic. \\ \midrule
    \nameref{def:Elevation_of_Privilege} & Authorization & Allowing someone to do something they're not authorized to do. & Process & Allow normal user to execute code as administrator. Remote person can run code. \\
    \bottomrule
  \end{tabular}
  \caption{STRIDE Acronym Definitions}
  \label{tab:STRIDE_Analysis_Definitions}
\end{table}
\subsubsection{Spoofing}\label{subsubsec:Spoofing}
\begin{definition}[Spoofing]\label{def:Spoofing}
  \emph{Spoofing} is pretending to be something or someone other than yourself.
  \Cref{tab:STRIDE_Analysis_Definitions} gives one example, there are several others.

  \begin{enumerate}[noitemsep]
  \item Spoofing the identity of an entity across a network.
    There is not mediating authority that takes reponsibility for telling that something is what it claims to be.
  \item Spoofing the identity of an entity, even when there is a mediating authority.
    For example, a modified \texttt{.dll} file is mediated by the operating system to make sure it is the right one.
    If it's modified, th emediating authority might let it through, even when it has been modified.
  \item Pretending to be a specific person, for example, Barack Obama.
  \item Pretending to be in a specific role.
  \end{enumerate}
\end{definition}

\subsubsection{Tampering}\label{subsubsec:Tampering}
\begin{definition}[Tampering]\label{def:Tampering}
  \emph{Tampering} is modifying something, typically on disk, on a network, or in memory.
  This can include changing the data in a spreadsheet, changing a binary or config file on dis, modifying a database, etc.
  On a network, packets can be added, modified, or removed.
\end{definition}

\subsubsection{Repudiation}\label{subsubsec:Repudiation}
\begin{definition}[Repudiation]\label{def:Repudiation}
  \emph{Repudiation} is claiming you didn't do something, or were not responsible for what happened.
  This can be done honestly or deceptively.

  Typically, this is done above any technical layer, and is instead in the business logic of buying products, for example.

  Repudiation often deals with logs and logging, which tracks everything that happened in order to find who/what was/is responsible.
\end{definition}

\subsubsection{Information Disclosure}\label{subsubsec:Information_Disclosure}
\begin{definition}[Information Disclosure]\label{def:Information_Disclosure}
  \emph{Information disclosure} is about allowing people to see information they are not authorized to see.
\end{definition}

\subsubsection{Denial of Service}\label{subsubsec:Denial_of_Service}
\begin{definition}[Denial of Service]\label{def:Denial_of_Service}
  \emph{Denial-of-service} attacks consume a resource that is needed to provide a service.
\end{definition}

\subsubsection{Elevation of Privilege}\label{subsubsec:Elevation_of_Privilege}
\begin{definition}[Elevation of Privilege]\label{def:Elevation_of_Privilege}
  \emph{Elevation of privilege} is allowing someone to do something they are not authorized to to.
  For example, allowing a remote user to execute code as an administrator, or allowing a remote person without privileges to run code.
\end{definition}

%%% Local Variables:
%%% mode: latex
%%% TeX-master: "../../EITP20-Secure_Systems_Engineering-Reference_Sheet"
%%% End:
