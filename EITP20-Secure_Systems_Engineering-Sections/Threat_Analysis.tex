\section{Threat Analysis}\label{sec:Threat_Analysis}
To perform any kind of threat analysis, the system's specification must be considered.
The state of the system (new or existing) must also be considered.
\begin{itemize}[noitemsep]
\item The specification of the system to be analyzed
  \begin{itemize}[noitemsep]
  \item If a completely new system is about to be built, there might not even exist a system specification and it needs to be created.
  \item If the analysis is to done on an existing system, the first task is to read existing system specifications, source code and/or make interviews with people familiar with the system.
  \end{itemize}

\item The specification must not necessarily be very detailed but can be a high-level description of the system.
\end{itemize}

\begin{definition}[Attack Goal]\label{def:Attack_Goal}
  An \emph{attack goal} is a technology, process, or thing that an attacker would like to gain access to.
  These are the things we are trying to protect against or mitigate.
\end{definition}

\subsection{Attack Trees}\label{subsec:Attack_Trees}
\begin{definition}[Attack Tree]\label{def:Attack_Tree}
  The Schneider \emph{attack tree} method is a straight-forward and step-by-step type of attack analysis.
  However, it is quite basic, so it might not be the best method to perform this analysis.

  The steps involved are:
  \begin{enumerate}[noitemsep]
  \item The starting point is a \emph{good} system description.
  \item Next, \nameref{def:Attack_Goal}s are identified.
  \item \nameref{def:Attack_Goal}s are then broken down to specific attacks to form a so-called attack tree. Identify different attack vectors for the same \nameref{def:Attack_Goal} with several iterations.
  \item Identify dependencies in the \nameref{def:Attack_Goal}s.
  \item Once the attack tree is created, it is transferred to a table where scores on risks/costs can be added as well as more details.
  \end{enumerate}
\end{definition}

In a Single Sign-On system (SSO), there any many different points of failure.
These will be illustrated with a basic \nameref{def:Attack_Tree}.
Some \nameref{def:Attack_Goal}s for this system are:
\begin{enumerate}[noitemsep]
\item Get access to all services provided by all entities in the system
  \begin{enumerate}[noitemsep]
  \item Get access to all services provided to all users at the ID Provider
  \item Get access to all services offered through an OIDC Client
  \item Get access to all services offered at the Hosting Server
  \end{enumerate}

\item Get access to all services provided to a single user provided by the ID Provider
\item Get access to all services provided to single user at the OIDC Client
\item Get access to all services provided to a single user at the Hosting Service
\item Get access to all services provided to an OIDC Client at the Hosting Service
\end{enumerate}

Now, we identify dependencies in the various \nameref{def:Attack_Goal}s.
\begin{itemize}[noitemsep]
\item \nameref{def:Attack_Goal} 1 depends on goals 2, 3, 4, and 5 being completed.
\item Goal 1.1 depends on goal 2 being completed.
\item Goal 1.2 depends on goal 3 being completed.
\item Goal 1.3 depends on goals 4 and 5 being completed.
\item Goal 5 depends on goal 4 being completed.
\end{itemize}

We can now break each of the smaller goals down into concrete attack vectors.
\begin{nestednums}
\item Access to all services provided to all entities in the system.
  \begin{nestednums}
  \item Access to all services offered by the ID Provider to all users.
    \begin{nestednums}
    \item Get KeyCloak admin rights.
      \begin{nestednums}
      \item Successful phishing attack.
      \item Break administrator authentication.
      \end{nestednums}
    \item Root access on the KeyCloak server.
      \begin{nestednums}
      \item Utilize an OS vulnerability
      \item Break server isolation (For example, Docker or VMs)
      \end{nestednums}
    \end{nestednums}
  \end{nestednums}
\item Access to all services provided to a single user by the ID Provider
  \begin{nestednums}
  \item Successful end-user password phishing attack.
  \item Take over the end-user's client device.
    \begin{nestednums}
    \item Place malware on the client device.
    \item Successful network attack on the end-user's client device.
    \end{nestednums}
  \item Break the end-user authentication mechanism.
    \begin{nestednums}
    \item Break authentication algorithm. (Difficult, if not impossible)
    \item Find flaw in authentication protocol.
    \item Find flaw in authentication algorithm implementation on the client or on the server.
    \end{nestednums}
  \end{nestednums}
\end{nestednums}

As this ``tree'' shows, the when accounting for attack vectors, it can become quite large.
You should start with the obvious cases, and move onto the les obvious as you go.
The most important thing while doing this is you should think like an attacker.

\begin{remark*}
  Depending on the information of the system available, a very detailed attack break-down might be possible or not. The tree can later be complemented when more implementation information is available
\end{remark*}

\begin{remark*}
  The attack tree is a useful tool, but is not a final or complete solution to a security analysis problem.
\end{remark*}

\subsection{Example Application of Attack Trees}\label{subsec:Attack_Tree_Example}

\subsection{Microsoft STRIDE Analysis}\label{subsec:STRIDE_Analysis}

\subsection{MITRE TARA Analysis}\label{subsec:TARA_Analysis}

%%% Local Variables:
%%% mode: latex
%%% TeX-master: "../EITP20-Secure_Systems_Engineering-Reference_Sheet"
%%% End: