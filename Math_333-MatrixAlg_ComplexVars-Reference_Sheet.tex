\documentclass[10pt,letterpaper,final,twoside,notitlepage]{article}
\usepackage[margin=.5in]{geometry} % 1/2 inch margins on all pages
\usepackage[utf8]{inputenc} % Define the input encoding
\usepackage[USenglish]{babel} % Define language used
\usepackage{amsmath}
\usepackage{mathtools} % Allow for text and math in align* environment.
\usepackage{amsfonts}
\usepackage{amssymb}
\usepackage{amsthm} % Gives us plain, definition, and remark to use in \theoremstyle{style}
\usepackage{thmtools}
\usepackage{thm-restate}
\usepackage{graphicx}

\usepackage[
backend=biber,
style=alphabetic,
citestyle=authoryear]{biblatex} % Must include citation somewhere in document to print bibliography
\usepackage{hyperref} % Generate hyperlinks to referenced items
\usepackage[nottoc]{tocbibind} % Prints the Reference/Bibliography in TOC as well
\usepackage[noabbrev,nameinlink]{cleveref} % Fancy cross-references in the document everywhere
\usepackage{nameref} % Can make references by name to places
\usepackage{caption} % Allows for greater control over captions in figure, algorithm, table, etc. environments
\usepackage{subcaption} % Allows for multiple figures in one Figure environment
\usepackage[binary-units=true]{siunitx} % Gives us ways to typeset units for stuff
\usepackage{csquotes} % Context-sensitive quotation facilities
\usepackage{enumitem} % Provides [noitemsep, nolistsep] for more compact lists
\usepackage{chngcntr} % Allows us to tamper with the counter a little more
\usepackage{empheq} % Allow boxing of equations in special math environments
\usepackage[x11names]{xcolor} % Gives access to coloring text in environments or just text, MUST be before tikz
\usepackage{tcolorbox} % Allows us to create boxes of various types for examples
\usepackage{tikz} % Allows us to create TikZ and PGF Pictures
\usepackage{ctable} % Greater control over tables and how they look
\usepackage{multirow} % Allow us to have a single cell in a table span multiple rows
\usepackage{titling} % Put document information throughout the document programmatically
\usepackage[linesnumbered,ruled,vlined]{algorithm2e} % Allows us to write algorithms in a nice style.

\counterwithin{figure}{section}
\counterwithin{table}{section}
\counterwithin{equation}{section}
\counterwithin{algocf}{section}
\crefname{algocf}{algorithm}{algorithms}
\Crefname{algocf}{Algorithm}{Algorithms}
\setcounter{secnumdepth}{4}
\setcounter{tocdepth}{4} % Include \paragraph in toc
\crefname{paragraph}{paragraph}{paragraphs}
\Crefname{paragraph}{Paragraph}{Paragraphs}

% Create a theorem environment
\theoremstyle{plain}
\newtheorem{theorem}{Theorem}[section]
% Create a numbered theorem-like environment for lemmas
\newtheorem{lemma}{Lemma}[theorem]

% Create a definition environment
\theoremstyle{definition}
\newtheorem{definition}{Defn}
\newtheorem{corollary}{Corollary}[section]
% \begin{definition}[Term] \label{def:}
% 		Make sure the term is emphasized with \emph{term}.
%		This ensures that if \emph is changed, it shows up everywhere
% \end{definition}

% Create a numbered remark environment numbered based on definition
% NOTE: This version of remark MUST go inside a definition environment
\theoremstyle{remark}
\newtheorem{remark}{Remark}[definition]
%\counterwithin{definition}{subsection} % Uncomment to have definitions use section.subsection numbering

% Create an unnumbered remark environment for general use
% NOTE: This version of remark has NO restrictions on placement
\newtheorem*{remark*}{Remark}

% Create a special list that handles properties. It can be broken and restarted
\newlist{propertylist}{enumerate}{1} % {Name}{Template}{Max Depth}
% [newlistname, LevelsToApplyTo]{formatting options}
\setlist[propertylist, 1]{label=\textbf{(\roman*)}, ref=\textbf{(\roman*)}, noitemsep, nolistsep}
\crefname{propertylisti}{property}{properties}
\Crefname{propertylisti}{Property}{Properties}

% Create a special list that handles enumerate starting with lower letters. Breakable/Restartable.
\newlist{boldalphlist}{enumerate}{1} % {Name}{Template}{Max Depth}
% [newlistname, LevelsToApplyTo]{formatting options}
\setlist[boldalphlist, 1]{label=\textbf{(\alph*)}, ref=\alph*, noitemsep, nolistsep} % Set options

\newlist{nocrefenumerate}{enumerate}{1} % {Name}{Template}{Max Depth}
% [newlistname, LevelsToApplyTo]{formatting options}
\setlist[nocrefenumerate, 1]{label=(\arabic*), ref=(\arabic*), noitemsep, nolistsep}

% Create a list that allows for deeper nesting of numbers. By default enumerate only allows depth=4.
\newlist{nestednums}{enumerate}{6}
% [newlistname, LevelsToApplyTo]{formatting options}
\setlist[nestednums]{noitemsep, label*=\arabic*.}

\tcbuselibrary{breakable} % Allow tcolorboxes to be broken across pages
% Create a tcolorbox for examples
% /begin{example}[extra name]{NAME}
% Create a tcolorbox for examples
% Argument #1 is optional, given by [], that is the textbook's problem number
% Argument #2 is mandatory, given by {}, that is the title for the example
% Avoid putting special characters, (), [], {}, ",", etc. in the title.
\newtcolorbox[auto counter,
number within=section,
number format=\arabic,
crefname={example}{examples}, % Define reference format for cref (No Capitals)
Crefname={Example}{Examples}, % Reference format for cleveref (With Capitals)
]{example}[2][]{ % The [2][] Means the first argument is optional
  width=\textwidth,
  title={Example \thetcbcounter: #2. #1}, % Parentheses and commas are not well supported
  fonttitle=\bfseries,
  label={ex:#2},
  nameref=#2,
  colbacktitle=white!100!black,
  coltitle=black!100!white,
  colback=white!100!black,
  upperbox=visible,
  lowerbox=visible,
  sharp corners=all,
  breakable
}

% Create a tcolorbox for general use
\newtcolorbox[% auto counter,
% number within=section,
% number format=\arabic,
% crefname={example}{examples}, % Define reference format for cref (No Capitals)
% Crefname={Example}{Examples}, % Reference format for cleveref (With Capitals)
]{blackbox}{
  width=\textwidth,
  % title={},
  fonttitle=\bfseries,
  % label={},
  % nameref=,
  colbacktitle=white!100!black,
  coltitle=black!100!white,
  colback=white!100!black,
  upperbox=visible,
  lowerbox=visible,
  sharp corners=all
}

% Redefine the 'end of proof' symbol to be a black square, not blank
\renewcommand\qedsymbol{$\blacksquare$} % Change proofs to have black square at end

\renewcommand{\Re}{\operatorname{Re}} % Redefine to use the command, but not the fraktur version
\renewcommand{\Im}{\operatorname{Im}} % Redefine to use the command, but not the fraktur version
% Math Operators that are useful to abstract the written math away to one spot
\DeclareMathOperator{\RealNumbers}{\mathbb{R}}
\newcommand{\TextRealNumbers}{$\RealNumbers$}
\DeclareMathOperator{\AllIntegers}{\mathbb{Z}}
\newcommand{\TextAllIntegers}{$\AllIntegers$}
\DeclareMathOperator{\PositiveInts}{\mathbb{Z}^{+}}
\newcommand{\TextPositiveInts}{$\PositiveInts$}
\DeclareMathOperator{\NegativeInts}{\mathbb{Z}^{-}}
\newcommand{\TextNegativeInts}{$\NegativeInts$}
\DeclareMathOperator{\NaturalNumbers}{\mathbb{N}}
\newcommand{\TextNaturalNumbers}{$\NaturalNumbers$}
\DeclareMathOperator{\ComplexNumbers}{\mathbb{C}}
\newcommand{\TextComplexNumbers}{$\ComplexNumbers$}
\DeclareMathOperator{\RationalNumbers}{\mathbb{Q}}
\newcommand{\TextRationalNumbers}{$\RationalNumbers$}
\DeclareMathOperator*{\argmax}{argmax} % Thin Space and subscripts are UNDER in display
\DeclareMathOperator{\Lapl}{\mathcal{L}} % Declare a Laplace symbol to be used
\DeclareMathOperator{\UnitStep}{\mathcal{U}}
\DeclareMathOperator{\sinc}{sinc} % sinc(x) = (sin(pi x)/(pi x))
\DeclareMathOperator{\XOR}{\oplus}

\newcommand{\rbpRegister}{\texttt{\%rbp}}
\newcommand{\rspRegister}{\texttt{\%rsp}}
\newcommand{\ripRegister}{\texttt{\%rip}}
\newcommand{\raxRegister}{\texttt{\%rax}}
\newcommand{\rbxRegister}{\texttt{\%rbx}}

%%% Local Variables:
%%% mode: latex
%%% TeX-master: shared
%%% End:


% These packages are more specific to certain documents, but will be availabe in the template
% \usepackage{esint} % Provides us with more types of integral symbols to use
% \usepackage[outputdir=./TeX_Output]{minted} % Allow us to nicely typeset 300+ programming languages
% \crefname{lstlisting}{listing}{listings}
% \Crefname{lstlisting}{Listing}{Listings}
% This document must be compiled with the -shell-escape flag if the packages above are uncommented

\graphicspath{{./Drawings/Math_333-MatrixAlg_ComplexVars/}} % Uncomment this to use pictures in this document
% \addbibresource{./Bibliographies/CourseNum-Name.bib}

\usepackage{nth}
% Math Operators that are useful to abstract the written math away to one spot
% These are supposed to be document-specific mathematical operators that will make life easier
% Many fundamental operators are defined in Reference_Sheet_Preamble.tex

\newcommand{\Path}{\ensuremath{\gamma}} % Path in complex plane to approach a point a.

% Taken from https://gitlab.com/jim.hefferon/linear-algebra/-/blob/master/src/sty/linalgjh.sty
%--------grstep
% For denoting a Gauss' reduction step.
% Use as: \grstep{\rho_1+\rho_3} % Step above arrow
% \grstep[2\rho_5 \\ 3\rho_6]{\rho_1+\rho_3} % Step above arrow, additional ones below
% \newcommand{\grstep}[2][\relax]{%
%    \ensuremath{\mathrel{
%        \mathop{\longrightarrow}\limits^{#2\mathstrut}_{
%                                    \begin{subarray}{l} #1 \end{subarray}}}}}

% Advantage of length formulation is that between adjacent
% \grstep's you can add \hspace{-\grsteplength} to make it look not too wide
\newlength{\grsteplength}
\setlength{\grsteplength}{1.5ex plus .1ex minus .1ex}

\newcommand{\grstep}[2][\relax]{%
   \ensuremath{\mathrel{
       \hspace{\grsteplength}\mathop{\longrightarrow}\limits^{#2\mathstrut}_{
                                     \begin{subarray}{l} #1 \end{subarray}}\hspace{\grsteplength}}}}
%-------------amatrix
% Augmented matrix.  Usage (note the argument does not count the aug col):
% \begin{amatrix}{2}
%   1  2  3 \\  4  5  6
% \end{amatrix}
\newenvironment{amatrix}[1]{%
  \left(\begin{array}{@{}*{#1}{c}|c@{}}
}{%
  \end{array}\right)
}
\newenvironment{amat}[2][c]{%
  % disable optional arg \left(\begin{array}{@{}*{#2}{#1}|#1@{}}
  \left(\begin{array}{@{}*{#2}{c}|#1@{}}
}{%
  \end{array}\right)
}

\renewcommand*{\arraystretch}{1.2}
\newcommand{\by}{\ensuremath{\times}}

\DeclareMathOperator{\Span}{span}

\begin{titlepage}
  \title{Math 333: Matrix Algebra and Complex Variables --- Reference Material \\ Illinois Institute of Technology}
  \author{Karl Hallsby}
  \date{Last Edited: \today} % We want to inform people when this document was last edited
\end{titlepage}

\begin{document}
\pagenumbering{gobble}
\maketitle
\pagenumbering{roman} % i, ii, iii on beginning pages, that don't have content
\tableofcontents
\clearpage
\listoftheorems[ignoreall, show={definition, Definition}]
\clearpage
\pagenumbering{arabic} % 1,2,3 on content pages

\section{Complex Numbers}
	\begin{equation} \label{eq:Exponential to Rectangular}
		A e^{-ix} = A \left[ \cos \left( x \right) + i\sin \left( x \right) \right]
	\end{equation}

\section{Complex Functions}\label{sec:Complex_Functions}
Complex functions, like their real-valued counterparts behave in much the same way.
\begin{equation}\label{eq:Complex_Function}
  f(z) = w
\end{equation}
\begin{description}[noitemsep]
\item $f$: The function or \nameref{def:Mapping} that corresponds the input to the output.
\item $z$: The input to the complex function/\nameref{def:Mapping}.
\item $w$: The output of the complex function/\nameref{def:Mapping}.
\end{description}


%%% Local Variables:
%%% mode: latex
%%% TeX-master: "../Math_333-MatrixAlg_ComplexVars-Reference_Sheet"
%%% End:


\section{Complex Series}\label{sec:Complex_Series}
We start our discussion of complex series by talking about \nameref{def:Complex_Power_Series}.

\begin{definition}[Power Series]\label{def:Complex_Power_Series}
  A \emph{power series} is an infinite summation of terms that form infinitely long polynomials.
  These are defined around a center $a$, $a \in \ComplexNumbers$.
  \begin{equation}\label{eq:Complex_Power_Series}
    \sum_{n=0}^{\infty} a_{n} {(z-a)}^{n} = a_{0} + a_{1} (z-a) + a_{2} {(z-a)}^{2} + \cdots
  \end{equation}
\end{definition}


%%% Local Variables:
%%% mode: latex
%%% TeX-master: "../Math_333-MatrixAlg_ComplexVars-Reference_Sheet"
%%% End:


% At this point, we have finished all the complex analysis we will be doing in
% this course, so now we are moving onto the linear/matrix algebra. A brand new
% page will be a nice thing to start with.
\clearpage

\section{Matrix Algebra}\label{sec:Matrix_Algebra}
\begin{definition}[Matrix]\label{def:Matrix}
  A \emph{matrix} is an array of numbers subject to special addition and multiplication rules.
  They are represented with capital letters, $A$, $B$, etc.
  However, other texts may bold those capital letters to, $\mathbf{A}$, $\mathbf{B}$, etc.
  When written, they can be written in one of two ways, but they both mean the same thing.
  \begin{equation}\label{eq:Matrix}
    \begin{aligned}
      &\begin{pmatrix}
        a_{11} & a_{12} \\
        a_{21} & a_{22} \\
      \end{pmatrix} \\
      &\begin{bmatrix}
        a_{11} & a_{12} \\
        a_{21} & a_{22} \\
      \end{bmatrix}
    \end{aligned}
  \end{equation}
\end{definition}

Matrices are very useful, as learning the intricacies of them allows for quick solving of larger problems.

\subsection{Properties of Matrices}\label{subsec:Properties_Matrices}
\subsubsection{Properties of Matrix Addition}\label{subsubsec:Properties_Matrix_Addition}
\begin{propertylist}
\item \nameref{def:Matrix} Addition is commutative.\label{prop:Matrix_Add_Commutative}
  \begin{equation}\label{eq:Matrix_Add_Commutative}
    A+B = B+A
  \end{equation}

\item \nameref{def:Matrix} Addition is associative.\label{prop:Matrix_Add_Associative}
  \begin{equation}\label{eq:Matrix_Add_Associative}
    A+(B+C) = (A+B) + C
  \end{equation}

\item For any \nameref{def:Matrix} $A$, there is a unique matrix, $I$ or $O$, such that the equation below holds true.\label{prop:Matrix_Additive_Identity}
  \begin{equation}\label{eq:Matrix_Additive_Identity}
    A+O = A
  \end{equation}

\item For each \nameref{def:Matrix} $A$, there is a unique matrix $-A$ such that the equation below holds true.\label{prop:Matrix_Additive_Inverse}
  \begin{equation}\label{eq:Matrix_Additive_Inverse}
    A+(-A) = O
  \end{equation}

\item $A+B$ is a \nameref{def:Matrix} of the same dimensions as $A$ and $B$.\label{prop:Matrix_Addition_Closure}
\end{propertylist}

\subsubsection{Properties of Matrix Multiplication}\label{subsubsec:Properties_Matrix_Multiplication}
Matrix multiplication is defined as follows:
\begin{equation}\label{eq:Matrix_Multiplication}
  \begin{bmatrix}
    a_{11} & a_{12} & a_{13} & \cdots & a_{1n} \\
    a_{21} & a_{22} & \ddots & \ddots & a_{2n} \\
    \vdots & \ddots & \ddots & \ddots & a_{mn} \\
    a_{n\,1} & a_{n\,2} & \cdots & \cdots & a_{nn}
  \end{bmatrix}
\end{equation}

\begin{propertylist}
\item In general, \nameref{def:Matrix} multiplication is \textbf{NOT} commutative.\label{prop:Matrix_Not_Commutative}
  \begin{equation}\label{eq:Matrix_Mult_Not_Commutative}
    AB \neq BA
  \end{equation}

\item \nameref{def:Matrix} multiplication is associative.\label{prop:Matrix_Associativity}
  \begin{equation}\label{eq:Matrix_Mult_Associativity}
    (AB) C = A (BC)
  \end{equation}
\item \nameref{def:Matrix} multiplication is distributive.\label{prop:Matrix_Distributivity}
  \begin{equation}\label{eq:Matrix_Mult_Distributivity}
    \begin{aligned}
      A (B+C) &= AB + AC \\
      (B+C) A &= BA + CA
    \end{aligned}
  \end{equation}

\item There exists a \nameref{def:Matrix} $I$ that forms the \nameref{def:Multiplicative_Identity_Matrix} that satisfies the following rule:\label{prop:Mult_Identity_Matrix}
  \begin{equation}\label{eq:Mult_Identity_Matrix}
    \begin{aligned}
      AI &= A \\
      IA &= A
    \end{aligned}
  \end{equation}
  \begin{remark*}
    This is one of the two general cases where \nameref{def:Matrix} multiplication \textbf{IS} commutative.
  \end{remark*}

\item There exists a \nameref{def:Matrix} $O$ that forms the \nameref{def:Zero_Matrix}, satisfying the following rule:\label{prop:Mult_Zero_Matrix}
  \begin{equation}\label{eq:Mult_Zero_Matrix}
    \begin{aligned}
      AO &= O \\
      OA &= O
    \end{aligned}
  \end{equation}
  \begin{remark*}
    This is one of the two general cases where \nameref{def:Matrix} multiplication \textbf{IS} commutative.
  \end{remark*}

\item When performing multiplication, the \textbf{dimension property} states that the product of an $m \by n$ and an $n \by k$ matrix is an $m \by k$ \nameref{def:Matrix}.\label{prop:Mult_Dimension_Property}
\end{propertylist}

\begin{definition}[Identity Matrix]\label{def:Multiplicative_Identity_Matrix}
  The \emph{identity matrix}, $I$ is one that satisfies \Cref{prop:Mult_Identity_Matrix}.
  These follow the forms of the ones shown below.
  \begin{equation}\label{eq:Multiplicative_Identity_Matrix}
    \begin{aligned}
      I &=
      \begin{bmatrix}
        1 & 0 \\
        0 & 1 \\
      \end{bmatrix} \\
      &=
      \begin{bmatrix}
        1 & 0 & 0 \\
        0 & 1 & 0 \\
        0 & 0 & 1 \\
      \end{bmatrix}
    \end{aligned}
  \end{equation}
\end{definition}

\begin{definition}[Zero Matrix]\label{def:Zero_Matrix}
  The \emph{zero matrix}, $O$, is one that satisfies \Cref{prop:Mult_Zero_Matrix}.
  For \nameref{def:Matrix} multiplication, it is a matrix of all $0$.
  \begin{equation}
    \label{eq:3}
    \begin{aligned}
      O &=
      \begin{bmatrix}
        0 & 0 \\
        0 & 0 \\
      \end{bmatrix} \\
      &=
      \begin{bmatrix}
        0 & 0 & 0 \\
        0 & 0 & 0 \\
        0 & 0 & 0 \\
      \end{bmatrix}
    \end{aligned}
  \end{equation}
\end{definition}




%%% Local Variables:
%%% mode: latex
%%% TeX-master: "../../Math_333-MatrixAlg_ComplexVars-Reference_Sheet"
%%% End:


\subsection{Systems of Equations}\label{subsec:Systems_Equations}
A \nameref{def:Matrix} can be used to represent a system of equations.

For example,
\begin{align*}
  x + 2y + 3z &= 5 \\
  2x - y + z &= 3 \\
  x - y + 5z &= 2
\end{align*}
can be represented with a set of matrices as follows
\begin{equation*}
  \begin{bmatrix}
    1 & 2 & 3 \\
    2 & -1 & 1 \\
    1 & -1 & 5 \\
  \end{bmatrix}
  \begin{bmatrix}
    x \\
    y \\
    z \\
  \end{bmatrix}
  =
  \begin{bmatrix}
    5 \\
    3 \\
    2 \\
  \end{bmatrix}
\end{equation*}


%%% Local Variables:
%%% mode: latex
%%% TeX-master: "../../Math_333-MatrixAlg_ComplexVars-Reference_Sheet"
%%% End:


\subsection{Elementary Row Operations}\label{subsec:Elementary_Row_Ops}
\nameref{def:Elementary_Row_Op}s are three operations that are commonly used to reduce matrices to an \nameref{thm:Echelon_Form}.

\begin{definition}[Elementary Row Operation]\label{def:Elementary_Row_Op}
  \emph{Elementary row operation}s are \nameref{def:Matrix} operations that can be applied to the \textbf{ROWS} of an \nameref{def:Augmented_Matrix}.
  These operations change the \textbf{appearance} of the rows in a matrix, but the solution remains unchanged.
  Using these operations on a matrix yields a \emph{row-equivalent} matrix, not an equivalent matrix.
\end{definition}

The three \nameref{def:Elementary_Row_Op}s are:
\begin{enumerate}[noitemsep]
\item Interchange rows.
\item Multiply a row by a non-zero scalar number.
\item Multiply a row by a non-zero scalar number, and add that row's product to a row below.
\end{enumerate}


%%% Local Variables:
%%% mode: latex
%%% TeX-master: "../../Math_333-MatrixAlg_ComplexVars-Reference_Sheet"
%%% End:


\subsection{Echelon Form}\label{subsec:Echelon_Form}
\begin{theorem}[Echelon Form]\label{thm:Echelon_Form}
  A \nameref{def:Matrix} $A =
  \begin{bmatrix}
    r_{1} \\
    r_{2} \\
    \vdots \\
    r_{n}
  \end{bmatrix}
  $ is said to be in \emph{Echelon form} if the first non-zero element of the $i$th row, $r_{i}$ where $i \geq 2$ appears on a column that is to the right from which the first non-zero element of the $r_{i-1}$ appears.

  Some examples of this are:
  \begin{equation*}
    \begin{bmatrix}
      0 & 1 & 2 & 3 & 4 & 5 \\
      0 & 0 & 5 & 6 & 0 & 0 \\
      0 & 0 & 0 & 0 & 7 & 0 \\
      0 & 0 & 0 & 0 & 0 & 1 \\
      0 & 0 & 0 & 0 & 0 & 0 \\
    \end{bmatrix}
  \end{equation*}
\end{theorem}

If we combine an \nameref{def:Augmented_Matrix} with \nameref{def:Elementary_Row_Op}s to yield an such a matrix in \nameref{thm:Echelon_Form}, it becomes easy to solve systems of equations.
This is called \nameref{def:Gaussian_Elimination}.

%%% Local Variables:
%%% mode: latex
%%% TeX-master: "../../Math_333-MatrixAlg_ComplexVars-Reference_Sheet"
%%% End:


\subsection{Gaussian Elimination}\label{subsec:Gaussian_Elimination}

%%% Local Variables:
%%% mode: latex
%%% TeX-master: "../../Math_333-MatrixAlg_ComplexVars-Reference_Sheet"
%%% End:


\subsection{Elementary Matrix Operations}\label{subsec:Elementary_Matrix_Ops}
\begin{definition}[Elementary Matrix]\label{def:Elementary_Matrix}
  An \emph{elementary matrix} is a \nameref{def:Matrix} obtained by performing an \nameref{def:Elementary_Row_Op} on the \nameref{def:Multiplicative_Identity_Matrix}.
  This allows us to encode the action we take on the identity matrix \textbf{into} the identity matrix, essentially saving it.
\end{definition}

\begin{blackbox}
  Given the \nameref{def:Multiplicative_Identity_Matrix}, $I$, a regular \nameref{def:Matrix} $A$, perform a row reduction by adding $r_{1}$, multiplied by $\lambda$, of $A$ to $r_{2}$.
  \begin{align*}
    I &= \begin{bmatrix}
      1 & 0 \\
      0 & 1
    \end{bmatrix} \\
    A &= \begin{bmatrix}
      a_{11} & a_{12} \\
      a_{21} & a_{22}
    \end{bmatrix}
  \end{align*}

  We perform the \nameref{def:Elementary_Row_Op} on the \nameref{def:Multiplicative_Identity_Matrix}.
  \begin{equation*}
    \begin{bmatrix}
      1 & 0 \\
      0 & 1
    \end{bmatrix}
    \grstep{\lambda r_{1}+r_{2}}
    \begin{bmatrix}
      1 & 0 \\
      \lambda & 1
    \end{bmatrix}
  \end{equation*}

  Now that we have an \nameref{def:Elementary_Matrix}, we can perform the action we saved in the elementary matrix on any matrix, in this case, $A$.
  \begin{equation*}
    \begin{bmatrix}
      1 & 0 \\
      \lambda & 1
    \end{bmatrix}
    \begin{bmatrix}
      a_{11} & a_{12} \\
      a_{21} & a_{22}
    \end{bmatrix}
    =
    \begin{bmatrix}
      a_{11} & a_{12} \\
      \lambda a_{11} + a_{21} & \lambda a_{12} + a_{22}
    \end{bmatrix}
  \end{equation*}

  We can see that this result is the exact same as if we have performed the \nameref{def:Elementary_Row_Op} on the matrix $A$ directly.
\end{blackbox}


%%% Local Variables:
%%% mode: latex
%%% TeX-master: "../../Math_333-MatrixAlg_ComplexVars-Reference_Sheet"
%%% End:


\subsection{Inverse Matrices}\label{subsec:Inverse_Matrices}
\begin{definition}[Inverse Matrix]\label{def:Inverse_Matrix}
  Let $A_{n \by n}$ be an $n \by n$ \nameref{def:Matrix}.
  A matrix $B$ is called an inverse of $A$ if and only if
  \begin{equation}\label{eq:Inverse_Matrix}
    AB = I
  \end{equation}
\end{definition}


%%% Local Variables:
%%% mode: latex
%%% TeX-master: "../../Math_333-MatrixAlg_ComplexVars-Reference_Sheet"
%%% End:


\subsection{Reduced Row Echelon Form}\label{subsec:RREF}
Using what we learned in \Cref{subsec:Inverse_Matrices} and \Cref{subsubsec:Inverse_Matrices_using_Elementary_Matrices}, we can solve systems of linear equations using a \nameref{def:Matrix}'s inverse.

This can be movtivated with a general example.
\begin{blackbox}
  Consider the system shown below.
  \begin{align*}
    a_{11} x_{1} + a_{12} x_{2} &= b_{1} \\
    a_{21} x_{1} + a_{22} x_{2} &= b_{2}
  \end{align*}

  We can convert this set of equations with 2 equations and 2 unknowns to a single linear equation with a single unknown.
  \textbf{Equivalently} to the system, we can write
  \begin{align*}
    \begin{pmatrix}
      a_{11} & a_{12} \\
      a_{21} & a_{22}
    \end{pmatrix}
    \begin{pmatrix}
      x_{1} \\
      x_{2}
    \end{pmatrix} &=
    \begin{pmatrix}
      a_{11} x_{1} + a_{12} x_{2} \\
      a_{21} x_{1} + a_{22} x_{2}
    \end{pmatrix} \\
    &=
      \begin{pmatrix}
        b_{1} \\
        b_{2}
      \end{pmatrix}
  \end{align*}

  The column matrices (column vectors) are technically one unknown with multiple unknown components.

  Just like with regular algebra, we can use the inverse of the items we know to solve the equation.
  \begin{align*}
    \intertext{Let}
    A &=
        \begin{pmatrix}
          a_{11} & a_{12} \\
          a_{21} & a_{22}
        \end{pmatrix} \\
    X &=
        \begin{pmatrix}
          x_{1} \\
          x_{2}
        \end{pmatrix} \\
    B &=
        \begin{pmatrix}
          b_{1} \\
          b_{2}
        \end{pmatrix} \\
    AX &= B \\
    \shortintertext{We make the assumption $A$ is invertible.}
    A^{-1} (A X) &= A^{-1} B \\
    (A^{-1} A) X &= A^{-1} B \\
    I X &= A^{-1} B
  \end{align*}

  Therefore, we can solve the system using $A^{-1}$.
\end{blackbox}


%%% Local Variables:
%%% mode: latex
%%% TeX-master: "../../Math_333-MatrixAlg_ComplexVars-Reference_Sheet"
%%% End:


\subsection{Linear Independence}\label{subsec:Linear_Independence}
\begin{theorem}[Linearly Dependent]\label{thm:Linearly_Dependent}
  Let $S$ be a set of vectors of size $n$, $S = \lbrace v_{1}, v_{2}, \ldots, v_{n} \rbrace$.
  Each vector $v_{k}$ has components in $\RealNumbers^{m}$ ($m$ components from $\RealNumbers$).

  The set is \emph{linearly dependent} if there exists scalars $c_{1}, c_{2}, \ldots, c_{n}$, at least one of which is not equal to zero, such that the below equation holds true.
  \begin{equation}\label{eq:Linearly_Dependent}
    c_{1}v_{1} + c_{2}v_{2} + \cdots + c_{n}v_{n} = 0
  \end{equation}

  \begin{remark*}
    Note that the $0$ in \Cref{eq:Linearly_Dependent} implies the zero vector, not necessarily the scalar value $0$.
  \end{remark*}
\end{theorem}


%%% Local Variables:
%%% mode: latex
%%% TeX-master: "../../Math_333-MatrixAlg_ComplexVars-Reference_Sheet"
%%% End:


\subsection{Matrix Rank}\label{subsec:Matrix_Rank}
\begin{definition}[Rank]\label{def:Matrix_Rank}
  The \emph{rank} of a \nameref{def:Matrix} $A_{n \by m}$ is the largest number of \nameref{def:Linearly_Independent} ``chunks'' of a matrix.

\end{definition}

%%% Local Variables:
%%% mode: latex
%%% TeX-master: "../../Math_333-MatrixAlg_ComplexVars-Reference_Sheet"
%%% End:


\subsection{Determinants}\label{subsec:Determinants}
Before calculating the \nameref{def:Determinant}, we need to have some additional vocabulary wto work with.

\begin{definition}[Minor]\label{def:Minor}
  The \emph{minor} $\Minor{i}{j}$ is the \nameref{def:Determinant} of the sub\nameref{def:Matrix} obtained by deleting the $i$th row and $j$th column.
\end{definition}

\begin{blackbox}
  Given the matrix $
  \begin{pmatrix}
    1 & -1 & 3 \\
    2 & 5 & -1 \\
    3 & 0 & 5
  \end{pmatrix}$, the \nameref{def:Minor} at the index $3,2$ is found by deleting the \nth{3} row and \nth{2} column and taking the \nameref{def:Determinant} of the resulting submatrix.
  Thus,
  \begin{align*}
    \Minor{3}{2} &= \det
                   \begin{pmatrix}
                     1 & 3 \\
                     2 & -1
                   \end{pmatrix} \\
    \intertext{Use the definition of the \nameref{def:Determinant} for a $2 \by 2$ \nameref{def:Matrix}.}
                 &= 1(-1) - 3(2) \\
                 &= -1 - 6 \\
                 &= -7
  \end{align*}
\end{blackbox}

\begin{definition}[Cofactor]\label{def:Cofactor}
  The \emph{cofactor} $\Cofactor{i}{j}$ is related to the \nameref{def:Minor}, $\Minor{i}{j}$ by the equation below.
  \begin{equation}\label{eq:Cofactor}
    \Cofactor{i}{j} = {(-1)}^{i + j} \Minor{i}{j}
  \end{equation}
\end{definition}

\begin{blackbox}
  Using the same \nameref{def:Matrix} from earlier, the \nameref{def:Cofactor} of of the element in the \nth{3} row and \nth{2} column is:
  \begin{align*}
    \Cofactor{3}{2} &= {(-1)}^{3+2} \Minor{3}{2} \\
                    &= (-1) (-7) \\
                    &= 7
  \end{align*}
\end{blackbox}

Now with these terms defined, we can define the general algorithm for the \nameref{def:Determinant}.

\begin{definition}[Determinant]\label{def:Determinant}
  The \emph{determinant} of a \nameref{def:Matrix} is a scalar value that is computed out of a \textbf{square} matrix and encodes certain properties of the transformation specified by the matrix.

  There are 2 equations in use for the determinant, depending on the size of the matrix.
  For a $2 \by 2$ matrix,
  \begin{equation}\label{eq:Determinant_2x2}
    \det
    \begin{pmatrix}
      a & b \\
      c & d
    \end{pmatrix}
    = ad - bc
  \end{equation}

  For a matrix larger than $2 \by 2$, we have many possible ways of finding the determinant.
  This is explored further in \Cref{subsubsec:Expand_Determinant}, with their equations given in \Cref{eq:Determinant_Expand_ith_Row} and \Cref{eq:Determinant_Expand_jth_Column}.
\end{definition}


%%% Local Variables:
%%% mode: latex
%%% TeX-master: "../../Math_333-MatrixAlg_ComplexVars-Reference_Sheet"
%%% End:


\subsection{Eigenvectors}\label{subsec:Eigenvectors}

\subsection{Eigenvalues}\label{subsec:Eigenvalues}

%%% Local Variables:
%%% mode: latex
%%% TeX-master: "../../Math_333-MatrixAlg_ComplexVars-Reference_Sheet"
%%% End:


\subsection{Cayley-Hamilton Theorem}\label{subsec:Cayley-Hamilton_Theorem}
\begin{theorem}[Cayley-Hamilton Theorem]\label{thm:Cayley-Hamilton_Theorem}
  A square \nameref{def:Matrix} $A_{n \by n}$ satisfies its own \nameref{def:Characteristic_Polynomial}.
  \begin{remark*}
    ``Satisfies'' in this context means that the \nameref{def:Characteristic_Polynomial} will return $0$ when the variable $\lambda$ is replaced by a value.
  \end{remark*}
\end{theorem}


%%% Local Variables:
%%% mode: latex
%%% TeX-master: "../../Math_333-MatrixAlg_ComplexVars-Reference_Sheet"
%%% End:


\subsection{Diagonalization}\label{subsec:Diagonalization}
To start off with, we need some background knowledge on what we mean by \nameref{def:Diagonalization}.

\begin{definition}[Diagonal Matrix]\label{def:Diagonal_Matrix}
  A \emph{diagonal matrix} is a \nameref{def:Matrix} whose only non-zero elements are on the main diagonal of the matrix.

  In general, this is seen as:
  \begin{equation}\label{eq:Diagonal_Matrix}
    A =
    \begin{pmatrix}
      a_{1,1} & 0 & 0 & \cdots \\
      0 & a_{2,2} & 0 & \cdots \\
      \vdots & \ddots & \ddots & \vdots \\
      0 & \cdots & \cdots & a_{n, n}
    \end{pmatrix}
  \end{equation}
\end{definition}

\begin{definition}[Diagonalizable]\label{def:Diagonalizable}
  Let the \nameref{def:Matrix} $A_{n \by n}$.
  We say $A$ is \emph{diagonalizable} to a \nameref{def:Diagonal_Matrix} $D$ if there exists an invertible matrix $P$ such that \Cref{eq:Diagonalizable} holds true.

  \begin{equation}\label{eq:Diagonalizable}
    P^{-1} A P = D
  \end{equation}
\end{definition}

\begin{definition}[Diagonalization]\label{def:Diagonalization}
  Let the matrix $A_{n \by n}$ be \nameref{def:Diagonalizable} by an invertible \nameref{def:Matrix} $P$ to form a \nameref{def:Diagonal_Matrix} $D$.
  $P$ is called a matrix that implements the \emph{diagonalization} shown in \Cref{eq:Diagonalization}.

  \begin{equation}\label{eq:Diagonalization}
    AP = PD
  \end{equation}
\end{definition}

Now that we have the terms and definitions out of the way, we can see something interesting about \Cref{eq:Diagonalization}.
\begin{blackbox}
  Let $A_{2 \by 2}$, $P =
  \begin{pmatrix}
    p_{1} & p_{2}
  \end{pmatrix}
  $, and $D =
  \begin{pmatrix}
    d_{1} & 0 \\
    0 & d_{2}
  \end{pmatrix}$.

  This means that if we apply \Cref{eq:Diagonalization}:
  \begin{align*}
    AP &= PD \\
    A_{2 \by 2}
    \begin{pmatrix}
      p_{1} & p_{2}
    \end{pmatrix} &=
                    \begin{pmatrix}
                      p_{1} & p_{2}
                    \end{pmatrix}
                              \begin{pmatrix}
                                d_{1} & 0 \\
                                0 & d_{2} \\
                              \end{pmatrix} \\
    \begin{pmatrix}
      A p_{1} & A p_{2}
    \end{pmatrix} &=
                    \begin{pmatrix}
                      d_{1} p_{1} & d_{2} p_{2}
                    \end{pmatrix}
  \end{align*}

  This means that:
  \begin{align*}
    A p_{1} &= d_{1} p_{1} \\
    A p_{2} &= d_{2} p_{2}
  \end{align*}

  If we study this, we can see that $d_{1}$ and $d_{2}$ are \nameref{def:Eigenvalue}s of $A$ (Remember, $AX = \lambda X$).
  Therefore, $P$ is a \nameref{def:Matrix} that implements the \nameref{def:Diagonalization} using a matrix of corresponding \nameref{def:Eigenvector}s.
\end{blackbox}


%%% Local Variables:
%%% mode: latex
%%% TeX-master: "../../Math_333-MatrixAlg_ComplexVars-Reference_Sheet"
%%% End:


%%% Local Variables:
%%% mode: latex
%%% TeX-master: "../Math_333-MatrixAlg_ComplexVars-Reference_Sheet"
%%% End:


\section{Complex Matrices}\label{sec:Complex_Matrices}
\subsection{Inner Product}\label{subsec:Inner_Product}

%%% Local Variables:
%%% mode: latex
%%% TeX-master: "../../Math_333-MatrixAlg_ComplexVars-Reference_Sheet"
%%% End:


\subsection{Inverses}\label{subsec:Complex_Matrix_Inverses}
The \nameref{def:Inverse_Matrix} of a \nameref{def:Complex_Matrix} has the exact same definition as its real-valued counterpart.
Refer back to \Cref{subsec:Inverse_Matrices} for a detailed explanation.


%%% Local Variables:
%%% mode: latex
%%% TeX-master: "../../Math_333-MatrixAlg_ComplexVars-Reference_Sheet"
%%% End:


\subsection{Unitary Matrices}\label{subsec:Unitary_Matrices}
\begin{definition}[Unitary Matrix]\label{def:Unitary_Matrix}
  A \emph{unitary matrix} is the \nameref{def:Complex_Matrix} equivalent of an \nameref{def:Orthogonal_Matrix} real-valued \nameref{def:Matrix}.

  This means that 2 properties must be satisfied:
  \begin{propertylist}
  \item Each row/column must be unit length ($=1$).\label{prop:Unitary_Matrix-Unit_Length}
  \item Each row/column is mutually orthogonal.
    This means the \nameref{def:Inner_Product} of any two distinct rows/columns sum to zero.\label{prop:Unitary_Matrix-Mutually_Orthogonal}
  \end{propertylist}
\end{definition}


%%% Local Variables:
%%% mode: latex
%%% TeX-master: "../../Math_333-MatrixAlg_ComplexVars-Reference_Sheet"
%%% End:


\subsection{Diagonalization}\label{subsec:Complex_Matrix_Diagonalization}
\nameref{def:Diagonalization} of a \nameref{def:Complex_Matrix} behaves much the same way as the real-valued case.

For a \nameref{def:Complex_Matrix} $A$, a matrix that implements a \nameref{def:Diagonalization} is $P$, and this yields a completely real-valued matrix $D$.
This means:
\begin{equation*}
  P^{-1} A P = D
\end{equation*}

\begin{remark*}
  If $P$ is a \nameref{def:Hermitian_Matrix}, then $P^{-1} = \MatrixConjTrans{P}$.
\end{remark*}

%%% Local Variables:
%%% mode: latex
%%% TeX-master: "../../Math_333-MatrixAlg_ComplexVars-Reference_Sheet"
%%% End:



%%% Local Variables:
%%% mode: latex
%%% TeX-master: "../Math_333-MatrixAlg_ComplexVars-Reference_Sheet"
%%% End:


%====================================APPENDIX====================================
\appendix
\counterwithin{definition}{subsection}

\clearpage
\subsection{Trigonometry} \label{app:Trig}
	\subsubsection{Trigonometric Formulas} \label{subsubsec:Trig Formulas}
		\begin{equation} \label{eq:Sin plus Sin with diff Angles}
			\sin \left( \alpha \right) + \sin \left( \beta \right) = 2 \sin \left( \frac{\alpha + \beta}{2} \right) \cos\left( \frac{\alpha - \beta}{2} \right)  
		\end{equation}
		\begin{equation} \label{eq:Cosine-Sine Product}
			\cos \left( \theta \right) \sin \left( \theta \right) = \frac{1}{2} \sin \left( 2 \theta \right)
		\end{equation}

\clearpage
\subsection{Calculus} \label{app:Calculus}
	\subsubsection{Fundamental Theorems of Calculus} \label{subsubsec:Fundamental Theorem of Calculus}
		\begin{definition}[First Fundamental Theorem of Calculus] \label{def:1st Fundamental Theorem of Calculus}
			The \emph{first fundamental theorem of calculus} states that, if $f$ is continuous on the closed interval $\left[ a,b \right]$ and $F$ is the indefinite integral of $f$ on $\left[ a,b \right]$, then 
			\begin{equation} \label{eq:1st Fundamental Theorem of Calculus}
				\int_{a}^{b}f \left( x \right) dx = F \left( b \right) - F \left( a \right)
			\end{equation}
		\end{definition}
		\begin{definition}[Second Fundamental Theorem of Calculus] \label{def:2nd Fundamental Theorem of Calculus}
			The \emph{second fundamental theorem of calculus} holds for $f$ a continuous function on an open interval $I$ and $a$ any point in $I$, and states that if $F$ is defined by
			\begin{equation*}
				F \left( x \right) = \int_{a}^{x} f \left( t \right) dt,
			\end{equation*}
			then
			\begin{equation} \label{eq:2nd Fundamental Theorem of Calculus}
				\begin{aligned}
					\frac{d}{dx} \int_{a}^{x} f \left( t \right) dt &= f \left( x \right) \\
					F' \left( x \right) &= f \left( x \right) \\
				\end{aligned}
			\end{equation}
		\end{definition}

\clearpage
\section{Laplace Transform}\label{app:Laplace_Transform}
\subsection{Laplace Transform}\label{subsec:Laplace_Transform}
\begin{definition}[Laplace Transform]\label{def:Laplace_Transform}
  The \emph{Laplace transformation} operation is denoted as $\Lapl \lbrace x(t) \rbrace$ and is defined as
  \begin{equation}\label{eq:Laplace_Transform}
    X(s) = \int\limits_{-\infty}^{\infty} x(t) e^{-st} dt
  \end{equation}
\end{definition}

\subsection{Inverse Laplace Transform}\label{subsec:Inverse_Laplace_Transform}
\begin{definition}[Inverse Laplace Transform]\label{def:Inverse_Laplace_Transform}
  The \emph{inverse Laplace transformation} operation is denoted as $\Lapl^{-1} \lbrace X(s) \rbrace$ and is defined as
  \begin{equation}\label{eq:Inverse_Laplace_Transform}
    x(t) = \frac{1}{2j \pi} \int_{\sigma-\infty}^{\sigma+\infty} X(s) e^{st} \, ds
  \end{equation}
\end{definition}



% To make this print, you must include a citation somewhere in the document
\clearpage
\printbibliography{}
\end{document}

%%% Local Variables:
%%% mode: latex
%%% TeX-master: t
%%% End:
