\documentclass[10pt,letterpaper,final,twoside,notitlepage]{article}
\usepackage[margin=.5in]{geometry} % 1/2 inch margins on all pages
\usepackage[utf8]{inputenc} % Define the input encoding
\usepackage[USenglish]{babel} % Define language used
\usepackage{amsmath,amsfonts,amssymb}
\usepackage{amsthm} % Gives us plain, definition, and remark to use in \theoremstyle{style}
\usepackage{mathtools} % Allow for text and math in align* environment.
\usepackage{thmtools}
\usepackage{thm-restate}
\usepackage{graphicx}

\usepackage[
backend=biber,
style=alphabetic,
citestyle=authoryear]{biblatex} % Must include citation somewhere in document to print bibliography
\usepackage{hyperref} % Generate hyperlinks to referenced items
\usepackage[nottoc]{tocbibind} % Prints the Reference/Bibliography in TOC as well
\usepackage[noabbrev,nameinlink]{cleveref} % Fancy cross-references in the document everywhere
\usepackage{nameref} % Can make references by name to places
\usepackage{caption} % Allows for greater control over captions in figure, algorithm, table, etc. environments
\usepackage{subcaption} % Allows for multiple figures in one Figure environment
\usepackage[binary-units=true]{siunitx} % Gives us ways to typeset units for stuff
\usepackage{csquotes} % Context-sensitive quotation facilities
\usepackage{enumitem} % Provides [noitemsep, nolistsep] for more compact lists
\usepackage{chngcntr} % Allows us to tamper with the counter a little more
\usepackage{empheq} % Allow boxing of equations in special math environments
\usepackage[x11names]{xcolor} % Gives access to coloring text in environments or just text, MUST be before tikz
\usepackage{tcolorbox} % Allows us to create boxes of various types for examples
\usepackage{tikz} % Allows us to create TikZ and PGF Pictures
\usepackage{ctable} % Greater control over tables and how they look
\usepackage{diagbox} % Allow us to have shared diagonal cells in tables
\usepackage{multirow} % Allow us to have a single cell in a table span multiple rows
\usepackage{titling} % Put document information throughout the document programmatically
\usepackage[linesnumbered,ruled,vlined]{algorithm2e} % Allows us to write algorithms in a nice style.

\counterwithin{figure}{section}
\counterwithin{table}{section}
\counterwithin{equation}{section}
\counterwithin{algocf}{section}
\crefname{algocf}{algorithm}{algorithms}
\Crefname{algocf}{Algorithm}{Algorithms}
\setcounter{secnumdepth}{4}
\setcounter{tocdepth}{4} % Include \paragraph in toc
\crefname{paragraph}{paragraph}{paragraphs}
\Crefname{paragraph}{Paragraph}{Paragraphs}

% Create a theorem environment
\theoremstyle{plain}
\newtheorem{theorem}{Theorem}[section]
% Create a numbered theorem-like environment for lemmas
\newtheorem{lemma}{Lemma}[theorem]

% Create a definition environment
\theoremstyle{definition}
\newtheorem{definition}{Defn}
\newtheorem{corollary}{Corollary}[section]
% \begin{definition}[Term] \label{def:}
%   Make sure the term is emphasized with \emph{term}.
%   This ensures that if \emph is changed, it shows up everywhere
% \end{definition}

% Create a numbered remark environment numbered based on definition
% NOTE: This version of remark MUST go inside a definition environment
\theoremstyle{remark}
\newtheorem{remark}{Remark}[definition]
%\counterwithin{definition}{subsection} % Uncomment to have definitions use section.subsection numbering

% Create an unnumbered remark environment for general use
% NOTE: This version of remark has NO restrictions on placement
\newtheorem*{remark*}{Remark}

% Create a special list that handles properties. It can be broken and restarted
\newlist{propertylist}{enumerate}{1} % {Name}{Template}{Max Depth}
% [newlistname, LevelsToApplyTo]{formatting options}
\setlist[propertylist, 1]{label=\textbf{(\roman*)}, ref=\textbf{(\roman*)}, noitemsep, nolistsep}
\crefname{propertylisti}{property}{properties}
\Crefname{propertylisti}{Property}{Properties}

% Create a special list that handles enumerate starting with lower letters. Breakable/Restartable.
\newlist{boldalphlist}{enumerate}{1} % {Name}{Template}{Max Depth}
% [newlistname, LevelsToApplyTo]{formatting options}
\setlist[boldalphlist, 1]{label=\textbf{(\alph*)}, ref=\alph*, noitemsep, nolistsep} % Set options

\newlist{nocrefenumerate}{enumerate}{1} % {Name}{Template}{Max Depth}
% [newlistname, LevelsToApplyTo]{formatting options}
\setlist[nocrefenumerate, 1]{label=(\arabic*), ref=(\arabic*), noitemsep, nolistsep}

% Create a list that allows for deeper nesting of numbers. By default enumerate only allows depth=4.
\newlist{nestednums}{enumerate}{6}
% [newlistname, LevelsToApplyTo]{formatting options}
\setlist[nestednums]{noitemsep, label*=\arabic*.}

\tcbuselibrary{breakable} % Allow tcolorboxes to be broken across pages
% Create a tcolorbox for examples
% /begin{example}[extra name]{NAME}
% Create a tcolorbox for examples
% Argument #1 is optional, given by [], that is the textbook's problem number
% Argument #2 is mandatory, given by {}, that is the title for the example
% Avoid putting special characters, (), [], {}, ",", etc. in the title.
\newtcolorbox[auto counter,
number within=section,
number format=\arabic,
crefname={example}{examples}, % Define reference format for cref (No Capitals)
Crefname={Example}{Examples}, % Reference format for cleveref (With Capitals)
]{example}[2][]{ % The [2][] Means the first argument is optional
  width=\textwidth,
  title={Example \thetcbcounter: #2. #1}, % Parentheses and commas are not well supported
  fonttitle=\bfseries,
  label={ex:#2},
  nameref=#2,
  colbacktitle=white!100!black,
  coltitle=black!100!white,
  colback=white!100!black,
  upperbox=visible,
  lowerbox=visible,
  sharp corners=all,
  breakable
}

% Create a tcolorbox for general use
\newtcolorbox[% auto counter,
% number within=section,
% number format=\arabic,
% crefname={example}{examples}, % Define reference format for cref (No Capitals)
% Crefname={Example}{Examples}, % Reference format for cleveref (With Capitals)
]{blackbox}{
  width=\textwidth,
  % title={},
  fonttitle=\bfseries,
  % label={},
  % nameref=,
  colbacktitle=white!100!black,
  coltitle=black!100!white,
  colback=white!100!black,
  upperbox=visible,
  lowerbox=visible,
  sharp corners=all
}

% Redefine the 'end of proof' symbol to be a black square, not blank
\renewcommand{\qedsymbol}{$\blacksquare$} % Change proofs to have black square at end

% Common Mathematical Stuff
\newcommand{\Abs}[1]{\ensuremath{\lvert #1 \rvert}}
\newcommand{\DNE}{\ensuremath{\mathrm{DNE}}} % Used when limit of function Does Not Exist

% Complex Numbers functions
\renewcommand{\Re}{\operatorname{Re}} % Redefine to use the command, but not the fraktur version
\renewcommand{\Im}{\operatorname{Im}} % Redefine to use the command, but not the fraktur version
\newcommand{\Real}[1]{\ensuremath{\Re \lbrace #1 \rbrace}}
\newcommand{\Imag}[1]{\ensuremath{\Im \lbrace #1 \rbrace}}
\newcommand{\Conjugate}[1]{\ensuremath{\overline{#1}}}
\newcommand{\Modulus}[1]{\ensuremath{\lvert #1 \rvert}}
\DeclareMathOperator{\PrincipalArg}{\ensuremath{Arg}}

% Math Operators that are useful to abstract the written math away to one spot
% Number Sets
\DeclareMathOperator{\RealNumbers}{\ensuremath{\mathbb{R}}}
\DeclareMathOperator{\AllIntegers}{\ensuremath{\mathbb{Z}}}
\DeclareMathOperator{\PositiveInts}{\ensuremath{\mathbb{Z}^{+}}}
\DeclareMathOperator{\NegativeInts}{\ensuremath{\mathbb{Z}^{-}}}
\DeclareMathOperator{\NaturalNumbers}{\ensuremath{\mathbb{N}}}
\DeclareMathOperator{\ComplexNumbers}{\ensuremath{\mathbb{C}}}
\DeclareMathOperator{\RationalNumbers}{\ensuremath{\mathbb{Q}}}

% Calculus operators
\DeclareMathOperator*{\argmax}{argmax} % Thin Space and subscripts are UNDER in display

% Signal and System Functions
\DeclareMathOperator{\UnitStep}{\mathcal{U}}
\DeclareMathOperator{\sinc}{sinc} % sinc(x) = (sin(pi x)/(pi x))

% Transformations
\DeclareMathOperator{\Lapl}{\mathcal{L}} % Declare a Laplace symbol to be used

% Logical Operators
\DeclareMathOperator{\XOR}{\oplus}

% x86 CPU Registers
\newcommand{\rbpRegister}{\texttt{\%rbp}}
\newcommand{\rspRegister}{\texttt{\%rsp}}
\newcommand{\ripRegister}{\texttt{\%rip}}
\newcommand{\raxRegister}{\texttt{\%rax}}
\newcommand{\rbxRegister}{\texttt{\%rbx}}

%%% Local Variables:
%%% mode: latex
%%% TeX-master: shared
%%% End:


% These packages are more specific to certain documents, but will be availabe in the template
% \usepackage{esint} % Provides us with more types of integral symbols to use
% \usepackage[outputdir=./TeX_Output]{minted} % Allow us to nicely typeset 300+ programming languages
% \crefname{lstlisting}{listing}{listings}
% \Crefname{lstlisting}{Listing}{Listings}
% This document must be compiled with the -shell-escape flag if the packages above are uncommented

\graphicspath{{./Drawings/Math_333-MatrixAlg_ComplexVars/}} % Uncomment this to use pictures in this document
% \addbibresource{./Bibliographies/CourseNum-Name.bib}

\usepackage{nth}
% Math Operators that are useful to abstract the written math away to one spot
% These are supposed to be document-specific mathematical operators that will make life easier
% Many fundamental operators are defined in Reference_Sheet_Preamble.tex

\newcommand{\Path}{\ensuremath{\gamma}} % Path in complex plane to approach a point a.

% Taken from https://gitlab.com/jim.hefferon/linear-algebra/-/blob/master/src/sty/linalgjh.sty
%--------grstep
% For denoting a Gauss' reduction step.
% Use as: \grstep{\rho_1+\rho_3} % Step above arrow
% \grstep[2\rho_5 \\ 3\rho_6]{\rho_1+\rho_3} % Step above arrow, additional ones below
% \newcommand{\grstep}[2][\relax]{%
%    \ensuremath{\mathrel{
%        \mathop{\longrightarrow}\limits^{#2\mathstrut}_{
%                                    \begin{subarray}{l} #1 \end{subarray}}}}}

% Advantage of length formulation is that between adjacent
% \grstep's you can add \hspace{-\grsteplength} to make it look not too wide
\newlength{\grsteplength}
\setlength{\grsteplength}{1.5ex plus .1ex minus .1ex}

\newcommand{\grstep}[2][\relax]{%
   \ensuremath{\mathrel{
       \hspace{\grsteplength}\mathop{\longrightarrow}\limits^{#2\mathstrut}_{
                                     \begin{subarray}{l} #1 \end{subarray}}\hspace{\grsteplength}}}}
%-------------amatrix
% Augmented matrix.  Usage (note the argument does not count the aug col):
% \begin{amatrix}{2}
%   1  2  3 \\  4  5  6
% \end{amatrix}
\newenvironment{amatrix}[1]{%
  \left(\begin{array}{@{}*{#1}{c}|c@{}}
}{%
  \end{array}\right)
}
\newenvironment{amat}[2][c]{%
  % disable optional arg \left(\begin{array}{@{}*{#2}{#1}|#1@{}}
  \left(\begin{array}{@{}*{#2}{c}|#1@{}}
}{%
  \end{array}\right)
}

\renewcommand*{\arraystretch}{1.2}
\newcommand{\by}{\ensuremath{\times}}

\DeclareMathOperator{\Span}{span}

\begin{titlepage}
  \title{Math 333: Matrix Algebra and Complex Variables --- Reference Material \\ Illinois Institute of Technology}
  \author{Karl Hallsby}
  \date{Last Edited: \today} % We want to inform people when this document was last edited
\end{titlepage}

\begin{document}
\pagenumbering{gobble}
\maketitle
\pagenumbering{roman} % i, ii, iii on beginning pages, that don't have content
\tableofcontents
\clearpage
\listoftheorems[ignoreall, show={definition, Definition}]
\clearpage
\pagenumbering{arabic} % 1,2,3 on content pages

\section{Complex Numbers}\label{sec:Complex_Numbers}
\begin{definition}[Complex Number]\label{def:Complex_Number}
  A \emph{complex number} is a hyper real number system.
  This means that two real numbers, $a, b \in \RealNumbers$, are used to construct the set of complex numbers, denoted $\ComplexNumbers$.

  A complex number is written, in Cartesian form, as shown in \Cref{eq:Complex_Number} below.
  \begin{equation}\label{eq:Complex_Number}
    z = a \pm ib
  \end{equation}
  where
  \begin{equation}\label{eq:Imaginary_Value}
    i = \sqrt{-1}
  \end{equation}

  \begin{remark*}[$i$ vs. $j$ for Imaginary Numbers]
    Complex numbers are generally denoted with either $i$ or $j$.
    Electrical engineering regularly makes use of $j$ as the imaginary value.
    This is because alternating current $i$ is already taken, so $j$ is used as the imaginary value instad.
  \end{remark*}
\end{definition}

\subsection{Parts of a Complex Number}\label{subsec:Complex_Number_Parts}
A \nameref{def:Complex_Number} is made of up 2 parts:
\begin{enumerate}[noitemsep]
\item \nameref{def:Real_Part}
\item \nameref{def:Imaginary_Part}
\end{enumerate}

\begin{definition}[Real Part]\label{def:Real_Part}
  The \emph{real part} of an imaginary number, denoted with the $\Re$ operator, is the portion of the \nameref{def:Complex_Number} with no part of the imaginary value $i$ present.

  If $z = x + iy$, then
  \begin{equation}\label{eq:Real_Part}
    \Real{z} = x
  \end{equation}

  \begin{remark}[Alternative Notation]\label{rmk:Real_Part_Alternative_Notation}
    The \nameref{def:Real_Part} of a number sometimes uses a slightly different symbol for denoting the operation.
    It is:
    \begin{equation*}
      \mathfrak{Re}
    \end{equation*}
  \end{remark}
\end{definition}

\begin{definition}[Imaginary Part]\label{def:Imaginary_Part}
  The \emph{imaginary part} of an imaginary number, denoted with the $\Im$ operator, is the portion of the \nameref{def:Complex_Number} where the imaginary value $i$ is present.

  If $z = x + iy$, then
  \begin{equation}\label{eq:Imaginary_Part}
    \Imag{z} = y
  \end{equation}

  \begin{remark}[Alternative Notation]\label{rmk:Imaginary_Part_Alternative_Notation}
    The \nameref{def:Imaginary_Part} of a number sometimes uses a slightly different symbol for denoting the operation.
    It is:
    \begin{equation*}
      \mathfrak{Im}
    \end{equation*}
  \end{remark}
\end{definition}

\subsection{Binary Operations}\label{subsec:Binary_Operations}

%%% Local Variables:
%%% mode: latex
%%% TeX-master: shared
%%% End:


\subsection{Complex Conjugates}\label{app:Complex_Conjugates}
\begin{definition}[Complex Conjugate]\label{def:Complex_Conjugate}
  The conjugate of a complex number is called its \emph{complex conjugate}.
  The complex conjugate of a complex number is the number with an equal real part and an imaginary part equal in magnitude but opposite in sign.
  If we have a complex number as shown below,
  \begin{equation*}
    z = a \pm bi
  \end{equation*}

  then, the conjugate is denoted and calculated as shown below.
  \begin{equation}\label{eq:Complex_Conjugates}
    \Conjugate{z} = a \mp bi
  \end{equation}
\end{definition}

The \nameref{def:Complex_Conjugate} can also be denoted with an asterisk ($*$).
This is generally done for complex functions, rather than single variables.
\begin{equation}\label{eq:Complex_Conjugates_Asterisk}
  z^{*} = \Conjugate{z}
\end{equation}

%%% Local Variables:
%%% mode: latex
%%% TeX-master: shared
%%% End:


\subsection{Geometry of Complex Numbers}\label{subsec:Geometry_Complex_Numbers}
So far, we have viewed \nameref{def:Complex_Number}s only algebraically.
However, we can also view them geometrically as points on a 2 dimensional \nameref{def:Argand_Plane}.

\begin{definition}[Argand Plane]\label{def:Argand_Plane}
  An \emph{Argane Plane} is a standard two dimensional plane whose points are all elements of the complex numbers, $z \in \ComplexNumbers$.
  This is taken from Descarte's definition of a completely real plane.

  The Argand plane contains 2 lines that form the axes, that indicate the real component and the imaginary component of the complex number specified.
\end{definition}

A \nameref{def:Complex_Number} can be viewed as a point in the \nameref{def:Argand_Plane}, where the \nameref{def:Real_Part} is the ``$x$''-component and the \nameref{def:Imaginary_Part} is the ``$y$''-component.

By plotting this, you see that we form a right triangle, so we can find the hypotenuse of that triangle.
This hypotenuse is the distance the point $p$ is from the origin, refered to as the \nameref{def:Complex_Number_Modulus}.
\begin{remark*}
  When working with \nameref{def:Complex_Number}s geometrically, we refer to the points, where they are defined like so:
  \begin{equation*}
    z = x + iy = p(x, y)
  \end{equation*}

  Note that $p$ is \textbf{not} a function of $x$ and $y$.
  Those are the values that inform us \textbf{where} $p$ is located on the \nameref{def:Argand_Plane}.
\end{remark*}

\subsubsection{Modulus of a Complex Number}\label{subsubsec:Complex_Number_Modulus}
\begin{definition}[Modulus]\label{def:Complex_Number_Modulus}
  The \emph{modulus} of a \nameref{def:Complex_Number} is the distance from the origin to the complex point $p$.
  This is based off the Pythagorean Theorem.
  \begin{equation}\label{eq:Complex_Number_Modulus}
    \begin{aligned}
      {\lvert z \rvert}^{2} = x^{2} + y^{2} &= z \Conjugate{z} \\
      \lvert z \rvert &= \sqrt{x^{2} + y^{2}}
    \end{aligned}
  \end{equation}
\end{definition}

\begin{propertylist}
\item The \emph{Law of Moduli} states that $\lvert z w \rvert = \lvert z \rvert \lvert w \rvert$.\label{prop:Law_of_Moduli}.
\end{propertylist}

We can prove \Cref{prop:Law_of_Moduli} using an algebraic identity.
\begin{proof}[Prove \Cref*{prop:Law_of_Moduli}]
  Let $z$ and $w$ be complex numbers ($z, w \in \ComplexNumbers$).
  We are asked to prove
  \begin{equation*}
    \lvert z w \rvert = \lvert z \rvert \lvert w \rvert
  \end{equation*}

  But, it is actually easier to prove
  \begin{equation*}
    {\lvert z w \rvert}^{2} = {\lvert z \rvert}^{2} {\lvert w \rvert}^{2}
  \end{equation*}

  We start by simplifying the ${\lvert z w \rvert}^{2}$ equation above.
  \begin{align*}
    {\lvert z w \rvert}^{2} &= {\lvert z \rvert}^{2} {\lvert w \rvert}^{2} \\
    \intertext{Using the definition of the \nameref{def:Complex_Number_Modulus} of a \nameref{def:Complex_Number} in \Cref{eq:Complex_Number_Modulus}, we can expand the modulus.}
                            &= (z w) (\Conjugate{z w}) \\
    \intertext{Using \Cref{prop:Complex_Conjugate_Split} for multiplication allows us to do the next step.}
                            &= (z w) (\Conjugate{z} \Conjugate{w}) \\
    \intertext{Using Multiplicative Associativity and Multiplicative Commutativity, we can simplify this further.}
                            &= (z \Conjugate{z}) (w \Conjugate{w}) \\
                            &= {\lvert z \rvert}^{2} {\lvert w \rvert}^{2}
  \end{align*}

  Note how we never needed to define $z$ or $w$, so this is as general a result as possible.
\end{proof}

\paragraph{Algebraic Effects of the Modulus' \Cref*{prop:Law_of_Moduli}}\label{par:Law_of_Moduli-Algebraic_Effects}
For this section, let $z = x_{1} + iy_{1}$ and $w = x_{2} + iy_{2}$.
Now,
\begin{align*}
  z w &= (x_{1}x_{2} - y_{1}y_{2}) + i(x_{1}y_{2} + x_{2}y_{1}) \\
  {\lvert z w \rvert}^{2} &= {(x_{1}x_{2} - y_{1}y_{2})}^{2} + {(x_{1}y_{2} + x_{2}y_{1})}^{2} \\
      &= \left( x_{1}^{2} + x_{2}^{2} \right) \left( x_{2}^{2} + y_{2}^{2} \right) \\
      &= {\lvert z \rvert}^{2} {\lvert w \rvert}^{2}
\end{align*}

However, the Law of Moduli (\Cref{prop:Law_of_Moduli}) does \textbf{not} hold for a hyper complex number system one that uses 2 or more imaginaries, i.e.\ $z = a + iy + jz$.
But, the Law of Moduli (\Cref{prop:Law_of_Moduli}) \textbf{does} hold for hyper complex number system that uses 3 imaginaries, $a = z + iy + jz + k \ell$.

\paragraph{Conceptual Effects of the Modulus' \Cref*{prop:Law_of_Moduli}}\label{par:Law_of_Moduli-Conceptual_Effects}
We are interested in seeing if $\lvert z w \rvert = (x_{1}^{2} + y_{1}^{2})(x_{2}^{2}+y_{2}^{2})$ can be extended to more complex terms (3 terms in the complex number).

However, Langrange proved that the equation below \textbf{always} holds.
Note that the $z$ below has no relation to the $z$ above.
\begin{equation*}
  (x_{1} + y_{1} + z_{1}) \neq X^{2} + Y^{2} + Z^{2}
\end{equation*}

%%% Local Variables:
%%% mode: latex
%%% TeX-master: shared
%%% End:


\subsection{Circles and Complex Numbers}\label{subsec:Circles_Complex_Numbers}
We need to define both a center and a radius, just like with regular purely real values.
\Cref{eq:Circles_Complex_Numbers} defines the relation required for a circle using \nameref{def:Complex_Number}s.
\begin{equation}\label{eq:Circles_Complex_Numbers}
  \lvert z - a \rvert = r
\end{equation}

\begin{example}[Lecture 2, Example 1]{Convert to Circle}
  Given the expression below, find the location of the center of the circle and the radius of the circle?
  \begin{equation*}
    \lvert 5 iz + 10 \rvert = 7
  \end{equation*}
  \tcblower{}
  This is just a matter of simplification and moving terms around.
  \begin{align*}
    \lvert 5 iz + 10 \rvert &= 7 \\
    \lvert 5i (z + \frac{10}{5i}) \rvert &= 7 \\
    \lvert 5i (z + \frac{2}{i}) \rvert &= 7 \\
    \lvert 5i (z + \frac{2}{i} \frac{-i}{-i}) \rvert &= 7 \\
    \lvert 5i (z - 2i) \rvert &= 7 \\
    \intertext{Now using the Law of Moduli (\Cref{prop:Law_of_Moduli}) $\lvert a b \rvert = \lvert a \rvert \lvert b \rvert$, we can simplify out the extra imaginary term.}
    \lvert 5i \rvert \lvert z-2i \rvert &= 7 \\
    5 \lvert z - 2i \rvert &= 7 \\
    \lvert z - 2i \rvert = \frac{7}{5}
  \end{align*}

  Thus, the circle formed by the equation $\lvert 5 iz + 10 \rvert = 7$ is actually $\lvert z - 2i \rvert = \frac{7}{5}$, with a center at $a = 2i$ and a radius of $\frac{7}{5}$.
\end{example}

\subsubsection{Annulus}\label{subsubsec:Annulus}
\begin{definition}[Annulus]\label{def:Annulus}
  An \emph{annulus} is a region that is bounded by 2 concentric circles.
  This takes the form of \Cref{eq:Annulus}.
  \begin{equation}\label{eq:Annulus}
    r_{1} \leq \lvert z - a \rvert \leq r_{2}
  \end{equation}

  In \Cref{eq:Annulus}, each of the $\leq$ symbols could also be replaced with $<$.
  This leads to 3 different possibilities for the annulus:
  \begin{enumerate}[noitemsep]
  \item If both inequality symbols are $\leq$, then it is a \textbf{Closed Annulus}.
  \item If both inequality symbols are $<$, then it is an \textbf{Open Annulus}.
  \item If \textbf{only one} inequality symbol $<$ and the other $\leq$, then it is not an \textbf{Open Annulus}.
  \end{enumerate}
\end{definition}


%%% Local Variables:
%%% mode: latex
%%% TeX-master: shared
%%% End:



%%% Local Variables:
%%% mode: latex
%%% TeX-master: shared
%%% End:

\section{Complex Functions}\label{sec:Complex_Functions}
Complex functions, like their real-valued counterparts behave in much the same way.
\begin{equation}\label{eq:Complex_Function}
  f(z) = w
\end{equation}
\begin{description}[noitemsep]
\item $f$: The function or \nameref{def:Mapping} that corresponds the input to the output.
\item $z$: The input to the complex function/\nameref{def:Mapping}.
\item $w$: The output of the complex function/\nameref{def:Mapping}.
\end{description}

\begin{definition}[Mapping]\label{def:Mapping}
  A \emph{mapping} is synonym for a function in mathematics.
  The term comes from set theory, where the input set is mapped to an output set by some operations.
  The conventional way to denote a mapping is with the $\mapsto$ symbol.

  An example of a mapping is shown in \Cref{eq:Mapping}
  \begin{equation}\label{eq:Mapping}
    z \mapsto z^{2}
  \end{equation}
\end{definition}

A complex function can only accept and will only return values in \textbf{Cartesian} or \textbf{polar} form.
Because the output of a complex function is also a complex value, \Cref{eq:Output_Value_Function} makes sense.
\begin{equation}\label{eq:Output_Value_Function}
  f(z) = U(x, y) + iV(x, y)
\end{equation}

$U(x, y)$ and $V(x, y)$ can be as general as we want in $x$ and $y$.
This means both could be constants, both could be polynomials, one could be trascendental, and anything in between.

The functions $U(x, y)$ and $V(x, y)$ are functions that yield real-values $u, v$.
This means that $u, v$ can also be graphed on an \nameref{def:Argand_Plane}.
By our definition of $U(x, y)$ and $V(x, y)$, $U(x, y), V(x, y)$ are parametric functions.

\begin{example}[Lecture 4]{Find Output Functions}
  Given the \nameref{def:Mapping} $z \mapsto z^{2}$, where $z = x + iy$, find the output functions for each term $U(x, y)$ and $V(x, y)$?
  \tcblower{}
  I will choose to represent the mapping $z \mapsto z^{2}$ with the complex function $f(z) = z^{2}$.
  \begin{align*}
    z &\mapsto z^{2} \\
    f(z) &= z^{2} \\
    \shortintertext{Apply the definition of $z$.}
      &= {(x + iy)}^{2} \\
      &= x^{2} + 2xyi + i^{2}y^{2} \\
      &= x^{2} + 2xyi + (-1)y^{2} \\
      &= \left( x^{2} - y^{2} \right) + 2xyi \\
  \end{align*}

  By out definition of $U(x, y)$ and $V(x, y)$ in \Cref{eq:Output_Value_Function}, we can finish solving this.
  \begin{align*}
    f(z) &= U(x, y) + iV(x, y) \\
    f(z) &= \left( x^{2} - y^{2} \right) + 2xyi \\
    U(x, y) &= x^{2} - y^{2} \\
    V(x, y) &= 2xy
  \end{align*}

  Thus, our output functions are $U(x, y) = x^{2} - y^{2}$ and $V(x, y) = 2xy$.
\end{example}

\subsubsection{Graphing Complex Functions}\label{subsubsec:Graphing_Complex_Functions}
If we take a closer look at the complex function $f(z)$, we notice something that makes handling complex function difficult.
$f(z)$ is really a function of $x$ and $y$, because $z$ depends on those 2 real-valued parameters.
Thus, all our inputs lie on a 2-dimensional plane.

Now if we look at the output $w$, we also notice it is complex-valued, meaning it also depends on some $u$ and $v$, which are equal to the value of their functions $U(x, y)$ and $V(x, y)$.
This means that the output of the function $f(z)$ \textbf{also} lies on a 2-dimensional plane.
Meaning, the function is 4-dimensional.
The intersection of 2 planes in our 3-dimensional world never yields a point in the hyperplane, and thus, we cannot graph it.

Instead, we choose to graph the inputs and outputs separately, effectively showing the mapping and the way the \nameref{def:Pre_Image} is transformed into the \nameref{def:Image} instead.

\begin{definition}[Pre-Image]\label{def:Pre_Image}
  The \emph{pre-image} consists of all points on the input plane.
  In this case, the input plane is the $z$-plane, being constructed out of the orthogonal intersection of the $\Re$ and $\Im$ axes.
\end{definition}


%%% Local Variables:
%%% mode: latex
%%% TeX-master: "../Math_333-MatrixAlg_ComplexVars-Reference_Sheet"
%%% End:


\section{Complex Series}\label{sec:Complex_Series}
We start our discussion of complex series by talking about \nameref{def:Complex_Power_Series}.

\begin{definition}[Power Series]\label{def:Complex_Power_Series}
  A \emph{power series} is an infinite summation of terms that form infinitely long polynomials.
  These are defined around a center $a$, $a \in \ComplexNumbers$.
  \begin{equation}\label{eq:Complex_Power_Series}
    \sum_{n=0}^{\infty} a_{n} {(z-a)}^{n} = a_{0} + a_{1} (z-a) + a_{2} {(z-a)}^{2} + \cdots
  \end{equation}
\end{definition}

Because our \nameref{def:Complex_Power_Series} are infinite, we now have to ask about its convergence.
There is a theorem for this, \Cref{thm:Region_of_Convergence}, called the \nameref{thm:Region_of_Convergence}.

\begin{theorem}[Region of Convergence]\label{thm:Region_of_Convergence}
  There exists a number $R$, $0 \leq R \leq \infty$, such that $\forall z, \Modulus{z-a} < R$ the series is convergent, and $\forall z, \Modulus{z-a} > R$ is divergent.
  $forall z, \Modulus{z-a} = R$ does not have guaranteed divergence or convergence.
  This value $R$ is called the \emph{Region of Convergence}, or the \emph{RoC}.
\end{theorem}

\begin{example}[Lecture 8, Example 3]{Finding the Region of Convergence}
  Compute the radius of convergence of the power series shown below.
  \begin{equation*}
    \sum_{n=0}^{\infty} \frac{3^{2n+5}}{2n+3} {(z-i)}^{7n+2}
  \end{equation*}
  \tcblower{}
  We can solve this by using d'Alembert's Ratio Test.
  This test states that for an infinite series to be convergent, $\lim_{n \to \infty} \Modulus{\frac{a_{n+1}}{a_{n}}} = C$, where $C$ is some constant value.

  Applying d'Alembert's Ratio test, we have
  \begin{align*}
    \lim_{n \to \infty} \Modulus{\frac{a_{n+1}}{a_{n}}} &= \lim_{n \to \infty} \Modulus{ \frac{\frac{3^{2(n+1) + 5}}{2(n+1) + 3} {(z-i)}^{7(n+1)+2}}{\frac{3^{2n+5}}{2n+3} {(z-i)}^{7n+2}}} \\
                                                        &= \lim_{n \to \infty} \Modulus{\frac{\frac{3^{2n+7}}{2n+5} {(z-i)}^{7n+9}}{\frac{3^{2n+5}}{2n+3} {(z-i)}^{7n+2}}} \\
    \shortintertext{We can cancel the values with exponents raised to the $n$.}
                                                        &= \lim_{n \to \infty} \frac{2n + 3}{2n + 5} 3^{2} \Modulus{{(z-i)}^{7}} \\
    \shortintertext{Evaluating the limit, we see that the fraction with $n$ will become $\frac{1}{1}$.}
                                                        &= 3^{2} \Modulus{{(z-i)}^{7}} \\
    \shortintertext{Using the Law of Moduli, we can simplify this.}
                                                        &= 3^{2} {\Modulus{z-i}}^{7}
  \end{align*}

  Now, we have to use \Cref{thm:Region_of_Convergence}.
  \begin{align*}
    \forall z, 3^{2} {\Modulus{z-i}}^{7} &< 1 \text{, Series is convergent} \\
    \forall z, 3^{2} {\Modulus{z-i}}^{7} &> 1 \text{, Series is divergent} \\
  \end{align*}

  Now, we move terms and exponents to the other side, leaving just $\Modulus{z-i}$ on the left.
  \begin{align*}
    \forall z, \Modulus{z-i} &< \frac{1}{3^{\frac{2}{7}}} \text{, Series is convergent} \\
    \forall z, \Modulus{z-i} &> \frac{1}{3^{\frac{2}{7}}} \text{, Series is divergent} \\
  \end{align*}
\end{example}

\begin{theorem}\label{thm:Power_Series-Analytic_Function}
  If the \nameref{def:Complex_Power_Series} $\sum_{n=0}^{\infty} a_{n} {(z-a)}^{n}$ has a \nameref{thm:Region_of_Convergence} $RoC > 0$, we can define $f(z)$ to be the summation of all the terms in the series, $f(z) = \sum_{n = 0}^{\infty} a_{n} {(z-a)}^{n}$, $\Modulus{z-a} < R$.

  Then, $f$ is \nameref{def:Analytic} in $\Modulus{z-a} < R$ and $f'(z)$ is term-by-term differentiable.
  \begin{align*}
    f(z) &= a_{0} + a_{1} (z-a) + a_{2} {(z-a)}^{2} + a_{3} {(z-a)}^{3} \cdots \\
    f'(z) &= 0 + a_{1} + 2 a_{2} (z-a) + 3 a_{3} {(z-a)}^{2} + \cdots \\
         &= \sum_{n=1}^{\infty} n a_{n} {(z-a)}^{n-1}, \, \forall z, \, \Modulus{z-a} < R
  \end{align*}
\end{theorem}

Likewise, the converse of \Cref{thm:Power_Series-Analytic_Function} holds true, seen in \Cref{thm:Analytic_Function-Power_Series}

\begin{theorem}\label{thm:Analytic_Function-Power_Series}
  Let $f$ be an \nameref{def:Analytic} \nameref{def:Complex_Function} on the disk $\Modulus{z-a} < R$.

  Then there exists a \nameref{def:Complex_Power_Series} with the same center $a$, such that
  \begin{equation*}
    f(z) = \sum_{n = 0}^{\infty} a_{n} {(z-a)}^{n}
  \end{equation*}
\end{theorem}

%%% Local Variables:
%%% mode: latex
%%% TeX-master: "../Math_333-MatrixAlg_ComplexVars-Reference_Sheet"
%%% End:


% At this point, we have finished all the complex analysis we will be doing in
% this course, so now we are moving onto the linear/matrix algebra. A brand new
% page will be a nice thing to start with.
\clearpage

\section{Matrix Algebra}\label{sec:Matrix_Algebra}
\begin{definition}[Matrix]\label{def:Matrix}
  A \emph{matrix} is an array of numbers subject to special addition and multiplication rules.
  They are represented with capital letters, $A$, $B$, etc.
  However, other texts may bold those capital letters to, $\mathbf{A}$, $\mathbf{B}$, etc.
  When written, they can be written in one of two ways, but they both mean the same thing.
  \begin{equation}\label{eq:Matrix}
    \begin{aligned}
      &\begin{pmatrix}
        a_{11} & a_{12} \\
        a_{21} & a_{22} \\
      \end{pmatrix} \\
      &\begin{bmatrix}
        a_{11} & a_{12} \\
        a_{21} & a_{22} \\
      \end{bmatrix}
    \end{aligned}
  \end{equation}
\end{definition}

Matrices are very useful, as learning the intricacies of them allows for quick solving of larger problems.

\subsection{Properties of Matrices}\label{subsec:Properties_Matrices}
\subsubsection{Properties of Matrix Addition}\label{subsubsec:Properties_Matrix_Addition}
\begin{propertylist}
\item \nameref{def:Matrix} Addition is commutative.\label{prop:Matrix_Add_Commutative}
  \begin{equation}\label{eq:Matrix_Add_Commutative}
    A+B = B+A
  \end{equation}

\item \nameref{def:Matrix} Addition is associative.\label{prop:Matrix_Add_Associative}
  \begin{equation}\label{eq:Matrix_Add_Associative}
    A+(B+C) = (A+B) + C
  \end{equation}

\item For any \nameref{def:Matrix} $A$, there is a unique matrix, $I$ or $O$, such that the equation below holds true.\label{prop:Matrix_Additive_Identity}
  \begin{equation}\label{eq:Matrix_Additive_Identity}
    A+O = A
  \end{equation}

\item For each \nameref{def:Matrix} $A$, there is a unique matrix $-A$ such that the equation below holds true.\label{prop:Matrix_Additive_Inverse}
  \begin{equation}\label{eq:Matrix_Additive_Inverse}
    A+(-A) = O
  \end{equation}

\item $A+B$ is a \nameref{def:Matrix} of the same dimensions as $A$ and $B$.\label{prop:Matrix_Addition_Closure}
\end{propertylist}

\subsubsection{Properties of Matrix Multiplication}\label{subsubsec:Properties_Matrix_Multiplication}
Matrix multiplication is defined as follows:
\begin{equation}\label{eq:Matrix_Multiplication}
  \begin{bmatrix}
    a_{11} & a_{12} & a_{13} & \cdots & a_{1n} \\
    a_{21} & a_{22} & \ddots & \ddots & a_{2n} \\
    \vdots & \ddots & \ddots & \ddots & a_{mn} \\
    a_{n\,1} & a_{n\,2} & \cdots & \cdots & a_{nn}
  \end{bmatrix}
\end{equation}

\begin{propertylist}
\item In general, \nameref{def:Matrix} multiplication is \textbf{NOT} commutative.\label{prop:Matrix_Not_Commutative}
  \begin{equation}\label{eq:Matrix_Mult_Not_Commutative}
    AB \neq BA
  \end{equation}

\item \nameref{def:Matrix} multiplication is associative.\label{prop:Matrix_Associativity}
  \begin{equation}\label{eq:Matrix_Mult_Associativity}
    (AB) C = A (BC)
  \end{equation}
\item \nameref{def:Matrix} multiplication is distributive.\label{prop:Matrix_Distributivity}
  \begin{equation}\label{eq:Matrix_Mult_Distributivity}
    \begin{aligned}
      A (B+C) &= AB + AC \\
      (B+C) A &= BA + CA
    \end{aligned}
  \end{equation}

\item There exists a \nameref{def:Matrix} $I$ that forms the \nameref{def:Multiplicative_Identity_Matrix} that satisfies the following rule:\label{prop:Mult_Identity_Matrix}
  \begin{equation}\label{eq:Mult_Identity_Matrix}
    \begin{aligned}
      AI &= A \\
      IA &= A
    \end{aligned}
  \end{equation}
  \begin{remark*}
    This is one of the two general cases where \nameref{def:Matrix} multiplication \textbf{IS} commutative.
  \end{remark*}

\item There exists a \nameref{def:Matrix} $O$ that forms the \nameref{def:Zero_Matrix}, satisfying the following rule:\label{prop:Mult_Zero_Matrix}
  \begin{equation}\label{eq:Mult_Zero_Matrix}
    \begin{aligned}
      AO &= O \\
      OA &= O
    \end{aligned}
  \end{equation}
  \begin{remark*}
    This is one of the two general cases where \nameref{def:Matrix} multiplication \textbf{IS} commutative.
  \end{remark*}

\item When performing multiplication, the \textbf{dimension property} states that the product of an $m \by n$ and an $n \by k$ matrix is an $m \by k$ \nameref{def:Matrix}.\label{prop:Mult_Dimension_Property}
\end{propertylist}

\begin{definition}[Identity Matrix]\label{def:Multiplicative_Identity_Matrix}
  The \emph{identity matrix}, $I$ is one that satisfies \Cref{prop:Mult_Identity_Matrix}.
  These follow the forms of the ones shown below.
  \begin{equation}\label{eq:Multiplicative_Identity_Matrix}
    \begin{aligned}
      I &=
      \begin{bmatrix}
        1 & 0 \\
        0 & 1 \\
      \end{bmatrix} \\
      &=
      \begin{bmatrix}
        1 & 0 & 0 \\
        0 & 1 & 0 \\
        0 & 0 & 1 \\
      \end{bmatrix}
    \end{aligned}
  \end{equation}
\end{definition}

\begin{definition}[Zero Matrix]\label{def:Zero_Matrix}
  The \emph{zero matrix}, $O$, is one that satisfies \Cref{prop:Mult_Zero_Matrix}.
  For \nameref{def:Matrix} multiplication, it is a matrix of all $0$.
  \begin{equation}
    \label{eq:3}
    \begin{aligned}
      O &=
      \begin{bmatrix}
        0 & 0 \\
        0 & 0 \\
      \end{bmatrix} \\
      &=
      \begin{bmatrix}
        0 & 0 & 0 \\
        0 & 0 & 0 \\
        0 & 0 & 0 \\
      \end{bmatrix}
    \end{aligned}
  \end{equation}
\end{definition}

%%% Local Variables:
%%% mode: latex
%%% TeX-master: "../../Math_333-MatrixAlg_ComplexVars-Reference_Sheet"
%%% End:


\subsection{Systems of Equations}\label{subsec:Systems_Equations}
A \nameref{def:Matrix} can be used to represent a system of equations.

For example,
\begin{align*}
  x + 2y + 3z &= 5 \\
  2x - y + z &= 3 \\
  x - y + 5z &= 2
\end{align*}
can be represented with a set of matrices as follows
\begin{equation*}
  \begin{bmatrix}
    1 & 2 & 3 \\
    2 & -1 & 1 \\
    1 & -1 & 5 \\
  \end{bmatrix}
  \begin{bmatrix}
    x \\
    y \\
    z \\
  \end{bmatrix}
  =
  \begin{bmatrix}
    5 \\
    3 \\
    2 \\
  \end{bmatrix}
\end{equation*}


%%% Local Variables:
%%% mode: latex
%%% TeX-master: "../../Math_333-MatrixAlg_ComplexVars-Reference_Sheet"
%%% End:


\subsection{Elementary Row Operations}\label{subsec:Elementary_Row_Ops}

%%% Local Variables:
%%% mode: latex
%%% TeX-master: "../../Math_333-MatrixAlg_ComplexVars-Reference_Sheet"
%%% End:


\subsection{Echelon Form}\label{subsec:Echelon_Form}

%%% Local Variables:
%%% mode: latex
%%% TeX-master: "../../Math_333-MatrixAlg_ComplexVars-Reference_Sheet"
%%% End:


\subsection{Gaussian Elimination}\label{subsec:Gaussian_Elimination}
\begin{definition}[Gaussian Elimination]\label{def:Gaussian_Elimination}
  \emph{Gaussian elimination} is a method of solving a system of linear equations using matrices and \nameref{def:Elementary_Row_Op}s.
\end{definition}

It is easier to show how \nameref{def:Gaussian_Elimination} works through an example.

\begin{example}[Lecture 12, Example 3]{Perform Gaussian Elimination}
  Solve the system of linear equations below.
  \begin{align*}
    x - 2y + 3z &= 5 \\
    2x + y - z &= 8 \\
    3x - y + 2z &= 13
  \end{align*}
  \tcblower{}
  Start by converting the system of linear equations to an \nameref{def:Augmented_Matrix}.
  \begin{align*}
    \begin{array}{cccl}
      1x &- 2y &+ 3z &= 5 \\
      2x &+ 1y &- 1z &= 8 \\
      3x &- 1y &+ 2z &= 13 \\
    \end{array}
         &=
           \begin{amat}{3}
             1 & -2 & 3 & 5 \\
             2 & 1 & -1 & 8 \\
             3 & -1 & 2 & 13 \\
           \end{amat} \\
    \intertext{Use repeated \nameref{def:Elementary_Row_Op}s to make lower rows have zeros, converting to \nameref{thm:Echelon_Form}.}
         &\grstep{-2r_{1}+r_{2}}\begin{amat}{3}
           1 & -2 & 3 & 5 \\
           0 & 5 & -7 & -2 \\
           3 & -1 & 2 & 13 \\
         \end{amat} \\
         &\grstep{-3r_{1}+r_{3}}\begin{amat}{3}
           1 & -2 & 3 & 5 \\
           0 & 5 & -7 & -2 \\
           0 & 5 & -7 & -2 \\
         \end{amat} \\
         &\grstep{\frac{1}{5}r_{2}}\begin{amat}{3}
           1 & -2 & 3 & 5 \\
           0 & 1 & \frac{-7}{5} & \frac{-2}{5} \\
           0 & 5 & -7 & -2 \\
         \end{amat} \\
         &\grstep{-5r_{2}+r_{3}}\begin{amat}{3}
           1 & -2 & 3 & 5 \\
           0 & 1 & \frac{-7}{5} & \frac{-2}{5} \\
           0 & 0 & 0 & 0 \\
         \end{amat} \\
  \end{align*}

  Now, we can reconvert back into a system of linear equations.
  \begin{equation*}
    \begin{amat}{3}
      1 & -2 & 3 & 5 \\
      0 & 1 & \frac{-7}{5} & \frac{-2}{5} \\
      0 & 0 & 0 & 0 \\
    \end{amat} \\
    =
    \begin{array}{cccl}
      1x &- 2y &+ 3z &= 5 \\
      0x &+ 1y &- \frac{7}{5}z &= \frac{-2}{5} \\
      0x &+ 0y &+ 0z &= 0 \\
    \end{array}
  \end{equation*}

  We now have 3 unknowns, but only 2 usable equations.
  Therefore, we solve for two of the unknowns in terms of a third.
  \begin{align*}
    y &= \frac{-2}{5} + \frac{7}{5}z \\
    x &= \frac{21}{5} - \frac{z}{5}
  \end{align*}
\end{example}

\begin{definition}[Consistent]\label{def:Consistent}
  A \emph{consistent} system of linear equations is one which has a solution.
\end{definition}


%%% Local Variables:
%%% mode: latex
%%% TeX-master: "../../Math_333-MatrixAlg_ComplexVars-Reference_Sheet"
%%% End:


\subsection{Elementary Matrix Operations}\label{subsec:Elementary_Matrix_Ops}
\begin{definition}[Elementary Matrix]\label{def:Elementary_Matrix}
  An \emph{elementary matrix} is a \nameref{def:Matrix} obtained by performing an \nameref{def:Elementary_Row_Op} on the \nameref{def:Multiplicative_Identity_Matrix}.
  This allows us to encode the action we take on the identity matrix \textbf{into} the identity matrix, essentially saving it.
\end{definition}

\begin{blackbox}
  Given the \nameref{def:Multiplicative_Identity_Matrix}, $I$, a regular \nameref{def:Matrix} $A$, perform a row reduction by adding $r_{1}$, multiplied by $\lambda$, of $A$ to $r_{2}$.
  \begin{align*}
    I &= \begin{bmatrix}
      1 & 0 \\
      0 & 1
    \end{bmatrix} \\
    A &= \begin{bmatrix}
      a_{11} & a_{12} \\
      a_{21} & a_{22}
    \end{bmatrix}
  \end{align*}

  We perform the \nameref{def:Elementary_Row_Op} on the \nameref{def:Multiplicative_Identity_Matrix}.
  \begin{equation*}
    \begin{bmatrix}
      1 & 0 \\
      0 & 1
    \end{bmatrix}
    \grstep{\lambda r_{1}+r_{2}}
    \begin{bmatrix}
      1 & 0 \\
      \lambda & 1
    \end{bmatrix}
  \end{equation*}

  Now that we have an \nameref{def:Elementary_Matrix}, we can perform the action we saved in the elementary matrix on any matrix, in this case, $A$.
  \begin{equation*}
    \begin{bmatrix}
      1 & 0 \\
      \lambda & 1
    \end{bmatrix}
    \begin{bmatrix}
      a_{11} & a_{12} \\
      a_{21} & a_{22}
    \end{bmatrix}
    =
    \begin{bmatrix}
      a_{11} & a_{12} \\
      \lambda a_{11} + a_{21} & \lambda a_{12} + a_{22}
    \end{bmatrix}
  \end{equation*}

  We can see that this result is the exact same as if we have performed the \nameref{def:Elementary_Row_Op} on the matrix $A$ directly.
\end{blackbox}


%%% Local Variables:
%%% mode: latex
%%% TeX-master: "../../Math_333-MatrixAlg_ComplexVars-Reference_Sheet"
%%% End:


\subsection{Inverse Matrices}\label{subsec:Inverse_Matrices}

%%% Local Variables:
%%% mode: latex
%%% TeX-master: "../../Math_333-MatrixAlg_ComplexVars-Reference_Sheet"
%%% End:


\subsection{Reduced Row Echelon Form}\label{subsec:RREF}
Using what we learned in \Cref{subsec:Inverse_Matrices} and \Cref{subsubsec:Inverse_Matrices_using_Elementary_Matrices}, we can solve systems of linear equations using a \nameref{def:Matrix}'s inverse.

This can be movtivated with a general example.
\begin{blackbox}
  Consider the system shown below.
  \begin{align*}
    a_{11} x_{1} + a_{12} x_{2} &= b_{1} \\
    a_{21} x_{1} + a_{22} x_{2} &= b_{2}
  \end{align*}

  We can convert this set of equations with 2 equations and 2 unknowns to a single linear equation with a single unknown.
  \textbf{Equivalently} to the system, we can write
  \begin{align*}
    \begin{pmatrix}
      a_{11} & a_{12} \\
      a_{21} & a_{22}
    \end{pmatrix}
    \begin{pmatrix}
      x_{1} \\
      x_{2}
    \end{pmatrix} &=
    \begin{pmatrix}
      a_{11} x_{1} + a_{12} x_{2} \\
      a_{21} x_{1} + a_{22} x_{2}
    \end{pmatrix} \\
    &=
      \begin{pmatrix}
        b_{1} \\
        b_{2}
      \end{pmatrix}
  \end{align*}

  The column matrices (column vectors) are technically one unknown with multiple unknown components.

  Just like with regular algebra, we can use the inverse of the items we know to solve the equation.
  \begin{align*}
    \intertext{Let}
    A &=
        \begin{pmatrix}
          a_{11} & a_{12} \\
          a_{21} & a_{22}
        \end{pmatrix} \\
    X &=
        \begin{pmatrix}
          x_{1} \\
          x_{2}
        \end{pmatrix} \\
    B &=
        \begin{pmatrix}
          b_{1} \\
          b_{2}
        \end{pmatrix} \\
    AX &= B \\
    \shortintertext{We make the assumption $A$ is invertible.}
    A^{-1} (A X) &= A^{-1} B \\
    (A^{-1} A) X &= A^{-1} B \\
    I X &= A^{-1} B
  \end{align*}

  Therefore, we can solve the system using $A^{-1}$.
\end{blackbox}


%%% Local Variables:
%%% mode: latex
%%% TeX-master: "../../Math_333-MatrixAlg_ComplexVars-Reference_Sheet"
%%% End:


\subsection{Linear Independence}\label{subsec:Linear_Independence}
\begin{theorem}[Linearly Dependent]\label{thm:Linearly_Dependent}
  Let $S$ be a set of vectors of size $n$, $S = \lbrace v_{1}, v_{2}, \ldots, v_{n} \rbrace$.
  Each vector $v_{k}$ has components in $\RealNumbers^{m}$ ($m$ components from $\RealNumbers$).

  The set is \emph{linearly dependent} if there exists scalars $c_{1}, c_{2}, \ldots, c_{n}$, at least one of which is not equal to zero, such that the below equation holds true.
  \begin{equation}\label{eq:Linearly_Dependent}
    c_{1}v_{1} + c_{2}v_{2} + \cdots + c_{n}v_{n} = 0
  \end{equation}

  \begin{remark*}
    Note that the $0$ in \Cref{eq:Linearly_Dependent} implies the zero vector, not necessarily the scalar value $0$.
  \end{remark*}
\end{theorem}


%%% Local Variables:
%%% mode: latex
%%% TeX-master: "../../Math_333-MatrixAlg_ComplexVars-Reference_Sheet"
%%% End:


\subsection{Matrix Rank}\label{subsec:Matrix_Rank}
\begin{definition}[Rank]\label{def:Matrix_Rank}
  The \emph{rank} of a \nameref{def:Matrix} $A_{n \by m}$ is the largest number of \nameref{def:Linearly_Independent} ``chunks'' of a matrix.

  There are 2 kinds of rank: \nameref{rmk:Row_Rank} and \nameref{rmk:Column_Rank}.

  \begin{remark}[Row Rank]\label{rmk:Row_Rank}
    The \emph{rank} of a \nameref{def:Matrix} $A_{n \by m}$ is the largest number of \nameref{def:Linearly_Independent} rows of a matrix.
  \end{remark}

  \begin{remark}[Column Rank]\label{rmk:Column_Rank}
    The \emph{column rank} of a \nameref{def:Matrix} $A_{n \by m}$ is the largest number of \nameref{def:Linearly_Independent} columns of a matrix.
  \end{remark}

  \begin{remark}[Row Rank Column Rank Equivalence]
    \nameref{rmk:Row_Rank} can be proven to be the same as the \nameref{rmk:Column_Rank} for all matrices.
  \end{remark}
\end{definition}

%%% Local Variables:
%%% mode: latex
%%% TeX-master: "../../Math_333-MatrixAlg_ComplexVars-Reference_Sheet"
%%% End:


\subsection{Determinants}\label{subsec:Determinants}
Before calculating the \nameref{def:Determinant}, we need to have some additional vocabulary wto work with.

\begin{definition}[Minor]\label{def:Minor}
  The \emph{minor} $\Minor{i}{j}$ is the \nameref{def:Determinant} of the sub\nameref{def:Matrix} obtained by deleting the $i$th row and $j$th column.
\end{definition}

\begin{blackbox}
  Given the matrix $
  \begin{pmatrix}
    1 & -1 & 3 \\
    2 & 5 & -1 \\
    3 & 0 & 5
  \end{pmatrix}$, the \nameref{def:Minor} at the index $3,2$ is found by deleting the \nth{3} row and \nth{2} column and taking the \nameref{def:Determinant} of the resulting submatrix.
  Thus,
  \begin{align*}
    \Minor{3}{2} &= \det
                   \begin{pmatrix}
                     1 & 3 \\
                     2 & -1
                   \end{pmatrix} \\
    \intertext{Use the definition of the \nameref{def:Determinant} for a $2 \by 2$ \nameref{def:Matrix}.}
                 &= 1(-1) - 3(2) \\
                 &= -1 - 6 \\
                 &= -7
  \end{align*}
\end{blackbox}

\begin{definition}[Cofactor]\label{def:Cofactor}
  The \emph{cofactor} $\Cofactor{i}{j}$ is related to the \nameref{def:Minor}, $\Minor{i}{j}$ by the equation below.
  \begin{equation}\label{eq:Cofactor}
    \Cofactor{i}{j} = {(-1)}^{i + j} \Minor{i}{j}
  \end{equation}
\end{definition}

\begin{blackbox}
  Using the same \nameref{def:Matrix} from earlier, the \nameref{def:Cofactor} of of the element in the \nth{3} row and \nth{2} column is:
  \begin{align*}
    \Cofactor{3}{2} &= {(-1)}^{3+2} \Minor{3}{2} \\
                    &= (-1) (-7) \\
                    &= 7
  \end{align*}
\end{blackbox}

Now with these terms defined, we can define the general algorithm for the \nameref{def:Determinant}.

\begin{definition}[Determinant]\label{def:Determinant}
  The \emph{determinant} of a \nameref{def:Matrix} is a scalar value that is computed out of a \textbf{square} matrix and encodes certain properties of the transformation specified by the matrix.

  There are 2 equations in use for the determinant, depending on the size of the matrix.
  For a $2 \by 2$ matrix,
  \begin{equation}\label{eq:Determinant_2x2}
    \det
    \begin{pmatrix}
      a & b \\
      c & d
    \end{pmatrix}
    = ad - bc
  \end{equation}

  For a matrix larger than $2 \by 2$, we have many possible ways of finding the determinant.
  This is explored further in \Cref{subsubsec:Expand_Determinant}, with their equations given in \Cref{eq:Determinant_Expand_ith_Row} and \Cref{eq:Determinant_Expand_jth_Column}.
\end{definition}


%%% Local Variables:
%%% mode: latex
%%% TeX-master: "../../Math_333-MatrixAlg_ComplexVars-Reference_Sheet"
%%% End:


\subsection{Eigenvectors}\label{subsec:Eigenvectors}
\begin{definition}[Eigenvector]\label{def:Eigenvector}
  Let $A$ be an $n \by n$ matrix, denoted $A_{n \by n}$ and $X_{n \by 1}$ be a column vector of unknowns.
  $X_{n \by 1}$ is said to be an \emph{eigenvector} of $A$ if:

  \begin{equation}\label{eq:Eigenvector}
    AX = \lambda X
  \end{equation}

  \begin{propertylist}
  \item $X \neq 0$.\label{prop:Eigenvector_Nonzero}
  \item $\lambda$ is a scalar, called an \nameref{def:Eigenvalue}.\label{prop:Eigenvector_Value}.
  \end{propertylist}

  \begin{remark}[Uniqueness of Eigenvectors]
    There can be infinitely many \nameref{def:Eigenvector}s for a given \nameref{def:Eigenvalue}.
  \end{remark}
\end{definition}

\begin{example}[Lecture 17, Example 1]{Verify Vector is not an Eigenvector}
  Given $A$ and $Y$, is $Y$ an \nameref{def:Eigenvector} of $A$?
  \begin{align*}
    A &=
        \begin{pmatrix}
          2 & 3 \\
          6 & -1
        \end{pmatrix} &
                        Y &=
                            \begin{pmatrix}
                              2 \\
                              -1
                            \end{pmatrix}
  \end{align*}
  \tcblower{}
  First, we check that the two properties of an \nameref{def:Eigenvector} are satisfied.
  \Cref{prop:Eigenvector_Nonzero} is satisfied ($Y \neq 0$).

  Now, we attempt to find the \nameref{def:Eigenvalue} for this particular \nameref{def:Eigenvector}.
  \begin{align*}
    AY &=
         \begin{pmatrix}
           2 & 3 \\
           6 & -1
         \end{pmatrix}
               \begin{pmatrix}
                 2 \\
                 -1
               \end{pmatrix} \\
    &=
      \begin{pmatrix}
        1 \\
        13
      \end{pmatrix} \\
  \end{align*}

  The problem here is that there is \textbf{no} $\lambda$ such that the equation for an \nameref{def:Eigenvalue} is satisfied.
  \begin{equation*}
    \lambda
    \begin{pmatrix}
      2 \\
      -1
    \end{pmatrix}
    =
    \begin{pmatrix}
      1 \\
      13
    \end{pmatrix}
  \end{equation*}

  $\therefore$ $Y$ is not an \nameref{def:Eigenvector} of $A$.
\end{example}

\begin{example}[Lecture 17, Example 2]{Verify Vector is Eigenvector}
  Verify that $X$ is an \nameref{def:Eigenvector} of $A$?
  \begin{align*}
    A &=
        \begin{pmatrix}
          6 & -1 \\
          2 & 3
        \end{pmatrix} &
                        X &=
                            \begin{pmatrix}
                              1 \\
                              1
                            \end{pmatrix}
  \end{align*}
  \tcblower{}
  Verifying \Cref{prop:Eigenvector_Nonzero} is done by simple observation.

  Now, we need to attempt to find a $\lambda$ such that \Cref{eq:Eigenvector} is satisfied.
  \begin{align*}
    AX &=
         \begin{pmatrix}
           6 & -1 \\
           2 & 3
         \end{pmatrix}
               \begin{pmatrix}
                 1 \\
                 1
               \end{pmatrix} \\
    &=
      \begin{pmatrix}
        5 \\
        5
      \end{pmatrix} \\
    &= 5
      \begin{pmatrix}
        1 \\
        1
      \end{pmatrix} \\
    &= 5 X
  \end{align*}

  $\therefore$ $X$ \textit{is} an \nameref{def:Eigenvector} of $A$, with the \nameref{def:Eigenvalue} $\lambda = 5$.
\end{example}

\begin{lemma}\label{lem:Matrix_Multiply_Zero_Vector}
  Let $B_{n \by n}$ be a \nameref{def:Matrix} with an \nameref{def:Inverse_Matrix} (it is invertible), meaning $B \neq 0$.
  Suppose $B X_{n \by 1} = 0$.

  Then, $X_{n \by 1} = 0$.

  \begin{remark*}
    This mirrors normal algebra, when $ab = 0$, and $a \neq 0$, then $b$ \textbf{must} be $0$.
  \end{remark*}
\end{lemma}

\begin{proof}[Proof of \Cref*{lem:Matrix_Multiply_Zero_Vector}]
  Suppose $BX = 0$.

  Then,
  \begin{align*}
    B^{-1} (BX) &= B^{-1} 0 \\
    \intertext{By the rules of \nameref{prop:Matrix_Associativity}.}
    (B^{-1} B) X &= 0 \\
    \intertext{By the definition of the \nameref{def:Multiplicative_Identity_Matrix}.}
    IX &= 0 \\
    X &= 0
  \end{align*}
\end{proof}

\begin{lemma}[Contrapositive of \Cref*{lem:Matrix_Multiply_Zero_Vector}]\label{lem:Contrapositive_Matrix_Multiply_Zero_Vector}
  If $BX = 0$ and $X \neq 0$, then $B$ has no inverse.
  Namely, this means that $\det B = 0$.
\end{lemma}

\subsection{Eigenvalues}\label{subsec:Eigenvalues}
\begin{definition}[Eigenvalue]\label{def:Eigenvalue}
  Let $A$ be an $n \by n$ matrix, denoted $A_{n \by n}$ and $X_{n \by 1}$ be a column vector of unknowns.
  If $X_{n \by 1}$ is an eigenvector of $A$, then
  \begin{equation}\label{eq:Eigenvalue}
    AX = \lambda X
  \end{equation}

  \begin{propertylist}
  \item $X \neq 0$.
  \item $\lambda$ is a scalar, called an \emph{Eigenvalue}.\label{prop:Eigenvalue}
  \end{propertylist}
\end{definition}

However, \Cref{eq:Eigenvalue} is difficult to work with when attempting to find the \nameref{def:Characteristic_Polynomial}.
So, we can perform some reductions that make it easier to use.
\begin{align*}
  AX &= \lambda X \\
  \shortintertext{Remember that $X \neq 0$.}
  AX - \lambda X &= 0 \\
  \intertext{We cannot factor $X$ as it is right now, because it multiplies two different things, a \nameref{def:Matrix} and a scalar.
  We can solve this by using the \nameref{def:Multiplicative_Identity_Matrix}.}
  AX - \lambda I X &= 0 \\
  (A - \lambda I) X &= 0 \\
  \intertext{Now, using \Cref{lem:Contrapositive_Matrix_Multiply_Zero_Vector}, $\det B = 0$.}
  \det(A-\lambda I) &= 0 \\
  \shortintertext{Where $lambda$ is an \nameref{def:Eigenvalue}.}
\end{align*}

\begin{definition}[Characteristic Polynomial]\label{def:Characteristic_Polynomial}
  Consider $\det(A_{n \by n}-\lambda I_{n \by n})$ where $A$ is indeterminant.
  This will form a polynomial of degree $n$ in $\lambda$, the \emph{characteristic polynomial}.

  If $\lambda$ \emph{is} an \nameref{def:Eigenvalue}, then $\det(A-\lambda I) = 0$.
\end{definition}

\begin{example}[Lecture 17, Example 4]{Find all Eigenvalues and Eigenvectors}
  Find all \nameref{def:Eigenvalue}s and \nameref{def:Eigenvector}s of $A$?
  \begin{equation*}
    A =
    \begin{pmatrix}
      2 & 3 & 6 \\
      0 & 4 & 4 \\
      0 & 0 & 2
    \end{pmatrix}
  \end{equation*}
  \tcblower{}
  Start by the \nameref{def:Characteristic_Polynomial} of $A$.
  \begin{align*}
    A &=
    \begin{pmatrix}
      2 & 3 & 6 \\
      0 & 4 & 4 \\
      0 & 0 & 2
    \end{pmatrix} \\
    A - \lambda I &=
                    \begin{pmatrix}
                      2 - \lambda & 3 & 6 \\
                      0 & 4 - \lambda & 4 \\
                      0 & 0 & 2 - \lambda \\
                    \end{pmatrix} \\
    \intertext{This \nameref{def:Matrix} is \nameref{def:Upper_Triangular}.}
    \det(A - \lambda I) &= (2 - \lambda) (4 - \lambda) (2 - \lambda)
  \end{align*}

  With $A$'s \nameref{def:Characteristic_Polynomial}, we can find its roots to determine the \nameref{def:Eigenvalue}s for $A$.
  As we can see, there are two possible eigenvalues:
  \begin{description}[noitemsep]
  \item[$\lambda = 2$] \nameref{def:Algebraic_Multiplicity} 2
  \item[$\lambda = 4$] \nameref{def:Algebraic_Multiplicity} 1, or simple.
  \end{description}

  Now we need to find all the possible \nameref{def:Eigenvector}s for these \nameref{def:Eigenvalue}s.
  Starting with $\lambda = 2$:
  \begin{align*}
    A - 4I &=
             \begin{pmatrix}
               0 & 3 & 6 \\
               0 & 2 & 4 \\
               0 & 0 & 0
             \end{pmatrix} \\
    (A-\lambda I) X &= 0 \\
    \begin{pmatrix}
      0 & 3 & 6 \\
      0 & 2 & 4 \\
      0 & 0 & 0
    \end{pmatrix}
              \begin{pmatrix}
                x_{1} \\
                x_{2} \\
                x_{3}
              \end{pmatrix}
    &=
      \begin{pmatrix}
        0 \\
        0 \\
        0
      \end{pmatrix}
  \end{align*}

  This yields a system of equations that we can solve.
  \begin{align*}
    3x_{2} + 6x_{3} &= 0 \\
    2x_{2} + 4x_{3} &= 0 \\
    \intertext{We can see that each equation is just a multiple of a simpler equation.}
    3 (x_{2} + 2x_{3}) &= 0 \\
    2 (x_{2} + 2x_{3}) &= 0 \\
    \shortintertext{Now, solving this system.}
    x_{2} &= -2x_{3}
  \end{align*}

  Now, we put it back into a solution matrix.
  \begin{align*}
    \begin{pmatrix}
      x_{1} \\
      x_{2} \\
      x_{3}
    \end{pmatrix} &=
                    \begin{pmatrix}
                      x_{1} \\
                      -2x_{3} \\
                      x_{3}
                    \end{pmatrix} \\
    \intertext{Now, because of superposition, we can say}
    &=
      \begin{pmatrix}
        x_{1} + 0x_{3} \\
        0x_{1} - 2x_{3} \\
        0x_{1} + x_{3}
      \end{pmatrix} \\
    &=
      \begin{pmatrix}
        x_{1} \\
        0 \\
        0
      \end{pmatrix} +
    \begin{pmatrix}
      0 \\
      -2x_{3} \\
      x_{3}
    \end{pmatrix} \\
    &= x_{1}
      \begin{pmatrix}
        1 \\
        0 \\
        0 \\
      \end{pmatrix} +
    x_{3}
    \begin{pmatrix}
      0 \\
      -2 \\
      1
    \end{pmatrix}
  \end{align*}
  Where both $x_{1}$ and $x_{3}$ cannot be $0$ at the same time.

  We need two matrices to find all \nameref{def:Eigenvector}s of the \nameref{def:Eigenvalue} $\lambda = 2$.
  Thus, the \nameref{def:Eigenvalue} $\lambda = 2$ has a \nameref{def:Geometric_Multiplicity} of 2.

  Similarly, we can solve for the \nameref{def:Eigenvalue} $\lambda = 4$, which will have \nameref{def:Geometric_Multiplicity} 1.
\end{example}

\begin{definition}[Algebraic Multiplicity]\label{def:Algebraic_Multiplicity}
  \emph{Algebraic multiplicity} is the number or identical roots a function may have.
  When it comes to finding the root of a \nameref{def:Matrix}'s \nameref{def:Characteristic_Polynomial}, the algebraic multiplicity is the exponent on a given root.
\end{definition}


%%% Local Variables:
%%% mode: latex
%%% TeX-master: "../../Math_333-MatrixAlg_ComplexVars-Reference_Sheet"
%%% End:


\subsection{Cayley-Hamilton Theorem}\label{subsec:Cayley-Hamilton_Theorem}
\begin{theorem}[Cayley-Hamilton Theorem]\label{thm:Cayley-Hamilton_Theorem}
  A square \nameref{def:Matrix} $A_{n \by n}$ satisfies its own \nameref{def:Characteristic_Polynomial}.
  \begin{remark*}
    ``Satisfies'' in this context means that the \nameref{def:Characteristic_Polynomial} will return $0$ when the variable $\lambda$ is replaced by a value.
  \end{remark*}
\end{theorem}

\begin{example}[Lecture 17, Example 5]{Verify the Cayley-Hamilton Theorem}
  Verify the \nameref{thm:Cayley-Hamilton_Theorem} for the \nameref{def:Matrix} $A$?
  \begin{equation*}
    A =
    \begin{pmatrix}
      6 & -1 \\
      2 & 3
    \end{pmatrix}
  \end{equation*}
  \tcblower{}
  We need to find $A$'s \nameref{def:Characteristic_Polynomial}.
  \begin{align*}
    A - \lambda I &=
                    \begin{pmatrix}
                      6 - \lambda & -1 \\
                      2 & 3 - \lambda
                    \end{pmatrix} \\
    \det(A - \lambda I) &= (6-\lambda) (3-\lambda) - (-1)(2) \\
                  &= \lambda^{2} - 9\lambda + 20
  \end{align*}

  The \nameref{thm:Cayley-Hamilton_Theorem} asserts:
  \begin{align*}
    A^{2} - 9 A + 20I &= 0 \\
    \begin{pmatrix}
      6 & -1 \\
      2 & 3
    \end{pmatrix}
          \begin{pmatrix}
            6 & -1 \\
            2 & 3
          \end{pmatrix} -
                9
                \begin{pmatrix}
                  6 & -1 \\
                  2 & 3
                \end{pmatrix} +
                      \begin{pmatrix}
                        20 & 0 \\
                        0 & 20
                      \end{pmatrix}
        &\qeq 0 \\
    \begin{pmatrix}
      34 & -9 \\
      18 & 7
    \end{pmatrix} -
           \begin{pmatrix}
             54 & -9 \\
             18 & 27
           \end{pmatrix} +
                  \begin{pmatrix}
                    20 & 0 \\
                    0 & 20
                  \end{pmatrix}
                        &\qeq
                          \begin{pmatrix}
                            0 & 0 \\
                            0 & 0
                          \end{pmatrix} \\
    \begin{pmatrix}
      -20 & 0 \\
      0 & -20
    \end{pmatrix} +
          \begin{pmatrix}
            20 & 0 \\
            0 & 20
          \end{pmatrix} &\qeq
                          \begin{pmatrix}
                            0 & 0 \\
                            0 & 0
                          \end{pmatrix} \\
    \begin{pmatrix}
      0 & 0 \\
      0 & 0
    \end{pmatrix} &\overset{\checkmark}{=}
                    \begin{pmatrix}
                      0 & 0 \\
                      0 & 0
                    \end{pmatrix}
  \end{align*}
\end{example}

\subsubsection{Express Higher Powers}\label{subsubsec:Cayley-Hamilton_Express_Higher_Powers}
The \nameref{thm:Cayley-Hamilton_Theorem} states that if $\lambda = A$, then the \nameref{def:Characteristic_Polynomial} will equal 0.
If that is the case, then we can solve for higher powers of the polynomial using repeated substitution.

It is easiest to show this with an example, \Cref{ex:Cayley-Hamilton Higher Powers}.

\begin{example}[Lecture 18, Example 1]{Cayley-Hamilton Higher Powers}
  Given the \nameref{def:Matrix} $A =
  \begin{pmatrix}
    2 & -2 \\
    1 & 5 \\
  \end{pmatrix}$, find all $A^{n}$?
  We start by finding the \nameref{def:Characteristic_Polynomial}.
  \begin{align*}
    \det (A - \lambda I) &= \det
                           \begin{pmatrix}
                             2-\lambda & -2 \\
                             1 & 5 - \lambda
                           \end{pmatrix} \\
                         &= (2-\lambda)(5 - \lambda) - (-2)(1) \\
                         &= \lambda^{2}- 7\lambda + 12
  \end{align*}

  If $A$ satistifies the \nameref{thm:Cayley-Hamilton_Theorem}, then $A^{2} - 7A + 12I = 0$ will hold true.
  In that case, we can express exponents greater than 1 like so:
  \begin{align*}
    A^{2} &= 7A - 12I \\
    A^{3} &= A A^{2} \\
          &= A (7A - 12I) \\
          &= 7A^{2} - 12A \\
          &= 7(7A-12I) - 12A \\
          &= 49A - 84I - 12A \\
          &= 37A - 84I
  \end{align*}

  This computation can be repeated for higher powers, and for equations whigh higher starting characteristic equation exponents.
\end{example}

If we generalize, we can see that any exponent power of a \nameref{def:Matrix} $A$ can be expressed as a linear combination of $A$ and $I$.
\begin{equation}\label{eq:Matrix_Constant_Time_Higher_Power}
  A^{n} = c_{n} A + d_{n} I
\end{equation}

Similar to how \Cref{eq:Matrix_Constant_Time_Higher_Power} exists, a similar equation exists for the \nameref{def:Eigenvalue}s of a given \nameref{def:Matrix}, seen in \Cref{eq:Eigenvalue_Constant_Time_Higher_Power}.
\begin{equation}\label{eq:Eigenvalue_Constant_Time_Higher_Power}
  \lambda^{n} = c_{n} \lambda + d_{n}
\end{equation}

If we solve for $c_{n}$ and $d_{n}$ in \Cref{eq:Eigenvalue_Constant_Time_Higher_Power}, using Cramer's Rule, we can find general forms of $c_{n}$ and $d_{n}$.

For the example in \Cref{ex:Cayley-Hamilton Higher Powers}, if we plug in $\lambda = 3$ and $\lambda = 4$, then we have two equationsto work with.
\begin{align*}
  3 c_{n} + d_{n} &= 3^{n} \\
  4c_{n} + d_{n} &= 3^{n}
\end{align*}

Using Cramer's Rule, we can solve for $c_{n}$ and $d_{n}$.
\begin{align*}
  c_{n} &= \frac{\det
          \begin{pmatrix}
            3^{n} & 1 \\
            4 ^{n} & 1
          \end{pmatrix}}{\det
                     \begin{pmatrix}
                       3 & 1 \\
                       3 & 1
                     \end{pmatrix}} \\
  d_{n} &= \frac{\det
          \begin{pmatrix}
            3 & 3^{n} \\
            4 & 4^{n}
          \end{pmatrix}}{\det
                \begin{pmatrix}
                  3 & 1 \\
                  4 & 1
                \end{pmatrix}
                      } \\
  \shortintertext{Now, solving for these scalars:}
  c_{n} &= 4^{n} - 3^{n} \\
  d_{n} &= 3 \bigl( 3^{n} \bigr) - 3 \bigl( 4^{n} \bigr)
\end{align*}

\subsubsection{Inverses}\label{subsubsec:Cayley-Hamilton_Inverses}
We can use \nameref{thm:Cayley-Hamilton_Theorem} to solve for \nameref{def:Inverse_Matrix} too.
\begin{example}[Lecture 18, Example 2]{Inverse with Cayley-Hamilton Theorem}
  Given the \nameref{def:Matrix} $A =
  \begin{pmatrix}
    2 & -2 \\
    1 & 5
  \end{pmatrix}$, find $A^{-1}$?
  Using the \nameref{thm:Cayley-Hamilton_Theorem}, we can express the \nameref{def:Characteristic_Polynomial} as shown:
  \begin{align*}
    A^{2} - 7A + 12I &= 0 \\
    A^{2} - 7A &= -12I \\
    A^{2} - 7IA &= -12I \\
    A (A - 7I) &= -12I \\
    A \frac{A-7I}{-12} &= I \\
    A^{-1} &= \frac{A-7I}{-12}
  \end{align*}
\end{example}

%%% Local Variables:
%%% mode: latex
%%% TeX-master: "../../Math_333-MatrixAlg_ComplexVars-Reference_Sheet"
%%% End:


\subsection{Diagonalization}\label{subsec:Diagonalization}
To start off with, we need some background knowledge on what we mean by \nameref{def:Diagonalization}.

\begin{definition}[Diagonal Matrix]\label{def:Diagonal_Matrix}
  A \emph{diagonal matrix} is a \nameref{def:Matrix} whose only non-zero elements are on the main diagonal of the matrix.

  In general, this is seen as:
  \begin{equation}\label{eq:Diagonal_Matrix}
    A =
    \begin{pmatrix}
      a_{1,1} & 0 & 0 & \cdots \\
      0 & a_{2,2} & 0 & \cdots \\
      \vdots & \ddots & \ddots & \vdots \\
      0 & \cdots & \cdots & a_{n, n}
    \end{pmatrix}
  \end{equation}
\end{definition}

\begin{definition}[Diagonalizable]\label{def:Diagonalizable}
  Let the \nameref{def:Matrix} $A_{n \by n}$.
  We say $A$ is \emph{diagonalizable} to a \nameref{def:Diagonal_Matrix} $D$ if there exists an invertible matrix $P$ such that \Cref{eq:Diagonalizable} holds true.

  \begin{equation}\label{eq:Diagonalizable}
    P^{-1} A P = D
  \end{equation}
\end{definition}

\begin{definition}[Diagonalization]\label{def:Diagonalization}
  Let the matrix $A_{n \by n}$ be \nameref{def:Diagonalizable} by an invertible \nameref{def:Matrix} $P$ to form a \nameref{def:Diagonal_Matrix} $D$.
  $P$ is called a matrix that implements the \emph{diagonalization} shown in \Cref{eq:Diagonalization}.

  \begin{equation}\label{eq:Diagonalization}
    AP = PD
  \end{equation}
\end{definition}

Now that we have the terms and definitions out of the way, we can see something interesting about \Cref{eq:Diagonalization}.
\begin{blackbox}
  Let $A_{2 \by 2}$, $P =
  \begin{pmatrix}
    p_{1} & p_{2}
  \end{pmatrix}
  $, and $D =
  \begin{pmatrix}
    d_{1} & 0 \\
    0 & d_{2}
  \end{pmatrix}$.

  This means that if we apply \Cref{eq:Diagonalization}:
  \begin{align*}
    AP &= PD \\
    A_{2 \by 2}
    \begin{pmatrix}
      p_{1} & p_{2}
    \end{pmatrix} &=
                    \begin{pmatrix}
                      p_{1} & p_{2}
                    \end{pmatrix}
                              \begin{pmatrix}
                                d_{1} & 0 \\
                                0 & d_{2} \\
                              \end{pmatrix} \\
    \begin{pmatrix}
      A p_{1} & A p_{2}
    \end{pmatrix} &=
                    \begin{pmatrix}
                      d_{1} p_{1} & d_{2} p_{2}
                    \end{pmatrix}
  \end{align*}

  This means that:
  \begin{align*}
    A p_{1} &= d_{1} p_{1} \\
    A p_{2} &= d_{2} p_{2}
  \end{align*}

  If we study this, we can see that $d_{1}$ and $d_{2}$ are \nameref{def:Eigenvalue}s of $A$ (Remember, $AX = \lambda X$).
  Therefore, $P$ is a \nameref{def:Matrix} that implements the \nameref{def:Diagonalization} using a matrix of corresponding \nameref{def:Eigenvector}s.
\end{blackbox}

\begin{example}[Lecture 19, Example 1]{Matrices Implementing Diagonalization}
  Let $A_{3 \by 3}$.
  Is $A$ \nameref{def:Diagonalizable}?
  Can $A$ be diagonalized to $B$, $C$, $E$, or $F$?
  If yes, what are $D$ and $P$?
  \begin{equation*}
    A =
        \begin{pmatrix}
          1 & 3 & 12 \\
          0 & 6 & 8 \\
          0 & 0 & 2
        \end{pmatrix}
  \end{equation*}
  \begin{align*}
    B &=
        \begin{pmatrix}
          0 & 0 & 0 \\
          0 & 2 & 0 \\
          0 & 0 & 6
        \end{pmatrix} &
    C &=
        \begin{pmatrix}
          6 & 0 & 0 \\
          0 & 6 & 0 \\
          0 & 0 & 2
        \end{pmatrix} \\
    E &=
        \begin{pmatrix}
          2 & 0 & 0 \\
          0 & 6 & 0 \\
          0 & 0 & 2
        \end{pmatrix} &
    F &=
        \begin{pmatrix}
          6 & 0 & 0 \\
          0 & 2 & 0 \\
          0 & 0 & 2
        \end{pmatrix}
  \end{align*}
  \tcblower{}
  Start by finding the \nameref{def:Eigenvalue}s of $A$.
  \begin{align*}
    A - \lambda I &=
                    \begin{pmatrix}
                      2 - \lambda & 3 & 12 \\
                      0 & 6 - \lambda & 8 \\
                      0 & 0 & 2 - \lambda \\
                    \end{pmatrix} \\
    \det(A - \lambda I) &= (2 - \lambda) (6 - \lambda) (2 - \lambda) \\
    &= {(2 - \lambda)}^{2} (6 - \lambda)
  \end{align*}

  To get \nameref{def:Eigenvalue}s, $\det(A - \lambda I) = 0$.
  Therefore, the eigenvalues of $A$ are:
  \begin{align*}
    \lambda &= 2, \; \text{Algebraic Multiplicity 2} \\
    \lambda &= 6, \; \text{Algebraic Multiplicity 1}
  \end{align*}

  Now, we see that $B$ and $C$ are not possible \nameref{def:Diagonalization}s.
  \begin{description}[noitemsep]
  \item[$B$] Not all columns are \nameref{def:Eigenvalue}s.
  \item[$C$] Not the proper amount of $2$s in the matrix.
  \end{description}

  $E$ and $F$ are \textit{possible} \nameref{def:Diagonalization}s, but we need to know more about the corresponding \nameref{def:Eigenvector}s before we can answer them.
  So, let's find the eigenvectors.

  For $\lambda = 6$:
  \begin{align*}
    A - \lambda I &=
                    \begin{pmatrix}
                      -4 & 3 & 12 \\
                      0 & 0 & 8 \\
                      0 & 0 & -4 \\
                    \end{pmatrix} \\
    \begin{pmatrix}
      -4 & 3 & 12 \\
      0 & 0 & 8 \\
      0 & 0 & -4 \\
    \end{pmatrix}
    \begin{pmatrix}
      x_{1} \\ x_{2} \\ x_{3}
    \end{pmatrix} &=
                    \begin{pmatrix}
                      0 \\ 0 \\ 0
                    \end{pmatrix}
  \end{align*}

  This matrix equation yields the system of equations:
  \begin{align*}
    -4x_{1} + 3x_{2} + 12x_{3} &= 0 \\
    8x_{3} &= 0 \\
    -4x_{3} &= 0
  \end{align*}

  This means that $x_{3} = 0$, implying $x_{1} = \frac{3}{4} x_{2}$.
  Therefore,
  \begin{align*}
    \begin{pmatrix}
      x_{1} \\ x_{2} \\ x_{3}
    \end{pmatrix} &=
                    \begin{pmatrix}
                      \frac{3}{4} x_{2} \\ x_{2} \\ 0
                    \end{pmatrix} \\
    &= x_{2}
      \begin{pmatrix}
        \frac{3}{4} \\ 1 \\ 0
      \end{pmatrix} \\
    \intertext{Remember, any non-zero scalar multiple is allowed, so we can simplify this matrix.}
    &= x_{2}
      \begin{pmatrix}
        3 \\ 4 \\ 0
      \end{pmatrix}
  \end{align*}

  Now, we find the \nameref{def:Eigenvector} for $\lambda = 2$:
  \begin{align*}
    A - \lambda I &=
                    \begin{pmatrix}
                      0 & 3 & 12 \\
                      0 & 4 & 8 \\
                      0 & 0 & 0
                    \end{pmatrix} \\
    \begin{pmatrix}
      0 & 3 & 12 \\
      0 & 4 & 8 \\
      0 & 0 & 0
    \end{pmatrix}
              \begin{pmatrix}
                x_{1} \\ x_{2} \\ x_{3}
              \end{pmatrix} &=
                              \begin{pmatrix}
                                0 \\ 0 \\ 0
                              \end{pmatrix}
    0x_{1} + 3x_{2} + 12x_{3} &= 0 \\
    0x_{1} + 4x_{2} + 8x_{3} &= 0 \\
    0x_{1} + 0x_{2} + 0x_{3} &= 0 \\
  \end{align*}

  Therefore, we say:
  \begin{align*}
    \begin{pmatrix}
      x_{1} \\ x_{2} \\ x_{3}
    \end{pmatrix} &=
                    \begin{pmatrix}
                      x_{1} \\ 0 \\ 0
                    \end{pmatrix} \\
    &= x_{1}
      \begin{pmatrix}
        1 \\ 0 \\ 0
      \end{pmatrix}
  \end{align*}

  Now, we can construct $P$ using any multiple of the \nameref{def:Eigenvector}'s scalar.
  \begin{align*}
    P &=
        \begin{pmatrix}
          \lambda = 6 & \lambda = 2 & \lambda = 2 \\
          3 & 1 & 1 \\
          4 & 0 & 0 \\
          0 & 0 & 0
        \end{pmatrix} \\
    &=
      \begin{pmatrix}
        1 & 3 & 1 \\
        0 & 4 & 0 \\
        0 & 0 & 0
      \end{pmatrix}
  \end{align*}

  Therefore, \textbf{both} $E$ and $F$ are possible \nameref{def:Diagonalization}s.
  However, because $P$ has two proportional columns, $\det(P) = 0$.

  The \nameref{def:Determinant} being equal to zero means $P$ is not invertible.
  Therefore, $A$ is \textbf{not} \nameref{def:Diagonalizable}.
\end{example}

\begin{example}[Lecture 19, Example 2]{Diagonalizable Matrix}
  Is the \nameref{def:Matrix} $B$ \nameref{def:Diagonalizable}?
  If so, what is $D$ and $P$?
  \begin{equation*}
    B = \begin{pmatrix}
      2 & 3 & 6 \\
      0 & 6 & 8 \\
      0 & 0 & 2
    \end{pmatrix}
  \end{equation*}
  \tcblower{}
  First, we start by finding the \nameref{def:Eigenvalue}s.
  \begin{align*}
    B - \lambda I &=
                    \begin{pmatrix}
                      2-\lambda & 3 & 6 \\
                      0 & 6-\lambda & 8 \\
                      0 & 0 & 2-\lambda
                    \end{pmatrix} \\
    \det(B - \lambda I) &= (2-\lambda) (6-\lambda) (2-\lambda)
  \end{align*}

  We get \nameref{def:Eigenvalue}s if and only if $\det(B-\lambda I) = 0$.
  Thus, our eigenvalues are:
  \begin{description}[noitemsep]
  \item $\lambda = 2$, with algebraic multiplicity 2
  \item $\lambda = 6$, with algebraic multiplicity 1
  \end{description}

  Now we find the \nameref{def:Eigenvector}s. \\
  For $\lambda = 6$:
  \begin{align*}
    \begin{pmatrix}
      -4 & 3 & 6 \\
      0 & 0 & 8 \\
      0 & 0 & -4
    \end{pmatrix}
              \begin{pmatrix}
                x_{1} \\ x_{2} \\ x_{3}
              \end{pmatrix} &=
                              \begin{pmatrix}
                                0 \\ 0 \\ 0
                              \end{pmatrix} \\
    \intertext{Turning this into a system of linear equations.}
    -4x_{1} + 3x_{2} + 6x_{3} &= 0 \\
    0x_{1} + 0x_{2} + 8x_{3} &= 0 \\
    0x_{1} + 0x_{2} - 4x_{3} &= 0 \\
    \intertext{Thus, $x_{3} = 0$.}
    -4x_{1} + 3x_{2} + 0 &= 0 \\
    x_{1} &= \frac{3}{4} x_{2}
  \end{align*}

  Now, we can construct our solutions vector.
  \begin{align*}
    \begin{pmatrix}
      x_{1} \\ x_{2} \\ x_{3}
    \end{pmatrix} &=
                    \begin{pmatrix}
                      \frac{3}{4}x_{2} \\ x_{2} \\ 0
                    \end{pmatrix} \\
    &= x_{2}
      \begin{pmatrix}
        \frac{3}{4} \\ 1 \\ 0
      \end{pmatrix} \\
    \intertext{By the definition of an eigenvector, we can scale this general vector by any scalar value and still have an eigenvector.}
    \begin{pmatrix}
      3 \\ 4 \\ 0
    \end{pmatrix}
  \end{align*}

  For $\lambda = 2$:
  \begin{align*}
    \begin{pmatrix}
      0 & 3 & 6 \\
      0 & 4 & 8 \\
      0 & 0 & 0
    \end{pmatrix}
              \begin{pmatrix}
                x_{1} \\ x_{2} \\ x_{3}
              \end{pmatrix} &=
                              \begin{pmatrix}
                                0 \\ 0 \\ 0
                              \end{pmatrix} \\
    \intertext{Turning this into a system of linear equations.}
    0x_{1} + 3x_{2} + 6x_{3} &= 0 \\
    0x_{1} + 4x_{2} + 8x_{3} &= 0 \\
    0x_{1} + 0x_{2} + 0x_{3} &= 0 \\
    \intertext{The first two equations are scalar multiples of the same equation.}
    x_{2} + 2x_{3} &= 0 \\
    x_{2} &= -2x_{3} \\
    \intertext{Lastly, $x_{1}$ must be a free variables.}
    x_{1} &= x_{1}
  \end{align*}

  Now, we can construct our solutions vector.
  \begin{align*}
    \begin{pmatrix}
      x_{1} \\ x_{2} \\ x_{3}
    \end{pmatrix} &=
                    \begin{pmatrix}
                      x_{1} \\ -2x_{3} \\ x_{3}
                    \end{pmatrix} \\
    &= x_{1}
      \begin{pmatrix}
        1 \\ 0 \\ 0
      \end{pmatrix} + x_{3}
    \begin{pmatrix}
      0 \\ -2 \\ 1
    \end{pmatrix}
  \end{align*}

  Now we can construct $P$ out of the \nameref{def:Eigenvector}s we found, and construct $D$ in correlation to $P$.
  \begin{equation*}
    P =
    \begin{pmatrix}
      1 & 0 & 3 \\
      0 & -2 & 4 \\
      0 & 1 & 0
    \end{pmatrix}
  \end{equation*}
  The columns are \nameref{def:Linearly_Independent}, because they are distinct \nameref{def:Eigenvector}s.
  Thus, $P$ must be invertible.
  Let's verify that.
  \begin{align*}
    \det(P) &= 1 c_{1,1} + 0 c_{2,1} + 0 c_{3,1} \\
    &= 1 {(-1)}^{1+1} \det
      \begin{pmatrix}
        -2 & 4 \\
        1 & 0
      \end{pmatrix} \\
            &= 1 \bigl( -2 (0) - 4 (1) \bigr) \\
            &= -4 \\
            &\neq 0
  \end{align*}

  Thus, $P$ is a \nameref{def:Matrix} of \nameref{def:Eigenvector}s which \textbf{is} invertible.
  Therefore $A$ is \nameref{def:Diagonalizable}.
  From the theory $P^{-1}AP=D$, we do not need to perform the matrix multiplication to get $D$.
  \begin{equation*}
    D =
    \begin{pmatrix}
      2 & 0 & 0 \\
      0 & 2 & 0 \\
      0 & 0 & 6
    \end{pmatrix}
  \end{equation*}
\end{example}

\subsubsection{Matrix Exponentials}\label{subsubsec:Matrix_Exponentials}
A reason to use \nameref{def:Diagonalization} is to solve for exponentials of a \nameref{def:Matrix} $A$.
From \Cref{eq:Diagonalizable}, we can see that:
\begin{align*}
  P^{-1} A P &= D \\
  AP &= PD \\
  A &= P D P^{-1}
\end{align*}

This is actually quite a useful tool for solving a variety of matrix problem.
\begin{equation}\label{eq:Diagonalizable_A}
  A = P D P^{-1}
\end{equation}

Say we wanted to compute $A^{n}$.
We could do it the way we have in the past (\Cref{subsubsec:Cayley-Hamilton_Express_Higher_Powers}), but that requires repeated substitution, and therefore repeated computation for earlier elements.
If we want a direct approach to calculate this, we now have all the tools for it.

\begin{blackbox}
  As an example, start with $A^{2}$ and $A^{3}$, where $A$ is any $m \by m$ matrix.
  \begin{align*}
    A^{2} &= A A \\
          &= (P A P^{-1}) (P A P^{-1}) \\
    \intertext{Only the middle $P^{-1}$ and $P$ cancel because matrix multiplication is \textbf{not} commutative.}
          &= P D^{2} P^{-1} \\
    A^{3} &= A A^{2} \\
          &= (P D P^{-1}) (P D^{2} P^{-1}) \\
          &= P D^{3} P^{-1}
  \end{align*}
\end{blackbox}

This is a general algorithm, where we can determine $A^{n}$ using \Cref{eq:Matrix_Exponent_Diagonalized}
\begin{equation}\label{eq:Matrix_Exponent_Diagonalized}
  A^{n} = P D^{n} P^{-1}
\end{equation}

Now, since $D$ is a \nameref{def:Diagonal_Matrix}, $D^{n}$ is very easy to compute.
\begin{align*}
  D_{2 \by 2} &=
                \begin{pmatrix}
                  d_{1} & 0 \\
                  0 & d_{2}
                \end{pmatrix} \\
  D^{2} &=
          \begin{pmatrix}
            d_{1} & 0 \\
            0 & d_{2}
          \end{pmatrix}
                \begin{pmatrix}
                  d_{1} & 0 \\
                  0 & d_{2}
                \end{pmatrix} \\
              &=
                \begin{pmatrix}
                  d_{1}^{2} & 0 \\
                  0 & d_{2}^{2}
                \end{pmatrix} \\
  D^{n} &=
          \begin{pmatrix}
            d_{1}^{n} & 0 \\
            0 & d_{2}^{n}
          \end{pmatrix}
\end{align*}

Therefore, we have a nice equation.
\begin{equation}\label{eq:Matrix_Exponent_Diagonalized-Expanded}
  A^{n} = P
  \begin{pmatrix}
    d_{1}^{n} & 0 & 0 & \cdots \\
    0 & d_{2}^{n} & 0 & \cdots \\
    \vdots & \ddots & \ddots & \vdots
  \end{pmatrix}
  P^{-1}
\end{equation}

\subsubsection{Sinusoids of Matrices}\label{subsubsec:Sinusoids_of_Matrices}
We can use diagonal matrices to find the $\cos$ or $\sin$ of a matrix.
This exploits the use of \Cref{eq:Diagonalizable_A} and Maclaurin expansions.

\begin{blackbox}
  We will compute $\sin(A_{2 \by 2})$ to illustrate the concept.
  However, \textbf{this concept is completely general} to $n \by n$ matrices.

  Start by expanding the $\sin$ according to its Maclaurin expansion.
  \begin{align*}
    \sin(A_{2 \by 2}) &= A - \frac{1}{3!} A^{3} + \frac{1}{5!} A^{5} - \cdots \\
    \sin(P D P^{-1}) &=  P D P^{-1} - \frac{1}{3!} {(P D P^{-1})}^{3} + \frac{1}{5!} {(P D P^{-1})}^{5} - \cdots \\
                      &= P (D - \frac{1}{3!} D^{3} + \frac{1}{5!} D^{5} - \cdots) P^{-1} \\
    \intertext{Expand the $D$ matrices and perform the exponentiation.}
    &= P \left(
      \begin{pmatrix}
        d_{1} & 0 \\
        0 & d_{2}
      \end{pmatrix} - \frac{1}{3!}
            \begin{pmatrix}
        d_{1}^{3} & 0 \\
        0 & d_{2}^{3}
      \end{pmatrix} + \frac{1}{5!}
            \begin{pmatrix}
        d_{1}^{5} & 0 \\
        0 & d_{2}^{5}
      \end{pmatrix} - \cdots \right) P^{-1} \\
    &= P
      \begin{pmatrix}
        d_{1} - \frac{1}{3!}d_{1}^{3} + \frac{1}{5!} d_{1}^{5} - \cdots & 0 \\
        0 & d_{2} - \frac{1}{3!}d_{2}^{3} + \frac{1}{5!} d_{2}^{5} - \cdots \\
      \end{pmatrix} P^{-1} \\
    \intertext{By using the Maclaurin expansion for $\sin$ again, we can simplify this.}
    &= P
      \begin{pmatrix}
        \sin(d_{1}) & 0 \\
        0 & \sin(d_{2}) \\
      \end{pmatrix} P^{-1}
  \end{align*}
\end{blackbox}

\begin{subequations}\label{eq:Matrix_Sinusoids}
  \begin{equation}\label{eq:Matrix_cos}
    \cos(A_{n \by n}) = P
    \begin{pmatrix}
      \cos(a_{1,1}) & 0 & 0 & \cdots \\
      0 & \cos(a_{2,2}) & 0 & \cdots \\
      \vdots & \ddots & \ddots & \vdots
    \end{pmatrix}
    P^{-1}
  \end{equation}
  \begin{equation}\label{eq:Matrix_sin}
    \sin(A_{n \by n}) = P
    \begin{pmatrix}
      \sin(a_{1,1}) & 0 & 0 & \cdots \\
      0 & \sin(a_{2,2}) & 0 & \cdots \\
      \vdots & \ddots & \ddots & \vdots
    \end{pmatrix}
    P^{-1}
  \end{equation}
\end{subequations}

%%% Local Variables:
%%% mode: latex
%%% TeX-master: "../../Math_333-MatrixAlg_ComplexVars-Reference_Sheet"
%%% End:


%%% Local Variables:
%%% mode: latex
%%% TeX-master: "../Math_333-MatrixAlg_ComplexVars-Reference_Sheet"
%%% End:


%====================================APPENDIX====================================
\appendix
\counterwithin{definition}{subsection}

\clearpage
\subsection{Trigonometry} \label{app:Trig}
	\subsubsection{Trigonometric Formulas} \label{subsubsec:Trig Formulas}
		\begin{equation} \label{eq:Sin plus Sin with diff Angles}
			\sin \left( \alpha \right) + \sin \left( \beta \right) = 2 \sin \left( \frac{\alpha + \beta}{2} \right) \cos\left( \frac{\alpha - \beta}{2} \right)  
		\end{equation}
		\begin{equation} \label{eq:Cosine-Sine Product}
			\cos \left( \theta \right) \sin \left( \theta \right) = \frac{1}{2} \sin \left( 2 \theta \right)
		\end{equation}
	
	\subsubsection{Euler Equivalents of Trigonometric Functions} \label{subsubsec:Euler Equivalents}
		\begin{equation} \label{eq:Euler Sin}
			\sin \left( x \right) = \frac{e^{\imath x} + e^{-\imath x}}{2}
		\end{equation}
		\begin{equation} \label{eq:Euler Cos}
			\cos \left( x \right) = \frac{e^{\imath x} - e^{-\imath x}}{2 \imath}
		\end{equation}
		\begin{equation} \label{eq:Euler Sinh}
			\sinh \left( x \right) = \frac{e^{x} - e^{-x}}{2}
		\end{equation}
		\begin{equation} \label{eq:Euler Cosh}
			\cosh \left( x \right) = \frac{e^{x} + e^{-x}}{2}
		\end{equation}

\clearpage
\section{Calculus}\label{app:Calculus}
\subsection{L'Hopital's Rule}\label{subsec:LHopitals_Rule}
L'Hopital's Rule can be used to simplify and solve expressions regarding limits that yield irreconcialable results.
\begin{lemma}[L'Hopital's Rule]\label{lemma:LHopitals_Rule}
  If the equation
  \begin{equation*}
    \lim\limits_{x \rightarrow a} \frac{f(x)}{g(x)} =
    \begin{cases}
      \frac{0}{0} \\
      \frac{\infty}{\infty} \\
    \end{cases}
  \end{equation*}
  then \Cref{eq:LHopitals_Rule} holds.
  \begin{equation}\label{eq:LHopitals_Rule}
    \lim\limits_{x \rightarrow a} \frac{f(x)}{g(x)} = \lim\limits_{x \rightarrow a} \frac{f'(x)}{g'(x)}
  \end{equation}
\end{lemma}

\subsection{Fundamental Theorems of Calculus}\label{subsec:Fundamental Theorem of Calculus}
\begin{definition}[First Fundamental Theorem of Calculus]\label{def:1st Fundamental Theorem of Calculus}
  The \emph{first fundamental theorem of calculus} states that, if $f$ is continuous on the closed interval $\left[ a,b \right]$ and $F$ is the indefinite integral of $f$ on $\left[ a,b \right]$, then

  \begin{equation}\label{eq:1st Fundamental Theorem of Calculus}
    \int_{a}^{b}f \left( x \right) dx = F \left( b \right) - F \left( a \right)
  \end{equation}
\end{definition}

\begin{definition}[Second Fundamental Theorem of Calculus]\label{def:2nd Fundamental Theorem of Calculus}
  The \emph{second fundamental theorem of calculus} holds for $f$ a continuous function on an open interval $I$ and $a$ any point in $I$, and states that if $F$ is defined by

  \begin{equation*}
    F \left( x \right) = \int_{a}^{x} f \left( t \right) dt,
  \end{equation*}
  then
  \begin{equation}\label{eq:2nd Fundamental Theorem of Calculus}
    \begin{aligned}
      \frac{d}{dx} \int_{a}^{x} f \left( t \right) dt &= f \left( x \right) \\
      F' \left( x \right) &= f \left( x \right) \\
    \end{aligned}
  \end{equation}
\end{definition}

\begin{definition}[argmax]\label{def:argmax}
  The arguments to the \emph{argmax} function are to be maximized by using their derivatives.
  You must take the derivative of the function, find critical points, then determine if that critical point is a global maxima.
  This is denoted as
  \begin{equation*}\label{eq:argmax}
    \argmax_{x}
  \end{equation*}
\end{definition}

\subsection{Rules of Calculus}\label{subsec:Rules of Calculus}
\subsubsection{Chain Rule}\label{subsubsec:Chain Rule}
\begin{definition}[Chain Rule]\label{def:Chain Rule}
  The \emph{chain rule} is a way to differentiate a function that has 2 functions multiplied together.

  If
  \begin{equation*}
    f(x) = g(x) \cdot h(x)
  \end{equation*}
  then,
  \begin{equation}\label{eq:Chain Rule}
    \begin{aligned}
      f'(x) &= g'(x) \cdot h(x) + g(x) \cdot h'(x) \\
      \frac{df(x)}{dx} &= \frac{dg(x)}{dx} \cdot g(x) + g(x) \cdot \frac{dh(x)}{dx} \\
    \end{aligned}
  \end{equation}
\end{definition}

\subsection{Useful Integrals}\label{subsec:Useful_Integrals}
\begin{equation}\label{eq:Cosine_Indefinite_Integral}
  \int \cos(x) \; dx = \sin(x)
\end{equation}

\begin{equation}\label{eq:Sine_Indefinite_Integral}
  \int \sin(x) \; dx = -\cos(x)
\end{equation}

\begin{equation}\label{eq:x_Cosine_Indefinite_Integral}
  \int x \cos(x) \; dx = \cos(x) + x \sin(x)
\end{equation}
\Cref{eq:x_Cosine_Indefinite_Integral} simplified with Integration by Parts.

\begin{equation}\label{eq:x_Sine_Indefinite_Integral}
  \int x \sin(x) \; dx = \sin(x) - x \cos(x)
\end{equation}
\Cref{eq:x_Sine_Indefinite_Integral} simplified with Integration by Parts.

\begin{equation}\label{eq:x_Squared_Cosine_Indefinite_Integral}
  \int x^{2} \cos(x) \; dx = 2x \cos(x) + (x^{2} - 2) \sin(x)
\end{equation}
\Cref{eq:x_Squared_Cosine_Indefinite_Integral} simplified by using Integration by Parts twice.

\begin{equation}\label{eq:x_Squared_Sine_Indefinite_Integral}
  \int x^{2} \sin(x) \; dx = 2x \sin(x) - (x^{2} - 2) \cos(x)
\end{equation}
\Cref{eq:x_Squared_Sine_Indefinite_Integral} simplified by using Integration by Parts twice.

\begin{equation}\label{eq:Exponential_Cosine_Indefinite_Integral}
  \int e^{\alpha x} \cos(\beta x) \; dx = \frac{e^{\alpha x} \bigl( \alpha \cos(\beta x) + \beta \sin(\beta x) \bigr)}{\alpha^{2} + \beta^{2}} + C
\end{equation}

\begin{equation}\label{eq:Exponential_Sine_Indefinite_Integral}
  \int e^{\alpha x} \sin(\beta x) \; dx = \frac{e^{\alpha x} \bigl( \alpha \sin(\beta x) - \beta \cos(\beta x) \bigr)}{\alpha^{2}+\beta^{2}} + C
\end{equation}

\begin{equation}\label{eq:Exponential_Indefinite_Integral}
  \int e^{\alpha x} \; dx = \frac{e^{\alpha x}}{\alpha}
\end{equation}

\begin{equation}\label{eq:x_Exponential_Indefinite_Integral}
  \int x e^{\alpha x} \; dx = e^{\alpha x} \left( \frac{x}{\alpha} - \frac{1}{\alpha^{2}} \right)
\end{equation}
\Cref{eq:x_Exponential_Indefinite_Integral} simplified with Integration by Parts.

\begin{equation}\label{eq:Inverse_x_Indefinite_Integral}
  \int \frac{dx}{\alpha + \beta x} = \int \frac{1}{\alpha + \beta x} \; dx = \frac{1}{\beta} \ln (\alpha + \beta x)
\end{equation}

\begin{equation}\label{eq:Inverse_x_Squared_Indefinite_Integral}
  \int \frac{dx}{\alpha^{2} + \beta^{2} x^{2}} = \int \frac{1}{\alpha^{2} + \beta^{2} x^{2}} \; dx = \frac{1}{\alpha \beta} \arctan \left( \frac{\beta x}{\alpha} \right)
\end{equation}

\begin{equation}\label{eq:a_Exponential_Indefinite_Integral}
  \int \alpha^{x} \; dx = \frac{\alpha^{x}}{\ln(\alpha)}
\end{equation}

\begin{equation}\label{eq:a_Exponential_Derivative}
  \frac{d}{dx} \alpha^{x} = \frac{d\alpha^{x}}{dx} = \alpha^{x} \ln(x)
\end{equation}

\subsection{Leibnitz's Rule}\label{subsec:Leibnitzs_Rule}
\begin{lemma}[Leibnitz's Rule]\label{lemma:Leibnitzs_Rule}
  Given
  \begin{equation*}
    g(t) = \int_{a(t)}^{b(t)} f(x, t) \, dx
  \end{equation*}
  with $a(t)$ and $b(t)$ differentiable in $t$ and $\frac{\partial f(x, t)}{\partial t}$ continuous in both $t$ and $x$, then
  \begin{equation}\label{eq:Leibnitzs_Rule}
    \frac{d}{dt} g(t) = \frac{d g(t)}{dt} = \int_{a(t)}^{b(t)} \frac{\partial f(x, t)}{\partial t} \, dx + f \bigl[ b(t), t \bigr] \, \frac{d b(t)}{dt} - f \bigl[ a(t), t \bigr] \, \frac{d a(t)}{dt}
  \end{equation}
\end{lemma}



\clearpage
\section{Laplace Transform}\label{app:Laplace_Transform}
\subsection{Laplace Transform}\label{subsec:Laplace_Transform}
\begin{definition}[Laplace Transform]\label{def:Laplace_Transform}
  The \emph{Laplace transformation} operation is denoted as $\Lapl \lbrace x(t) \rbrace$ and is defined as
  \begin{equation}\label{eq:Laplace_Transform}
    X(s) = \int\limits_{-\infty}^{\infty} x(t) e^{-st} dt
  \end{equation}
\end{definition}

\subsection{Inverse Laplace Transform}\label{subsec:Inverse_Laplace_Transform}
\begin{definition}[Inverse Laplace Transform]\label{def:Inverse_Laplace_Transform}
  The \emph{inverse Laplace transformation} operation is denoted as $\Lapl^{-1} \lbrace X(s) \rbrace$ and is defined as
  \begin{equation}\label{eq:Inverse_Laplace_Transform}
    x(t) = \frac{1}{2j \pi} \int_{\sigma-\infty}^{\sigma+\infty} X(s) e^{st} \, ds
  \end{equation}
\end{definition}

\subsection{Properties of the Laplace Transform}\label{subsec:Laplace_Transform_Properties}
\subsubsection{Linearity}\label{subsubsec:Laplace_Linearity}
The \nameref{def:Laplace_Transform} is a linear operation, meaning it obeys the laws of linearity.
This means \Cref{eq:Laplace_Linearity} must hold.
\begin{subequations}\label{eq:Laplace_Linearity}
  \begin{equation}\label{eq:Laplace_Linearity_Time}
    x(t) = \alpha_{1} x_{1}(t) + \alpha_{2} x_{2}(t)
  \end{equation}
  \begin{equation}\label{eq:Laplace_Linearity_Frequency}
    X(s) = \alpha_{1} X_{1}(s) + \alpha_{2} X_{2}(s)
  \end{equation}
\end{subequations}

\subsubsection{Time Scaling}\label{subsubsec:Laplace_Time_Scaling}
Scaling in the time domain (expanding or contracting) yields a slightly different transform.
However, this only makes sense for $\alpha > 0$ in this case.
This is seen in \Cref{eq:Laplace_Time_Scaling}.
\begin{equation}\label{eq:Laplace_Time_Scaling}
  \Lapl \bigl\lbrace x(\alpha t) \bigr\rbrace = \frac{1}{\alpha} X \left( \frac{s}{\alpha} \right)
\end{equation}

\subsubsection{Time Shift}\label{subsubsec:Laplace_Time_Shift}
Shifting in the time domain means to change the point at which we consider $t=0$.
\Cref{eq:Laplace_Time_Shifting} below holds for shifting both forward in time and backward.
\begin{equation}\label{eq:Laplace_Time_Shifting}
  \Lapl \bigl\lbrace x(t-a) \bigr\rbrace = X(s) e^{-a s}
\end{equation}

\subsubsection{Frequency Shift}\label{subsubsec:Laplace_Frequency_Shift}
Shifting in the frequency domain means to change the complex exponential in the time domain.
\begin{equation}\label{eq:Laplace_Frequency_Shift}
  \Lapl^{-1} \bigl\lbrace X(s-a) \bigr\rbrace = x(t)e^{at}
\end{equation}

\subsubsection{Integration in Time}\label{subsubsec:Laplace_Time_Integration}
Integrating in time is equivalent to scaling in the frequency domain.
\begin{equation}\label{eq:Laplace_Time_Integration}
  \Lapl \left\lbrace \int_{0}^{t} x(\lambda) \, d\lambda \right\rbrace = \frac{1}{s} X(s)
\end{equation}

\subsubsection{Frequency Multiplication}\label{subsubsec:Laplace_Frequency_Multiplication}
Multiplication of two signals in the frequency domain is equivalent to a convolution of the signals in the time domain.
\begin{equation}\label{eq:Laplace_Frequency_Multiplication}
  \Lapl \bigl\lbrace x(t) * v(t) \bigr\rbrace = X(s) V(s)
\end{equation}

\subsubsection{Relation to Fourier Transform}\label{subsubsec:Fourier_Transform_Relation}
The Fourier transform looks and behaves very similarly to the Laplace transform.
In fact, if $X(\omega)$ exists, then \Cref{eq:Fourier_Laplace_Transform_Relation} holds.
\begin{equation}\label{eq:Fourier_Laplace_Transform_Relation}
  X(s) = X(\omega) \vert_{\omega = \frac{s}{j}}
\end{equation}

\subsection{Theorems}\label{subsec:Laplace_Theorems}
There are 2 theorems that are most useful here:
\begin{enumerate}[noitemsep]
\item \nameref{thm:Laplace_Initial_Value_Theorem}
\item \nameref{thm:Laplace_Final_Value_Theorem}
\end{enumerate}


%%% Local Variables:
%%% mode: latex
%%% TeX-master: shared
%%% End:


% To make this print, you must include a citation somewhere in the document
\clearpage
\printbibliography{}
\end{document}

%%% Local Variables:
%%% mode: latex
%%% TeX-master: t
%%% End:
