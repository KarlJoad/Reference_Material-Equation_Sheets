\section{Threat Analysis and Security Requirements}\label{sec:Threat_Analysis-Security_Requirements}
\begin{questions}
\question{} List three different typical security threats to an IT system?
  \begin{solution}
    \begin{enumerate}[noitemsep]
    \item Network-based threats
    \item Physical threats
    \item Software vulnerabilities
    \end{enumerate}
  \end{solution}

\question{} What is the first step in an attack tree threat analysis process?
  \begin{solution}
    There is a ``Step Zero'' that stipulates you have a \emph{good} system description.

    The first real step is to identify potential goals that the attacker could have when attcking your system as it currently exists.
  \end{solution}

\question{} Make an attack tree based analysis of a BankID system.
  \begin{solution}
    There is no single solution for this, everyone's will be different.

    For the Non-Swedes reading this, BankID is a way of verifying banking instructions and transactions through your phone as a second factor of authentication.
    To use it, you initiate some banking operation, which must be signed.
    Then, your phone will open the BankID app and ask you to input a secure 6-digit (minimum) code to ``Digitally Sign'' the transaction.
    Once done, the app closes and the trasaction is recorded and completed at the appropriate time.
  \end{solution}

\question{} List three well established threat assessment methodologies
  \begin{solution}
    \begin{enumerate}[noitemsep]
    \item Shneier Attack Trees
    \item Microsoft's STRIDE Analysis
    \item MITRE's TARA Analysis
    \end{enumerate}
  \end{solution}

\question{} Spell out the acronym STRIDE
  \begin{solution}
    \begin{description}[noitemsep]
    \item[S] Spoofing
    \item[T] Tampering
    \item[R] Repudiation
    \item[I] Information Disclosure
    \item[D] Denial of Service
    \item[E] Elevation of Privilege
    \end{description}
  \end{solution}

  \begin{parts}
  \part{} Explain the meaning of the six different concepts in STRIDE
    \begin{solution}
      \begin{description}[noitemsep]
      \item[Spoofing] Pretending to be someone or something you are not.
        Getting the correct service to communicate with someone it thinks is also the correct thing, when it really isn't.
      \item[Tampering] Modifying something (file, config, etc.) somewhere (disk, network, memory, etc.).
      \item[Repudiation] Claiming you did/didn't do something when you didn't/did.
        Essentially, the system claims you did something that you didn't do (can be honest or dishonest).
        The real question to answer here is what evidence do we have to trace this?
      \item[Information Disclosure] Providing information to someone \textbf{NOT} authorized to view it.
      \item[Denial of Service] Absorbing resources required for regular system function for malicious reasons.
      \item[Elevation of Privilege] Allowing someone to do something they should \textbf{NOT} be allowed to do.
      \end{description}
    \end{solution}

  \part{} Give examples of attacks for the six different concepts in STRIDE
    \begin{solution}
      \begin{description}[noitemsep]
      \item[Spoofing] Man-in-the-Middle.
        Impersonating the intended recipient.
      \item[Tampering] Changing an Excel Spreadsheet.
        Changing a configuration file somewhere to change the functions of the system.
      \item[Repudiation] Writing a file, then deleting the kernel log about the file requests.
      \item[Information Disclosure] Forwarding an email to someone that should not be able to see the email.
      \item[Denial of Service] Using a botnet to deny network service to some online provider.
      \item[Elevation of Privilege] You are supposed to be a regular user, but you need to elevate to administrator to run your usual programs, and IT allows people to do so.
      \end{description}
    \end{solution}
  \end{parts}

\question{} Which are the three basic steps in STRIDE?\@
  \begin{solution}
    \begin{enumerate}[noitemsep]
    \item Identify the main entities/actors in the system.
      \begin{itemize}[noitemsep]
      \item These are the people and computers that interact with the system.
      \end{itemize}
    \item Identify the main entities' interactions.
      \begin{itemize}[noitemsep]
      \item How do the people interact with the computers in the system?
      \item How do the computers interact with other computers in the system?
      \end{itemize}
    \item For each entity, perform a STRIDE analysis on each of the following items:
      \begin{description}[noitemsep]
      \item[Process] A Process is a program or set of programs that achieve or do something.
      \item[External Entity] An External Entity is one that is not \textbf{EXPLICITLY} part of \textbf{YOUR} system design.
      \item[Data Flow] A Data Flow is how information can be moved through the system.
      \item[Data Store] A Data Store is how information can be stored throughout the system.
      \end{description}
    \end{enumerate}
  \end{solution}

\question{} Which are the two main activities in a MITRE TARA security analysis
  \begin{solution}
    There is a zeroth step, the Crown Jewel Analysis for the identification of things in the system that are important to protect.
    \begin{enumerate}[noitemsep]
    \item Cyber Threat Susceptibility Analysis (CTSA)
    \item Cyber Risk Remediation Analysis (CRRA)
    \end{enumerate}
  \end{solution}

  \begin{parts}
  \part{} Which are the main input sources to these two analysis
    activities?
  \part{} The output of these two activities are stored in special databases. What is the name of these two databases?
  \end{parts}

\question{} Describe briefly the different steps performed during a TARA CTSA
\question{} Spell out the acronyms CAPEC, CWE and CVE
  \begin{parts}
  \part{} What does CAPEC contain and how it is used in a TARA analysis?
  \part{} What does CWE contain and how it is used in a TARA analysis?
  \part{} What does CVE contain and how it is used in a TARA analysis?
  \end{parts}

\question{} Describe briefly the different steps performed during a TARA CRRA
\question{} Where can one find TTP mitigation solutions?
\question{} Which are the four different mitigation types?
\question{} Which are the steps used to obtain a final ranking table for countermeasures?
\question{} How do one select final TARA recommendations based on a countermeasure ranking table?
\question{} Which are the three mandatory parts of a TARA TTP recommendation?
\question{} Which are the different input sources to the security requirements derivation process?
\question{} Which are the main outputs from the attack tree, the STRIDE and the TARA process respectively which are used to derive high-level security requirements?
\question{} Give example of high level security requirements for a Bank ID system
\question{} Give example of low level security requirements  for a Bank ID system
\question{} Describe a four step approach for security requirements identification and documentation
\end{questions}

%%% Local Variables:
%%% mode: latex
%%% TeX-master: "../EITP20-Secure_Systems_Engineering-Study_Questions"
%%% End:
