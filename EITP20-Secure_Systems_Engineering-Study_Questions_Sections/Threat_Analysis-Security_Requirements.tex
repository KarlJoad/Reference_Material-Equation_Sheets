\section{Threat Analysis and Security Requirements}\label{sec:Threat_Analysis-Security_Requirements}
\begin{itemize}
\item List three different typical security threats to an IT system?
\item What is the first step in an attack tree threat analysis process?
\item Make an attack tree based analysis of a BankID system
\item List three well established threat assessment methodologies
\item Spell out the acronym STRIDE
  \begin{itemize}[noitemsep]
  \item Explain the meaning of the six different concepts in
    STRIDE
  \item Give examples of attacks for the six different concepts in STRIDE
  \end{itemize}

\item Which are the three basic steps in STRIDE?
\item Which are the two main activities in a MITRE TARA security analysis
  \begin{itemize}[noitemsep]
  \item Which are the main input sources to these two analysis
    activities?
  \item The output of these two activities are stored in special databases. What is the name of these two databases?
  \end{itemize}

\item Describe briefly the different steps performed during a TARA CTSA
\item Spell out the acronyms CAPEC, CWE and CVE
  \begin{itemize}[noitemsep]
  \item What does CAPEC contain and how it is used in a TARA
    analysis?
  \item What does CWE contain and how it is used in a TARA analysis?
  \item What does CVE contain and how it is used in a TARA analysis?
  \end{itemize}

\item Describe briefly the different steps performed during a TARA CRRA
\item Where can one find TTP mitigation solutions?
\item Which are the four different mitigation types?
\item Which are the steps used to obtain a final ranking table for countermeasures?
\item How do one select final TARA recommendations based on a countermeasure ranking table?
\item Which are the three mandatory parts of a TARA TTP recommendation?
\item Which are the different input sources to the security requirements derivation process?
\item Which are the main outputs from the attack tree, the STRIDE and the TARA process respectively which are used to derive high-level security requirements?
\item Give example of high level security requirements for a Bank ID system
\item Give example of low level security requirements  for a Bank ID system
\item Describe a four step approach for security requirements identification and documentation
\end{itemize}

%%% Local Variables:
%%% mode: latex
%%% TeX-master: "../EITP20-Secure_Systems_Engineering-Study_Questions"
%%% End:
