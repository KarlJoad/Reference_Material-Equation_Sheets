\section{Threat Analysis and Security Requirements}\label{sec:Threat_Analysis-Security_Requirements}
\begin{questions}
\question{} List three different typical security threats to an IT system?
  \begin{solution}
    \begin{enumerate}[noitemsep]
    \item Network-based threats
    \item Physical threats
    \item Software vulnerabilities
    \end{enumerate}
  \end{solution}

\question{} What is the first step in an attack tree threat analysis process?
  \begin{solution}
    There is a ``Step Zero'' that stipulates you have a \emph{good} system description.

    The first real step is to identify potential goals that the attacker could have when attcking your system as it currently exists.
  \end{solution}

\question{} Make an attack tree based analysis of a BankID system.
  \begin{solution}
    There is no single solution for this, everyone's will be different.

    For the Non-Swedes reading this, BankID is a way of verifying banking instructions and transactions through your phone as a second factor of authentication.
    To use it, you initiate some banking operation, which must be signed.
    Then, your phone will open the BankID app and ask you to input a secure 6-digit (minimum) code to ``Digitally Sign'' the transaction.
    Once done, the app closes and the trasaction is recorded and completed at the appropriate time.
  \end{solution}

\question{} List three well established threat assessment methodologies
  \begin{solution}
    \begin{enumerate}[noitemsep]
    \item Shneier Attack Trees
    \item Microsoft's STRIDE Analysis
    \item MITRE's TARA Analysis
    \end{enumerate}
  \end{solution}

\question{} Spell out the acronym STRIDE
  \begin{parts}
  \part{} Explain the meaning of the six different concepts in STRIDE
  \part{} Give examples of attacks for the six different concepts in STRIDE
  \end{parts}

\question{} Which are the three basic steps in STRIDE?\@
\question{} Which are the two main activities in a MITRE TARA security analysis
  \begin{parts}
  \part{} Which are the main input sources to these two analysis
    activities?
  \part{} The output of these two activities are stored in special databases. What is the name of these two databases?
  \end{parts}

\question{} Describe briefly the different steps performed during a TARA CTSA
\question{} Spell out the acronyms CAPEC, CWE and CVE
  \begin{parts}
  \part{} What does CAPEC contain and how it is used in a TARA analysis?
  \part{} What does CWE contain and how it is used in a TARA analysis?
  \part{} What does CVE contain and how it is used in a TARA analysis?
  \end{parts}

\question{} Describe briefly the different steps performed during a TARA CRRA
\question{} Where can one find TTP mitigation solutions?
\question{} Which are the four different mitigation types?
\question{} Which are the steps used to obtain a final ranking table for countermeasures?
\question{} How do one select final TARA recommendations based on a countermeasure ranking table?
\question{} Which are the three mandatory parts of a TARA TTP recommendation?
\question{} Which are the different input sources to the security requirements derivation process?
\question{} Which are the main outputs from the attack tree, the STRIDE and the TARA process respectively which are used to derive high-level security requirements?
\question{} Give example of high level security requirements for a Bank ID system
\question{} Give example of low level security requirements  for a Bank ID system
\question{} Describe a four step approach for security requirements identification and documentation
\end{questions}

%%% Local Variables:
%%% mode: latex
%%% TeX-master: "../EITP20-Secure_Systems_Engineering-Study_Questions"
%%% End:
