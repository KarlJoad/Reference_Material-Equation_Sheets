\section{Threat Analysis and Security Requirements}\label{sec:Threat_Analysis-Security_Requirements}
\begin{questions}
\question{} List three different typical security threats to an IT system?
  \begin{solution}
    \begin{enumerate}[noitemsep]
    \item Network-based threats
    \item Physical threats
    \item Software vulnerabilities
    \end{enumerate}
  \end{solution}

\question{} What is the first step in an attack tree threat analysis process?
  \begin{solution}
    There is a ``Step Zero'' that stipulates you have a \emph{good} system description.

    The first real step is to identify potential goals that the attacker could have when attcking your system as it currently exists.
  \end{solution}

\question{} Make an attack tree based analysis of a BankID system.
  \begin{solution}
    There is no single solution for this, everyone's will be different.

    For the Non-Swedes reading this, BankID is a way of verifying banking instructions and transactions through your phone as a second factor of authentication.
    To use it, you initiate some banking operation, which must be signed.
    Then, your phone will open the BankID app and ask you to input a secure 6-digit (minimum) code to ``Digitally Sign'' the transaction.
    Once done, the app closes and the trasaction is recorded and completed at the appropriate time.
  \end{solution}

\question{} List three well established threat assessment methodologies
  \begin{solution}
    \begin{enumerate}[noitemsep]
    \item Shneier Attack Trees
    \item Microsoft's STRIDE Analysis
    \item MITRE's TARA Analysis
    \end{enumerate}
  \end{solution}

\question{} Spell out the acronym STRIDE
  \begin{solution}
    \begin{description}[noitemsep]
    \item[S] Spoofing
    \item[T] Tampering
    \item[R] Repudiation
    \item[I] Information Disclosure
    \item[D] Denial of Service
    \item[E] Elevation of Privilege
    \end{description}
  \end{solution}

  \begin{parts}
  \part{} Explain the meaning of the six different concepts in STRIDE
    \begin{solution}
      \begin{description}[noitemsep]
      \item[Spoofing] Pretending to be someone or something you are not.
        Getting the correct service to communicate with someone it thinks is also the correct thing, when it really isn't.
      \item[Tampering] Modifying something (file, config, etc.) somewhere (disk, network, memory, etc.).
      \item[Repudiation] Claiming you did/didn't do something when you didn't/did.
        Essentially, the system claims you did something that you didn't do (can be honest or dishonest).
        The real question to answer here is what evidence do we have to trace this?
      \item[Information Disclosure] Providing information to someone \textbf{NOT} authorized to view it.
      \item[Denial of Service] Absorbing resources required for regular system function for malicious reasons.
      \item[Elevation of Privilege] Allowing someone to do something they should \textbf{NOT} be allowed to do.
      \end{description}
    \end{solution}

  \part{} Give examples of attacks for the six different concepts in STRIDE
    \begin{solution}
      \begin{description}[noitemsep]
      \item[Spoofing] Man-in-the-Middle.
        Impersonating the intended recipient.
      \item[Tampering] Changing an Excel Spreadsheet.
        Changing a configuration file somewhere to change the functions of the system.
      \item[Repudiation] Writing a file, then deleting the kernel log about the file requests.
      \item[Information Disclosure] Forwarding an email to someone that should not be able to see the email.
      \item[Denial of Service] Using a botnet to deny network service to some online provider.
      \item[Elevation of Privilege] You are supposed to be a regular user, but you need to elevate to administrator to run your usual programs, and IT allows people to do so.
      \end{description}
    \end{solution}
  \end{parts}

\question{} Which are the three basic steps in STRIDE?\@
  \begin{solution}
    \begin{enumerate}[noitemsep]
    \item Identify the main entities/actors in the system.
      \begin{itemize}[noitemsep]
      \item These are the people and computers that interact with the system.
      \end{itemize}
    \item Identify the main entities' interactions.
      \begin{itemize}[noitemsep]
      \item How do the people interact with the computers in the system?
      \item How do the computers interact with other computers in the system?
      \end{itemize}
    \item For each entity, perform a STRIDE analysis on each of the following items:
      \begin{description}[noitemsep]
      \item[Process] A Process is a program or set of programs that achieve or do something.
      \item[External Entity] An External Entity is one that is not \textbf{EXPLICITLY} part of \textbf{YOUR} system design.
      \item[Data Flow] A Data Flow is how information can be moved through the system.
      \item[Data Store] A Data Store is how information can be stored throughout the system.
      \end{description}
    \end{enumerate}
  \end{solution}

\question{} Which are the two main activities in a MITRE TARA security analysis
  \begin{solution}
    There is a zeroth step, the Crown Jewel Analysis for the identification of things in the system that are important to protect.
    \begin{enumerate}[noitemsep]
    \item Cyber Threat Susceptibility Analysis (CTSA)
    \item Cyber Risk Remediation Analysis (CRRA)
    \end{enumerate}
  \end{solution}

  \begin{parts}
  \part{} Which are the main input sources to these two analysis activities?
    \begin{solution}
      The question asks for 2, but I will give 3.
      \begin{enumerate}[noitemsep]
      \item Common Attack Pattern Enumeration and Classification (CAPEC)
      \item Common Weakness Enumeration (CWE)
      \item Common Vulnerabilities and Exposures (CVE)
      \end{enumerate}
    \end{solution}

  \part{} The output of these two activities are stored in special databases. What is the name of these two databases?
    \begin{solution}
      \begin{enumerate}[noitemsep]
      \item Various vulnerability and exploit databases that detail various attacks and how they can/may be executed.
        These are known as \textbf{Attack TTP Catalogs}.
      \item Countermeasure databases that explain how to mitigate the vulnerabilities and exploits that are found in the previous databases.
        These are known as \textbf{Countermeasure Catalogs}.
      \end{enumerate}
    \end{solution}
  \end{parts}

\question{} Describe briefly the different steps performed during a TARA CTSA.\@
  \begin{solution}
    \begin{enumerate}[noitemsep]
    \item Establish the Assessment Scope.
      \begin{itemize}[noitemsep]
      \item Typically this is partly done during the Crown Jewel Analysis.
      \item This is used to make sure you don't evaluate parts of the existing or new system that you are not concerned aobut.
      \end{itemize}
    \item Identify the candidate Tactics, Techniques, and Procedures (TTP).
      \begin{itemize}[noitemsep]
      \item Here, you collect \emph{\textbf{ALL}} potential TTPs that could affect your properly scoped system.
      \end{itemize}
    \item Eliminate Implausible TTPs.
      \begin{itemize}[noitemsep]
      \item If you perform Step 2 correctly, you will have WAY too many TTPs to mitigate for you system to ever work.
      \item You must use your best judgment to remove the ones that are implausible to occur in your system.
      \end{itemize}
    \item Apply some scoring model.
      \begin{itemize}[noitemsep]
      \item MITRE has a scoring model provided that can be quite useful.
      \item These will be the scores that you use in the Threat Matrix to find the best TTPs to design against.
      \end{itemize}
    \item Construct a Threat Matrix.
      \begin{itemize}[noitemsep]
      \item The scores used in the Threat Matrix are based off the scoring model that is created in the previous step.
      \item This will show you the which TTPs that remain are the most important to design against.
      \end{itemize}
    \end{enumerate}
  \end{solution}

\question{} Spell out the acronyms CAPEC, CWE and CVE.\@
  \begin{solution}
    \begin{description}[noitemsep]
    \item[C] Common
    \item[A] Attack
    \item[P] Pattern
    \item[E] Enumeration
    \item[C] Classification
    \end{description}

    \begin{description}[noitemsep]
    \item[C] Common
    \item[W] Weakness
    \item[E] Enumeration
    \end{description}

    \begin{description}[noitemsep]
    \item[C] Common
    \item[V] Vulnerabilities
    \item[E] Exposures
    \end{description}
  \end{solution}

  \begin{parts}
  \part{} What does CAPEC contain and how it is used in a TARA analysis?
    \begin{solution}
      CAPEC contains general attack patterns and threats (TTPs) to system that are broken down into a family hierarchy.
      This also contains the potential threat of a TTP and how difficult it is to mitigate.
      It is used to find TTPs, their potential score, and to help construct the Threat Matrix.
    \end{solution}

  \part{} What does CWE contain and how it is used in a TARA analysis?
    \begin{solution}
      CWE describes software weaknesses.
      It is best used when finishing the choice of software to implement.
    \end{solution}

  \part{} What does CVE contain and how it is used in a TARA analysis?
    \begin{solution}
      CVE contains known software vulnerabilities.
      It is best used when the software implementation decisions have been made.
    \end{solution}
  \end{parts}

\question{} Describe briefly the different steps performed during a TARA CRRA.\@
  \begin{solution}
    There is the obvious requirement that the TARA CTSA step has been performed.
    \begin{enumerate}[noitemsep]
    \item Select which TTPs to mitigate.
      \begin{itemize}[noitemsep]
      \item You may not be able to mitigate \textbf{ALL} TTPs.
      \item For example, you can attempt to mitigate Social Engineering attacks, but these are \emph{\textbf{NOTORIOUSLY, INCREDIBLY}} difficult to protect against.
      \end{itemize}
    \item Identify plausible countermeasures.
      \begin{itemize}[noitemsep]
      \item There may be several possible countermeasures for your particular TTP.\@
      \item You must use your best judgment to ensure you counter as much of the TTP's threat as possible.
      \end{itemize}
    \item Assess countermeasure merits.
      \begin{itemize}[noitemsep]
      \item You can perform a Cost-Benefit Analysis here to mathematically determine the best countermeasures.
      \item You will get a \emph{Utility} score that estimates how beneficial your countermeasure is and how effectively it counters the threat the TTP provides.
      \item You will also get a \emph{Cost} score that estimates how costly your countermeasure is to implement and let run.
      \item To find the ``best bang for your buck'' solution, you must take $\emph{Utility}/\emph{Cost}$.
      \end{itemize}
    \item Identify ``Optimal'' Countermeasure Solution
      \begin{itemize}[noitemsep]
      \item This is what you think would be the best set of countermeasures to implement to secure the system.
      \end{itemize}
    \item Prepare recommendations.
      \begin{itemize}[noitemsep]
      \item These will be used in conjunction with original project description to build your Security Requirements.
      \end{itemize}
    \end{enumerate}
  \end{solution}

\question{} Where can one find TTP mitigation solutions?
  \begin{solution}
    There are a large variety of databases available, but we used the CAPEC database throughout the course to find countermeasures.
  \end{solution}

\question{} Which are the four different mitigation types?
  \begin{solution}
    \begin{enumerate}[noitemsep]
    \item Detect: \textbf{FIND} the problem that is occurring.
    \item Limit: \textbf{LIMIT} the potential damage a harmful action can do.
    \item Neutralize: The harmful action has already begun, but not yet finished.
      How do you \textbf{DEAL} with it?
    \item Recover: The harmful action may have finished, or you may have interrupted it.
      How do you \textbf{COME BACK} from this harmful action, to make it seem like it never happened?
    \end{enumerate}
  \end{solution}

\question{} Which are the steps used to obtain a final ranking table for countermeasures?
  \begin{solution}
    The main step is to perform a Cost-Benefit analysis.
    Here you identify and create a score of benefits and potential costs each countermeasure has, and find countermeasures that maximize the benefits, while minimizing the costs.
  \end{solution}

\question{} How do one select final TARA recommendations based on a countermeasure ranking table?
  \begin{solution}
    Typically, you would select TARA countermeasures that have as high a $U/C$ ratio as possible, so they maximize utility, while minimizing cost.
    Although, these can also be somewhat arbitrarily chosen if there are specific items that you must counter.
  \end{solution}

\question{} Which are the three mandatory parts of a TARA TTP recommendation?
  \begin{solution}
    \begin{enumerate}[noitemsep]
    \item The Action/Device/Procedure/Technique that is recommended.
      Which countermeasure(s) should be applied?
    \item The reason why that particular countermeasure is required.
      Which TTP(s) does this countermeasure mitigate?
    \item The implication/effect if the countermeasure is \emph{not} applied.
      What is the potential impact(s) to the system's/mission's capabilities resulting from the compromise of a cyber asset?
    \end{enumerate}
  \end{solution}

\question{} Which are the different input sources to the security requirements derivation process?
  \begin{solution}
    \begin{enumerate}[noitemsep]
    \item Client
      \begin{itemize}[noitemsep]
      \item Use-case
      \item Needs
      \item Costs
      \item Performance
      \item Other Business considerations
      \end{itemize}
    \item Engineer (Us)
      \begin{itemize}[noitemsep]
      \item Threats
      \item Analysis
      \item Breakdowns
      \item Discussions with client
      \end{itemize}
    \end{enumerate}
  \end{solution}

\question{} Which are the main outputs from the attack tree, the STRIDE and the TARA process respectively which are used to derive high-level security requirements?
  \begin{solution}
    \begin{itemize}[noitemsep]
    \item Attack Tree:
      \begin{itemize}[noitemsep]
      \item \textbf{Methods of preventing} the potential attacks identified, through the trees, that can be used to compromise the system.
      \end{itemize}
    \item STRIDE:\@
      \begin{itemize}[noitemsep]
      \item Key system elements.
      \item Potential methods of compromising each letter of STRIDE.\@
        Each of the letters gets its own table that gives potential attacks and mitigations that must be provided.
      \end{itemize}
    \item TARA:\@
      \begin{itemize}[noitemsep]
      \item TTPs from CTSA
      \item The CTSA Threat Matrix
      \item Countermeasures from CRRA
      \item The CRRA $U/C$ matrix
      \end{itemize}
    \end{itemize}
  \end{solution}

\question{} Give example of high level security requirements for a BankID system.
  \begin{solution}
    The BankID discussed here is the same as the one discussed above.

    There is \textbf{no} general solution for this.
    Everyone's will be different.
  \end{solution}

\question{} Give example of low level security requirements  for a BankID system.
  \begin{solution}
    The BankID discussed here is the same as the one discussed above.

    There is \textbf{no} general solution for this.
    Everyone's will be different.
  \end{solution}

\end{questions}

%%% Local Variables:
%%% mode: latex
%%% TeX-master: "../EITP20-Secure_Systems_Engineering-Study_Questions"
%%% End:
