\section{Security Evaluation}\label{sec:Security_Evaluation}
\begin{questions}
\question{} What is the purpose with a security review and when should it be performed?
\question{} At which four main levels do you typically perform a security evaluation?
  \begin{parts}
  \part{} At which occasion should they be done?
  \end{parts}

\question{} Mention three different aspects to consider at an architecture ``sanity check'' review
  \begin{parts}
  \part{} For each aspect, list what should be considered?
  \end{parts}

\question{} Mention three different aspects to consider at an architecture business review
  \begin{parts}
  \part{} For each aspect, list what should be considered?
  \end{parts}

\question{} What do you perform during a security requirements review?
\question{} Mention four different aspects to consider at a design review
  \begin{parts}
  \part{} For each aspect, list what should be considered?
  \end{parts}

\question{} Describe a process for issuing and security testing of a software product
  \begin{parts}
  \part{} What is the role in the process for the security officer, the security architect, the security master and security penetration tester respectively?
  \end{parts}

\question{} Give examples of things to consider during a performance review of a design?

\question{} What is the purpose of trying to ``measure'' the security of a design?
\question{} A simple security measurement takes three basic security characteristics into account:
  \begin{parts}
  \part{} Which three characteristics are then measured and how to you combine the measurements to get an overall measurement of a system?
  \end{parts}

\question{} Consider the smart card security system in \Cref{fig:Smart_Card_Security_System}.
  Assume the side-channel and physical channel break are equal important and the overall smart card security is the minmum strength of the two nodes.
  Calculate the smart card security score using the weighted weakest link approach.
  \begin{figure}[h!]
    \centering
    \includegraphics[scale=0.35]{./Drawings/EITP20-Secure_Systems_Engineering/Smart_Card_Security_System.png}
    \caption{Smart Card Security System}
    \label{fig:Smart_Card_Security_System}
  \end{figure}


\question{} How can the sensitivity for a certain security component be calculated?
\question{} What is a CVE database? Which organizations maintain global CVEs?
\question{} Which are the tree basic categories for which CVE scoring is based
  \begin{parts}
  \part{} Briefly explain each of the three categories and which security aspects are considered for each of them?
  \end{parts}

\question{} A communication product have three different categories of weaknesses, buffer overflow, TPM weaknesses, and authentication weakness with the CVE list below. Calculate an overall vulnerability score for the product (use the NIST CVE database to obtain the individual scores).
  \begin{parts}
  \part{} Buffer overflow: CVE--2019--2304, CVE--2019--2242, CVE--2019--10572
  \part{} TPM:\@ CVE--2019--16863, CVE--2018--6622
  \part{} Authentication: CVE--2019--3768, CVE--2019--5108,CVE--2019--17627, CVE--2018--5389
  \end{parts}

\question{} Explain the terms TOI, PP, ST and EAL used in CC evaluations.
\question{} What is the purpose with the PPs?
\question{} What is the main differences between the different EAL levels?
\end{questions}

%%% Local Variables:
%%% mode: latex
%%% TeX-master: "../EITP20-Secure_Systems_Engineering-Study_Questions"
%%% End:
