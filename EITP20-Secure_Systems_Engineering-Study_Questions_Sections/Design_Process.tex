\section{Design Process}\label{sec:Design_Process}
\begin{questions}
\question{} Which are the steps involved in the overall design process of a secure system?
  \begin{solution}
    \begin{enumerate}[noitemsep]
    \item Threat Analysis
    \item Security Requirements
    \item \nameref{sec:Security_Architectures}
    \item \nameref{sec:Security_Design}
    \item \nameref{sec:Security_Evaluation}
    \end{enumerate}
  \end{solution}

  \begin{parts}
  \part{} Describe the relationship between the different steps?
    \begin{solution}
      In the Threat Analysis, the system is first scoped, to ensure that we limit our possibilities a little bit.
      Then, we attempt to find potential threats to the scoped system by performing one or more techniques (``Attacker can steal sensitive personal information.'').

      Once we find the potential threats, we use those along with any information/requirements given by the client to develop Security Requirements (``We require the use of encryption on personal data.'').

      The Security Architecture is the phase where we determine the kind of infrastructure we will need, the communication flows that are possible, find the interactions in the system, etc.
      This allows us to place our requirements onto the system as it exists already, and allow us to design for the system AND its longevity (``The personal info server is kept physically separate from others, only only communicates over mutually authenticated means using an appropriate cryptographic key.'').

      The Security Design is the process of actually choosing the standard, protocols, and hardware used to build the system (``The server will use keys generated by a Hardware Security Module, to ensure randomness.'').

      The Security Evaluation is done by going back to the Security Requirements and comparing the solutions you built in the Security Design and flows you developed in the Security Architecture to see if the requirements were satisfied.
      Other considerations can be put here too, including business ones (``Using a too difficult encryption algorithm will cause slowdowns of data requests/uses of the personal information.'').
    \end{solution}
  \end{parts}

\question{} In which way does a use-case description assist in the secure system design process?
  \begin{solution}
    It gives a general idea of \emph{how} the system will be used.
    This is important, because you can design a secure system in multiple ways to counter multiple threats.
    But, given the use-case of the client, there may be restrictions on what you can design for.
    It will help you derive the Security Requirements for the system, helping you Architect, Design, and Evaluate later in the process.
  \end{solution}

\question{} What is a threat analysis?
  \begin{solution}
    The process of finding potential threats to a system.
    These can be from inside the client's group, outside, or anywhere else.
    Using one or more methods, the applier can find current vulnerabilities, and future problems with the system.
    These are used after being given the use-case description, so that potential attack vectors can be limited to some extent.
  \end{solution}

\question{} What is the main purpose of performing a threat analysis?
  \begin{solution}
    It helps find the potential vulnerabilities in a system that \textbf{HAVE NOT} been found already.
    This will give a broader overview of the system, allowing for a more general, and hopefully, more all-encompassing secure system design.
  \end{solution}

\question{} Can you give examples of threat analysis approaches?
  \begin{solution}
    \begin{enumerate}[noitemsep]
    \item Schneier Attack Trees
    \item Microsoft's STRIDE Analysis
      \begin{enumerate}[noitemsep]
      \item Spoofing
      \item Tampering
      \item Repudiation
      \item Information Disclosure
      \item Denial of Service
      \item Elevation of Privilege
      \end{enumerate}
    \item MITRE's TARA Analysis
      \begin{enumerate}[noitemsep]
      \item Crown Jewel Analysis
      \item Cyber Risk Remediation Analysis
      \item Cyber Threat Susceptability Analysis
      \end{enumerate}
    \end{enumerate}
  \end{solution}

\question{} What is the purpose of a security requirements list?
  \begin{solution}
    To ensure that you (the secure system designer) and the stakeholders/clients/users of your system design can agree on what must be done.
    Agreement here will ensure the system is secure from both perspectives and that all parites are on the same page when it comes to the issue of security.
  \end{solution}

  \begin{parts}
  \part{} Can you list input sources for deriving security requirements?
    \begin{solution}
      \begin{itemize}[noitemsep]
      \item The use-case description.
      \item The client's system needs.
      \item The client's limitations on cost and performance.
      \item The results from the threat analysis.
      \item Further discussions and interaction between the designer and the client.
      \end{itemize}
    \end{solution}

  \part{} Which are the duties of the security engineering in the security requirements gathering process?
    \begin{solution}
      \begin{itemize}[noitemsep]
      \item To identify the threats and use them to develop the Security Requirements.
      \item To analyze the Security Requirements and correctly design to handle them.
      \item To breakdown the Security Requirements into manageable pieces that allow for simpler system design.
      \end{itemize}
    \end{solution}
  \end{parts}

\question{} Can you elaborate on the main differences between security requirements and other system requirements?
  \begin{solution}
    Security Requirements have a mixture of \emph{\textbf{BOTH}} functional and non-functional requirements.
    By having both, it can be dfficult to verify that the original requirements are fulfilled.

    Some of these requirements may be ways to ``handle'' or ``work'' with the system, such as processes, rather than properties of the system.
    A common example of this is requiring that people can only access data by giving a password, or correctly setting user permissions.

    There is always the possibility that the system may be secure now, but not in the future.
    This means that the system might not have the expected security properties, even if all the evaluations and tests indicated it did.
    This means that Security Requirements \textbf{MUST} be continually updated.
  \end{solution}

\question{} Please list at least three different “types” of security architectures?
  \begin{solution}
    The three that we discussed the most in-class and used in our reports were:
    \begin{enumerate}[noitemsep]
    \item Logical Security Architecture
    \item Physical Security Architecture
    \item Security Service Management Architecture
    \end{enumerate}
  \end{solution}

  \begin{parts}
  \part{} Explain the main differences between the different listed types?
    \begin{solution}
      Logical Security Architecture is taking the view of the \textbf{information to be secured} and designing around that.
      This involves finding the entities/actors in our system, then identifying the functions that the system must fulfill/perform.
      Next, we identify security domains/levels in the system that are in play.
      These determine where we need what kind of security and how much we trust each entity and function in each domain.

      Physical Security Architecture is done by representing the \textbf{security data model structures} and showing the ``builder's'' view of the system.
      Here, we map the security services/functions/entities identified in the logical architecture to physical security mechanisms.
      Essentially, if we specify that there is supposed to be a unique random number that acts as an ID for something, then we might say that a Hardware Security Module is required to generate those numbers.
      We do not specify any specific HSM to use, but just state that one must be used to ensure that the numbers are genuinely unique.

      Security Service Management Architecture takes a step back from designing the system for the security of the system in mind, and rather consider the maintainability of the system's security aspects.
      This is technically done at every layer of the SABSA Security Architecture model, but it is also an explicit job.
      Here, we determine how maintainable a system is given the security decisions made.
    \end{solution}
  \end{parts}

\question{} Which are the main steps preceeding the actual security design step?
  \begin{solution}
    The performance of a Security Architecture, and ensuring that the Architecture is followed when designing the system.
    Additionally, the Security Requirements make a difference on the choice of standards/protocols/hardware to use in the system.
  \end{solution}

\question{} Explain the main design choices that need to be done at the security design process.
  \begin{solution}
    \begin{itemize}[noitemsep]
    \item To find suitable security ``elements'' that can implement our chosen Secure Architecture.
    \item To find suitable security ``elements'' that can meet both our and the provided Security Requirements.
    \item To find suitable security ``elements'' that can fulfill the expected system performance and cost expectations.
    \item To make design choices that allow us to handle future, currently unknown, security weaknesses.
    \end{itemize}
  \end{solution}

\question{} List and explain different type of security evaluations that typically are done ``in-house''
  \begin{solution}
    Typically, if there is little threat if the system's security is compromised, any and all testing can be done in-house.
    However, if the system contains more valuable information, or has higher risks of being compromised and may be dangerous, typically external experts are used.
  \end{solution}

\question{} List and explain different type of security evaluations that typically are done by external experts.
  \begin{solution}
    Pen-tests, Protocol Analyses, Common Criteria Evaluations, etc.\ can all be done by external experts.
    In addition, the external experts will likely have certifications in various aspects of secure system design.

    They can Attack (Red), Build (Yellow), or Defend (Blue) from attacks and designs to attempt to subvert the system.
  \end{solution}

\question{} What is a pen test and what is the purpose of such test?
  \begin{solution}
    To deliberately attempt to attack and compromise the system.
    This helps find security vulnerabilities that may have been missed, or that the system doesn't counter against.
    It is also a way to test the recovery infrastructure, and test the people that are tasked with working with the system and ensure that they know what to do.
  \end{solution}

\question{} What is a protocol analysis tool and what is the purpose of using such tool?
  \begin{solution}
    To see if the communciation flows used in the system are actually secure.
    These can be used to ensure that when entities are communicating,t hey will actually be doing so in a safe and secure manner.
    This also helps formally prove the security of a system.
  \end{solution}

\question{} What is the Common Criteria (CC) standard?
  \begin{parts}
  \part{} List the 7 evaluation levels defined in CC and explain the main differences between the levels?
    \begin{solution}
      \begin{description}[noitemsep]
      \item[EAL 1] Functionally Tested
      \item[EAL 2] Structurally Tested
      \item[EAL 3] Methodically Tested and Checked
      \item[EAL 4] Methodically Designed, Tested, and Reviewed
      \item[EAL 5] Semi-Formally Designed and Tested
      \item[EAL 6] Semi-Formally Verified, Designed, and Tested
      \item[EAL 7] Formally Verified, Designed, and Tested
      \end{description}
    \end{solution}

  \part{} List the three different system documents part of a CC and describe their main purpose.
    \begin{solution}
      \begin{enumerate}[noitemsep]
      \item Protection Profile (PP)
        \begin{itemize}[noitemsep]
        \item Security needs for a particular class of devices.
        \item The product that is being designed for will use one or more Protection Profiles to evaluate against.
        \end{itemize}
      \item Security Target (ST)
        \begin{itemize}[noitemsep]
        \item The specification of the actual security properties that we will be evaluating.
        \item This is usually published to give a clear understanding of the scope of the evaluation.
        \end{itemize}
      \item Security Functional Requirements (SFR)
        \begin{itemize}[noitemsep]
        \item Specification of the security functions of the target product that is provided for individuals
        \end{itemize}
      \end{enumerate}
    \end{solution}
  \end{parts}

\question{} What is CISSP?\@
  \begin{solution}
    CISSP is the Certified Information Systems Security Professional certification and certification program.
    This essentially says that you are skilled in designing secure information systems.
  \end{solution}
\end{questions}

%%% Local Variables:
%%% mode: latex
%%% TeX-master: "../EITP20-Secure_Systems_Engineering-Study_Questions"
%%% End:
