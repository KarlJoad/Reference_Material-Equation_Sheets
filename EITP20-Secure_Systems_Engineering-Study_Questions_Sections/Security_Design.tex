\section{Security Design}\label{sec:Security_Design}
\begin{questions}
\question{} Can you list four different principles upon which a security design should be based?
  \begin{solution}
    \begin{enumerate}[noitemsep]
    \item Use well-proven techniques, standards, libraries, hardware modules, and solutions.
    \item Keep system complexity as low as possible.
    \item Minimize the utilization of trusted components.
    \item Only develop your own solutions when really needed.
    \item Use open designs whenever possible.
    \end{enumerate}
  \end{solution}

  \begin{parts}
  \part{} Give a motivation for each of the listed design principles
    \begin{solution}
      \begin{enumerate}[noitemsep]
      \item Well-proven solutions are constantly tested, so they have more verification of their security.
      \item By reducing the number of things that can go wrong, the system can be more secure.
        The reduced number of vulnerabilities means we can focus more on the ones that remain.
      \item The greater the reliance on trusted components, the more trust that must be placed in the hardware of the system.
      \item Developing novel and unique solutions requires a lot of design and engineering time. Additionally, if they are not well-designed, they can cause more problems than they solve.
      \item By using open designs, there is a lower risk of licensing issues.
      \end{enumerate}
    \end{solution}
  \end{parts}

\question{} Describe the process steps to perform when going from a security architecture to a design specification.
  \begin{solution}
    \begin{enumerate}[noitemsep]
    \item Start with the Security Architecture.
    \item Map the desired security services to security standards.
    \item Identify gaps that are present in the System Design/Architecture.
      \begin{itemize}[noitemsep]
      \item Complement the existing Security Design/Architecture with new Designs if needed.
      \end{itemize}
    \item Perform a \nameref{sec:Security_Evaluation}.
    \item Document the \nameref{sec:Security_Design}.
    \item Identify implementation efforts (missing hardware/software).
    \end{enumerate}
  \end{solution}

\question{} What is typically the role of unique identities in security systems?
  \begin{solution}
    To create a unique mapping from name to key, allowing for reliable authentication.
    This unique name can be used for discrete access control and authorization decisions in the system.
  \end{solution}

  \begin{parts}
  \part{} Give example of three different widely used identity types?
    \begin{solution}
      \begin{enumerate}[noitemsep]
      \item ITU
      \item IEEE's MAC Addresses
      \item IETF and URIs
      \end{enumerate}
    \end{solution}
  \end{parts}

\question{} What is the problem from privacy perspective with using fixed identities?
  \begin{solution}
    If the identity has been compromised (on the back-end), it can be traced back to a user/device.
  \end{solution}

  \begin{parts}
  \part{} Describe two different methods for avoiding the identity privacy problem in a system design
    \begin{solution}
      \begin{enumerate}[noitemsep]
      \item Randomly Generated Pseudonym.
        There is a middle-man that receives the unique identifier and assigns a random value to that, which is changed each time, and allows for identification of that user/device for that session.
      \item Encrypted Identity that is not visible outside trusted container boundaries.
      \end{enumerate}
    \end{solution}
  \end{parts}

\question{} What is a digital certificate and how it is used?
  \begin{solution}
    A digital certificate is a way to securely distribute cryptographic keys while ensuring that they come from the correct person.
  \end{solution}

  \begin{parts}
  \part{} Which are the most important data fields in an X.509 certificate?
    \begin{solution}
      \begin{itemize}[noitemsep]
      \item The algorithm used in the signature
      \item The Certificate Issuer's name
      \item The period of validity
      \item The Receipient's public key information
      \end{itemize}
    \end{solution}
  \end{parts}

\question{} Which is the most used authentication principle over HTTP?\@
  \begin{solution}
    The most common priniciple is the Challenge principle.
    The only way that the correct user can authenticate with the service is by solving a challenge only the real user should know.
  \end{solution}

  \begin{parts}
  \part{} Describe the different steps in an HTTP basic authentication.
    \begin{solution}
      \begin{enumerate}[noitemsep]
      \item The client performs a \texttt{GET} request.
      \item The server responds with \texttt{Error 401: Unauthorized}, which asks the user to input information.
      \item The client returns their credentials.
      \item The server checks their credentials.
      \item The server either returns \texttt{Error 200: OK} or \texttt{Error 403: Forbidden}.
      \end{enumerate}
    \end{solution}

  \part{} Under which circumstances can basic authentication be used?
    \begin{solution}
      Basic authentication can \textbf{\emph{ONLY}} be used over already-secured TLS channels.
    \end{solution}

  \part{} Typically, how is basic authentication treated at the server side and what is the main reason for using this type of storage?
    \begin{solution}
      The main reason is that it is fairly quick to perform and easy to implement.
      The user inputs their username and password, which is securely transformed, and the server stores the hashed version of the password and a randomly generated number as salt.
      The database can combine the password and salt, hashing them, and checking in the user database, to verify the password.
    \end{solution}
  \end{parts}

\question{} Describe the principles behind hardware token-based authentication.
  \begin{solution}
    By giving \textbf{\emph{ONLY}} the correct user a hardware token, only they can correctly give the information required by the system and provided by the token.
    This is combined with a simpler end-user local authentication procedure.
    This is also quite user-friendly, as the token should always be with the user too.

    This is typically categorized as a 2nd-factor authentication step.
  \end{solution}

\question{} What is the rationale behind two factor authentication?
  \begin{solution}
    By providing only the correct user with the correct information, even if one form of authentication is compromised, the attacker must also compromise another.
    This second compromise should be a different attack vector, reducing the likelihood that one attack compromises both at the same time.
  \end{solution}

\question{} What are the main differences between:
  \begin{parts}
  \part{} TLS server authentication?
    \begin{solution}
      In this implementation, the hosting server is the one that performs the authentication.
      The end-user device only provides credentials in a secure form that the server then verifies.
      This is done at before every session.
    \end{solution}

  \part{} TLS client certificate authentication?
    \begin{solution}
      Here, a Public Key Infrastructure must be setup and maintained.
      This has the benefit that the end-user does \textbf{not} have provide credentials.
      However, the PKI must be maintained and customized, which is quite complex, expensive, and difficult on the client-side.
    \end{solution}

  \part{} TLS pre-share key authentication?
    \begin{solution}
      The server and client pre-share a key, then perform a challenge-based authentication to ensure that each other say are who they say they are.
    \end{solution}
  \end{parts}

\question{} Explain the main principle behind an object security scheme.
  \begin{solution}
    Instead of ensuring that the communication session between the client and server is secure, we instead secure the things sent between them.
    This has the benefit of securing the data throughout the entire transport process, including intermediate buffers and storage.
  \end{solution}

  \begin{parts}
  \part{} How does it differ from a session protection scheme like TLS or IPsec?
    \begin{solution}
      It is more difficult to verify who is sending the information to whom, but the information is still secure.
      MACs and hashes do not provide enough power for non-repudiation.
      MACs and hashes \textbf{must} be combined with Digital Signatures and secure logs to create non-repudiation.

      There is also different types of packets that can be sent to setup the object security scheme.
    \end{solution}
  \end{parts}

\question{} What does RBAC and ABAC stand for with respect to access control systems?
  \begin{solution}
    \begin{description}[noitemsep]
    \item[RBAC] Role-Baesd Access Control
      \begin{itemize}[noitemsep]
      \item Users are grouped into roles.
      \item These roles have certain permissions, and can perform their allowed actions only in selected places.
      \item Roles can have a hierarchical structure, allowing for inheritance schemes to be used.
      \item Removing access to information can be done on a per-user basis after this too.
      \end{itemize}
    \item[ABAC] Attribute-Based Access Control
      \begin{itemize}[noitemsep]
      \item Users are given attributes about themselves.
      \item An Access Control Policy is setup with rules about what attributes can do what.
      \item Each object being stored has attributes about it, describing what it is, what can be done with it, etc.
      \item The current environment conditions are also considered, the time for instance.
      \item These are all combined with into a logical Access Control Mechanism, which grants access/permission to files based on the attributes and rules of a system.
      \end{itemize}
    \end{description}
  \end{solution}

  \begin{parts}
  \part{} Explain the main differences between RBAC and ABAC.\@
    \begin{solution}
      RBAC groups users into Roles, which are then given permissions as a whole.

      ABAC keeps users separate, but gives each attributes.
      Each object is also given a set of attributes.
      A set of rules is created for each user attribute and object attribute to allow access.
      These are then combined with the current environment's conditions to make an access decision.
    \end{solution}
  \end{parts}

\question{} What is the purpose with an access token?
  \begin{solution}
    These tokens contain information about the authorization decisions made by the access controller for the user.
  \end{solution}

  \begin{parts}
  \part{} What does a SAML assertion contain?
    \begin{solution}
      \begin{enumerate}[noitemsep]
      \item AssertionID
      \item IssueInstant
      \item Issuer
      \item MajorVersion
      \item MinorVersion
      \item Conditions
        \begin{itemize}[noitemsep]
        \item NotBefore
        \item NotOnOrAfter
        \item Audience Restriction
        \end{itemize}
      \item Authentication Statement
        \begin{itemize}[noitemsep]
        \item Subject (User's Identity)
        \item Authentication Instant
        \item Authentication Mechanism
        \end{itemize}
      \item Attribute Statement
        \begin{itemize}[noitemsep]
        \item Subject
        \item Asserted Attributes
        \end{itemize}
      \item Digital Signature
      \end{enumerate}
    \end{solution}
  \end{parts}

\question{} How can a Hardware Security Module (HSM) assist in protection of cloud data storage?
\question{} List a couple of widely used commercial server anti-virus tools
  \begin{solution}
    The HSM will store the actual secret key, that it generated.
    It also encrypts this key, which is then returned to the user, and contains an index to the actual secret key in the HSM.\@

    This allows the customer to authenticate against the HSM, then perform actions using the information the HSM returns.
  \end{solution}

\question{} How can the security of a Docker container be enhanced beyond using default configurations?
\question{} How can a Web design be made to make ``clickjacking'' less likely?
\question{} What is the main difference between key provisioning of a public key system compare to a symmetric key-based systems?
  \begin{parts}
  \part{} What are the key issues to consider when making a public key issuing design?
  \part{} Which are the key issues to consider when making a symmetric key issuing design?
  \end{parts}

\question{} List the three main different intrusion detection principles and explain how they work on high level
\question{} What is syslog and how is it typically used?
\question{} What is the main risks with a debug interface (like JTAG) and how should it be treated in a product design to avoid these risks?
\question{} Which are the three different most severe attacks threats against smart card designs and which are the typical countermeasures?
\question{} Give three examples of widely used NIST security standards and what they specify?
\question{} What does IETF stand for?
  \begin{parts}
  \part{} Give example of two well-knows IETF security standards and explain what they specify?
  \end{parts}

\question{} Which type of security standards are done by IEEE and 3GPP respectively?
\question{} What is the difference between public industry bodies and industry standards?
\question{} What is meant by a cancellable biometric protection scheme?
  \begin{parts}
  \part{} Describe how to achieve a cancellable biometric matching system
  \part{} Which alternative biometrics protection approaches can be used?
  \end{parts}
\end{questions}

%%% Local Variables:
%%% mode: latex
%%% TeX-master: "../EITP20-Secure_Systems_Engineering-Study_Questions"
%%% End:
