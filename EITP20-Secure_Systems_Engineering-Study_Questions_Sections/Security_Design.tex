\section{Security Design}\label{sec:Security_Design}
\begin{questions}
\question{} Can you list four different principles upon which a security design should be based?
  \begin{solution}
    \begin{enumerate}[noitemsep]
    \item Use well-proven techniques, standards, libraries, hardware modules, and solutions.
    \item Keep system complexity as low as possible.
    \item Minimize the utilization of trusted components.
    \item Only develop your own solutions when really needed.
    \item Use open designs whenever possible.
    \end{enumerate}
  \end{solution}

  \begin{parts}
  \part{} Give a motivation for each of the listed design principles
    \begin{solution}
      \begin{enumerate}[noitemsep]
      \item Well-proven solutions are constantly tested, so they have more verification of their security.
      \item By reducing the number of things that can go wrong, the system can be more secure.
        The reduced number of vulnerabilities means we can focus more on the ones that remain.
      \item The greater the reliance on trusted components, the more trust that must be placed in the hardware of the system.
      \item Developing novel and unique solutions requires a lot of design and engineering time. Additionally, if they are not well-designed, they can cause more problems than they solve.
      \item By using open designs, there is a lower risk of licensing issues.
      \end{enumerate}
    \end{solution}
  \end{parts}

\question{} Describe the process steps to perform when going from a security architecture to a design specification.
  \begin{solution}
    \begin{enumerate}[noitemsep]
    \item Start with the Security Architecture.
    \item Map the desired security services to security standards.
    \item Identify gaps that are present in the System Design/Architecture.
      \begin{itemize}[noitemsep]
      \item Complement the existing Security Design/Architecture with new Designs if needed.
      \end{itemize}
    \item Perform a \nameref{sec:Security_Evaluation}.
    \item Document the \nameref{sec:Security_Design}.
    \item Identify implementation efforts (missing hardware/software).
    \end{enumerate}
  \end{solution}

\question{} What is typically the role of unique identities in security systems?
  \begin{solution}
    To create a unique mapping from name to key, allowing for reliable authentication.
    This unique name can be used for discrete access control and authorization decisions in the system.
  \end{solution}

  \begin{parts}
  \part{} Give example of three different widely used identity types?
  \end{parts}

\question{} What is the problem from privacy perspective with using fixed identities?
  \begin{parts}
  \part{} Describe two different methods for avoiding the identity privacy problem in a system design
  \end{parts}

\question{} What is a digital certificate and how it is used?
  \begin{parts}
  \part{} Which are the most important data fields in an X.509 certificate?
  \end{parts}

\question{} Which is the far most used authentication principle over http?
  \begin{parts}
  \part{} Describe the different steps in an HTTP basic authentication
  \part{} Under which circumstances can basic authentication be used
  \part{} How are typically basic authentication treated at the server side and what is the main reason for using this type of storage?
  \end{parts}

\question{} Describe the principles behind hardware token based authentication
\question{} What is the rationale behind two factor authentication?
\question{} What are the main differences between:
  \begin{parts}
  \part{} TLS server authentication
  \part{} TLS client certificate authentication
  \part{} TLS pre-share key authentication
  \end{parts}

\question{} Explain the main principle behind an object security scheme
  \begin{parts}
  \part{} How does it differs from a session protection scheme like TLS or IPsec?
  \end{parts}

\question{} What does RBAC and ABAC stand for with respect to access control systems?
  \begin{parts}
  \part{} Explain the main differences between RBAC and ABAC
  \end{parts}

\question{} What is the purpose with an access token?
  \begin{parts}
  \part{} What does a SAML assertion contain?
  \end{parts}

\question{} How can a Hardware Security Module (HSM) assist in protection of cloud data storage?
\question{} List a couple of widely used commercial server anti-virus tools
\question{} How can the security of a Docker container be enhanced beyond using default configurations?
\question{} How can a Web design be made to make ``clickjacking'' less likely?
\question{} What is the main difference between key provisioning of a public key system compare to a symmetric key-based systems?
  \begin{parts}
  \part{} What are the key issues to consider when making a public key issuing design?
  \part{} Which are the key issues to consider when making a symmetric key issuing design?
  \end{parts}

\question{} List the three main different intrusion detection principles and explain how they work on high level
\question{} What is syslog and how is it typically used?
\question{} What is the main risks with a debug interface (like JTAG) and how should it be treated in a product design to avoid these risks?
\question{} Which are the three different most severe attacks threats against smart card designs and which are the typical countermeasures?
\question{} Give three examples of widely used NIST security standards and what they specify?
\question{} What does IETF stand for?
  \begin{parts}
  \part{} Give example of two well-knows IETF security standards and explain what they specify?
  \end{parts}

\question{} Which type of security standards are done by IEEE and 3GPP respectively?
\question{} What is the difference between public industry bodies and industry standards?
\question{} What is meant by a cancellable biometric protection scheme?
  \begin{parts}
  \part{} Describe how to achieve a cancellable biometric matching system
  \part{} Which alternative biometrics protection approaches can be used?
  \end{parts}
\end{questions}

%%% Local Variables:
%%% mode: latex
%%% TeX-master: "../EITP20-Secure_Systems_Engineering-Study_Questions"
%%% End:
