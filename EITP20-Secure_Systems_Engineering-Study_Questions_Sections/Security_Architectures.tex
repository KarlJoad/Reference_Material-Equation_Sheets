\section{Security Architectures}\label{sec:Security_Architectures}
\begin{itemize}
\item Describe what constitutes a security architecture and give some examples.
\item The Sherwood Applied Business Security Architecture (SABSA) consist of 5 layers and one cross layer.
  \begin{itemize}[noitemsep]
  \item Describe the different layer views and list the names of the different layers.
  \item Give examples of questions the different SABSA architecture views are supposed to answer
  \end{itemize}

\item Which are the three different types of security services in a logical security architecture?
  \begin{itemize}[noitemsep]
  \item \item List and explain examples of services part of the non-prevention type of security services.
  \end{itemize}

\item Describe each of the different prevention security services in more details.
  \begin{itemize}[noitemsep]
  \item \item Give example of at least four different entity security services and how they contribute to security prevention.
  \item Give example of at least four different communication security services and how they contribute to security prevention.
  \item Give example of at least four different application level security services and how they contribute to security prevention.
  \item Give example of at least four security management services and how they contribute to security prevention.
  \end{itemize}

\item Draw a picture showing the relations between major different security services from a systems point of view.
\item A logical security architecture can be created using a six steps methodology:
  \begin{itemize}[noitemsep]
  \item Describe each of the different steps
  \item What is the end result?
  \item Give an example of a logical security architecture
  \end{itemize}

\item What is a physical security architecture?
\item A physical security architecture when using the SABSA include making a mapping to physical security mechanisms
  \begin{itemize}[noitemsep]
  \item Describe what is meant by a ``Naming and registration'' mechanism and give examples
  \item Describe what is meant by a ``Storage and runtime'' mechanism and give examples
  \item Describe what is meant by a ``Physical security'' mechanism and give examples
  \item Describe what is meant by an ``Authentication and session'' protection mechanism and give examples
  \item Describe what is meant by a ``User interface and naming'' mechanism and give examples
  \item Describe what is meant by a ``Authorization and access control'' mechanism and give examples
  \item Describe what is meant by a ``Monitoring and incident'' mechanism and give examples
  \end{itemize}

\item What must in addition to the security mechanisms be specified in the physical security architecture?
\item Give example of platform security solution that can be used to build solutions meeting a logical security services and can be used to protect the chosen physical security mechanism?
\item For the SSO logical security architecture given at the lecture, perform the following:
  \begin{itemize}[noitemsep]
  \item Identify the main physical security mechanisms needed in the corresponding physical security architecture.
  \item Identify the main platform security components needed to fulfil the architecture
    \begin{itemize}[noitemsep]
    \item Suggest concrete platform security mechanisms to use for the physical realization.
    \end{itemize}
  \end{itemize}
\end{itemize}

%%% Local Variables:
%%% mode: latex
%%% TeX-master: "../EITP20-Secure_Systems_Engineering-Study_Questions"
%%% End:
