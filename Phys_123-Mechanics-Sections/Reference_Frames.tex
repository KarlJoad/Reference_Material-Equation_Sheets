\section{Reference Frames}\label{sec:Reference Frames}
\begin{definition}[Reference Frames]\label{def:Reference Frames}
  An \emph{inertial reference frame} is a frame for the world.
  It is easiest to think of of an inertial reference frame with an example.
  For instance, when you're in a car going 40 \si{\mph} and you see someone going 45, you can only tell they're going 5 \si{\mph} faster than you.
\end{definition}

\begin{definition}[Galileo Transformation/Relativity Principle]\label{def:Galileo Transformation}
  This is a transformation that happens when you're calculating in one \nameref{def:Reference Frames} and the event is happening in a different \nameref{def:Reference Frames}.

  \begin{equation}\label{eq:Galileo Transformation}
    \begin{aligned}
      \frac{d \vec{R}}{dt} &= \frac{d \vec{r}}{dt} + \frac{d \vec{r}'}{dt} \\
      \vec{V} &= \vec{v} + \vec{v}'
    \end{aligned}
  \end{equation}
\end{definition}

%%% Local Variables:
%%% mode: latex
%%% TeX-master: "../Phys_123-Mechanics-Reference_Sheet"
%%% End:
