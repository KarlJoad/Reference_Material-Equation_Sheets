\section{Dynamics of Circular Motion}\label{sec:Dynamics Circular Motion}
There are three cases when the \nameref{sec:Dynamics Circular Motion} are easily visible.
Each of these is illustrated with an example.
\begin{enumerate}[noitemsep, nolistsep]
  \item Turning on a flat curve, neglecting friction (\Cref{ex:Car Turning on a Flat Curve})
  \item Turning on an angled curve, neglecting friction (\Cref{ex:Car Turning on Angled Curve with No Friction})
  \item Turning on an angled curve, with friction (\Cref{ex:Car Turning on Angled Curve with Friction})
\end{enumerate}

\begin{example}[]{Car Turning on a Flat Curve}
  A car of mass $m$ is turning on a flat curve with radius $r = \SI{50}{\meter}$.
  The \nameref{def:Coefficient of Static Friction} is $\mu_{s} = 0.7$.
  What is the car's maximum velocity, $v_{\text{Max}}$ before it cannot turn anymore?
  \tcblower{}

  First, we start by finding the net force in the system, considering the \nameref{def:Normal Force} and \nameref{def:Frictional Force} to be positive.
  \begin{align*}
    \shortintertext{All Forces Present:}
    \vec{F}_{Net} &= \vec{N} - m\vec{g} + \vec{F}_{f} \\
    \shortintertext{Newton's Second Law:}
    \vec{F}_{Net} &= m\vec{a} \\
  \end{align*}

  Now that all forces have been identified, we can set both equations equal to each other and being substituting other expressions in.
  \begin{align*}
    \vec{N} - m\vec{g} + \vec{F}_{f} &= m\vec{a} \\
    \shortintertext{Remember the formula for static friction:}
    \vec{F}_{f} &= \mu_{s} \vec{N} \\
  \end{align*}

  With those 2 relations gathered, we now need to consider the \nameref{def:Normal Force}.
  This \emph{must} be $0$, otherwise the car is starting to float or fall into the ground.
  \begin{align*}
    \vec{N} - m\vec{g} &= 0 \\
    \vec{N} &= m\vec{g} \\
  \end{align*}

  Now, we can substitute everything back into our main equation.
  \begin{align*}
    \vec{N} - m\vec{g} + \vec{F}_{f} &= m\vec{a} \\
    \shortintertext{Subsitute $F_{f}$:}
    \vec{N} - m\vec{g} + \mu_{s}\vec{N} &= m\vec{a} \\
    \shortintertext{Substitute $\vec{N}$:}
    m\vec{g} - m\vec{g} + \mu_{s} m\vec{g} &= m\vec{a} \\
    \shortintertext{Reduce:}
    \mu_{s} m\vec{g} &= m\vec{a} \\
    \shortintertext{Substitute $\vec{a}$ with \Cref{eq:Relate Acceleration Angular Acceleration}:}
    \mu_{s} m\vec{g} &= m \frac{v_{Max}^{2}}{r} \\
    \shortintertext{Simplify:}
    v_{Max} &= \sqrt{\mu_{s} \vec{g} r} \\
  \end{align*}

  With this expression, we can solve for the maximum velocity this vehicle can move.
  The expression $v_{Max} = \sqrt{\mu_{s} \vec{g} r}$ can be thought of as the speed where the \nameref{def:Coefficient of Static Friction} ``breaks.''
  \begin{equation*}
    v_{Max} = \sqrt{\mu_{s} \vec{g} r} = \sqrt{0.7 (\SI{9.81}{\meter/\second}) \SI{50}{\meter}} = \SI{18.5}{\meter/\second}
  \end{equation*}
\end{example}

\begin{example}[]{Car Turning on Angled Curve with No Friction}
  A car of mass $m$ is turning on a curve $r$ meters from the center, with an angle $\theta$ from the horizontal.
  If there is no friction, what is its maximum velocity, $v_{\text{Max}}$?
  \tcblower{}

  Here, because there is an angle involved, we need to decompose the vectors into their components.
  Starting with the vertical component, we find the normalizing force at an angle to the force of gravity.
  \begin{align*}
    \vec{N} \cos(\theta) - m\vec{g} &= \vec{0} \\
    \vec{N} \cos(\theta) &= m\vec{g} \\
  \end{align*}

  Now, we need to consider the horizontal components, which is the sliding that happens up or down the curve of the turn.
  \begin{align*}
    \vec{N} \sin(\theta) &= m a_{c} \\
    \intertext{Substitute $a_{c}$ with \Cref{eq:Relate Acceleration Angular Acceleration}:}
    \vec{N} \sin(\theta) &= m \frac{v_{Max}^{2}}{r} \\
  \end{align*}

  With both components of the final movement calculated, we single out $\vec{N}$ and set each relation equal to each other.
  \begin{align*}
    \shortintertext{Vertical:}
    \vec{N} &= \frac{m\vec{g}}{\cos(\theta)} \\
    \shortintertext{Horizontal:}
    \vec{N} &= \frac{m v_{Max}^{2}}{r \sin(\theta)} \\
    \shortintertext{Substitute for $\vec{N}$:}
    \frac{m\vec{g}}{\cos(\theta)} &= \frac{m v_{Max}^{2}}{r \sin(\theta)} \\
    \shortintertext{Simplify:}
    v_{Max}^{2} &= \frac{\vec{g} r \sin(\theta)}{\cos(\theta)} \\
    v_{Max} &= \sqrt{\vec{g} r \tan(\theta)} \\
  \end{align*}

  Thus, our final expression for this problem is
  \begin{equation*}
    v_{Max} = \sqrt{\vec{g} r \tan(\theta)}
  \end{equation*}
\end{example}

\begin{example}[]{Car Turning on Angled Curve with Friction}
  A car of mass $m$ is turning on a curve $r$ meters from the center, with an angle $\theta$ from the horizontal.
  There is friction, with the coefficient of static friction being $\mu_{s}$ and coefficient of kinetic friction being $\mu_{k}$.
  What is its maximum velocity, $v_{Max}$?
  \tcblower{}

  First, we can eliminate the use of $\mu_{k}$.
  This is because we are not concerned with the vehicle's sliding forward or backward in the direction of travel.
  Instead we want to know about the car's movement up or down the ramp due to its velocity.

  Again, just like the previous example, the car is traveling at an angle, so the forces acting on the vehicle must be broken down into their components.
  Like \Cref{ex:Car Turning on a Flat Curve}:
  \begin{equation*}
    \vec{F}_{f} = \mu_{s} \vec{N}
  \end{equation*}

  Starting with the vertical again:
  \begin{align*}
    \vec{N} \cos(\theta) - m\vec{g} + \vec{F}_{f} \sin(\theta) &= 0 \\
    \shortintertext{Substitute $F_{f}$:}
    \vec{N} \cos(\theta) - m\vec{g} + \mu_{s} \vec{N} \sin(\theta) &= 0 \\
    \shortintertext{Reduce and Simplify:}
    \vec{N} \cos(\theta) + \mu_{s} \vec{N} \sin(\theta) &= m \vec{g} \\
    \vec{N} \bigl( \cos(\theta) + \mu_{s} \sin(\theta) \bigr) &= m \vec{g} \\
    \vec{N} &= \frac{m \vec{g}}{\cos(\theta) + \mu_{s} \sin(\theta)} \\
  \end{align*}

  Now onto the horizontal:
  \begin{align*}
    \vec{N} \sin(\theta) - \vec{F}_{f} \cos(\theta) &= m a_{c} \\
    \intertext{Substitute $a_{c}$ with \Cref{eq:Relate Acceleration Angular Acceleration}:}
    \vec{N} \sin(\theta) - \vec{F}_{f} \cos(\theta) &= m \frac{v^{2}}{r} \\
    \shortintertext{Substitute $F_{f}$:}
    \vec{N} \sin(\theta) - \mu_{s} \vec{N} \cos(\theta) &= m \frac{v^{2}}{r} \\
    \shortintertext{Reduce and Simplify:}
    \vec{N} \bigl( \sin(\theta) - \mu_{s} \cos(\theta) \bigr) &= m \frac{v^{2}}{r} \\
    \vec{N} &= \frac{m v^{2}}{r \bigl( \sin(\theta) - \mu_{s} \cos(\theta) \bigr)} \\
  \end{align*}

  With both components simplified, we can subsitute one for the other.
  \begin{align*}
    \shortintertext{Vertical:}
    \vec{N} &= \frac{m \vec{g}}{\cos(\theta) + \mu_{s} \sin(\theta)} \\
    \shortintertext{Horizontal:}
    \vec{N} &= \frac{m v^{2}}{r \bigl( \sin(\theta) - \mu_{s} \cos(\theta) \bigr)} \\
    \shortintertext{Substitute:}
    \frac{m \vec{g}}{\cos(\theta) + \mu_{s} \sin(\theta)} &= \frac{m v^{2}}{r \bigl( \sin(\theta) - \mu_{s} \cos(\theta) \bigr)} \\
    \shortintertext{Reduce and Simplify for $v_{Max}$:}
    v_{Max}^{2} &= \frac{\vec{g} r \bigl( \sin(\theta) - \mu_{s} \cos(\theta) \bigr)}{\cos(\theta) + \mu_{s} \sin(\theta)} \\
    v_{Max} &= \sqrt{\vec{g} r \left( \frac{\sin(\theta) - \mu_{s} \cos(\theta)}{\cos(\theta) + \mu_{s} \sin(\theta)} \right)} \\
  \end{align*}

  Thus, our final expression for the maximum velocity of a car on a curved turn with friction is:
  \begin{equation*}
    v_{Max} = \sqrt{\vec{g} r \left( \frac{\sin(\theta) - \mu_{s} \cos(\theta)}{\cos(\theta) + \mu_{s} \sin(\theta)} \right)}
  \end{equation*}

  However, one thing to note is the way I chose to interpret $F_{f}$.
  I chose to point it out of the curve, which is what happens when the car is moving slowly enough.
  If you were to point the $F_{f}$ vector towards the inside of the curve, you would end up with this instead.
  \begin{equation*}
    v_{Max} = \sqrt{\vec{g} r \left( \frac{\sin(\theta) + \mu_{s} \cos(\theta)}{\cos(\theta) - \mu_{s} \sin(\theta)} \right)}
  \end{equation*}

  Another thing to note is that if we consider $\mu_{s} = 0$, i.e.\ there is no friction, we get the same result back as in \Cref{ex:Car Turning on Angled Curve with No Friction}.
\end{example}

%%% Local Variables:
%%% mode: latex
%%% TeX-master: "../Phys_123-Mechanics-Reference_Sheet"
%%% End: