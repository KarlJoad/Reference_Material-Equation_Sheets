\begin{example}[]{Elastic Collision with Stationary Object}
  Saw we have 2 cars, with springs attached to them.
  The first car is moving at $v_{1} = 5 \si{\meter / \second}$ and has a mass of $m_{1} = 2 \si{\kilo \gram}$.
  The second car is no moving $v_{2} = 0 \si{\meter / \second}$ and has a mass of $m_{2} = 1 \si{\kilo \gram}$.
  What is the final velocities of the cars?

  \tcblower

  Start by referencing \Cref{eq:Elastic Collision}.
  \begin{align*}
    \frac{1}{2} m_{1} v_{1}^{2} + \frac{1}{2} m_{2} v_{2}^{2} &= \frac{1}{2} m_{1} v_{1}'^{\,2} + \frac{1}{2} m_{2} v_{2}'^{\,2} \\
    m_{1} v_{1} + m_{2} v_{2} &= m_{1} v_{1}'^{\,2} + m_{2} v_{2}'^{\,2}
  \end{align*}

  Now we start plugging our given values in.
  Starting with the \nameref{def:Kinetic Energy} portion of the above equations.

  \begin{align*}
    \frac{1}{2} (2) 5^{2} + \frac{1}{2} (1) 0^{2} &= \frac{1}{2} (2) v_{1}'^{\,2} + \frac{1}{2} (1) v_{2}'^{\,2} \\
    25 + 0 &= v_{1}'^{\,2} + \frac{1}{2} v_{2}'^{\,2} \\
    25 &= v_{1}'^{\,2} + \frac{1}{2} v_{2}'^{\,2}
  \end{align*}

  Now, moving to the \nameref{def:Momentum} portion of the equation.

  \begin{align*}
    2 (5) + 1 (0) &= 2 v_{1}'^{\,2} + 1 v_{2}'^{\,2} \\
    10 &= 2 v_{1}'^{\,2} + 1 v_{2}'^{\,2}
  \end{align*}

  Now we have to solve this system of equations.
  \begin{align*}
    25 &= v_{1}'^{\,2} + \frac{1}{2} v_{2}'^{\,2} \\
    10 &= 2 v_{1}'^{\,2} + 1 v_{2}'^{\,2}
  \end{align*}

  When solved, you end up with,
  \begin{align*}
    v_{1}' &= \frac{5}{3} \si{\meter / \second} \\
    v_{2}' &= \frac{20}{3} \si{\meter / \second}
  \end{align*}

  After the collision, both cars are going the same direction.
  The originally moving one is moving at $\frac{5}{3}$ \si{\meter / \second}.
  The originally stationary one is moving at $\frac{20}{3}$ \si{\meter / \second}.
\end{example}

%%% Local Variables:
%%% mode: latex
%%% TeX-master: "../Phys_123-Mechanics-Reference_Sheet"
%%% End:
