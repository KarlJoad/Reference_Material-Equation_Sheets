\section{Springs}\label{sec:Springs}
Springs are useful things.
We can model many things as a spring for \nameref{sec:Statics}
\subsection{Hooke's Law}\label{subsec:Hookes Law}
\begin{definition}[Hooke's Law]\label{def:Hookes Law}
  \emph{Hooke's law} relates the distance a spring is pulled from equilibrium to the force it will exert as it returns to equilibrium.
  Because of \nameref{sec:Newtons Laws}, this will also tell us the force required to move a spring from its equilibrium position.

  \begin{equation}\label{eq:Hookes Law}
    \vec{F} = -k \Delta x
  \end{equation}
  \begin{itemize}[noitemsep, nolistsep]
    \item $k$ - The spring constant. A unique value for the ``springyness'' of any spring
    \item $\Delta x$ - The distance displaced from the equilibrium position
  \end{itemize}

  \begin{remark}
    \nameref{def:Hookes Law} only works for distances where $\Delta x$ are relatively small.
    If $\Delta x$ were to become too large, your spring would cease to be a spring and become a straight piece of metal.
  \end{remark}
\end{definition}

%%% Local Variables:
%%% mode: latex
%%% TeX-master: "../Phys_123-Mechanics-Reference_Sheet"
%%% End: