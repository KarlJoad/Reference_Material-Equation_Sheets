\section{Collisions}\label{sec:Collisions}
Collisions are a fundamental part of classical physics.
For our uses, we will assume that these are closed systems.

In practice, there are 2 types of collisions.

\begin{enumerate}[noitemsep, nolistsep]
  \item \nameref{def:Elastic Collision}s
  \item \nameref{def:Inelastic Collision}s
\end{enumerate}

\subsection{Elastic Collision}\label{subsec:Elastic Collision}
\begin{definition}[Elastic Collision]\label{def:Elastic Collision}
  An \emph{elastic collision} is a collision where the objects colliding:

  \begin{itemize}[noitemsep, nolistsep]
    \item ``Bounce'' apart
    \item The objects' internal structure is unchanged
  \end{itemize}

  There are some conclusions that we can draw about these collisions based on the criteria above, assuming it is a closed system.
  \begin{enumerate}
    \item Mechanical energy is conserved. No kinetic energy is lost, it is converted to potential energies.
    \item Linear \nameref{def:Momentum} is conserved.
  \end{enumerate}

  The set of equations you will need to solve to solve an \nameref{def:Elastic Collision} problem are:
  \begin{equation}\label{eq:Elastic Collision}
    \begin{aligned}
      \sum\limits_{i=1}^{n} \frac{1}{2} m_{i}v_{i}^{2} &= \sum\limits_{i=1}^{n} \frac{1}{2} m_{i}v_{i}'^{\,2} \\
      \sum\limits_{i=1}^{n} m_{i}v_{i} &= \sum\limits_{i=1}^{n} m_{i}v_{i}'
    \end{aligned}
  \end{equation}

  \begin{itemize}[noitemsep, nolistsep]
    \item The $v_{i}'$ are \textbf{\emph{NOT}} derivatives, but the velocities after the collision
    \item Each equation is just the sum of each object's \nameref{def:Momentum} or \nameref{def:Kinetic Energy} before and after the collision
  \end{itemize}

  \begin{remark}
    Remember to keep your signs ($+$, $-$) in order!!
    The initial velocity and final velocity might be going in different directions, keep that in mind!!
  \end{remark}
\end{definition}

\begin{example}[]{Elastic Collision with Stationary Object}
  Say we have 2 cars, with springs attached to them.
  The first car is moving at $v_{1} = 5 \si{\meter / \second}$ and has a mass of $m_{1} = 2 \si{\kilo \gram}$.
  The second car is not moving $v_{2} = 0 \si{\meter / \second}$ and has a mass of $m_{2} = 1 \si{\kilo \gram}$.
  What is the final velocities of the cars?

  \tcblower

  Start by referencing \Cref{eq:Elastic Collision}.
  \begin{align*}
    \frac{1}{2} m_{1} v_{1}^{2} + \frac{1}{2} m_{2} v_{2}^{2} &= \frac{1}{2} m_{1} v_{1}'^{\,2} + \frac{1}{2} m_{2} v_{2}'^{\,2} \\
    m_{1} v_{1} + m_{2} v_{2} &= m_{1} v_{1}'^{\,2} + m_{2} v_{2}'^{\,2}
  \end{align*}

  Now we start plugging our given values in.
  Starting with the \nameref{def:Kinetic Energy} portion of the above equations.

  \begin{align*}
    \frac{1}{2} (2) 5^{2} + \frac{1}{2} (1) 0^{2} &= \frac{1}{2} (2) v_{1}'^{\,2} + \frac{1}{2} (1) v_{2}'^{\,2} \\
    25 + 0 &= v_{1}'^{\,2} + \frac{1}{2} v_{2}'^{\,2} \\
    25 &= v_{1}'^{\,2} + \frac{1}{2} v_{2}'^{\,2}
  \end{align*}

  Now, moving to the \nameref{def:Momentum} portion of the equation.

  \begin{align*}
    2 (5) + 1 (0) &= 2 v_{1}'^{\,2} + 1 v_{2}'^{\,2} \\
    10 &= 2 v_{1}'^{\,2} + 1 v_{2}'^{\,2}
  \end{align*}

  Now we have to solve this system of equations.

  \begin{align*}
    25 &= v_{1}'^{\,2} + \frac{1}{2} v_{2}'^{\,2} \\
    10 &= 2 v_{1}'^{\,2} + 1 v_{2}'^{\,2}
  \end{align*}

  When solved, you end up with
  \begin{align*}
    v_{1}' &= \frac{5}{3} \si{\meter / \second} \\
    v_{2}' &= \frac{20}{3} \si{\meter / \second}
  \end{align*}

  After the collision, both cars are going the same direction.
  The originally moving one is moving at $\frac{5}{3}$ \si{\meter / \second}.
  The originally stationary one is moving at $\frac{20}{3}$ \si{\meter / \second}.
\end{example}

\begin{example}[]{Elastic Collision with Non-Stationary Objects}
  Say we have 2 cars, with springs attached to them.
  The first car is moving at $v_{1} = 5 \si{\meter / \second}$ and has a mass of $m_{1} = 2 \si{\kilo \gram}$.
  The second car is moving towards the first $v_{2} = 10 \si{\meter / \second}$ and has a mass of $m_{2} = 1 \si{\kilo \gram}$.
  What is the final velocities of the cars?

  \tcblower

  Start by referencing \Cref{eq:Elastic Collision}.
  \begin{align*}
    \frac{1}{2} m_{1} v_{1}^{2} + \frac{1}{2} m_{2} v_{2}^{2} &= \frac{1}{2} m_{1} v_{1}'^{\,2} + \frac{1}{2} m_{2} v_{2}'^{\,2} \\
    m_{1} v_{1} + m_{2} v_{2} &= m_{1} v_{1}'^{\,2} + m_{2} v_{2}'^{\,2}
  \end{align*}

  Now we start plugging our given values in.
  Starting with the \nameref{def:Kinetic Energy} portion of the above equations.
  Keep in mind that the velocity of one car will have to be defined as negative, since the two cars are moving towards each other.

  \begin{align*}
    \frac{1}{2} (2) 5^{2} + \frac{1}{2} (1) (-10)^{2} &= \frac{1}{2} (2) v_{1}'^{\,2} + \frac{1}{2} (1) v_{2}'^{\,2} \\
    25 + 50 &= v_{1}'^{\,2} + \frac{1}{2} v_{2}'^{\,2} \\
    75 &= v_{1}'^{\,2} + \frac{1}{2} v_{2}'^{\,2}
  \end{align*}

  Now, moving to the \nameref{def:Momentum} portion of the equation.

  \begin{align*}
    2 (5) + 1 (-10) &= 2 v_{1}'^{\,2} + 1 v_{2}'^{\,2} \\
    10 + -10 &= 2 v_{1}'^{\,2} + v_{2}'^{\,2} \\
    0 &= 2 v_{1}'^{\,2} + v_{2}'^{\,2}
  \end{align*}

  Now we have to solve this system of equations.

  \begin{align*}
    v_{1}'^{\,2} + \frac{1}{2} v_{2}'^{\,2} &= 75 \\
    2 v_{1}'^{\,2} + v_{2}'^{\,2} &= 0
  \end{align*}

  When solved, you end up with

  \begin{align*}
    v_{1}' &= -5 \si{\meter / \second} \\
    v_{2}' &= 10 \si{\meter / \second}
  \end{align*}

  After the collision, the cars are going away from each other. The first car is going backward at 5 \si{\meter / \second}. The second car is moving forward (what we defined as forward) at 10 \si{\meter / \second}.
\end{example}

\subsection{Inelastic Collisions}\label{subsuec:Inelastic Collisions}
\begin{definition}[Inelastic Collision]\label{def:Inelastic Collision}
  An \emph{inelastic collision} is a collision where the objects colliding:

  \begin{itemize}[noitemsep, nolistsep]
    \item ``Stick'' together
    \item The objects' internal structure is changed
  \end{itemize}

  There are some conclusions that we can draw about these collisions based on the criteria above, assuming it is a closed system.
  \begin{enumerate}[noitemsep, nolistsep]
    \item Mechanical energy is \textbf{\emph{NOT}} conserved.
    \item Linear \nameref{def:Momentum} is conserved.
  \end{enumerate}

  The equation that you will need to solve an \nameref{def:Inelastic Collision} problem is:
  \begin{equation}\label{eq:Inelastic Collision}
    \sum\limits_{i=1}^{n} m_{i}v_{i} = \left( \sum\limits_{i=1}^{n} m_{i} \right) v'
  \end{equation}

  \begin{itemize}[noitemsep, nolistsep]
    \item $v'$ is the final velocity of the objects after they have collided. It is the same for all objects because they stuck together.
  \end{itemize}

  \begin{remark}
    Remember to keep your signs ($+$, $-$) in order!!
    The initial velocity and final velocity might be going in different directions, keep that in mind!!
  \end{remark}
\end{definition}

\subsection{Collisions in Multiple Dimensions}\label{subsec:Multi-D Collisions}
To solve a collision problem that is in multiple dimensions, you break it down into its component vectors, and solve each separately.

\begin{example}[]{Multiple Dimension Collision}
  Say 2 objects are travelling in an $xy$-plane.
  When they have an \emph{inelastic} collision, they veer off at different angles.
  Object A has an angle of $\alpha$ from the $x$-axis and is travelling at $\vec{v}_{A,f}$.
  Object A is initally travelling on the $x$-axis, with no $y$-component.
  Object B has an angle of $\beta$ from the $x$-axis and is travelling at $v_{B,f}$.
  Object B initially starts at rest, $\vec{v}_{B,i} = \langle 0, 0 \rangle$.

  What are the initial velocity of object A?

  \tcblower

  Since this is an \nameref{def:Inelastic Collision}, we want to use \Cref{eq:Inelastic Collision}.

  \begin{equation*}
    m_{A} \, \vec{v}_{A,i} + m_{B} \, \vec{v}_{B,i} = m_{A} \, \vec{v}_{A,f} + m_{B} \, \vec{v}_{B,f}
  \end{equation*}

  Now start plugging in information that we know.

  \begin{align*}
    m_{A} \, \vec{v}_{A,i} + m_{B} \left( \vec{0} \right) &= m_{A} \, \vec{v}_{A,f} + m_{B} \, \vec{v}_{B,f} \\
    m_{A} \, \vec{v}_{A,i} + 0 &= m_{A} \, \vec{v}_{A,f} + m_{B} \, \vec{v}_{B,f} \\
    m_{A} \, \vec{v}_{A,i} &= m_{A} \, \vec{v}_{A,f} + m_{B} \, \vec{v}_{B,f}
  \end{align*}

  Since this is a multi-dimensional collision problem, we need need to break the velocity vectors down into their component vectors.

  \begin{align*}
    \mathbf{X} \rightarrow m_{A} \, \vec{v}_{A,i} \cos (0) &= m_{A} v_{A,f} \cos (\alpha) + m_{B} \, v_{B,f} \cos (\beta) \\
    \mathbf{Y} \rightarrow m_{A} \, \vec{v}_{A,i} \sin (0) &= m_{A} v_{A,f} \sin (\alpha) - m_{B} \, v_{B,f} \sin (\beta) \\
  \end{align*}

  Object B is moving the the negative $y$-direction, so it has a negative in front of its term.
  We know that Object A was not moving in the $y$-direction at all, so the above equations simplify down to:

  \begin{align*}
    \mathbf{X} &\rightarrow m_{A} \, \vec{v}_{A,i} \cos (0) = m_{A} v_{A,f} \cos (\alpha) + m_{B} \, v_{B,f} \cos (\beta) \\
    \mathbf{Y} &\rightarrow 0 = m_{A} v_{A,f} \sin (\alpha) - m_{B} \, v_{B,f} \sin (\beta) \\
  \end{align*}

  Since we know the final velocity of Object A, the $y$-component equation can give us the one unknown we have, $v_{B,f}$.

  \begin{align*}
    0 &= m_{A} \, v_{A,f} \sin (\alpha) - m_{B} \, v_{B,f} \sin (\beta) \\
    m_{A} \, v_{A,f} \sin (\alpha) &= m_{B} \, v_{B,f} \sin (\beta) \\
    v_{B,f} &= \frac{m_{A}v_{A,f} \sin (\alpha)}{m_{B} \sin (\beta)}
  \end{align*}

  Now, plugging in that value to the $x$-component equation, we can solve for $v_{A,i}$.

  \begin{align*}
    m_{A} \, v_{A,i} &= m_{A} \, v_{A,f} \cos (\alpha) + m_{B} \left( \frac{m_{A}v_{A,f} \sin (\alpha)}{m_{B} \sin (\beta)} \right) \cos (\beta) \\
    v_{A,i} &= v_{A,f} \cos (\alpha) + \left( \frac{v_{A,f} \sin (\alpha)}{\sin (\beta)} \right) \cos (\beta)
  \end{align*}

  So, the answer for the inital velocity of Object A is:

  \begin{equation*}
    \vec{v}_{A,i} = \biggl \langle v_{A,f} \cos (\alpha) + \frac{v_{A,f}\sin (\alpha) \cos (\beta)}{\sin (\beta)}, 0 \biggr \rangle \si{\meter / \second}
  \end{equation*}
\end{example}

%%% Local Variables:
%%% mode: latex
%%% TeX-master: "../Phys_123-Mechanics-Reference_Sheet"
%%% End: