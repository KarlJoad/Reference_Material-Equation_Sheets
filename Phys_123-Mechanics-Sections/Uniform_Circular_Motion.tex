\section{Uniform Circular Motion}\label{sec:Uniform Circular Motion}
\begin{definition}[Uniform Circular Motion]\label{def:Uniform Circular Motion}
  \emph{Uniform circular motion} is when an object of moving in a perpetual circular motion.
  There is no outside source of acceleration changing the state of the system.
  \begin{remark}
    This does \emph{not} happen in real life.
    However, it is useful for modelling things under ideal conditions that do happen in real life.
  \end{remark}
\end{definition}

\subsection{Angular Part of Circular Motion}\label{subsec:Angular Circular Motion}
During \nameref{def:Uniform Circular Motion}, your terminology changes a little bit.
\begin{definition}[Angular Position]\label{def:Angular Position}
  \emph{Angular position} is determined with radians around a circle.
  It is denoted with \[ \vec{\theta} \]
\end{definition}

\begin{definition}[Angular Velocity]\label{def:Angular Velocity}
  \emph{Angular velocity} is orthogonal to the flat 2-D plane that the object is traveling in.
  It is the dervative of the \nameref{def:Angular Position}.

  \begin{equation}\label{eq:Angular Velocity}
    \begin{aligned}
      \vec{\omega} &= \frac{d \theta}{dt} \\
      &= \langle 0, 0, \omega \rangle \\
    \end{aligned}
  \end{equation}
\end{definition}

\begin{definition}[Angular Acceleration]\label{def:Angular Acceleration}
  \emph{Angular acceleration} is the derivative of the \nameref{def:Angular Velocity}.

  \begin{equation}\label{eq:Angular Acceleration}
    \vec{\alpha} = \frac{d \vec{\omega}}{dt} = \frac{d^{2} \vec{\theta}}{dt}
  \end{equation}

  \begin{remark}
    Note that under \nameref{def:Uniform Circular Motion}, by its very definition, there cannot be any acceleration on the object.
    Therefore, when an object is in uniform circular motion, $\vec{\alpha} = 0$.

    However, when the object is \textbf{\emph{NOT}} in \nameref{def:Uniform Circular Motion} the object is undergoing \nameref{def:Linear Acceleration}.
  \end{remark}
\end{definition}

\subsection{Linear Part of Circular Motion}\label{subsec:Linear Circular Motion}
\begin{definition}[Linear Position]\label{def:Linear Position}
  \emph{Linear position} relates the position of an object from the cartesian coordinate plane to the polar.
  This means that:

  \begin{equation}\label{eq:Linear Position}
    \begin{aligned}
      x &= r \cos ( \theta ) & y &= r \sin ( \theta )
    \end{aligned}
  \end{equation}
\end{definition}

\begin{definition}[Linear Velocity]\label{def:Linear Velocity}
  \emph{Linear velocity} relates the velocity of an object in a line to its \nameref{def:Angular Velocity}.

  \begin{equation}\label{eq:Linear Velocity}
    \begin{aligned}
      \vec{v} &= \frac{d \vec{r}}{dt} \\
      \frac{d \vec{r}}{dt} &= \biggl \langle \frac{dx}{dt}, \frac{dy}{dt}, 0 \biggr \rangle = \biggl \langle \frac{d}{dt} r \cos ( \theta ), \frac{d}{dt} r \sin ( \theta ), 0 \biggr \rangle \\
      &= \Bigl \langle -r \sin ( \theta ) \omega, r \cos ( \theta ) \omega, 0 \Bigr \rangle \\
      &= \vec{\omega} \cross \vec{r}
    \end{aligned}
  \end{equation}
\end{definition}

\begin{definition}[Linear Acceleration]\label{def:Linear Acceleration}
  \emph{Linear acceleration} is the derivative of \nameref{def:Linear Velocity}.
  It relates the acceleration of an object in a line is relative to its \nameref{def:Angular Acceleration}

  \begin{equation}\label{eq:Linear Acceleration}
    \begin{aligned}
      \frac{d \vec{a}}{dt} &= \frac{d \vec{v}}{dt} \\
      &= \langle -r \omega^{2} \cos ( \theta ), -r \omega^{2} \sin ( \theta ), 0 \rangle = \omega^{2} \langle -r \cos ( \theta ), -r \sin ( \theta ), 0 \rangle \left( \right) \\
      &= - \omega^{2} \vec{r}
    \end{aligned}
  \end{equation}
\end{definition}

\subsection{Relation Between the \nameref{subsec:Angular Circular Motion} and the \nameref{subsec:Linear Circular Motion}}\label{subsec:Relations Circular Motion}
There are a few equations that relate both the \nameref{subsec:Angular Circular Motion} and the \nameref{subsec:Linear Circular Motion}.
\paragraph{Velocity and Angular Velocity}\label{par:Relate Velocity Angular Velocity}
\begin{equation}\label{eq:Relate Velocity Angular Velocity}
  v = \omega r
\end{equation}

\paragraph{Acceleration and Angular Acceleration}\label{par:Relate Acceleration Angular Acceleration}
\begin{equation}
  \begin{aligned}
    a &= \omega^{2} r \\
    &= \frac{v^{2}}{r} \\
  \end{aligned}
\end{equation}

%%% Local Variables:
%%% mode: latex
%%% TeX-master: "../Phys_123-Mechanics-Reference_Sheet"
%%% End: