\section{Statistics}\label{sec:Statistics}
In applying probability models to real situations, we perform experiments and collect data to answer questions such as:
\begin{enumerate}[noitemsep, nolistsep]
\item What are the values of the parameters of the distribution of a random variable of interest?
  \begin{itemize}[noitemsep, nolistsep]
  \item Mean or Expected value
  \item Variance
  \end{itemize}

\item Is the data set consistent with some model?
  \begin{itemize}[noitemsep, nolistsep]
  \item Some assumed distribution, which must be true, otherwise the model is wrong.
  \end{itemize}

\item Is the data set consistent with some parameter value of the assumed value?
\end{enumerate}

\begin{definition}[Random Sample]\label{def:Random Sample}
  A \emph{random sample} is a set of $n$ \nameref{def:Random Variable, Full} or \nameref{def:Statistic} that are drawn with \nameref{def:Independent and Identically Distributed}.
  \begin{equation}\label{eq:Random Sample}
    \mathbf{X}_{n} = \left( X_{1},X_{2},\ldots,X_{n} \right)
  \end{equation}
  \begin{remark}
    This is \emph{similar} to the definition of a \nameref{def:Random Vector}.
    The difference here is that the values in a \nameref{def:Random Sample} must be \nameref{def:Independent and Identically Distributed} and \emph{may} be related to each other somehow.
  \end{remark}
  \begin{remark}[Random Sample Parameters]\label{rmk:Random Sample Parameters}
    These are an additional variable that is added onto the \nameref{def:Probability Density Function} or \nameref{def:Probability Mass Function}.
    When we were using these functions in the previous sections, these parameters were either constant or assumed to be constant.
    When considering these samples in \nameref{sec:Statistics}, you must also account for the \nameref{rmk:Random Sample Parameters}.
  \end{remark}
\end{definition}
\begin{definition}[Statistic]\label{def:Statistic}
  A \emph{statistic} $W (\mathbf{X}_{n})$ is a function of the random sample $X_{1},X_{2},\ldots,X_{n}$.
  \begin{equation}\label{eq:Statistic}
    W \left( \mathbf{X}_{n} \right) = g \left( X_{1},X_{2},\ldots,X_{n} \right)
  \end{equation}
\end{definition}
\begin{definition}[Unit Variance]\label{def:Unit Variance}
  The \emph{unit variance} means that the standard deviation, $\sigma$ of a sample, as well as the variance, $\sigma^{2}$ will tend towards 1 as the sample size increases to infinity.
\end{definition}

\subsection{Sample Mean}\label{subsec:Sample Mean}
\begin{definition}[Sample Mean]\label{def:Sample Mean}
  The \emph{sample mean} of a sequence is denoted as,
  \begin{equation}\label{eq:Sample Mean}
    \bar{X} = M_{n} = \frac{\sum_{i=1}^{n} X_{i}}{n}
  \end{equation}
\end{definition}
\begin{definition}[Expected Value of Sample Mean]\label{def:Expected Value of Sample Mean}
  The \emph{expected value of the sample mean} is defined as:
  \begin{equation}\label{eq:Expected Value of Sample Mean}
    \ExpectedValue \left[ \bar{X} \right]
    = \ExpectedValue \left[ M_{n} \right]
    = \frac{\ExpectedValue \left[ S_{n} \right]}{n}
    = \frac{n \mu}{n}
    = \mu
  \end{equation}
  \begin{remark}
    The sample mean $M_{n}$ is an \emph{\nameref{def:Unbiased Estimator}} of population mean $\mu$.
  \end{remark}
\end{definition}
\begin{definition}[Variance of Sample Mean]\label{def:Variance of Sample Mean}
  The \emph{variance of the sample mean} is denoted as:
  \begin{equation}\label{eq:Variance of Sample Mean}
    \Variance \left[ \bar{X} \right]
    = \Variance \left[ M_{n} \right]
    = \Variance \left[ \frac{S_{n}}{n} \right]
    = \frac{1}{n^{2}} \Variance \left[ S_{n} \right]
    = \frac{\sigma^{2}}{n}
  \end{equation}
  \begin{remark}
    The larger $n$ gets, the smaller $\Variance \left[ M_{n} \right]$ gets, and the closer $M_{n}$ gets to $\mu$.
  \end{remark}
\end{definition}

Also, we can use the \nameref{eq:Chebychev Inequality} to approximate many values. In this case, we change the Chebychev Inequality from \Cref{eq:Chebychev Inequality} to \Cref{eq:Statistics Chebychev Inequality} like so:
\begin{equation}\label{eq:Statistics Chebychev Inequality}
  P \left[ \lvert M_{n} - \ExpectedValue \left[ M_{n} \right] \rvert \geq \varepsilon \right] \leq \frac{\Variance \left[ M_{n} \right]}{\varepsilon^{2}}
\end{equation}
\begin{example}[Problem 7.15]{Chebychev Inequality to Bound Probability}
  Suppose that the number of particle emissions by a radioactive mass in $t$ seconds is a Poisson random variable with mean $\lambda t$.
  Use the Chebychev inequality to obtain a boind for the probability that $\lvert \frac{N \left( t \right)}{t} - \lambda \rvert$ exceeds $\varepsilon$.

  \tcblower

  Solution to Problem 7.15 from Homework 11 (Extra Credit).
\end{example}

\subsection{Important Probability and Statistics Theorems}\label{subsec:Important Probability and Statistics Theorems}
There are 3 very import theorems that are used quite frequently in both \nameref{sec:Probability Theory} and \nameref{sec:Statistics}.
\begin{enumerate}[noitemsep, nolistsep]
\item \nameref{thm:Weak Law of Large Numbers}
\item \nameref{thm:Strong Law of Large Numbers}
\item \nameref{thm:Central Limit Theorem}
\end{enumerate}
\begin{theorem}[Weak Law of Large Numbers]\label{thm:Weak Law of Large Numbers}
  Let $X_{1},X_{2},\ldots,X_{n}$ be a sequence of \nameref{def:Independent and Identically Distributed} random variables form a population with mean $\ExpectedValue \left[ X \right] = \mu$, then for $\varepsilon > 0$,
  \begin{equation}\label{eq:Weak Law of Large Numbers}
    \lim\limits_{n \to \infty} P \left[ \lvert M_{n} - \mu \rvert < \varepsilon \right] = 1
  \end{equation}
  \begin{remark*}
    In words this means, for large enough fixed values of $n$, $M_{n}$ is close to $\mu$ with high probability.
  \end{remark*}
\end{theorem}
\begin{theorem}[Strong Law of Large Numbers]\label{thm:Strong Law of Large Numbers}
  Let $X_{1},X_{2},\ldots,X_{n}$ be a sequence of \nameref{def:Independent and Identically Distributed} random variables form a population with mean $\ExpectedValue \left[ X \right] = \mu$ and finite variance, then
  \begin{equation}\label{eq:Strong Law of Large Numbers}
    P \left[ \lim\limits_{n \to \infty} M_{n} = \mu \right] = 1
  \end{equation}
  \begin{remark*}
    With probability 1, every sequence of sample mean calculations will eventually approach and stay close to the population mean.
  \end{remark*}
\end{theorem}
\begin{theorem}[Central Limit Theorem]\label{thm:Central Limit Theorem}
  Let $X_{1},X_{2},\ldots,X_{n}$ be a sequence of \nameref{def:Independent and Identically Distributed} random variables form a population with mean $\ExpectedValue \left[ X \right] = \mu < \infty$ and finite variance $\sigma^{2}$ and let
  \begin{equation*}
    Z_{n} = \frac{S_{n} - n\mu}{\sigma \sqrt{n}} = \frac{\bar{X} - \mu}{\frac{\sigma}{\sqrt{n}}}
  \end{equation*}
  then,
  \begin{equation}\label{eq:Central Limit Theorem}
    \lim\limits_{n \to \infty} P \left[ Z_{n} \leq z \right] = \frac{1}{\sqrt{2 \pi}} \int_{-\infty}^{z} e^{-\frac{x^{2}}{2}} dx
  \end{equation}
  \begin{itemize} % I preferred the way it looked without noitemsep/nolistsep
  \item $S_{n}=X_{1}+X_{2}+\ldots+X_{n}$, The sum of the random variables
  \item $\mu=\ExpectedValue \left[ X_{1} \right]$, The mean for an individual random variable
  \item $\sigma=\sqrt{\Variance \left[ X_{1} \right]}$, The variance of an individual random variable
  \item $n$ is the number of trials/recordings/samples/etc.
  \item $\bar{X} = \frac{1}{n} \sum\limits_{i=1}^{n} X_{i}$, The \nameref{def:Sample Mean}
  \end{itemize}
  \begin{remark*}
    This means that over time, as you gain more and more sample means, they will start to resemble the \nameref{def:Gaussian Random Variable}, or the Normal Random Variable.
  \end{remark*}
\end{theorem}
\begin{example}[Problem 7.25]{Central Limit Theorem}
  The lifetime of a cheap light bulb is an exponential random variable with mean 36 hours.
  Suppose that 16 light bulbs are tested and their lifetimes measured.
  Use the \nameref{thm:Central Limit Theorem} to estimate the probability that the sum of the lifetimes is less than 600 hours.

  \tcblower

  Solution to Problem 7.25 from Homework 11 (Extra Credit).
\end{example}

\subsection{Estimators}\label{subsec:Estimators}
\begin{itemize}[noitemsep, nolistsep]
\item A \nameref{def:Statistic} is a function of the data $X_{1},X_{2},\ldots,X_{n}$
\item An \emph{estimator} for a parameter, $\theta$, usually denoted $\hat{\theta}$, is also a statistic
\end{itemize}
\begin{definition}[Unbiased Estimator]\label{def:Unbiased Estimator}
  In general we say that a \nameref{def:Statistic} $\Theta (X)$ (a function of data $X_{1},X_{2},\ldots,X_{n}$) is an \emph{unbiased estimator} of a parameter $\theta$ if $\ExpectedValue \left[ W \left( \mathbf{X} \right) \right] = \theta$.
  \begin{remark}[What makes a good estimator of any parameter, $\theta$?]
    A \emph{good estimator} of any parameter, $\theta$, should:
    \begin{itemize}[noitemsep, nolistsep]
    \item Give the correct value of $\theta$
    \item Not vary too much around $\theta$
    \end{itemize}
  \end{remark}
  \begin{remark}
    This is the definition of \emph{unbiased}, drawn from the definition of \nameref{def:Bias}
  \end{remark}
\end{definition}

\subsubsection{Goodness of an Estimator}\label{subsubsec:Estimator Goodness}
There are 4 measures we use to determine how good our estimator is.
\begin{enumerate}[noitemsep, nolistsep]
\item \nameref{def:Bias}
\item \nameref{def:Variance of Sample Mean}
\item \nameref{def:Mean Squared Error}
\item \nameref{def:Consistency}
\end{enumerate}
If our estimator is an \nameref{def:Unbiased Estimator}, then:
\begin{itemize}
\item Accuracy is defined as $\Bias [\hat{\theta} ] = \ExpectedValue [\hat{\theta}] - \theta$
\item Precision is defined as $\Variance [\hat{\theta}]$
\end{itemize}

\begin{definition}[Bias]\label{def:Bias}
  \emph{Bias} is defined as:
  \begin{equation}\label{eq:Accuracy}
    \Bias [ \hat{\Theta} ] = \ExpectedValue [ \hat{\Theta} ] - \theta
  \end{equation}
  \begin{remark}\label{rmk:Unbiased}
    The estimator $\hat{\Theta}$ is \emph{unbiased} for $\theta$ if
    \begin{equation}\label{Unbiased}
      \ExpectedValue [ \hat{\Theta} ] = \theta
    \end{equation}
  \end{remark}
\end{definition}
\begin{example}[Problem 8.14]{Show Bias in Estimator}
  The output of a communication system is $Y = \theta + N$, where $\theta$ is an input signal and $N$ is a noise signal that is uniformly distributed in the interval $\left[ 0,2 \right]$.
  Suppose the signal is transmitted $n$ times and that the noise terms are iid random variables.
  Show that the sample mean of the outputs is a biased estimator for $\theta$.

  \tcblower

  Solution to Problem 8.14, Part a from Homework 11 (Extra Credit).
\end{example}

\begin{definition}[Mean Squared Error]\label{def:Mean Squared Error}
  The \emph{Mean Squared Error} of an estimator for parameter $\hat{\theta}$ is:
  \begin{equation}\label{eq:Mean Squared Error}
    \MeanSqErr [ \hat{\theta} ]
    = \ExpectedValue \left[ \left( \hat{\Theta} - \theta \right)^{2} \right]
    = \Variance \left[ \hat{\theta} \right] + \left( \Bias \left[ \hat{\theta} \right] \right)^{2}
  \end{equation}
\end{definition}
\begin{example}[Problem 8.14]{Find Mean Square Error of Estimator}
  The output of a communication system is $Y = \theta + N$, where $\theta$ is an input signal and $N$ is a noise signal that is uniformly distributed in the interval $\left[ 0,2 \right]$.
  Suppose the signal is transmitted $n$ times and that the noise terms are iid random variables.
  Find the \nameref{def:Mean Squared Error} of the estimator.

  \tcblower

  Solution to Problem 8.14, Part b from Homework 11 (Extra Credit).
\end{example}

\begin{remark*}
  When doing statistical analysis, there is something called the \emph{Bias-Variance Tradeoff}.
  When doing the analysis, if you try to minimize bias, your variance will increase and vice-versa.
  There is a happy medium, which is not discussed in this class.
\end{remark*}

\begin{definition}[Consistency]\label{def:Consistency}
  $\hat{\theta}$ is a \emph{consistent estimator} for $\theta$ if $\hat{\Theta}$ converges to $\theta$ in probability.
  \begin{equation}\label{eq:Consistency}
    \lim\limits_{n \to \infty} \Prob \left[ \lvert \hat{\Theta} - \theta \rvert > \varepsilon \right] = 0
  \end{equation}
\end{definition}
\begin{example}[Problem 8.17]{Confirm Consistency of Estimator}
  To extimate the variance of a Bernoilli random variable $X$, we perform $n$ iid trials and count the number of successes $k$ and obtain the estimate $\hat{p} = \frac{k}{n}$.
  We then estimate the variance of $X$ by
  \begin{equation*}
    \hat{\sigma}^{2} = \hat{p} \left( 1-\hat{p} \right) = \frac{k}{n} \left( 1- \frac{k}{n} \right)
  \end{equation*}
  Is $\hat{\sigma}^{2}$ a consistent estimator for the variance of $X$?

  \tcblower

  Solution to Problem 8.17, Part b from Homework 11 (Extra Credit).
\end{example}

\subsection{How to Find a Good Estimator}\label{subsec:Find Good Estimator}
There are several methods, two of which are:
\begin{enumerate}[noitemsep, nolistsep]
\item \nameref{subsubsec:Method of Moments}
  \begin{itemize}[noitemsep, nolistsep]
  \item Sample Moments and Population Moments, $\bar{X}_{n} = \mu$
  \item You needs as many moments as parameters to get enough equations
  \end{itemize}
\item \nameref{def:Maximum Likelihood Estimation}
\end{enumerate}

\subsubsection{Method of Moments}\label{subsubsec:Method of Moments}
\begin{example}{Method of Moments}
  Let the sample $X_{1}, X_{2}, \ldots, X_{n}$ consist of iid version of the random variable $X$. The method of moments involves estimating the moments of $X$ as follows:
  \begin{equation*}
    \hat{m}_{k} = \frac{1}{n} \sum\limits_{j=1}^{n} X_{j}^{k}
  \end{equation*}
  \begin{boldalphlist}
  \item Suppose that $X$ is a uniform random variable in the interval $\left[ 0, \theta \right]$. Use $\hat{m}_{1}$ to find an estimator for $\theta$.\label{prob:8.6a}
  \item Find the mean and variance of the estimator in part \ref{prob:8.6a}.
  \end{boldalphlist}

  \tcblower

  Solution to Problem 8.6 from Homework 11 (Extra Credit).
\end{example}

\subsubsection{Maximum Likelihood Estimation}\label{subsubsec:Maximum Likelihood Estimation}
\begin{definition}[Maximum Likelihood Estimation]\label{def:Maximum Likelihood Estimation}
  Let $ X_{1},X_{2},\ldots,X_{n} \DrawnIID f \left( x \Given \theta \right) $.
  \begin{equation}\label{eq:Maximum Likelihood Estimation}
    \hat{\Theta}_{\MaxLikeEstim} = \argmax_{\theta \in \Theta} \Likelihood \left( \theta \Given x_{1},x_{2},\ldots,x_{n} \right)
  \end{equation}
  \begin{remark}\label{rmk:Likelihood}
    Likelihood, denoted $\Likelihood$ is defined as the \nameref{def:Joint PDF} of the \nameref{def:Random Sample} and its \nameref{rmk:Random Sample Parameters}.
    \begin{equation}\label{eq:Likelihood}
      \Likelihood \left( \theta \Given x_{1},x_{2},\ldots,x_{n} \right) = f_{X_{1}} \left( x_{1} \Given \theta \right) \cdot f_{X_{2}} \left( x_{2} \Given \theta \right) \cdot \ldots \cdot f_{X_{n}} \left( x_{n} \Given \theta \right)
    \end{equation}
  \end{remark}
  \begin{remark}
    It is often easier to maximize $\hat{\Theta}_{\MaxLikeEstim}$ over the log-likelihood.
    \begin{equation*}
      \hat{\Theta}_{\MaxLikeEstim} = \argmax_{\theta \in \Theta} \log \Likelihood \left( \theta \Given x_{1},x_{2},\ldots,x_{n} \right)
    \end{equation*}
  \end{remark}
  \begin{remark}
    \[ \argmax_{\theta \in \Theta} \] is the global maxima of the function. This is further described in \Cref{def:argmax}.
  \end{remark}
\end{definition}

\subsection{Confidence Intervals}\label{subsec:Confidence Interval}
Because certain estimators are not discrete, but continuous, the estimators expected value might not be quite right.
This is where \nameref{def:Confidence Interval}s come in.
\begin{definition}[Confidence Interval]\label{def:Confidence Interval}
  A \emph{confidence interval} is an interval or set of values that is highly likely to contain the true value of the parameter.
\end{definition}
\begin{definition}[$1-\alpha$ Confidence Interval]\label{def:1-alpha Confidence Interval}
  The \emph{$1-\alpha$ confidence interval} is a \nameref{def:Confidence Interval} where we estimate the probability of the parameter, $\theta$, being in the random interval to $1-\alpha$.
  The problem is, find a random interval $\left[ \ell \left( \mathbf{X} \right), u \left( \mathbf{X} \right) \right]$ such that
  \begin{equation}\label{eq:1-alpha Confidence Interval}
    \Prob \left[ \ell \left( \mathbf{X}_{n} \right) , u \left( \mathbf{X}_{n} \right) \right] = 1-\alpha
  \end{equation}
  \begin{remark}
    $\ell \left( \mathbf{X} \right)$ is the \emph{lower bound of the random interval}.
    $u \left( \mathbf{X} \right)$ is the \emph{upper bound of the random interval}.
    \textbf{Only ONE of these may be $\infty$ at a time.} Otherwise, you're including the whole sample space, making the confidence interval useless.
  \end{remark}
  \begin{remark}
    If the problem says, with a confidence interval of 95\%, that is the $1-\alpha$ portion; i.e. $1-\alpha = 0.95 \%$.
  \end{remark}
\end{definition}
\begin{example}{Confidence Intervals}
  Let $X_{1}, X_{2}, \ldots, X_{n} \DrawnIID N \left( \mu, \sigma^{2} \right)$.
  The population mean, $\mu$ is unknown.
  The population variance, $\sigma^{2}$ is konwn.
  What is a 95\% \nameref{def:Confidence Interval} for the population mean, $\mu$?

  \tcblower

  Solution in Example 1 from Lecture 28.
\end{example}

\subsection{Hypothesis Testing}\label{subsec:Hypothesis Testing}
\begin{definition}[Hypothesis Testing]\label{def:Hypothesis Testing}
  There are 4 parts to \emph{hypothesis testing}.
  \begin{enumerate}[noitemsep, nolistsep]
  \item Identify the hypotheses. $H_{0}$ is the \emph{null hypothesis}. It is usually compared against another, complementary hypothesis, $H_{1}$.
  \item The Rejection Region, which is evidence against the null hypothesis that we may or may not choose to include. $R \subset \mathcal{X}$, where $\mathcal{X}$ is the sample space.
  \item The Decision Rule:
    \begin{enumerate}[noitemsep, nolistsep]
    \item Reject $H_{0}$ if $X \in R$
    \item Accept/Do not reject $H_{0}$ if $X \notin R$
    \end{enumerate}
  \item The Test Statistic
  \end{enumerate}
  \begin{remark}
    There are 2 errors that can occur in \nameref{def:Hypothesis Testing}.
    \begin{enumerate}[label=\textbf{Type \Roman* Error: }, ref=Hypothesis Testing Type \Roman* Error, align=left, noitemsep, nolistsep]
    \item Reject $H_{0}$ if $H_{0}$ is true.
    \item Accept $H_{0}$ if $H_{0}$ is false.
    \end{enumerate}
  \end{remark}
\end{definition}

\begin{example}{Use Hypothesis Testing}
  Using \nameref{def:Hypothesis Testing}, how can you determine if a coin is fair?

  \tcblower

  Solution from Example 2 from Lecture 28.
\end{example}
%%% Local Variables:
%%% mode: latex
%%% TeX-master: "../Math_374-Prob_Stats-Reference_Sheet"
%%% End:
