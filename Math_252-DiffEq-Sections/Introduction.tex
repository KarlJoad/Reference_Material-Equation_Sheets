\section{Introduction} \label{sec:Introduction}
This section will introduce the basic terminology and definitions for solving ordinary differential equations.
\subsection{Definitions and Terminology} \label{subsec:Definitions and Terminology}
\begin{definition}[Differential Equation] \label{def:Differential Equation}
  A \emph{differential equation (DE)} is an equation with 1 or more derivatives.
  \begin{remark}
    The highest differential determines the order of the differential equation.
    This means that the differential equation below is of order 2.
    \begin{align*}
      y'' + y &= 0 \\
      \frac{d^{2}y}{dx^{2}} + y &= 0 \\
    \end{align*}
  \end{remark}
\end{definition}
\begin{definition}{Initial Value Problem} \label{def:Initial Value Problem}
  A differential equation with one or more initial conditions is called an \emph{initial value problem (IVP)}.
  \begin{remark}
    To solve an initial value problem, you must have the same number of initial conditions as the order of the differential equation.
  \end{remark}
  \begin{remark}[Existence of Unique Solution]
    $R$ is a rectangular region on the xy-plane $a \leq x \leq b$, $c \leq y \leq d$ that contains $\left( x_{0}, y_{0} \right)$ interior.
    If $f \left( x,y \right)$ and $\frac{df}{dy}$ are continuous on $R$, then an interval exists $I_{0}$ such that $\left( x_{0}-h, x_{0}+h \right)$ where $h>0$, on the interval $\left[ a,b \right]$, and a unique function $y \left( x \right)$, defined on $I_{0}$ that is a solution of the initial value problem.
  \end{remark}
\end{definition}

\subsection{Confirm If Differential Equation} \label{subsec:Confirm Differential Equation}
You can confirm if the solution $y \left( x \right)$ found for a differential equation $y \left( x \right)'$ is the solution by differentiating the solution and putting that in the solved differential equation and verfiying that the equation holds true.
This is shown in \Cref{ex:Confirm Differential Solution}.
\begin{example}[]{Confirm Differential Solution}
  Given the differential equation, $2y' + y = 0$, is $y = e^{\frac{-x}{2}}$ a solution? \newline

  \tcblower

  \begin{align*}
    y' &= \frac{-1}{2} e^{\frac{-x}{2}} \\
    2 \left( \frac{-1}{2} e^{\frac{-x}{2}} \right) + \left( e^{\frac{-x}{2}} \right) &= 0 \\
    -e^{\frac{-x}{2}} + e^{\frac{-x}{2}} &= 0 \\
    0 &= 0 \text{ \checkmark}
  \end{align*}
\end{example}

\subsection{Separable Differential Equation} \label{subsec:Separable Differential Equation}
\begin{definition}[Separable] \label{def:Separable}
  A \emph{separable} differential equation allows you to move various elements around to solve the equation.
  For example,
  \begin{align*}
    \frac{dP}{dt} &= kP \\
    \frac{1}{P} dP &= k dt \\
    \ln \left( P \right) &= kt + C \\
    P &= Ce^{kt}
  \end{align*}
  \begin{remark}
    These are used extensively in modelling phenomena with differential equations.
    These include: \nameref{subsubsec:Population Growth}, \nameref{subsubsec:Radioactive Decay}, \nameref{subsubsec:Newton Law of Cooling/Heating}, and \nameref{subsubsec:Spread of Disease}.
  \end{remark}
\end{definition}

\subsection{Modeling with Differential Equations} \label{subsec:Modeling with DEs}
\subsubsection{Population Growth} \label{subsubsec:Population Growth}
\begin{definition}[Population Growth]
  \emph{Population growth} can be modelled with a separable differential equation. Namely,
  \begin{equation} \label{eq:Population Growth}
    \frac{dP}{dt} = kP
  \end{equation}
  \begin{remark}[Population Growth Parameters] \label{rmk:Population Growth Parameters}
    The parameters for the \nameref{eq:Population Growth}~equation are given below.
    \begin{itemize}[noitemsep, nolistsep]
    \item $k>0$
    \item $P>0$
    \end{itemize}
  \end{remark}
\end{definition}
\subsubsection{Radioactive Decay} \label{subsubsec:Radioactive Decay}
\begin{definition}[Radioactive Decay] \label{def:Radioactive Decay}
  \emph{Radioactive decay} is the process that some particularly heave atoms undergo to become lighter, more stable atoms.
  \begin{definition}[Half-Life] \label{subdef:Half-Life}
    The \emph{half-life} is the usual reported metric, and is defined as the amount of time required for an element to half its mass through \nameref{def:Radioactive Decay}.
  \end{definition}
  \begin{equation} \label{eq:Radioactive Decay}
    \frac{1}{2} A_{0} = A_{0} e^{kt}
  \end{equation}
  \begin{remark}[Radioactive Decay Parameters] \label{rmk:Radioactive Decay Parameters}
    The parameters for the \nameref{eq:Radioactive Decay}~equation are given below.
    \begin{itemize}[noitemsep, nolistsep]
    \item $k<0$
    \item $A>0$
    \end{itemize}
  \end{remark}
\end{definition}
\subsubsection{Newton's Law of Cooling/Heating} \label{subsubsec:Newton Law of Cooling/Heating}
\begin{definition}[Newton's Law of Cooling/Heating] \label{def:Newton Law of Cooling/Heating}
  \emph{Newton's Law of Cooling/Heating} is the same equation, but some of the parameters change.
  This equation is defined as:
  \begin{equation} \label{eq:Newton Law of Cooling/Heating}
    \frac{dT}{dt} = k \left( T-T_{m} \right)
  \end{equation}
  \begin{remark} \label{rmk:Newton Law of Cooling/Heating Parameters}
    The parameters for the \nameref{eq:Newton Law of Cooling/Heating}~equation are given below.
    \begin{itemize}[noitemsep, nolistsep]
    \item $\frac{dT}{dt}$; The rate of change of temperature in the object per unit time.
    \item $k<0$; The cooling constant and is unique to every object.
    \item $T$; The starting temperature.
    \item $T_{m}$; The temperature of the surrounding medium.
    \end{itemize}
  \end{remark}
\end{definition}
\subsubsection{Spread of Disease} \label{subsubsec:Spread of Disease}
\begin{definition}[Spread of Disease] \label{def:Spread of Disease}
  This is used to model the spread of something throughout a society or group of people.
  \begin{equation} \label{eq:Spread of Disease}
    \frac{dx}{dt} = kxy
  \end{equation}
  \begin{remark}
    The parameters for the \nameref{eq:Spread of Disease}~equation are given below.
    \begin{itemize}[noitemsep, nolistsep]
    \item $\frac{dx}{dt}$; Change in the number of infected per unit time.
    \item $k<0$; Transmission Constant
    \item$x$; Number of Infected
    \item $y$; Number of non-infected, $y$ is really a function of $x$
      \begin{itemize}[noitemsep, nolistsep]
      \item $y = n+1-x$
      \end{itemize}
    \end{itemize}
  \end{remark}
\end{definition}

\subsubsection{Chemical Reactions} \label{subsubsec:Chemical Reactions}
\begin{definition}[Chemical Reactions] \label{def:Chemical Reactions}
  These model how molecules interact in certain proportions to achieve some resultant molecule.
  \begin{equation} \label{eq:Chemical Reactions}
    \frac{dx}{dt} = k \left( \alpha - x \right) \left( \beta - x \right)
  \end{equation}
  \begin{remark}
    The parameters for the \nameref{def:Chemical Reactions}~equation are given below.
    \begin{itemize}[noitemsep, nolistsep]
    \item $x$; Amount of resultant chemical
    \item $k$; Reaction rate, must be greater than 0, $k > 0$
    \item $\frac{dx}{dt}$; Rate of creation of resultant molecule per unit time
    \item $\alpha$; Initial amount of Chemical ``A''
    \item $\beta$; Initial amount of Chemical ``B''
    \item $x \left( 0 \right) = 0$; Initial amount of resultant molecule must be 0 at the start
    \end{itemize}
  \end{remark}
\end{definition}

\subsubsection{Tank Mixture} \label{subsubsec:Tank Mixture}
\begin{definition}[Tank Mixture] \label{def:Tank Mixture}
  A well-mixed dissolved influent ``thing'' is brought into a tank and drained at some rate.
  What is the change in the amount of dissolved ``thing'' at any point in time?
  \begin{equation}\label{eq:Tank Mixture}
    \frac{dA}{dt} = R_{\text{in}} - R_{\text{out}}
  \end{equation}
  \begin{remark}
    The parameters for the \nameref{def:Tank Mixture}~equation are given below.
    \begin{itemize}[noitemsep, nolistsep]
    \item $A$; The amount of dissolved ``thing''
    \item $t$; The time of time the tank has taken
    \item $R_{\text{in}}$; The rate of dissolved ``thing'' into the tank
    \item $R_{\text{out}}$; The rate of dissolved ``thing'' out of the tank
    \end{itemize}
  \end{remark}
\end{definition}

\subsubsection{Torricelli's Law} \label{subsubsec:Torricelli's Law}
\begin{definition}[Torricelli's Law] \label{def:Torricelli's Law}
  This equation relates the rate the volume in a tank changes to the height of the water to the hole in the tank.
  \begin{equation} \label{eq:Torricelli's Law}
    \frac{dV}{dt} = -A_{h} \sqrt{2gh}
  \end{equation}
  \begin{remark}
    The parameters for the \nameref{def:Torricelli's Law}~ equation are given below.
    \begin{itemize}[noitemsep, nolistsep]
    \item $V = A_{w}h$; The volume of water above the hole
    \item $\frac{dV}{dt} = A_{w} \frac{dh}{dt}$; The change in the volume of the water above the hole
    \item $h$; Height of the water
    \item $A_{h}$; Width of the hole
    \item $A_{w}$; Cross-sectional area of the tank
    \end{itemize}
  \end{remark}
\end{definition}

\subsubsection{LRC Circuits} \label{subsubsec:LRC Circuits}
\begin{definition}[LRC Circuits] \label{def:LRC Circuits}
  An \emph{LRC Circuit} is analyzed in terms of the energy moving through the circuit.
  There is a unique relationship for the energy in each element:
  \begin{equation} \label{eq:Energy in Capacitor}
    E \left( t \right) = \frac{q}{C}
  \end{equation}
  \begin{equation} \label{eq:Energy in Resistor}
    E \left( t \right) = RI = R \frac{dq}{dt}
  \end{equation}
  \begin{equation} \label{eq:Energy in Inductor}
    E \left( t \right) = L \frac{dI}{dt} = L \frac{d^{2}q}{dt^{2}}
  \end{equation}
  \begin{remark}
    Depending on the circuit given, you might use a combination of these, but you \emph{\textbf{must}} have at least one capacitor or inductor, otherwise it is not a differential equation.
  \end{remark}
  \begin{remark}
    These equations \emph{add} together when the entire circuit is in series, i.e. the elements are put together back-to-back.
  \end{remark}
\end{definition}

\subsection{Linear and Non-Linear Differential Equations} \label{Linear vs. Non-Linear Differential Equations}
\begin{definition}[Linear Differential Equation] \label{def:Linear Differential Equation}
  A \emph{linear differential equation} is one that satisfies one of the following equations below.
  \begin{equation} \label{eq:Linear Differential Equation}
    \begin{aligned}
      a_{1} \left( x \right) \frac{dy}{dx} + a_{0} \left( x \right) &= g \left( x \right) \\
      a_{2} \left( x \right) \frac{d^{2}y}{dx^{2}} + a_{1} \left( x \right) \frac{dy}{dx} + a_{0} \left( x \right) &= g \left( x \right) \\
    \end{aligned}
  \end{equation}
  \begin{remark}
    The equations in Equation~\eqref{eq:Linear Differential Equation} can be generalized to the $n$th order as shown below.
    \begin{equation} \label{eq:General Linear Differential Equation}
      a_{n} \left( x \right) \frac{d^{n}y}{dx^{n}} + a_{n-1} \left( x \right) \frac{d^{n-1}y}{dx^{n-1}} + \ldots + a_{1} \left( x \right) \frac{dy}{dx} + a_{0} \left( x \right) = g \left( x \right)
    \end{equation}
  \end{remark}
\end{definition}
\begin{definition}[Non-Linear] \label{def:Non-Linear Differential Equation}
  A \emph{non-linear} differential equation is one that does not satisfy the definition of a \nameref{def:Linear Differential Equation}.
  It does not obey Equation~\eqref{eq:General Linear Differential Equation}.
\end{definition}

%%% Local Variables:
%%% mode: latex
%%% TeX-master: "../Math_252-DiffEq-Reference_Sheet"
%%% End: