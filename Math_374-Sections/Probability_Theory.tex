\section{Probability Theory} \label{sec:Probability Theory}
There are 3 main components to \nameref{sec:Probability Theory}.
\begin{enumerate}[noitemsep, nolistsep]
	\item \nameref{sec:Set Theory}
	\item \nameref{subsec:Probability Law Corollary}
	\item \nameref{subsec:Conditional Probability}~and~\nameref{subsec:Event Independence}
\end{enumerate}

	\subsection{Random Experiments} \label{subsec:Random Experiments}
	\begin{definition}[Random Experiment] \label{def:Random Experiment}
		A \emph{random experiment} is an experiment whose outcome varies in an unpredictable fashion when performed under the same conditions.
	\end{definition}
	\begin{definition}[Sample Space] \label{def:Sample Space}
		A \emph{sample space, $S$} of a random experiment is the set of all possible experiments.
	\end{definition}
	\begin{definition}[Outcome/Sample Point] \label{def:Outcome}
		An \emph{outcome}, or \emph{sample point} of a random experiment is a result that cannot be decomposed into other results.
	\end{definition}
	\begin{definition}[Event] \label{def:Event}
		An \emph{event} corresponds to a subset of the sample space. We say an event occurs if and only if (iff) the outcome of the experiment is in the subset representing the event.
	\end{definition}
	\begin{definition}[Event Classes] \label{def:Event Classes}
		An \emph{event class} $\EventClass$ is the collection of the all the events' sets. $\EventClass$ should be closed under unions, intersections, and complements.
		\begin{itemize}[noitemsep, nolistsep]
			\item For $S$ finite, or countably infinite, then we can let $\EventClass$ be all subsets of $S$.
			\item For $S$ uncountably infinite, instead we can let $\EventClass$ consist of the subsets that can be obtained as countable unions and intersections of some sets of $\EventClass$.
		\end{itemize}
	\end{definition}
	\begin{definition}[Probability Law] \label{def:Probability Law}
		A \emph{probability law} for a random experiment $E$, with sample space $S$, and an event class $\EventClass$ is a rule that assigns to each event $A \in \EventClass$ a number $P \left[A \right]$, called the probability of $A$ that satisfies the axioms:
		\begin{enumerate}[label=Axiom~\Roman*:, align=left, noitemsep, nolistsep] \label{subdef:Probability Law Axioms}
			\item $0 \leq P\left[ A \right]$
			\item $P \left[ S \right] = 1$
			\item If $A \cap B = \emptyset$, then $P \left[ A \cup B \right] = P \left[ A \right] + P \left[ B \right]$
			\item[Axiom III':] If $A_{1}$, $A_{2}$, $\ldots$ is a sequence of events such that $A_{i} \cap A_{j} = \emptyset$ for all $i \neq j$, then $P \left[ \bigcup_{k=1}^{\infty} A_{k} \right] = \sum_{k=1}^{\infty} P \left[ A_{k} \right]$
		\end{enumerate}
	\end{definition}
	
	\subsection{Probability Law Corollaries} \label{subsec:Probability Law Corollary}
		\begin{enumerate}[label=Axiom~\Roman*:, align=left, noitemsep, nolistsep] % Probability Law Axioms
			\item $0 \leq P\left[ A \right]$
			\item $P \left[ S \right] = 1$
			\item If $A \cap B = \emptyset$, then $P \left[ A \cup B \right] = P \left[ A \right] + P \left[ B \right]$
			\item[Axiom III':] If $A_{1}$, $A_{2}$, $\ldots$ is a sequence of events such that $A_{i} \cap A_{j} = \emptyset$ for all $i \neq j$, then $P \left[ \bigcup_{k=1}^{\infty} A_{k} \right] = \sum_{k=1}^{\infty} P \left[ A_{k} \right]$
		\end{enumerate}
		\begin{corollary} \label{cor:Probability Parts}
			$P \left[ A^{C} \right] = 1 - P \left[ A \right]$
		\end{corollary}
		\begin{corollary} \label{cor:Probability of Event}
			$P \left[ A \right] \leq 1$
		\end{corollary}
		\begin{corollary} \label{cor:Probability of Empty Set}
			$P \left[ \emptyset \right] = 0$
		\end{corollary}
		\begin{corollary} \label{cor: Probability Addition of Disjoint Pairs}
			If $A_{1}$, $A_{2}$, $\ldots$, $A_{n}$ are pairwise mutually exclusive ($A_{1} \cap A_{2} \cap \ldots \cap A_{n} = \emptyset$), then $P \left[ \bigcup_{k=1}^{n} \right] = \sum_{k=1}^{n} P \left[ A_{k} \right]$ for $n \geq 2$
		\end{corollary}
		\begin{corollary} \label{cor:Inclusion-Exclusion Principle to 2 Sets}
			$P \left[ A \cup B \right] = P \left[ A \right] + P \left[ B \right] - P \left[ A \cap B \right]$
		\end{corollary}
		\begin{corollary} \label{cor:Inclusion-Exclusion Principle to n Sets}
			$P \left[ A \cup B \right] = \sum\limits_{j=1}^{n} P \left[ A_{j} \right] - \sum_{j<k} P \left[A_{j} \cap A_{k} \right] + \ldots + \left( -1 \right)^{n+1} P \left[ A_{1} \cap \ldots \cap A_{n} \right]$
		\end{corollary}
		\begin{corollary} \label{cor:Subset Probability to Superset}
			If $A \subset B$, then $P \left[ A \right] \leq P \left[ B \right]$
		\end{corollary}
	
	\subsection{Conditional Probability} \label{subsec:Conditional Probability}
		\begin{definition}[Conditional Probability] \label{def:Conditional Probability}
			The \emph{conditional probability} of event $A$ \textbf{GIVEN THAT} event $B$ occurred is denoted $P \left[ A \vert B \right]$ and is defined as
			\begin{equation} 
				P \left[ A \vert B \right] = \frac{P \left[ A \cap B \right]}{P \left[ B \right]}
			\end{equation}
		\end{definition}
		\begin{theorem}[Theorem of Total Probability]
			Let $B_{1}$, $B_{2}$, $\ldots$, $B_{n}$ be mutually exclusive events whose union equals the sample space $S$, i.e. $B_{1}$, $B_{2}$, $\ldots$, $B_{n}$ is a \emph{partition of $S$}.
		\end{theorem}
		\begin{definition}[Baye's Rule]
			Let $B_{1}$, $B_{2}$, $\ldots$, $B_{n}$ be a partition of sample space $S$.
			\begin{equation}
				P \left[ B_{j} \vert A \right] = \frac{P \left[ A \cap B_{j} \right]}{P \left[ A \right]}
				= \frac{P \left[ A \vert B_{j} \right] * P \left[ B_{j} \right]}{\sum\limits_{k=1}^{n} P \left[ A \vert B_{k} \right] * P \left[ B_{k} \right]}
			\end{equation}
		\end{definition}
	
	\subsection{Event Independence} \label{subsec:Event Independence}
		\begin{definition}[Independent] \label{def:Event Independence}
			Two events $A$ and $B$ are \emph{independent} if 
			\begin{equation} \label{eq:Event Independence}
				P \left[ A \cap B \right] = P \left[ A \right] * P \left[ B \right], P\left[ A \right] \neq 0, P\left[ B \right] \neq 0
			\end{equation}
			\begin{itemize}[noitemsep, nolistsep]
				\item If $A \cap B = \emptyset$, the $A$ and $B$ are \textbf{dependent}.
				\item If checking for independence between more than 2 events, you must check each pair, each triple, etc. until you check the independence of each event against each other. For 3 events, $A$, $B$, $C$:
					\begin{itemize}[noitemsep, nolistsep]
						\item Check $P \left[ A \cap B \cap C \right] = P \left[ A \right] * P \left[ B \right] * P \left[ C \right]$
						\item Also need to check:
							\begin{enumerate}[noitemsep, nolistsep]
								\item $P \left[ A \cap B \right] = P \left[ A \right] * P \left[ B \right]$
								\item $P \left[ B \cap C \right] = P \left[ B \right] * P \left[ C \right]$
								\item $P \left[ A \cap C \right] = P \left[ A \right] * P \left[ C \right]$
							\end{enumerate}
					\end{itemize}
			\end{itemize}
		\end{definition}
	If 2 events $A$ and $B$ are independent, then their complements are also independent. This is shown in \nameref{proof:Independence of Complements of Events}.
		\begin{proof}[Independence of Complements of Events] \label{proof:Independence of Complements of Events}
			We assumed that $A$ and $B$ were independent, so $P \left[ A \cap B \right] = P \left[ A \right] \cdot P \left[ B \right]$.
			There are 2 more facts we will need:
			\begin{enumerate}[leftmargin=1.0in, label=Fact \arabic*: , ref=Fact \arabic*, noitemsep, nolistsep]
				% leftmargin sets a distance for left margin
				% ref sets the way items will be cross-referenced, and can differ from the label.
				\item $P \left[ B \right] + P \left[ B^{C} \right] = 1$ \label{proof:Independence of Complements of Events:Fact 1}
				\item $P \left[ A \cap B^{C} \right] + P \left[ A \cap B \right] = P \left[ A \right]$ \label{proof:Independence of Complements of Events:Fact 2}
			\end{enumerate}
			From \ref{proof:Independence of Complements of Events:Fact 1}, we have:
			\begin{equation*}
				P \left[ A \cap B \right] = P \left[ A \right] \cdot \left( 1-P \left[ B^{C} \right] \right)
			\end{equation*}
			From \ref{proof:Independence of Complements of Events:Fact 2}, we have $P \left[ A \cap B \right] = P \left[ A \right] - P \left[ A \cap B^{C} \right]$.
			Substituting these into the equation above:
			\begin{align*}
				P \left[ A \right] - P \left[ A \cap B^{C} \right] &= P \left[ A \right] \cdot \left( 1-P \left[ B^{C} \right] \right)\\
				P \left[ A \right] - P \left[ A \cap B^{C} \right] &= P \left[ A \right] - P \left[ A \right] \cdot P \left[ B^{C} \right] \\
				- P \left[ A \cap B^{C} \right] &= -P \left[ A \right] \cdot P \left[ B^{C} \right] \\
				P \left[ A \cap B^{C} \right] &= P \left[ A \right] \cdot P \left[ B^{C} \right] \\
			\end{align*}
			$\therefore$ $A$ and $B^{C}$ are independent, according to the definition of \nameref{def:Event Independence}~events in Equation~\eqref{eq:Event Independence}.
		\end{proof}