\section{Single Discrete Random Variables} \label{sec:Single Discrete Random Variables}
	\begin{definition}[Random Variable] \label{def:Random Variable, Simple}
		A \emph{random variable} $X$ is a function that assigns a real number $X \left( \zeta \right)$ to each outcome $\zeta$ in the sample space of the random experiment.
	\end{definition}
	\begin{definition}[Discrete Random Variable] \label{def:Discrete Random Variable}
		A \emph{discrete random variable} is a random variable that assumes values in a countable set. For example, the number of heads in 3 coin flips is a discrete random variable.
	\end{definition}

	\subsection{Probability Mass Function  (PMF)} \label{subsec:Probability Mass Function}
		\begin{definition}[Probability Mass Function] \label{def:Probability Mass Function}
			The \emph{probability mass function (PMF)} of a discrete random variable $X$ is defined as:
			\begin{equation} \label{eq:Probability Mass Function}
				p_{X} \left( x \right) = P \left[ X=x \right]
			\end{equation}
			Using the coin example from the definition of a~\nameref{def:Discrete Random Variable},
			\begin{equation}
				p_{X} \left( x \right) = 
				\begin{cases}
					\frac{1}{8} & x=0 \\
					\frac{3}{8} & x=1 \\
					\frac{3}{8} & x=2 \\
					\frac{1}{8} & x=3 \\
				\end{cases}
			\end{equation}
		\end{definition}
		\begin{example}[Problem 3.2]{Probability Mass Function}
		A die is tossed and the random variable $X$ is defined as the number of full pairs of dots on the fact showing up.
			\begin{boldalphlist}[label=\textbf{(\alph*)}, ref=(\alph*)]
				\item Describe the underlying space $S$ of this random experiment and specify the probabilities of the elementary events. \label{ex:3.2a}
				\item Show the mapping from $S$ to $S_{X}$, the range of $S$ \label{ex:3.2b}
				\item Find the probabilities for the various values of $X$. \label{ex:3.2c}
				\item Repeat parts \ref{ex:3.2a}, \ref{ex:3.2b}, and \ref{ex:3.2c} if $Y$ is the number of full or partial pairs of dots in the face showing up.
				\item Explain why $\Prob \left[ X = 0 \right]$ and $\Prob \left[ Y = 0 \right]$ are not equal.
			\end{boldalphlist}
		
		\tcblower
		
		Solution from Homework 3.
		\end{example}
		
		\subsubsection{Properties of Probability Mass Functions} \label{subsubsec:Properties of Probability Mass Functions}
			\begin{propertylist}%[label=\textbf{(\roman*)}, noitemsep, nolistsep]
				\item
					\begin{equation}
						p_{X} \left( x \right) \geq 0 \text{, } \forall x \in \RealNums
					\end{equation}
				\item
					\begin{equation}
						\sum\limits_{x \in S_{X}} p_{X} \left( x \right) = 1
					\end{equation}
			\end{propertylist}
					\begin{example}[Problem 3.13]{Find Normalizing Constant}
					Let $X$ be a random varaible with PMF $p_{k} = \frac{c}{k^{2}}$ for $k = 1,2,\ldots$
						\begin{enumerate}[label=\textbf{(\alph*)}]
							\item Estimate the value of $c$ numerically. Note that the series converges.
							\item Find $\Prob \left[ X > 4 \right]$.
							\item Find $\Prob \left[ 6 \leq X \leq 8 \right]$.
						\end{enumerate}
					
					\tcblower
					
						Solution from Homework 4.
					\end{example}
			\begin{propertylist}[resume]
				\item
					\begin{equation}
						P \left[ x \in B \right] = \sum\limits_{x \in B} p_{X} \left( x \right) \text{, where } B \subset S_{X}
					\end{equation}
			\end{propertylist}
		
	\subsection{Expected Value/Mean of Single Discrete Random Variable} \label{subsec:Expected Value of Single Discrete}
		\begin{definition}[Expected Value/Mean of Single Discrete Random Variable] \label{def:Expected Value of Single Discrete}
			The \emph{expected value} or \emph{mean} of a single discrete random variable $X$ is defined by
			\begin{equation} \label{eq:Expected Value of Single Discrete}
				m_{X} = \ExpectedValue \left[ X \right] = \sum_{x \in S_{X}} x \cdot p_{X} \left( x \right)
			\end{equation}
			\begin{remark} \label{rmk:Expected Value of Single Discrete Countably Infinite}
				If $X$ is countably infinite, you will have an infinite series that exists only if
				\begin{equation} \label{eq:Expected Value of Single Discrete Countably Infinite}
					\sum_{s \in S_{X}} \lvert x \rvert \cdot p_{X} \left( x \right)
				\end{equation}
				is absolutely convergent.
			\end{remark}
		\end{definition}
		\begin{example}[Problem 3.27]{Expectation of Discrete Random Variable}
			Find the expectation of $X$ where the PMF of $X$ is \[p_{k} = \frac{\frac{pi^{2}}{6}}{k^{2}}\]. (Note that this PMF is the same as in \Cref{ex:Find Normalizing Constant}).
			
			\tcblower
			
			Solution from Homework 4. ONLY THE Expectation PART.
		\end{example}
	
		\subsubsection{Properties of Expected Values} \label{subsubsec:Properties of Discrete Expected Value}
			\begin{definition}[Linearity of Expectation] \label{def:Linearity of Expectation}
				Let $Y = X_{1} + X_{2}$
				\begin{equation}
					\ExpectedValue \left[ X \right] = \ExpectedValue \left[ X_{1} \right] + \ExpectedValue \left[ X_{2} \right]
				\end{equation}
				This can be generalized to
				\begin{equation} \label{eq:General Linearity of Expectation}
					\ExpectedValue \left[ \sum_{i=1}^{k} x_{i} \right] = \sum_{i=1}^{k} \ExpectedValue \left[ X_{i} \right]
				\end{equation}
			\end{definition}
			\begin{propertylist}
				\item
					\begin{equation}
						\ExpectedValue \left[ X_{1} + X_{2} \right] = \ExpectedValue \left[ X_{1} \right] + \ExpectedValue \left[ X_{2} \right]
					\end{equation}
				\item
					\begin{equation}
						\ExpectedValue \left[ g \left( X \right) \right] = \sum\limits_{s \in S_{X}} g \left( x \right) \cdot p_{X} \left[ X \right]
					\end{equation}
				\item
					\begin{equation}
						\ExpectedValue \left[ c g\left( X \right) \right] = c \ExpectedValue \left[ g\left( X \right) \right]
					\end{equation}
				\item
					\begin{equation}
						\ExpectedValue \left[ g_{1} \left( X \right) + g_{2} \left( X \right) + \ldots + g_{m} \left( X \right) \right] = \sum\limits_{i=1}^{m} \ExpectedValue \left[ g_{i} \left( X \right) \right]
					\end{equation}
			\end{propertylist}
		
		\subsubsection{Moments of Random Variable} \label{subsubsec:Moments of a Random Variable}
			\begin{definition}[Moment]
				The \emph{moment} of a random variable, $X$ is defined as the expectation of the random variable raised to the moment.
				\begin{equation} \label{eq:Moments of a Random Variable}
					\begin{aligned}
						\ExpectedValue \left[ X^{1} \right] &= \text{First Moment} \\
						\ExpectedValue \left[ X^{2} \right] &= \text{Second Moment} \\
						&\vdots \\
						\ExpectedValue \left[ X^{k} \right] &= \text{kth Moment} \\
					\end{aligned}
				\end{equation}
			\end{definition}					
	
	\subsection{Variance of Single Discrete Random Variable} \label{subsec:Variance of Single Discrete}
		\begin{definition}[Variance] \label{def:Variance of Single Discrete}
			The \emph{variance} of a single discrete random variable $X$ is defined as:
			\begin{equation} \label{eq:Variance of Single Discrete-Form 1}
				\ExpectedValue \left[ \left( X - \ExpectedValue \left[ X \right] \right)^{2} \right]
			\end{equation}
			\begin{equation} \label{eq:Variance of Single Discrete-Form 2}
				\Variance \left[ X \right] = \ExpectedValue \left[ X^{2} \right] - \left( \ExpectedValue \left[ X \right] \right)^{2}
			\end{equation}
			and is denoted as $\sigma_{X}^{2}$, or as the operator $\Variance \left[ X \right]$.
			\begin{remark} \label{rmk:Constant in Variance}
				If $X$ is a random variable, and $c$ is some constant coefficient, then:
				\begin{equation}
					\Variance \left[ cX \right] = c^{2} \Variance \left[ X \right]
				\end{equation}
			\end{remark}
		\end{definition}
		\begin{example}[Problem 3.27]{Variance of Discrete Random Variable}
			Find the variance of $X$ where the PMF of $X$ is \[p_{k} = \frac{\frac{pi^{2}}{6}}{k^{2}}\]. (Note that this PMF is the same as in \Cref{ex:Find Normalizing Constant}).
			
			\tcblower
			
			Solution from Homework 4. ONLY THE Variance PART.
		\end{example}
		\begin{definition}[Standard Deviation] \label{def:Standard Deviation}
			The standard deviation of a random variable $X$ is:
			\begin{equation} \label{eq:Standard Deviation}
				\sigma_{X} = \sqrt{\Variance \left[ X \right]}
			\end{equation}
		\end{definition}
	
	\subsection{Conditional Probability Mass Function} \label{subsec:Conditional Probability Mass Function}
		\begin{definition}[Conditional Probability Mass of Function] \label{def:Conditional Probability Mass Function}
			Let $X$ be a discrete random variable, with PMF $p_{X} \left( x \right)$ and let $C$ be the event with non-zero probability, i.e. $P \left[ C \right] > 0$.
			The \emph{conditional probability mass function of $X$ given $C$ (Conditional PMF)} is defined as:
			\begin{equation} \label{eq:Conditional Probability Mass Function}
				p_{X \Given C} \left( x \Given C \right) = P \left[ X=x \Given C \right] \text{ for } x \in \RealNums
			\end{equation}
			\begin{remark} \label{rmk:Properties of Conditional Probability Mass Functions}
				The conditional PMF, $p_{X \Given C} \left( x \Given C \right)$, satisfies \emph{\textbf{all}} properties of \nameref{subsubsec:Properties of Probability Density Functions}.
			\end{remark}
		\end{definition}
	
	\subsection{Conditional Expected Value of Single Discrete Random Variable} \label{subsec:Conditional Expected Value of Single Discrete}
		\begin{definition}[Conditional Expected Value of Discrete Random Variable] \label{def:Conditional Expected Value of Single Discrete}
			The \emph{conditional expected value of the discrete random variable} $X$ given $B$ is defined as:
			\begin{equation} \label{eq:Conditional Expected Value of Single Discrete}
				m_{X \Given B} = \ExpectedValue \left[ X \Given B \right] = \sum\limits_{x \in S_{X}} s \cdot p_{X} \left( x \Given B \right)
			\end{equation}
		\end{definition}
		\begin{example}[Problem 3.39]{Conditional Expected Value}
			\begin{boldalphlist}
				\item Find the conditional expected value of $X$ in Problem 3.5 given that no message gets through in the first time slow. SHow that $\ExpectedValue \left[ X \Given X>1 \right] = \ExpectedValue \left[ X \right] + 1$. \label{ex:3.39a}
				\item Find the conditional expected value of $X$ in problem 3.5 given that a message gets through in the first time slot. \label{ex:3.39b}
				\item Find $\ExpectedValue \left[ X \right]$ by using the results of Parts \ref{ex:3.39a} and \ref{ex:3.39b}. \label{ex:3.39c}
				\item Find $\ExpectedValue \left[ X^{2} \right]$ and $\Variance \left[ X \right]$ using the approach in parts \ref{ex:3.39b} and \ref{ex:3.39c}.
			\end{boldalphlist}
		
			\tcblower
			
			Solution from Homework 5.
		\end{example}
	
	\subsection{Conditional Variance of Single Discrete Random Variable} \label{subsec:Conditional Variance of Single Discrete}
		\begin{definition}[Conditional Variance of Discrete Random Variable] \label{def:Conditional Variance of Single Discrete}
			The \emph{conditional variance of a discrete random variable} $X$ given event $B$ as defined as:
			\begin{equation} \label{eq:Conditional Variance of Single Discrete}
				\begin{aligned}
					\sigma_{X \Given B}^{2} &= \Variance \left[ X \Given B \right] \\
						&= \ExpectedValue \left[ \left( X - \ExpectedValue \left[ X \Given B \right] \right)^{2} \Given B \right] \\
						&= \sum\limits_{x \in S_{X}} \left( x - m_{X \Given B} \right)^{2} \cdot p_{X} \left( x \Given B \right) \\
					\Variance \left[ X \Given B \right] &= \ExpectedValue \left[ X^{2} \Given B \right] - \left( \ExpectedValue \left[ X \Given B \right] \right)^{2} \\
				\end{aligned}
			\end{equation}
		\end{definition}