\section{Probability Models} \label{sec:Probability Models}
	\subsection{Relative Frequency} \label{subsec:Relative Frequency}
		\begin{definition}[Relative Frequency] \label{def:Relative Frequency}
			\emph{Relative frequency} is defined in \Cref{eq:Relative Frequency}:
			\begin{equation} \label{eq:Relative Frequency}
				f_k (n) = \frac{N_k (n)}{n}
			\end{equation}
			\begin{itemize}[noitemsep, nolistsep]
				\item $k$ is the outcome
				\item $N_k (n)$ is the number of times outcome $k$
			\end{itemize}
		\end{definition}
	
		\subsubsection{Properties of Relative Frequencies} \label{subsec:Properties Relative Frequency}
		\begin{enumerate}[label=\textbf{(\roman*)}, noitemsep, nolistsep]
			\item 
				\begin{equation}
					f_k (n) = \frac{N_k (n)}{n}
				\end{equation}
			\item
				\begin{equation}
					0 \leq N_k (n) \leq n
				\end{equation}
			\item
				\begin{equation}
					0 \leq f_k (n) \leq 1 = \frac{0}{n} \leq \frac{N_k (n)}{n} \leq \frac{n}{n}
				\end{equation}
			\item
				\begin{equation}
					\sum\limits_{k=1}^{k} f_k (n) = \sum\limits_{k=1}^{k} \frac{N_k (n)}{n} = \frac{\sum\limits_{k=1}^{k} N_k (n)}{n} = \frac{n}{n} = 1
				\end{equation}
			\item
				\begin{equation}
					\sum\limits_{k=1}^{k} f_k (n) = 1
				\end{equation}
			\item If events $A$ and $B$ are disjoint and event $C$ is "$A$ or $B$", then 
				\begin{equation}
					F_C = F_A (n) + F_B (n)
				\end{equation}
		\end{enumerate}
	\subsection{Statistical Regularity} \label{subsec:Statistical Regularity}
		\begin{definition} \label{def:Statistical Regularity}
			The averages obtained in long sequences of trials that lead to approximately the same value have a property called \emph{statistical regularity}.
			This is defined in \Cref{eq:Statistical Regularity}.
			\begin{equation} \label{eq:Statistical Regularity}
				\lim\limits_{n \rightarrow \infty} f_k (n) = p_k
			\end{equation}
			\begin{itemize}[noitemsep, nolistsep]
				\item $p_k$ is the probability of event $k$ occurring
			\end{itemize}
		\end{definition}