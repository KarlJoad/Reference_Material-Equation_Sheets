\section{Random Vectors} \label{sec:Random Vectors}
Random Vectors are usually denoted:
	\begin{equation} \label{eq:Random Vector Notation}
		\vec{X} = \langle X_{1}, X_{2} X_{3}, \ldots, X_{n} \rangle
	\end{equation}
	
	\subsection{Joint CDF of a Random Vector} \label{subsec:Joint CDF of Random Vector}
		\begin{equation} \label{eq:Joint CDF of Random Vector}
			\begin{aligned}
				F_{\vec{X}} \left( \vec{x} \right) 
					&= F_{X_{1}, X_{2}, X_{3}, \ldots, X_{n}} \left( x_{1}, x_{2}, x_{3}, \ldots, x_{n} \right) \\
					&= P \left[ X_{1} \leq x_{1}, X_{2} \leq x_{2}, X_{3} \leq x_{3}, \ldots, X_{n} \leq x_{n} \right] \\
			\end{aligned}
		\end{equation}
		
	\subsection{Joint PDF of a Random Vector} \label{subsec:Joint PDF of Random Vector}
		\begin{equation} \label{eq:Joint PDF of Random Vector}
			f_{\vec{X}} \left( \vec{x} \right) = \frac{\partial^{n} F_{\vec{X}} \left( \vec{x} \right)}{\partial x_{1} \partial x_{2} \partial x_{3} \cdots \partial x_{n}}
		\end{equation}
		
		\subsubsection{Marginal PDF of a Random Vector} \label{subsubsec:Marginal PDF of Random Vector}
		Integrate out the terms that you're not interested in.
		\begin{equation} \label{eq:Marginal PDF of Random Vector}
			f_{\vec{X}} = \int_{-\infty}^{\infty} \cdots \int_{-\infty}^{\infty} f_{\vec{X}} \left( \vec{x} \right) \partial x_{2} \partial x_{3} \cdots \partial x_{n}
		\end{equation}
		For instance, say we want the marginal PDF of some function with respect to $X_{1}$, $X_{3}$, and $X_{4}$.
		\begin{equation} \label{eq:Marginal PDF of Random Vector Multiple Variables}
			f_{X_{1}, X_{3}, X_{4}} \left( x_{1}, x_{3}, x_{4} \right) = \int_{-\infty}^{\infty} \cdots \int_{-\infty}^{\infty} f_{\vec{X}} \left( \vec{x} \right) \partial x_{2} \partial x_{5} \partial x_{6} \cdots \partial x_{n}
		\end{equation}
	
	\subsection{Conditional Probability Functions of Random Vectors} \label{subsec:Random Vector Conditional Probability Functions}
	This section is just an extension of Section~\ref{subsec:Multiple Variable Conditional Probability Functions}, \nameref{subsec:Multiple Variable Conditional Probability Functions}.
	There are 3 major cases for these:
		\begin{enumerate}[noitemsep, nolistsep]
			\item \nameref{subsubsec:Conditional Probability Discrete Random Vectors}
			\item \nameref{subsubsec:Conditional Probability Mixed Random Vectors}
			\item \nameref{subsubsec:Conditional Probability Continuous Random Vectors}
		\end{enumerate}
	
		\begin{remark*} \label{rmk:Define Random Vector Y for Example}
			\begin{large}
				For the sections below, let $\vec{Y}= \langle Y_{1},Y_{2},Y_{3} \rangle$ and $\vec{y}= \langle y_{1},y_{2},y_{3} \rangle$.
			\end{large} \newline
			While I am using $\vec{Y}$ and $\vec{y}$, these equations can be further generalized to higher dimensions.
			All that would be required for this is to keep track of everything.
		\end{remark*}
	
		\subsubsection{Discrete Random Vectors} \label{subsubsec:Conditional Probability Discrete Random Vectors}
			\begin{definition}[Conditional Probability Mass Function] \label{def:Discrete Random Vector-Conditional PMF}
				The \emph{conditional Probability Mass Function (Conditional PMF)} of $Y_{3}$ given that $Y=y$ is:
				\begin{equation} \label{eq:Discrete Random Vector-Conditional PMF}
					p_{Y_{3}} \left( y_{3} \Given y_{1},y_{2} \right)
					= \frac{P \left[ \lbrace Y_{3}=y_{3} \rbrace \cap \left( \lbrace Y_{1}=y_{1} \rbrace \cap \lbrace Y_{2}=y_{2} \rbrace \right) \right]}{P \left[ \lbrace Y_{1}=y_{1} \rbrace \cap \lbrace Y_{2}=y_{2} \rbrace \right]} 
					= \frac{p_{\vec{Y}} \left( \vec{y} \right)}{p_{Y_{1},Y_{2}} \left( y_{1},y_{2} \right)}
				\end{equation}
				\begin{remark}
					This also implies that
					\begin{equation} \label{eq:Discrete Random Vector-Joint PMF}
						p_{\vec{Y}} \left( \vec{y} \right) = p_{Y_{3}} \left( y_{3} \Given y_{1},y_{2} \right) \cdot p_{Y_{2}} \left( y_{2} \Given y_{1} \right) \cdot p_{Y_{1}} \left( y_{1} \right)
					\end{equation}
				\end{remark}
				\begin{remark}
					If all elements of $\vec{Y}$ are \emph{independent} (Remember that you need to check each subgroup too, like shown in Section~\ref{subsec:Event Independence}), then:
					\begin{equation} \label{eq:Discrete Random Vector-Independent Conditional PMF}
						p_{Y_{3}} \left( y_{3} \Given y_{1},y_{2} \right)
						= \frac{p_{\vec{Y}} \left( \vec{y} \right)}{p_{Y_{1},Y_{2}} \left( y_{1},y_{2} \right)}
						= \frac{p_{Y_{1},Y_{2}} \left( y_{1},y_{2} \right) p_{Y_{3}} \left( y_{3} \right)}{p_{Y_{1},Y_{2}} \left( y_{1},y_{2} \right)}
						= p_{Y_{3}} \left( y_{3} \right)
					\end{equation}
				\end{remark}
				\begin{remark}
					The \nameref{def:Discrete Random Vector-Conditional PMF} of 2 discrete random variables satisfies all \nameref{subsubsec:Properties of Probability Mass Functions}.
				\end{remark}
			\end{definition}
		
		\subsubsection{Mixed Random Vectors} \label{subsubsec:Conditional Probability Mixed Random Vectors}
		
		
		\subsubsection{Continuous Random Vectors} \label{subsubsec:Conditional Probability Continuous Random Vectors}
		
		
	
	