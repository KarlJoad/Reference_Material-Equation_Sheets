\subsection{\texorpdfstring{The Integers modulo $n$}{The Integers modulo n}}\label{subsec:Integer_Modulo_n}
Let $n$ be a positive integer.
\begin{definition}[Congruence]\label{def:Congruence}
  If $a$ and $b$ are integers, then \emph{$a$ is said to be congruent to $b$ modulo $n$}, which is written as
  \begin{equation}\label{eq:A_Congruent_B}
    a \equiv b \pmod{n}
  \end{equation}

  If $n$ divides $(a-b)$, i.e. $n \Divides (a-b)$, then we call $n$ the \emph{modulus} of the congruence.

\end{definition}

\begin{theorem}
  For $a, a_{1}, b, b_{1}, c\in \AllIntegers$, we have
  \begin{propertylist}
  \item $a \equiv b \pmod{n}$ \emph{if and only if} $a$ and $b$ leave the same \nameref{def:Integer_Remainder} when divided by $n$.
  \item $a \equiv a \pmod{n}$ \label{prop:A_Congruent_B_Reflexivity}
  \item If $a \equiv b \pmod{n}$, then $b \equiv a \pmod{n}$ \label{prop:A_Congruent_B_Symmetry}
  \item If $a \equiv b \pmod{n}$ adn $b \equiv c \pmod{n}$, then $a \equiv c \pmod{n}$ \label{prop:A_Congruent_B_Transitivity}
  \item If $a \equiv a_{1} \pmod{n}$ and $b \equiv b_{1} \pmod{n}$, then $a+b = a_{1} + b_{1} \pmod{n}$ and $ab = a_{1}b_{1} \pmod{n}$.
  \end{propertylist}

  \Crefrange{prop:A_Congruent_B_Reflexivity}{prop:A_Congruent_B_Transitivity} are called \emph{reflexivity}, \emph{symmetry}, and \emph{transitivity}, respectively.
\end{theorem}

\begin{example}[Exercise 1, Question 1.2b]{Integers modulo n}
  Write all the units (\nameref{def:Invertible_Element}) in \TextIntsMod{36}?
  \tcblower{}
  First, we start by constructing our set of integers modulo $n$.
  \begin{equation*}
    \IntsMod{36} = \bigl\lbrace [0], [1], [2], [3], [4], \ldots, [35] \bigr\rbrace
  \end{equation*}

  Since we are only worried about the units of \TextIntsMod{36}, we need to find the integers that satisfy \Cref{eq:Invertible_Element}.
  This is done by finding an $a$ value that has a \nameref{def:Multiplicative_Inverse}, which requires that \Cref{eq:Invertible} be true, namely
  \begin{equation*}
    \gcd(a, n) = 1
  \end{equation*}
  This leaves us with
  \begin{equation*}
    \IntsMod{36} = \bigl\lbrace [1], [5], [7], [11], [13], [17], [19], [23], [25], [29], [31], [35] \bigr\rbrace
  \end{equation*}
  which is the solution.
\end{example}

%%% Local Variables:
%%% mode: latex
%%% TeX-master: "../../EDIN01-Cryptography-Reference_Sheet"
%%% End:
