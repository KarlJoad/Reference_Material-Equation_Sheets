\subsection{Euler Phi Function}\label{subsec:Euler_Phi_Function}
\begin{definition}[Euler Phi Function]\label{def:Euler_Phi_Function}
  For $n \geq 1$, let $\phi(n)$ denote the number of integers in the interval $[1, n]$, which are \nameref{def:Relatively_Prime} to $n$.
  This function is called the \emph{Euler Phi Function}.
  \begin{equation}\label{eq:Euler_Phi_Function}
    \phi(n) = \left( p_{1}^{e_{1}} - p_{1}^{e_{1}-1} \right) \left( p_{2}^{e_{2}} - p_{2}^{e_{2}-1} \right) \cdots \left( p_{k}^{e_{k}} - p_{k}^{e_{k}-1} \right)
  \end{equation}
  \begin{remark}
    The \nameref{def:Euler_Phi_Function} is closely related to \nameref{def:Set_Order}.
  \end{remark}
\end{definition}

\begin{theorem}[Euler Phi Function]\label{thm:Euler_Phi_Function}
  There are a few properties of the \nameref{def:Euler_Phi_Function} that we will treat as true because of this theorem.
  \begin{propertylist}
  \item If $p$ is a \nameref{def:Prime}, then $\phi(p) = p - 1$.\label{prop:Euler_Phi_Function_Properties-Prime_Number}
  \item If $\gcd(a, b) = 1$, then $\phi(ab) = \phi(a) \phi(b)$.\label{prop:Euler_Phi_Function_Properties-Split_Multiplication}
  \item If $n = p_{1}^{e_{1}} p_{2}^{e_{2}} \cdots p_{k}^{e_{k}}$, then
    \begin{equation*}
      \phi(n) = \left( p_{1}^{e_{1}} - p_{1}^{e_{1}-1} \right) \left( p_{2}^{e_{2}} - p_{2}^{e_{2}-1} \right) \cdots \left( p_{k}^{e_{k}} - p_{k}^{e_{k}-1} \right)
    \end{equation*}\label{prop:Euler_Phi_Function_Properties-Product_of_Primes}
  \end{propertylist}
\end{theorem}

\begin{example}[Exercise 1, Question 1.2a]{Euler Phi Function}
  Find the value of $\phi(36)$?
  \tcblower{}
  First, we note that 36 is not a \nameref{def:Prime} number, thus we need to find a set of primes that are equal to 36.
  The divisors of 36 are
  \begin{equation*}
    \lbrace \pm 1, \pm 2, \pm 3, \pm 4, \pm 9, \pm 12, \pm 18, \pm 36 \rbrace
  \end{equation*}
  And all possible prime numbers present as divisors of 36 are
  \begin{equation*}
    \lbrace 2, 3 \rbrace
  \end{equation*}

  So, there must be some combination of $2^{x}\cdot 3^{y}$ that yields 36.
  In fact, $2^{2} \cdot 3^{2} = 4 \cdot 9 = 36$.
  Since 36 can be broken up as a product of 2 \nameref{def:Prime} numbers raised to some power, we can use \Cref{prop:Euler_Phi_Function_Properties-Product_of_Primes} of the \nameref{def:Euler_Phi_Function} to simplify this.
  But first, we need to separate the two values from each other using \Cref{prop:Euler_Phi_Function_Properties-Split_Multiplication} of the \nameref{def:Euler_Phi_Function} to apply \Cref{prop:Euler_Phi_Function_Properties-Product_of_Primes}.

  We need to check $\gcd \left( 2^{2}, 3 ^{2} \right) = 1$.
  \begin{align*}
    \gcd \left( 2^{2}, 3^{2} \right) &= \gcd(4, 9) \\
    9 &= a4 + b \\
                                     &= 2 \cdot 4 + 1 \\
    4 &= a \cdot 1 + b \\
                                     &= 4 \cdot 1 + 0
  \end{align*}
  So, $\gcd \left( 2^{2}, 3^{2} \right) = 1$, so we can use \Cref{prop:Euler_Phi_Function_Properties-Split_Multiplication}.
  \begin{equation*}
    \phi \left( 2^{2} \cdot 3^{2} \right) = \phi \left( 2^{2} \right) \phi \left( 3^{2} \right)
  \end{equation*}
  And now we can apply \Cref{prop:Euler_Phi_Function_Properties-Product_of_Primes} to find our answer.

  \begin{align*}
    \phi \left( 2^{2} \right) \phi \left( 3^{2} \right) &= \left( 2^{2} - 2^{2-1} \right) \left( 3^{2} - 3^{2-1} \right) \\
                                                        &= (4-2)(9-3) \\
                                                        &= (2)(6) \\
                                                        &= 12
  \end{align*}
  Thus, $\phi(36) = 12$.
\end{example}

\begin{lemma}[Computing the \nameref{def:GCD}]\label{lemma:Compute_GCD}
  If $a$ and $b$ are positive integers where $a > b$, then
  \begin{equation}\label{eq:Compute_GCD}
    \gcd(a, b) = \gcd(b, a \bmod b)
  \end{equation}
  \begin{remark*}
    This can be repeated to efficiently calculate the $\gcd(a, b)$.
    This is called the \nameref{def:Euclidean_Algorithm}.
  \end{remark*}
\end{lemma}

\begin{definition}[Euclidean Algorithm]\label{def:Euclidean_Algorithm}
  The \emph{euclidean algorithm} is a way to efficiently calculate the $\gcd(a, b)$.
  
  \begin{algorithm}[H]
    \DontPrintSemicolon{}
    \SetKwInOut{Input}{Input}\SetKwInOut{Output}{Output}

    \Input{Two non-negative integers $a, b$ where $a \geq b$.}
    \Output{$\gcd(a, b)$.}
    \BlankLine{}
    Set $r_{0} \leftarrow a$, $r_{1} \leftarrow b$, $i \leftarrow 1$ \;
    \While{$r_{i} \neq 0$}{
      Set $r_{i+1} \leftarrow r_{i-1} \bmod r_{i}$ \;
      $i \leftarrow i+1$ \;
    }
    \Return{$r_{i}$}
    \caption{Euclidean Algorithm}
    \label{algo:Euclidean_Algorithm}
  \end{algorithm}
\end{definition}

\begin{example}[Exercise 1, Question 1.1a]{Euclidean Algorithm}
  Find the \nameref{def:GCD} of 222 and 1870?

  \tcblower{}

  \begin{align*}
    1870 &= a 222 + b \\
         &= 8 \cdot 222 + 94 \\
    222 &= a 94 + b \\
         &= 2 \cdot 94 + 34 \\
    94 &= a 34 + b \\
         &= 2 \cdot 34 + 26 \\
    34 &= a 26 + b \\
         &= 1 \cdot 26 + 8 \\
    26 &= a 8 + b \\
         &= 3 \cdot 8 + 2 \\
    8 &= a 2 + b \\
         &= 4 \cdot 2 + 0
  \end{align*}

  Thus, since $4 \cdot 2 = 8$, 2 is the \nameref{def:GCD} of 222 and 1870.
\end{example}

\begin{theorem}
  There exist integers $x, y$ such that $\gcd(a, b)$ can be written as
  \begin{equation}\label{eq:Extended_Euclidean_Algorithm_Basis}
    \gcd(a, b) = ax + by
  \end{equation}
\end{theorem}
\begin{proof}
  \begin{align*}
    \gcd(a, b) &= r_{i} \\
               &= r_{i-2} - q_{i-1}r_{i-1} \\
               &= r_{i-2} - q_{i-1}(r_{i-3} - q_{i-2}r_{i-2}) \\
               &\vdots \\
               &= r_{0}x + r_{1}y \\
               &= ax + by
  \end{align*}
  for some integers $x, y \in \AllIntegers$.
\end{proof}

This means that the \nameref{def:Euclidean_Algorithm} can be extended to return the values of $x$ and $y$ from \Cref{eq:Extended_Euclidean_Algorithm_Basis}.

\begin{definition}[Extended Euclidean Algorithm]\label{def:Extended_Euclidean_Algorithm}
  The \emph{extended euclidean algorithm} is a way to efficiently calculate the linear pair of integers ($x, y \in \AllIntegers$) that satisfy \Cref{eq:Extended_Euclidean_Algorithm_Basis}.
  \begin{equation*}
    \gcd(a, b) = ax + by
  \end{equation*}

  \begin{algorithm}[H]
    \DontPrintSemicolon{}
    \SetKwInOut{Input}{Input}\SetKwInOut{Output}{Output}

    \Input{Two non-negative integers $a, b$ where $a \geq b$.}
    \Output{$d = \gcd(a, b)$ and two integers $x, y$ such that $d = ax + by$.}
    \BlankLine{}

    \If{$b=0$}{
      \Return{$a$, $x \leftarrow 1$, $y \leftarrow 0$}
    }
    Set $x_{2} \leftarrow 1$, $x_{1} \leftarrow 0$, $y_{2} \leftarrow 0$, and $y_{1} \leftarrow 1$. \;
    \While{$b > 0$}{
      $q \leftarrow a \div b$. \;
      $r \leftarrow a-qb$, $x \leftarrow x_{2} - qx_{1}$, $y \leftarrow y_{2} - qy_{1}$. \;
      $a \leftarrow b$
      $b \leftarrow r$
      $x_{2} \leftarrow x_{1}$, $x_{1} \leftarrow x$, $y_{2} \leftarrow y_{1}$, and $y_{1} \leftarrow y$. \;
    }
    Set $d \leftarrow a$, $x \leftarrow x_{2}$, $y \leftarrow y_{2}$. \;
    \Return{$d, x, y$}
    \caption{Extended Euclidean Algorithm (Bezout's Theorem)}
    \label{algo:Extended_Euclidean_Algorithm}
  \end{algorithm}
\end{definition}

\begin{example}[Exercise 1, Question 1.1b]{Extended Euclidean Algorithm}
  Find the integers $x$ and $y$ such that $\gcd(222, 1870) = 222x + 1870y$?
  \tcblower{}
  From \Cref{ex:Euclidean Algorithm}, we know $\gcd(222, 1870) = 2$, so we can plug that in.
  We now know that
  \begin{equation*}
    2 = 222x + 1870y
  \end{equation*}

  Now, we essentially run the \nameref{def:Euclidean_Algorithm} backwards.
  \begin{align*}
    2 &= 26 - 3 \cdot 8 \\
      &= 26 - 3(34 - 1 \cdot 26) = 26 - 3 \cdot 34 + 3 \cdot 26 \\
      &= 4 \cdot 26 - 3 \cdot 34 \\
      &= 4(94 - 2 \cdot 34) - 3 \cdot 34 = 4 \cdot 94 - 8 \cdot 34 - 3 \cdot 34\\
      &= 4 \cdot 94 - 11 \cdot 34 \\
      &= 4 \cdot 94 - 11(222 - 2 \cdot 94) = 4 \cdot 94 - 11 \cdot 222 + 22 \cdot 94 \\
      &= 26 \cdot 94 - 22 \cdot 222 \\
      &= 26(1870 - 8 \cdot 222) - 22 \cdot 222 = 26 \cdot 1870 - 208 \cdot 222 - 11 \cdot 222 \\
      &= -219 \cdot 222 + 26 \cdot 1870
  \end{align*}
  Now we need to check the solution we might have found
  \begin{align*}
    -219 \cdot 222 + 26 \cdot 1870 &= 2 \\
    -48618 + 48620 &= 2 \\
    2 &= 2
  \end{align*}

  Thus,
  \begin{align*}
    x &= -219 \\
    y &= 26
  \end{align*}
\end{example}

%%% Local Variables:
%%% mode: latex
%%% TeX-master: "../../EDIN01-Cryptography-Reference_Sheet"
%%% End:
