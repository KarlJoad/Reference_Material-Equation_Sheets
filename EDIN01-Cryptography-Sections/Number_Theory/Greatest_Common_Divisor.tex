\subsection{Greatest Common Divisor}\label{subsec:Greatest_Common_Divisor}
\begin{definition}[Common Divisor]\label{def:Common_Divisor}
  An integer $c$ is a \emph{common divisor} of $a$ and $b$ if $c \Divides a$ and $c \Divides b$.
\end{definition}

\begin{definition}[Greatest Common Divisor]\label{def:GCD}
  A non-negative integer $d$ is called the \emph{greatest common divisor} (\emph{GCD}) of integers $a$ and $b$ if:
  \begin{enumerate}[noitemsep]
  \item $d$ is a \nameref{def:Common_Divisor} of $a$ and $b$.
  \item For every other common divisor $c$ it holds that $c \Divides d$.
  \end{enumerate}

  The greatest common divisor is denoted
  \begin{equation}\label{eq:GCD}
    \gcd(a, b)
  \end{equation}

  $\gcd(a,b)$ is the \textbf{largest positive} integer dividing both $a$ and $b$ (except for $\gcd(0,0)=0$).
  
  \begin{remark}
    If $a, b \in \PositiveInts$, then $\lcm(a, b) \cdot \gcd(a, b) = a \cdot b$
  \end{remark}
\end{definition}

\begin{example}[Lecture 1, Example 2]{Greatest Common Divisor}
  What is the $\gcd(18, 24)$?
  \tcblower{}
  Common Divisors = $\lbrace \pm 1, \pm 2, \pm 4, \pm 6 \rbrace$.

  Since we can only allow positive integers,
  \begin{equation*}
    \gcd(18, 24) = +6
  \end{equation*}
\end{example}

%%% Local Variables:
%%% mode: latex
%%% TeX-master: "../../EDIN01-Cryptography-Reference_Sheet"
%%% End:
