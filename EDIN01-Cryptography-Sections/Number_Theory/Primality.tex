\subsection{Primality}\label{subsec:Primality}
\begin{definition}[Relatively Prime]\label{def:Relatively_Prime}
  $a, b$ are called \emph{relatively prime} if $\gcd(a, b) = 1$.
\end{definition}

\begin{definition}[Prime]\label{def:Prime}
  An integer $p \geq 2$ is called \emph{prime} if its only positive divisors are $1$ and $p$.
  Otherwise, $p$ is called a \emph{\nameref{def:Composite}}.
\end{definition}

\begin{definition}[Composite]\label{def:Composite}
  An integer $p \geq 2$ is called \emph{composite} if it has more positive divisors than just $1$ and $p$.
  Otherwise, $p$ is called a \emph{\nameref{def:Prime}}.
\end{definition}

\subsubsection{Number of Primes}\label{subsubsec:Number_of_Primes}
The number of primes $\leq x$ is denoted
\begin{equation}\label{eq:Number_of_Primes}
  \pi(x)
\end{equation}
\begin{enumerate}[noitemsep]
\item There are infinitely many primes
\item $\lim\limits_{x \to \infty} \frac{\pi(x)}{\frac{x}{\ln(x)}} = 1$
\item For $x \geq 17$, $\frac{x}{\ln(x)} < \pi(x) < \frac{1.25506x}{\ln(x)}$
\end{enumerate}

%%% Local Variables:
%%% mode: latex
%%% TeX-master: "../../EDIN01-Cryptography-Reference_Sheet"
%%% End:
