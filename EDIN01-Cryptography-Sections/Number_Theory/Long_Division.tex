\subsection{Integer Long Division}\label{subsec:Integer_Long_Division}
For $a, b \in \AllIntegers$, with $b \geq 1$.
Then an ordinary long division of $a$ by $b$, i.e. $a \div b$ yields two integers $q$ and $r$ such that
\begin{equation}\label{eq:Integer_Long_Division}
  a = qb + r, \text{where } 0 \leq r < b
\end{equation}

$q$ and $r$ are called the \nameref{def:Integer_Quotient} and \nameref{def:Integer_Remainder}, respectively, and are \textbf{unique}.

\begin{definition}[Quotient]\label{def:Integer_Quotient}
  The \emph{quotient}, $q$, of $a$ divided by $b$ is denoted $a \DIV b$.
\end{definition}

\begin{definition}[Remainder]\label{def:Integer_Remainder}
  The \emph{remainder}, $r$, of $a$ divided by $b$ is denoted $a \bmod b$.
\end{definition}

\begin{example}[Lecture 1, Example 1]{Integer Long Division}
  If $a=53$ and $b=9$, what is $a \bmod b$?
  \tcblower{}
  \begin{align*}
    53 &= q9 + r \\
    q &= 5 \\
    r &= 8
  \end{align*}

  Thus, $53 \bmod 9 = 8$.
\end{example}

%%% Local Variables:
%%% mode: latex
%%% TeX-master: "../../EDIN01-Cryptography-Reference_Sheet"
%%% End:
