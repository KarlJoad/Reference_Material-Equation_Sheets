\section{Public-Key Encryption}\label{sec:Public_Key_Encryption}
\begin{definition}[Public-Key Encryption Scheme]\label{def:Public_Key_Encryption_Scheme}
  A public-key encryption scheme is a set of encryption transformations $\lbrace E_{e} : e \in \Keyspace \rbrace$ and a set of decryption transformations $\lbrace D_{d} : d \in \Keyspace \rbrace$.
  For each $e \in \Keyspace$ there is a corresponding $d \in \Keyspace$ such that $D_{d} \bigl(E_{e} (M) \bigr) = M,  \forall M$.

  Furthermore, after choosing such a pair $(e, d)$, the \emph{public key} $e$ (or the \emph{public parameter}) is made public, while the associated \emph{secret key} $d$ is kept secret.
  For the scheme to be secure, it must be computationally infeasible to compute $d$ and/or $E_{e}^{-1}(C)$, knowing the public value $e$.
  These types of schemes are built with \nameref{def:One_Way_Function}s and/or \nameref{def:Trapdoor_One_Way_Function}s.

  This means that the encryption key can be public, allowing anyone to send an encrypted message to the receiver.
  Then, only the receiver can decrypt the message, because the decryption key is kept secret.

  \begin{remark}[Construction]
    These are constructed through multiple \nameref{def:Trapdoor_One_Way_Function}s.
  \end{remark}
\end{definition}

\begin{definition}[One-Way Function]\label{def:One_Way_Function}
  An informal definition of a \emph{one-way function} $f(x)$ is a function from a set $\mathcal{X}$ to a set $\mathcal{Y}$ such that $f(x)$ is easy to compute for all $x \in \mathcal{X}$, but for ``essentially all'' elements $y \in \mathcal{Y}$ it is ``computationally infeasible'' to find any $x \in \mathcal{X}$ such that $f(x) = y$.

  \begin{remark}[``Essentially All'']\label{rmk:One_Way_Function-Essentially_All}
    There are some special values where \nameref{def:One_Way_Function}s to not behave normally, in that the function becomes computationally feasible to solve.
  \end{remark}
\end{definition}

\begin{definition}[Trapdoor One-Way Function]\label{def:Trapdoor_One_Way_Function}
  A \emph{trapdoor one-way function} $f(x)$ is a \nameref{def:One_Way_Function} $f : \mathcal{X} \mapsto \mathcal{Y}$ such that if one knows some specific information $T$, called the \emph{trapdoor information}, then $f(x)$ is computationally easy to invert $f$, i.e., for any $y \in \mathcal{Y}$ it is easy to find a $x \in \mathcal{X}$ such that $f(x) = y$.
  For anyone without knowledge of the trapdoor information $T$, $f(x)$ is a \nameref{def:One_Way_Function}.
\end{definition}

\subsection{RSA Public-Key Encryption Scheme}\label{subsec:RSA_Public_Key_Encryption_Scheme}
\begin{definition}[RSA Public-Key Encryption Scheme]\label{def:RSA_Public_Key_Encryption_Scheme}
  The \emph{RSA public-key encryption scheme} is defined as follows.
  Let $n = pq$, where $p$ and $q$ are two large \nameref{def:Prime}s.
  Let $\Messages = \Ciphertexts = \IntsMod{n}$.
  Pick a number $e$ that is \nameref{def:Relatively_Prime} to $\phi(n)$ (the \nameref{def:Set_Order} of \TextIntsMod{n}) and calculate a number $d$ such that $ed = 1 \bmod \phi(n)$.
  The \emph{public key} is the \underline{two numbers} $(n, e)$ and the public encryption transformation $E(M)$ is
  \begin{equation}\label{eq:RSA_Public_Key_Encryption_Scheme-Encryption}
    E(M) = M^{e} \bmod n
  \end{equation}

  and the associated decryption transformation is
  \begin{equation}
    \label{eq:RSA_Public_Key_Encryption_Scheme-Decryption}
    D(C) = C^{d} \bmod n
  \end{equation}
\end{definition}

\begin{proof}[RSA Public-Key Encryption Scheme]\label{proof:RSA_Public_Key_Encryption_Scheme}
  To verify that the \nameref{def:RSA_Public_Key_Encryption_Scheme} returns the plaintext after it was encrypted by the public key, we first assume that $n$, $e$, and $d$ are all properly defined.
  By extension, this means that $E(M)$ and $D(C)$ are also well-defined.

  We start be substituting the ciphertext, $C$ in \Cref{eq:RSA_Public_Key_Encryption_Scheme-Decryption} with the definition of the cipher text.
  \begin{equation*}
    \begin{aligned}
      D(C) &= C^{d} \bmod n \\
      &= {(M^{e})}^{d} \bmod n \\
      &= M^{ed} \bmod n
    \end{aligned}
  \end{equation*}

  Now, we note that $ed = 1 \bmod \phi(n)$, which means we can write
  \begin{equation*}
    ed = 1 + t \phi(n)
  \end{equation*}
  for some integer $t$.

  So, we continue, and use our two equations together.
  \begin{equation*}
    \begin{aligned}
      D(C) &= M^{ed} \bmod n \\
      &= M^{1 + \phi(n)} \bmod n \\
      &= M \cdot M^{\phi(n)} \bmod n \\
    \end{aligned}
  \end{equation*}

  From Euler's formula, we know $x^{\phi(n)} = 1$ for any $x \in \MultiplicativeGroup{n}$ (From \Cref{def:Multiplicative_Inverse} of \nameref{def:Multiplicative_Inverse}).
  We also assume that $M$ is invertible, otherwise the entire scheme falls apart, since a non-invertible message means it cannot be decrypted once encrypted.
  \begin{equation*}
    \begin{aligned}
      D(C) &= M \cdot M^{\phi(n)} \bmod n \\
      &= (M \cdot 1) \bmod n \\
      &= M \bmod n
    \end{aligned}
  \end{equation*}
\end{proof}

\begin{example}[Lecture 12, Example 3]{RSA Public-Key Encryption Scheme}
  Let $p = 47$ and $q = 167$.
  Then, $n = pq = 7849$.
  Compute $\phi(n)$,
  \begin{equation*}
    \begin{aligned}
      \phi(n) &= \phi(7849) \\
      &= \phi(47 \cdot 167) \\
      &= (47 - 1) (167 - 1) \\
      &= 7636
    \end{aligned}
  \end{equation*}

  Now, we choose a value of $e$ such that $e$ and $\phi(n)$ are \nameref{def:Relatively_Prime}, $\gcd(e, \phi(n)) = 1$.
  We choose $e=25$. (This is just one correct $e$. There are many more.)
  Since we chose $e$ where $\gcd(e, \phi(n)) = 1$, $e$ is a \nameref{def:Multiplicative_Inverse} in $\IntsMod{\phi(n)} = \IntsMod{7636}$.

  We can use the \nameref{def:Euclidean_Algorithm} and Bezout's lemma to find the inverse.
  \begin{equation*}
    d = 2749
  \end{equation*}

  Thus,
  \begin{equation*}
    25 \cdot 2749 = 1 \bmod 7636
  \end{equation*}

  Since this is true, we publish our public key, $(n, e) = (7636, 25)$.
  \tcblower{}
  Now, to send a message, Alice uses \Cref{eq:RSA_Public_Key_Encryption_Scheme-Encryption}.
  Her message has $M = 2728$.
  \begin{equation*}
    \begin{aligned}
      C &= M^{e} \bmod n \\
      &= 2728^{25} \bmod 7849 \\
      &= 2401 \bmod 7849
    \end{aligned}
  \end{equation*}

  When Bob receives this ciphertext, he can decrypt it using \Cref{eq:RSA_Public_Key_Encryption_Scheme-Decryption}.
  \begin{equation*}
    \begin{aligned}
      M &= C^{d} \bmod n \\
      &= 2401^{2749} \bmod 7849 \\
      &= 2728 \bmod 7849
    \end{aligned}
  \end{equation*}

  Thus, Bob gets Alice's original message, and he is likely the only one who was able to decrypt it.
\end{example}

\subsubsection{Security of the \nameref*{subsec:RSA_Public_Key_Encryption_Scheme}}\label{subsubsec:RSA_Encryption_Scheme-Security}
The \nameref{def:RSA_Public_Key_Encryption_Scheme} relies on the factorization problem.
Namely, that for large semi-prime numbers, it is difficult to find its constiuent \nameref{def:Prime} factors.
However, this \textbf{does not} mean that breaking RSA is equivalent to solving a factorization problem.
It is currently unknown if there is a way to break RSA without factoring the integer $n$.

\subsubsection{Implementation of the \nameref*{subsec:RSA_Public_Key_Encryption_Scheme}}\label{subsubsec:RSA_Encryption_Scheme-Implementation}
In order to compute $M^{e} \bmod n$ we need $L = \lceil \log_{2} n \rceil$ bits to store a value from \TextIntsMod{n}.

To easily compute the exponentiation of $M$ with $e$, we apply the \emph{square and multiply algorithm}.
\begin{definition}[Square and Multiply Algorithm]\label{def:Square_Multiply_Algorithm}
  The \emph{square and multiply algorithm} is a way to efficiently compute exponents.
  If the expression in question is $M^{e}$, we first rewrite $e$ as a sum of powers of 2.
  \begin{equation*}
    e = e_{0} + e_{1} \cdot 2 + e_{2} \cdot 2^{2} + \cdots + e_{L-1} w^{L-1}
  \end{equation*}
  where $e_{i} \in \IntsMod{2}$ and $0 \leq i \leq L-1$.

  Next, $L-1$ squarings of $M$ occurs.
  \begin{equation*}
    M^{2}, \Bigl( M^{4} = {\left( M^{2} \right)}^{2} \Bigr), \Bigl( M^{4} = {\left( M^{4} \right)}^{2} \Bigr), \ldots
  \end{equation*}
  in \TextIntsMod{n} (i.e.\ reducing modulo $n$ every time).

  Lastly, we perform at most $L-1$ multiplications by computing $M^{e}$ as
  \begin{equation*}
    M^{e} = M^{e_{0}} \cdot {\left( M^{2} \right)}^{e_{1}} \cdot {\left( M^{4} \right)}^{e_{2}} \cdots {\left( M^{2^{L-1}} \right)}^{e_{L-1}}
  \end{equation*}
\end{definition}

%%% Local Variables:
%%% mode: latex
%%% TeX-master: "../EDIN01-Cryptography-Reference_Sheet"
%%% End:
