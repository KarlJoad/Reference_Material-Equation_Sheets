\section{Public-Key Encryption}\label{sec:Public_Key_Encryption}
\begin{definition}[Public-Key Encryption Scheme]\label{def:Public_Key_Encryption_Scheme}
  A public-key encryption scheme is a set of encryption transformations $\lbrace E_{e} : e \in \Keyspace \rbrace$ and a set of decryption transformations $\lbrace D_{d} : d \in \Keyspace \rbrace$.
  For each $e \in \Keyspace$ there is a corresponding $d \in \Keyspace$ such that $D_{d} \bigl(E_{e} (M) \bigr) = M,  \forall M$.

  Furthermore, after choosing such a pair $(e, d)$, the \emph{public key} $e$ (or the \emph{public parameter}) is made public, while the associated \emph{secret key} $d$ is kept secret.
  For the scheme to be secure, it must be computationally infeasible to compute $d$ and/or $E_{e}^{-1}(C)$, knowing the public value $e$.
  These types of schemes are built with \nameref{def:One_Way_Function}s and/or \nameref{def:Trapdoor_One_Way_Function}s.

  This means that the encryption key can be public, allowing anyone to send an encrypted message to the receiver.
  Then, only the receiver can decrypt the message, because the decryption key is kept secret.

  \begin{remark}[Construction]
    These are constructed through multiple \nameref{def:Trapdoor_One_Way_Function}s.
  \end{remark}
\end{definition}

\begin{definition}[One-Way Function]\label{def:One_Way_Function}
  An informal definition of a \emph{one-way function} $f(x)$ is a function from a set $\mathcal{X}$ to a set $\mathcal{Y}$ such that $f(x)$ is easy to compute for all $x \in \mathcal{X}$, but for ``essentially all'' elements $y \in \mathcal{Y}$ it is ``computationally infeasible'' to find any $x \in \mathcal{X}$ such that $f(x) = y$.

  \begin{remark}[``Essentially All'']\label{rmk:One_Way_Function-Essentially_All}
    There are some special values where \nameref{def:One_Way_Function}s to not behave normally, in that the function becomes computationally feasible to solve.
  \end{remark}
\end{definition}

\begin{definition}[Trapdoor One-Way Function]\label{def:Trapdoor_One_Way_Function}
  A \emph{trapdoor one-way function} $f(x)$ is a \nameref{def:One_Way_Function} $f : \mathcal{X} \mapsto \mathcal{Y}$ such that if one knows some specific information $T$, called the \emph{trapdoor information}, then $f(x)$ is computationally easy to invert $f$, i.e., for any $y \in \mathcal{Y}$ it is easy to find a $x \in \mathcal{X}$ such that $f(x) = y$.
  For anyone without knowledge of the trapdoor information $T$, $f(x)$ is a \nameref{def:One_Way_Function}.
\end{definition}

\subsection{RSA Public-Key Encryption Scheme}\label{subsec:RSA_Public_Key_Encryption_Scheme}
\begin{definition}[RSA Public-Key Encryption Scheme]\label{def:RSA_Public_Key_Encryption_Scheme}
  The \emph{RSA public-key encryption scheme} is defined as follows.
  Let $n = pq$, where $p$ and $q$ are two large \nameref{def:Prime}s.
  Let $\Messages = \Ciphertexts = \IntsMod{n}$.
  Pick a number $e$ that is \nameref{def:Relatively_Prime} to $\phi(n)$ (the \nameref{def:Set_Order} of \TextIntsMod{n}) and calculate a number $d$ such that $ed = 1 \bmod \phi(n)$.
  The \emph{public key} is the \underline{two numbers} $(n, e)$ and the public encryption transformation $E(M)$ is
  \begin{equation}\label{eq:RSA_Public_Key_Encryption_Scheme-Encryption}
    E(M) = M^{e} \bmod n
  \end{equation}

  and the associated decryption transformation is
  \begin{equation}
    \label{eq:RSA_Public_Key_Encryption_Scheme-Decryption}
    D(C) = C^{d} \bmod n
  \end{equation}
\end{definition}

\begin{proof}[RSA Public-Key Encryption Scheme]\label{proof:RSA_Public_Key_Encryption_Scheme}
  To verify that the \nameref{def:RSA_Public_Key_Encryption_Scheme} returns the plaintext after it was encrypted by the public key, we first assume that $n$, $e$, and $d$ are all properly defined.
  By extension, this means that $E(M)$ and $D(C)$ are also well-defined.

  We start be substituting the ciphertext, $C$ in \Cref{eq:RSA_Public_Key_Encryption_Scheme-Decryption} with the definition of the cipher text.
  \begin{equation*}
    \begin{aligned}
      D(C) &= C^{d} \bmod n \\
      &= {(M^{e})}^{d} \bmod n \\
      &= M^{ed} \bmod n
    \end{aligned}
  \end{equation*}

  Now, we note that $ed = 1 \bmod \phi(n)$, which means we can write
  \begin{equation*}
    ed = 1 + t \phi(n)
  \end{equation*}
  for some integer $t$.

  So, we continue, and use our two equations together.
  \begin{equation*}
    \begin{aligned}
      D(C) &= M^{ed} \bmod n \\
      &= M^{1 + \phi(n)} \bmod n \\
      &= M \cdot M^{\phi(n)} \bmod n \\
    \end{aligned}
  \end{equation*}

  From Euler's formula, we know $x^{\phi(n)} = 1$ for any $x \in \MultiplicativeGroup{n}$ (From \Cref{def:Multiplicative_Inverse} of \nameref{def:Multiplicative_Inverse}).
  We also assume that $M$ is invertible, otherwise the entire scheme falls apart, since a non-invertible message means it cannot be decrypted once encrypted.
  \begin{equation*}
    \begin{aligned}
      D(C) &= M \cdot M^{\phi(n)} \bmod n \\
      &= (M \cdot 1) \bmod n \\
      &= M \bmod n
    \end{aligned}
  \end{equation*}
\end{proof}

\begin{example}[Lecture 12, Example 3]{RSA Public-Key Encryption Scheme}
  Let $p = 47$ and $q = 167$.
  Then, $n = pq = 7849$.
  Compute $\phi(n)$,
  \begin{equation*}
    \begin{aligned}
      \phi(n) &= \phi(7849) \\
      &= \phi(47 \cdot 167) \\
      &= (47 - 1) (167 - 1) \\
      &= 7636
    \end{aligned}
  \end{equation*}

  Now, we choose a value of $e$ such that $e$ and $\phi(n)$ are \nameref{def:Relatively_Prime}, $\gcd(e, \phi(n)) = 1$.
  We choose $e=25$. (This is just one correct $e$. There are many more.)
  Since we chose $e$ where $\gcd(e, \phi(n)) = 1$, $e$ is a \nameref{def:Multiplicative_Inverse} in $\IntsMod{\phi(n)} = \IntsMod{7636}$.

  We can use the \nameref{def:Euclidean_Algorithm} and Bezout's lemma to find the inverse.
  \begin{equation*}
    d = 2749
  \end{equation*}

  Thus,
  \begin{equation*}
    25 \cdot 2749 = 1 \bmod 7636
  \end{equation*}

  Since this is true, we publish our public key, $(n, e) = (7636, 25)$.
  \tcblower{}
  Now, to send a message, Alice uses \Cref{eq:RSA_Public_Key_Encryption_Scheme-Encryption}.
  Her message has $M = 2728$.
  \begin{equation*}
    \begin{aligned}
      C &= M^{e} \bmod n \\
      &= 2728^{25} \bmod 7849 \\
      &= 2401 \bmod 7849
    \end{aligned}
  \end{equation*}

  When Bob receives this ciphertext, he can decrypt it using \Cref{eq:RSA_Public_Key_Encryption_Scheme-Decryption}.
  \begin{equation*}
    \begin{aligned}
      M &= C^{d} \bmod n \\
      &= 2401^{2749} \bmod 7849 \\
      &= 2728 \bmod 7849
    \end{aligned}
  \end{equation*}

  Thus, Bob gets Alice's original message, and he is likely the only one who was able to decrypt it.
\end{example}

\subsubsection{Security of the \nameref*{subsec:RSA_Public_Key_Encryption_Scheme}}\label{subsubsec:RSA_Encryption_Scheme-Security}
The \nameref{def:RSA_Public_Key_Encryption_Scheme} relies on the factorization problem.
Namely, that for large semi-prime numbers, it is difficult to find its constiuent \nameref{def:Prime} factors.
However, this \textbf{does not} mean that breaking RSA is equivalent to solving a factorization problem.
It is currently unknown if there is a way to break RSA without factoring the integer $n$.

\subsubsection{Implementation of the \nameref*{subsec:RSA_Public_Key_Encryption_Scheme}}\label{subsubsec:RSA_Encryption_Scheme-Implementation}
In order to compute $M^{e} \bmod n$ we need $L = \lceil \log_{2} n \rceil$ bits to store a value from \TextIntsMod{n}.

To easily compute the exponentiation of $M$ with $e$, we apply the \emph{square and multiply algorithm}.
\begin{definition}[Square and Multiply Algorithm]\label{def:Square_Multiply_Algorithm}
  The \emph{square and multiply algorithm} is a way to efficiently compute exponents.

  \begin{algorithm}[H]
    \DontPrintSemicolon{}
    \SetKwInOut{Input}{Input}\SetKwInOut{Output}{Output}

    \Input{An integer $M$ raised to some exponent $e$ modulo some number $n$, written $M^{e} \bmod n$.}
    \Output{The result of the integer exponentiation modulo $n$.}
    \BlankLine{}
    Rewrite $e$ as a sum of powers of 2.
    \begin{equation*}
      e = e_{0} + e_{1} \cdot 2 + e_{2} \cdot 2^{2} + \cdots + e_{L-1} w^{L-1}
    \end{equation*}
    where $e_{i} \in \IntsMod{2}$ and $0 \leq i \leq L-1$. \;
    Next, $L-1$ squarings of $M$ occurs.
    \begin{equation*}
      M^{2}, \Bigl( M^{4} = {\left( M^{2} \right)}^{2} \Bigr), \Bigl( M^{4} = {\left( M^{4} \right)}^{2} \Bigr), \ldots
    \end{equation*}
    in \TextIntsMod{n} (i.e.\ reducing modulo $n$ every time). \;

    Lastly, we perform at most $L-1$ multiplications by computing $M^{e}$ as
    \begin{equation*}
      M^{e} = M^{e_{0}} \cdot {\left( M^{2} \right)}^{e_{1}} \cdot {\left( M^{4} \right)}^{e_{2}} \cdots {\left( M^{2^{L-1}} \right)}^{e_{L-1}}
    \end{equation*}
    \caption{Square and Multiply Algorithm}
    \label{algo:Square_Multiply_Algorithm}
  \end{algorithm}
\end{definition}

\begin{example}[Lecture 14, Example 1]{Square and Multiply Algorithm}
  Compute $2728^{25}$ in \TextIntsMod{7849} using the \nameref{def:Square_Multiply_Algorithm}?
  \tcblower{}
  First, rewrite 25 as a result of products of 2.
  \begin{align*}
    25 &= 16 + 8 + 1 \\
    &= 1 (2^{4}) + 1 (2^{3}) + 0 (2^{2}) + 1 (2^{1}) + 1 (2^{0})
  \end{align*}
  This results in
  \begin{equation*}
    2728^{25} = 2728^{1} \cdot 2728^{8} \cdot 2728^{16}
  \end{equation*}

  Now we perform the squarings required until we get to $2728^{16}$.
  \begin{align*}
    2728^{2} &= 7441984 \bmod 7849 = 1132 \\
    2728^{4} &= {(2728^{2})}^{2} = 1132^{2} \bmod 7849 = 2037 \\
    2728^{8} &= {(2728^{4})}^{2} = 2037^{2} \bmod 7849 = 5097 \\
    2728^{16} &= {(2728^{8})}^{2} = 5097^{2} \bmod 7849 = 7068
  \end{align*}

  Now substituting our values in and multiplying through (ensuring we reduce modulo $n$ at the end).
  \begin{align*}
    2728^{25} &= 2728^{16+8+1} \\
              &= (2728 \cdot 5097 \cdot 7068) \bmod 7849 \\
              &= 2401
  \end{align*}

  Our answer is $2728^{25} \bmod 7849 = 2401$.
\end{example}

\subsection{Primality Testing}\label{subsec:Primality_Testing}
How do we check whether a given number $m$ is a \nameref{def:Prime} number?

The na\"{\i}ve approach said to perform trial divisions such that $x \Divides m$ for all integers $x$ where $2 \leq x \leq \sqrt{m}$.
The problem with this is that if $m$ is a 1024-bit number, for instance, then $\sqrt{m}$ is 512-bits, and performing $2^{512}$ tests is not possible.

So, we have probabilistic theorems for determining primality of numbers.

\subsubsection{Probabilistic Algorithms for Testing Primality}\label{subsubsec:Probabilistic_Algorithms_Testing_Primality}
\paragraph{Fermat's Little Theorem}\label{par:Fermats_Little_Theorem}
\begin{theorem}[Fermat's Little Theorem]\label{thm:Fermats_Little_Theorem}
  Fermat's Little Theorem states that
  \begin{equation}\label{eq:Fermats_Little_Theorem}
    a^{m-1} = 1 \bmod m
  \end{equation}
  if $m$ is a \nameref{def:Prime} and $a$ such that $1 \leq a \leq m-1$.

  When $m$ is not a \nameref{def:Prime}, we cannot know what we will receive when computing $a^{m-1}$.
  However, if $a^{m-1} \neq 1 \bmod m$, we know \textbf{for certain} that $m$ is \textbf{not} a \nameref{def:Prime} number.
\end{theorem}

\begin{remark*}[Error Probability of \nameref{thm:Fermats_Little_Theorem}]
  The error probability after repeating the test $k$ times would be less than $\frac{1}{2^{k}}$.
\end{remark*}

\begin{definition}[Pseudo-Prime]\label{def:Pseudo_Prime}
  A composite $m$ such that $a^{m-1} = 1 \bmod m$ is said to be a \emph{pseudo-prime} to the base $a$.
\end{definition}

\begin{definition}[Carmichael Number]\label{def:Carmichael_Number}
  A number is called a \emph{Carmichael number} if a composite $m$ is such that it is a \nameref{def:Pseudo_Prime} for every base $a$ with $\gcd(a, m) = 1$.

  \begin{remark}[Smallest \nameref*{def:Carmichael_Number}]
    The smallest \nameref{def:Carmichael_Number} is 561.
  \end{remark}
\end{definition}

\begin{theorem}[Miller-Rabin Test]\label{thm:Miller_Rabin_Test}
  The Miller-Rabin test states that we choose a random integer as base, $a$.
  Then write $m -1 = 2^{b} \cdot q$, where $q$ is odd.

  If $a^{q} = 1 \bmod m$ or $a^{2^{c} \cdot q} = -1 \bmod m$ for any $c < b$ then return ``$m$ probably \nameref{def:Prime}'' and go back and choose a new base.
  Otherwise, return ``$m$ is definitely \textbf{not} \nameref{def:Prime}''.

  One can prove that the probability that $m$ is probably \nameref{def:Prime} with to the below equation.
  \begin{equation}\label{eq:Miller_Rabin_Test_Success_Probability}
    \Prob \left( \text{``$m$ probably prime''} \Given \text{``$m$ not prime''} \right) < \frac{1}{4}
  \end{equation}
\end{theorem}

\subsection{Factoring}\label{subsec:Factoring}
All \nameref{def:Public_Key_Encryption_Scheme}s used today rely on the intractability of factoring large semi-primes.

\subsubsection{\texorpdfstring{Pollard's $(p-1)$-Method}{Pollard's Method}}\label{subsubsec:Pollards_Method}
\begin{theorem}[\texorpdfstring{Pollard's $(p-1)$-Method}{Pollard's Method}]\label{thm:Pollards_Method}
  Let $n = pq$ and let $p-1$ factor as
  \begin{equation*}
    p-1 = q_{1}q_{2} \cdots q_{k}
  \end{equation*}
  where each $q_{i}$ is a \nameref{def:Prime} power.

  There is a condition where each $q_{i} < B$ where $B$ is a predetermined ``bound''.
  If this condition is valid, then we have
  \begin{equation*}
    (p-1) \Divides B!
  \end{equation*}

  We compute $a = 2^{B!} \bmod n$.
  Now let $a' = 2^{B!} \bmod p$.
  Since $p \Divides n$, we must have $a \bmod p = a'$.

  \nameref{thm:Fermats_Little_Theorem} states that
  \begin{equation*}
    2^{p-1} = 1 \bmod p
  \end{equation*}
  and since $(p-1) \Divides B!$, we also have $a' = 1 \bmod p$, which leads to $a = 1 \bmod p$.

  So, $p \Divides (a-1)$ and since $p \Divides n$, we also have $p = \gcd(a-1, n)$.
\end{theorem}

The algorithm stands as:
\begin{enumerate}[noitemsep]
\item Compute $a^{2B!} \bmod n$.
\item Compute the unknown prime $p$ as $p = \gcd(a-1, n)$.
\end{enumerate}

An important conclusion to draw here is that the \nameref{def:Prime}s $p$ and $q$ must be chosen such that $p-1$ and $q-1$ each contains a large \nameref{def:Prime} in their factorization.
The usual approach is to generate a random \nameref{def:Prime} $p_{1}$ and test whether $p = 2p_{1} + 1$ is a prime number.
If so, we choose $p$.

\subsubsection{Other Factoring Methods}\label{subsubsec:Other_Factoring_Methods}
There are many modern factoring methods that are more efficient than \nameref{thm:Pollards_Method}.
\begin{enumerate}[noitemsep]
\item Quadratic Sieve
\item Number Field Sieve
\item Elliptic Curve Factorization
\item Pollard's Rho Algorithm Method
\item etc.
\end{enumerate}

\subsubsection{Computational Complexity of Factoring}\label{subsubsec:Factoring_Computational_Complexity}
We often use the function to model complexity of algorithms with sub-exponential behavior.
\begin{equation}\label{eq:Sub_Exponential_Behavior}
  L_{N}(\alpha, \beta) = e^{\bigl( \beta + O(1){(\log(N))}^{\alpha} {\log(\log(N))}^{1-\alpha}\bigr)}
\end{equation}

Number field sieves are currently the most successful method for factoring semi-\nameref{def:Prime} numbers with more than 100 decimal digits.
It can factor numbers of the size $2^{512}$
Its complexity behavior is sub-exponential, with $L_{N}(\frac{1}{3}, 1.923)$.

\subsection{Uses of Public-Key Cryptography}\label{subsec:Uses_Public_Key_Cryptography}
There are 3 basic \nameref{def:Cryptographic_Primitive}s that \nameref{def:Public_Key_Encryption_Scheme}s are used for.
\begin{enumerate}[noitemsep]
\item \nameref{def:Public_Key_Encryption_Scheme}: A messis encrypted with a recipient’s public key and cannot be decrypted by anyone except the recipient possessing the private key.
\item \nameref{subsec:Digital_Signatures}: A message signed with a sender’s private key can be verified by anyone who has access to the sender’s public key, thereby proving that the correct original sender signed it and that the message has not been tampered with during message transmission (authenticity and nonrepudiation).
\item \nameref{subsec:Key_Exchange}: A cryptographic protocol that allows two parties that have no prior knowledge of each other to jointly establish a shared secret key.
\end{enumerate}

The major problem with using a public key is linking it an entity or principal.
The most common solution to this problem is the use of a \nameref{def:Digital_Certificate}.

\begin{definition}[Digital Certificate]\label{def:Digital_Certificate}
  A \emph{digital certificate} is a ``stamp of approval'' that the public key belongs to who the principal says it belongs to.
  There is a third-party called a \nameref{def:Certificate_Authority} that handles the signing of entity's/principal's public keys to ensure they are authentic.

  A digital certificate has the form of
  \begin{center}
    (Alice, $\ldots$, Alice's public key, CA's signature)
  \end{center}
\end{definition}

\begin{definition}[Certificate Authority]\label{def:Certificate_Authority}
  A \emph{certificate authority} or (\emph{CA}) is a third-party that vouches for the validity of the public keys in use by an entity/principal.
\end{definition}

\subsubsection{Digital Certificates and Certificate Authorities}\label{subsubsec:Digital_Certificate_Certificate_Authorities}
A \nameref{def:Certificate_Authority}-based system works by establishing a ``web of trust''.
\begin{enumerate}[noitemsep]
\item All users have a trusted copy of the public key of the \textbf{\nameref{def:Certificate_Authority}}. For example, these are embedded in your web browser.
\item The \nameref{def:Certificate_Authority} signs data strings containing the following information.
  \begin{center}
    (Alice, $\ldots$, Alice's public key, CA's signature)
  \end{center}
  This data string and the associated signature is called a \nameref{def:Digital_Certificate}.
  The \nameref{def:Certificate_Authority} only signs the data if it believes the public key really belongs to Alice.
\item When Alice sends her public key, contained in the digital certificate, you now trust that the public key is really Alice's, since you trust the \nameref{def:Certificate_Authority} and you checked their signature too.
\end{enumerate}

\subsection{The Discrete Logarithm Problem}\label{subsec:Discrete_Log_Problem}
\begin{definition}[Discrete Logarithm Problem]\label{def:Discrete_Log_Problem}
  Let $(G, *)$ be an \nameref{def:Abelian} \nameref{def:Group}.
  The \emph{discrete logarithm problem} states that given $g, h \in G$, find an $x$ (if it exists) such that $g^{x} = h$.

  The difficulty of this problem depends on the group $G$ chosen.
  \begin{itemize}[noitemsep]
  \item Very Easy: Polynomial Time Algorithm. $(\IntsModN{}, +)$
  \item Hard: Sub-Exponential Time Algorithm. $(\FiniteMathField{F}{p}{*})$
  \item Very Hard: Exponential Time Algorithm. Elliptic Curve Groups (Not discussed in this course).
  \end{itemize}
\end{definition}

\subsection{Key Exchange}\label{subsec:Key_Exchange}
\begin{definition}[Key Exchange]\label{def:Key_Exchange}
  \emph{Key exchange} is a cryptographic protocol that allows two parties that have no prior knowledge of each other to jointly establish a shared secret key.
\end{definition}

\subsubsection{Diffie Hellman Key Exchange}\label{subsubsec:Diffie_Hellman_Key_Exchange}
\begin{definition}[Diffie Hellman Key Exchange]\label{def:Diffie_Hellman_Key_Exchange}
  The \emph{Diffie Hellman Key Exchange} is a type of \nameref{def:Key_Exchange} that allows 2 parties to agree to a secret key over an insecure channel without having met before.

  Let $G = \FiniteMathField{F}{p}{*}$ and $g \in \FiniteMathField{F}{p}{*}$.
  The basic message flows for this protocol is:
  \begin{enumerate}[noitemsep]
  \item Alice selects a secret $a$, and Bob selects a secret $b$.
  \item Alice sends $A = g^{a} \bmod p$ to Bob.
  \item Bob sends $B = g^{b} \bmod p$ to Alice.
  \item Alice computes $K_{A} = B^{a} \bmod p = g^{b^{a}} = g^{ba}$.
  \item Bob computes $K_{B} = A^{b} \bmod p= g^{a^{b}} = g^{ab}$.
  \item If both $K_{A} = K_{B}$, then the secret key has been created for Alice and Bob.
  \end{enumerate}

  \begin{remark}[Flaws in \nameref*{def:Diffie_Hellman_Key_Exchange}]\label{rmk:Flaws_Diffie_Hellman_Key_Exchange}
    While this would work for most cases, the entire \nameref{def:Diffie_Hellman_Key_Exchange} protocol is vulnerable to a \emph{Man-in-the-Middle attack}.
  \end{remark}
\end{definition}

\begin{example}[Lecture 15, Example 1]{Diffie Hellman Key Exchange}
  Given the comain parameters $p = 2147482659$, $g = 2$ and the secret keys chosen by Alice $a = 12345$ and Bob $b = 654323$, compute their shared key?
  \tcblower{}
  \begin{align*}
    &\text{Alice} & &\text{Bob} \\
    a &= 12345 & b &=654323 \\
    A &= g^{a} = 2^{12345} \bmod 2147482659 = 428647416 & B &= g^{b} = 2^{654323} \bmod 2147482659 = 450904856 \\
    K_{A} &= B^{a} \bmod 2147482659 & K_{B} &= A^{b} \bmod 2147482659\\
    &= 450904856^{12345} \bmod 2147482659 & &= 428647416^{654323} \bmod 2147482659 \\
    &= 1333327162 & &= 1333327162
  \end{align*}

  Thus, Alice and Bob have managed to agree on a secret key of $K=1333327162$.
\end{example}

%%% Local Variables:
%%% mode: latex
%%% TeX-master: "../EDIN01-Cryptography-Reference_Sheet"
%%% End:
