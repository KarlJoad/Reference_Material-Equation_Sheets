\section{Information Theory}\label{sec:Information_Theory}
Information theory is heavily influenced by probability and its theories.
For \textbf{much} greater detail on some of the concepts presented here, refer to the \href{file:./Math_374-Reference_Sheet.pdf}{Math 374 - Probability and Statistics} document.

\begin{definition}[Random Experiment]\label{def:Random_Experiment}
  A \emph{random experiment} is an experiment whose outcome varies in an unpredictable fashion when performed under the same conditions.
\end{definition}

\subsection{Sample Space}\label{subsec:Sample_Space}
\begin{definition}[Sample Space]\label{def:Sample_Space}
  The \emph{sample space} is the set of \textbf{all} possible outcomes, the \nameref{def:Elementary_Event}s, denoted
  \begin{equation}\label{eq:Sample_Space}
    \Omega = \lbrace \omega_{1}, \omega_{2}, \ldots, \omega_{n} \rbrace
  \end{equation}
\end{definition}

\subsection{Discrete Random Variables}\label{subsec:Discrete_Random_Variables}
\subsubsection{Independent Discrete Random Variables}\label{subsubsec:Independent_Discrete_Random_Variables}
\subsubsection{Conditional Probability of  Discrete Random Variables}\label{subsubsec:Conditional_Probability_Discrete_Random_Variables}

\subsection{Entropy}\label{subsec:Entropy}
\subsubsection{Properties of \nameref*{subsec:Entropy}}\label{subsubsec:Entropy_Properties}

\subsubsection{Conditional Entropy}\label{subsubsec:Conditional_Entropy}
\paragraph{Properties of \nameref*{subsubsec:Conditional_Entropy}}\label{par:Conditional_Entropy_Properties}

\subsubsection{Relative Entropy}\label{subsubsec:Relative_Entropy}

\subsection{Mutual Information}\label{subsec:Mutual_Information}
\subsubsection{Properties of \nameref*{subsec:Mutual_Information}}\label{subsubsec:Mutual_Information_Properties}
%%% Local Variables:
%%% mode: latex
%%% TeX-master: "../EDIN01-Cryptography-Reference_Sheet"
%%% End:
