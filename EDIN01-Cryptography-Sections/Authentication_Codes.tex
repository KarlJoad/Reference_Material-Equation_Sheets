\section{Authentication}\label{sec:Authentication}
There are 3 ways we can confirm authenticity with authentication schemes:
\begin{enumerate}[noitemsep]
\item Unconditionally Secure \nameref{subsec:Authentication_Codes}
\item \nameref{def:MAC}
\item \nameref{subsec:Digital_Signatures}
\end{enumerate}

\subsection{\nameref*{subsec:MACs} with Authentication}\label{subsec:MAC_Authentication}
\nameref{def:MAC}s can be used as an authentication technique that uses \nameref{def:Cryptographic_Primitive}s (\nameref{def:Block_Cipher}s and \nameref{def:Hash_Function}s) to provide authentication.

\begin{remark*}
  It is assumed that the sender and reciever both share the same key used for the \nameref{def:MAC}.
\end{remark*}

\subsubsection{Problems with \nameref*{subsec:MACs} and Authentication}\label{subsubsec:MAC_Problems_Authentication}
\nameref{def:MAC}s do not protect against an unlimited enemy, i.e.\ an unlimited number of attackers or an attacker with infinite computing power.
However, they are able to authenticate many messages without changing the key.

\subsubsection{CBC-MAC for Authentication}\label{subsubsec:CBC_MAC_Authentication}
\subsection{Digital Signatures}\label{subsec:Digital_Signatures}
\subsection{Authentication Codes}\label{subsec:Authentication_Codes}
\begin{definition}[Authentication Code]\label{def:Authentication_Code}
\end{definition}

\subsubsection{The Unconditionally Secure Model}\label{subsubsec:Authentication_Code_Unconditionally_Secure_Model}
\subsubsection{Attacks on Authentication Codes}\label{subsubsec:Attacks_Authentication_Codes}
  
\paragraph{Impersonation Attacks}\label{par:Attack_Authentication_Code-Impersonation}
\paragraph{Substitution Attacks}\label{par:Attack_Authentication_Code-Substitution}
\subsection{Systematic Authentication Codes}\label{subsec:Systematic_Authentication_Codes}
\end{definition}

%%% Local Variables:
%%% mode: latex
%%% TeX-master: "../EDIN01-Cryptography-Reference_Sheet"
%%% End:
