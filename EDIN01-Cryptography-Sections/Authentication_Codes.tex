\section{Authentication}\label{sec:Authentication}
There are 3 ways we can confirm authenticity with authentication schemes:
\begin{enumerate}[noitemsep]
\item Unconditionally Secure \nameref{subsec:Authentication_Codes}
\item \nameref{def:MAC}
\item \nameref{subsec:Digital_Signatures}
\end{enumerate}

\subsection{\nameref*{subsec:MACs} with Authentication}\label{subsec:MAC_Authentication}
\nameref{def:MAC}s can be used as an authentication technique that uses \nameref{def:Cryptographic_Primitive}s (\nameref{def:Block_Cipher}s and \nameref{def:Hash_Function}s) to provide authentication.

\begin{remark*}
  It is assumed that the sender and reciever both share the same key used for the \nameref{def:MAC}.
\end{remark*}

\subsubsection{Problems with \nameref*{subsec:MACs} and Authentication}\label{subsubsec:MAC_Problems_Authentication}
\nameref{def:MAC}s do not protect against an unlimited enemy, i.e.\ an unlimited number of attackers or an attacker with infinite computing power.
However, they are able to authenticate many messages without changing the key.

\subsubsection{CBC-MAC for Authentication}\label{subsubsec:CBC_MAC_Authentication}
CBC-MAC is a \nameref{def:Mode_of_Operation} for \nameref{def:Block_Cipher}s to operate with \nameref{def:MAC}s.
A \nameref{def:Block_Cipher} is secure for fixed-length messages, but insecure for variable-length messages.
The problem is that the same key $K$ cannot be reused anywhere.

\subsection{Digital Signatures}\label{subsec:Digital_Signatures}
This is an asymmetric (\nameref{def:Public_Key_Encryption_Scheme}) solution.
The advantages of this are:
\begin{itemize}[noitemsep]
\item No need to distribute/establish a common secret key
\item Nonrepudiation, if the receiver received an authentic message, the sender cannot deny having sending it
\end{itemize}

The disadvantages of this are:
\begin{itemize}[noitemsep]
\item Signature schemes rely on the hardness of problems, like semiprime integer factoring.
\item These signature schemes also rely on large numbers.
\item Both of these factors slow down the process of this technique compared to others.
\end{itemize}

\subsection{Authentication Codes}\label{subsec:Authentication_Codes}
\begin{definition}[Authentication Code]\label{def:Authentication_Code}
  An \emph{authentication code} is used to check if the received message was sent by the claimed sender.
  They are also used to verify that the message was not modified (by a third-party, not random errors) during transmission.

  The authentication code requires there be \emph{secret keys} that are known to the sender and receiver, but not to the enemy.

  There can be several models of authentication codes.
  In this course, we only look at \nameref{subsubsec:Authentication_Code_Unconditionally_Secure_Model}.
  The formula defining this type of authentication code is given in \Cref{eq:Unconditional_Authentication_Code}.
\end{definition}

\subsubsection{The Unconditionally Secure Model}\label{subsubsec:Authentication_Code_Unconditionally_Secure_Model}
An unconditionally secure solution is given below.
\begin{itemize}[noitemsep]
\item The transmitted information, the \emph{source message} (\nameref{def:Plaintext}) is denoted $s$ and $s \in \SourceMessages$.
\item The source message is mapped to a \emph{channel message} $m$ where $m \in \ChannelMessages$.
\item Using the secret key $k$ where $k \in \Keyspace$.
\end{itemize}

\begin{equation}\label{eq:Unconditional_Authentication_Code}
  f : \SourceMessages \times \Keyspace \rightarrow \mathcal{M} : (s, k) \mapsto m
\end{equation}
\begin{itemize}[noitemsep]
\item An important property of $f$ is that if $f(s, e) = m$ and $f(s', e) = m$, then $s = s'$ (Injective for each $k \in Keyspace$).
\item Meaning the receiver must check whether a source message $s$ even exists.
\item If such an $s$ exists, $m$ is valid.
\item Otherwise, the message $m$ is not authentic.
\end{itemize}

\begin{large}
  \textbf{The mapping $f$, along with $\SourceMessages$, $\ChannelMessages$, and $\Keyspace$ define an \nameref{def:Authentication_Code} (\emph{A-Code}).}
\end{large}

\subsubsection{Attacks on Authentication Codes}\label{subsubsec:Attacks_Authentication_Codes}
There are only 2 reasonable attacks on \nameref{def:Authentication_Code}s possible.
\begin{enumerate}[noitemsep]
\item \nameref{par:Attack_Authentication_Code-Impersonation}
\item \nameref{par:Attack_Authentication_Code-Substitution}
\end{enumerate}

\begin{definition}[Probability of Deception]\label{def:Probability_of_Deception}
  The \emph{probability of deception} is the probability that a non-authentic message is authenticated by the \nameref{def:Authentication_Code} system as a valid message.
  It is based off probabilities of success of \nameref{def:Attack_Authentication_Code-Impersonation}s and \nameref{def:Attack_Authentication_Code-Substitution}s.
  
  \begin{equation}\label{eq:Probability_of_Deception}
    \Prob_{D} = \max(\Prob_{i}, \Prob_{S})
  \end{equation}
\end{definition}

\begin{theorem}[Square Root Bound]\label{thm:Prob_Deception_Square_Root_Bound}
  By first multiplying the Simmons' Bounds on the probabilities of success of each attack, we can find their combined probability.
  Using \Cref{thm:Impersonation-Simmons_Bound} and \Cref{thm:Substitution-Simmons_Bound} and multiplying:
  \begin{equation*}
    \Prob_{I}\Prob_{S} \geq 2^{-\MutualInformation(M; K) - \Entropy(K \Given M)} = 2^{-\Entropy(K)}
  \end{equation*}

  From $\Entropy(K) \leq \log_{2} \SetOrder{\Keyspace}$ we can find the \emph{square root bound} of the \nameref{def:Probability_of_Deception} \textbf{for any \nameref{def:Authentication_Code}}.
  \begin{equation}\label{eq:Prob_Deception_Square_Root_Bound}
    \Prob_{D} \geq \frac{1}{\sqrt{\SetOrder{\Keyspace}}}
  \end{equation}
\end{theorem}

\begin{theorem}[Tightness of the \nameref*{thm:Prob_Deception_Square_Root_Bound}]\label{thm:Prob_Deception_Square_Root_Bound_Tightness}
  The \nameref{thm:Prob_Deception_Square_Root_Bound} can only be tight if
  \begin{equation}\label{eq:10}
    \SetOrder{\SourceMessages} \leq \sqrt{\SetOrder{\Keyspace}} + 1
  \end{equation}

  Meaning a large source size demands the key be twice as large.
  This is not very practical.
\end{theorem}

\paragraph{Impersonation Attacks}\label{par:Attack_Authentication_Code-Impersonation}
\begin{definition}[Impersonation Attack]\label{def:Attack_Authentication_Code-Impersonation}
  An \emph{impersonation attack} is a form of a Man-in-the-Middle attack where the attacker inserts a message $m'$ into the channel, and hope that it is accepted as authentic.

  The equation to determine the probability of success is shown below.
  \begin{equation}\label{eq:Attack_Authentication_Code-Impersonation-Success}
    \Prob_{I} = \max_{m'} P(m' \text{ is valid})
  \end{equation}
\end{definition}

\begin{theorem}[Bounds of \nameref*{def:Attack_Authentication_Code-Impersonation} Success]\label{thm:Attack_Authentication_Code-Impersonation-Success_Bounds}
  For any \nameref{def:Authentication_Code},
  \begin{equation}\label{eq:Attack_Authentication_Code-Impersonation-Success_Bound}
    \Prob_{I} \geq \frac{\SetOrder{\SourceMessages}}{\SetOrder{\ChannelMessages}}
  \end{equation}
  where $\SetOrder{\ChannelMessages} >> \SetOrder{\SourceMessages}$.
\end{theorem}

\begin{theorem}[Simmons' Bound for \nameref*{def:Attack_Authentication_Code-Impersonation}]\label{thm:Impersonation-Simmons_Bound}
  For any \nameref{def:Authentication_Code},
  \begin{equation}\label{eq:Impersonation-Simmons_Bound}
    \Prob_{I} \geq 2^{-\MutualInformation(M; E)}
  \end{equation}

  Meaning, for good protection $\Prob_{I}$ is small, we must give away a lot of information about they key.
\end{theorem}

\paragraph{Substitution Attacks}\label{par:Attack_Authentication_Code-Substitution}
\begin{definition}[Substitution Attack]\label{def:Attack_Authentication_Code-Substitution}
  A \emph{substitution attack} is one in which an attacker observes a valid message sent by the sender, $m$, and replaces it with their message $m'$, where $m \neq n'$.
  The attacker then hopes that the system recognizes $m'$ as valid.

  The equation to determine the probability of success is shown below.
  \begin{equation}\label{eq:Attack_Authentication_Code-Substitution-Success}
    \begin{aligned}
      \Prob_{S} &= \underset{m \neq m'}{\max_{m,m'}} \Prob(m' \text{ is valid} \Given m \text{ is valid}) \\
      &= \underset{m \neq m'}{\max_{m,m'}} \frac{\SetOrder{\Keyspace(m) \cap \Keyspace(m')}}{\SetOrder{\Keyspace(m)}}
    \end{aligned}
  \end{equation}
  which requires that they keys $k \in \Keyspace$ be uniformly distributed.
\end{definition}

\begin{theorem}[Bounds of \nameref*{def:Attack_Authentication_Code-Substitution} Success]\label{thm:Attack_Authentication_Code-Substitution-Success_Bounds}
  For any \nameref{def:Authentication_Code},
  \begin{equation}\label{eq:Attack_Authentication_Code-Substitution-Success_Bound}
    \Prob_{S} \geq \frac{\SetOrder{\SourceMessages} - 1}{\SetOrder{\ChannelMessages} - 1}
  \end{equation}
  where $\SetOrder{\ChannelMessages} >> \SetOrder{\SourceMessages}$.
\end{theorem}

\begin{theorem}[Simmons' Bound for \nameref*{def:Attack_Authentication_Code-Substitution}]\label{thm:Substitution-Simmons_Bound}
  For any \nameref{def:Authentication_Code},
  \begin{equation}\label{eq:Substitution-Simmons_Bound}
    \Prob_{S} \geq 2^{-\Entropy(K \Given M)}, \text{ if } \SetOrder{\SourceMessages} \geq 2
  \end{equation}
\end{theorem}

\subsubsection{Constructing \nameref*{def:Authentication_Code}s}\label{subsubsec:Authentication_Code-Construction}
Define $\Keyspace(m)$ as the set of keys for which a message $m$ is valid.
\begin{equation}\label{eq:Authentication_Code-Keys_Message_Valid}
  \Keyspace(m) = \lbrace k \in \Keyspace; exists s \in \SourceMessages, f(s, e) = m \rbrace
\end{equation}

\subsection{Systematic Authentication Codes}\label{subsec:Systematic_Authentication_Codes}
\begin{definition}[Systematic Authentication Code]\label{def:Systematic_Authentication_Code}
  An \nameref{def:Authentication_Code} for which the mapping function $f : \SourceMessages \times \Keyspace \rightarrow \ChannelMessages$ can be written in the form
  \begin{equation}\label{eq:Systematic_Authentication_Codes}
    f : \SourceMessages \times \Keyspace \rightarrow \SourceMessages \times \MessageTags : (s, k) \mapsto (s, z)
  \end{equation}
  where $s \in \SourceMessages$ and $z \in \MessageTags$ is called a \emph{systematic authentication code} (or \emph{Cartesian authentication code}).
  $z$ is the \emph{tag} (or \emph{authenticator}) and is taken from the tag alphabet $\MessageTags$.
\end{definition}

\begin{theorem}[Bounds of Attacks on \nameref*{def:Systematic_Authentication_Code}s]\label{thm:Bounds_Systematic_Authentication_Code}
  For any \nameref{def:Systematic_Authentication_Code},
  \begin{equation}\label{eq:Bounds_Systematic_Authentication_Code}
    \Prob_{S} \geq \Prob_{I}
  \end{equation}
\end{theorem}

\subsubsection{Vector Space Construction}\label{subsubsec:Systematic_Authentication_Code-Vector_Space}
Let $\SetOrder{\SourceMessages} = q^{n}$, $\SetOrder{\MessageTags} = q^{n}$, and $\SetOrder{\Keyspace} = q^{2n}$.
Decompose the keys as $k = (k_{1}, k_{2})$, where $s, z, k_{1}, k_{2} \in \FiniteMathField{F}{q}{m}$
For the transmission of a source message $s$, generate a message $m = (s, z)$, where
\begin{equation*}
  z = k_{1} + sk_{2}
\end{equation*}

\subsubsection{Polynomial Evaluation Construction}\label{subsubsec:Systematic_Authentication_Code-Polynomial_Evaluation}
Let $\SourceMessages = \lbrace \mathbf{s} = (s_{1}, s_{2}, \ldots, s_{n}); s_{i} \in \FiniteMathField{F}{q}{} \rbrace$.
Define the source message \nameref{def:Polynomial} to be
\begin{equation*}
  s(x) = s_{1}x + s_{2}x^{2} + \cdots + s_{n}x^{n}
\end{equation*}.
Let $\Keyspace = \lbrace k = (k_{1}, k_{2}); k_{1}, k_{2} \in \FiniteMathField{F}{q}{} \rbrace$ and $\MessageTags = \FiniteMathField{F}{q}{}$.
For the transmission of the source message $\mathbf{s}$, the sender sends $\mathbf{s}$ together with the tag
\begin{equation*}
  z = k_{1} + s(k_{2})
\end{equation*}
$s(k_{2})$ is the application of the source message polynomial where $x$ is replaced with $k_{2}$.

\begin{theorem}[\nameref*{def:Systematic_Authentication_Code} Parameters]\label{thm:Systematic_Authentication_Code Parameters}
  This \nameref{subsubsec:Systematic_Authentication_Code-Polynomial_Evaluation} given a \nameref{def:Systematic_Authentication_Code} has parameters
  \begin{equation}\label{eq:Systematic_Authentication_Code Parameters}
    \begin{aligned}
      \SetOrder{\SourceMessages} &= q^{n} \\
      \SetOrder{\Keyspace} &= q^{2} \\
      \SetOrder{\MessageTags} &= q \\
      \Prob_{I} &= \frac{1}{q} \\
      \Prob_{S} &= \frac{k}{q} \\
    \end{aligned}
  \end{equation}
\end{theorem}

%%% Local Variables:
%%% mode: latex
%%% TeX-master: "../EDIN01-Cryptography-Reference_Sheet"
%%% End:
