\section{Classical Cryptography}\label{sec:Classical_Cryptography}
\begin{definition}[Cryptosystem]\label{def:Cryptosystem}
  A \emph{cryptosystem} five-tuple $(\Plaintexts, \Ciphertexts, \Keyspace, \EncryptionRules, \DecryptionRules)$ where the following conditions are satisfied:
  \begin{enumerate}[noitemsep]
  \item $\Plaintexts$ is a finite set of possible \emph{\nameref{def:Plaintext}}
  \item $\Ciphertexts$ is a finite set of possible \emph{\nameref{def:Ciphertext}}
  \item $\Keyspace$, the \emph{\nameref{def:Keyspace}}, is a finite set of all possible keys
  \item For each $K \in \Keyspace$, there is an \emph{\nameref{def:Encryption_Rule}} $e_{K}\in \EncryptionRules$ and a corresponding \emph{\nameref{def:Decryption_Rule}} $d_{K} \in \DecryptionRules$.
    Each $e_{K} : \Plaintexts \rightarrow \Ciphertexts$ and $d_{K} : \Ciphertexts \rightarrow \Plaintexts$ are functions such that $d_{K} \left( e_{K}(x) \right) = x$ for every \nameref{def:Plaintext} element $x \in \Plaintexts$.
  \end{enumerate}
\end{definition}

\begin{definition}[Plaintext]\label{def:Plaintext}
  \emph{Plaintext} is the information that Alice wants to send to Bob, and is denoted $\Plaintexts$
  This information can be in any arbitrary format, we do not care.
  However, Alice does not want Oscar to be able to understand what she sends to Bob.
\end{definition}

\begin{definition}[Ciphertext]\label{def:Ciphertext}
  A \emph{ciphertext} is a piece of \nameref{def:Plaintext} information that has been run through an element of \nameref{def:Encryption_Rule} set.
\end{definition}

\begin{definition}[Keyspace]\label{def:Keyspace}
  \emph{Keyspace}, denoted $\Keyspace$.
  \textbf{TODO}
\end{definition}

\begin{definition}[Encryption Rule]\label{def:Encryption_Rule}
  The \emph{encryption rule} is an element $e_{K}$ from the set of all encryption rules, $\EncryptionRules$.
  \begin{equation}\label{eq:Encryption_Rule}
    e_{K} : \Plaintexts \rightarrow \Ciphertexts \text{where } e_{K} \in \EncryptionRules
  \end{equation}
  Namely, the encryption rule element is used to map the \nameref{def:Plaintext} pieces of information that Alice wants to send to a corresponding \nameref{def:Ciphertext} that she can send to Bob.
\end{definition}

\begin{definition}[Decryption Rule]\label{def:Decryption_Rule}
  The \emph{decryption rule} is an element $d_{K}$ from the set of all decryption rules, $\DecryptionRules$.
  \begin{equation}\label{eq:Decryption_Rule}
    d_{K} : \Ciphertexts \rightarrow \Plaintexts \text{where } d_{K} \in \DecryptionRules
  \end{equation}
    Namely, the decryption rule element is used to map the \nameref{def:Ciphertext} pieces of information that Alice sent to a corresponding \nameref{def:Plaintext} that Bob can use.
\end{definition}
%%% Local Variables:
%%% mode: latex
%%% TeX-master: "../EDIN01-Cryptography-Reference_Sheet"
%%% End:
