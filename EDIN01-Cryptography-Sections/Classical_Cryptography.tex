\section{Classical Cryptography}\label{sec:Classical_Cryptography}
\begin{definition}[Cryptosystem]\label{def:Cryptosystem}
  A \emph{cryptosystem} five-tuple $(\Plaintexts, \Ciphertexts, \Keyspace, \EncryptionRules, \DecryptionRules)$ where the following conditions are satisfied:
  \begin{enumerate}[noitemsep]
  \item $\Plaintexts$ is a finite set of possible \emph{\nameref{def:Plaintext}}, producing a message, $\Message$.
  \item $\Ciphertexts$ is a finite set of possible \emph{\nameref{def:Ciphertext}}
  \item $\Keyspace$, the \emph{\nameref{def:Keyspace}}, is a finite set of all possible keys
  \item For each $K \in \Keyspace$, there is an \emph{\nameref{def:Encryption_Rule}} $e_{K}\in \EncryptionRules$ and a corresponding \emph{\nameref{def:Decryption_Rule}} $d_{K} \in \DecryptionRules$.
    Each $e_{K} : \Plaintexts \rightarrow \Ciphertexts$ and $d_{K} : \Ciphertexts \rightarrow \Plaintexts$ are functions such that $d_{K} \left( e_{K}(x) \right) = x$ for every \nameref{def:Plaintext} element $x \in \Plaintexts$.
  \end{enumerate}
\end{definition}

This can be illustrated with the \nameref{fig:Shannon_Model_Symmetric_Encryption}.

\begin{figure}[h!]
  \centering
  % \includegraphic{Shannon_Model_Symmetric_Encryption.png}
  \caption{Shannon Model for Symmetric Encryption}
  \label{fig:Shannon_Model_Symmetric_Encryption}
\end{figure}

A convention is to give the various parties in play names:
\begin{itemize}[noitemsep]
\item Alice
\item Bob
\item Caesar
\item Eve (The enemy)
\end{itemize}

\begin{definition}[Plaintext]\label{def:Plaintext}
  \emph{Plaintext} is the information that Alice wants to send to Bob, and is denoted $\Plaintexts$
  This information can be in any arbitrary format, we do not care, however the elements in the message are drawn from the \nameref{def:Alphabet}.
  However, Alice does not want Eve to be able to understand what she sends to Bob.

  The message to be sent is:
  \begin{equation}\label{eq:Plaintext}
    \Message \in \Plaintexts = \left( m_{1}, m_{2}, \ldots, m_{n} \right) \forall m \in \Alphabet
  \end{equation}
\end{definition}

\begin{definition}[Alphabet]\label{def:Alphabet}
  Let $\Alphabet$ be a finite set, which is called the \emph{alphabet}.
  We often use the English letters as the alphabet, we number them $\mathtt{a} = 0, \mathtt{b} = 1, \ldots, \mathtt{z} = 25$.
  This given $\Alphabet = \IntsMod{26}$ and $\SetOrder{\Alphabet} = 26$.

  \begin{table}[h!]
    \centering
    \begin{tabular}{cccccc}
      \toprule
      \texttt{a} & 0.0804 & \texttt{j} & 0.0016 & \texttt{s} & 0.0654 \\
      \texttt{b} & 0.0154 & \texttt{k} & 0.0067 & \texttt{t} & 0.0925 \\
      \texttt{c} & 0.0306 & \texttt{l} & 0.0414 & \texttt{u} & 0.0271 \\
      \texttt{d} & 0.0399 & \texttt{m} & 0.0253 & \texttt{v} & 0.0099 \\
      \texttt{e} & \textbf{0.1251} & \texttt{n} & 0.0709 & \texttt{w} & 0.0192 \\
      \texttt{f} & 0.0230 & \texttt{0} & 0.0760 & \texttt{x} & 0.0019 \\
      \texttt{g} & 0.0196 & \texttt{p} & 0.0200 & \texttt{y} & 0.0173 \\
      \texttt{h} & 0.0549 & \texttt{q} & 0.0011 & \texttt{z} & 0.0009 \\
      \texttt{i} & 0.0726 & \texttt{r} & 0.0612 & & \\
      \bottomrule
    \end{tabular}
    \caption{Frequency of Letters in the English Alphabet}
    \label{tab:Frequency_Letters_English_Alphabet}
  \end{table}
\end{definition}

\begin{definition}[Ciphertext]\label{def:Ciphertext}
  A \emph{ciphertext} is a piece of \nameref{def:Plaintext} information that has been run through an element of \nameref{def:Encryption_Rule} set.

  A message that has been transformed with an \nameref{def:Encryption_Rule} is a cipher message.
  \begin{equation}\label{eq:Ciphertext}
    \CipherMessage = e_{K} (\Message)
  \end{equation}
\end{definition}

\begin{definition}[Keyspace]\label{def:Keyspace}
  \emph{Keyspace}, denoted $\Keyspace$.
  Each \emph{key} in the keyspace describes a certain function
  \begin{equation}\label{eq:Keyspace}
    e_{K} : \Plaintexts^{n} \rightarrow \Ciphertexts^{n'}
  \end{equation}
  \textbf{TODO}
\end{definition}

\begin{definition}[Encryption Rule]\label{def:Encryption_Rule}
  The \emph{encryption rule} is an element $e_{K}$ from the set of all encryption rules, $\EncryptionRules$.
  \begin{equation}\label{eq:Encryption_Rule}
    e_{K} : \Plaintexts \rightarrow \Ciphertexts \text{where } e_{K} \in \EncryptionRules
  \end{equation}
  Namely, the encryption rule element is used to map the \nameref{def:Plaintext} pieces of information that Alice wants to send to a corresponding \nameref{def:Ciphertext} that she can send to Bob.

  \begin{remark}[Invertible]\label{rmk:Encryption_Rule_Invertible}
    Each \nameref{def:Encryption_Rule} \textbf{must} be \emph{invertible}, thus allowing decryption of the ciphertext.
  \end{remark}
\end{definition}

\begin{definition}[Decryption Rule]\label{def:Decryption_Rule}
  The \emph{decryption rule} is an element $d_{K}$ from the set of all decryption rules, $\DecryptionRules$.
  \begin{equation}\label{eq:Decryption_Rule}
    d_{K} : \Ciphertexts \rightarrow \Plaintexts \text{where } d_{K} \in \DecryptionRules
  \end{equation}
  Namely, the decryption rule element is used to map the \nameref{def:Ciphertext} pieces of information that Alice sent to a corresponding \nameref{def:Plaintext} that Bob can use.

  \begin{remark}
    \begin{equation}\label{eq:Decrypt_Encryption}
p      d_{K}(e_{K}(\Message)) = \Message \: \forall \Message \in \Plaintexts
    \end{equation}
  \end{remark}
\end{definition}

\subsection{Types of Attacks}\label{subsec:Attack_Types}
There are 2 main types of attacks:
\begin{enumerate}[noitemsep]
\item Ciphertext-only: A ciphertext-only attack is an attack in which the enemy has access to the ciphertext $\Ciphertexts$ and tries to recover the plaintext $\Plaintexts$ or the key $K$.
\item Known Plaintext: A known-plaintext attack is an attack in which the enemy has access to not only the ciphertext, $\Ciphertexts$, but may also have some or all of the $\Plaintexts$, and is trying to recover the key $K$.
\end{enumerate}

\subsection{The Caesar Cipher}\label{subsec:The_Caesar_Cipher}
The Caesar Cipher is a \nameref{def:Monoalphabetic_Cipher} in which each letter is shifted 3 steps.

This can be generalized to shift not just 3, but $K$ positions, where $K \in \lbrace 0, 1, \ldots, 25 \rbrace$ is the secret key.
In this case, the \nameref{def:Keyspace}, $\Keyspace$, is the set of shifts in position possible.
\begin{equation*}
  K \in \Keyspace, \:\: \Keyspace = \lbrace 0, 1, \ldots, 25 \rbrace
\end{equation*}

To encrypt a message, each individual character is encrypted according to \Cref{eq:The_Caesar_Cipher-Encryption}.
\begin{equation}\label{eq:The_Caesar_Cipher-Encryption}
  e_{K}(m) = m + k \bmod 26
\end{equation}

The \nameref{def:Decryption_Rule} for \nameref*{subsec:The_Caesar_Cipher} is used to decrypt a message encrypted according to \Cref{eq:The_Caesar_Cipher-Encryption}.
\begin{equation}\label{eq:The_Caesar_Cipher-Decryption}
  d_{K}(c) = c - k
\end{equation}

\subsubsection{\nameref*{def:Cryptanalysis} of \nameref*{subsec:The_Caesar_Cipher}}\label{subsubsec:Cryptanalysis_Caesar_Cipher}
The Caesar Cipher can be broken with an \emph{exhaustive key search}.
\begin{example}[Lecture 4]{Cryptanalysis of The Caesar Cipher}
  Given the input string \texttt{wklvlvdphvvdjhwrbrx}, decrypt and find the message.
  The message was encrypted using \nameref{subsec:The_Caesar_Cipher}
  \tcblower{}
  \begin{center}
    \begin{tabular}{cc}
      \toprule
      $K$ & Plaintext Equivalent \\
      \midrule
      0 & \texttt{wklvlvdphvvdjhwrbrx} \\
      1 & \texttt{vjkukucoguucigvqaqw} \\
      2 & \texttt{uijtjtbnfttbhfupzpv} \\
      \midrule
      3 & \texttt{thisisamessagetoyou} \\
      \midrule
      4 & \texttt{sghrhrzldrrzfdsnxnt} \\
      5 & \texttt{rfgqgqykcqqyecrmwms} \\
      $\vdots$ & $\vdots$ \\
      25 & \texttt{xlmwmweqiwwekixscsy} \\
      \bottomrule
    \end{tabular}
  \end{center}

  So, the key used in this Caesar Cipher was $K=3$, and the message was \texttt{thisisamessagetoyou}.
\end{example}

%%% Local Variables:
%%% mode: latex
%%% TeX-master: "../EDIN01-Cryptography-Reference_Sheet"
%%% End:
