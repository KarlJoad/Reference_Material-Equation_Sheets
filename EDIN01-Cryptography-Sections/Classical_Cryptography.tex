\section{Classical Cryptography}\label{sec:Classical_Cryptography}
\begin{definition}[Cryptosystem]\label{def:Cryptosystem}
  A \emph{cryptosystem} five-tuple $(\Plaintexts, \Ciphertexts, \Keyspace, \EncryptionRules, \DecryptionRules)$ where the following conditions are satisfied:
  \begin{enumerate}[noitemsep]
  \item $\Plaintexts$ is a finite set of possible \emph{\nameref{def:Plaintext}}, producing a message, $\Message$.
  \item $\Ciphertexts$ is a finite set of possible \emph{\nameref{def:Ciphertext}}
  \item $\Keyspace$, the \emph{\nameref{def:Keyspace}}, is a finite set of all possible keys
  \item For each $K \in \Keyspace$, there is an \emph{\nameref{def:Encryption_Rule}} $e_{K}\in \EncryptionRules$ and a corresponding \emph{\nameref{def:Decryption_Rule}} $d_{K} \in \DecryptionRules$.
    Each $e_{K} : \Plaintexts \rightarrow \Ciphertexts$ and $d_{K} : \Ciphertexts \rightarrow \Plaintexts$ are functions such that $d_{K} \left( e_{K}(x) \right) = x$ for every \nameref{def:Plaintext} element $x \in \Plaintexts$.
  \end{enumerate}
\end{definition}

This can be illustrated with the \nameref{fig:Shannon_Model_Symmetric_Encryption}.

\begin{figure}[h!]
  \centering
  % \includegraphic{Shannon_Model_Symmetric_Encryption.png}
  \caption{Shannon Model for Symmetric Encryption}
  \label{fig:Shannon_Model_Symmetric_Encryption}
\end{figure}

\begin{definition}[Symmetric Encryption]\label{def:Symmetric_Encryption}
  A \emph{symmetric encryption} system has Alice and Bob both share the \textbf{same} key.
  We make the assumption that Alice can share the key with Bob in a way such that Eve cannot use the key.
  \Cref{fig:Shannon_Model_Symmetric_Encryption} gives a visualization of this type of system.
\end{definition}

A convention is to give the various parties in play names:
\begin{itemize}[noitemsep]
\item Alice
\item Bob
\item Caesar
\item Eve (The enemy)
\end{itemize}

\begin{definition}[Plaintext]\label{def:Plaintext}
  \emph{Plaintext} is the information that Alice wants to send to Bob, and is denoted $\Plaintexts$
  This information can be in any arbitrary format, we do not care, however the elements in the message are drawn from the \nameref{def:Alphabet}.
  However, Alice does not want Eve to be able to understand what she sends to Bob.

  The message to be sent is:
  \begin{equation}\label{eq:Plaintext}
    \Message \in \Plaintexts = \left( m_{1}, m_{2}, \ldots, m_{n} \right) \forall m \in \Alphabet
  \end{equation}
\end{definition}

\begin{definition}[Alphabet]\label{def:Alphabet}
  Let $\Alphabet$ be a finite set, which is called the \emph{alphabet}.
  We often use the English letters as the alphabet, we number them $\mathtt{a} = 0, \mathtt{b} = 1, \ldots, \mathtt{z} = 25$.
  This given $\Alphabet = \IntsMod{26}$ and $\SetOrder{\Alphabet} = 26$.

  \begin{table}[h!]
    \centering
    \begin{tabular}{cccccc}
      \toprule
      \texttt{a} & 0.0804 & \texttt{j} & 0.0016 & \texttt{s} & 0.0654 \\
      \texttt{b} & 0.0154 & \texttt{k} & 0.0067 & \texttt{t} & 0.0925 \\
      \texttt{c} & 0.0306 & \texttt{l} & 0.0414 & \texttt{u} & 0.0271 \\
      \texttt{d} & 0.0399 & \texttt{m} & 0.0253 & \texttt{v} & 0.0099 \\
      \texttt{e} & \textbf{0.1251} & \texttt{n} & 0.0709 & \texttt{w} & 0.0192 \\
      \texttt{f} & 0.0230 & \texttt{0} & 0.0760 & \texttt{x} & 0.0019 \\
      \texttt{g} & 0.0196 & \texttt{p} & 0.0200 & \texttt{y} & 0.0173 \\
      \texttt{h} & 0.0549 & \texttt{q} & 0.0011 & \texttt{z} & 0.0009 \\
      \texttt{i} & 0.0726 & \texttt{r} & 0.0612 & & \\
      \bottomrule
    \end{tabular}
    \caption{Frequency of Letters in the English Alphabet}
    \label{tab:Frequency_Letters_English_Alphabet}
  \end{table}
\end{definition}

\begin{definition}[Ciphertext]\label{def:Ciphertext}
  A \emph{ciphertext} is a piece of \nameref{def:Plaintext} information that has been run through an element of \nameref{def:Encryption_Rule} set.

  A message that has been transformed with an \nameref{def:Encryption_Rule} is a cipher message.
  \begin{equation}\label{eq:Ciphertext}
    \CipherMessage = e_{K} (\Message)
  \end{equation}
\end{definition}

\begin{definition}[Keyspace]\label{def:Keyspace}
  \emph{Keyspace}, denoted $\Keyspace$.
  Each \emph{key} in the keyspace describes a certain function
  \begin{equation}\label{eq:Keyspace}
    e_{K} : \Plaintexts^{n} \rightarrow \Ciphertexts^{n'}
  \end{equation}
  \textbf{TODO}
\end{definition}

\begin{definition}[Encryption Rule]\label{def:Encryption_Rule}
  The \emph{encryption rule} is an element $e_{K}$ from the set of all encryption rules, $\EncryptionRules$.
  \begin{equation}\label{eq:Encryption_Rule}
    e_{K} : \Plaintexts \rightarrow \Ciphertexts \text{where } e_{K} \in \EncryptionRules
  \end{equation}
  Namely, the encryption rule element is used to map the \nameref{def:Plaintext} pieces of information that Alice wants to send to a corresponding \nameref{def:Ciphertext} that she can send to Bob.

  \begin{remark}[Invertible]\label{rmk:Encryption_Rule_Invertible}
    Each \nameref{def:Encryption_Rule} \textbf{must} be \emph{invertible}, thus allowing decryption of the ciphertext.
  \end{remark}
\end{definition}

\begin{definition}[Decryption Rule]\label{def:Decryption_Rule}
  The \emph{decryption rule} is an element $d_{K}$ from the set of all decryption rules, $\DecryptionRules$.
  \begin{equation}\label{eq:Decryption_Rule}
    d_{K} : \Ciphertexts \rightarrow \Plaintexts \text{where } d_{K} \in \DecryptionRules
  \end{equation}
  Namely, the decryption rule element is used to map the \nameref{def:Ciphertext} pieces of information that Alice sent to a corresponding \nameref{def:Plaintext} that Bob can use.

  \begin{remark}
    \begin{equation}\label{eq:Decrypt_Encryption}
p      d_{K}(e_{K}(\Message)) = \Message \: \forall \Message \in \Plaintexts
    \end{equation}
  \end{remark}
\end{definition}

\subsection{Types of Attacks}\label{subsec:Attack_Types}
There are 2 main types of attacks:
\begin{enumerate}[noitemsep]
\item Ciphertext-only: A ciphertext-only attack is an attack in which the enemy has access to the ciphertext $\Ciphertexts$ and tries to recover the plaintext $\Plaintexts$ or the key $K$.
\item Known Plaintext: A known-plaintext attack is an attack in which the enemy has access to not only the ciphertext, $\Ciphertexts$, but may also have some or all of the $\Plaintexts$, and is trying to recover the key $K$.
\end{enumerate}

\subsection{The Caesar Cipher}\label{subsec:The_Caesar_Cipher}
The Caesar Cipher is a \nameref{def:Monoalphabetic_Cipher} in which each letter is shifted 3 steps.

This can be generalized to shift not just 3, but $K$ positions, where $K \in \lbrace 0, 1, \ldots, 25 \rbrace$ is the secret key.
In this case, the \nameref{def:Keyspace}, $\Keyspace$, is the set of shifts in position possible.
\begin{equation*}
  K \in \Keyspace, \:\: \Keyspace = \lbrace 0, 1, \ldots, 25 \rbrace
\end{equation*}

To encrypt a message, each individual character is encrypted according to \Cref{eq:The_Caesar_Cipher-Encryption}.
\begin{equation}\label{eq:The_Caesar_Cipher-Encryption}
  e_{K}(m) = m + k \bmod 26
\end{equation}

The \nameref{def:Decryption_Rule} for \nameref*{subsec:The_Caesar_Cipher} is used to decrypt a message encrypted according to \Cref{eq:The_Caesar_Cipher-Encryption}.
\begin{equation}\label{eq:The_Caesar_Cipher-Decryption}
  d_{K}(c) = c - k
\end{equation}

\subsubsection{\nameref*{def:Cryptanalysis} of \nameref*{subsec:The_Caesar_Cipher}}\label{subsubsec:Cryptanalysis_Caesar_Cipher}
The Caesar Cipher can be broken with an \emph{exhaustive key search}.
\begin{example}[Lecture 4]{Cryptanalysis of The Caesar Cipher}
  Given the input string \texttt{wklvlvdphvvdjhwrbrx}, decrypt and find the message.
  The message was encrypted using \nameref{subsec:The_Caesar_Cipher}
  \tcblower{}
  \begin{center}
    \begin{tabular}{cc}
      \toprule
      $K$ & Plaintext Equivalent \\
      \midrule
      0 & \texttt{wklvlvdphvvdjhwrbrx} \\
      1 & \texttt{vjkukucoguucigvqaqw} \\
      2 & \texttt{uijtjtbnfttbhfupzpv} \\
      \midrule
      3 & \texttt{thisisamessagetoyou} \\
      \midrule
      4 & \texttt{sghrhrzldrrzfdsnxnt} \\
      5 & \texttt{rfgqgqykcqqyecrmwms} \\
      $\vdots$ & $\vdots$ \\
      25 & \texttt{xlmwmweqiwwekixscsy} \\
      \bottomrule
    \end{tabular}
  \end{center}

  So, the key used in this Caesar Cipher was $K=3$, and the message was \texttt{thisisamessagetoyou}.
\end{example}

\subsection{The Simple Substitution Cipher}\label{subsec:Simple_Substitution_Cipher}
The Simple Substitution Cipher operates on the individual characters in $\Plaintexts$.
However, the key, $K$ is now an arbitrary permutation of the characters.

The set of all permutations on \TextIntsMod{26} is written as
\begin{equation*}
  \EncryptionRules = \lbrace \pi_{1}, \pi_{2}, \ldots, \pi_{\SetOrder{\Keyspace}} \rbrace
\end{equation*}

The keyspace, $\Keyspace$ for this cipher is:
\begin{equation*}
  \begin{aligned}
    \Keyspace &= \lbrace 0, 1, 2, \ldots, 26!-1 \rbrace \\
    \SetOrder{\Keyspace} &= 26!
  \end{aligned}
\end{equation*}

The \nameref{def:Encryption_Rule} for this cipher is:
\begin{equation}\label{eq:Simple_Substitution_Cipher-Encryption}
  e_{K}(m) = \pi_{K}(m)
\end{equation}

The \nameref{def:Decryption_Rule} for this cipher's output is:
\begin{equation}\label{eq:Simple_Substitution_Cipher-Decryption}
  d_{K}(c) = \pi_{k}^{-1}(c)
\end{equation}

\subsubsection{\nameref*{def:Cryptanalysis} of \nameref*{subsec:Simple_Substitution_Cipher}}\label{subsubsec:Cryptanalysis_Simple_Substitution_Cipher}
We could attemp to perform an exhaustive key search, but because the keyspace is all integers from 0 to $26!-1$, which is incredibly large, it makes more sense to exploit the statistical nature of the plaintext source.

We could try matching the most common letters present in the \nameref{def:Ciphertext} against the most common letter in the \nameref{def:Plaintext} \nameref{def:Alphabet}, using \Cref{tab:Frequency_Letters_English_Alphabet}.
However, these 1-grams are only good for the most common letters in the English \nameref{def:Alphabet}.

If we take use n-grams we will have better luck, because there is a dependence between consecutive letters.

\subsection{Polyalphabetic Ciphers}\label{subsec:Polyalphabetic_Ciphers}
Polyalphabetic ciphers operate differently on different portions of the \nameref{def:Plaintext}.

The simplist \nameref{def:Polyalphabetic_Cipher} is a number of \nameref{def:Monoalphabetic_Cipher}s with different keys are used sequentially, the cyclically repeated.
\nameref{subsubsec:The_Vigenere_Cipher} is one example of this kind of cipher.

\subsubsection{The Vigen\'{e}re Cipher}\label{subsubsec:The_Vigenere_Cipher}
Cyclically uses $t$ Caesar ciphers, where $t$ is the period of the cipher.

The \nameref{def:Encryption_Rule} maps the \nameref{def:Plaintext} message $\Message = m_{1},m_{2},\ldots$ to the \nameref{def:Ciphertext} $\CipherMessage = c_{1},c_{2},\ldots$ through
\begin{equation}\label{eq:The_Vigenere_Cipher-Encryption}
  \CipherMessage = e_{\mathbf{K}} (m_{1}, m_{2}, \ldots, m_{t} ), e_{\mathbf{K}} (m_{t+1}, m_{t+2}, \ldots, m_{t+t}),\ldots
\end{equation}
where
\begin{equation*}
  e_{\mathbf{K}} (m_{1}, m_{2}, \ldots, m_{t}) = (m_{1}+k_{1}, m_{2}+k_{2}, \ldots, m_{t}+k_{t})
\end{equation*}
and $\mathbf{K}$ consists of $t$ characters $\mathbf{K} = (k_{1}, k_{2}, \ldots, k_{t})$.

\begin{remark*}
  The key, $\mathbf{K}$ is often chosen to be a word, called a keyword.
\end{remark*}

\begin{example}[Lecture 4]{Encryption Using The Vigenere Cipher}
  The \nameref{def:Plaintext} \texttt{youmustvisitmetonight} is to be encrypted using a Vigen\'{e}re Cipher, with a period 4, where $\mathbf{K} = \mathtt{lucy}$.
  \tcblower{}
  \begin{center}
    \begin{tabular}{cccc|cccc|cccc|cccc|c}
      \toprule
      \texttt{y} &  \texttt{o} &  \texttt{u} &  \texttt{m} &  \texttt{u} &  \texttt{s} &  \texttt{t} &  \texttt{v} &  \texttt{i} &  \texttt{s} &  \texttt{i} &  \texttt{t} &  \texttt{m} &  \texttt{e} &  \texttt{t} &  \texttt{o} &  $\cdots$ \\
      \texttt{+} &  \texttt{+} &  \texttt{+} &  \texttt{+} &  \texttt{+} &  \texttt{+} &  \texttt{+} &  \texttt{+} &  \texttt{+} &  \texttt{+} &  \texttt{+} &  \texttt{+} &  \texttt{+} &  \texttt{+} &  \texttt{+} &  \texttt{+} &  $\cdots$ \\
      \texttt{l} &  \texttt{u} &  \texttt{c} &  \texttt{y} &  \texttt{l} &  \texttt{u} &  \texttt{c} &  \texttt{y} &  \texttt{l} &  \texttt{u} &  \texttt{c} &  \texttt{y} &  \texttt{l} &  \texttt{u} &  \texttt{c} &  \texttt{y} &  $\cdots$ \\
      \midrule
      \texttt{j} &  \texttt{i} &  \texttt{w} &  \texttt{k} &  \texttt{f} &  \texttt{m} &  \texttt{v} &  \texttt{t} &  \texttt{t} &  \texttt{m} &  \texttt{k} &  \texttt{r} &  \texttt{x} &  \texttt{y} &  \texttt{v} &  \texttt{m} &  $\cdots$ \\
      \bottomrule
    \end{tabular}
  \end{center}

  The resulting \nameref{def:Ciphertext} is \texttt{jiwkfmvttmkrxyvm}.
\end{example}

\paragraph{\nameref*{def:Cryptanalysis} of \nameref*{subsubsec:The_Vigenere_Cipher}}\label{par:Cryptanalysis_Vigenere_Cipher}
There are 2 steps to breaking \nameref{subsubsec:The_Vigenere_Cipher}:
\begin{enumerate}[noitemsep]
\item Determine the period of the cipher, $t$. Generally, \nameref{def:Kasiskis_Method} is used.
\item Reconstruct the $t$ different \nameref{subsec:Simple_Substitution_Cipher} \nameref{def:Alphabet}s used.
\end{enumerate}

\begin{definition}[Kasiski's Method]\label{def:Kasiskis_Method}
  \emph{Kasiski's Method} relies on the observation that repeated portions of \nameref{def:Plaintext} input will yield a repeating \nameref{def:Ciphertext} output, with some cyclic distance, $t$ between each occurrence.
  This distance will be a multiple of the keyword length.

  To solve this, you calculate the \nameref{def:GCD} of these distances.
\end{definition}

Now we move onto the second part of breaking \nameref{subsubsec:The_Vigenere_Cipher}.
There are 2 main ways to do this.
\begin{enumerate}[noitemsep]
\item Split the \nameref{def:Ciphertext} characters into $t$ different multisets, where each character should have been shifted according to a Simple Substitution Cipher.
  The key of each \nameref{def:Monoalphabetic_Cipher} can be determined with the statistics in \Cref{tab:Frequency_Letters_English_Alphabet} and some trial-and-error.
\item \nameref{def:Measure_of_Roughness} and \nameref{def:Index_of_Coincidence}
\end{enumerate}

\begin{definition}[Measure of Roughness]\label{def:Measure_of_Roughness}
  \begin{equation}\label{eq:Measure_of_Roughness}
    \begin{aligned}
      \mathrm{MR} &= \sum\limits_{i=a\Rightarrow1}^{z\Rightarrow26} {\left( p_{i} - \frac{1}{26} \right)}^{2} \\
      &= \sum\limits_{i=a\Rightarrow1}^{z\Rightarrow26} {\left( p_{i} \right)}^{2} - \frac{1}{26} \\
    \end{aligned}
  \end{equation}
\end{definition}

\begin{definition}[Index of Coincidence]\label{def:Index_of_Coincidence}
  The \emph{index of coincidence} uses values found in the \nameref{def:Measure_of_Roughness}.
  \begin{equation}\label{eq:Index_of_Coincidence}
    \mathrm{IC} = \frac{\sum\limits_{i=a\Rightarrow1}^{z\Rightarrow26} f_{i} \left( f_{1} - 1 \right)}{n(n-1)}
  \end{equation}

  However, it does \emph{not} return the period of \nameref{subsubsec:The_Vigenere_Cipher}.
\end{definition}

\subsubsection{The Vernam Cipher (One-Time Pad)}\label{subsubsec:The_Vernam_Cipher}
This is a special case of the \nameref{subsubsec:The_Vigenere_Cipher}, where the period of the key, $\mathbf{K}$ is chosen to be the same length as the output ciphertext, $\mathbf{C}$.
More concretely, for a ciphertext, $\mathbf{C}$ of length $n$, pick a Vigen\'{e}re Cipher key with a period $t$.
Where,
\begin{equation*}
  n = t
\end{equation*}

\paragraph{Cryptanalysis of \nameref*{subsubsec:The_Vernam_Cipher}}
\nameref{subsubsec:The_Vernam_Cipher} is an example of \nameref{def:Perfect_Secrecy} (unbreakable) if the the elements of the key $\mathbf{K}$ are chosen uniformly at random among all possible $\IntsMod{26}^{n}$ possible values.

\subsubsection{Transposition Cipher (Permutation Cipher)}\label{subsubsec:Transposition_Cipher}
This cipher reorders all the letters in a block of length $t$.
The \nameref{def:Encryption_Rule}'s key, $K$ that permutes all the characters in a word block.
\Cref{eq:Transposition_Cipher-Encryption} is an example of how the \nameref{def:Encryption_Rule} is constructed.
\begin{equation}\label{eq:Transposition_Cipher-Encryption}
  \begin{aligned}
    \pi_{e}(2, 5, 1, 3, 4) \Longrightarrow \pi(1) &= 2 \\
    \pi(2) &= 5 \\
    \pi(3) &= 1 \\
    \pi(4) &= 3 \\
    \pi(5) &= 4 \\
  \end{aligned}
\end{equation}

The \nameref{def:Decryption_Rule} for \Cref{eq:Transposition_Cipher-Encryption} is given in \Cref{eq:Transposition_Cipher-Decryption}.
\begin{equation}\label{eq:Transposition_Cipher-Decryption}
  \begin{aligned}
    \pi_{d}(4, 3, 1, 5, 2) \Longrightarrow \pi(1) &= 4 \\
    \pi(2) &= 3 \\
    \pi(3) &= 1 \\
    \pi(4) &= 5 \\
    \pi(5) &= 2 \\
  \end{aligned}
\end{equation}

\begin{remark}
  The number on the \textbf{right-hand side} is assigned to the position present on the \textbf{left-hand side}.
\end{remark}

\begin{example}[Lecture 5]{Transposition Cipher}
  Given the plaintext message $\mathbf{M}=$ findi, and the \nameref{def:Encryption_Rule} shown in \Cref{eq:Transposition_Cipher-Encryption}, what is the resulting ciphertext message?
  \tcblower{}
  \begin{align*}
    \pi(1) &= 2 \\
    \pi(2) &= 5 \\
    \pi(3) &= 1 \\
    \pi(4) &= 3 \\
    \pi(5) &= 4 \\
  \end{align*}
  So, each character in the input ``findi'' is moved around to result in ``iifnd''.
\end{example}

\subsubsection{The Hill Cipher}\label{subsubsec:The_Hill_Cipher}
\nameref{subsubsec:The_Hill_Cipher} acts on $t$-grams from \TextIntsMod{26} through a key that is an \nameref{def:Invertible} $t \times t$ matrix.

%%% Local Variables:
%%% mode: latex
%%% TeX-master: "../EDIN01-Cryptography-Reference_Sheet"
%%% End:
