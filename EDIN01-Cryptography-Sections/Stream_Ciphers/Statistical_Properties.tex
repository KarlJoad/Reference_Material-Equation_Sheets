\subsection{Statistical Properties}\label{subsec:LFSR_Statistical_Properties}
The importance of \nameref{def:LFSR} sequences and \nameref{def:M_Sequence}s are due to their pseudorandomness properties.
$\mathbf{s} = s_{0}, s_{1}, \ldots$ is an \nameref{def:M_Sequence}, and an $n$-gram is a subsequence of length $n$,
\begin{equation*}
  \left( s_{t}, s_{t+1}, \ldots, s_{t+n-1} \right)
\end{equation*}
for $t=0, 1, \ldots$

\begin{theorem}
  Among the $q^{L}-1$ $L$-grams that can be constructed for $t = 0, 1, \ldots, q^{L}-2$, ever nonzero vector appears exactly once.
\end{theorem}

\subsubsection{Autocorrelation Function}\label{subsubsec:Autocorrelation_Function}
\begin{definition}[Autocorrelation Function]\label{def:LFSR_Autocorrelation_Function}
  The \emph{autocorrelation function} has:
  \begin{itemize}[noitemsep]
  \item 2 binary sequences, $\mathbf{x}, \mathbf{y}$ of the same length $n$.
  \item The autocorrelation $C(\mathbf{x}, \mathbf{y})$ between the 2 sequences is defined as the number of positions of agreements minus the number of disagreements.
  \item The autocorrelation function $C(\tau)$ is defined to be the correlation between a sequence $\mathbf{x}$ and its $\tau$th cyclic shift, i.e.,
    \begin{equation}\label{eq:LFSR_Autocorrelation_Function}
      C(\tau) = \sum\limits_{i=1}^{n} {(-1)}^{x_{i}+x_{i+\tau}}
    \end{equation}
    where subscripts are taken modulo $n$ and addition in the exponent is $\mod{2}$ addition.
  \end{itemize}
\end{definition}

\begin{theorem}[Autocorrelation of $m$-Sequence]\label{thm:M_Sequence_Autocorrelation}
  If $\mathbf{s}$ is an $m$-sequence of length $2^{L}-1$, then
  \begin{equation}\label{eq:M_Sequence_Autocorrelation}
    C(\tau) =
    \begin{cases}
      2^{L} - 1 & \text{ if } \tau \equiv 0 \bmod n \\
      -1 & \text{otherwise} \\
    \end{cases}
  \end{equation}
\end{theorem}

%%% Local Variables:
%%% mode: latex
%%% TeX-master: "../../EDIN01-Cryptography-Reference_Sheet"
%%% End:
