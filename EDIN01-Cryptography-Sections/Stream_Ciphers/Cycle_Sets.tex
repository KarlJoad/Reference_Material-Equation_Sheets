\subsection{Cycle Sets}\label{subsec:Cycle_Sets}
\begin{definition}[Cycle Set]\label{def:Cycle_Set}
  The \emph{cycle set} for $C(D)$ is the number of cycles of a given length.
  This is assuming $L = \Degree \bigl( C(D) \bigr)$.
  It is written
  \begin{equation}\label{eq:Cycle_Set}
    n_{1} \left( T_{1} \right) \XOR n_{1} \left( T_{1} \right) \XOR \cdots
  \end{equation}

  For example, $1(1) \XOR 3(5)$ means there is one cycle of length one, and three cycles of length 5.

  \begin{remark}[Combining]
    You can combine cycle with the same period by simple addition.
    \begin{equation}\label{eq:Cycle_Set_Addition}
      n_{1}(T) \XOR n_{2}(T) =  \left( n_{1} + n_{2} \right) (T)
    \end{equation}
  \end{remark}
\end{definition}

\subsubsection{Properties of \nameref*{subsec:Cycle_Sets}}\label{subsubsec:Cycle_Set_Properties}
\begin{propertylist}
\item If $C(D)$ is a \nameref{def:Polynomial_Ring_Properties-Primitive_Element} of \nameref{def:Polynomial_Degree} $L$ over \TextFiniteMathField{F}{q}{}, then the \nameref{def:Cycle_Set} is\label{prop:Cycle_Set_Properties-Primitive_Polynomial}
  \begin{equation}\label{eq:Primitive_Polynomial_Cycle_Set}
    1(1) \XOR 1 \left( q^{L}-1 \right)
  \end{equation}
\item If $C(D)$ is an \nameref{def:Irreducible_Polynomial} \nameref{def:Polynomial}, then the \nameref{def:Cycle_Set} is\label{prop:Cycle_Set_Properties-Irreducible_Polynomial}
  \begin{equation}\label{eq:Irreducible_Polynomial_Cycle_Set}
    1(1) \XOR \frac{ \left(q^{L} - 1 \right)}{T} (T)
  \end{equation}
  where $T$ is the \nameref{def:Polynomial_Period} of $C(D)$, or the \nameref{def:Element_Order} of $\alpha$ when $\pi(\alpha)=0$.
\end{propertylist}

\begin{theorem}[Connection Polynomial Multiple]\label{thm:Cycle_Set_Multiple_of_Connection_Polynomial}
  If $C(D) = {C_{1}(D)}^{n}$, then the \nameref{def:Cycle_Set} of $C(D)$ is
  \begin{equation}\label{eq:Cycle_Set_Multiple_of_Connection_Polynomial}
    1(1) \XOR \frac{\left( q^{L_{1}}-1 \right)}{T_{1}} \left( T_{1} \right) \XOR \frac{q^{L_{1}} \left( q^{L_{1}}-1 \right)}{T_{2}} \left( T_{2} \right) \XOR \cdots \XOR \frac{q^{(n-1)L_{1}}\left( q^{L_{1}}-1 \right)}{T_{n}} \left( T_{n} \right)
  \end{equation}
  where $\Degree \bigl( C(D) \bigr) = L$ and $T_{j}$ is the \nameref{def:Polynomial_Period} of ${C_{1}(D)}^{j}$.
\end{theorem}

\begin{theorem}[Irreducible Connection Polynomial Multiple]\label{thm:Irreducible_Connection_Polynomial_Multiple_Cycle_Set}
  if $C_{1}(D)$ is \nameref{def:Irreducible_Polynomial} with period $T_{1}$, then the \nameref{def:Polynomial_Period} of ${C_{1}(D)}^{j}$ is
  \begin{equation}\label{eq:Irreducible_Connection_Polynomial_Multiple_Cycle_Set}
    T_{j} = p^{m}T_{1}
  \end{equation}
  where $p$ is the \nameref{def:Field_Characteristic} of the \nameref{def:Field} and $m$ is the integer satisfying
  \begin{equation}\label{eq:Irreducible_Connection_Polynomial_Multiple_Cycle_Set_Requirement}
    p^{m-1} < j \leq p^{m}
  \end{equation}
\end{theorem}

\begin{example}[Lecture 9, Example 4.9]{Cycle Set of Connection Polynomial}
  Compute the \nameref{def:Cycle_Set} for the \nameref{def:Connection_Polynomial} $C(D) = {\left( 1+D+D^{2} \right)}^{3}$ in \TextFiniteMathField{F}{2}{}?
  \tcblower{}
  The \nameref{def:Connection_Polynomial} here is actually in the form $C(D) = {C_{1}(D)}^{3}$.
  In this case, $C_{1}(D) = 1+D+D^{2}$, which is both \nameref{def:Irreducible_Polynomial} and \nameref{def:Polynomial_Ring_Properties-Primitive_Element}.

  We can find the \nameref{def:Polynomial_Period} in 3 ways.
  \begin{enumerate}[noitemsep]
  \item Because it's $C_{1}(D)$ is a primitive \nameref{def:Polynomial}, $T_{1}=q^{L}-1 = 3$.
  \item Could perform a \nameref{def:Polynomial_Period}, which yields the same thing.
  \item Because it's $C_{1}(D)$ is a primitive \nameref{def:Polynomial}, $\pi(\alpha)=0$, and the $\ElementOrder \left( \alpha^{2} \right) = 3$.
  \end{enumerate}

  Now, $T_{2}$ must be calculated with $T_{2} = 2^{m_{2}} \cdot 3$, where $2^{m_{2}-1} < 2 \leq 2^{m_{2}}$.
  $m_{2} = 1$, plugging that back in gives $T_{2} = 2$.

  Doing the same for $T_{3}$ yields $T_{3} = 12$.

  Now we combine these into a single line.
  \begin{align*}
    1(1) &\XOR \frac{ \left( 2^{2} - 1 \right)}{3} (3) \XOR \frac{ 2^{2} \left( 2^{2} - 1 \right)}{3} (6) \XOR \frac{2^{6} \left( 2^{2} - 1 \right)}{3} (12) \\
    1(1) &\XOR 1 (3) \XOR 2 (6) \XOR 4 (12)
  \end{align*}
\end{example}

\begin{theorem}[Arbitrary Connection Polynomial]\label{thm:Arbitrary_Connection_Polynomial}
  For a \nameref{def:Connection_Polynomial} $C(D)$ factoring like
  \begin{equation*}
    C(D) = {C_{1}(D)}^{n_{1}} {C_{2}(D)}^{n_{2}} \cdots {C_{m}(D)}^{n_{m}}
  \end{equation*}
  $C_{i}(D)$ is \nameref{def:Irreducible_Polynomial}, has a \nameref{def:Cycle_Set} $S_{1} \times S_{2} \times \cdots S_{m}$, where $S_{i}$ is the \nameref{def:Cycle_Set} for $C_{i}^{e_{i}}$, and
  \begin{equation}\label{eq:Arbitrary_Connection_Polynomial}
    n_{1} \left( T_{1} \right) \times n_{2} \left( T_{2} \right) = n_{1}n_{2} \gcd(T_{1}, T_{2}) \bigl( \lcm(T_{1}, T_{2}) \bigr)
  \end{equation}
  and the distributive law holds for $\times$ and $\XOR$.
\end{theorem}

%%% Local Variables:
%%% mode: latex
%%% TeX-master: "../../EDIN01-Cryptography-Reference_Sheet"
%%% End:
