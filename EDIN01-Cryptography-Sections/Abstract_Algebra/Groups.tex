\subsection{Groups}\label{subsec:Groups}
\begin{definition}[Binary Operation]\label{def:Binary_Operation}
  A \emph{binary operation}, denoted $\BinaryOperation$ on a set $S$, is a mapping from $S \BinaryOperation S$ to $S$.
  \begin{remark}
    Note that $\BinaryOperation$ is \textbf{NOT} multiplication nor any kind of convolution.
    It is just a placeholder for some \nameref{def:Binary_Operation} that can be done.
  \end{remark}
\end{definition}

\begin{definition}[Group]\label{def:Group}
  A \emph{group} is a set $G$ and a \nameref{def:Binary_Operation}, $*$, on $G$ denoted as $(G, *)$, which satisfies the following properties:
  \begin{propertylist}
  \item $a \BinaryOperation (b \BinaryOperation c) = (a \BinaryOperation b) \BinaryOperation c$ for all $a, b, c, \in G$ (Associativity).\label{prop:Group_Properties-Associativity}.
  \item There is a special element $1 \in G$ such that $a \BinaryOperation 1 = 1 \BinaryOperation a = a$ for all $a \in G$ (Identity).\label{prop:Group_Properties-Identity}
    \begin{equation*}
      \exists 1 \in G \forall a \in G \, \left( a \BinaryOperation 1 = 1 \BinaryOperation a = a \right)
    \end{equation*}

  \item For each $a \in G$ there is an element $a^{-1} \in G$ such that $a \BinaryOperation a^{-1} = a^{-1} \BinaryOperation a = 1$  (Inverse).\label{prop:Group_Properties-Inverse}
  \end{propertylist}

  We call 1 the \emph{identity element} as defined in \Cref{prop:Group_Properties-Identity}, and call $a^{-1}$ the \emph{inverse} of $a$.
  Furthermore,
  \begin{propertylist}[resume]
  \item If $a \BinaryOperation b = b \BinaryOperation a$ for all $a,b \in G$ (\nameref{def:Abelian}/Commutativity).\label{prop:Group_Properties-Commutativity}
    \begin{equation*}
      \forall a,b \in G \, \left( a \BinaryOperation b = b \BinaryOperation a \right)
    \end{equation*}
  \end{propertylist}
\end{definition}

\begin{example}[Exercise 1, Problem 1.8a]{Show Set Is Group}
  Let $S$ be the set of binary triples, $S = \lbrace \left( s_{0}, s_{1}, s_{2} \right) \vert s_{i} \in \IntsMod{2} \rbrace$.
  Let the operation be bitwise addition.
  \tcblower{}
  \begin{remark*}
    In this case, ``bitwise addition'' means element-wise addition, i.e.\ each element of each triple gets added together.
    Additionally, all additions are done in modulo 2.
  \end{remark*}
  I will define the bitwise addition \nameref{def:Binary_Operation} with the symbol $.+$, like how MATLAB or GNU Octave define it.
  We will need to prove that \Crefrange{prop:Group_Properties-Associativity}{prop:Group_Properties-Identity} are true.
  \Cref{prop:Group_Properties-Commutativity} is \emph{not} necessary to show that a set is a \nameref{def:Group}.
  \Cref{prop:Group_Properties-Commutativity} only shows that the \nameref{def:Group} is \nameref{def:Abelian}

  First, the elements present in the set $S$:
  \begin{equation*}
    S = \Bigl\lbrace (0, 0, 0), (0, 0, 1), (0, 1, 0), (0, 1, 1), (1, 0, 0), (1, 0, 1), (1, 1, 0), (1, 1, 1) \Bigr\rbrace
  \end{equation*}

  To prove that $S$ is a group with an element-wise addition \nameref{def:Binary_Operation}, I will arbitrarily select
  \begin{align*}
    a &= (0, 1, 0) \\
    b &= (0, 1, 1) \\
    c &= (1, 1, 0)
  \end{align*}

  Starting with \Cref{prop:Group_Properties-Associativity}:
  \begin{align*}
    a \, .+ (b \, .+ c) &= (a \, .+ b) \, .+ c \\
    (0, 1, 0) \, .+ \bigl( (0, 1, 1) \, .+ (1, 1, 0) \bigr) &= \bigl( (0, 1, 0) \, .+ (0, 1, 1) \bigr) \, .+ (1, 1, 0) \\
    (0, 1, 0) \, .+ \bigl( (1, 0, 1) \bigr) &= \bigl( (0, 0, 1) \bigr) \, .+ (1, 1, 0) \\
    (1, 1, 1)  &= (1, 1, 1) \\
  \end{align*}
  Thus, \Cref{prop:Group_Properties-Associativity} is satisfied.

  Now, \Cref{prop:Group_Properties-Identity}
  \begin{align*}
    a \, .+ 1 &= a \\
    (0, 1, 0) \, .+ 1 &= (0, 1, 0) \\
    1 &= (0, 1, 0) \, .+ \Bigl( -(0, 1, 0) \Bigr)
  \end{align*}
  It is important to remamber that these additions are done in the modulo 2 domain, so addition and subtraction yield the same results, making negatives irrelevant.
  \begin{align*}
    1 &= (0, 1, 0) \, .+ (0, 1, 0) \\
    1 &= (0, 0, 0)
  \end{align*}
  Thus, \Cref{prop:Group_Properties-Identity} is satisfied, and $1 = (0, 0, 0)$.

  Lastly, we need to prove \Cref{prop:Group_Properties-Inverse}.
  \begin{align*}
    a \, .+ a^{-1} &= 1 \\
    (0, 1, 0) \, .+ a^{-1} &= (0, 0, 0) \\
    a^{-1} &= (0, 0, 0) \, .+ \Bigl( -(0, 1, 0) \Bigr)
  \end{align*}
  Like before, addition and subtraction are irrelevant here.
  \begin{align*}
    a^{-1} &= (0, 0, 0) \, .+ (0, 1, 0) \\
    a^{-1} &= (0, 1, 0)
  \end{align*}
  Thus, \Cref{prop:Group_Properties-Inverse} is satisfied, where each element $a$ is its own inverse.

  Since \Crefrange{prop:Group_Properties-Associativity}{prop:Group_Properties-Inverse} are satisfied, $S$ \textbf{is} a \nameref{def:Group} with the bitwise addition \nameref{def:Binary_Operation}.
\end{example}

\begin{definition}[Abelian]\label{def:Abelian}
  An \emph{abelian} group, also called a \emph{commutative \nameref{def:Group}}, is a group in which the result of applying the group operation to two group elements does not depend on the order in which they are written.
  That is, these are the groups that obey the axiom of commutativity.
\end{definition}

\subsubsection{Examples of \nameref*{subsec:Groups}}\label{subsubsec:Examples_of_Groups}
\begin{itemize}[noitemsep]
\item \TextAllIntegers{} with the addition \nameref{def:Binary_Operation}, denoted $(\AllIntegers, +)$ is a \nameref{def:Group}.
\item Finite \nameref{def:Group}s are $(\IntsModN{}, +)$ and $(\MultiplicativeGroupN{}, \cdot)$, where $\cdot$ denotes multiplication modulo $n$
\item \textbf{NOTE:} $(\IntsModN{}, \cdot)$ is \textbf{NOT} a \nameref{def:Group}, nor is $(\AllIntegers, \cdot)$.
\end{itemize}

\subsubsection{Definitions for \nameref*{subsec:Groups}}\label{subsubsec:Definitions_for_Groups}
\begin{definition}[Subgroup]\label{def:Subgroup}
  A non-empty subset $H$ of a \nameref{def:Group} $G$ is called a \emph{subgroup} of $G$, if $H$ itself is a \nameref{def:Group} with respect to the operation of $G$.
\end{definition}

\begin{example}[Exercise 1, Problem 1.10]{Find Subgroups}
  Find all \nameref{def:Subgroup}s in the multiplicative group \TextMultiplicativeGroup{19} (under the multiplication \nameref{def:Binary_Operation})?
  \tcblower{}
  The goal here is to find a \nameref{def:Generator} that can be raised to some set of exponents that can generate \nameref{def:Subgroup}s of \TextMultiplicativeGroup{19}.

  The first thing to note is that 19 is a \nameref{def:Prime}, so
  \begin{equation*}
    \SetOrder{\MultiplicativeGroup{19}} = 18 = \phi(n)
  \end{equation*}

  We can also apply \nameref{thm:Lagranges_Theorem} to restrict the number of \nameref{def:Subgroup}s possible.
  Because \TextMultiplicativeGroup{19} is a finite \nameref{def:Group}, and we are considering $G$ to be a \nameref{def:Subgroup} of \TextMultiplicativeGroup{19}, we apply \nameref{thm:Lagranges_Theorem}.
  This means $\SetOrder{G} \Divides \SetOrder{\MultiplicativeGroup{19}}$ must be true.
  Thus,
  \begin{equation*}
    \SetOrder{G} = \lbrace 1, 2, 3, 6, 9, 18 \rbrace
  \end{equation*}

  Now, we can find a \nameref{def:Generator} for the \nameref{def:Group} \TextMultiplicativeGroup{19}.
  This means, we need to find some element $a$ that raised to some power $t$, modulo 19 will yield 1.
  We know that $t$ \textbf{must} equal 18, because $\phi(n) = 18$, and we want to find a \nameref{def:Generator}.
  So, we solve for $a$.
  \begin{equation*}
    a^{18} \bmod 18 = 1
  \end{equation*}
  In this case, $a = 2$, making $2$ a \nameref{def:Generator} of \TextMultiplicativeGroup{19}, or \TextCyclicSubgroup{2}.

  Now, all possible \nameref{def:Subgroup}s of \TextMultiplicativeGroup{19} are:
  \begin{align*}
    G_{1} &= \biggl \lbrace \Bigl[ 2^{18} \Bigr] \biggr \rbrace \\
    G_{2} &= \biggl \lbrace \Bigl[ 2^{9} \Bigr], \Bigl[ 2^{18}\Bigr] \biggr \rbrace \\
    G_{3} &= \biggl \lbrace \Bigl[ 2^{6} \Bigr], \Bigl[ 2^{12} \Bigr], \Bigl[ 2^{18} \Bigr] \biggr \rbrace \\
    G_{6} &= \biggl \lbrace \Bigl[ 2^{3} \Bigr], \Bigl[ 2^{6} \Bigr], \Bigl[ 2^{9} \Bigr], \Bigl[ 2^{12} \Bigr], \Bigl[ 2^{15} \Bigr], \Bigl[ 2^{18} \Bigr] \biggr \rbrace \\
    G_{9} &= \biggl \lbrace \Bigl[ 2^{2} \Bigr], \Bigl[ 2^{4} \Bigr], \Bigl[ 2^{6} \Bigr], \Bigl[ 2^{8} \Bigr], \Bigl[ 2^{10} \Bigr], \Bigl[ 2^{12} \Bigr], \Bigl[ 2^{14} \Bigr], \Bigl[ 2^{16} \Bigr], \Bigl[ 2^{18} \Bigr] \biggr \rbrace \\
    G_{18} &= \biggl \lbrace \Bigl[ 2^{0} \Bigr], \Bigl[ 2^{1} \Bigr], \ldots, \Bigl[ 2^{18} \Bigr] \biggr \rbrace \\
  \end{align*}
\end{example}

\begin{definition}[Cyclic]\label{def:Cyclic}
  $G$ is \emph{cyclic} if there is an element $a \in G$ such that each $b \in G$ can be written as $a^{i}$ for some integer $i$.
  The element $a$ is called a \emph{\nameref{def:Generator}} of $G$.
\end{definition}

\begin{example}[]{Cyclic}
  Find $a$, the \nameref{def:Cyclic} term in the \nameref{def:Group} \TextIntsMod{15}?
  \tcblower{}
  \begin{equation*}
    \begin{aligned}
      \IntsMod{15} &= \lbrace 3, 6, 9, 12, 0, \ldots \rbrace \\
      &= \lbrace 3, 3+3, 3+3+3, 3+3+3+3, 5 \cdot 3, \ldots \rbrace \\
    \end{aligned}
  \end{equation*}

  Thus, $\CyclicSubgroup{a} = 3 \rightarrow \CyclicSubgroup{3}$.
\end{example}

\begin{definition}[Element Order]\label{def:Element_in_Group_Order}
  The \emph{order of an element} $a$, denoted $\ElementOrder(a)$ is defined as the least positive integer $t$ ($t \in \AllIntegers$) such that
  \begin{equation}\label{eq:Element_in_Group_Order}
    a^{t} = 1
  \end{equation}
  If such an integer $t$ does not exist, then the order of $a$ is defined to be $\infty$.

  \begin{remark}\label{rmk:Element_in_Group_Order_Term_Explanation}
    \textbf{NOTE:}
    \begin{itemize}[noitemsep]
    \item The $a^{t}$ part does not mean exponentiation, but rather repeated uses of the \nameref{def:Binary_Operation} used to create the \nameref{def:Group}.
    \item The $1$ is not a value $1$, but the Identity of the \nameref{def:Group}, as defined by \Cref{prop:Group_Properties-Identity}.
    \end{itemize}
  \end{remark}
\end{definition}

\begin{example}[Exercise 1, Problem 1.8c]{Order of Element in Group}
  Given the element $(1, 1, 1,)$ from the group of bit triples, $S = \lbrace \left( s_{0}, s_{1}, s_{2} \right) \vert s_{i} \in \IntsMod{2} \rbrace$, using an bitwise addition \nameref{def:Binary_Operation}, what is the \nameref{def:Element_in_Group_Order} of $(1, 1, 1)$?
  The identity element of this \nameref{def:Group} is $1 = (0, 0, 0)$.
  \tcblower{}
  It is important to remember the note in \Cref{rmk:Element_in_Group_Order_Term_Explanation}, especially for this problem.

  Since the given \nameref{def:Binary_Operation} was ``bitwise'', i.e.\ element-wise, I will define the operation to be $.+\,$.
  In this case, the \nameref{def:Element_in_Group_Order} of any element in this \nameref{def:Group} will be the number of element-wise additions that must occur to get the identity element, $1$.

  For this problem, since we are working in the modulo 2 domain, we have some basic facts:
  \begin{align*}
    (0 + 0) \bmod 2 &= 0 \\
    (1 + 0) \bmod 2 &= 1 \\
    (0 + 1) \bmod 2 &= 1 \\
    (1 + 1) \bmod 2 &= 0
  \end{align*}
  And for subtraction:
  \begin{align*}
    (0 - 0) \bmod 2 &= 0 \\
    (1 - 0) \bmod 2 &= 1 \\
    (0 - 1) \bmod 2 &= -1 \bmod 2 = 1 \\
    (1 - 1) \bmod 2 &= 0
  \end{align*}

  We start by constructing the equation needed to solve this problem.
  \begin{equation*}
    (1, 1, 1) .+ a = (0, 0, 0)
  \end{equation*}
  Now we can move the $(1, 1, 1)$ term over to the left, and since we are working in the modulo 2 domain, the ``negative'' that would be introduced by normal subtraction (negative addition) is irrelevant.
  Thus, we end up with
  \begin{align*}
    a &= (0, 0, 0) .+ (1, 1, 1) \\
      &= (1, 1, 1)
  \end{align*}
  This means that if $t$ from \Cref{eq:Element_in_Group_Order} is 2, we get the identity element $1$.
  So, our answer is $t = 2$, thus $\ElementOrder \bigl[ (1, 1, 1) \bigr] = 2$.
\end{example}

\begin{lemma}
  Let $a \in G$ be an element of finite \nameref{def:Element_in_Group_Order} $t$.
  Then the set of all powers of $a$ forms a \nameref{def:Cyclic} \nameref{def:Subgroup} of $G$, denoted by \TextCyclicSubgroup{a}.
  Furthermore, the \nameref{def:Element_in_Group_Order} of \TextCyclicSubgroup{a} is $t$.
\end{lemma}

\begin{definition}[Left Coset]\label{def:Left_Coset}
  Let $H$ be a \nameref{def:Subgroup} in $G$ and pick an element $a \in G$.
  A set of elements of the form
  \begin{equation}\label{eq:Left_Coset}
    aH = \lbrace ah \vert h \in H \rbrace
  \end{equation}
  is called the \emph{left coset} of $H$.
  If $G$ is commutative, it is simply called a \emph{coset}.

  The set consisting of all such left cosets is written $G/H$.
  Note that $H$ itself is a left coset.
  Furthermore, every left coset contains the same number of elements (the order of $H$) and every element is contained in exactly one left coset.

  So, the elements of $G$ are partitioned into $\SetOrder{G} / \SetOrder{H}$ different cosets, each containing \TextSetOrder{H} elements.
\end{definition}

%%% Local Variables:
%%% mode: latex
%%% TeX-master: "../../EDIN01-Cryptography-Reference_Sheet"
%%% End:
