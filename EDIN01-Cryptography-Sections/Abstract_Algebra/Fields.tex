\subsection{Fields}\label{subsec:Fields}
\begin{definition}[Field]\label{def:Field}
  A \emph{field} is a commutative \nameref{def:Ring} where all nonzero elements from the underlying set have \nameref{def:Multiplicative_Inverse}s.
\end{definition}

\begin{example}[Exercise 1, Problem 1.11]{Prove Set is Not Field}
  Prove that \TextIntsMod{4} is not a \nameref{def:Field}?
  \tcblower{}
  We begin by checking that all elements from the underlying set, \TextIntsMod{4}, have \nameref{def:Multiplicative_Inverse}s, for all the \textbf{\underline{non-zero}} elements.
  \begin{equation*}
    \IntsMod{4} = \Bigl \lbrace [0], [1], [2], [3] \Bigr \rbrace
  \end{equation*}

  First, we check $1$.
  \begin{equation*}
    1^{-1} = \gcd(1, 4) = 1a + 4b
  \end{equation*}
  \begin{align*}
    \gcd(1, 4) \Rightarrow 4 &= q1 + r \\
                             &= 4 \cdot 1 + 0 \\
    \gcd(1, 4) &= 1
  \end{align*}
  \begin{align*}
    1 &= 1 (-3) + 4 (1) \\
    a &= -3 \\
    b &= 1
  \end{align*}
  Since $a = -3$, we need to find the additive inverse:
  \begin{align*}
    -3 \bmod 4 &= (-3 + 4) \bmod 4 \\
               &= 1 \bmod 4 \\
               &= 1
  \end{align*}
  So, 1 has a \nameref{def:Multiplicative_Inverse}.

  Now we check 2.
  \begin{equation*}
    2^{-1} = \gcd(2, 4) = 2a + 4b
  \end{equation*}
  \begin{align*}
    \gcd(2, 4) \Rightarrow 4 &= q2 + r \\
                             &= 2 \cdot 2 + 0 \\
    \gcd(2, 4) &= 2
  \end{align*}
  Since $\gcd(2, 4) \neq 0$, 2 is \textbf{\underline{not}} invertible, meaning it has no \nameref{def:Multiplicative_Inverse}s.

  \TextIntsMod{4} is NOT a \nameref{def:Field} because 2 is \textbf{\underline{not}} invertible.
\end{example}

\begin{definition}[Characteristic]\label{def:Field_Characteristic}
  The \emph{characteristic} of a field is the smallest integer $m > 0$ such that
  \begin{equation}\label{eq:Field_Characteristic}
    \overbrace{1 + 1 + \cdots + 1}^{m} = 0
  \end{equation}

  If no such integer $m$ exists, the characteristic is defined to be 0.
\end{definition}

\begin{definition}[Finite Field]\label{def:Finite_Field}
  A \nameref{def:Field} is \emph{finite} only if \Cref{thm:Finite_Field} is satisfied.
\end{definition}

\begin{theorem}[Finite Field]\label{thm:Finite_Field}
  \TextIntsModN{} is a \nameref{def:Field} if and only if $n$ is a \nameref{def:Prime} number.
  If $n$ is a \nameref{def:Prime}, the \nameref{def:Field_Characteristic} of \TextIntsModN{} is $n$.
\end{theorem}

\begin{definition}[Subfield]\label{def:Subfield}
  A subset $F$ of a \nameref{def:Field} $E$ is called a \emph{subfield} of $E$ if $F$ is itself a \nameref{def:Field} with respect to the operations of $E$.

  \begin{remark}[Extension Field]\label{rmk:Extension_Field}
    Likewise, we say $E$ is an \emph{extension field} of $F$.
  \end{remark}
\end{definition}

\begin{definition}[Isomorphism]\label{def:Isomorphism}
  Two \nameref{def:Field}s are \emph{isomorphic} if they are structurally the same, but elements hvae a different representation.
  For example, for a \nameref{def:Prime}, $p$, \TextIntsMod{p} is a \nameref{def:Field} or \nameref{def:Set_Order} $p$.
  So we associate the \nameref{def:Finite_Field} \TextFiniteMathField{F}{p} with \TextIntsMod{p} and its representation.
\end{definition}

\subsubsection{Examples of \nameref*{subsec:Fields}}\label{subsubsec:Examples_of_Fields}
\begin{itemize}[noitemsep]
\item The rational Numbers: $\RationalNumbers$
\item The Real Numbers: $\RealNumbers$
\item The Complex Numbers: $\ComplexNumbers$
\item \nameref{def:Finite_Field}:
  \begin{itemize}[noitemsep]
  \item \TextIntsMod{p}, where $p$ is \nameref{def:Prime}
  \end{itemize}
\end{itemize}

%%% Local Variables:
%%% mode: latex
%%% TeX-master: "../../EDIN01-Cryptography-Reference_Sheet"
%%% End:
