\section{Number Theory}\label{sec:Number_Theory}
Before we can start with any of the deeper cryptography stuff, we need to start with some basic number theory.
\begin{definition}[Number Theory]\label{def:Number_Theory}
  \emph{Number theory} is a branch of pure mathematics devoted primarily to the study of the integers and integer-valued functions.
  Number theorists study prime numbers as well as the properties of objects made out of integers (for example, rational numbers) or defined as generalizations of the integers (for example, algebraic integers).
\end{definition}

\begin{definition}[Divides]\label{def:Divides}
  For $a, b \in \AllIntegers$, we say that $a$ \emph{divides} $b$ (written $a \Divides b$) if there exists an integer $c$ such that $b = ac$.

  Properties:
  \begin{propertylist}
  \item $a \Divides a$
  \item If $a \Divides b$ and $b \Divides c$, then $a \Divides c$.
  \item If $a \Divides b$ and $a \Divides c$, then $a \Divides (bx + cy)$ for any $x, y \in \AllIntegers$.
  \item If $a \Divides b$ and $b \Divides a$, then $a = \pm b$.
  \end{propertylist}
\end{definition}

\subsection{Integer Long Division}\label{subsec:Integer_Long_Division}
For $a, b \in \AllIntegers$, with $b \geq 1$.
Then an ordinary long division of $a$ by $b$, i.e. $a \div b$ yields two integers $q$ and $r$ such that
\begin{equation}\label{eq:Integer_Long_Division}
  a = qb + r, \text{where } 0 \leq r < b
\end{equation}

$q$ and $r$ are called the \nameref{def:Integer_Quotient} and \nameref{def:Integer_Remainder}, respectively, and are \textbf{unique}.

\begin{definition}[Quotient]\label{def:Integer_Quotient}
  The \emph{quotient}, $q$, of $a$ divided by $b$ is denoted $a \DIV b$.
\end{definition}

\begin{definition}[Remainder]\label{def:Integer_Remainder}
  The \emph{remainder}, $r$, of $a$ divided by $b$ is denoted $a \bmod b$.
\end{definition}

\begin{example}[Lecture 1, Example 1]{Integer Long Division}
  If $a=53$ and $b=9$, what is $a \bmod b$?
  \tcblower{}
  \begin{align*}
    53 &= q9 + r \\
    q &= 5 \\
    r &= 8
  \end{align*}

  Thus, $53 \bmod 9 = 8$.
\end{example}

%%% Local Variables:
%%% mode: latex
%%% TeX-master: "../../EDIN01-Cryptography-Reference_Sheet"
%%% End:


\subsection{Greatest Common Divisor}\label{subsec:Greatest_Common_Divisor}
\begin{definition}[Common Divisor]\label{def:Common_Divisor}
  An integer $c$ is a \emph{common divisor} of $a$ and $b$ if $c \Divides a$ and $c \Divides b$.
\end{definition}

\begin{definition}[Greatest Common Divisor]\label{def:GCD}
  A non-negative integer $d$ is called the \emph{greatest common divisor} (\emph{GCD}) of integers $a$ and $b$ if:
  \begin{enumerate}[noitemsep]
  \item $d$ is a \nameref{def:Common_Divisor} of $a$ and $b$.
  \item For every other common divisor $c$ it holds that $c \Divides d$.
  \end{enumerate}

  The greatest common divisor is denoted
  \begin{equation}\label{eq:GCD}
    \gcd(a, b)
  \end{equation}

  $\gcd(a,b)$ is the \textbf{largest positive} integer dividing both $a$ and $b$ (except for $\gcd(0,0)=0$).
  
  \begin{remark}
    If $a, b \in \PositiveInts$, then $\lcm(a, b) \cdot \gcd(a, b) = a \cdot b$
  \end{remark}
\end{definition}

\begin{example}[Lecture 1, Example 2]{Greatest Common Divisor}
  What is the $\gcd(18, 24)$?
  \tcblower{}
  Common Divisors = $\lbrace \pm 1, \pm 2, \pm 4, \pm 6 \rbrace$.

  Since we can only allow positive integers,
  \begin{equation*}
    \gcd(18, 24) = +6
  \end{equation*}
\end{example}

%%% Local Variables:
%%% mode: latex
%%% TeX-master: "../../EDIN01-Cryptography-Reference_Sheet"
%%% End:


\subsection{Least Common Multiple}\label{subsec:Least_Common_Multiple}

%%% Local Variables:
%%% mode: latex
%%% TeX-master: "../../EDIN01-Cryptography-Reference_Sheet"
%%% End:


\subsection{Primality}\label{subsec:Primality}
\begin{definition}[Relatively Prime]\label{def:Relatively_Prime}
  $a, b$ are called \emph{relatively prime} if $\gcd(a, b) = 1$.
\end{definition}

\begin{definition}[Prime]\label{def:Prime}
  An integer $p \geq 2$ is called \emph{prime} if its only positive divisors are $1$ and $p$.
  Otherwise, $p$ is called a \emph{\nameref{def:Composite}}.
\end{definition}

\begin{definition}[Composite]\label{def:Composite}
  An integer $p \geq 2$ is called \emph{composite} if it has more positive divisors than just $1$ and $p$.
  Otherwise, $p$ is called a \emph{\nameref{def:Prime}}.
\end{definition}

\subsubsection{Number of Primes}\label{subsubsec:Number_of_Primes}
The number of primes $\leq x$ is denoted
\begin{equation}\label{eq:Number_of_Primes}
  \pi(x)
\end{equation}
\begin{enumerate}[noitemsep]
\item There are infinitely many primes
\item $\lim\limits_{x \to \infty} \frac{\pi(x)}{\frac{x}{\ln(x)}} = 1$
\item For $x \geq 17$, $\frac{x}{\ln(x)} < \pi(x) < \frac{1.25506x}{\ln(x)}$
\end{enumerate}

%%% Local Variables:
%%% mode: latex
%%% TeX-master: "../../EDIN01-Cryptography-Reference_Sheet"
%%% End:


\subsection{Unique Factorization}\label{subsec:Unique_Factorization}

%%% Local Variables:
%%% mode: latex
%%% TeX-master: "../../EDIN01-Cryptography-Reference_Sheet"
%%% End:


\subsection{Euler Phi Function}\label{subsec:Euler_Phi_Function}

%%% Local Variables:
%%% mode: latex
%%% TeX-master: "../../EDIN01-Cryptography-Reference_Sheet"
%%% End:


\subsection{\texorpdfstring{The Integers modulo $n$}{The Integers modulo n}}\label{subsec:Integer_Modulo_n}
Let $n$ be a positive integer.
\begin{definition}[Congruence]\label{def:Congruence}
  If $a$ and $b$ are integers, then \emph{$a$ is said to be congruent to $b$ modulo $n$}, which is written as
  \begin{equation}\label{eq:A_Congruent_B}
    a \equiv b \pmod{n}
  \end{equation}

  If $n$ divides $(a-b)$, i.e. $n \Divides (a-b)$, then we call $n$ the \emph{modulus} of the congruence.

\end{definition}

\begin{theorem}
  For $a, a_{1}, b, b_{1}, c\in \AllIntegers$, we have
  \begin{propertylist}
  \item $a \equiv b \pmod{n}$ \emph{if and only if} $a$ and $b$ leave the same \nameref{def:Integer_Remainder} when divided by $n$.
  \item $a \equiv a \pmod{n}$ \label{prop:A_Congruent_B_Reflexivity}
  \item If $a \equiv b \pmod{n}$, then $b \equiv a \pmod{n}$ \label{prop:A_Congruent_B_Symmetry}
  \item If $a \equiv b \pmod{n}$ adn $b \equiv c \pmod{n}$, then $a \equiv c \pmod{n}$ \label{prop:A_Congruent_B_Transitivity}
  \item If $a \equiv a_{1} \pmod{n}$ and $b \equiv b_{1} \pmod{n}$, then $a+b = a_{1} + b_{1} \pmod{n}$ and $ab = a_{1}b_{1} \pmod{n}$.
  \end{propertylist}

  \Crefrange{prop:A_Congruent_B_Reflexivity}{prop:A_Congruent_B_Transitivity} are called \emph{reflexivity}, \emph{symmetry}, and \emph{transitivity}, respectively.
\end{theorem}

\begin{example}[Exercise 1, Question 1.2b]{Integers modulo n}
  Write all the units (\nameref{def:Invertible_Element}) in \TextIntsMod{36}?
  \tcblower{}
  First, we start by constructing our set of integers modulo $n$.
  \begin{equation*}
    \IntsMod{36} = \bigl\lbrace [0], [1], [2], [3], [4], \ldots, [35] \bigr\rbrace
  \end{equation*}

  Since we are only worried about the units of \TextIntsMod{36}, we need to find the integers that satisfy \Cref{eq:Invertible_Element}.
  This is done by finding an $a$ value that has a \nameref{def:Multiplicative_Inverse}, which requires that \Cref{eq:Invertible} be true, namely
  \begin{equation*}
    \gcd(a, n) = 1
  \end{equation*}
  This leaves us with
  \begin{equation*}
    \IntsMod{36} = \bigl\lbrace [1], [5], [7], [11], [13], [17], [19], [23], [25], [29], [31], [35] \bigr\rbrace
  \end{equation*}
  which is the solution.
\end{example}

\subsection{Equivalence Classes}\label{subsec:Equivalence_Classes}
\begin{definition}[Equivalence Class]\label{def:Equivalence_Class}
  \nameref{def:Congruence} modulo $n$ partitions $\AllIntegers$ into $n$ sets, called \emph{equivalence class}es, where each integer belongs to exactly one equivalence class.

  For example, these are all congruent to each other modulo $n$:
  \begin{subequations}\label{eq:Equivalence_Class}
    \begin{equation}\label{subeq:Equivalence_Class_Remainder_0}
      [0] = \lbrace \ldots, -2n, -n,\, 0, n, 2n, \ldots \rbrace
    \end{equation}
    \begin{equation}\label{subeq:Equivalence_Class_Remainder_1}
      [1] = \lbrace \ldots -2n + 1, -n+1,\, 1, n+1, 2n+1 \ldots \rbrace
    \end{equation}
    \begin{equation}\label{eq:General_Equivalent_Class_Remainder}
      [r] = \IntsMod{r} = (x \bmod r) + n\AllIntegers
    \end{equation}
  \end{subequations}

  Since all elements in an equivalent class have the same \nameref{def:Integer_Remainder}, $r$, we use $r$ as a \emph{represenatative} for the equivalence class.
  \begin{remark}
    In this case, the representatives of the equivalence classes shown in \Crefrange{subeq:Equivalence_Class_Remainder_0}{subeq:Equivalence_Class_Remainder_1} are 0 and 1, respectively, and consist of all integers that are mod 0 or mod 1, respectively.
  \end{remark}
\end{definition}

%%% Local Variables:
%%% mode: latex
%%% TeX-master: "../EDIN01-Cryptography-Reference_Sheet"
%%% End:
