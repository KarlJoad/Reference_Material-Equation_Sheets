\section{Number Theory}\label{sec:Number_Theory}
Before we can start with any of the deeper cryptography stuff, we need to start with some basic number theory.
\begin{definition}[Number Theory]\label{def:Number_Theory}
  \emph{Number theory} is a branch of pure mathematics devoted primarily to the study of the integers and integer-valued functions.
  Number theorists study prime numbers as well as the properties of objects made out of integers (for example, rational numbers) or defined as generalizations of the integers (for example, algebraic integers).
\end{definition}

\begin{definition}[Divides]\label{def:Divides}
  For $a, b \in \AllIntegers$, we say that $a$ \emph{divides} $b$ (written $a \Divides b$) if there exists an integer $c$ such that $b = ac$.

  Properties:
  \begin{propertylist}
  \item $a \Divides a$
  \item If $a \Divides b$ and $b \Divides c$, then $a \Divides c$.
  \item If $a \Divides b$ and $a \Divides c$, then $a \Divides (bx + cy)$ for any $x, y \in \AllIntegers$.
  \item If $a \Divides b$ and $b \Divides a$, then $a = \pm b$.
  \end{propertylist}
\end{definition}

\subsection{Integer Long Division}\label{subsec:Integer_Long_Division}
For $a, b \in \AllIntegers$, with $b \geq 1$.
Then an ordinary long division of $a$ by $b$, i.e. $a \div b$ yields two integers $q$ and $r$ such that
\begin{equation}\label{eq:Integer_Long_Division}
  a = qb + r, \text{where } 0 \leq r < b
\end{equation}

$q$ and $r$ are called the \nameref{def:Integer_Quotient} and \nameref{def:Integer_Remainder}, respectively, and are \textbf{unique}.

\begin{definition}[Quotient]\label{def:Integer_Quotient}
  The \emph{quotient}, $q$, of $a$ divided by $b$ is denoted $a \DIV b$.
\end{definition}

\begin{definition}[Remainder]\label{def:Integer_Remainder}
  The \emph{remainder}, $r$, of $a$ divided by $b$ is denoted $a \bmod b$.
\end{definition}

\begin{example}[Lecture 1, Example 1]{Integer Long Division}
  If $a=53$ and $b=9$, what is $a \bmod b$?
  \tcblower{}
  \begin{align*}
    53 &= q9 + r \\
    q &= 5 \\
    r &= 8
  \end{align*}

  Thus, $53 \bmod 9 = 8$.
\end{example}

%%% Local Variables:
%%% mode: latex
%%% TeX-master: "../../EDIN01-Cryptography-Reference_Sheet"
%%% End:


\subsection{Greatest Common Divisor}\label{subsec:Greatest_Common_Divisor}

%%% Local Variables:
%%% mode: latex
%%% TeX-master: "../../EDIN01-Cryptography-Reference_Sheet"
%%% End:


\subsection{Least Common Multiple}\label{subsec:Least_Common_Multiple}
\begin{definition}[Least Common Multiple]\label{def:LCM}
  A non-negative integer $d$ is called the \emph{least common multiple} (\emph{LCM}) of integers $a$ and $b$ if:
  \begin{enumerate}[noitemsep]
  \item $a \Divides d$ and $b \Divides d$
  \item For every integer $c$ such that $a \Divides c$ and $b \Divides c$, we have $d \Divides c$.
  \end{enumerate}

  The least common multiple is denoted
  \begin{equation}\label{eq:LCM}
    \lcm(a, b)
  \end{equation}

  $\lcm(a, b)$ is the \textbf{smallest positive} integer divisible by both $a$ and $b$.

  \begin{remark}
    If $a, b \in \PositiveInts$, then $\lcm(a, b) \cdot \gcd(a, b) = a \cdot b$
  \end{remark}
\end{definition}

%%% Local Variables:
%%% mode: latex
%%% TeX-master: "../../EDIN01-Cryptography-Reference_Sheet"
%%% End:


\subsection{Primality}\label{subsec:Primality}
\begin{definition}[Relatively Prime]\label{def:Relatively_Prime}
  $a, b$ are called \emph{relatively prime} if $\gcd(a, b) = 1$.
\end{definition}

\begin{definition}[Prime]\label{def:Prime}
  An integer $p \geq 2$ is called \emph{prime} if its only positive divisors are $1$ and $p$.
  Otherwise, $p$ is called a \emph{\nameref{def:Composite}}.
\end{definition}

\begin{definition}[Composite]\label{def:Composite}
  An integer $p \geq 2$ is called \emph{composite} if it has more positive divisors than just $1$ and $p$.
  Otherwise, $p$ is called a \emph{\nameref{def:Prime}}.
\end{definition}

\subsubsection{Number of Primes}\label{subsubsec:Number_of_Primes}
The number of primes $\leq x$ is denoted
\begin{equation}\label{eq:Number_of_Primes}
  \pi(x)
\end{equation}
\begin{enumerate}[noitemsep]
\item There are infinitely many primes
\item $\lim\limits_{x \to \infty} \frac{\pi(x)}{\frac{x}{\ln(x)}} = 1$
\item For $x \geq 17$, $\frac{x}{\ln(x)} < \pi(x) < \frac{1.25506x}{\ln(x)}$
\end{enumerate}

%%% Local Variables:
%%% mode: latex
%%% TeX-master: "../../EDIN01-Cryptography-Reference_Sheet"
%%% End:


\subsection{Unique Factorization}\label{subsec:Unique_Factorization}
\begin{theorem}[Unique Factorization Theorem]\label{thm:Unique_Factorization_Theorem}
  Every integer $n \geq 2$ can be written as a product of prime powers,
  \begin{equation*}
    n = p_{1}^{e_{1}} p_{2}^{e_{2}} \cdots p_{k}^{e_{k}}
  \end{equation*}
  where $p_{1}, p_{2}, \ldots p_{k}$ are distinct primes and $e_{1}, e_{2}, \ldots e_{k}$ are positive integers.
  Furthermore, the factorization is unique up to rearrangement of the factors.
\end{theorem}

\subsubsection{\nameref*{def:GCD} and \nameref*{def:LCM} with Unique Factors}\label{subsubsec:GCD_LCM_Unique_Factors}
If $a = p_{1}^{e_{1}} p_{2}^{e_{2}} \cdots p_{k}^{e_{k}}$ and $b = p_{1}^{f_{1}} p_{2}^{f_{2}} \cdots p_{k}^{e_{k}}$, where $e_{i}, f_{i}, i = 1, 2, \ldots k$ are non-negative integers, then
\begin{equation}\label{eq:GCD_Unique_Factors}
  \gcd(a, b) = p_{1}^{\min(e_{1}, f_{1})} p_{2}^{\min(e_{2}, f_{2})} \cdots p_{k}^{\min(e_{k}, f_{k})}
\end{equation}
and
\begin{equation}\label{eq:LCM_Unique_Factors}
  \lcm(a, b) = p_{1}^{\max(e_{1}, f_{1})} p_{2}^{\max(e_{2}, f_{2})} \cdots p_{k}^{\max(e_{k}, f_{k})}
\end{equation}

%%% Local Variables:
%%% mode: latex
%%% TeX-master: "../../EDIN01-Cryptography-Reference_Sheet"
%%% End:


\subsection{Euler Phi Function}\label{subsec:Euler_Phi_Function}

%%% Local Variables:
%%% mode: latex
%%% TeX-master: "../../EDIN01-Cryptography-Reference_Sheet"
%%% End:


\subsection{\texorpdfstring{The Integers modulo $n$}{The Integers modulo n}}\label{subsec:Integer_Modulo_n}

%%% Local Variables:
%%% mode: latex
%%% TeX-master: "../../EDIN01-Cryptography-Reference_Sheet"
%%% End:

\subsection{Equivalence Classes}\label{subsec:Equivalence_Classes}
\begin{definition}[Equivalence Class]\label{def:Equivalence_Class}
  \nameref{def:Congruence} modulo $n$ partitions $\AllIntegers$ into $n$ sets, called \emph{equivalence class}es, where each integer belongs to exactly one equivalence class.

  For example, these are all congruent to each other modulo $n$:
  \begin{subequations}\label{eq:Equivalence_Class}
    \begin{equation}\label{subeq:Equivalence_Class_Remainder_0}
      [0] = \lbrace \ldots, -2n, -n,\, 0, n, 2n, \ldots \rbrace
    \end{equation}
    \begin{equation}\label{subeq:Equivalence_Class_Remainder_1}
      [1] = \lbrace \ldots -2n + 1, -n+1,\, 1, n+1, 2n+1 \ldots \rbrace
    \end{equation}
    \begin{equation}\label{eq:General_Equivalent_Class_Remainder}
      [r] = \IntsMod{r} = (x \bmod r) + n\AllIntegers
    \end{equation}
  \end{subequations}

  Since all elements in an equivalent class have the same \nameref{def:Integer_Remainder}, $r$, we use $r$ as a \emph{represenatative} for the equivalence class.
  \begin{remark}
    In this case, the representatives of the equivalence classes shown in \Crefrange{subeq:Equivalence_Class_Remainder_0}{subeq:Equivalence_Class_Remainder_1} are 0 and 1, respectively, and consist of all integers that are mod 0 or mod 1, respectively.
  \end{remark}
\end{definition}

%%% Local Variables:
%%% mode: latex
%%% TeX-master: "../EDIN01-Cryptography-Reference_Sheet"
%%% End:
