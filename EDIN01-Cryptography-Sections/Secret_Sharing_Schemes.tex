\section{Secret Sharing Schemes}\label{sec:Secret_Sharing_Schemes}
\begin{definition}[Threshold Scheme]\label{def:Threshold_Scheme}
  Let $k$ and $n$ be positive integers, where $k \leq n$.
  A $(k, n)$-\emph{threshold scheme} is a method of sharing a secret key $K$ among a set of $n$ participants in such a way that any $k$ participants can compute the value of the secret, but no group of $k-1$ or fewer may do so.

  \begin{itemize}[noitemsep]
  \item The set of participants is denoted $\Participants$.
  \item The secret $K$ is chosen by the \emph{dealer} $D$.
  \item When $D$ shares the secret $K$ amount the participants in $\Participants$, $D$ gives each participant some partial information, called a \emph{share}.
  \item The shares are secret. No participant knows any other participant's share.
  \item At a later time, some participants, where $\SomeParticipants \subseteq \Participants$, would like to compute $K$. They members of $\SomeParticipants$ pool their shares together and compute $K$.
  \end{itemize}
\end{definition}

\subsection{Shamir Threshold Scheme}\label{subsec:Shamir_Threshold_Scheme}
\begin{definition}[Shamir Threshold Scheme]\label{def:Shamir_Threshold_Scheme}
  The \emph{Shamir threshold scheme} is a famous \nameref{def:Threshold_Scheme}.
  It uses:
  \begin{itemize}[noitemsep]
  \item $\Participants = \lbrace P_{1}, P_{2}, \ldots, P_{n} \rbrace$.
  \item $\Keyspace$ is the set of all secrets.
  \item $\Shares$ is the set of all shares.
  \item $\Keyspace = \IntsMod{p}$, where $p \geq n + 1$ is a \nameref{def:Prime}.
  \item $\Shares = \IntsMod{p}$.
  \end{itemize}

  \begin{equation}\label{eq:Shamir_Threshold_Scheme}
    a(x) = \sum\limits_{j=0}^{k-1} a_{j} x^{j} \bmod p
  \end{equation}

  \begin{algorithm}[H]
    \DontPrintSemicolon{}
    \SetKwInOut{Init}{Initialization}

    \Init{$D$ chooses $n$ distinct non-zero elements from $\IntsMod{p}$, denoted $x_{i}$, where $1 \leq i \leq n$.
      All these values are public.}
    \BlankLine{}

    \textbf{Distribution of Shares} \;
    $D$ wants to share the secret $K \in \IntsMod{p}$.
    $D$ constructs a random \nameref{def:Polynomial} of \nameref{def:Polynomial_Degree} at most $k-1$.
    $D$ randomly chooses $k-1$ elements of $\IntsMod{p}$, denoted $a_{1}, a_{2}, \ldots, a_{k-1}$. \;
    \textbf{$K = a_{0} = a(0)$}.
    \BlankLine{}
    $D$ computes $y_{i} = a(x_{i})$ for $1 \leq i \leq n$, using \Cref{eq:Shamir_Threshold_Scheme}. \;
    \begin{equation*}
      a(x) = \sum\limits_{j=0}^{k-1} a_{j}x^{j} \bmod p
    \end{equation*} \;
    $D$ gives participant $P_{i}$ the share $y_{i}$ and the value $x_{i}$ as a pair, $(x_{i}, y_{i})$. \;
    \BlankLine{}
    Now, any $k$ participants can reconstruct the polynomial $a(x)$ and get the secret $K$ back. \;
    However, if there are $k-1$ or fewer participants, there is no way to recover $K = a_{0} = a(0)$.
    \caption{Shamir Threshold Scheme}
    \label{algo:Shamir_Threshold_Scheme}
  \end{algorithm}
\end{definition}

\subsubsection{\texorpdfstring{How can $k$ Participants Reconstruct $a(x)$?}{Successfully Reconstruct the Key}}\label{subsubsec:How_k_Participants_Reconstruct}
\begin{enumerate}[noitemsep]
\item Assume that $\SomeParticipants = \lbrace P_{1}, P_{2}, \ldots, P_{k} \rbrace$.
\item The participants in $\SomeParticipants$ know
  \begin{equation*}
    y_{i} = a(x_{i}), \; 1 \leq i \leq k
  \end{equation*}
  where $a(x) \in \PolynomialRing{Z}{p}{x}$ is the secret \nameref{def:Polynomial}.
\item The \nameref{def:Polynomial} $a(x)$ has \nameref{def:Polynomial_Degree} at most $k-1$, and can be written
  \begin{equation*}
    a(x) = a_{0} + a_{1}x + \cdots + a_{k-1} x^{k-1}
  \end{equation*}
  where the coefficients $a_{0}, a_{1}, \ldots, a_{k-1}$ are unknown elements of $\IntsMod{p}$ and $a_{0} = a(0) = K$ is the secret.
\item Each participant, knowing their $y_{i} = a(x_{i})$ can obtain one linear equation in the $k$ unknowns ($a_{0}, a_{1}, \ldots, a_{k-1}$).
\item Now, the group of participants $\SomeParticipants$ has $k$ linear equations at its disposal.
\item If all $k$ linear equations are linearly independent, there will be a unique solution, and $a_{0}$ will be revealed. It can be written in 2 ways:
  \begin{enumerate}[noitemsep]
  \item As a system of linear equations.
    \begin{align*}
      P_{0} : a_{0} + a_{1}x_{0} + x_{2}x_{0}^{2} + \cdots + a_{k-1}x_{0}^{k-1} &= y_{0} \\
      P_{1} : a_{0} + a_{1}x_{1} + x_{2}x_{1}^{2} + \cdots + a_{k-1}x_{1}^{k-1} &= y_{1} \\
       &\vdots \\
      P_{k-1} : a_{0} + a_{1}x_{k-1} + x_{2}x_{k-1}^{2} + \cdots + a_{k-1}x_{k-1}^{k-1} &= y_{k-1}
    \end{align*}
  \item As a matrix.
    \begin{equation*}
      \begin{pmatrix}
        1 & x_{0} & x_{0}^{2} & \cdots & x_{0}^{k-1} \\
        1 & x_{1} & x_{1}^{2} & \cdots & x_{1}^{k-1} \\
        \vdots & \vdots & \vdots & \ddots & \vdots \\
        1 & x_{k-1} & x_{k-1}^{2} & \cdots & x_{k-1}^{k-1}
      \end{pmatrix}
      \begin{pmatrix}
        a_{0} \\
        a_{1} \\
        \vdots \\
        a_{k-1}
      \end{pmatrix} =
      \begin{pmatrix}
        y_{0} \\
        y_{1} \\
        \vdots \\
        y_{k-1}
      \end{pmatrix}
    \end{equation*}
    \begin{itemize}[noitemsep]
    \item The matrix of coefficients, $a_{0}, a_{1}, \ldots, a_{k-1}$ called $A$ is a Vandermonde Matrix.
    \item The determinant of a Vandermonde Matrix has a formula, \Cref{eq:Vandermonde_Matrix_Determinant}.
      \begin{equation}\label{eq:Vandermonde_Matrix_Determinant}
        \det A = \prod\limits_{1 \leq i \leq j \leq k} (x_{i} - x_{j}) \bmod p
      \end{equation}
    \item Since all $x_{i}$'s are distinct, the product cannot be 0, thus the $\det A \neq 0$.
    \item A non-zero determinant implies a unique solution over $\IntsMod{p}$.
    \end{itemize}
  \end{enumerate}
\end{enumerate}

\begin{example}[Lecture 15, Example 1]{Shamir Threshold Scheme Reconstruct Secret}
  Given $p=17$, $k=3$, $n=5$, and $x_{i} = i$, where $i$ is the participant number, and
  \begin{center}
    \begin{tabular}{c|c}
      \toprule
      $i$ & Shares \\
      \midrule
      $P_{1}$ & 8 \\
      $P_{3}$ & 10 \\
      $P_{5}$ & 11 \\
      \bottomrule
    \end{tabular}
  \end{center}

  Find $K$?
  \tcblower{}
  First thing to note is that $k=3$ the degree of the secret \nameref{def:Polynomial}.
  \begin{equation*}
    a(x) = a_{0} + a_{1}x + a_{2}x^{2}
  \end{equation*}

  Using the public shares that each participant brings in, we can compute $a_{0} = K$.
  Start by creating the system of linear equations.
  The first is:
  \begin{align*}
    a_{0} + a_{1}x_{1} + a_{2}x_{1}^{2} &= 8 \\
    a_{0} + a_{1}(1) + a_{2}{(1)}^{2} &= \\
    a_{0} + a_{1} + a_{2} &= \\
  \end{align*}

  The second is:
  \begin{align*}
    a_{0} + a_{1}x_{3} + a_{2}x_{3}^{2} &= 10 \\
    a_{0} + a_{1}(3) + a_{2}{(3)}^{2} &= \\
    a_{0} + 3a_{1} + 9a_{2} &= \\
  \end{align*}

  The third is:
  \begin{align*}
    a_{0} + a_{1}x_{5} + a_{2}x_{5}^{2} &= 11 \\
    a_{0} + a_{1}(5) + a_{2}{(5)}^{2} &= \\
    a_{0} + 5a_{1} + 25a_{2} \bmod 17 &= \\
    a_{0} + 5a_{1} + 8a_{2} &= \\
  \end{align*}

  So,
  \begin{align*}
    a_{0} + a_{1} + a_{2} &= 8 \\
    a_{0} + 3a_{1} + 9a_{2} &= 10 \\
    a_{0} + 5a_{1} + 8a_{2} &= 11 \\
  \end{align*}

  Solving this system yields:
  \begin{align*}
    a_{2} &= 2 \\
    a_{1} &= 10 \\
    a_{0} &= 13 \\
  \end{align*}

  So, our secret is $K=13$.
\end{example}

\subsubsection{\texorpdfstring{Why $k-1$ Participants Cannot Reconstruct $a(x)$?}{Fail to Reconstruct Secret}}\label{subsubsec:Fail_Reconstruct_Secret}
Why can't $k-1$ participants reconstruct the secret?
\begin{enumerate}[noitemsep]
\item Proceeding as above, the participants end up with $k-1$ linear equations, but $k$ unknowns.
\item Suppose that $K=y_{0}$. Since $K=a_{0}=y_{0}$, we have
  \begin{equation*}
    y_{0} = a(0)
  \end{equation*}
  which gives us our last $k$th equation.
\item Now there is a unique solution for $a(x)$.
\item However, there is a unique solution for every possible value of $y_{0}$ in the field.
  For every possible value $y_{0}$ of the secret $K$, there is a unique \nameref{def:Polynomial} $a_{y_{0}}(x)$ such that
  \begin{equation*}
    y_{i} = a_{y_{0}}(x_{i})
  \end{equation*}
  for $1 \leq i \leq k-1$ and such that
  \begin{equation*}
    y_{0} = a_{y_{0}}(0)
  \end{equation*}
\item Thus, this group cannot find the single unique solution, because all values are possible.
\end{enumerate}

\subsection{Alternative Way to Calculate Secret Polynomial}\label{subsec:Alternative_Calculate_Secret_Polynomial}
\begin{definition}[Lagrange Interpolation Formula]\label{def:Lagrange_Interpolation_Formula}
  The \emph{Lagrange interpolation formula} is an explicit formula for the unique \nameref{def:Polynomial} $a(x)$ with degree at most $k-1$, when the values in $k$ different points are given, i.e.
  \begin{align*}
    y_{0} &= a(x_{0}) \\
    y_{1} &= a(x_{1}) \\
          &\vdots \\
    y_{k-1} &= a(x_{k-1})
  \end{align*}

  \begin{equation}\label{eq:Lagrange_Interpolation_Formula}
    a(x) = \sum\limits_{i=0}^{k-1} y_{i} \prod\limits_{0 \leq j \leq k,\; j \neq i} \frac{x-x_{j}}{x_{i}-x_{j}}
  \end{equation}
\end{definition}

However, the group calculating the secret value $K = a_{0} = a(0)$ is only interested in the $a(0)$ term, which yields this formula.
\begin{equation}\label{eq:Lagrange_Interpolation_Formula-Useful}
  a(0) = K = \sum\limits_{i=0}^{k-1} y_{i} \prod\limits_{0 \leq j \leq k,\; j \neq i} \frac{x_{j}}{x_{j}-x_{i}}
\end{equation}

\begin{example}[Lecture 16, Example 2]{Lagrange Interpolation Function}
  Given $p=17$, $k=3$, $n=5$, and $x_{i} = i$, where $i$ is the participant number, and
  \begin{center}
    \begin{tabular}{c|c}
      \toprule
      $i$ & Shares \\
      \midrule
      $P_{1}$ & 8 \\
      $P_{3}$ & 10 \\
      $P_{5}$ & 11 \\
      \bottomrule
    \end{tabular}
  \end{center}

  Find $K$ using \Cref{eq:Lagrange_Interpolation_Formula-Useful}?
  \tcblower{}
  We start by plugging in our values, and ``interpreting'' the actual equation.
  \begin{equation*}
    a(0) = K = \sum\limits_{i=0}^{k-1} y_{i} \prod\limits_{0 \leq j \leq k,\; j \neq i} \frac{x_{j}}{x_{j}-x_{i}}
  \end{equation*}
  \begin{align*}
    K &= 8 \left( \frac{3 \cdot 5}{(3-1)(5-1)} \right) + 10 \left( \frac{1 \cdot 5}{(1-3)(5-3)} \right) + 11 \left( \frac{1 \cdot 3}{(1-5)(1-3)} \right) \\
      &= 8 \left( \frac{15}{2 \cdot 4} \right) + 10 \left( \frac{5}{-2 \cdot 2} \right) + 11 \left( \frac{3}{-4 \cdot -2} \right) \\
      &= 8 \left( \frac{15}{8} \right) + 5 \left( \frac{5}{-2} \right) + 11 \left( \frac{3}{8} \right) \\
      &= 15 + 5 (5)(-2^{-1} \bmod 17) + 11(3)(8^{-1} \bmod 17)
  \end{align*}

  After evaluating the last line, you end up with
  \begin{equation*}
    K = 13
  \end{equation*}
  just like before.
\end{example}

\subsection{\texorpdfstring{Simplified Construction for $(k, k)$-Threshold Schemes}{Simplified Construction for Threshold Schemes}}\label{subsec:Simplified_Construction_Threshold_Schemes}
Consider the following $(k, k)$-threshold scheme.
\begin{enumerate}[noitemsep]
\item The dealer $D$ chooses $k$ random shares, $y_{1}, y_{2}, \ldots, y_{k}$ from $\IntsMod{m}$ and gives $y_{i}$ to participant $i$ as their share.
\item The secret $K$ is chosen to be
  \begin{equation}\label{eq:Simplified_Construction_Threshold_Scheme}
    K = \sum\limits_{i=1}^{k}y_{i} \bmod m
  \end{equation}
\item The $k$ participants can recover the secret by just adding all their shares together. Though this requires \textbf{ALL} participants.
\end{enumerate}

\subsection{Access Structures}\label{subsec:Access_Structures}
\begin{definition}[Access Structure]\label{def:Access_Structure}
  An \emph{access structure}, $\AccessStructure$, is the subset of participants that are qualified to compute the secret $K$.
  The access structure $\AccessStructure$ is a method of sharing a secret $K$ among a set of $n$ participants in such a way that the following properties hold.
  \begin{propertylist}
  \item If $\SomeParticipants \in \AccessStructure$, then $\Entropy(K \Given B) = 0$
  \item If $\SomeParticipants \notin \AccessStructure$, then $\Entropy(K \Given B) \geq \log_{2} \alpha$, where $\alpha$ is a fixed value and $\alpha > 1$.
  \end{propertylist}

  Let $\AccessStructure$ be a subset of $\Participants$, $\AccessStructure \subseteq 2^{\Participants}$.
  If $\SomeParticipants \subseteq \Participants$, then $B$ denotes the set of shares for $\SomeParticipants$.

  \begin{remark}[$(k, n)$-Threshold Access Structure]\label{rmk:kn_Threshold_Access_Structure}
    A \nameref{def:Shamir_Threshold_Scheme} is a form of an access structure where
    \begin{equation*}
      \AccessStructure = \lbrace \SomeParticipants \subseteq \Participants; \SetOrder{\SomeParticipants} \geq k \lbrace
    \end{equation*}
  \end{remark}
\end{definition}

\begin{definition}[Perfect]\label{def:Perfect_Secret_Sharing_Scheme}
  A secret sharing scheme for which the equation below holds is called a \emph{perfect} secret sharing scheme.
  \begin{equation}\label{eq:Perfect_Secret_Sharing_Scheme}
    \alpha = \SetOrder{\Keyspace}
  \end{equation}

  In a perfect sharing scheme, any non-qualified subset of participants cannot gather any information about the secret.
  Because no non-qualified subset of participants can gather \textbf{any} information about the secret, this is the most desirable aspect of a secret sharing scheme.

  \begin{remark}[Ability to Construct]\label{rmk:Ability_Construct_Perfect_Secret_Sharing_Scheme}
    A \nameref{def:Perfect_Secret_Sharing_Scheme} \nameref{def:Access_Structure} can \textbf{always} be constructed with any \nameref{def:Access_Structure}.
  \end{remark}
\end{definition}

\begin{definition}[Ideal]\label{def:Ideal_Secret_Sharing_Scheme}
  A secret sharing scheme is \emph{ideal} if it is both \nameref{def:Perfect_Secret_Sharing_Scheme} and if
  \begin{equation}\label{eq:Ideal_Secret_Sharing_Scheme}
    \SetOrder{\Shares} = \SetOrder{\Keyspace}
  \end{equation}

  An ideal secret sharing scheme is both \nameref{def:Perfect_Secret_Sharing_Scheme} and the number of shares distributed is equal to the total keyspace.

  \begin{remark}[Ability to Construct]\label{rmk:Ability_Construct_Ideal_Secret_Sharing_Scheme}
    A \nameref{def:Ideal_Secret_Sharing_Scheme} \nameref{def:Access_Structure} can be created only with certain \nameref{def:Access_Structure}s.
  \end{remark}
\end{definition}

\subsubsection{Properties of \nameref*{subsec:Access_Structures}}\label{subsubsec:Properties_Access_Structures}
\begin{definition}[Closure]\label{def:Closure_Access_Structure}
  The \emph{closure} of an \nameref{def:Access_Structure}, denoted $\Closure(\AccessStructure)$ are all the subsets that are qualified from the \nameref{def:Access_Structure} $\AccessStructure$.
  The closure is defined in the equation below.
  \begin{equation}\label{eq:Closure_Access_Structure}
    \Closure(\AccessStructure) = \lbrace \MoreParticipants \subseteq \Participants; \SomeParticipants \subseteq \MoreParticipants, \SomeParticipants \in \AccessStructure \rbrace
  \end{equation}

  For example,
  \begin{align*}
    \AccessStructure &= \lbrace \lbrace P_{1} \rbrace, \lbrace P_{1}, P_{2} \rbrace, \lbrace P_{1}, P_{3} \rbrace, \lbrace P_{1}, P_{2}, P_{3} \rbrace, \lbrace P_{2}, P_{3} \rbrace \rbrace \\
    \Closure(\AccessStructure) &= \lbrace P_{1}, P_{2}P_{3} \rbrace
  \end{align*}
  In this case, $P_{1}$ can access the secret by themself.
  If anyone joins $P_{1}$, they also get access to the secret.
  However, $P_{2}$ \textbf{and} $P_{3}$ must combine their information to gather the secret if $P_{1}$ is not available.
\end{definition}

\begin{definition}[Monotone]\label{def:Monotone_Access_Structure}
  Suppose $\SomeParticipants \in \AccessStructure$ and we add one more participant (qualified or non-qualified) to the set $\SomeParticipants$, creating the set $\MoreParticipants$.
  The set $\MoreParticipants$ must also be qualified, since it contains a subset that is qualified, namely $\SomeParticipants$.
  This \nameref{def:Access_Structure} is called a \emph{monotone} access structure.
  \begin{equation}\label{eq:Monotone_Access_Structure}
    \text{If } \SomeParticipants \in \AccessStructure \text{ and } \SomeParticipants \subseteq \MoreParticipants \subseteq \Participants, \text{ then } \MoreParticipants \in \AccessStructure
  \end{equation}

  \begin{remark}[Monotone and Closures]\label{rmk:Monotone_Closure_Access_Structure}
    An \nameref{def:Access_Structure} is \nameref{def:Monotone_Access_Structure} \textbf{if and only if}
    \begin{equation}\label{eq:Monotone_Closure_Access_Structure}
      \AccessStructure = \Closure(\AccessStructure)
    \end{equation}
  \end{remark}

  \begin{remark}[Use of Access Structures in this Document]
    In this document and course, we assume that if $\AccessStructure$ is \textbf{not} \nameref{def:Monotone_Access_Structure}, we really mean the \nameref{def:Closure_Access_Structure} of $\AccessStructure$ ($\Closure(\AccessStructure)$).
  \end{remark}
\end{definition}

\subsubsection{Constructing a \nameref*{def:Perfect_Secret_Sharing_Scheme} For Any Access Structure}\label{subsubsec:Construct_Perfect_Secret_Sharing_Scheme}
%%% Local Variables:
%%% mode: latex
%%% TeX-master: "../EDIN01-Cryptography-Reference_Sheet"
%%% End:
