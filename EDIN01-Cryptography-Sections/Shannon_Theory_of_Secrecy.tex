\section{Shannon's Theory of Secrecy}\label{sec:Shannon_Theory_of_Secrecy}
\begin{definition}[Shannon's Theory of Secrecy]\label{def:Shannon_Theory_of_Secrecy}
  \emph{Shannon's Theory of Secrecy} was developed by \href{https://en.wikipedia.org/wiki/Claude_Shannon}{Claude Shannon}.
  It is an attack/defense model for a \nameref{def:Symmetric_Encryption} system.

  This model operates under the assumptions that there is \nameref{def:Security-Unconditional} and only \nameref{def:Attack-Ciphertext_Only}s.

  \begin{remark}[Flaws of this Model's Assumptions]\label{rmk:Shannon_Theory_of_Secrecy_Assumption_Flaws}
    These assumptions are very strong.
    In reality, these assumptions would correspond to the \textbf{absolute best-case scenario}.
  \end{remark}

  There is a set of \nameref{def:Encryption_Rule}s, one for each $k$, which maps \nameref{def:Plaintext} letters in a message $\mathbf{m} = m_{1}, m_{2}, \ldots, m_{i} \in \Plaintexts$ to \nameref{def:Ciphertext} letters $\mathbf{c} = c_{1}, c_{2}, \ldots, c_{i} \in \Ciphertexts$.
\end{definition}

\subsection{Attack and Security Assumptions}\label{subsec:Shannon_Attack_Security_Assumptions}
There are several possible assumptions for Eve's attack on Alice's message.
\begin{enumerate}[noitemsep]
\item \emph{\nameref{def:Attack-Ciphertext_Only}}: Eve has only the ciphertext $C$, and wants to get the key $K$ or the plaintext message $M$.
\item \emph{\nameref{def:Attack-Known_Plaintext}}: Eve has both the ciphertext $C$ and the plaintext message $M$, and wants the key $K$.
\item \emph{\nameref{def:Attack-Chosen_Plaintext}}: Eve knows $M$ and and can arbitrarily choose $M$ to get $C$ back. She wants to get the key $K$.
\item \emph{\nameref{def:Attack-Chosen_Ciphertext}}: Eve has the \nameref{def:Decryption_Rule}, $d_{K}$ and can feed in any arbitrary $C$. She is attempting to find the key $K$.
\item \emph{\nameref{def:Attack-Related_Key}}: The message $M$ is encrypted with a key $K$ similar to another.
\item \emph{\nameref{def:Attack-Side_Channel}}: Eve is attacking the implementation of the \nameref{def:Encryption_Rule}, $e_{K}$, by observing some other outputs.
  \begin{itemize}[noitemsep]
  \item The power consumed by the algorithm
  \item The network communcations occurring
  \end{itemize}
\end{enumerate}

\begin{definition}[Ciphertext-Only Attack]\label{def:Attack-Ciphertext_Only}
  In a \emph{ciphertext-only attack}, Eve has only the ciphertext $C$, and wants to get the key $K$ or the plaintext message $M$.
\end{definition}

\begin{definition}[Known-Plaintext Attack]\label{def:Attack-Known_Plaintext}
  In a \emph{known-plaintext attack}, Eve has both the ciphertext $C$ and the plaintext message $M$, and wants the key $K$.
\end{definition}

\begin{definition}[Chosen-Plaintext Attack]\label{def:Attack-Chosen_Plaintext}
  In a \emph{chosen-plaintext attack}, Eve knows $M$ and and can arbitrarily choose $M$ to get $C$ back. She wants to get the key $K$.
\end{definition}

\begin{definition}[Chosen-Ciphertext Attack]\label{def:Attack-Chosen_Ciphertext}
  In a \emph{chosen-ciphertext attack}, Eve has the \nameref{def:Decryption_Rule}, $d_{K}$ and can feed in any arbitrary $C$.
  She is attempting to find the key $K$.

  \begin{remark}[Lunchtime Attack]\label{rmk:Attack-Lunchtime}
    Sometimes the \nameref{def:Attack-Chosen_Ciphertext} is called a \emph{lunchtime attack}, because an arbitrary \nameref{def:Plaintext} can be fed through the \nameref{def:Encryption_Rule} quickly.
    This means the output can be found quickly, for example, while a coworker is having lunch.
  \end{remark}
\end{definition}

\begin{definition}[Related-Key Attack]\label{def:Attack-Related_Key}
  In a \emph{related-key attack}, the message $M$ is encrypted with a key $K$ similar to another, already broken key.
\end{definition}

\begin{definition}[Side Channel Attack]\label{def:Attack-Side_Channel}
  In a \emph{side channel attack}, Eve is attacking the implementation of the \nameref{def:Encryption_Rule}, $e_{K}$, by observing some other outputs.
  \begin{itemize}[noitemsep]
  \item The power consumed by the algorithm
  \item The network communcations occurring
  \end{itemize}
\end{definition}

\subsection{Security Scenarios}\label{subsec:Shannon_Security_Scenarios}
There are several scenarios we have for our \nameref{def:Encryption_Rule} and \nameref{def:Decryption_Rule}.
In the order of \textbf{strongest security} to \textbf{weakest security}:
\begin{enumerate}[noitemsep]
\item \emph{\nameref{def:Security-Unconditional}}: A \nameref{def:Cryptographic_Primitive} is unconditionally secure if it cannot be broken even if Eve has infinite computational power.
  \begin{itemize}[noitemsep]
  \item This means she can also perform an exhautive key search (Brute Force)
  \end{itemize}
\item \emph{\nameref{def:Security-Computational}}: A \nameref{def:Cryptographic_Primitive} is computationally secure if the best algorithm to break the key requires at least $T$ operations, where $T$ is a very large number.
\item \emph{\nameref{def:Security-Provable}}: A \nameref{def:Cryptographic_Primitive} is provably secure if the key can be reduced to a well known and well-studied problem.
  \begin{itemize}[noitemsep]
  \item Semiprime Integer Factorization is an example a key that is provably secure.
  \end{itemize}
\item \emph{\nameref{def:Security-Heuristic}}: If there is no known method of breaking the \nameref{def:Cryptographic_Primitive}, but the security cannot be proven in any sense.
\end{enumerate}

\begin{definition}[Unconditional Security]\label{def:Security-Unconditional}
  A \nameref{def:Cryptographic_Primitive} has \emph{unconditional security}, is \emph{unconditionally secure}, if it cannot be broken even if Eve has infinite computational power.
  \begin{itemize}[noitemsep]
  \item This means she can also perform an exhautive key search (Brute Force)
  \end{itemize}
\end{definition}

\begin{definition}[Computational Security]\label{def:Security-Computational}
  A \nameref{def:Cryptographic_Primitive} has \emph{computational security}, is \emph{computationally secure}, if the best algorithm to break the key requires at least $T$ operations, where $T$ is a very large number.
\end{definition}

\begin{definition}[Provable Security]\label{def:Security-Provable}
  A \nameref{def:Cryptographic_Primitive} has \emph{provable security}, is \emph{provably secure}, if the key can be reduced to a well known and well-studied problem.
  \begin{itemize}[noitemsep]
  \item Semiprime Integer Factorization is an example a key that is provably secure.
  \end{itemize}
\end{definition}

\begin{definition}[Heuristic Security]\label{def:Security-Heuristic}
  A \nameref{def:Cryptographic_Primitive} has \emph{heuristic security}, is \emph{heuristically secure}, if there is no known method of breaking the \nameref{def:Cryptographic_Primitive}, but the security cannot be proven in any sense.
\end{definition}

\begin{definition}[Perfect Secrecy]\label{def:Perfect_Secrecy}
  A cryptosystem has \emph{perfect secrecy} if \Cref{eq:Perfect_Secrecy} is true.
  \begin{equation}\label{eq:Perfect_Secrecy}
    \MutualInformation(M;C) = 0 = H(M) - H(M \Given C)
  \end{equation}
\end{definition}

\subsection{Shannon's Theory with Discrete Random Variables}\label{subsec:Shannon_Theory_Discrete_Random_Variables}
\nameref{def:Shannon_Theory_of_Secrecy} can be modelled with 3 random variables:
\begin{equation}\label{eq:Shannon_Theory_Discrete_Random_Variables}
  \begin{aligned}
    \mathbf{M} &= (M_{1}, M_{2}, \ldots, M_{N}) \text{ and } \Prob(\mathbf{M}) \\
    \mathbf{C} &= (C_{1}, C_{2}, \ldots, C_{N}) \text{ and } \Prob(\mathbf{C}) \\
    K &\in \Keyspace \text{ and } \Prob(K) \\
  \end{aligned}
\end{equation}

\begin{definition}[Key \nameref*{def:Entropy}]\label{def:Key_Entropy}
  The \emph{key \nameref{def:Entropy}} is the uncertainty Eve faces regarding an unknown key that she has no experience with.
  \begin{equation}\label{eq:Key_Entropy}
    \Entropy(K) = - \sum\limits_{k \in \Keyspace} \Prob(k) \log_{2} \bigl( \Prob(k) \bigr)
  \end{equation}

  \begin{remark}[Key Entropy Upper Bound]\label{rmk:Key_Entropy_Upper_Bound}
    \begin{equation}\label{eq:Key_Entropy_Upper_Bound}
      \Entropy(K) \leq \log_{2} \SetOrder{\Keyspace}
    \end{equation}
  \end{remark}
\end{definition}

\begin{definition}[Message \nameref*{def:Entropy}]\label{def:Message_Entropy}
  The \emph{message \nameref{def:Entropy}} is the uncertainty regarding the transmitted message.
  \begin{equation}\label{eq:Message_Entropy}
    \Entropy(\mathbf{M}) = - \sum\limits_{\mathbf{m} \in \Plaintexts^{N}} \Prob(\mathbf{m}) \log_{2} \bigl( \Prob(\mathbf{m}) \bigr)
  \end{equation}

  \begin{remark}[Message Entropy Upper Bound]\label{rmk:Message_Entropy_Upper_Bound}
    \begin{equation}\label{eq:Message_Entropy_Upper_Bound}
      \Entropy(\mathbf{M}) \leq \log_{2} \SetOrder{\Plaintexts}^{N}
    \end{equation}
  \end{remark}
\end{definition}

\begin{definition}[Key Equivocation]\label{def:Key_Equivocation}
  \emph{Key equivocation} is the case when Eve uses her knowledge of the \nameref{def:Ciphertext} to help break the key.
  \begin{equation}\label{eq:Key_Equivocation}
    \Entropy(K \Given \mathbf{C}) = - \sum\limits_{k \in \Keyspace, \mathbf{c} \in \Ciphertexts^{N}} \Prob(k, \mathbf{c}) \log_{2} \bigl( \Prob(k \Given \mathbf{c}) \bigr)
  \end{equation}

  \begin{remark}[\nameref*{def:Key_Equivocation} Upper Bound]\label{def:Key_Equivocation_Upper_Bound}
    The uncertainty of the key can never increase by observing $\mathbf{C}$.
    Otherwise, why even encrypt the \nameref{def:Plaintext}?
    \begin{equation}\label{eq:Key_Equivocation_Upper_Bound}
      \Entropy(K \Given \mathbf{C}) \leq \Entropy(K)
    \end{equation}
  \end{remark}
\end{definition}

\begin{definition}[Message Equivocation]\label{def:Message_Equivocation}
  \emph{Message equivocation} is the case when Eve uses her knowledge of the \nameref{def:Ciphertext} to help break the \nameref{def:Plaintext}.
  \begin{equation}\label{eq:Message_Equivocation}
    \Entropy(\mathbf{M} \Given \mathbf{C}) = - \sum\limits_{\mathbf{m} \in \Plaintexts^{N}, \mathbf{c} \in \Ciphertexts^{N}} \Prob(\mathbf{m}, \mathbf{c}) \log_{2} \bigl( \Prob(\mathbf{m} \Given \mathbf{c}) \bigr)
  \end{equation}

  \begin{remark}[\nameref*{def:Message_Equivocation} Upper Bound]\label{def:Message_Equivocation_Upper_Bound}
    The uncertainty of the message can never increase by observing $\mathbf{C}$.
    Otherwise, why even encrypt the \nameref{def:Plaintext}?
    \begin{equation}\label{eq:Message_Equivocation_Upper_Bound}
      \Entropy(\mathbf{M} \Given \mathbf{C}) \leq \Entropy(\mathbf{M})
    \end{equation}
  \end{remark}
\end{definition}

\begin{theorem}[\nameref*{def:Message_Equivocation} Bounded by \nameref*{def:Key_Equivocation}]\label{thm:Message_Equivocation_Bounded_Key_Equivocation}
  For an encryption scheme, we have
  \begin{equation}\label{eq:Message_Equivocation_Bounded_Key_Equivocation}
    \Entropy(\mathbf{M} \Given \mathbf{C}) \leq \Entropy(K \Given \mathbf{C})
  \end{equation}
\end{theorem}

\begin{proof}[\nameref*{def:Message_Equivocation} Bounded by \nameref*{def:Key_Equivocation}]\label{proof:Message_Equivocation_Bounded_Key_Equivocation}
  The expression $\Entropy(K, \mathbf{M} \Given \mathbf{C})$ can be written
  \begin{equation*}
    \Entropy(K, \mathbf{M} \Given \mathbf{C}) = \Entropy(K \Given \mathbf{C}) + \Entropy(\mathbf{M} \Given \mathbf{C}, K)
  \end{equation*}

  When the key and ciphertext are given, the plaintext can be uniquely determined, so
  \begin{equation*}
    \Entropy(\mathbf{M} \Given \mathbf{C}, K) = 0
  \end{equation*}

  Since $Entropy(\mathbf{M} \Given \mathbf{C}) \leq \Entropy(K, \mathbf{M} \Given \mathbf{C})$, we get
  \begin{equation*}
    \Entropy(\mathbf{M} \Given \mathbf{C}) \leq \Entropy(K \Given \mathbf{C})
  \end{equation*}
\end{proof}

\begin{definition}[Alphabet Size]\label{def:Alphabet_Size}
  The \emph{alphabet size}, denoted $L$, is
  \begin{equation}\label{eq:Alphabet_Size}
    \SetOrder{\Plaintexts} = \SetOrder{\Ciphertexts} = L
  \end{equation}
\end{definition}

\begin{definition}[Rate of the Alphabet]\label{def:Rate_of_Alphabet}
  The \emph{rate of the alphabet} is the \textbf{maximum \nameref{def:Entropy}} possible.
  \begin{equation}\label{eq:Rate_of_Alphabet}
    H_{0} = \log_{2} (L)
  \end{equation}
\end{definition}

\begin{definition}[\nameref*{def:Entropy} Per Alphabet Symbol]\label{def:Entropy_per_Alphabet_Symbol}
  The actual \nameref{def:Entropy} of the message source \textbf{per alphabet symbol} is denoted $H(M)$ and is defined as
  \begin{equation}\label{eq:Entropy_per_Alphabet_Symbol}
    \Entropy(M) = \frac{\Entropy(\mathbf{M})}{N}
  \end{equation}
  for a message $\mathbf{M} = (M_{1}, M_{2}, \ldots, M_{N})$ of length $N$.

  \begin{remark}[Cases of \nameref*{def:Entropy_per_Alphabet_Symbol}]\label{rmk:Entropy_per_Alphabet_Symbol-Cases}
    There are 2 cases:
    \begin{enumerate}[noitemsep]
    \item A memoryless source
    \item A non-memoryless source
    \end{enumerate}
  \end{remark}
\end{definition}

\begin{definition}[Redundancy]\label{def:Redundancy}
  The \emph{redundancy} of a source, denoted $D$ is
  \begin{equation}\label{eq:Redundancy}
    D = H_{0} - \Entropy(M)
  \end{equation}

  A higher redundancy means there is more regularity in the message, which makes it easier to break.

  \begin{remark}[Benefits for Eve]\label{rmk:Redundancy_Benefits_Eve}
    \nameref{def:Redundancy} helps Eve!
    There are 2 ways to reduce redundancy in digital systems:
    \begin{enumerate}[noitemsep]
    \item Compress the message before encryption, because regularities in the message are ``removed'' by compression.
    \item Channel coding (the addition of parity bits for correct/detection of errors) should be done after encryption, because their regularity before will increase redundancy, hurting the key's strength.
    \end{enumerate}
  \end{remark}
\end{definition}

\begin{theorem}[Shannon's Theory of Secrecy]\label{thm:Shannon_Theory_of_Secrecy}
  \begin{equation}\label{eq:Shannon_Theory_of_Secrecy}
    \Entropy(K \Given \mathbf{C}) \geq \Entropy(K) - ND
  \end{equation}
\end{theorem}

\begin{proof}[Shannon's Theory of Secrecy]\label{proof:Shannon_Theory_of_Secrecy}
  Since $\Entropy(K, \mathbf{M}, \mathbf{C}) = \Entropy(K, \mathbf{M}) = \Entropy(K, \mathbf{C})$ (assuming unique encryption/decryption, and nonprobabilistic encryption).

  Starting with $\Entropy(K, \mathbf{M})$, we have the relation
  \begin{equation*}
    \Entropy(K, \mathbf{M}) = \Entropy(K) + \Entropy(\mathbf{M})
  \end{equation*}
  which can be simplified to
  \begin{equation*}
    \Entropy(K, \mathbf{M}) = \Entropy(K) + N \Entropy(M)
  \end{equation*}

  On the other side, $\Entropy(K, \mathbf{C})$, we have the relation
  \begin{equation*}
    \Entropy(K, \mathbf{C}) = \Entropy(\mathbf{C}) + \Entropy(K \Given \mathbf{C}) \leq N H_{0} + \Entropy(K \Given \mathbf{C})
  \end{equation*}

  Putting these 2 together gives us
  \begin{equation*}
    \Entropy(K \Given \mathbf{C}) \geq N \Entropy(M) + \Entropy(K) - NH_{0} = \Entropy(K) - ND
  \end{equation*}
\end{proof}

\begin{definition}[Unicity Distance]\label{def:Unicity_Distance}
  The \emph{unicity distance}, denoted $N_{0}$ is defined as
  \begin{equation}\label{eq:Unicity_Distance}
    N_{0} = \frac{\Entropy(K)}{D}
  \end{equation}

  The unicity distance, practically, gives the number of symbols in the \nameref{def:Ciphertext} we need to observe to go from having multiple possible keys to a single possible key.
\end{definition}
%%% Local Variables:
%%% mode: latex
%%% TeX-master: "../EDIN01-Cryptography-Reference_Sheet"
%%% End:
