\section{Stream Ciphers --- LFSR Sequences}\label{sec:Stream_Ciphers}
There are 2 main types of \nameref{def:Symmetric_Encryption} algorithms:
\begin{enumerate}[noitemsep]
\item \nameref{def:Block_Cipher}s
\item \nameref{def:Stream_Cipher}s
\end{enumerate}

\begin{definition}[Block Cipher]\label{def:Block_Cipher}
  \emph{Block cipher}s encrypt a block of symbols of a fixed size from the \nameref{def:Plaintext} message using an fixed-size encryption transformation.
\end{definition}

\begin{definition}[Stream Cipher]\label{def:Stream_Cipher}
  \emph{Stream cipher}s encrypt individual characters of the \nameref{def:Plaintext} using an encryption transofrmation that varies with time.
\end{definition}

\begin{definition}[Keystream]\label{def:Keystream}
  The \emph{keystream} is the output of the generator, which gets used on the \nameref{def:Plaintext} message to form the \nameref{def:Ciphertext}
  \begin{equation}\label{eq:Keystream}
    \mathbf{z} = z_{1}, z_{2}, \ldots
  \end{equation}

  \begin{remark}[Attacks]\label{rmk:Keystream_Attack_Vectors}
    For a synchronous \nameref{def:Stream_Cipher}, a \nameref{def:Attack-Known_Plaintext}, \nameref{def:Attack-Chosen_Plaintext}, or \nameref{def:Attack-Chosen_Ciphertext} is equivalent to having access to the \nameref{def:Keystream}.
  \end{remark}

  \begin{remark}[Secret Key]\label{rmk:Keystream_Secret_Key}
    When the \nameref{def:Keystream} first starts, because it is a synchronous system, it needs to have a starting value with which to generate subsequent values.
    This first value may be fixed, and is the secret key, $K$, in this case.
  \end{remark}
\end{definition}

\subsection{Assumptions}\label{subsec:Stream_Cipher_Assumptions}
We assume that
\begin{itemize}[noitemsep]
\item The output \nameref{def:Keystream} of length $N$ is known to Eve.
\end{itemize}

\subsection{Attacks}\label{subsec:Stream_Cipher_Attacks}
\begin{itemize}[noitemsep]
\item \emph{\nameref{def:Attack-Key_Recovery}}: Eve tries to recover the \nameref{rmk:Keystream_Secret_Key}, $K$.
\item \emph{\nameref{def:Attack-Distinguishing}}: Eve tries to determine whether a given sequence $\hat{\mathbf{z}} = z_{1}, z_{2}, \ldots, z_{N}$ is likely to have been generated from the considered \nameref{def:Stream_Cipher}, or if it's a truly random sequence.
\end{itemize}

\begin{definition}[Key Recovery Attack]\label{def:Attack-Key_Recovery}
  In a \emph{key recovery attack}, Eve tries to recover the \nameref{rmk:Keystream_Secret_Key}, $K$.
\end{definition}

\begin{definition}[Distinguishing Attack]\label{def:Attack-Distinguishing}
  In a \emph{distinguishing attack}, Eve tries to determine whether a given sequence $\hat{\mathbf{z}} = z_{1}, z_{2}, \ldots, z_{N}$ is likely to have been generated from the considered \nameref{def:Stream_Cipher}, or if it's a truly random sequence.

  \begin{remark}
    A \nameref{def:Attack-Distinguishing} is a much weaker attack than a \nameref{def:Attack-Key_Recovery}.
  \end{remark}
\end{definition}

%%% Local Variables:
%%% mode: latex
%%% TeX-master: "../EDIN01-Cryptography-Reference_Sheet"
%%% End:
