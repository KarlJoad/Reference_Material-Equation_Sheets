\section{Abstract Algebra}\label{sec:Abstract_Algebra}
In the beginning of this course, we covered some basic abstract algebra.
These include:
\begin{itemize}[noitemsep]
\item Aspects covering integers and calculus modulo $n$
\item Covering a few examples of algebraic structures
\item Some basic concepts from abstract algebra, which provide a more general treatement of algebraic structures
\item In cryptography, we are generally interested in \emph{finite} sets
\end{itemize}

\subsection{Groups}\label{subsec:Groups}
\begin{definition}[Binary Operation]\label{def:Binary_Operation}
  A \emph{binary operation}, denoted $\BinaryOperation$ on a set $S$, is a mapping from $S \BinaryOperation S$ to $S$.
  \begin{remark}
    Note that $\BinaryOperation$ is \textbf{NOT} multiplication nor any kind of convolution.
    It is just a placeholder for some \nameref{def:Binary_Operation} that can be done.
  \end{remark}
\end{definition}

\begin{definition}[Group]\label{def:Group}
  A \emph{group} is a set $G$ and a \nameref{def:Binary_Operation}, $*$, on $G$ denoted as $(G, *)$, which satisfies the following properties:
  \begin{propertylist}
  \item $a \BinaryOperation (b \BinaryOperation c) = (a \BinaryOperation b) \BinaryOperation c$ for all $a, b, c, \in G$ (Associativity).\label{prop:Group_Properties-Associativity}.
  \item There is a special element $1 \in G$ such that $a \BinaryOperation 1 = 1 \BinaryOperation a = a$ for all $a \in G$ (Identity).\label{prop:Group_Properties-Identity}
    \begin{equation*}
      \exists 1 \in G \forall a \in G \, \left( a \BinaryOperation 1 = 1 \BinaryOperation a = a \right)
    \end{equation*}

  \item For each $a \in G$ there is an element $a^{-1} \in G$ such that $a \BinaryOperation a^{-1} = a^{-1} \BinaryOperation a = 1$  (Inverse).\label{prop:Group_Properties-Inverse}
  \end{propertylist}

  We call 1 the \emph{identity element} as defined in \Cref{prop:Group_Properties-Identity}, and call $a^{-1}$ the \emph{inverse} of $a$.
  Furthermore,
  \begin{propertylist}[resume]
  \item If $a \BinaryOperation b = b \BinaryOperation a$ for all $a,b \in G$ (\nameref{def:Abelian}/Commutativity).\label{prop:Group_Properties-Commutativity}
    \begin{equation*}
      \forall a,b \in G \, \left( a \BinaryOperation b = b \BinaryOperation a \right)
    \end{equation*}
  \end{propertylist}
\end{definition}

\begin{example}[Exercise 1, Problem 1.8a]{Show Set Is Group}
  Let $S$ be the set of binary triples, $S = \lbrace \left( s_{0}, s_{1}, s_{2} \right) \vert s_{i} \in \IntsMod{2} \rbrace$.
  Let the operation be bitwise addition.
  \tcblower{}
  \begin{remark*}
    In this case, ``bitwise addition'' means element-wise addition, i.e.\ each element of each triple gets added together.
    Additionally, all additions are done in modulo 2.
  \end{remark*}
  I will define the bitwise addition \nameref{def:Binary_Operation} with the symbol $.+$, like how MATLAB or GNU Octave define it.
  We will need to prove that \Crefrange{prop:Group_Properties-Associativity}{prop:Group_Properties-Identity} are true.
  \Cref{prop:Group_Properties-Commutativity} is \emph{not} necessary to show that a set is a \nameref{def:Group}.
  \Cref{prop:Group_Properties-Commutativity} only shows that the \nameref{def:Group} is \nameref{def:Abelian}

  First, the elements present in the set $S$:
  \begin{equation*}
    S = \Bigl\lbrace (0, 0, 0), (0, 0, 1), (0, 1, 0), (0, 1, 1), (1, 0, 0), (1, 0, 1), (1, 1, 0), (1, 1, 1) \Bigr\rbrace
  \end{equation*}

  To prove that $S$ is a group with an element-wise addition \nameref{def:Binary_Operation}, I will arbitrarily select
  \begin{align*}
    a &= (0, 1, 0) \\
    b &= (0, 1, 1) \\
    c &= (1, 1, 0)
  \end{align*}

  Starting with \Cref{prop:Group_Properties-Associativity}:
  \begin{align*}
    a \, .+ (b \, .+ c) &= (a \, .+ b) \, .+ c \\
    (0, 1, 0) \, .+ \bigl( (0, 1, 1) \, .+ (1, 1, 0) \bigr) &= \bigl( (0, 1, 0) \, .+ (0, 1, 1) \bigr) \, .+ (1, 1, 0) \\
    (0, 1, 0) \, .+ \bigl( (1, 0, 1) \bigr) &= \bigl( (0, 0, 1) \bigr) \, .+ (1, 1, 0) \\
    (1, 1, 1)  &= (1, 1, 1) \\
  \end{align*}
  Thus, \Cref{prop:Group_Properties-Associativity} is satisfied.

  Now, \Cref{prop:Group_Properties-Identity}
  \begin{align*}
    a \, .+ 1 &= a \\
    (0, 1, 0) \, .+ 1 &= (0, 1, 0) \\
    1 &= (0, 1, 0) \, .+ \Bigl( -(0, 1, 0) \Bigr)
  \end{align*}
  It is important to remamber that these additions are done in the modulo 2 domain, so addition and subtraction yield the same results, making negatives irrelevant.
  \begin{align*}
    1 &= (0, 1, 0) \, .+ (0, 1, 0) \\
    1 &= (0, 0, 0)
  \end{align*}
  Thus, \Cref{prop:Group_Properties-Identity} is satisfied, and $1 = (0, 0, 0)$.

  Lastly, we need to prove \Cref{prop:Group_Properties-Inverse}.
  \begin{align*}
    a \, .+ a^{-1} &= 1 \\
    (0, 1, 0) \, .+ a^{-1} &= (0, 0, 0) \\
    a^{-1} &= (0, 0, 0) \, .+ \Bigl( -(0, 1, 0) \Bigr)
  \end{align*}
  Like before, addition and subtraction are irrelevant here.
  \begin{align*}
    a^{-1} &= (0, 0, 0) \, .+ (0, 1, 0) \\
    a^{-1} &= (0, 1, 0)
  \end{align*}
  Thus, \Cref{prop:Group_Properties-Inverse} is satisfied, where each element $a$ is its own inverse.

  Since \Crefrange{prop:Group_Properties-Associativity}{prop:Group_Properties-Inverse} are satisfied, $S$ \textbf{is} a \nameref{def:Group} with the bitwise addition \nameref{def:Binary_Operation}.
\end{example}

\begin{definition}[Abelian]\label{def:Abelian}
  An \emph{abelian} group, also called a \emph{commutative \nameref{def:Group}}, is a group in which the result of applying the group operation to two group elements does not depend on the order in which they are written.
  That is, these are the groups that obey the axiom of commutativity.
\end{definition}

\subsubsection{Examples of \nameref*{subsec:Groups}}\label{subsubsec:Examples_of_Groups}
\begin{itemize}[noitemsep]
\item \TextAllIntegers{} with the addition \nameref{def:Binary_Operation}, denoted $(\AllIntegers, +)$ is a \nameref{def:Group}.
\item Finite \nameref{def:Group}s are $(\IntsModN{}, +)$ and $(\MultiplicativeGroupN{}, \cdot)$, where $\cdot$ denotes multiplication modulo $n$
\item \textbf{NOTE:} $(\IntsModN{}, \cdot)$ is \textbf{NOT} a \nameref{def:Group}, nor is $(\AllIntegers, \cdot)$.
\end{itemize}

\subsubsection{Definitions for \nameref*{subsec:Groups}}\label{subsubsec:Definitions_for_Groups}
\begin{definition}[Subgroup]\label{def:Subgroup}
  A non-empty subset $H$ of a \nameref{def:Group} $G$ is called a \emph{subgroup} of $G$, if $H$ itself is a \nameref{def:Group} with respect to the operation of $G$.
\end{definition}

\begin{example}[Exercise 1, Problem 1.10]{Find Subgroups}
  Find all \nameref{def:Subgroup}s in the multiplicative group \TextMultiplicativeGroup{19} (under the multiplication \nameref{def:Binary_Operation})?
  \tcblower{}
  The goal here is to find a \nameref{def:Generator} that can be raised to some set of exponents that can generate \nameref{def:Subgroup}s of \TextMultiplicativeGroup{19}.

  The first thing to note is that 19 is a \nameref{def:Prime}, so
  \begin{equation*}
    \SetOrder{\MultiplicativeGroup{19}} = 18 = \phi(n)
  \end{equation*}

  We can also apply \nameref{thm:Lagranges_Theorem} to restrict the number of \nameref{def:Subgroup}s possible.
  Because \TextMultiplicativeGroup{19} is a finite \nameref{def:Group}, and we are considering $G$ to be a \nameref{def:Subgroup} of \TextMultiplicativeGroup{19}, we apply \nameref{thm:Lagranges_Theorem}.
  This means $\SetOrder{G} \Divides \SetOrder{\MultiplicativeGroup{19}}$ must be true.
  Thus,
  \begin{equation*}
    \SetOrder{G} = \lbrace 1, 2, 3, 6, 9, 18 \rbrace
  \end{equation*}

  Now, we can find a \nameref{def:Generator} for the \nameref{def:Group} \TextMultiplicativeGroup{19}.
  This means, we need to find some element $a$ that raised to some power $t$, modulo 19 will yield 1.
  We know that $t$ \textbf{must} equal 18, because $\phi(n) = 18$, and we want to find a \nameref{def:Generator}.
  So, we solve for $a$.
  \begin{equation*}
    a^{18} \bmod 18 = 1
  \end{equation*}
  In this case, $a = 2$, making $2$ a \nameref{def:Generator} of \TextMultiplicativeGroup{19}, or \TextCyclicSubgroup{2}.

  Now, all possible \nameref{def:Subgroup}s of \TextMultiplicativeGroup{19} are:
  \begin{align*}
    G_{1} &= \biggl \lbrace \Bigl[ 2^{18} \Bigr] \biggr \rbrace \\
    G_{2} &= \biggl \lbrace \Bigl[ 2^{9} \Bigr], \Bigl[ 2^{18}\Bigr] \biggr \rbrace \\
    G_{3} &= \biggl \lbrace \Bigl[ 2^{6} \Bigr], \Bigl[ 2^{12} \Bigr], \Bigl[ 2^{18} \Bigr] \biggr \rbrace \\
    G_{6} &= \biggl \lbrace \Bigl[ 2^{3} \Bigr], \Bigl[ 2^{6} \Bigr], \Bigl[ 2^{9} \Bigr], \Bigl[ 2^{12} \Bigr], \Bigl[ 2^{15} \Bigr], \Bigl[ 2^{18} \Bigr] \biggr \rbrace \\
    G_{9} &= \biggl \lbrace \Bigl[ 2^{2} \Bigr], \Bigl[ 2^{4} \Bigr], \Bigl[ 2^{6} \Bigr], \Bigl[ 2^{8} \Bigr], \Bigl[ 2^{10} \Bigr], \Bigl[ 2^{12} \Bigr], \Bigl[ 2^{14} \Bigr], \Bigl[ 2^{16} \Bigr], \Bigl[ 2^{18} \Bigr] \biggr \rbrace \\
    G_{18} &= \biggl \lbrace \Bigl[ 2^{0} \Bigr], \Bigl[ 2^{1} \Bigr], \ldots, \Bigl[ 2^{18} \Bigr] \biggr \rbrace \\
  \end{align*}
\end{example}

\begin{definition}[Cyclic]\label{def:Cyclic}
  $G$ is \emph{cyclic} if there is an element $a \in G$ such that each $b \in G$ can be written as $a^{i}$ for some integer $i$.
  The element $a$ is called a \emph{\nameref{def:Generator}} of $G$.
\end{definition}

\begin{example}[]{Cyclic}
  Find $a$, the \nameref{def:Cyclic} term in the \nameref{def:Group} \TextIntsMod{15}?
  \tcblower{}
  \begin{equation*}
    \begin{aligned}
      \IntsMod{15} &= \lbrace 3, 6, 9, 12, 0, \ldots \rbrace \\
      &= \lbrace 3, 3+3, 3+3+3, 3+3+3+3, 5 \cdot 3, \ldots \rbrace \\
    \end{aligned}
  \end{equation*}

  Thus, $\CyclicSubgroup{a} = 3 \rightarrow \CyclicSubgroup{3}$.
\end{example}

\begin{definition}[Element Order]\label{def:Element_in_Group_Order}
  The \emph{order of an element} $a$, denoted $\ElementOrder(a)$ is defined as the least positive integer $t$ ($t \in \AllIntegers$) such that
  \begin{equation}\label{eq:Element_in_Group_Order}
    a^{t} = 1
  \end{equation}
  If such an integer $t$ does not exist, then the order of $a$ is defined to be $\infty$.

  \begin{remark}\label{rmk:Element_in_Group_Order_Term_Explanation}
    \textbf{NOTE:}
    \begin{itemize}[noitemsep]
    \item The $a^{t}$ part does not mean exponentiation, but rather repeated uses of the \nameref{def:Binary_Operation} used to create the \nameref{def:Group}.
    \item The $1$ is not a value $1$, but the Identity of the \nameref{def:Group}, as defined by \Cref{prop:Group_Properties-Identity}.
    \end{itemize}
  \end{remark}
\end{definition}

\begin{example}[Exercise 1, Problem 1.8c]{Order of Element in Group}
  Given the element $(1, 1, 1,)$ from the group of bit triples, $S = \lbrace \left( s_{0}, s_{1}, s_{2} \right) \vert s_{i} \in \IntsMod{2} \rbrace$, using an bitwise addition \nameref{def:Binary_Operation}, what is the \nameref{def:Element_in_Group_Order} of $(1, 1, 1)$?
  The identity element of this \nameref{def:Group} is $1 = (0, 0, 0)$.
  \tcblower{}
  It is important to remember the note in \Cref{rmk:Element_in_Group_Order_Term_Explanation}, especially for this problem.

  Since the given \nameref{def:Binary_Operation} was ``bitwise'', i.e.\ element-wise, I will define the operation to be $.+\,$.
  In this case, the \nameref{def:Element_in_Group_Order} of any element in this \nameref{def:Group} will be the number of element-wise additions that must occur to get the identity element, $1$.

  For this problem, since we are working in the modulo 2 domain, we have some basic facts:
  \begin{align*}
    (0 + 0) \bmod 2 &= 0 \\
    (1 + 0) \bmod 2 &= 1 \\
    (0 + 1) \bmod 2 &= 1 \\
    (1 + 1) \bmod 2 &= 0
  \end{align*}
  And for subtraction:
  \begin{align*}
    (0 - 0) \bmod 2 &= 0 \\
    (1 - 0) \bmod 2 &= 1 \\
    (0 - 1) \bmod 2 &= -1 \bmod 2 = 1 \\
    (1 - 1) \bmod 2 &= 0
  \end{align*}

  We start by constructing the equation needed to solve this problem.
  \begin{equation*}
    (1, 1, 1) .+ a = (0, 0, 0)
  \end{equation*}
  Now we can move the $(1, 1, 1)$ term over to the left, and since we are working in the modulo 2 domain, the ``negative'' that would be introduced by normal subtraction (negative addition) is irrelevant.
  Thus, we end up with
  \begin{align*}
    a &= (0, 0, 0) .+ (1, 1, 1) \\
      &= (1, 1, 1)
  \end{align*}
  This means that if $t$ from \Cref{eq:Element_in_Group_Order} is 2, we get the identity element $1$.
  So, our answer is $t = 2$, thus $\ElementOrder \bigl[ (1, 1, 1) \bigr] = 2$.
\end{example}

\begin{lemma}
  Let $a \in G$ be an element of finite \nameref{def:Element_in_Group_Order} $t$.
  Then the set of all powers of $a$ forms a \nameref{def:Cyclic} \nameref{def:Subgroup} of $G$, denoted by \TextCyclicSubgroup{a}.
  Furthermore, the \nameref{def:Element_in_Group_Order} of \TextCyclicSubgroup{a} is $t$.
\end{lemma}

\begin{definition}[Left Coset]\label{def:Left_Coset}
  Let $H$ be a \nameref{def:Subgroup} in $G$ and pick an element $a \in G$.
  A set of elements of the form
  \begin{equation}\label{eq:Left_Coset}
    aH = \lbrace ah \vert h \in H \rbrace
  \end{equation}
  is called the \emph{left coset} of $H$.
  If $G$ is commutative, it is simply called a \emph{coset}.

  The set consisting of all such left cosets is written $G/H$.
  Note that $H$ itself is a left coset.
  Furthermore, every left coset contains the same number of elements (the order of $H$) and every element is contained in exactly one left coset.

  So, the elements of $G$ are partitioned into $\SetOrder{G} / \SetOrder{H}$ different cosets, each containing \TextSetOrder{H} elements.
\end{definition}

%%% Local Variables:
%%% mode: latex
%%% TeX-master: "../../EDIN01-Cryptography-Reference_Sheet"
%%% End:


\subsection{Properties of \nameref*{subsec:Groups}}\label{subsec:Properties_of_Groups}

%%% Local Variables:
%%% mode: latex
%%% TeX-master: "../../EDIN01-Cryptography-Reference_Sheet"
%%% End:


\subsection{Rings}\label{subsec:Rings}
\begin{definition}[Ring]\label{def:Ring}
  A \emph{ring} (with unity) consists of a set $R$ with two \nameref{def:Binary_Operation}s.
  A ring is denoted as $(R, +, *)$., where $+$ and $*$ are \textbf{not} addition and multiplication respectively, but placeholders for the two \nameref{def:Binary_Operation}s.
  A ring must also satisfy the following conditions:
  \begin{propertylist}
  \item $(R, +)$ is an \nameref{def:Abelian} \nameref{def:Group} with an identity element denoted 0.
  \item $a \cdot (b \cdot c) = (a \cdot b) \cdot c$ for all $a, b, c \in R$ (Associativity).\label{prop:Ring_Properties-Associativity}
    \begin{equation*}
      \forall a, b, c \in R \, a \cdot (b \cdot c) = (a \cdot b) \cdot c
    \end{equation*}

  \item There is a multiplicative identity, denoted 1, with the multiplicative identity not being equal to the identity element of the underlying \nameref{def:Abelian} \nameref{def:Group} ($1 \neq 0$), such that $1 \cdot a = a \cdot 1 = a$ for all $a \in R$ (Multiplicative Identity).\label{prop:Ring_Properties-Multiplicative_Identity}
    
  \item $a \cdot (b+c) = (a \cdot b) + (a \cdot c)$ and $(b+c) \cdot a = (b \cdot a) + (c \cdot a)$ for all $a, b, c \in R$ (Distributive).\label{prop:Ring_Properties-Distributivity}
    \begin{equation*}
      \begin{aligned}
        \forall a, b, c \in R \: a \cdot (b+c) &= (a \cdot b) + (a \cdot c) \\
        \forall a, b, c \in R \: (b+c) \cdot a &= (b \cdot a) + (c \cdot a) \\
      \end{aligned}
    \end{equation*}
  \end{propertylist}
  
  A \emph{commutative ring} is a ring where additionally
  \begin{propertylist}[resume]
  \item $a \cdot b = b \cdot a$ for all $a, b \in R$ (Commutativity)\label{prop:Ring_Properties-Commutativity}
    \begin{equation*}
      \forall a, b \in R \: a \cdot b = b \cdot a
    \end{equation*}
  \end{propertylist}

  \begin{remark}[Ring Multiplicative Inverses]\label{rmk:Ring_Multiplicative_Inverses}
    Note that we haven't said anything about multiplicative inverses yet.
  \end{remark}
\end{definition}

\begin{definition}[Invertible Element]\label{def:Invertible_Element}
  An element $a \in R$ is called an \emph{invertible element} or a \emph{unit} if there is an element $b \in R$ such that
  \begin{equation}\label{eq:Invertible_Element}
    a \cdot b = b \cdot a = 1
  \end{equation}

  \begin{remark}
    The set of \nameref{def:Invertible_Element}s in a \nameref{def:Ring} $R$ forms a \nameref{def:Group} under multiplication.
    For example, the \nameref{def:Group} of \nameref{def:Invertible_Element}s of the \nameref{def:Ring} \TextIntsModN{} is \TextMultiplicativeGroupN{}.
  \end{remark}
  
  \begin{remark}[Multiplicative Inverse]\label{rmk:Ring_Multiplicative_Inverse}
    The \nameref{def:Multiplicative_Inverse} of an element $a \in R$ is denoted by $a^{-1}$, assuming it exists.
    The division expression $a/b$ should then be interpreted as $a \cdot b^{-1}$.
  \end{remark}
\end{definition}

\subsubsection{Examples of \nameref*{subsec:Rings}}\label{subsubsec:Examples_of_Rings}
\begin{itemize}[noitemsep]
\item A commutative ring is $(\AllIntegers, +, \cdot)$, where $+$ and $\cdot$ are the susual operations of addition and multiplication.
\item Finite Ring: \TextIntsModN{} with addition and multiplication modulo $n$
  \begin{itemize}[noitemsep]
  \item The additive inverse of $a \in R$ is denoted $-a$. So the subtraction expression $a-b$ should be interpreted as $a + (-b)$.
  \item The multiplication of $a \cdot b$ is equivalently written $ab$.
  \item Similarly $a^{2} = aa = a \cdot a$.
  \end{itemize}
\end{itemize}

%%% Local Variables:
%%% mode: latex
%%% TeX-master: "../../EDIN01-Cryptography-Reference_Sheet"
%%% End:


\subsection{Fields}\label{subsec:Fields}
\begin{definition}[Field]\label{def:Field}
  A \emph{field} is a commutative \nameref{def:Ring} where all nonzero elements from the underlying set have \nameref{def:Multiplicative_Inverse}s.
\end{definition}

\begin{example}[Exercise 1, Problem 1.11]{Prove Set is Not Field}
  Prove that \TextIntsMod{4} is not a \nameref{def:Field}?
  \tcblower{}
  We begin by checking that all elements from the underlying set, \TextIntsMod{4}, have \nameref{def:Multiplicative_Inverse}s, for all the \textbf{\underline{non-zero}} elements.
  \begin{equation*}
    \IntsMod{4} = \Bigl \lbrace [0], [1], [2], [3] \Bigr \rbrace
  \end{equation*}

  First, we check $1$.
  \begin{equation*}
    1^{-1} = \gcd(1, 4) = 1a + 4b
  \end{equation*}
  \begin{align*}
    \gcd(1, 4) \Rightarrow 4 &= q1 + r \\
                             &= 4 \cdot 1 + 0 \\
    \gcd(1, 4) &= 1
  \end{align*}
  \begin{align*}
    1 &= 1 (-3) + 4 (1) \\
    a &= -3 \\
    b &= 1
  \end{align*}
  Since $a = -3$, we need to find the additive inverse:
  \begin{align*}
    -3 \bmod 4 &= (-3 + 4) \bmod 4 \\
               &= 1 \bmod 4 \\
               &= 1
  \end{align*}
  So, 1 has a \nameref{def:Multiplicative_Inverse}.

  Now we check 2.
  \begin{equation*}
    2^{-1} = \gcd(2, 4) = 2a + 4b
  \end{equation*}
  \begin{align*}
    \gcd(2, 4) \Rightarrow 4 &= q2 + r \\
                             &= 2 \cdot 2 + 0 \\
    \gcd(2, 4) &= 2
  \end{align*}
  Since $\gcd(2, 4) \neq 0$, 2 is \textbf{\underline{not}} invertible, meaning it has no \nameref{def:Multiplicative_Inverse}s.

  \TextIntsMod{4} is NOT a \nameref{def:Field} because 2 is \textbf{\underline{not}} invertible.
\end{example}

\begin{definition}[Characteristic]\label{def:Field_Characteristic}
  The \emph{characteristic} of a field is the smallest integer $m > 0$ such that
  \begin{equation}\label{eq:Field_Characteristic}
    \overbrace{1 + 1 + \cdots + 1}^{m} = 0
  \end{equation}

  If no such integer $m$ exists, the characteristic is defined to be 0.
\end{definition}

\begin{definition}[Finite Field]\label{def:Finite_Field}
  A \nameref{def:Field} is \emph{finite} only if \Cref{thm:Finite_Field} is satisfied.
\end{definition}

\begin{theorem}[Finite Field]\label{thm:Finite_Field}
  \TextIntsModN{} is a \nameref{def:Field} if and only if $n$ is a \nameref{def:Prime} number.
  If $n$ is a \nameref{def:Prime}, the \nameref{def:Field_Characteristic} of \TextIntsModN{} is $n$.
\end{theorem}

\begin{definition}[Subfield]\label{def:Subfield}
  A subset $F$ of a \nameref{def:Field} $E$ is called a \emph{subfield} of $E$ if $F$ is itself a \nameref{def:Field} with respect to the operations of $E$.

  \begin{remark}[Extension Field]\label{rmk:Extension_Field}
    Likewise, we say $E$ is an \emph{extension field} of $F$.
  \end{remark}
\end{definition}

\begin{definition}[Isomorphism]\label{def:Isomorphism}
  Two \nameref{def:Field}s are \emph{isomorphic} if they are structurally the same, but elements hvae a different representation.
  For example, for a \nameref{def:Prime}, $p$, \TextIntsMod{p} is a \nameref{def:Field} or \nameref{def:Set_Order} $p$.
  So we associate the \nameref{def:Finite_Field} \TextFiniteMathField{F}{p} with \TextIntsMod{p} and its representation.
\end{definition}

\subsubsection{Examples of \nameref*{subsec:Fields}}\label{subsubsec:Examples_of_Fields}
\begin{itemize}[noitemsep]
\item The rational Numbers: $\RationalNumbers$
\item The Real Numbers: $\RealNumbers$
\item The Complex Numbers: $\ComplexNumbers$
\item \nameref{def:Finite_Field}:
  \begin{itemize}[noitemsep]
  \item \TextIntsMod{p}, where $p$ is \nameref{def:Prime}
  \end{itemize}
\end{itemize}

%%% Local Variables:
%%% mode: latex
%%% TeX-master: "../../EDIN01-Cryptography-Reference_Sheet"
%%% End:


\subsection{Polynomial \nameref*{subsec:Rings}}\label{subsec:Polynomial_Rings}
\nameref{subsec:Polynomial_Rings} are getting their own subsection separate from other \nameref{subsec:Rings}, because they are actually a more general case of what we have learned already.
\begin{definition}[Polynomial]\label{def:Polynomial}
  A \emph{polynomial} in the indeterminate $x$ over the \nameref{def:Ring} $R$ is an expression of the form
  \begin{equation}\label{eq:Polynomial_Ring}
    f(x) = a_{n}x^{n} + \cdots + a_{2}x^{2} + a_{1}x + a_{0}
  \end{equation}
  where each $a_{i} \in R$, $a_{n} \neq 0$, and $n \geq 0$.

  \begin{remark}
    \textbf{Remember that even integers can be polynomials!}
    This means that everything we have learned about \nameref{subsec:Rings} already is actually a special case of \nameref{subsec:Polynomial_Rings}.
    \begin{equation*}
      1 = 0x^{m} + 0x^{m-1} + \cdots + 0x + 1
    \end{equation*}
  \end{remark}
\end{definition}

\begin{definition}[Degree]\label{def:Polynomial_Degree}
  We say that $f(x)$ has \emph{degree} $n$, denoted
  \begin{equation}\label{eq:Polynomial_Degree}
    \Degree f(x) = n
  \end{equation}

  \begin{remark}
    We also allow $f(x)$ to be the \nameref{def:Polynomial} with all coefficients being zero, in which case, the \nameref{def:Polynomial_Degree} is defined to be $-\infty$.
  \end{remark}
\end{definition}

\begin{definition}[Monic]\label{def:Monic_Polynomial}
  A \nameref{def:Polynomial} is said to be \emph{monic} if the leading coefficient is equal to 1.
  \begin{equation}\label{eq:Monic_Polynomial}
    a_{n} = 1
  \end{equation}
\end{definition}

\begin{definition}[Polynomial Ring]\label{def:Polynomial_Ring}
  Let $R$ be a commutative \nameref{def:Ring} (i.e. \Cref{prop:Ring_Properties-Commutativity} is fulfilled).
  Then the \emph{polynomial ring}, denoted by $R[x]$ is the ring formed by the set of all \nameref{def:Polynomial}s in the indeterminate $x$ having coefficients from $R$.
  The operations are addition and multiplication of \nameref{def:Polynomial}s, with the coefficient arithmetic performed in $R$.
\end{definition}

\begin{example}[Lecture 2]{Polynomial Ring}
  Find the \nameref{def:Polynomial_Ring} formed when the underlying \nameref{def:Ring} is \TextIntsMod{2}?
  \tcblower{}
  \begin{equation*}
    \IntsMod{2} \underset{R[x]}{\Longrightarrow} \IntsMod{2}[x] = \lbrace 0, 1, x, x+1, x^{2}, x^{2}+1, x^{2}+x, x^{2}+x, x^{2}+x+1, \ldots \rbrace
  \end{equation*}
\end{example}

If you consider the \TextMathField{F}{x}, where $F$ denotes an arbitrary \nameref{def:Field}.
\TextMathField{F}{x} has many properties in common with integers.

\begin{definition}[Irreducible]\label{def:Irreducible_Polynomial}
  A polynomial $f(x) \in \MathField{F}{x}$, of \nameref{def:Polynomial_Degree} $d \geq 1$ is \emph{irreducible} if $f(x)$ cannot be written as a product of 2 polynomials, $g(x), h(x) \in \MathField{F}{x}$, where the $\Degree g(x)$ and $\Degree h(x)$ are both less than $d$.

  \begin{remark}[Sum to Zero]\label{rmk:Irreducible_Polynomials_Sum_to_Zero}
    It seems that \nameref{def:Irreducible_Polynomial} \nameref{def:Polynomial}s, $f(x)$, where $f(x) \in \FiniteMathField{F}{p}{q}$ must sum to a non-zero value for all values of $x$ from the underlying set of integers, $\IntsMod{p}$.
  \end{remark}

  \begin{remark}[Relation Between Irreducible Polynomials and Prime Numbers]\label{rmk:Irreducible_Polynomials_and_Prime_Numbers}
    \nameref{def:Irreducible_Polynomial} polynomials are the \nameref{def:Polynomial_Ring} counterpart of \nameref{def:Prime} numbers.
  \end{remark}
\end{definition}

\begin{example}[Exercise 1, Problem 1.14a]{Irreducible Polynomial}
  Is $f(x) = x^{4} + x + 1$ in \TextIntsMod{2} an \nameref{def:Irreducible_Polynomial} \nameref{def:Polynomial} and/or a \nameref{def:Polynomial_Ring_Properties-Primitive_Element}?
  \tcblower{}
  We start by checking the \nameref{rmk:Irreducible_Polynomials_Sum_to_Zero} conditions.
  \begin{align*}
    f(0) = 0^{4} + 0 + 1 &= 1 \\
    f(1) = 1^{4} + 1 + 1 &= 1
  \end{align*}
  The \nameref{rmk:Irreducible_Polynomials_Sum_to_Zero} conditions for this \nameref{def:Polynomial}.

  Because our \nameref{def:Polynomial} is of \nameref{def:Polynomial_Degree} 4, we can factor it 2 ways:
  \begin{enumerate}[noitemsep]
  \item \begin{align*}
          \left( x^{2} + y \right) \left( x^{2} + z \right) &= x^{4} + yx^{2} + zx^{2} + yz \\
          x^{4} + yx^{2} + zx^{2} + yz &= x^{4} + x + 1
        \end{align*}
        This cannot work, because there are no $x^{2}$ terms in the potential factorization.
  \item \begin{align*}
          \left( x + y \right) \left( x^{3} + z \right) &= x^{4} + yx^{3} + zx + yz \\
          x^{4} + yx^{3} + zx + yz &= x^{4} + x + 1
        \end{align*}
        This also cannot work, because the coefficient's terms do not line up ($y \neq 0$ because $yz = 1$).
  \end{enumerate}
  Thus, $f(x)$ is an \nameref{def:Irreducible_Polynomial}.

  Now we need to check if $f(x)$ is a \nameref{def:Polynomial_Ring_Properties-Primitive_Element}.
  \begin{align*}
      \alpha^{0} &= 1 & \alpha^{8} &= \alpha^{4} + \alpha^{2} + \alpha = \alpha^{2} + \alpha + \alpha + 1 = \alpha^{2} + 1 \\
      \alpha^{1} &= \alpha & \alpha^{9} &= \alpha^{3} + \alpha \\
      \alpha^{2} &= \alpha^{2} & \alpha^{10} &= \alpha^{4} + \alpha^{2} = \alpha^{2} + \alpha + 1 \\
      \alpha^{3} &= \alpha^{3} & \alpha^{11} &= \alpha^{3} + \alpha^{2} + \alpha \\
      \alpha^{4} &= \alpha + 1 & \alpha^{12} &= \alpha^{4} + \alpha^{3} + \alpha^{2} = \alpha^{3} + \alpha^{2} + \alpha + 1 = \alpha^{3} + \alpha^{2} + \alpha + 1 \\
      \alpha^{5} &= \alpha^{2} + \alpha & \alpha^{13} &= \alpha^{4} + \alpha^{3} + \alpha^{2} + \alpha = \alpha + 1 + \alpha + 1 = \alpha^{3} + \alpha^{2} + 1 \\
      \alpha^{6} &= \alpha^{3} + \alpha^{2} & \alpha^{14} &= \alpha^{4} + \alpha^{3} + \alpha = \alpha + 1 + \alpha^{3} + \alpha = \alpha^{3} + 1 \\
      \alpha^{7} &= \alpha^{4} + \alpha^{3} = \alpha^{3} + \alpha + 1 & \alpha^{15} &= \alpha^{4} + \alpha = \alpha + 1 + \alpha = 1\\
  \end{align*}
  Because $\alpha^{4} = \alpha + 1$ can generate all \nameref{def:Polynomial}s in \TextFiniteMathField{F}{2}{4}, $f(x)$ is also \nameref{def:Polynomial_Ring_Properties-Primitive_Element}.

  Thus, the \nameref{def:Polynomial}, $f(x)$ is both \nameref{def:Irreducible_Polynomial} and a \nameref{def:Polynomial_Ring_Properties-Primitive_Element}.
\end{example}

\begin{example}[Exercise 1, Problem 1.14c]{Reducible Polynomial}
  Is $f(x) = x^{4} + x + 1$ in \TextIntsMod{2} an \nameref{def:Irreducible_Polynomial} \nameref{def:Polynomial} and/or a \nameref{def:Polynomial_Ring_Properties-Primitive_Element}?
  \tcblower{}
  Just by inspection, \nameref{rmk:Irreducible_Polynomials_Sum_to_Zero} holds up.

  Now we have to check if the \nameref{def:Polynomial} is reducible.
  I will arbitrarily choose to start with a factorization of
  \begin{align*}
    \left( x^{2} + c \right) \left( x^{2} + c \right) &= x^{4} + 2cx^{2} + c^{2} \\
    x^{4} + 2cx^{2} + c^{2} &= x^{4} + x^{2} + 1 \\
    x^{4} + c^{2} &= x^{4} + x^{2} + 1 \\
    c^{2} &= x^{2} + 0x + 1 \\
  \end{align*}
  Because of the modulo addition we are doing here, $2 \bmod 2 = 0$, eases the factorization.
  \begin{align*}
    c^{2} &= x^{2} + 2x + 1 \\
    c^{2} &= (x + 1) (x + 1) \\
    c &= x+1
  \end{align*}
  So, $f(x)$ is reducible.

  We can check if $f(x)$ is a \nameref{def:Irreducible_Polynomial}, though we don't have to.
  \begin{equation*}
    \alpha^{4} = \alpha^{2} + 1
  \end{equation*}
  \begin{align*}
    \alpha^{0} &= 1 \\
    \alpha^{1} &= \alpha \\
    \alpha^{2} &= \alpha^{2} \\
    \alpha^{3} &= \alpha^{3} \\
    \alpha^{4} &= \alpha^{2} + 1 \\
    \alpha^{5} &= \alpha^{3} + \alpha \\
    \alpha^{6} &= \alpha^{4} + \alpha^{2} + \alpha^{2} + 1 + \alpha^{2} = 1 \\
  \end{align*}

  Thus, $f(x)$ is reducible and NOT a \nameref{def:Polynomial_Ring_Properties-Primitive_Element}.
\end{example}

\begin{example}[Exercise 1, Problem 1.14d]{Reducible Polynomial By Sum}
  Is $f(x) = x^{4} + x + 1$ in \TextIntsMod{3} an \nameref{def:Irreducible_Polynomial} \nameref{def:Polynomial} and/or a \nameref{def:Polynomial_Ring_Properties-Primitive_Element}?
  \tcblower{}
  We can start by looking at \Cref{rmk:Irreducible_Polynomials_Sum_to_Zero}.
  \begin{align*}
    f(x=0) = 0^{4} + 0 + 1 &= 1 \bmod 3 = 1 \\
    f(x=1) = 1^{4} + 1 + 1 &= 3 \bmod 3 = 0 \\
    f(x=2) = 2^{4} + 2 + 1 &= 19 \bmod 3 = 1
  \end{align*}
  Since there is a case where $f(x) = 0$ in the set \TextIntsMod{3}, $f(x)$ is reducible.
  We can make the table of $\alpha$:
  \begin{equation*}
    \alpha^{4} = \alpha + 1
  \end{equation*}
  \begin{align*}
    \alpha^{0} &= 1 & \alpha^{8} &= \alpha^{4} + \alpha^{2} + \alpha = \alpha + 1 + \alpha^{2} + \alpha = \alpha^{2} + 1 \\
    \alpha^{1} &= \alpha & \alpha^{9} &= \alpha^{3} + \alpha \\
    \alpha^{2} &= \alpha^{2} & \alpha^{10} &= \alpha^{4} + \alpha^{2} = \alpha^{2} + \alpha + 1\\
    \alpha^{3} &= \alpha^{3} & \alpha^{11} &= \alpha^{3} + \alpha^{2} + \alpha \\
    \alpha^{4} &= \alpha + 1 & \alpha^{12} &= \alpha^{4} + \alpha^{3} + \alpha^{2} = \alpha^{3} + \alpha^{2} + \alpha + 1 \\
    \alpha^{5} &= \alpha^{2} + \alpha & \alpha^{13} &= \alpha^{4} + \alpha^{3} + \alpha^{2} + \alpha = \alpha + \alpha + \alpha^{3} + \alpha^{2} + 1 = \alpha^{3} + \alpha^{2} + 1 \\
    \alpha^{6} &= \alpha^{3} + \alpha^{2} & \alpha^{14} &= \alpha^{4} + \alpha^{3} + \alpha = \alpha + 1 + \alpha^{3} + \alpha = \alpha^{3} + 1 \\
    \alpha^{7} &= \alpha^{4} + \alpha^{3} = \alpha^{3} + \alpha + 1 & \alpha^{15} &= \alpha^{4} + \alpha = \alpha + 1 + \alpha = 1 \\
  \end{align*}

  Thus, $f(x)$ is reducible and MAY be a \nameref{def:Polynomial_Ring_Properties-Primitive_Element}.
\end{example}

\subsubsection{Long Division of \nameref*{def:Polynomial}s}\label{subsubsec:Polynomial_Long_Division}
Similarly as for integers (\Cref{subsec:Integer_Long_Division}), we have a division algorithm for polynomials.
\begin{definition}[Polynomial Long Division]\label{def:Polynomial_Long_Division}
  If $a(x), b(x) \in \MathField{F}{x}$, with $b(x) \neq 0$, then there are polynomials $q(x), r(x) \in \MathField{F}{x}$ such that
  \begin{equation}\label{eq:Polynomials_Long_Division}
    a(x) = q(x) b(x) + r(x)
  \end{equation}
  where
  \begin{itemize}[noitemsep]
  \item $\Degree r(x) < \Degree b(x)$.
  \item $q(x)$ and $r(x)$ are unique.
  \item $q(x)$ is referred to as the \nameref{def:Polynomial_Quotient}
  \item $r(x)$ is referred to as the \nameref{def:Polynomial_Remainder}
  \end{itemize}
\end{definition}

\begin{definition}[Polynomial Quotient]\label{def:Polynomial_Quotient}
  The \emph{\nameref{def:Polynomial} quotient}, $q(x)$, of $a(x)$ divided by $b(x)$ ($a(x) \div b(x)$, $a(x) / b(x)$) is denoted $a(x) \DIV b(x)$.
\end{definition}

\begin{definition}[Polynomial Remainder]\label{def:Polynomial_Remainder}
  The \emph{\nameref{def:Polynomial} remainder}, $r(x)$, of $a(x)$ divided by $b(x)$ ($a(x) \div b(x)$, $a(x) / b(x)$) is denoted $a(x) \bmod b(x)$.
\end{definition}

\begin{example}[Lecture 3]{Long Division of Polynomials}
  Calculate $\left( x^{3} + 11x^{2} + x + 7 \right) \bmod \left( x^{2} + x + 3 \right)$ while working in the \nameref{def:Field} \TextPolynomialRing{Z}{13}{x}?
  \tcblower{}
  First thing to note is that \TextPolynomialRing{Z}{13}{x} is a \nameref{def:Finite_Field} because 13 is a \nameref{def:Prime}.
  Now, we perform long division:

  \begin{center}
    \polylongdiv{x^3 + 11x^2 + x + 7}{x^2 + x + 3}
  \end{center}

  Since our \nameref{def:Polynomial_Remainder} is $(-12x - 23) \bmod 13$, we perform some reduction.
  Thus, the \nameref{def:Polynomial_Remainder} is:
  \begin{align*}
    -12x \bmod 13 &= 1x \bmod 13 \\
    -23 \bmod 13 = -10 \bmod 13 &= 3 \bmod 13
  \end{align*}
  So, the reduced \nameref{def:Polynomial_Quotient} and \nameref{def:Polynomial_Remainder} are:
  \begin{align*}
    q(x) &= x + 10 \\
    r(x) &= x + 3
  \end{align*}
\end{example}

\subsubsection{Properties of \nameref*{subsec:Polynomial_Rings}}\label{subsubsec:Polynomial_Ring_Properties}
\begin{definition}[Divide]\label{def:Polynomial_Ring_Properties-Divide}
  If $g(x), h(x) \in \MathField{F}{x}$, then $h(x)$ is said to \emph{divide} $g(x)$, written
  \begin{equation}\label{eq:Polynomial_Ring_Properties-Divide}
    h(x) \Divides g(x) \text{if } g(x) \bmod h(x) = 0
  \end{equation}
\end{definition}

\begin{definition}[Congruent]\label{def:Polynomial_Ring_Properties-Congruent}
  Let $g(x), h(x) \in \MathField{F}{x}$.
  Then, $g(x)$ is said to be \emph{congruent} to $h(x) \bmod f(x)$ if $f(x) \Divides \bigl( g(x) - h(x) \bigr)$.
  We denote this
  \begin{equation}\label{eq:Polynomial_Ring_Properties-Congruent}
    g(x) \equiv h(x) \pmod{f(x)}
  \end{equation}
\end{definition}

\begin{propertylist}
\item $g(x) \equiv h(x) \pmod{f(x)}$ if and only if $g(x)$ and $h(x)$ leave the same remainder when divided by $f(x)$.\label{prop:Polynomial_Ring_Properties-Equivalence}
\item $g(x) \equiv g(x) \pmod{f(x)}$.\label{prop:Polynomial_Ring_Properties-Reflexivity}
\item If $g(x) \equiv h(x) \pmod{f(x)}$, the $h(x) \equiv g(x) \pmod{f(x)}$.\label{prop:Polynomial_Ring_Properties-Symmetry}
\item If $g(x) \equiv h(x) \pmod{f(x)}$ and $h(x) \equiv s(x) \pmod{f(x)}$, then $g(x) \equiv s(x) \pmod{f(x)}$.\label{prop:Polynomial_Ring_Properties-Transitivity}
\item If $g(x) \equiv g_{1}(x) \pmod{f(x)}$ and $h(x) \equiv h_{1}(x) \pmod{f(x)}$, then:\label{prop:Polynomial_Ring_Properties-Linearity}
\begin{itemize}[noitemsep]
\item $g(x) + h(x) \equiv g_{1}(x) + h_{1}(x) \pmod{f(x)}$
\item $g(x) h(x) \equiv g_{1}(x) h_{1}(x) \pmod{f(x)}$
\end{itemize}
\end{propertylist}

We can divide \TextMathField{F}{x} into sets called \nameref{def:Polynomial_Ring_Properties-Equivalence_Classes}, where each \nameref{def:Polynomial_Ring_Properties-Equivalence_Classes} contains all \nameref{def:Polynomial}s that leaves a certain \nameref{def:Polynomial_Remainder} when divided by $f(x)$.

\begin{definition}[Equivalence Class]\label{def:Polynomial_Ring_Properties-Equivalence_Classes}
  By $\MathField{F}{x} / f(x)$, we denote the set of \emph{equivalence class}es of \nameref{def:Polynomial}s in \TextMathField{F}{x} of degree less than $\Degree f(x0$.
  The addition and multiplication operations are performed $\mod f(x)$.
\end{definition}

\begin{definition}[Representative]\label{def:Polynomial_Ring_Properties-Representative}
  Since the \nameref{def:Polynomial_Remainder}, $r(x)$ itself is a \nameref{def:Polynomial} in the \nameref{def:Polynomial_Ring_Properties-Equivalence_Classes} we use it as a \emph{representative} of the \nameref{def:Polynomial_Ring_Properties-Equivalence_Classes}.
\end{definition}

\begin{definition}[Commutative Ring]\label{def:Polynomial_Ring_Properties-Commutative_Ring}
  A commutative \nameref{def:Ring} for \nameref{def:Polynomial}s is defined as
  \begin{equation}\label{eq:Polynomial_Ring_Properties-Commutative_Ring}
    \MathField{F}{x} / f(x)
  \end{equation}
  \begin{remark}
    Note that this is the inverse condition of when a \nameref{def:Polynomial_Ring} is a \nameref{def:Polynomial_Ring_Properties-Field}.
  \end{remark}
\end{definition}

\begin{definition}[Field]\label{def:Polynomial_Ring_Properties-Field}
  If $f(x) \in \MathField{F}{x}$ is \nameref{def:Irreducible_Polynomial}, then $\MathField{F}{x} / f(x)$ is a \nameref{def:Field}.
  \begin{remark}
    Note that this is the inverse condition of when a \nameref{def:Polynomial_Ring} is a \nameref{def:Polynomial_Ring_Properties-Commutative_Ring}
  \end{remark}
\end{definition}

\begin{definition}[Finite Field]\label{def:Polynomial_Ring_Properties-Finite_Field}
  A \emph{finite \nameref{def:Polynomial_Ring_Properties-Field}} is a \nameref{def:Polynomial_Ring_Properties-Field} which contains a finite number of elements, i.e. the \nameref{def:Set_Order} of the \nameref{def:Polynomial_Ring_Properties-Field} is not $\infty$.

  \begin{propertylist}
  \item If $F$ is a \nameref{def:Polynomial_Ring_Properties-Finite_Field}, then the \nameref{def:Set_Order} of $F$ is $p^{m}$ for some \nameref{def:Prime} $p$ and integer $m \geq 1$.\label{prop:Polynomial_Ring_Properties-Order}
  \item For every \nameref{def:Prime} power order $p^{m}$, there is a unique (up to \nameref{def:Isomorphism}) \nameref{def:Polynomial_Ring_Properties-Finite_Field} of \nameref{def:Set_Order} $p^{m}$. This \nameref{def:Polynomial_Ring_Properties-Field} is denoted by \TextFiniteMathField{F}{p}{m} or $GF \left(p^{m} \right)$.\label{prop:Polynomial_Ring_Properties-Uniqueness}
  \item If \TextFiniteMathField{F}{q}{} is a \nameref{def:Polynomial_Ring_Properties-Finite_Field} of \nameref{def:Set_Order} $q = p^{m}$, i.e. \TextFiniteMathField{F}{p}{m}, the the \emph{\nameref{def:Field_Characteristic}} of \TextFiniteMathField{F}{q}{} is $p$.
    Furthermore, \TextMathField{F}{q} contains a copy of \TextIntsMod{p} as a \emph{\nameref{def:Subfield}}.\label{prop:Polynomial_Ring_Properties-Characteristic}
  \item Let \TextFiniteMathField{F}{q}{} be a \nameref{def:Polynomial_Ring_Properties-Finite_Field} of \nameref{def:Set_Order} $q = p^{m}$, i.e. \TextFiniteMathField{F}{p}{m}.
    Then every \nameref{def:Subfield} of \TextFiniteMathField{F}{q}{} has \nameref{def:Set_Order} $p^{n}$ for some positive integer $n$ where $n \Divides m$.
    Conversely, if $n \Divides m$, then there is exactly one \nameref{def:Subfield} of \TextFiniteMathField{F}{q}{} of \nameref{def:Set_Order} $p^{n}$.\label{prop:Polynomial_Ring_Properties-Subfield}
  \item An element $a \in \FiniteMathField{F}{q}{}$ is in the \nameref{def:Subfield} \TextFiniteMathField{F}{p}{n} if and only if $a^{p^{n} - 1} = 1$.\label{prop:Polynomial_Ring_Properties-Element_in_Subfield}
  \end{propertylist}
\end{definition}

\begin{definition}[Multiplicative Group]\label{def:Polynomial_Ring_Properties-Multiplicative_Group}
  The non-zero elements of \TextFiniteMathField{F}{q} all have inverses and thus, they form a \nameref{def:Group} under multiplication.
  This \nameref{def:Group} is called the \emph{multiplicative group} of \TextFiniteMathField{F}{q} and denoted by $\FiniteMathField{F}{q}{}^{*}$.
  It can be shown that $\FiniteMathField{F}{q}{}^{*}$ is a \nameref{def:Cyclic} \nameref{def:Group} (of \nameref{def:Set_Order} $q-1$).
  Especially, this means that $a^{q-1} = 1$ for all $a \in \FiniteMathField{F}{q}{}$.
\end{definition}

\begin{definition}[Primitive Element]\label{def:Polynomial_Ring_Properties-Primitive_Element}
  A \nameref{def:Generator} of the \nameref{def:Cyclic} \nameref{def:Group} $\FiniteMathField{F}{q}{}^{*}$ is called a \emph{primitive element}.
\end{definition}

\subsubsection{Extension of \nameref*{def:GCD}, \nameref*{def:Euclidean_Algorithm}, and \nameref*{def:Extended_Euclidean_Algorithm}}\label{subsubsec:Extend_GCD_Euclidean_Algorithms}
\begin{definition}[Greatest Common Divisor]\label{def:Polynomial_Ring_GCD}
  Let $g(x), h(x) \in \FinitePolynomialRing{Z}{p}{x}$, where not both are zero.
  Then the \emph{greatest common divisor}, \emph{GCD}, of $g(x)$ and $h(x)$, denoted $\gcd\bigl( g(x), h(x) \bigr)$, is the \nameref{def:Monic_Polynomial} of greatest \nameref{def:Polynomial_Degree} in \TextFinitePolynomialRing{Z}{p}{x} which \nameref{def:Polynomial_Ring_Properties-Divide}s both $g(x)$ and $h(x)$.

  \begin{remark}
    By definition $\gcd(0, 0) = 0$.
  \end{remark}
\end{definition}

\begin{theorem}[Unique Factorization of Polynomials]\label{thm:Unique_Factorization_of_Polynomials}
  Every non-zero polynomial $f(x) \in \FinitePolynomialRing{Z}{p}{x}$ has a factorization
  \begin{equation}\label{eq:Unique_Factorization_of_Polynomials}
    f(x) = a {f_{1}(x)}^{e_{1}} {f_{2}(x)}^{e_{2}} \cdots {f_{k}(x)}^{e_{k}}
  \end{equation}
  where each $f_{i}(x)$ is a distinct \nameref{def:Monic_Polynomial} \nameref{def:Irreducible_Polynomial} \nameref{def:Polynomial} in \TextFinitePolynomialRing{Z}{p}{x}, the $e_{i}$ are positive integers, and $a \in \IntsMod{p}$, where $p$ is for \nameref{def:Prime}s.
  The factorization is unique up to the rearrangement of factors.
\end{theorem}

\begin{definition}[Polynomial Euclidean Algorithm]\label{def:Polynomial_Euclidean_Algorithm}
  The \emph{polynomial \nameref*{def:Euclidean_Algorithm}} is an extension of the traditional \nameref{def:Euclidean_Algorithm} to \nameref{def:Polynomial} \nameref{def:Field}s.

  \begin{algorithm}[H]
    \DontPrintSemicolon{}
    \SetKwInOut{Input}{Input}\SetKwInOut{Output}{Output}

    \Input{Two non-negative \nameref{def:Polynomial}s $a(x), b(x) \in \FinitePolynomialRing{F}{q}{x}$.}
    \Output{$\gcd \bigl( a(x), b(x) \bigr)$.}
    \BlankLine{}

    Set $r_{0}(x) \leftarrow a(x)$, $r_{1}(x) \leftarrow b(x)$, $i \leftarrow 1$. \;
    \While{$r_{i}(x) \neq 0$}{
      Set $r_{i+1}(x) \leftarrow r_{i-1}(x) \bmod r_{i}(x)$, $i \leftarrow i+1$. \;
    }
    \Return{$r_{i}(x)$}
    \caption{Polynomial Euclidean Algorithm}
    \label{algo:Polynomial_Euclidean_Algorithm}
  \end{algorithm}
  
  This is demonstrated in \Cref{ex:Polynomial Euclidean Algorithm}
\end{definition}

\begin{example}[Lecture 3]{Polynomial Euclidean Algorithm}
  Given $f(x) = x^{3} + x$, where $f(x) \in \PolynomialRing{Z}{2}{x}$, find the \nameref{def:GCD} of $f(x)$ given that $f(x) g(x) = 1 \bmod \left( x^{4} + x + 1 \right)$.
    Assume $f(x) g(x)$ is \nameref{def:Irreducible_Polynomial}.
    \tcblower{}
    Considering that $\PolynomialRing{Z}{2}{x} / f(x)$ where $f(x) \in \PolynomialRing{Z}{2}{x}$ and $f(x)$ is \nameref{def:Irreducible_Polynomial}, this is a \nameref{def:Polynomial_Ring}.
    Since the order of the \nameref{def:Polynomial_Ring} is a \nameref{def:Prime} power, this is a \nameref{def:Finite_Field}.
    Because this is a \nameref{def:Finite_Field}, $\FinitePolynomialRing{F}{2^{4}}{} = 2^{4} = 16$ elements.
    
    Now, we calculate the \nameref{def:Polynomial_Euclidean_Algorithm}.
    \begin{align*}
      \gcd \Bigl( &\left( x^{4} + x + 1 \right), \left( x^{3} + x \right) \Bigl) \\
      x^{4} + x + 1 &= q(x) \left( x^{3} + x \right) + r(x) \\
    \end{align*}
    We can perform \nameref{def:Polynomial_Long_Division} to find $q(x)$ and $r(x)$ for this iteration.
    \begin{center}
      \polylongdiv{x^4+x+1}{x^3+x}
    \end{center}
    So,
    \begin{align*}
      q(x) &= x \\
      r(x) &= \left( -x^{2} + x + 1 \right) \bmod 2
    \end{align*}
    We need to correct the coefficients in $r(x)$ with modulo 2 (where the coefficients come from in the first place).
    \begin{align*}
      r(x) &= \left( -x^{2} + x + 1 \right) \bmod 2 \\
           &= x^{2} + x + 1 \\
      x^{4} + x + 1 &= x \left( x^{3} + x \right) + \left( x^{2} + x + 1 \right)
    \end{align*}

    Now, the next iteration of \nameref{def:Polynomial_Euclidean_Algorithm}.
    \begin{equation*}
      x^{3} + x = q(x) \left( x^{2} + x + 1 \right) + r(x)
    \end{equation*}
    \begin{center}
      \polylongdiv{x^{3} + x}{x^{2} + x + 1}
    \end{center}
    \begin{align*}
      q(x) &= x-1 \\
      r(x) &= x+1 \\
      x^{3} + x &= (x-1) \left( x^{2} + x + 1 \right) + (x+1)
    \end{align*}

    Another iteration:
    \begin{equation*}
      x^{2} + x + 1 = q(x) (x+1) + r(x)
    \end{equation*}
    \begin{center}
      \polylongdiv{x^{2} + x + 1}{x+1}
    \end{center}
    \begin{align*}
      q(x) &= x \\
      r(x) &= 1 \\
      x^{2} + x + 1 &= x (x+1) + 1
    \end{align*}

    Another iteration (the last):
    \begin{equation*}
      x+1 = q(x) \cdot 1 + r(x)
    \end{equation*}
    \begin{align*}
      q(x) &= x+1 \\
      r(x) &= 0 \\
      x+1 &= (x+1) \cdot 1 + 0
    \end{align*}

    Thus, the $\gcd \Bigl( x^{4} + x + 1, x^{3} + x \Bigr) = 1$.
\end{example}

\newpage
\begin{definition}[Polynomial Extended Euclidean Algorithm]\label{def:Polynomial_Extended_Euclidean_Algorithm}
  Let $a(x)$ and $b(x)$ be two non-negative polynomials in \TextPolynomialRing{F}{q}{x}.
  Then, there exists polynomials $s(x), t(x)$ such that $\gcd \bigl( a(x), b(x) \bigr)$ can be written as
  \begin{equation}\label{eq:Polynomial_Extended_Euclidean_Algorithm}
    \gcd \bigl( a(x), b(x) \bigr) = a(x) s(x) + b(x) t(x)
  \end{equation}

    \begin{algorithm}[H]
    \DontPrintSemicolon{}
    \SetKwInOut{Input}{Input}\SetKwInOut{Output}{Output}

    \Input{Two non-negative \nameref{def:Polynomial}s $a(x), b(x) \in \FinitePolynomialRing{F}{q}{x}$.}
    \Output{$d(x) = \gcd \bigl( a(x), b(x) \bigr)$ and two \nameref{def:Polynomial}s $u(x), v(x)$ such that $d(x) = a(x)u(x) + b(x)v(x)$.}
    \BlankLine{}

    \If{$b(x)=0$}{
      \Return{$a(x)$, $u(x) \leftarrow 1$, $v(x) \leftarrow 0$}
    }
    Set $u_{2}(x) \leftarrow 1$, $u_{1}(x) \leftarrow 0$, $v_{2}(x) \leftarrow 0$, and $v_{1}(x) \leftarrow 1$. \;
    \While{$b(x) > 0$}{
      Set $q(x) \leftarrow a(x) \div b(x)$. \;
      Set $r(x) \leftarrow a(x)-q(x)b(x)$ \;
      Set $u(x) \leftarrow u_{2}(x) - q(x)u_{1}(x)$ \;
      Set $v(x) \leftarrow v_{2}(x) - q(x)v_{1}(x)$ \;
      Set $a(x) \leftarrow b(x)$ \;
      Set $b(x) \leftarrow r(x)$ \;
      Set $u_{2}(x) \leftarrow u_{1}(x)$ \;
      Set $u_{1}(x) \leftarrow u(x)$ \;
      Set $v_{2}(x) \leftarrow v_{1}(x)$ \;
      Set $v_{1}(x) \leftarrow v(x)$ \;
    }
    Set $d(x) \leftarrow a(x)$, $u(x) \leftarrow u_{2}(x)$, $v(x) \leftarrow v_{2}(x)$. \;
    \Return{$d(x), u(x), v(x)$}
    \caption{Polynomial Extended Euclidean Algorithm (Bezout's Theorem)}
    \label{algo:Polynomial_Extended_Euclidean_Algorithm}
  \end{algorithm}

  This is demonstrated in \Cref{ex:Polynomial Extended Euclidean Algorithm}
\end{definition}

\begin{example}[Lecture 3]{Polynomial Extended Euclidean Algorithm}
  Given $f(x) = x^{3} + x$, where $f(x) \in \PolynomialRing{Z}{2}{x}$, find $g(x)$ of $f(x)$ given that $f(x) g(x) = 1 \bmod \left( x^{4} + x + 1 \right)$.
  Assume $f(x) g(x)$ is \nameref{def:Irreducible_Polynomial}.
  \tcblower{}
  Since we know that the $\gcd \left( x^{4} + x + 1, x^{3} + x \right) = 1$, from \Cref{ex:Polynomial Euclidean Algorithm}, we can solve \Cref{eq:Polynomial_Extended_Euclidean_Algorithm}.

  \begin{align*}
    1 &= \left( x^{2} + x + 1 \right) - x (x+1) \\
      &= \left( x^{2} + x + 1 \right) - x \Bigl( \left( x^{3} + x \right) - (x-1) \left( x^{2} + x + 1 \right) \Bigr) = \left( x^{2} + x + 1 \right) - x \left( x^{3} + x \right) - x(x-1) \left( x^{2} + x + 1 \right) \\
      &= \bigl( 1 - x(x-1) \bigr) \left( x^{2} + x + 1 \right) - x \left( x^{3} + x \right) \\
      &= \bigl (1 - x(x-1) \bigr) \Bigl( \left( x^{4} + x + 1 \right) - x \left(x^{3} + x \right) \Bigr) - x \left( x^{3} +x  \right) \\
      &= \bigl( 1 - x(x-1) \bigr) \left( x^{4} + x + 1 \right) - x \bigl( 1 - x(x-1) \bigr) \left( x^{3} + x \right) - x \left( x^{3} + x \right) \\
    &= \bigl[ 1 - x(x-1) \bigr] \left( x^{4} + x + 1 \right) - \Bigl[ x \bigl( 1 - x(x-1) \bigr) - x \Bigr] \left( x^{3} + x \right)
  \end{align*}

  \begin{align*}
    s(x) &= 1 - x(x-1) = -x^{2} - x - 1 \\
    t(x) &= x \bigl( 1-x (x-1) \bigr) - x = -x^{3} + x^{2}
  \end{align*}
  Now, we reduce the coefficients with respect to modulo 2.
  \begin{align*}
    s(x) &= x^{2} + x + 1 \\
    t(x) &= x^{3} + x^{2}
  \end{align*}

  So, we end up with:
  \begin{align*}
    1 \bmod \left( x^{4} + x + 1 \right) &= \left( x^{2} + x + 1 \right) \left( x^{4} + x + 1 \right) + \left( x^{3} + x^{2} \right) \left( x^{3} + x \right) \\
                                        &= \biggl( \left( x^{2} + x + 1 \right) \left( x^{4} + x + 1 \right) + \left( x^{3} + x^{2} \right) \left( x^{3} + x \right) \biggr) \bmod \left( x^{4} + x + 1 \right) \\
                                        &= 0 + \left( x^{3} + x^{2} \right) \left( x^{3} + x \right) \\
    f(x) &= \left( x^{3} + x \right) \\
                                         &= \left( x^{3} + x^{2} \right) f(x)
  \end{align*}

  Thus, $g(x) = x^{3} + x^{2}$.
\end{example}

\begin{definition}[Polynomial Basis Representation]
  The most common representation of element of a \nameref{def:Finite_Field} \TextFiniteMathField{F}{q}, where $q = p^{m}$, $p$ is a \nameref{def:Prime}, and is a \emph{polynomial basis representation}.
\end{definition}

\begin{theorem}\label{thm:Polynomial_Ring-Addition_Multiplication_Elements}
  Let $f(x) \in \FinitePolynomialRing{Z}{p}{x}$ be an \nameref{def:Irreducible_Polynomial} of \nameref{def:Polynomial_Degree} $m$.
  Then $\FinitePolynomialRing{Z}{p}{x} / f(x)$ is a \nameref{def:Polynomial_Ring_Properties-Finite_Field} of \nameref{def:Set_Order} $p^{m}$.
  The elements are all \nameref{def:Polynomial}s of \nameref{def:Polynomial_Degree} less than $m$.
  Addition and multiplication of elements is performed modulo $f(x)$.
\end{theorem}

\begin{lemma}\label{lemma:Polynomial_Ring-Monic_Irreducible_Polynomial_Existence}
  For each $m \geq 1$, there exists a \nameref{def:Monic_Polynomial} that is also an \nameref{def:Irreducible_Polynomial} of \nameref{def:Polynomial_Degree} $m$ over \TextIntsMod{p}.
\end{lemma}
%%% Local Variables:
%%% mode: latex
%%% TeX-master: "../EDIN01-Cryptography-Reference_Sheet"
%%% End:
