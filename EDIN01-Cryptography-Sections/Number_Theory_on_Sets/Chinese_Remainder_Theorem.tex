\subsection{Chinese Remainder Theorem}\label{subsec:Chinese_Remainder_Theorem}
\begin{theorem}[Chinese Remainder Theorem]\label{thm:Chinese_Remainder_Theorem}
  Let the integers $n_{1}, n_{2}, \ldots, n_{k}$ be pairwise \nameref{def:Relatively_Prime}.
  Then the system of \nameref{def:Congruence}s
  \begin{align*}
    x &\equiv a_{1} \pmod{n_{1}} \\
    x &\equiv a_{2} \pmod{n_{2}} \\
      &\vdots \\
    x &\equiv a_{k} \pmod{n_{k}}
  \end{align*}
  has a unique solution modulo $n = n_{1}n_{2} \cdots n_{k}$.
\end{theorem}

\begin{definition}[Gauss's Algorithm]\label{def:Gauss_Algorithm}
  The solution $x$ to the system of \nameref{def:Congruence}s promised by the \nameref{thm:Chinese_Remainder_Theorem} can be calculated as
  \begin{equation}\label{eq:Gauss_Algorithm}
    x = \Biggl( \sum\limits_{i=1}^{k}a_{i} N_{i} M_{i} \Biggr) \bmod n
  \end{equation}
  where $N_{i} = \frac{n}{n_{i}}$ and $M_{i} = N_{i}^{-1} = {\left( \frac{n}{n_{i}} \right)}^{-1} \bmod n_{i}$ ($M_{i}$ is the \nameref{def:Multiplicative_Inverse} of $N_{i} \bmod n_{i}$).
\end{definition}

\begin{definition}[Chinese Remainder Theorem]\label{def:Chinese_Remainder_Theorem}
  The \emph{\nameref{thm:Chinese_Remainder_Theorem}} allows us to change the way we represent elements of our set, \TextIntsModN{}.
  
  The integers modulo $n$, \TextIntsModN{}, where $n = n_{1}n_{2}$ and $\gcd(n_{1}, n_{2}) = 1$.
  An element $a \in$ \TextIntsModN{} has a unique representation: $(a \bmod n_{1}, a \bmod n_{2})$.
  We can denote this mapping by $\gamma : \IntsModN{} \rightarrow \IntsMod{n_{1}} \times \IntsMod{n_{2}}$.
  \begin{propertylist}
  \item $\gamma(a) = \gamma(b)$ if and only if $a = b$. \label{prop:Chinese_Remainder_Theorem_Property-Equivalence}
  \item For all $(a_{1}, a_{2}) \in \IntsMod{n_{1}} \times \IntsMod{n_{2}}$, there exists an $a$ such that $\gamma(a) = (a_{1}, a_{2})$.
  \item $\gamma(a+b) = \gamma(a) + \gamma(b)$
  \item $\gamma(ab) = \gamma(a) \gamma(b)$ \label{prop:Chinese_Remainder_Theorem_Property-Multiplication}
  \end{propertylist}
  These properties (\Crefrange{prop:Chinese_Remainder_Theorem_Property-Equivalence}{prop:Chinese_Remainder_Theorem_Property-Multiplication}) make $\gamma$ an \emph{\nameref{def:Isomorphism}}.

  \begin{remark}
    In the case of large integers for cryptography, knowing just one part of the number can ehlp get the other part.
    However, if the number is very large, 2048 bits for instance, these calculations start becoming \nameref{def:Intractable}.
  \end{remark}
\end{definition}

\begin{example}[]{Chinese Remainder Theorem Mapping}
  Find the mapping of $7$ when in \TextIntsMod{15}?
  \tcblower{}
  We need to find a prime factorization for $15$.
  \begin{equation*}
    15 = 3^{1} \cdot 5^{1}
  \end{equation*}
  So, the 2 modulos we will use are $3$ and $5$.
  Additionally, since 7 is an element in \TextIntsMod{15} we can simply say,
  \begin{equation*}
    7 \Leftrightarrow (7 \bmod 3,\, 7 \bmod 5) = (1, 2)
  \end{equation*}
\end{example}

\begin{example}[]{Reverse Chinese Remainder Theorem Mapping 1}
  Which element of \TextIntsMod{168} corresponds to the pair $([7],[5]) \in \IntsMod{8} \times \IntsMod{21}$?
  \tcblower{}
  If we apply the \nameref{def:Extended_Euclidean_Algorithm} to 8 and 21, we get $1 = 8 \cdot 8 + (-3) \cdot 21$.
  Note that $8 \cdot 8 = 64$ is congruent to $0 \bmod 8$ and $1 \bmod 21$, and $(-3) \cdot 21 = -63$ is congruent to $1 \bmod 8$ and $0 \bmod 21$.
  Therefore
  \begin{align*}
    5 \cdot 64+7 \cdot (-63) &\equiv 5 \cdot 0+7 \cdot 1 = 7 (\bmod 8) \\
    5 \cdot 64+7 \cdot (-63) &\equiv 5 \cdot 1+7 \cdot 0 = 5 (\bmod 21)
  \end{align*}
  so,
  \begin{equation*}
    5 \cdot 64 + 7 \cdot (-63) = -121 \equiv 47 (\bmod 168)
  \end{equation*}
  works, that is, $[47] \Leftrightarrow ([7],[5])$.
\end{example}

\begin{example}[]{Reverse Chinese Remainder Theorem Mapping 2}
  Given elements $z_{2} \in \IntsMod{2}$ and $z_{5} \in \IntsMod{5}$, find the corresponding element $z_{10} \in \IntsMod{10}$?
  \tcblower{}
  Because the $\gcd(2, 5) = 1$, we can also use the \nameref{def:Extended_Euclidean_Algorithm}.
  It returns $1 = 1 \cdot 5 - 2 \cdot 2$.
  Now we need to match the terms up with modulo factors that do not reduce them, i.e.\ any element $z_{5}$ should not be matched with 5, and likewise with $z_{2}$ and 2.

  Thus,
  \begin{equation*}
    z_{10} = (z_{2} \cdot 5 - z_{5} \cdot 2) \bmod 10
  \end{equation*}
\end{example}

%%% Local Variables:
%%% mode: latex
%%% TeX-master: "../../EDIN01-Cryptography-Reference_Sheet"
%%% End:
