\subsection{Euler's Theorem}\label{subsec:Eulers_Theorem}
\begin{theorem}[Euler's Theorem]\label{thm:Eulers_Theorem}
  If $a \in \MultiplicativeGroupN{}$, then
  \begin{equation}\label{eq:Eulers_Theorem}
    a^{\phi(n)} \equiv 1 \pmod{n}
  \end{equation}
\end{theorem}

\begin{proof}[Euler's Theorem]\label{proof:Eulers_Theorem}
  Let $\MultiplicativeGroupN{} = \lbrace a_{1}, a_{2}, \ldots, a_{\phi(n)} \rbrace$.
  Looking at the set of elements $A = \lbrace aa_{1}, aa_{2}, \ldots, aa_{\phi(n)} \rbrace$, it is easy to see that $A = a \MultiplicativeGroupN{}$.
  So we have 2 ways of writing the product of all of the elements, i.e.
  \begin{equation*}
    \prod\limits_{i=1}^{\phi(n)} a a_{i} = \prod\limits_{i=1}^{\phi(n)} a_{i}
  \end{equation*}
  
  This leads to
  \begin{equation*}
    \prod\limits_{i=1}^{\phi(n)} a = a^{\phi(n)} = 1
  \end{equation*}
  which is the same as what we said in \Cref{eq:Eulers_Theorem}.
\end{proof}

%%% Local Variables:
%%% mode: latex
%%% TeX-master: "../../EDIN01-Cryptography-Reference_Sheet"
%%% End:
