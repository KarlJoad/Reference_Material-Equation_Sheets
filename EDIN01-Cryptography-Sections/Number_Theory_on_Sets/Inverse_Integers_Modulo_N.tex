\subsection{\texorpdfstring{Inverse in \TextIntsModN{}}{Inverse in Integers Modulo n}}\label{subsec:Inverse_Z_mod_n}
Addition, subtraction, and multiplication can be performed trivially in \TextIntsModN{}, as shown in \Crefrange{ex:Addition on Integers mod n}{ex:Multiplication on Integers mod n}.
However, the concept of division is a little bit more difficult.
\begin{definition}[Multiplicative Inverse]\label{def:Multiplicative_Inverse}
  Let $a \in \IntsModN{}$.
  The \emph{multiplicative inverse} of $a$ is an integer $x \in \IntsModN{}$, such that $ax = 1$.
  If such an integer, $x$, exists, then $a$ is said to be \emph{invertible} and $x$ is called the inverse of $a$, denoted as $a^{-1}$.
\end{definition}

\begin{definition}[Division in \TextIntsModN{}]\label{def:Division_Z_mod_n}
  \emph{Division} of $a$ by an element $b \in \IntsModN{}$ (written $a/b$) is defined as $ab^{-1}$, and is only defined if $b$ has a \nameref{def:Multiplicative_Inverse}.
\end{definition}

\begin{definition}[Invertible]\label{def:Invertible}
  Let $a \in \IntsModN{}$.
  Then $a$ is \emph{invertible} if and only iff
  \begin{equation}\label{eq:Invertible}
    \gcd(a, n) = 1
  \end{equation}
\end{definition}

\begin{proof}
  Assume that $\gcd(a, n) = 1$.
  We know that $1 = \gcd(a, n) = xa + yn$ for some $x, y \in \AllIntegers$.
  Since $yn$ is a multiple of $n$, namely $yn \bmod n = 0$, it is removed from the equation.
  Then $x \bmod n$ is an inverse to $a$.

  Now assume $\gcd(a, n) > 1$.
  If $a$ has an inverse $x$, then $ax = 1 \bmod n$, which means $1 = ax + ny$ for some $x, y \in \AllIntegers$, directly contradicting the assumption that $\gcd(a, n) = 1$.
\end{proof}

The two possible cases of division, i.e.\ possible and impossible, are shown in \Crefrange{ex:Possible Division on Integers mod n}{ex:Impossible Division on Integers mod n}.

\begin{example}[]{Possible Division on Integers mod n}
  When dealing with the set of integers \TextIntsMod{15}, what is the result from the division of 5 by 11?
  \tcblower{}
  \begin{align*}
    5 \bmod 15 \div 11 \bmod 15 &= 5 \cdot 11^{-1}
  \end{align*}
  Now we need to find the \nameref{def:Multiplicative_Inverse} of $1$.
  \begin{equation*}
    11^{-1} = \gcd(11, 15)
  \end{equation*}
  We can compute the \nameref{def:GCD} efficiently with the \nameref{def:Euclidean_Algorithm}.
  \begin{align*}
    15 &= 1 \cdot 11 + 4 \\
    11 &= 2 \cdot 4 + 3 \\
    4 &= 1 \cdot 3 + 1 \\
    3 &= 3 \cdot 1 + 0 \\
  \end{align*}
  Thus, the \nameref{def:Euclidean_Algorithm} gives us $\gcd(11, 15) = 1$.
  Since $\gcd(11, 15) = 1 = 1$, 11 \textbf{does} have a \nameref{def:Multiplicative_Inverse}, making the division possible.
  Now we need to run through the \nameref{def:Extended_Euclidean_Algorithm}, to find the values $x, y \in \AllIntegers$.
  \begin{align*}
    1 &= 4 - 1 \cdot 3 \\
      &= 4 - 1 \cdot (11 - 2 \cdot 4) = 3 \cdot 4 - 1 \cdot 11 \\
      &= 3 (15 - 1 \cdot 11) - 1 \cdot 11 \\
      &= 3 \cdot 15 - 3 \cdot 11 - 1 \cdot 11 \\
      &= 3 \cdot 15 - 4 \cdot 11
  \end{align*}
  Thus,
  \begin{align*}
    x &= -4 \\
    y &= 3
  \end{align*}
  Now we know
  \begin{equation*}
    {(11 \bmod 15)}^{-1} = -4 \bmod 15
  \end{equation*}
  Since $-4$ is not part of \TextIntsMod{15}, we need to find the additive inverse.
  $-4 + 15 = 11$.
  Thus,
  \begin{equation*}
    {(11 \bmod 15)}^{-1} = 11 \bmod 15
  \end{equation*}
  Now, we perform a simple substitution.
  \begin{align*}
    5 \bmod 15 / 11 \bmod 15 &= 5 \bmod 15 \cdot {(11 \bmod 15)}^{-1} \\
                             &= 5 \bmod 15 \cdot 11 \bmod 15 \\
                             &= 55 \bmod 15 \\
                             &= 10
  \end{align*}

  So, the result of the division of $5$ by $11$ is $10$.
\end{example}

\begin{example}[]{Impossible Division on Integers mod n}
  When dealing with the set of integers \TextIntsMod{15}, what is the result from the division of 5 by 9?
  \tcblower{}
  \begin{equation*}
    5 \bmod 15 \div 9 \bmod 15 = 5 \cdot 9^{-1}
  \end{equation*}
  Now we need to find the \nameref{def:Multiplicative_Inverse} of $9$.
  \begin{equation*}
    9^{-1} = \gcd(9, 15)
  \end{equation*}
  We can compute the \nameref{def:GCD} efficiently with the \nameref{def:Euclidean_Algorithm}.
  \begin{align*}
    15 &= 1 \cdot 9 + 6 \\
    9 &= 1 \cdot 6 + 3 \\
    3 &= 1 \cdot 3 + 0 \\
  \end{align*}
  Thus, the \nameref{def:Euclidean_Algorithm} gives us $\gcd(9, 15) = 3$.
  Since $\gcd(9, 15) = 3 \neq 1$, 9 does \textbf{not} have a \nameref{def:Multiplicative_Inverse}, making the division impossible.
\end{example}

%%% Local Variables:
%%% mode: latex
%%% TeX-master: "../../EDIN01-Cryptography-Reference_Sheet"
%%% End:
