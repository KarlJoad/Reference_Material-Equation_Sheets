\subsection{\texorpdfstring{Multiplicative Groups, \TextMultiplicativeGroupN{}}{Multiplicative Groups}}\label{Multiplicative_Groups}
\begin{definition}[Multiplicative Group, \TextMultiplicativeGroupN{}]\label{def:Multiplicative_Group}
  Define the \emph{multiplicative group} of \TextIntsModN{}, denoted \TextMultiplicativeGroupN{} as the set of all elements in \TextIntsModN{} with \nameref{def:Multiplicative_Inverse}s.
  \begin{equation}\label{eq:Multiplicative_Group}
    \MultiplicativeGroupN{} = \lbrace a \in \IntsModN{} \vert \gcd(a, b) = 1 \rbrace
  \end{equation}
\end{definition}

\begin{definition}[Set Order]\label{def:Set_Order}
  The \emph{order a set}, for example, \TextMultiplicativeGroupN{}, is the number of elements in \TextMultiplicativeGroupN{} (\TextSetOrder{\MultiplicativeGroupN{}}).
  From the definition of the \nameref{def:Euler_Phi_Function}
  \begin{equation}\label{eq:Set_Order_Euler_Phi_Function}
    \SetOrder{\MultiplicativeGroupN{}} = \phi(n)
  \end{equation}

  \begin{remark}[Closed Under Multiplication]\label{rmk:Set_Order_Closed_Multiplication}
    Since the produce of two elements with \nameref{def:Multiplicative_Inverse}s is another element with a \nameref{def:Multiplicative_Inverse}, we say that \TextSetOrder{\MultiplicativeGroupN{}} is \emph{closed under multiplication}.
  \end{remark}
\end{definition}

\begin{definition}[Element Order]\label{def:Element_Order}
  The \emph{order of an element} $a \in \MultiplicativeGroupN{}$, denoted $\ElementOrder(a)$ is defined as the least positive integer $t$ ($t \in \AllIntegers$) such that
  \begin{equation}\label{eq:Element_Order}
    a^{t} \bmod n = 1
  \end{equation}
\end{definition}

\begin{lemma}[Element Order]\label{lemma:Element_Order}
  Let $a \in \MultiplicativeGroupN{}$.
  If $a^{s}$ for some $s$, then $\ElementOrder(a) \Divides s$.
  In particular, $\ElementOrder(a) \Divides \phi(n)$ must be true.
\end{lemma}

\begin{example}[Exercise 1, Problem 1.6b]{Element Order}
  Find the $\ElementOrder(5)$ in \TextMultiplicativeGroup{8}?
  \tcblower{}
  We need to solve
  \begin{equation*}
    a^{t} \bmod n = 1
  \end{equation*}
  where $a = 5$ and $n = 8$.
  \begin{equation*}
    5^{t} \bmod 8 = 1
  \end{equation*}

  Now we test values for $t$ until we satisfy the equation.
  \begin{align*}
    t &= 1 \rightarrow 5^{1} \bmod 8 = 5 \bmod 8 = 5 \neq 1 \\
    t &= 2 \rightarrow 5^{2} \bmod 8 = 25 \bmod 8 = 1 = 1
  \end{align*}

  Since $t=2$ satisfies our equation, the $\ElementOrder(5) = 2$.
\end{example}

\begin{proof}[Element Order]\label{proof:Element_Order}
  Let $t = \ElementOrder(a)$.
  By long division, $s = qt + r$, where $r < t$.
  Then $a^{s} = a^{qt + r} = a^{qt}a^{r}$ and since $a^{t} = 1$, from \Cref{eq:Element_Order}, we have $a^{s} = a^{r}$ and $a^{r} = 1$.
  This reduction is shown below:
  \begin{align*}
    a^{s} &= a^{qt + r} \\
          &= a^{qt}a^{r} \\
          &= {\left( a^{t} \right)}^{q} a^{r} \\
          &= {\left( 1 \right)}^{q} a^{r} \\
          &= 1^{q} a^{r} \\
          &= 1 a^{r} \\
          &= a^{r}
  \end{align*}

  But, $r<t$, so we must have $r=0$, and so $\ElementOrder(a) \Divides s$.
\end{proof}

%%% Local Variables:
%%% mode: latex
%%% TeX-master: "../../EDIN01-Cryptography-Reference_Sheet"
%%% End:
