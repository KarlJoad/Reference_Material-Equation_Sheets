\subsection{Generators}\label{subsec:Generators}
\begin{definition}[Generator]\label{def:Generator}
  Let $a \in \MultiplicativeGroupN{}$.
  If the \nameref{def:Element_Order} of $a$ is equal to the \nameref{def:Euler_Phi_Function}, i.e.\ $\ElementOrder(a) = \phi(n)$, then $a$ is said to be a \emph{generator} (or a \emph{primitive element}) of \TextMultiplicativeGroupN{}.
  \begin{subequations}\label{eq:Generator}
    \begin{equation}\label{eq:Generator-Equivalent_Element_Euler_Phi_Function}
      \ElementOrder(a) = \phi(n)
    \end{equation}
    \begin{equation}\label{eq:Generator-Equivalent_Element_Set_Orders}
      \ElementOrder(a) = \SetOrder{\MultiplicativeGroupN{}}
    \end{equation}
  \end{subequations}

  Furthermore, if \TextMultiplicativeGroupN{} has a generator, then \TextMultiplicativeGroupN{} is said to be\emph{\nameref{def:Cyclic}}.
  \begin{remark}
    It is clear that if $a \in \MultiplicativeGroupN{}$ is a \nameref{def:Generator}, then every element in \TextMultiplicativeGroupN{} can be expressed as $a^{i}$ for some integer $i$ ($i \in \AllIntegers$).
    So, we can write
    \begin{equation}\label{eq:Generator_in_Multiplicative_Group}
      \MultiplicativeGroupN{} = \lbrace a^{i} \vert 0 \leq i \leq \phi(n) - 1 \rbrace
    \end{equation}
  \end{remark}
\end{definition}

\begin{example}[Exercise 1, Problem 1.6c]{Cyclic Group}
  Is \TextMultiplicativeGroup{8} a \nameref{def:Cyclic} \nameref{def:Group}?
  \tcblower{}
    We need to find an $a$ that satisfies $\ElementOrder(a) = \phi(n)$.
  In this case $n=8$.
  First, we will calculate $\phi(n)$.
  \begin{equation*}
    \phi(n) = \phi(8)
  \end{equation*}
  We need to find a \nameref{def:Prime} factorization of $8$.
  \begin{align*}
    \phi(8) &= \phi \left( 2^{3} \right) \\
            &= \left( 2^{3} - 2^{3-1} \right) \\
            &= \left( 2^{3} - 2^{2} \right) \\
            &= (8-4) \\
            &= 4
  \end{align*}
  So, $\phi(8) = 4$.

  Now we need to solve
  \begin{equation*}
    \ElementOrder (a) = 4
  \end{equation*}

  All of the terms in \TextMultiplicativeGroup{8} must be \nameref{def:Invertible}, so they \textbf{must} satisfy $\gcd(z, 8) = 1, \, z \in \IntsMod{8}$.
  All of the terms in \TextIntsMod{8} are:
  \begin{equation*}
    \IntsMod{8} = \Bigl\lbrace [0], [1], [2], [3], [4], [5], [6], [7] \Bigr\rbrace
  \end{equation*}
  Now we check the \nameref{def:GCD}s for $z \in \IntsMod{8}$.
  \begin{align*}
    \gcd(0, 8) &= 8 & \gcd(4, 8) &= 2 \\
    \gcd(1, 8) &= 1 & \gcd(5, 8) &= 1 \\
    \gcd(2, 8) &= 2 & \gcd(6, 8) &= 2 \\
    \gcd(3, 8) &= 1 & \gcd(7, 8) &= 1
  \end{align*}
  So,
  \begin{equation*}
    \MultiplicativeGroup{8} = \Bigl\lbrace [1], [3], [5], [7] \Bigr\rbrace
  \end{equation*}

  Now we test $\ElementOrder{a} = 4, \, a \in \MultiplicativeGroup{8}$.
  \begin{align*}
    \ElementOrder(1) &= 1 \\
    \ElementOrder(3) = 3^{t} \bmod 8 = 1 \rightarrow t = 2 \rightarrow \ElementOrder(3) &= 2 \\
    \ElementOrder(5) &= 2 \\
    \ElementOrder(7) = 7^{t} \bmod 8 = 1 \rightarrow t = 2 \rightarrow \ElementOrder(7) &= 2
  \end{align*}

  Since no element from $\ElementOrder (\MultiplicativeGroup{8}) = \phi(8) = 4$, \TextMultiplicativeGroup{8} is \textbf{NOT} \nameref{def:Cyclic}.
\end{example}

%%% Local Variables:
%%% mode: latex
%%% TeX-master: "../../EDIN01-Cryptography-Reference_Sheet"
%%% End:
