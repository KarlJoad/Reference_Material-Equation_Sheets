\subsection{Quadratic Residues}\label{subsec:Quadratic_Residues}
\begin{definition}[Quadratic Residue]\label{def:Quadratic_Residue}
  An element $a \in \MultiplicativeGroupN{}$ is said to be a \emph{quadratic residue} modulo $n$ (or a \emph{square}) if there exists an $x \in \MultiplicativeGroupN{}$ such that $x^{2} = a$.
  \begin{subequations}\label{eq:Quadratic_Residue}
    \begin{equation}\label{subeq:Quadratic_Residue_1}
      a \in \IntsModN{} \exists x \in \MultiplicativeGroupN{}\:\: x^{2} = a \pmod{n}
    \end{equation}
    \begin{equation}\label{subeq:Quadratic_Residue_2}
      a \in \IntsModN{} \exists x \in \MultiplicativeGroupN{}\:\: a \equiv x^{2} \pmod{n}
    \end{equation}
  \end{subequations}

  \begin{remark}[Square Root]\label{rmk:Square_Root}
    If $x^{2} = a$, then $x$ is called the \emph{square root} of $a \bmod n$.
  \end{remark}

  Otherwise, $a$ is called a \emph{\nameref{def:Quadratic_Non_Residue} modulo $n$}.
\end{definition}

\begin{definition}[Quadratic Non-Residue]\label{def:Quadratic_Non_Residue}
  An element $a \in \MultiplicativeGroupN{}$ is said to be a \emph{quadratic non-residue} modulo $n$ if there does not exist an $x \in \MultiplicativeGroupN{}$ such that $x^{2} = a$.
  \begin{equation*}
    a \in \IntsModN{} \nexists x \in \MultiplicativeGroupN{}\:\: x^{2} = a \pmod{n}
  \end{equation*}
  Otherwise, $a$ is called a \emph{\nameref{def:Quadratic_Residue} modulo $n$}.
\end{definition}
%%% Local Variables:
%%% mode: latex
%%% TeX-master: "../../EDIN01-Cryptography-Reference_Sheet"
%%% End:
