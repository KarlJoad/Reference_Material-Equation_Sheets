\subsection{\texorpdfstring{\TextIntsModN{}}{Sets of Integers Modulo n}}\label{subsec:Z_mod_n}
\begin{definition}[\TextIntsModN{}]\label{def:Z_mod_n}
  \nameref{subsec:Integer_Modulo_n}, denoted \TextIntsModN{}, is the set of (\nameref{def:Equivalence_Class}es of) integers $\lbrace [0], [1], \ldots , [n-1] \rbrace$.
  Addition, subtraction, and multiplication are all performed with modulo $n$.
  \Crefrange{ex:Addition on Integers mod n}{ex:Multiplication on Integers mod n} demonstrate this.
\end{definition}

\begin{example}[]{Addition on Integers mod n}
  When dealing with the set of integers \TextIntsMod{15}, what is the sum of 5 and 9?
  \tcblower{}
  \begin{equation*}
    5 \bmod 15 + 9 \bmod 15 = 11 \bmod 15
  \end{equation*}

  Thus, the answer is 11.
\end{example}

\begin{example}[]{Subtraction on Integers mod n}
  When dealing with the set of integers \TextIntsMod{15}, what is 5 minus 9?
  \tcblower{}
  \begin{align*}
    5 \bmod 15 - 9 \bmod 15 &= 5 \bmod 15 + (-9 \bmod 15) \\
                            &= 5 \bmod 15 + \underbrace{-9 \bmod 15}_{-9 + 15 = 6} \\
                            &= 5 \bmod 15 + 6 \bmod 15 \\
                            &= 11 \bmod 15
  \end{align*}

  Thus, the answer is, again, 11.
\end{example}

\begin{example}[]{Multiplication on Integers mod n}
  When dealing with the set of integers \TextIntsMod{15}, what is the product of 5 and 9?
  \tcblower{}
  \begin{align*}
    5 \bmod 15 \cdot 9 \bmod 15 &= 45 \bmod 15 \\
                                &= 0
  \end{align*}

  Thus, the answer is 0, because $45 = 3 \cdot 15$.
\end{example}

%%% Local Variables:
%%% mode: latex
%%% TeX-master: "../../EDIN01-Cryptography-Reference_Sheet"
%%% End:
