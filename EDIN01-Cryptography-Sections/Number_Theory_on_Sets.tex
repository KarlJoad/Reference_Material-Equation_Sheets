\section{\nameref*{sec:Number_Theory} on Sets}\label{sec:Number_Theory_on_Sets}
While this section is not technically different than \Cref{sec:Number_Theory}, it is worth it to split these up, since \Cref{sec:Number_Theory} does not deal with sets.
However, using what we learned in \Cref{sec:Number_Theory}, \nameref{sec:Number_Theory}, it is natural to extend these to sets of numbers.

\subsection{\texorpdfstring{\TextIntsModN{}}{Sets of Integers Modulo n}}\label{subsec:Z_mod_n}
\begin{definition}[\TextIntsModN{}]\label{def:Z_mod_n}
  \nameref{subsec:Integer_Modulo_n}, denoted \TextIntsModN{}, is the set of (\nameref{def:Equivalence_Class}es of) integers $\lbrace [0], [1], \ldots , [n-1] \rbrace$.
  Addition, subtraction, and multiplication are all performed with modulo $n$.
  \Crefrange{ex:Addition on Integers mod n}{ex:Multiplication on Integers mod n} demonstrate this.
\end{definition}

\begin{example}[]{Addition on Integers mod n}
  When dealing with the set of integers \TextIntsMod{15}, what is the sum of 5 and 9?
  \tcblower{}
  \begin{equation*}
    5 \bmod 15 + 9 \bmod 15 = 11 \bmod 15
  \end{equation*}

  Thus, the answer is 11.
\end{example}

\begin{example}[]{Subtraction on Integers mod n}
  When dealing with the set of integers \TextIntsMod{15}, what is 5 minus 9?
  \tcblower{}
  \begin{align*}
    5 \bmod 15 - 9 \bmod 15 &= 5 \bmod 15 + (-9 \bmod 15) \\
                            &= 5 \bmod 15 + \underbrace{-9 \bmod 15}_{-9 + 15 = 6} \\
                            &= 5 \bmod 15 + 6 \bmod 15 \\
                            &= 11 \bmod 15
  \end{align*}

  Thus, the answer is, again, 11.
\end{example}

\begin{example}[]{Multiplication on Integers mod n}
  When dealing with the set of integers \TextIntsMod{15}, what is the product of 5 and 9?
  \tcblower{}
  \begin{align*}
    5 \bmod 15 \cdot 9 \bmod 15 &= 45 \bmod 15 \\
                                &= 0
  \end{align*}

  Thus, the answer is 0, because $45 = 3 \cdot 15$.
\end{example}

\subsection{\texorpdfstring{Inverse in \TextIntsModN{}}{Inverse in Integers Modulo n}}\label{subsec:Inverse_Z_mod_n}
Addition, subtraction, and multiplication can be performed trivially in \TextIntsModN{}, as shown in \Crefrange{ex:Addition on Integers mod n}{ex:Multiplication on Integers mod n}.
However, the concept of division is a little bit more difficult.
\begin{definition}[Multiplicative Inverse]\label{def:Multiplicative_Inverse}
  Let $a \in \IntsModN{}$.
  The \emph{multiplicative inverse} of $a$ is an integer $x \in \IntsModN{}$, such that $ax = 1$.
  If such an integer, $x$, exists, then $a$ is said to be \emph{invertible} and $x$ is called the inverse of $a$, denoted as $a^{-1}$.
\end{definition}

\begin{definition}[Division in \TextIntsModN{}]\label{def:Division_Z_mod_n}
  \emph{Division} of $a$ by an element $b \in \IntsModN{}$ (written $a/b$) is defined as $ab^{-1}$, and is only defined if $b$ has a \nameref{def:Multiplicative_Inverse}.
\end{definition}

\begin{definition}[Invertible]\label{def:Invertible}
  Let $a \in \IntsModN{}$.
  Then $a$ is \emph{invertible} if and only iff
  \begin{equation}\label{eq:Invertible}
    \gcd(a, n) = 1
  \end{equation}
\end{definition}

\begin{proof}
  Assume that $\gcd(a, n) = 1$.
  We know that $1 = \gcd(a, n) = xa + yn$ for some $x, y \in \AllIntegers$.
  Since $yn$ is a multiple of $n$, namely $yn \bmod n = 0$, it is removed from the equation.
  Then $x \bmod n$ is an inverse to $a$.

  Now assume $\gcd(a, n) > 1$.
  If $a$ has an inverse $x$, then $ax = 1 \bmod n$, which means $1 = ax + ny$ for some $x, y \in \AllIntegers$, directly contradicting the assumption that $\gcd(a, n) = 1$.
\end{proof}

The two possible cases of division, i.e.\ possible and impossible, are shown in \Crefrange{ex:Possible Division on Integers mod n}{ex:Impossible Division on Integers mod n}.

\begin{example}[]{Possible Division on Integers mod n}
  When dealing with the set of integers \TextIntsMod{15}, what is the result from the division of 5 by 11?
  \tcblower{}
  \begin{align*}
    5 \bmod 15 \div 11 \bmod 15 &= 5 \cdot 11^{-1}
  \end{align*}
  Now we need to find the \nameref{def:Multiplicative_Inverse} of $1$.
  \begin{equation*}
    11^{-1} = \gcd(11, 15)
  \end{equation*}
  We can compute the \nameref{def:GCD} efficiently with the \nameref{def:Euclidean_Algorithm}.
  \begin{align*}
    15 &= 1 \cdot 11 + 4 \\
    11 &= 2 \cdot 4 + 3 \\
    4 &= 1 \cdot 3 + 1 \\
    3 &= 3 \cdot 1 + 0 \\
  \end{align*}
  Thus, the \nameref{def:Euclidean_Algorithm} gives us $\gcd(11, 15) = 1$.
  Since $\gcd(11, 15) = 1 = 1$, 11 \textbf{does} have a \nameref{def:Multiplicative_Inverse}, making the division possible.
  Now we need to run through the \nameref{def:Extended_Euclidean_Algorithm}, to find the values $x, y \in \AllIntegers$.
  \begin{align*}
    1 &= 4 - 1 \cdot 3 \\
      &= 4 - 1 \cdot (11 - 2 \cdot 4) = 3 \cdot 4 - 1 \cdot 11 \\
      &= 3 (15 - 1 \cdot 11) - 1 \cdot 11 \\
      &= 3 \cdot 15 - 3 \cdot 11 - 1 \cdot 11 \\
      &= 3 \cdot 15 - 4 \cdot 11
  \end{align*}
  Thus,
  \begin{align*}
    x &= -4 \\
    y &= 3
  \end{align*}
  Now we know
  \begin{equation*}
    {(11 \bmod 15)}^{-1} = -4 \bmod 15
  \end{equation*}
  Since $-4$ is not part of \TextIntsMod{15}, we need to find the additive inverse.
  $-4 + 15 = 11$.
  Thus,
  \begin{equation*}
    {(11 \bmod 15)}^{-1} = 11 \bmod 15
  \end{equation*}
  Now, we perform a simple substitution.
  \begin{align*}
    5 \bmod 15 / 11 \bmod 15 &= 5 \bmod 15 \cdot {(11 \bmod 15)}^{-1} \\
                             &= 5 \bmod 15 \cdot 11 \bmod 15 \\
                             &= 55 \bmod 15 \\
                             &= 10
  \end{align*}

  So, the result of the division of $5$ by $11$ is $10$.
\end{example}

\begin{example}[]{Impossible Division on Integers mod n}
  When dealing with the set of integers \TextIntsMod{15}, what is the result from the division of 5 by 9?
  \tcblower{}
  \begin{equation*}
    5 \bmod 15 \div 9 \bmod 15 = 5 \cdot 9^{-1}
  \end{equation*}
  Now we need to find the \nameref{def:Multiplicative_Inverse} of $9$.
  \begin{equation*}
    9^{-1} = \gcd(9, 15)
  \end{equation*}
  We can compute the \nameref{def:GCD} efficiently with the \nameref{def:Euclidean_Algorithm}.
  \begin{align*}
    15 &= 1 \cdot 9 + 6 \\
    9 &= 1 \cdot 6 + 3 \\
    3 &= 1 \cdot 3 + 0 \\
  \end{align*}
  Thus, the \nameref{def:Euclidean_Algorithm} gives us $\gcd(9, 15) = 3$.
  Since $\gcd(9, 15) = 3 \neq 1$, 9 does \textbf{not} have a \nameref{def:Multiplicative_Inverse}, making the division impossible.
\end{example}

\subsection{Chinese Remainder Theorem}\label{subsec:Chinese_Remainder_Theorem}
\begin{theorem}[Chinese Remainder Theorem]\label{thm:Chinese_Remainder_Theorem}
  Let the integers $n_{1}, n_{2}, \ldots, n_{k}$ be pairwise \nameref{def:Relatively_Prime}.
  Then the system of \nameref{def:Congruence}s
  \begin{align*}
    x &\equiv a_{1} \pmod{n_{1}} \\
    x &\equiv a_{2} \pmod{n_{2}} \\
      &\vdots \\
    x &\equiv a_{k} \pmod{n_{k}}
  \end{align*}
  has a unique solution modulo $n = n_{1}n_{2} \cdots n_{k}$.
\end{theorem}

\begin{definition}[Gauss's Algorithm]\label{def:Gauss_Algorithm}
  The solution $x$ to the system of \nameref{def:Congruence}s promised by the \nameref{thm:Chinese_Remainder_Theorem} can be calculated as
  \begin{equation}\label{eq:Gauss_Algorithm}
    x = \Biggl( \sum\limits_{i=1}^{k}a_{i} N_{i} M_{i} \Biggr) \bmod n
  \end{equation}
  where $N_{i} = \frac{n}{n_{i}}$ and $M_{i} = N_{i}^{-1} = {\left( \frac{n}{n_{i}} \right)}^{-1} \bmod n_{i}$ ($M_{i}$ is the \nameref{def:Multiplicative_Inverse} of $N_{i} \bmod n_{i}$).
  
  This simplifies to
  \begin{equation}\label{eq:Gauss_Algorithm_Simplified}
    x = \sum\limits_{i=1}^{k}a_{i} \frac{n}{n_{i}} \Biggl( \frac{n_{i}}{n} \bmod n \Biggr)
  \end{equation}
\end{definition}

\begin{definition}[Chinese Remainder Theorem]\label{def:Chinese_Remainder_Theorem}
  The \emph{\nameref{thm:Chinese_Remainder_Theorem}} allows us to change the way we represent elements of our set, \TextIntsModN{}.
  
  The integers modulo $n$, \TextIntsModN{}, where $n = n_{1}n_{2}$ and $\gcd(n_{1}, n_{2}) = 1$.
  An element $a \in$ \TextIntsModN{} has a unique representation: $(a \bmod n_{1}, a \bmod n_{2})$.
  We can denote this mapping by $\gamma : \IntsModN{} \rightarrow \IntsMod{n_{1}} \times \IntsMod{n_{2}}$.
  \begin{propertylist}
  \item $\gamma(a) = \gamma(b)$ if and only if $a = b$. \label{prop:Chinese_Remainder_Theorem_Property-Equivalence}
  \item For all $(a_{1}, a_{2}) \in \IntsMod{n_{1}} \times \IntsMod{n_{2}}$, there exists an $a$ such that $\gamma(a) = (a_{1}, a_{2})$.
  \item $\gamma(a+b) = \gamma(a) + \gamma(b)$
  \item $\gamma(ab) = \gamma(a) \gamma(b)$ \label{prop:Chinese_Remainder_Theorem_Property-Multiplication}
  \end{propertylist}
  These properties (\Crefrange{prop:Chinese_Remainder_Theorem_Property-Equivalence}{prop:Chinese_Remainder_Theorem_Property-Multiplication}) make $\gamma$ an \emph{\nameref{def:Isomorphism}}.

  \begin{remark}
    In the case of large integers for cryptography, knowing just one part of the number can ehlp get the other part.
    However, if the number is very large, 2048 bits for instance, these calculations start becoming \nameref{def:Intractable}.
  \end{remark}
\end{definition}

\begin{example}[]{Chinese Remainder Theorem Mapping}
  Find the mapping of $7$ when in \TextIntsMod{15}?
  \tcblower{}
  Since 7 is an element in \TextIntsMod{15},
  \begin{equation*}
    7 \Leftrightarrow (7 \bmod 3, 7 \bmod 5) = (1, 2)
  \end{equation*}
\end{example}

\subsection{\texorpdfstring{Multiplicative Groups, \TextMultiplicativeGroupN{}}{Multiplicative Groups}}\label{Multiplicative_Groups}
\begin{definition}[Multiplicative Group, \TextMultiplicativeGroupN{}]\label{def:Multiplicative_Group}
  Define the \emph{multiplicative group} of \TextIntsModN{}, denoted \TextMultiplicativeGroupN{} as the set of all elements in \TextIntsModN{} with \nameref{def:Multiplicative_Inverse}s.
  \begin{equation}\label{eq:Multiplicative_Group}
    \MultiplicativeGroupN{} = \lbrace a \in \IntsModN{} \vert \gcd(a, b) = 1 \rbrace
  \end{equation}
\end{definition}

\begin{definition}[Set Order]\label{def:Set_Order}
  The \emph{order a set}, for example, \TextMultiplicativeGroupN{}, is the number of elements in \TextMultiplicativeGroupN{} (\TextSetOrder{\MultiplicativeGroupN{}}).
  From the definition of the \nameref{def:Euler_Phi_Function}
  \begin{equation}\label{eq:Set_Order_Euler_Phi_Function}
    \SetOrder{\MultiplicativeGroupN{}} = \phi(n)
  \end{equation}

  \begin{remark}[Closed Under Multiplication]\label{rmk:Set_Order_Closed_Multiplication}
    Since the produce of two elements with \nameref{def:Multiplicative_Inverse}s is another element with a \nameref{def:Multiplicative_Inverse}, we say that \TextSetOrder{\MultiplicativeGroupN{}} is \emph{closed under multiplication}.
  \end{remark}
\end{definition}

\begin{definition}[Element Order]\label{def:Element_Order}
  The \emph{order of an element} $a \in \MultiplicativeGroupN{}$, denoted $\ElementOrder(a)$ is defined as the least positive integer $t$ ($t \in \AllIntegers$) such that
  \begin{equation}\label{eq:Element_Order}
    a^{t} \bmod n = 1
  \end{equation}
\end{definition}

\begin{lemma}[Element Order]\label{lemma:Element_Order}
  Let $a \in \MultiplicativeGroupN{}$.
  If $a^{s}$ for some $s$, then $\ElementOrder(a) \Divides s$.
  In particular, $\ElementOrder(a) \Divides \phi(n)$ must be true.
\end{lemma}

\begin{example}[Exercise 1, Problem 1.6b]{Element Order}
  Find the $\ElementOrder(5)$ in \TextMultiplicativeGroup{8}?
  \tcblower{}
  \textbf{TODO}
  \textit{Remember to check with $\Divides \phi(n)$, since $5 \in \MultiplicativeGroup{8}$.}
\end{example}

\begin{proof}[Element Order]\label{proof:Element_Order}
  Let $t = \ElementOrder(a)$.
  By long division, $s = qt + r$, where $r < t$.
  Then $a^{s} = a^{qt + r} = a^{qt}a^{r}$ and since $a^{t} = 1$, from \Cref{eq:Element_Order}, we have $a^{s} = a^{r}$ and $a^{r} = 1$.
  This reduction is shown below:
  \begin{align*}
    a^{s} &= a^{qt + r} \\
          &= a^{qt}a^{r} \\
          &= {\left( a^{t} \right)}^{q} a^{r} \\
          &= {\left( 1 \right)}^{q} a^{r} \\
          &= 1^{q} a^{r} \\
          &= 1 a^{r} \\
          &= a^{r}
  \end{align*}

  But, $r<t$, so we must have $r=0$, and so $\ElementOrder(a) \Divides s$.
\end{proof}

\subsection{Euler's Theorem}\label{subsec:Eulers_Theorem}
\begin{theorem}[Euler's Theorem]\label{thm:Eulers_Theorem}
  If $a \in \MultiplicativeGroupN{}$, then
  \begin{equation}\label{eq:Eulers_Theorem}
    a^{\phi(n)} \equiv 1 \pmod{n}
  \end{equation}
\end{theorem}

\begin{proof}[Euler's Theorem]\label{proof:Eulers_Theorem}
  Let $\MultiplicativeGroupN{} = \lbrace a_{1}, a_{2}, \ldots, a_{\phi(n)} \rbrace$.
  Looking at the set of elements $A = \lbrace aa_{1}, aa_{2}, \ldots, aa_{\phi(n)} \rbrace$, it is easy to see that $A = a \MultiplicativeGroupN{}$.
  So we have 2 ways of writing the product of all of the elements, i.e.
  \begin{equation*}
    \prod\limits_{i=1}^{\phi(n)} a a_{i} = \prod\limits_{i=1}^{\phi(n)} a_{i}
  \end{equation*}
  
  This leads to
  \begin{equation*}
    \prod\limits_{i=1}^{\phi(n)} a = a^{\phi(n)} = 1
  \end{equation*}
  which is the same as what we said in \Cref{eq:Eulers_Theorem}.
\end{proof}

\subsection{Fermat's Little Theorem}\label{subsec:Fermats_Little_Theorem}
\begin{theorem}[Fermat's Little Theorem]\label{thm:Fermats_Little_Theorem}
  Let $p$ be a \nameref{def:Prime} number.
  If $\gcd(a, p) = 1$, then
  \begin{equation}\label{eq:Fermats_Little_Theorem}
    a^{p-1} \equiv 1 \pmod{p}
  \end{equation}
\end{theorem}
\begin{remark*}
  This ties in with \nameref{thm:Eulers_Theorem}, because working in \TextIntsModN{}, all exponents can be reduced by modulo $\phi(n)$.
\end{remark*}

\subsection{Generators}\label{subsec:Generators}
\begin{definition}[Generator]\label{def:Generator}
  Let $a \in \MultiplicativeGroupN{}$.
  If $\ElementOrder(a) = \phi(n)$, then $a$ is said to be a \emph{generator} (or a \emph{primitive element}) of \TextMultiplicativeGroupN{}.
  Furthermore, if \TextMultiplicativeGroupN{} has a generator, then \TextMultiplicativeGroupN{} is said to be\emph{\nameref{def:Cyclic}}.
  \begin{remark}
    It is clear that if $a \in \MultiplicativeGroupN{}$ is a \nameref{def:Generator}, then every element in \TextMultiplicativeGroupN{} can be expressed as $a^{i}$ for some integer $i$ ($i \in \AllIntegers$).
    So, we can write
    \begin{equation}\label{eq:Generator_in_Multiplicative_Group}
      \MultiplicativeGroupN{} = \lbrace a^{i} \vert 0 \leq i \leq \phi(n) - 1 \rbrace
    \end{equation}
  \end{remark}
\end{definition}

\begin{example}[Exercise 1, Question 1.6c]{Cyclic Group}
  Is \TextMultiplicativeGroup{8} a \nameref{def:Cyclic} \nameref{def:Group}?
  \tcblower{}
  \textbf{TODO}
\end{example}

\subsection{Quadratic Residues}\label{subsec:Quadratic_Residues}
\begin{definition}[Quadratic Residue]\label{def:Quadratic_Residue}
  An element $a \in \MultiplicativeGroupN{}$ is said to be a \emph{quadratic residue} modulo $n$ (or a \emph{square}) if there exists an $x \in \MultiplicativeGroupN{}$ such that $x^{2} = a$.
  \begin{subequations}\label{eq:Quadratic_Residue}
    \begin{equation}\label{subeq:Quadratic_Residue_1}
      a \in \IntsModN{} \exists x \in \MultiplicativeGroupN{}\:\: x^{2} = a \pmod{n}
    \end{equation}
    \begin{equation}\label{subeq:Quadratic_Residue_2}
      a \in \IntsModN{} \exists x \in \MultiplicativeGroupN{}\:\: a \equiv x^{2} \pmod{n}
    \end{equation}
  \end{subequations}

  \begin{remark}[Square Root]\label{rmk:Square_Root}
    If $x^{2} = a$, then $x$ is called the \emph{square root} of $a \bmod n$.
  \end{remark}

  Otherwise, $a$ is called a \emph{\nameref{def:Quadratic_Non_Residue} modulo $n$}.
\end{definition}

\begin{definition}[Quadratic Non-Residue]\label{def:Quadratic_Non_Residue}
  An element $a \in \MultiplicativeGroupN{}$ is said to be a \emph{quadratic non-residue} modulo $n$ if there does not exist an $x \in \MultiplicativeGroupN{}$ such that $x^{2} = a$.
  \begin{equation*}
    a \in \IntsModN{} \nexists x \in \MultiplicativeGroupN{}\:\: x^{2} = a \pmod{n}
  \end{equation*}
  Otherwise, $a$ is called a \emph{\nameref{def:Quadratic_Residue} modulo $n$}.
\end{definition}
%%% Local Variables:
%%% mode: latex
%%% TeX-master: "../EDIN01-Cryptography-Reference_Sheet"
%%% End:
