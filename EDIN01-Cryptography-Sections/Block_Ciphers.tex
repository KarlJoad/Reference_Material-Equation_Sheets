\section{Block Ciphers}\label{sec:Block_Ciphers}
\begin{definition}[Block Cipher]\label{def:Block_Cipher}
  A \emph{block cipher} encrypts a fixed-length block of plaintext bits $x$ to a fixed-length block of ciphertext $y$.
  This transformation is controlled by the secret key $K$, and is written $E_{K}(x) = y$.

  The secret key defines a fixed mapping of the plaintext block $x$ to the ciphertext block $y$.
  This can sometimes make block ciphers a form of \nameref{subsec:Simple_Substitution_Cipher}.

  Block ciphers are usually implemented with several \nameref{def:Round_Function}s, based on the \nameref{def:Round_Key}.
\end{definition}

\begin{definition}[Round Function]\label{def:Round_Function}
  \nameref{def:Block_Cipher}s are usually implemented as iterated ciphers, where a simple encryption function is iteratively applied for $N$ rounds.
  This is called a \emph{round function}.
\end{definition}

\subsection{Modes of Operation}\label{subsec:Modes_of_Operation}
\begin{definition}[Mode of Operation]\label{def:Mode_of_Operation}
  
\end{definition}

\subsection{Advanced Encryption Scheme}\label{subsec:AES}
\textbf{TODO}
\begin{definition}[Advanced Encryption Scheme]\label{def:AES}
  \textbf{TODO}
\end{definition}

%%% Local Variables:
%%% mode: latex
%%% TeX-master: "../EDIN01-Cryptography-Reference_Sheet"
%%% End:
