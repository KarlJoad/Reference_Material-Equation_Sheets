\section{Block Ciphers}\label{sec:Block_Ciphers}
\begin{definition}[Block Cipher]\label{def:Block_Cipher}
  A \emph{block cipher} encrypts a fixed-length block of plaintext bits $x$ to a fixed-length block of ciphertext $y$.
  This transformation is controlled by the secret key $K$, and is written $E_{K}(x) = y$.

  The secret key defines a fixed mapping of the plaintext block $x$ to the ciphertext block $y$.
  This can sometimes make block ciphers a form of \nameref{subsec:Simple_Substitution_Cipher}.

  Block ciphers are usually implemented with several \nameref{def:Round_Function}s, based on the \nameref{def:Round_Key}.
\end{definition}

\begin{definition}[Round Function]\label{def:Round_Function}
  \nameref{def:Block_Cipher}s are usually implemented as iterated ciphers, where a simple encryption function is iteratively applied for $N$ rounds.
  This is called a \emph{round function}.
  It is commonly denoted
  \begin{equation}\label{eq:Round_Function}
    h(w_{i-1}, k_{i})
  \end{equation}
  where
  \begin{itemize}[noitemsep]
  \item $i$ is the current round (current iteration).
  \item $w$ is the input plaintext block.
  \item $k$ is the \nameref{def:Round_Key} on the $i$th iteration.
  \item $h$ is the \nameref{def:Round_Function}.
  \end{itemize}

  These functions must also be invertible, namely,
  \begin{equation}\label{eq:Round_Function_Invertible}
    h^{-1} \bigl( h(w_{i-1}, k_{i}), k_{i} \bigr) = w_{i-1}
  \end{equation}

  \textbf{These functions must be efficient to compute, and be efficient to compute the inverse round function.}
\end{definition}

\begin{definition}[Round Key]\label{def:Round_Key}
  The \emph{round key} is derived from the key $K$.
  The way in which the round key is derived from $K$ is called the \emph{key schedule}.
\end{definition}

If we want to mathematically illustrate the implementation of a \nameref{def:Block_Cipher}'s encryption with a iterative cipher, it is defined as:
\begin{align*}
  w_{0} &= x \\
  w_{1} &= h(w_{0}, k_{1}) \\
  w_{2} &= h(w_{1}, k_{2}) \\
        &\vdots \\
  w_{N-1} &= h(w_{N-2}, k_{N-1}) \\
  w_{N} &= h(w_{N-1}, k_{N}) \\
\end{align*}
\begin{itemize}[noitemsep]
\item $x$ is the plaintext block.
\item $w_{i}$ are intermediate values in the implementation of the iteration.
\item $w_{N}$ is the final output from the cipher.
\item $h(w_{i-1}, k_{i})$ denotes the round function.
\item $k_{i}$ is the round key used in the $i$th round.
\end{itemize}

\subsection{Modes of Operation}\label{subsec:Modes_of_Operation}
\begin{definition}[Mode of Operation]\label{def:Mode_of_Operation}
  
\end{definition}

\subsection{Advanced Encryption Scheme}\label{subsec:AES}
\textbf{TODO}
\begin{definition}[Advanced Encryption Scheme]\label{def:AES}
  \textbf{TODO}
\end{definition}

%%% Local Variables:
%%% mode: latex
%%% TeX-master: "../EDIN01-Cryptography-Reference_Sheet"
%%% End:
