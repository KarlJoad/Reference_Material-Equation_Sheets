\section{Cryptography Introduction}\label{sec:Intro_Cryptography}
\begin{definition}[Cryptographic Primitive]\label{def:Cryptographic_Primitive}
  A \emph{cryptographic primitive} is an algorithm with basic cryptographic properties.
\end{definition}

\begin{definition}[Cryptographic Protocol]\label{def:Cryptographic_Protocol}
  A \emph{cryptographic protocol} involves the back-and-forth communication among two or more parties.

  \begin{remark}[Bob and Alice]\label{rmk:Bob_and_Alice}
    Typically, the parties are named Bob and Alice.
    These are arbitrary names, but these are the most commonly used ones.
  \end{remark}
\end{definition}

There are have been several \nameref{def:Cryptographic_Protocol}s.
\begin{enumerate}[noitemsep]
\item \textit{Symmetric-key cryptography} - Methods in which both the sender and receiver share the same key
  \begin{enumerate}[noitemsep]
  \item Block ciphers
  \item Stream ciphers
  \item MAC algorithms
  \end{enumerate}
\item \textit{Public-key cryptography}: 2 different, but mathematically related keys are used.
  A public key and a private key.
  \begin{enumerate}[noitemsep]
  \item The public key cannot decrypt something that was encrypted with the private key.
  \item The public key can be shared freely, because the private key cannot be generated from the public key.
  \end{enumerate}
\item \textit{Cryptographic hash functions} are a related and important class of cryptographic algorithms.
  \begin{enumerate}[noitemsep]
  \item This is a keyless \nameref{def:Cryptographic_Primitive}.
  \item Takes an arbitrary length input and produces a fixed-length output.
  \item The mapping between the input and output is such that the output cannot generate the input, therefore making it cryptographic.
  \end{enumerate}
\end{enumerate}

\subsection{Historical Cryptography}\label{subsec:Historical_Cryptography}
Just to give a super quick background on how we've gotten to where we are today when it comes to cryptography.

\subsubsection{Monoalphabetic Ciphers}\label{subsubsec:Monoalphabetic_Ciphers}
\begin{definition}[Monoalphabetic Cipher]\label{def:Monoalphabetic_Cipher}
  In a \emph{monoalphabetic cipher} a single letter is replaced by the cipher's mapping.
  Since the cipher can do this to arbitrary letters, this could continue indefinitely for any single letter.

  These were some of the first ciphers developed by Man.
  These include simple substitute ciphers, and letter shifting ciphers.
  However, these can be broken with \emph{frequency analysis}.
\end{definition}

\subsubsection{Polyalphabetic Ciphers}\label{subsubsec:Polyalphabetic_Ciphers}
\begin{definition}[Polyalphabetic Cipher]\label{def:Polyalphabetic_Cipher}
  In a \emph{polyalphabetic cipher} multiple letters are replaced by the cipher's mapping.
  Additionally, since the cipher can output multiple letters, the ciphered letters could be run through the cipher again.

  These were developed in response to \nameref{def:Monoalphabetic_Cipher}s being broken.
  However, these can also be broken, with \emph{extended frequency analysis}.
\end{definition}

Eventually, it was realized that the secrecy of the cipher is not sensible/possible.
This leads us to the conclusion that \textbf{any cryptographic scheme should remain secure even if the adversary understands the cipher algorithm itself}.

\subsubsection{Cryptographic Keys}\label{subsubsec:Cryptographic_Keys}
The use of keys as ciphers is a slightly more modern occurrence.
\begin{definition}[Kerckhoff's Principle]\label{def:Kerckhoffs_Principle}
  \emph{Kerckhoff's Principle} states that the security of the key used should alone be sufficient for a good cipher to maintain confidentiality under an attack.
  Essentially, the security of the key used should be sufficient such that the cipher can be maintained confidently while under attack.
\end{definition}

However, only since the mid-1970s, has public key cryptography has been possible.

\begin{table}[h!]
  \centering
  \begin{tabular}{cc}
    \toprule
    Symmetric-Key Cryptography & Public-Key Cryptography \\
    \midrule
    Block ciphers & Public-Key encryption \\
    Stream ciphers & Digital Signature Schemes \\
    Cryptographic Hash Functions & Key exchange protocols \\
                               & Electronic Cash/Cryptocurrency \\
                               & Interactive Proof Systems \\
    \bottomrule
  \end{tabular}
  \caption{Uses of Key-Based Cryptography}
  \label{tab:Key_Cryptography_Uses}
\end{table}

Computers can efficiently encrypt, given the following constraints:
\begin{enumerate}[noitemsep]
\item Some modern techniques can only keep the keys secret if certain mathematical problems are \nameref{def:Intractable}.
  \begin{enumerate}[noitemsep]
  \item Integer factorization
  \item Discrete logarithm problems
  \end{enumerate}
\item However, there are no absolute proofs that a cryptographic technique is secure.
\end{enumerate}

\begin{definition}[Intractable]\label{def:Intractable}
  An \emph{intractable} problem is one in which there are no \textbf{efficient} algorithms to solve them.
\end{definition}

%%% Local Variables:
%%% mode: latex
%%% TeX-master: "../EDIN01-Cryptography-Reference_Sheet"
%%% End:
