\section{Pointers, References, and Arrays}\label{sec:Pointers}


\subsection{}
Concepts: pointers and references

\begin{itemize}
\item Pointer:
  \begin{itemize}[noitemsep]
  \item Points to a location in memory
  \item Can be dereferenced to get actual value in memory
  \item Pointers can be nested
  \item Pointers can have arithmetic done on them to move them around arbitrarily
  \end{itemize}

\item Reference:
  \begin{itemize}[noitemsep]
  \item Refers to an object or value in memory
  \item Can be dereferenced to get the actual value
  \item References cannot be moved by pointer arithmetic
  \end{itemize}
\end{itemize}

\subsection{}
Concept: the dangling pointer problem

\begin{itemize}[noitemsep]
\item If there are 2 pointers, but one of them deallocates the memory from the heap, the other one is pointing to garbage.
\item This is the dangling pointer problem.
\end{itemize}

\begin{itemize}[noitemsep]
\item If a pointer goes out of scope and is deallocated, but the memory isn't, that's a memory leak.
\end{itemize}

\subsection{}
Concepts: garbage collection in the forms of reference counting and mark-and-sweep collection

\begin{itemize}
\item This only happens for implicitly heap-allocated programming languages.
\item Reference Counting:
  \begin{itemize}[noitemsep]
  \item The runtime system keeps count of the number of references to a thing in memory
  \item When the number of references increases, nothing happens.
  \item When the count gets to 0, the system will deallocate the memory.
  \item This doesn't work for cyclical structures.
  \item While incremental, it also introduces a lot of operations between any actual programmer calculations.
  \end{itemize}

\item Mark-and-Sweep:
  \begin{itemize}[noitemsep]
  \item The runtime system will pause program execution occassionally and perform this algorithm.
    \begin{itemize}[noitemsep]
    \item Consider all memory invalid/not-in-use.
    \item Mark all memory in static memory and on the stack as valid.
    \item Go through pointers on the stack and in static memory and follow them.
    \item Recursively descend through the pointers, marking that memory as valid.
    \item Once the entire memory segment has been visited, anything not marked as valid is deallocated.
    \end{itemize}
  \item This is done less frequently, but it uses up a lot more system resources.
  \item It also handles deallocation of circular structures that may still be on the heap because the structure isn't reachable from anywhere else.
  \end{itemize}
\end{itemize}

\subsection{}
Concept: arrays

\begin{itemize}[noitemsep]
\item Arrays are continuous blocks of memory
\item The size is calculated based on the number of elements expected to be stored and the size of the elements.
\item These are quick to access because the index of any element is a fixed distance away from the start of the array.
\item The head of the array, the thing that is at index 0, is actually a pointer to the memory of index 0.
\item The \texttt{array[index]} syntax is just a nice way to present \texttt{*(array + (index * sizeOfElement))}
\end{itemize}

\subsubsection{}
static arrays

\begin{itemize}[noitemsep]
\item The array is statically sized
\item The storage is statically bound
\item Depending on the compiler, language, and memory setup, the values in storage could be bound statically too
\item They can only hold one type, to allow for indexing.
\item This exists inside the static memory of the program
\item Like static variables, these exist for the duration of a program's execution.
\end{itemize}

\subsubsection{}
fixed stack-dynamic arrays

\begin{itemize}[noitemsep]
\item These are the usual arrays (relatively small-sized)
\item Placed on the stack in memory
\item The memory storage is allocated dynamically, during runtime
\item The index range is bound statically, meaning they are of fixed size
\item They can only hold one type, to allow for indexing.
\end{itemize}

\subsubsection{}
fixed heap-dynamic arrays

\begin{itemize}[noitemsep]
\item These are the usual arrays (relatively large-sized)
\item Placed on the heap in memory
\item The memory storage is allocated dynamically, during runtime
\item The index range is bound statically, meaning they are of fixed size
\item They can only hold one type, to allow for indexing.
\end{itemize}

\subsubsection{}
heap-dynamic arrays

\begin{itemize}[noitemsep]
\item These are less commonly found
\item Placed on the heap in memory
\item The memory storage is allocated dynamically
\item The index range is also bound dynamically, meaning they can be of any size
\item They can only hold one type, to allow for indexing.
\end{itemize}

%%% Local Variables:
%%% mode: latex
%%% TeX-master: "../EDAP05-Concepts_Programming_Languages-Skills"
%%% End:
