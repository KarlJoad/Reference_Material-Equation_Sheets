\section{Natural Semantics}\label{sec:Natural_Semantics}
There are many ways to describe semantics.
In this course, we focus on natural semantics (also known as Big-Step Operational Semantics).

\begin{enumerate}[noitemsep]
\item You should be able to read a specification of natural semantics.
\item You should be able to undestand how \textbf{expressions} and \textbf{values} relate to each other.
\item Given a natural semantics and a parse tree (or unambiguous expression), you should be able to compute the semantics of the given program.
\item Given a natural semantics and a BNF grammar, you should be able to tell whether any parts of the semantics are undefined.
\item Given an understanding of what a state-free expression language is supposed to do, you should be able to write down a simple natural semantics for it.
\item You should be able to understand the concepts of Environments in the context of natural semantics and be able to utilise it when reading and reasoning about natural semantics.
\end{enumerate}

%%% Local Variables:
%%% mode: latex
%%% TeX-master: "../EDAP05-Concepts_Programming_Languages-Skills"
%%% End:
