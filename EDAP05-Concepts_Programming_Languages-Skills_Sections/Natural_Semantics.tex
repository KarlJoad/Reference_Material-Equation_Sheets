\section{Natural Semantics}\label{sec:Natural_Semantics}
There are many ways to describe semantics.
In this course, we focus on natural semantics (also known as Big-Step Operational Semantics).

\subsection{}
You should be able to read a specification of natural semantics.

\subsection{}
You should be able to undestand how \textbf{expressions} and \textbf{values} relate to each other.

\begin{enumerate}[noitemsep]
\item An expression is built up of values.
\item An expression can return a value or another expression.
\item Values must be defined in terms of some mathematical numerical thing. If $\mathtt{v}$ is input as an integer, then $v \in \mathbb{Z}$ is the conditional requirement for $\mathtt{v} \Downarrow v$.
\end{enumerate}

\subsection{}
Given a natural semantics and a parse tree (or unambiguous expression), you should be able to compute the semantics of the given program.

\begin{itemize}[noitemsep]
\item You must use the semantic rules provided, and repreatedly recurse until the semantics of the entire program has been defined.
\end{itemize}

\subsection{}
Given a natural semantics and a BNF grammar, you should be able to tell whether any parts of the semantics are undefined.

\begin{itemize}[noitemsep]
\item The BNF grammar defines what symbols may be used. So if the semantics uses symbols that the BNF does not defined, the semantics are undefined.
\item If there are operator symbols defined, but the semantics does not use them, then the semantics are undefined.
\item If an operator's semantics do not recurse enough to handle all cases, then they are undefined.
\end{itemize}

\subsection{}
Given an understanding of what a state-free expression language is supposed to do, you should be able to write down a simple natural semantics for it.

\begin{itemize}[noitemsep]
\item Given a basic state-free expression, like an expression $1 + x$ where $x \in \mathbb{Z}$
\item Create the necessary rules for such an expression to be able to exist
\end{itemize}

\subsection{}
You should be able to understand the concepts of Environments in the context of natural semantics and be able to utilise it when reading and reasoning about natural semantics.

\begin{itemize}[noitemsep]
\item These behave like a map, where if a meta-language's variable (meta-variable) has been defined, then it stores a value.
\item They are updated like this: $E[c \mapsto 2]$
\item They are read (Meta-variable values are retrieved) from like this: $E(c) \Rightarrow 2$
\item The empty environment is $E_{\emptyset} = \lbrace \rbrace$
\end{itemize}

%%% Local Variables:
%%% mode: latex
%%% TeX-master: "../EDAP05-Concepts_Programming_Languages-Skills"
%%% End:
