\section{Functional Programming}\label{sec:Functional_Programming}

\subsection{}
Concept: local let bindings of variables

\begin{itemize}[noitemsep]
\item This allows declarations to occur between the \texttt{let} and \texttt{in} keywords.
\item These definitions/declarations will shadow any that exist outside of the \texttt{let}-expression.
\item Between the \texttt{in} and \texttt{end} keywords are expressions.
\item Any expression that uses a  declaration from inside the first portion will use that value.
\item After the end of the \texttt{let}-expression, any information from the \texttt{let} portion of the \texttt{let}-expression is discarded.
\item The \texttt{Let}-expression returns an expression, which may be used inside another \texttt{let}-expression, or used in another expression of any other type.
\end{itemize}

\subsection{}
Concept: anonymous functions, also known as lambda expressions

\begin{itemize}[noitemsep]
\item Since functions are expressions that map input types to output types, they don't necessarily need names.
\item If a function does not have a name, they are anonymous functions, or lambda expressions.
\item They \textbf{can} take parameters and use them.
\item They are subject to the same type checking that all other functions get.
\end{itemize}

\subsection{}
Concept: first-class functions

\begin{itemize}[noitemsep]
\item Functions are treated just like data.
\item Meaning functions can be assigned to variables (every function is anonymous until it is given a name to bind to).
\item Functions can be passed into other functions, and returned from other functions, just like traditional data.
\end{itemize}

\subsection{}
Concepts: pattern matching, including wildcards

\begin{itemize}[noitemsep]
\item This is closely related to algebraic data types.
\item You can pattern match on tuples too.
\end{itemize}

\subsection{}
Concepts: exhaustiveness and redundancy in pattern matching

\begin{itemize}
\item Exhaustiveness
  \begin{itemize}[noitemsep]
  \item The pattern match must cover all cases to be exhaustive.
  \item There is a wildcard operator \texttt{\textunderscore{}} that matches anything, but doesn't preserve its information, meaning you can't use it.
  \end{itemize}

\item Redundancy
  \begin{itemize}[noitemsep]
  \item When multiple patterns match the same thing.
  \item Cannot have this, otherwise the pattern matching won't know which to choose.
  \item It might choose the first pattern that it matches with instead.
  \end{itemize}
\end{itemize}

\subsection{}
Concept: lists as algebraic datatypes

\begin{itemize}[noitemsep]
\item Think of these like BNF lists.
\item There is either a ``null'', or a node with another list after it
\item You can pattern match on this using \texttt{head::list} in SML.
\item After this, the name \texttt{head} will refer to the first element in the original list.
\item The \texttt{list} name will refer to the rest of the list after \texttt{head}.
\end{itemize}

\subsection{}
Concept: the \texttt{map} function

\begin{itemize}[noitemsep]
\item In SML \texttt{fn : ('a -> 'b) -> 'a list -> 'b list}
\item This function takes a function, which performs some action on elements of a list, which returns another function that takes in a list and returns a new list where the function was applied to all element of the original list.
\end{itemize}

\subsection{}
Concept: currying

\begin{itemize}[noitemsep]
\item Like in the \texttt{map} function, the act of taking in parameters that will return functions.
\item This allows for great generalization in the language.
\item However, the tradeoff to make is the programmer must determine the first things to be evaluated to create the function.
\item If the first things to evaluate are chosen correctly, large performance benefits can ensue.
\item It also allows for functions to be created more generally and slowly become more specfic, as needed.
\end{itemize}

For example, \texttt{map'} is a non-curried version.
\texttt{map'} takes in a function and a list \textbf{AT THE SAME TIME}!
\texttt{map} is a curried version.
\texttt{map} takes in a function, and returns a function that takes in a list and returns a list.

\begin{minted}[frame=lines,linenos]{sml}
fun map' (f, nil) = nil
  | map' (f, h::t) = (f h) :: map' (f,t);
(* val map' = fn : ('a -> 'b) * 'a list -> 'b list *)
(*      NOTE THE ASTERISK HERE^                    *)

fun map f nil = nil
  | map f (h::t) = (f h) :: (map f t);       
(* val map = fn : ('a -> 'b) -> 'a list -> 'b list *)
(*        NOTE THE ARROW HERE^^                    *)
\end{minted}

\subsection{}
Functional Programming in Standard ML

%%% Local Variables:
%%% mode: latex
%%% TeX-master: "../EDAP05-Concepts_Programming_Languages-Skills"
%%% End:
