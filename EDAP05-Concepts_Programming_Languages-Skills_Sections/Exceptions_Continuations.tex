\section{Exceptions and Continuations}\label{sec:Exceptions_Continuations}

\subsection{}
Concepts: exceptions and exception handlers

\begin{itemize}
\item Exceptions:
  \begin{itemize}[noitemsep]
  \item In SML, exceptions are raised with \texttt{raise \textit{ExceptionName}}
  \item Where \textit{ExceptionName} must be defined by \texttt{exception \textit{ExceptionName}}
  \item Checked Exceptions:
    \begin{itemize}[noitemsep]
    \item These exceptions \textbf{MUST} be handled, or else during static evaluation (compilation) the process stops.
    \item For example, a file not found exception MUST be handled before compilation is allowed.
    \end{itemize}

  \item Unchecked Exceptions:
    \begin{itemize}[noitemsep]
    \item These exceptions are not checked during static evaluation (compilation).
    \item They are only found during runtime.
    \item For example, division by 0 in some languages throws a divide by zero exception.
    \item The compiler won't check to see if the error isn't handled. If it isn't, then the program may prematurely terminate execution.
    \end{itemize}
  \end{itemize}

\item Exception Handlers:
  \begin{itemize}[noitemsep]
  \item In SML, \texttt{\textit{expr} handle \textit{match}}, where \textit{expr} can raise an exception.
  \item The match is also pattern matched.
  \item If an exception goes unhandled, it continues up the stack until it reaches a handler that correctly handles the exception.
  \end{itemize}
\end{itemize}

%%% Local Variables:
%%% mode: latex
%%% TeX-master: "../EDAP05-Concepts_Programming_Languages-Skills"
%%% End:
