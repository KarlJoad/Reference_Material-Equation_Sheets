\section{Statements and Control Structures}\label{sec:Statements_Control_Structures}

\subsection{}
Concept: assignment statements, including compound assignment

\begin{itemize}
\item Assignment:
  \begin{itemize}[noitemsep]
  \item The regular assignment.
  \item Use the storage binding that was set aside and perform a value binding.
  \end{itemize}

\item Compound Assignment:
  \begin{itemize}[noitemsep]
  \item Short-hand to specify an operation on a value and reassign that new value to the variable with the same name used.
  \item \texttt{count += 1}
  \end{itemize}
\end{itemize}

\subsection{}
Concept: two-way selection statements

\begin{itemize}[noitemsep]
\item Standard \texttt{if} statement.
\item If the condition is true, take one path, else take the other path.
\item When we reach the end of either path, they come back together.
\end{itemize}

\subsection{}
Concept: multiple-selection statements

\begin{itemize}[noitemsep]
\item Standard \texttt{switch} statement or \texttt{if-elseif-else} statement
\item Based on the condition, choose one path of many.
\item If the first condition is not met, then choose another.
\item The different paths usually need to be broken up by a \texttt{break} statement for them to be completely separate.
\item Without the \texttt{break}, one path might run into another, because the condition is only evaluated once and then checked in a \texttt{switch} statement.
\item In an \texttt{if-elseif-else} statement, the condition is evaluated as many times as there are \texttt{if} and \texttt{elseif} branches.
\item At the end of all paths, they come back together.
\end{itemize}

\subsection{}
Concept: counter-controlled loops

\begin{itemize}[noitemsep]
\item \texttt{while}-loops can be used for these, where their condition checks if the count is correct.
\item \texttt{for}-loops are a construction that eases the writing of counter-controlled loops
\item The counter can increment at the beginning or at the end of the loop.
  \begin{itemize}[noitemsep]
  \item \texttt{for}-loops have this done at the end.
  \item \texttt{while}-loops can have it done wherever the programmer specifies, so long as it is done.
  \item If a programmer forgets to manage the counter in a \texttt{while}-loop, an infinite loops occurs.
  \end{itemize}
\item Counter-controlled loops are special logic controlled loop, where the condition is checking if a counter has met some threshold.
\end{itemize}

\subsection{}
Concept: logically controlled loops

\begin{itemize}[noitemsep]
\item \texttt{while}-loops are used for these.
\item The condition is evaluated before entering the loop, and evaluated after each iteration through the loop.
\item These can be used to create an infinite loop while waiting for some stateful change, like keyboard input, or a network packet.
\item The condition must evaluate to a boolean value, in old versions of C either 0 or 1.
\end{itemize}

\subsection{}
Concept: data structure-controlled loops

\begin{itemize}[noitemsep]
\item These are colloquially called \texttt{for-each}-loops.
\item They are used on more complex data structures, like vectors, lists, etc.
\item They return one item on each iteration of the loop.
\item The loop ends when the data structure has iterated over each element once.
\end{itemize}

%%% Local Variables:
%%% mode: latex
%%% TeX-master: "../EDAP05-Concepts_Programming_Languages-Skills"
%%% End:
