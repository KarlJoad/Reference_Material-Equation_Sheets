\section{Bindings and Lifetimes}\label{sec:Bindings_Lifetimes}

\subsection{}
You should understand the difference between static and dynamic type binding and be able to take advantage of either property in your programming.

\begin{itemize}
\item Static Type Binding
  \begin{itemize}[noitemsep]
  \item C-Family Languages
  \item Once things have been given a type, they cannot change their type for their lifetime.
  \item Once something has reached the end of its lifetime, it is deallocated.
    \begin{itemize}[noitemsep]
    \item The name and storage bindings can be reused.
    \end{itemize}
  \end{itemize}

\item Dynamic Type Binding
  \begin{itemize}[noitemsep]
  \item Python, JavaScript
  \item Throughout a single thing's lifetime, its type can change as many times as the programmer wants. Only subsequent actions will require the binding to be of a certain type.
  \end{itemize}
\end{itemize}

\subsection{}
Given a syntax, a compiler and a run-time system for a language, you should be able to determine whether the language is using static or dynamic type binding.

\begin{itemize}
\item Static
  \begin{itemize}[noitemsep]
  \item You try to redefine the type binding for a variable.
  \item For instance, you declare \texttt{x} to be an integer, then later declare \texttt{x} to be a string.
  \item Java/Rust/SML/Haskell, for examples.
  \item In a strongly-typed language, this would be a (before runtime) compiler error.
  \item In a weakly-typed language, this error would occur during runtime, and may actually be passed over completely by giving default values or storing the error in the place of the variable's result.
  \end{itemize}

\item Dynamic
  \begin{itemize}[noitemsep]
  \item You try to redeclare something that is of one type to another and the language allows that.
  \item Python, for example.
  \item For instance, you declare \texttt{x} to be of type integer, then later declare \texttt{x} to be of type string.
  \item The language allows this, and any subsequent operations happens with the new type binding.
  \item In a strongly-typed language, this would be allowed, but invalid operations would be found during (before runtime) compilation stage.
  \item In a weakly-typed language, this would be allowed, but may ``fly under the radar'' until you try to use the results of an erroneous operation.
  \end{itemize}
\end{itemize}

\subsection{}
Concepts: static binding, stack-dynamic binding, explicit heap-dynamic binding, and implicit heap-dynamic binding.
\begin{itemize}
\item Static
  \begin{itemize}[noitemsep]
  \item \textbf{Global} and \textbf{Constant} variable bindings are put here.
  \item \texttt{const int x = 4;}
  \item The storage binding lasts as long as the program is in execution.
  \item These can be stored in the actual program binary when it is compiled and saved to disk, because they are unchanging.
  \end{itemize}

\item Stack-Dynamic
  \begin{itemize}[noitemsep]
  \item The typical variable bindings go here.
  \item \texttt{int x = 4;}
  \item They are usable within the scope they are declared in.
  \item Their lifetime is the same as their scope.
  \end{itemize}

\item Explicit Heap-Dynamic
  \begin{itemize}[noitemsep]
  \item The programmer must \textbf{EXPLICITLY} allocate memory from the heap for the storage binding.
  \item The programmer must \textbf{EXPLICITLY} deallocate memory from the heap when the binding reaches its end of life.
  \item The lifetime of the variable is determined by the programmer.
  \item \texttt{int* x = malloc(); *x = 4; free(x);}
  \item The lifetime of an explicit heap-dynamic variable storage binding is separate from its scope.
  \end{itemize}

\item Implicit Heap-Dynamic
  \begin{itemize}[noitemsep]
  \item The compiler/runtime system determines the lifetime of the binding.
  \item The compiler/runtime system handles the allocation of memory and deallocation of memory.
  \item The compiler/runtime creates a garbage collection to ensure things are deallocated when their lifetime is over.
  \item The lifetime of an explicit heap-dynamic variable storage binding is related to its scope, for the most part, they are the same.
  \end{itemize}
\end{itemize}

\subsection{}
Given a syntax, a compiler and a run-time system for a language, you should be able to determine which storage binding(s) the language is using.

\begin{itemize}
\item Implicit vs. Explicit Heap-Dynamic:
  \begin{itemize}[noitemsep]
  \item There needs to be a keyword/reserved word for explicit heap operations. If none present, not explicit.
  \item The explicit keyword/reserved word may be given in the grammar, or it is a core function built into the language.
  \item If none of that, implicit.
  \end{itemize}

\item Stack-Dynamic vs. Static:
  \begin{itemize}[noitemsep]
  \item Static bindings are done \emph{statically}, and last the \emph{life of the \textbf{program}}.
  \item If the static value is modified INSIDE a subprogram, it is updated \textbf{throughout the rest of the program's} execution.
  \item Thus, static bindings do not allow for recursion to happen with the static binding's value.
  \item If the stack-dynamic value is modified inside a subprogram, and this value is not visibly global, the value is not modified.
  \end{itemize}

\item Stack-Dynamic vs. Heap-Dynamic:
  \begin{itemize}[noitemsep]
  \item The difference here is where they are stored.
  \item This can be hard to tell, you would need to be able to see the memory in which the program is running.
  \item Larger things (arrays) are usually stored on the Heap, while smaller things (integers) are more likely to be stored on the stack.
  \item However, the previous rule is \textbf{INCREDIBLY} flexible and is not a certainty. (It also depends on language implementation, to a degree.)
  \item If there is a way to explicitly allocate memory on the heap, then you can find out much more easily.
  \end{itemize}
\end{itemize}

\subsection{}
Concept: lifetime of a variable.

\begin{itemize}
\item The lifetime of a variable is the period of time which it has a value binding.
\item This is distinct from being in-scope.
  \begin{itemize}[noitemsep]
  \item In-scope, out of lifetime: An explicit heap-dynamic storage binding, the memory could be freed too early and the value that was there is going to be used later.
  \item Out-of-Scope, in lifetime: A stack-dynamic variable is used in a program, but is not passed to the subprogram. It is out-of-scope of the subprogram, but still alive in the parent program.
  \end{itemize}
\item For stack-dynamic storage bindings, their lifetime and scope are usually the same, but can be different (as seen above).
\item For static storage bindings, they are alive all the time, but may be out-of-scope or shadowed by other bindings with the same name.
\end{itemize}

\subsection{}
Concepts: allocation and deallocation of a heap-dynamic variable.

\begin{itemize}
\item Allocation
  \begin{itemize}[noitemsep]
  \item If implicit, programmer doesn't need to worry about allocation, at worst has to write a constructor.
  \item If explicit,
    \begin{itemize}[noitemsep]
    \item Programmer must tell compiler/runtime system \textbf{how much} memory to allocate.
    \item Programmer must keep track of the pointer/reference that points to that memory.
    \item Programmer must make sure that the pointer is used correctly (including pointer arithmetic)
    \end{itemize}
  \end{itemize}

\item Deallocation
  \begin{itemize}[noitemsep]
  \item If implicit, programmer doesn't need to worry about deallocation, at worst, has to write a destructor. Lets the garbage collector handle it.
    \begin{itemize}[noitemsep]
    \item Many garbage collector algorithms.
      \begin{itemize}[noitemsep]
      \item Reference counter
        \begin{enumerate}[noitemsep]
        \item Keep count of how many pointers point to things in memory.
        \item If $count \leq 0$, deallocate.
        \item Cannot handle linked lists or circular structures.
        \end{enumerate}
      \item Mark-and-sweep
        \begin{enumerate}[noitemsep]
        \item Start by identifing all memory as safe (includes stack-dynamic and static variables).
        \item Follow all pointers to heap and traverse the heap, marking everything for deletion.
        \item Once done iterating over pointers and recursively descending, delete everything marked
        \end{enumerate}
      \end{itemize}
    \end{itemize}
  \item If explicit,
    \begin{itemize}[noitemsep]
    \item Programmer must tell compiler/runtime system to deallocate the memory at some specified time.
    \item Most of these errors are only found during runtime (Rust has special capabilities to figure this out),
      \begin{itemize}[noitemsep]
      \item If programmer forgets $\rightarrow$ \textbf{memory-leak}.
      \item If programmer deallocates twice $\rightarrow$ \textbf{double-free}.
      \item If programmer deallocates, but still uses it $\rightarrow$ \textbf{corrupted memory}.
      \end{itemize}
    \end{itemize}
  \end{itemize}
\end{itemize}

\subsection{}
Concept: binding time, especially the difference between static and dynamic binding times.

\begin{itemize}
\item Static Binding Time
  \begin{itemize}[noitemsep]
  \item These are things that are bound when the program is compiled (before it is run).
  \item Static storage bindings are done statically, during compilation.
  \item That goes along with their names too.
  \item The binding of names and types to functions is static.
  \end{itemize}

\item Dynamic Binding Time
  \begin{itemize}[noitemsep]
  \item The binding of storage is dynamic for all categories except static.
  \item In a dynamically typed language, the type bindings for variables is dynamic.
  \item In an OOP language, you can dynamically type objects if you declare them with a supertype.
  \end{itemize}
\end{itemize}

%%% Local Variables:
%%% mode: latex
%%% TeX-master: "../EDAP05-Concepts_Programming_Languages-Skills"
%%% End:
