\section{Syntax}\label{sec:Syntax}
Syntax describes the possible structure (or form) of programs of a given programming language.
Backus-Naur Form (BNF) grammars have emerged as the standard mechanism for describing language syntax.
BNF grammars used to describe languages when communicating with language adopters and compiler implementors.
There are also many tools (particularly the yacc and antlr families of programs) for automatically generating parsers, programs that recognise whether an input program matches a grammar and, if it does, execute user-defined actions upon encountering certain language constructs.

\begin{enumerate}[noitemsep]
\item You should be able to determine whether a given property of a language is part of the syntax, of the static semantics, or of the dynamic semantics.
\item You should be able to read a BNF grammar and understand the difference between terminals and nonterminals.
\item Given a BNF grammar, you should be able to write down examples of programs that can be generated by the grammar.
\item Given a BNF grammar, you should be able to tell whether a given program can be generated by the grammar. If the program is generated by the grammar, you should also be able to generate a parse tree for the program.
\item You should be able to determine whether a given (small) BNF grammar is ambiguous (the problem is undecidable in general, so this skill only pertains to practically relevant examples as covered in the textbook).
\item Given a BNF grammar, you should be able to determine the associativity of any operator used therein.
\item You should be able to describe the difference between an object language and a meta-language.
\item Understand \textbf{arity}, \textbf{fixity}, and \textbf{precedence}, and \textbf{associativity} of operators
\end{enumerate}

%%% Local Variables:
%%% mode: latex
%%% TeX-master: "../EDAP05-Concepts_Programming_Languages-Skills"
%%% End:
