\section{Language Implementation}\label{sec:Language_Implementation}
You should be familiar with the following concepts:
\subsection{}
Language implementation via \textbf{pure interpretation}, \textbf{compilation}, and \textbf{hybrid implementation}.
\begin{itemize}
\item Pure Interpretation
  \begin{itemize}[noitemsep]
  \item Python
  \item The program runs on an interpretation layer between the hardware and itself. (VM usually)
  \item Quicker development with faster feedback on issues
  \item No compilation (May be possible to compile down to VM-specific code, but not machine code.)
  \item Slower execution time (Must interpret each command, break down to possible CPU actions, and write the machine code on-the-fly.)
  \end{itemize}

\item Compilation
  \begin{itemize}[noitemsep]
  \item C
  \item Program runs ON the computer's hardware
  \item Slower development, must compile for some issues to be found (Type Error). Must run for other issues to be found (Null pointer).
  \item Compile down to machine code for quick runtime.
  \item Fast execution. Human-readable code already in machine code, just need to load into memory and begin execution.
  \end{itemize}

\item Hybrid Implementation
  \begin{itemize}[noitemsep]
  \item Java
  \item Program may start by being interpreted, but parts in use will start to be compiled while running.
  \item Runs some parts through an interpreter, and some pieces are compiled.
  \item Theoretically, same kind of fast development as interpreted languages, with the more frequently-used code being compiled for speed.
  \item Hypothetically, Fast execution for the parts that are compiled.
  \end{itemize}
\end{itemize}

\subsection{}
Language implementation with the help of a \textbf{just-in-time compiler}.

\begin{itemize}[noitemsep]
\item A Just-In-Time (JIT) compiler starts by having the code run in an interpreted mode.
\item After some time, the most frequently used portions of code are compiled, to improve performance, and the interpreted parts are replaced by the compiled parts.
\item Java uses this.
\end{itemize}

\subsection{}
The concept of language \textbf{run-time system}.

\begin{itemize}[noitemsep]
\item These are the systems in place that were put in by the compiler/interpreter to help the program run.
\item These can include: Error handlers, resource deallocators (Garbage collectors), and type coercers.
\end{itemize}

%%% Local Variables:
%%% mode: latex
%%% TeX-master: "../EDAP05-Concepts_Programming_Languages-Skills"
%%% End:
