\subsection{Power Series of Rational Functions}\label{subsec:Complex_Power_Series-Rational_Functions}
The key functions to know a \nameref{def:Complex_Power_Series} for are $\frac{1}{1-z}$ and $\frac{1}{1+z}$.
\begin{equation}\label{eq:Complex_Power_Series-1-z}
  \begin{aligned}
    \frac{1}{1-z} &= \sum_{n=0}^{\infty} z^{n} \\
    &= 1 + z + z^{2} + z^{3} + \cdots \, , \Modulus{z} < 1
  \end{aligned}
\end{equation}

\begin{equation}\label{eq:Complex_Power_Series-1+z}
  \begin{aligned}
    \frac{1}{1+z} &= \sum_{n=0}^{\infty} z^{n} \\
    &= 1 - z + z^{2} - z^{3} + \cdots \, , \Modulus{z} < 1
  \end{aligned}
\end{equation}

\begin{example}[Lecture 9, Example 4]{Power Series of Rational Function 1}
  Given the rational function $f(z) = \frac{1}{1-2z^{3}}$ is \nameref{def:Analytic}, find its \nameref{def:Complex_Power_Series}?
  \tcblower{}
  We start be determining the \nameref{thm:Region_of_Convergence}.

  If in the standard format, we have $\frac{1}{1-z}$ where $\Modulus{z} < 1$, we can follow that format.
  \begin{align*}
    \Modulus{2z^{3}} &< 1 \\
    \shortintertext{Property of Modulus, Constants can Factor out.}
    2 \Modulus{z^{3}} &< 1 \\
    \shortintertext{Use the Law of Moduli.}
    {\Modulus{z}}^{3} &< \frac{1}{2} \\
    \Modulus{z} &< \frac{1}{\sqrt[3]{2}}
  \end{align*}

  Now that we have our \nameref{thm:Region_of_Convergence}, which is a circle with center $a=0$ and $r = \frac{1}{\sqrt[3]{2}}$, there \textbf{MUST} be a \nameref{def:Complex_Power_Series} of $f(z)$.
  \begin{align*}
    \Modulus{z} &\Rightarrow \sum_{n=0}^{\infty} z^{n} \\
    \Modulus{2z^{3}} &\Rightarrow \sum_{n=0}^{\infty} {(2z^{3})}^{n} \\
                &= \sum_{n=0}^{\infty} 2^{n} z^{3n}
  \end{align*}
\end{example}

\begin{example}[Lecture 9, Example 4]{Power Series of Rational Function 2}
  Consider $z \to \frac{1}{2+3z}$ as a sum of \nameref{def:Complex_Power_Series} of a circle with center $z = 2 + 5i$.
  Find the power series of this function?
  \tcblower{}
  The function is \nameref{def:Analytic} everywhere exception $z = \frac{-3}{2}$.
  Thus, the \nameref{thm:Region_of_Convergence} is the distance from the center $a$ to the hole $z = \frac{-3}{2}$.

  We start by taking our function, and plugging the left-hand side of the circle, the modulus portion, into this function for $z$.
  \begin{align*}
    \frac{1}{2 + 3z} &= \frac{1}{2 + 3 \bigl( z - (2+5i) \bigr) + 3 (2+5i)} \\
    \intertext{The extra $3(2+5i)$ was needed be cancel out the added $-3(2+5i)$.
    This way, we are only adding by 0, which is allowed by algebra.}
                     &= \frac{1}{8 + 15i + 3 \bigl( z - (2+5i) \bigr)} \\
    \intertext{Now, we force our constants to match our standard equation, \Cref{eq:Complex_Power_Series-1+z}, by factoring out the constants.}
                     &= \frac{1}{(8 + 15i) \left( 1 + \frac{3}{8+15i} \bigl( z - (2+5i) \bigr) \right)} \\
    \intertext{To make our lives easier, we will create a new variable that will represent the variable expression in the denominator.}
    w &= \frac{3 \bigl( z - (2+5i) \bigr)}{8+15i} \\
    \frac{1}{2 + 3z} &= \frac{1}{(8+15i) (1+w)} \\
                     &= \frac{1}{8+15i} \left( \frac{1}{1+w} \right) \\
    \intertext{Now, we can use \Cref{eq:Complex_Power_Series-1+z} as our template.}
    &= \frac{1}{8+15i} \sum_{n=0}^{\infty} {(-1)}^{n} w^{n}, \: \Modulus{w} < 1
  \end{align*}

  Now, we have 2 derivations left to perform:
  \begin{enumerate}[noitemsep]
  \item We need to substitute the series back into $z$, and simplify it.
  \item We need to solve for the \nameref{thm:Region_of_Convergence} by substituting our $z$ expression back in for $w$.
  \end{enumerate}

  I will start with the \nameref{def:Complex_Power_Series}, and then complete the \nameref{thm:Region_of_Convergence}.
  \begin{align*}
    \frac{1}{2 + 3z} &= \frac{1}{8+15i} \sum_{n=0}^{\infty} {(-1)}^{n} w^{n} \\
                     &= \frac{1}{8+15i} \sum_{n=0}^{\infty} {(-1)}^{n} {\left( \frac{3 \bigl( z - (2+5i) \bigr)}{8+15i} \right)}^{n} \\
    \shortintertext{Pull the exponent into all the sub-terms.}
                     &= \frac{1}{8+15i} \sum_{n=0}^{\infty} {(-1)}^{n} \left( \frac{3^{n} {\bigl( z - (2+5i) \bigr)}^{n}}{{(8+15i)}^{n}} \right) \\
                     &= \frac{1}{8+15i} \sum_{n=0}^{\infty} \left( \frac{{(-1)}^{n} 3^{n} }{{(8+15i)}^{n}} \right) {\bigl( z - (2+5i) \bigr)}^{n} \\
                     &= \frac{1}{8+15i} \sum_{n=0}^{\infty} {\left( \frac{(-1)\, 3}{(8+15i)} \right)}^{n} {\bigl( z - (2+5i) \bigr)}^{n} \\
  \end{align*}

  With the \nameref{def:Complex_Power_Series} found, we now focus on the \nameref{thm:Region_of_Convergence}.
  \begin{align*}
    \Modulus{w} &< 1 \\
    \Modulus{\frac{3 \bigl( z - (2+5i) \bigr)}{8+15i}} &< 1 \\
    \shortintertext{Factor out the constants.}
    \Modulus{\frac{3}{8+15i}} \Modulus{z - (2+5i)} &< 1 \\
    \Modulus{z - (2+5i)} &< \Modulus{\frac{8+15i}{3}} \\
    \shortintertext{Use the Law of Moduli to separate the moduli in the fraction.}
    \Modulus{z - (2+5i)} &< \frac{\Modulus{8+15i}}{\Modulus{3}} \\
                &< \frac{\sqrt{64 + 225}}{3} \\
    &< \frac{\sqrt{289}}{3}
  \end{align*}

  Thus, our solution is:
  \begin{equation*}
    \frac{1}{2 + 3z} = \frac{1}{8+15i} \sum_{n=0}^{\infty} {\left( \frac{(-1)\, 3}{(8+15i)} \right)}^{n} {\bigl( z - (2+5i) \bigr)}^{n} , \: \Modulus{z - (2+5i)} < \frac{\sqrt{289}}{3}
  \end{equation*}
\end{example}

%%% Local Variables:
%%% mode: latex
%%% TeX-master: "../../Math_333-MatrixAlg_ComplexVars-Reference_Sheet"
%%% End:
