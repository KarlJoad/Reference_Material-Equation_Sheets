\subsection{Power Series of Common Functions}\label{subsec:Common_Function_Power_Series}
All of the common transcendental functions below are \nameref{def:Analytic} on the entire complex place, $\Modulus{z} < \infty$.

\begin{equation}\label{eq:Complex_Power_Series-e}
  \begin{aligned}
    e^{z} &= \sum_{n=0}^{\infty} \frac{z^{n}}{n!} \\
    & = 1 + \frac{1}{1!} z + \frac{1}{2!} z + \frac{1}{3!} z^{3} + \cdots
  \end{aligned}
\end{equation}

\begin{equation}\label{eq:Complex_Power_Series-cos}
  \begin{aligned}
    \cos(z) &= \sum_{n=0}^{\infty} {(-1)}^{n} \frac{z^{2n}}{(2n)!} \\
    &= 1 - \frac{z^{2}}{2!} + \frac{z^{4}}{4!} - \frac{z^{6}}{6!} + \cdots
  \end{aligned}
\end{equation}

\begin{equation}\label{eq:Complex_Power_Series-sin}
  \begin{aligned}
    \sin(z) &= \sum_{n=0}^{\infty} {(-1)}^{n} \frac{z^{2n+1}}{(2n+1)!} \\
    &= z - \frac{z^{3}}{3!} + \frac{z^{5}}{5!} - \cdots
  \end{aligned}
\end{equation}

Now, some examples on how to use these series to solve problems.
\begin{example}[Lecture 9 Example 1]{Exponential not Centered on Origin}
  The mapping $z \to e^{z}$ is \nameref{def:Analytic} on $\Modulus{z-2} < \infty$.
  Determine its \nameref{def:Complex_Power_Series}?
  \tcblower{}
  We start by remembering that we already have a general power series for $e^{z}$, but this one is centered on $z = 0$.
  \begin{equation*}
    e^{z} = \sum_{n=0}^{\infty} \frac{z^{n}}{n!}
  \end{equation*}

  We can try just replacing $e^{z}$ weith $e^{z-2}$.
  \begin{align*}
    e^{z-2} &= \sum_{n=0}^{\infty} \frac{{(z-2)}^{n}}{n!} \\
    \intertext{The series above is technically correct, but not in the form we want.}
    e^{z} e^{-2} &= \sum_{n=0}^{\infty} \frac{{(z-2)}^{n}}{n!} \\
            &= e^{2} \sum_{n=0}^{\infty} \frac{{(z-2)}^{n}}{n!} \\
            &= \sum_{n=0}^{\infty} \frac{e^{2}}{n!} {(z-2)}^{n} \\
  \end{align*}
\end{example}

\begin{example}[Lecture 9, Example 2]{Power Series of Sinusoid with Angle Squared}
  Find the \nameref{def:Complex_Power_Series} for $z \to \sin(z^{2})$ for $\Modulus{z} < \infty$?
  \tcblower{}
  We already know what $\sin(z)$ is on $\Modulus{z} < \infty$, so we start there.
  \begin{equation*}
    \sin(z) = \sum_{n=0}^{\infty} {(-1)}^{n} \frac{z^{2n+1}}{(2n+1)!}
  \end{equation*}

  Now, we need to modify it to handle $z^{2}$.
  We can just replace all instances of $z$ with $z^{2}$.
  \begin{align*}
    \sin(z^{2}) &= \sum_{n=0}^{\infty} {(-1)}^{n} \frac{{\left( z^{2} \right)}^{2n+1}}{(2n+1)!} \\
                &= \sum_{n=0}^{\infty} {(-1)}^{n} \frac{z^{4n+2}}{(2n+1)!} \\
  \end{align*}
\end{example}

\begin{example}[Lecture 9, Example 3]{Power Series of Sinusoid not Centered on Origin}
  $z \to \sin(z)$ is \nameref{def:Analytic} on $\Modulus{z-2} < \infty$.
  Find the \nameref{def:Complex_Power_Series} of this function?
  \tcblower{}
  We start by taking what we already know, $\sin(z)$ on $\Modulus{z} < \infty$.
  \begin{equation*}
    \sin(z) = \sum_{n=0}^{\infty} {(-1)}^{n} \frac{z^{2n+1}}{(2n+1)!}
  \end{equation*}

  Now, we attempt to replace all instances of $z$ with $z-2$.
  \begin{align*}
    \sin(z) &= \sum_{n=0}^{\infty} {(-1)}^{n} \frac{z^{2n+1}}{(2n+1)!} \\
    \shortintertext{This is if $\sin$ were centered on $z=0$ though.}
            &= \sum_{n=0}^{\infty} {(-1)}^{n} \frac{{(z-2)}^{2n+1}}{(2n+1)!}
  \end{align*}

  But, that doesn't work, as we have no reasonable way to put this into our ``standard'' \nameref{def:Complex_Power_Series} form.
  Instead, start from the beginning again and say that $\sin(z) = \sin(z - 2 + 2)$.
  Now, we treat $\sin(z-2+2)$ as $\sin \bigl( (z-2) + 2 \bigr)$.
  We use the \nameref{subsec:Angle Sum and Difference Identities} to simplify this, namely \Cref{eq:Sin Angle Sum and Difference}.
  \begin{align*}
    \sin \bigl( (z-2) + 2 \bigr) &= \sin(z-2) \cos(2) + \cos(z-2) \sin(2) \\
    \shortintertext{Now we expand the sinusoids that have $z$ terms in them.}
                                 &= \cos(2) \left( \sum_{n=0}^{\infty} {(-1)}^{n} \frac{{(z-2)}^{2n+1}}{(2n+1)!} \right) + \sin(2) \left( \sum_{n=0}^{\infty} {(-1)}^{n} \frac{{(z-2)}^{2n}}{(2n)!} \right) \\
  \end{align*}

  Again, this is not in our ``standard'' form, but we can change it so it is, using 2 cases.
  \begin{equation*}
    \sin \bigl( (z-2) + 2 \bigr) = \sum_{n=0}^{\infty} a_{n} {(z-2)}^{n}
  \end{equation*}
  Where
  \begin{equation*}
    a_{n} =
    \begin{cases}
      \cos(2) {(-1)}^{\frac{n-1}{2}} \frac{{(z-2)}^{n}}{n!} & \text{If $n$ is odd} \\[5pt]

      \sin(2) {(-1)}^{\frac{n}{2}} \frac{{(z-2)}^{n}}{n!} & \text{If $n$ is even} \\
    \end{cases}
  \end{equation*}
\end{example}

One more useful series that we frequently use is the \nameref{def:Complex_Taylor_Series}.

\begin{definition}[Taylor Series]\label{def:Complex_Taylor_Series}
  The \emph{Taylor series} of a \nameref{def:Complex_Function} is created out of 2 equations.
  The first is \Cref{eq:Complex_Power_Series}, and the other is shown below.
  \begin{equation*}
    \begin{aligned}
      a_{n} &= \frac{1}{n!} \frac{d^{n}}{{dz}^{n}} f(a) \\
      &= \frac{f^{(n)}(a)}{n!}
    \end{aligned}
  \end{equation*}

  By replacing the $a_{n}$ terms, we end up with the general form of a Taylor series for a function $f$.
  \begin{equation}\label{eq:Complex_Taylor_Series}
    f(z) = \sum_{n=0}^{\infty} \frac{f^{(n)}(a)}{n!} {(z-a)}^{n}
  \end{equation}

  \begin{remark}[Maclaurin Series]\label{rmk:Complex_Maclaurin_Series}
    A \emph{Maclaurin series} is a special case of a \nameref{def:Complex_Taylor_Series}, where the function is centered on $a=0$.
  \end{remark}
\end{definition}

%%% Local Variables:
%%% mode: latex
%%% TeX-master: "../../Math_333-MatrixAlg_ComplexVars-Reference_Sheet"
%%% End:
