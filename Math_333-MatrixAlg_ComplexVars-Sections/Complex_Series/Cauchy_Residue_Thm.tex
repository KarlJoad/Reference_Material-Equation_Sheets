\subsection{Cauchy's Residue Theorem}\label{subsec:Cauchys_Residue_Theorem}
This is the 3rd and final method of integration we will discuss.

\begin{theorem}[Cauchy's Residue Theorem]\label{thm:Cauchys_Residue_Theorem}
  Let $\Omega$ be a \nameref{def:Connected}, \nameref{def:Open}, \nameref{def:Simply_Connected} region.
  Let $\Gamma$ be a \nameref{def:Path} in $\Omega$ which is \nameref{def:Simple_Path}, \nameref{def:Closed_Path}, and counterclockwise oriented.
  Let $f$ be \nameref{def:Analytic} on $\Omega$, except perhaps  with finitely many singularities, with no \nameref{def:Singularity} on the image of the path $\Gamma$.

  Then,
  \begin{equation}\label{eq:Cauchys_Residue_Theorem}
    \int_{\Gamma} f(z) dz = 2\pi i \left( \sum_{k=1}^{n} \Residue{f}{z=a_{k}} \right)
  \end{equation}
  The integral is equal to $2\pi i \times (\text{Sum of residues at singularities inside $\Gamma$})$.
\end{theorem}

\begin{remark*}
  Note that \nameref{thm:Cauchys_Integral_Theorem} is actually a special case of \nameref{thm:Cauchys_Residue_Theorem}.
\end{remark*}

\begin{example}[Lecture 11, Example 7]{Use Cauchy's Residue Theorem}
  Evaluate $\int_{\Gamma} \frac{\cos(z)}{e^{z} -1}$ where $\Gamma$ is the rectangle with vertices $\pm 5\pi \pm \pi i$?
  \tcblower{}
  Let $f(z) = \frac{\cos(z)}{e^{z} -1}$.
  From \Cref{ex:Using General Residue Equation}, we know that the singularities for $f$ are located at $z = 2\pi k i$, where $k = 0, 1, 2, \ldots$.

  Now, we look at the singularities \textbf{inside} the \nameref{def:Closed_Path} \nameref{def:Path} $\Gamma$.
  The only \nameref{def:Singularity} that is inside this region is when $k=0$, so $z=0$.
  \begin{align*}
    \int_{\Gamma} \frac{\cos(z)}{e^{z} - 1} dz &= 2\pi i \left( \Residue{f}{z=0} \right) \\
                                               &= {\left[ 2\pi i \bigl( \cosh(2\pi k) \bigr) \right]}_{k=0} \\
                                               &= 2\pi i \bigl( \cosh(0) \bigr) \\
                                               &= 2\pi i (1) \\
                                               &= 2\pi i
  \end{align*}
\end{example}

%%% Local Variables:
%%% mode: latex
%%% TeX-master: "../../Math_333-MatrixAlg_ComplexVars-Reference_Sheet"
%%% End:
