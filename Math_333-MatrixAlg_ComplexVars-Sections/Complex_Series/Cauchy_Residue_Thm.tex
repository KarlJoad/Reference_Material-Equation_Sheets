\subsection{Cauchy's Residue Theorem}\label{subsec:Cauchys_Residue_Theorem}
This is the 3rd and final method of integration we will discuss.

\begin{theorem}[Cauchy's Residue Theorem]\label{thm:Cauchys_Residue_Theorem}
  Let $\Omega$ be a \nameref{def:Connected}, \nameref{def:Open}, \nameref{def:Simply_Connected} region.
  Let $\Gamma$ be a \nameref{def:Path} in $\Omega$ which is \nameref{def:Simple_Path}, \nameref{def:Closed_Path}, and counterclockwise oriented.
  Let $f$ be \nameref{def:Analytic} on $\Omega$, except perhaps  with finitely many singularities, with no \nameref{def:Singularity} on the image of the path $\Gamma$.

  Then,
  \begin{equation}\label{eq:Cauchys_Residue_Theorem}
    \int_{\Gamma} f(z) dz = 2\pi i \left( \sum_{k=1}^{n} \Residue{f}{z=a_{k}} \right)
  \end{equation}
  The integral is equal to $2\pi i \times (\text{Sum of residues at singularities inside $\Gamma$})$.
\end{theorem}

\begin{remark*}
  Note that \nameref{thm:Cauchys_Integral_Theorem} is actually a special case of \nameref{thm:Cauchys_Residue_Theorem}.
\end{remark*}

\begin{example}[Lecture 11, Example 7]{Use Cauchy's Residue Theorem}
  Evaluate $\int_{\Gamma} \frac{\cos(z)}{e^{z} -1}$ where $\Gamma$ is the rectangle with vertices $\pm 5\pi \pm \pi i$?
  \tcblower{}
  Let $f(z) = \frac{\cos(z)}{e^{z} -1}$.
  From \Cref{ex:Using General Residue Equation}, we know that the singularities for $f$ are located at $z = 2\pi k i$, where $k = 0, 1, 2, \ldots$.

  Now, we look at the singularities \textbf{inside} the \nameref{def:Closed_Path} \nameref{def:Path} $\Gamma$.
  The only \nameref{def:Singularity} that is inside this region is when $k=0$, so $z=0$.
  \begin{align*}
    \int_{\Gamma} \frac{\cos(z)}{e^{z} - 1} dz &= 2\pi i \left( \Residue{f}{z=0} \right) \\
                                               &= {\left[ 2\pi i \bigl( \cosh(2\pi k) \bigr) \right]}_{k=0} \\
                                               &= 2\pi i \bigl( \cosh(0) \bigr) \\
                                               &= 2\pi i (1) \\
                                               &= 2\pi i
  \end{align*}
\end{example}

\subsubsection{Alternative Uses of \nameref*{thm:Cauchys_Residue_Theorem}}\label{subsubsec:Alternative_Uses_Cauchys_Residue_Theorem}
There are 2 major cases of evaluating real-valued Riemann integrals that is simplified using complex integrals and \nameref{thm:Cauchys_Residue_Theorem}.
They are of the forms:
\begin{enumerate}[noitemsep]
\item $\int_{0}^{2\pi} f \bigl( \cos(\theta), \sin(\theta) \bigr) \, d\theta$
\item $\int_{-\infty}^{\infty} \frac{p(x)}{q(x)} \, dx$
\end{enumerate}

\paragraph{Sinusoids}\label{par:Sinusoids_Cauchys_Residue_Theorem}
To integrate sinusoids, regardless of form, we need to convert the real-valued function $f(\theta)$ to a complex function $f(z)$ on the unit circle, in the counterclockwise direction.

To do this, we can describe a function that creates the unit circle like so, $z(\theta) = 1 \bigl( \cos(\theta) + i \sin(\theta) \bigr)$.
In addition, if we define the inverse of $z(\theta)$ as $z^{-1}(\theta) = 1 \bigl( \cos(\theta) + i \sin(\theta) \bigr)$, then we can just add these to yield an equation for $\cos(\theta)$ and subtract them for $\sin(\theta)$.
Eventually, we can generalize these down to the equations in \Cref{eq:Real_to_Imaginary_Sinusoids}.

\begin{equation}\label{eq:Real_to_Imaginary_Sinusoids}
  \begin{aligned}
    \cos(n \theta) &= \frac{1}{2} \left( z^{n} + \frac{1}{z^{n}} \right) \\
    \sin(n \theta) &= \frac{1}{2i} \left( z^{n} - \frac{1}{z^{n}} \right) \\
    d\theta &= \frac{dz}{i z}
  \end{aligned}
\end{equation}

By replacing the $\cos(\theta)$ and $\sin(\theta)$ in the given integral with the equivalents in \Cref{eq:Real_to_Imaginary_Sinusoids}, we can solve the integral, typically with \nameref{thm:Cauchys_Residue_Theorem}.

Another thing to keep in mind is the differential for the integral.
\begin{align*}
  \shortintertext{If}
  z &= e^{i \theta} \\
  dz &= i e^{i \theta} \, d\theta \\
    &= i z \, d\theta \\
  d\theta &= \frac{dz}{i z}
\end{align*}

\begin{example}[Lecture 12, Example 1]{Real-Valued Sinusoid Integrals using Cauchy's Residue Theorem}
  Evaluate the integral.
  \begin{equation*}
    \int_{0}^{2\pi} \frac{d\theta}{13 - 5\cos(\theta)}
  \end{equation*}
  \tcblower{}
  We can use \Cref{eq:Real_to_Imaginary_Sinusoids} to convert this to a \nameref{def:Complex_Number} and solve it.
  \begin{align*}
    \int_{0}^{2\pi} \frac{d\theta}{13 - 5\cos(\theta)} &= \int_{C} \frac{1}{13 - \frac{5}{2} \left( z + \frac{1}{z} \right)} \frac{dz}{i z} \\
                                                       &= \frac{1}{i} \int_{C} \frac{1}{13 - \frac{5}{2} \left( z + \frac{1}{z} \right)} \\
    \shortintertext{Multiple $z$ through.}
                                                       &= \frac{1}{i} \int_{C} \frac{1}{13z - \frac{5}{2} \left( z^{2} + 1 \right)} \, dz \\
    \shortintertext{Multiple fraction by $2$ to move $\frac{5}{2}$ to $5$.}
                                                       &= \frac{1}{i} \int_{C} \frac{2}{26z - 5 \left( z^{2} + 1 \right)} \, dz \\
                                                       &= \frac{2}{i} \int_{C} \frac{1}{26z - 5z^{2} + 5} \, dz \\
                                                       &= \frac{-2}{i} \int_{C} \frac{1}{5z^{2} - 26z + 5} \, dz \\
                                                       &= \frac{-2}{i} \int_{C} \frac{1}{(5z-1) (z-5)} \, dz
  \end{align*}

  Because we are working on the complex plane ($\Omega = \ComplexNumbers$), $C$ is the unit circle (which is inside $\Omega$), and the contents of the integral are \nameref{def:Analytic}, except at finitely many points, the two \nameref{def:Pole}s.
  Both of these poles are of \nameref{def:Pole_Order} two (2).
  Therefore, we can apply \nameref{thm:Cauchys_Residue_Theorem}.

  \begin{align*}
    \int_{0}^{2\pi} \frac{d\theta}{13 - 5\cos(\theta)} &= \frac{-2}{i} \int_{C} \frac{1}{(5z-1) (z-5)} \, dz \\
                                                       &= \frac{-2}{i} \left( 2 \pi i \left( \sum_{k=1}^{n} \Residue{\frac{1}{(5z-1) (z-5)}}{z=a_{k}} \right) \right) \\
    \intertext{The only \nameref{def:Pole} inside $C$ is the $5z-1$ pole, at $z = \frac{1}{5}$.}
                                                       &= -4 \pi \left( \Residue{\frac{1}{(5z-1) (z-5)}}{z=\frac{1}{5}} \right) \\
                                                       &= -4 \pi \left( \lim\limits_{z \to \frac{1}{5}} \left( z-\frac{1}{5} \right) \left( \frac{1}{(5z-1) (z-5)} \right) \right) \\
    \shortintertext{Factor out the $5$, so the $5z-1$ can be cancelled.}
                                                       &= -4 \pi \left( \lim\limits_{z \to \frac{1}{5}} \left( z-\frac{1}{5} \right) \left( \frac{1}{5 \left( z-\frac{1}{5} \right) (z-5)} \right) \right) \\
                                                       &= -4 \pi \left( \lim\limits_{z \to \frac{1}{5}} \frac{1}{5(z-5)} \right) \\
                                                       &= -4 \pi \left( \frac{1}{5 \left( \frac{1}{5} - 5 \right)} \right) \\
                                                       &= -4 \pi \left( \frac{1}{1-25} \right) \\
                                                       &= -4 \pi \left(\frac{1}{-24} \right) \\
                                                       &= \frac{\pi}{6}
  \end{align*}
\end{example}

%%% Local Variables:
%%% mode: latex
%%% TeX-master: "../../Math_333-MatrixAlg_ComplexVars-Reference_Sheet"
%%% End:
