\subsection{Cauchy's Residue Theorem}\label{subsec:Cauchys_Residue_Theorem}
This is the 3rd and final method of integration we will discuss.

\begin{theorem}[Cauchy's Residue Theorem]\label{thm:Cauchys_Residue_Theorem}
  Let $\Omega$ be a \nameref{def:Connected}, \nameref{def:Open}, \nameref{def:Simply_Connected} region.
  Let $\Gamma$ be a \nameref{def:Path} in $\Omega$ which is \nameref{def:Simple_Path}, \nameref{def:Closed_Path}, and counterclockwise oriented.
  Let $f$ be \nameref{def:Analytic} on $\Omega$, except perhaps  with finitely many singularities, with no \nameref{def:Singularity} on the image of the path $\Gamma$.

  Then,
  \begin{equation}\label{eq:Cauchys_Residue_Theorem}
    \int_{\Gamma} f(z) dz = 2\pi i \left( \sum_{k=1}^{n} \Residue{f}{z=a_{k}} \right)
  \end{equation}
  The integral is equal to $2\pi i \times (\text{Sum of residues at singularities inside $\Gamma$})$.
\end{theorem}


%%% Local Variables:
%%% mode: latex
%%% TeX-master: "../../Math_333-MatrixAlg_ComplexVars-Reference_Sheet"
%%% End:
