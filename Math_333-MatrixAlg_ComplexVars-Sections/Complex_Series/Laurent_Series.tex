\subsection{Laurent Series}\label{def:Laurent_Series}
\nameref{thm:Laurent_Series} are generalizations of the \nameref{def:Complex_Power_Series} that we have already been discussing.

\begin{theorem}[Laurent Series]\label{thm:Laurent_Series}
  Suppose $f$ is \nameref{def:Analytic} on the \nameref{def:Annulus} $r_{1} < \Modulus{z-a} < r_{2}$.

  Then there exists a \emph{Laurent series} such that $\forall z, r_{1} < \Modulus{z-a} < r_{2}$
  \begin{equation}\label{eq:Laurent_Series}
    f(z) = \sum_{n=0}^{\infty} a_{n} {(z-a)}^{n} + \frac{b_{1}}{z-a} + \frac{b_{2}}{{(z-a)}^{2}} + \frac{b_{3}}{{(z-a)}^{3}} + \cdots
  \end{equation}
\end{theorem}

\begin{definition}[Principal Part]\label{def:Principal_Part}
  The \emph{principal part} of the \nameref{thm:Laurent_Series} $f(z)$ is the summation of all the $b_{n}$ elements.
  \begin{equation}\label{eq:Principal_Part}
    \frac{b_{1}}{z-a} + \frac{b_{2}}{{(z-a)}^{2}} + \frac{b_{3}}{{(z-a)}^{3}} + \cdots
  \end{equation}
\end{definition}

Before we continue any farther, we need to add some more terms to our dictionary.

\begin{definition}[Deleted]\label{def:Deleted}
  \emph{Deleted} is a term that is used to describe disks.
  A deleted disk is one that is of the form $0 < \Modulus{z-a} < r$.
\end{definition}

\begin{definition}[Singularity]\label{def:Singularity}
  A \emph{singularity} is a point where a function is non-\nameref{def:Analytic}.
\end{definition}

\begin{definition}[Isolated]\label{def:Isolated_Singularity}
  In this definition, \emph{isolated} refers to a \nameref{def:Singularity}.
  For a singularity to be isolated, this means that for \textbf{all} points surrounding the singularity $z=a$, the function \textbf{IS} \nameref{def:Analytic}.
\end{definition}


%%% Local Variables:
%%% mode: latex
%%% TeX-master: "../../Math_333-MatrixAlg_ComplexVars-Reference_Sheet"
%%% End:
