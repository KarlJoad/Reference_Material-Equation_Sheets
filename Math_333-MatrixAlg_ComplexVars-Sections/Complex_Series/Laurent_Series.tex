\subsection{Laurent Series}\label{def:Laurent_Series}
\nameref{thm:Laurent_Series} are generalizations of the \nameref{def:Complex_Power_Series} that we have already been discussing.

\begin{theorem}[Laurent Series]\label{thm:Laurent_Series}
  Suppose $f$ is \nameref{def:Analytic} on the \nameref{def:Annulus} $r_{1} < \Modulus{z-a} < r_{2}$.

  Then there exists a \emph{Laurent series} such that $\forall z, r_{1} < \Modulus{z-a} < r_{2}$
  \begin{equation}\label{eq:Laurent_Series}
    f(z) = \sum_{n=0}^{\infty} a_{n} {(z-a)}^{n} + \frac{b_{1}}{z-a} + \frac{b_{2}}{{(z-a)}^{2}} + \frac{b_{3}}{{(z-a)}^{3}} + \cdots
  \end{equation}
\end{theorem}

\begin{definition}[Principal Part]\label{def:Principal_Part}
  The \emph{principal part} of the \nameref{thm:Laurent_Series} $f(z)$ is the summation of all the $b_{n}$ elements.
  \begin{equation}\label{eq:Principal_Part}
    \frac{b_{1}}{z-a} + \frac{b_{2}}{{(z-a)}^{2}} + \frac{b_{3}}{{(z-a)}^{3}} + \cdots
  \end{equation}
\end{definition}

Before we continue any farther, we need to add some more terms to our dictionary.

\begin{definition}[Deleted]\label{def:Deleted}
  \emph{Deleted} is a term that is used to describe disks.
  A deleted disk is one that is of the form $0 < \Modulus{z-a} < r$.
\end{definition}

\subsubsection{Function Singularities}\label{subsubsec:Function_Singularities}
\begin{definition}[Singularity]\label{def:Singularity}
  A \emph{singularity} is a point where a function is non-\nameref{def:Analytic}.
\end{definition}

\begin{definition}[Isolated]\label{def:Isolated_Singularity}
  In this definition, \emph{isolated} refers to a \nameref{def:Singularity}.
  For a singularity to be isolated, this means that for \textbf{all} points surrounding the singularity $z=a$, the function \textbf{IS} \nameref{def:Analytic}.
\end{definition}

\begin{example}[Lecture 10, Example 1]{Simple Isolated Singularity}
  Given the function $f(z) = \frac{\sin(z)}{z^{2}}$, find its \nameref{def:Singularity}?
  \tcblower{}
  If we look at $f(z)$, we notice that $f$ is undefined at $z=0$.
  This also means that $f$ is \textbf{not} differentiable at $z=0$, making $z=0$ a \nameref{def:Singularity}.

  $\therefore f$ has a \nameref{def:Singularity} at $z=0$.
  In addition, $z=0$ is an \nameref{def:Isolated_Singularity} \nameref{def:Singularity}, because at \textbf{every} other point, $f$ is differentiable.
\end{example}

\begin{example}[Lecture 10, Example 2]{Multiple Isolated Singularities}
  Given the function $f(z) = \frac{z+1}{(z-2) (z+3)}$, find the singularities?
  \tcblower{}
  Looking at $f$, we immediately notice two points where $f$ is undefined, and therefore does not have a derivative.
  $z=2$ and $z=-3$.
  In addition, both of these singularities are \nameref{def:Isolated_Singularity}, because we can find a disk/\nameref{def:Annulus} surrounding the \nameref{def:Singularity} such that $f$ is \nameref{def:Analytic}.
\end{example}

With the terms above defined, we can move onto seeing a theorem regarding the \nameref{thm:Laurent_Series_Existence}.

\begin{theorem}[Existence of Laurent Series]\label{thm:Laurent_Series_Existence}
  A function $f$, that is \nameref{def:Analytic} on a \nameref{def:Deleted} disk, $0 < \Modulus{z-a} < r$ is said to have an (\nameref{def:Isolated_Singularity}) \nameref{def:Singularity} of $z=a$ if $f$ is \textbf{not} differentiable at $z=a$.

  Then, there exists a \nameref{thm:Laurent_Series} of the form:
  \begin{equation}\label{eq:Laurent_Series_Existence}
    f(z) = \sum_{n=0}^{\infty} a_{n} {(z-a)}^{n} + \frac{b_{1}}{z-a} + \frac{b_{2}}{{(z-a)}^{2}} + \cdots
  \end{equation}
\end{theorem}

\begin{definition}[Removable]\label{def:Removable}
  A \emph{removable} \nameref{def:Singularity} is one that can be canceled out, somehow.
  Sometimes these are obvious, and can just be factored out.
  Other times, these can only be found by working with the \nameref{thm:Laurent_Series}.
\end{definition}

If \Cref{thm:Laurent_Series_Existence} is satisfied, then there are 3 cases:
\begin{enumerate}[noitemsep]
\item All terms $b_{1}, b_{2}, b_{3}, \ldots = 0$.
  Then the \nameref{def:Singularity} is said to be \nameref{def:Removable}.
  Refer to \Cref{ex:Laurent Series Removable Singularity}.
\item The number of non-zero $b_{n}$ is finite.
  This means that the \nameref{def:Principal_Part} of the \nameref{thm:Laurent_Series} is finite.
  This implies that there is a \nameref{def:Pole} with an \nameref{def:Pole_Order}.
  Refer to \Cref{ex:Laurent Series Pole Singularity}.
\item The number of non-zero $b_{n}$ is infinite.
\end{enumerate}

\begin{example}[Lecture 10, Example 3]{Laurent Series Removable Singularity}
  Given the function $f(z) = \frac{\sin \left( z^{3} \right)}{z^{2}}$, find the singularity and determine if it is \nameref{def:Removable}?
  \tcblower{}
  $f$ has an \nameref{def:Isolated_Singularity} \nameref{def:Singularity} at $z=0$.

  Now, look at the \nameref{def:Annulus} where $f$ is defined; which also determines where $f$ is \nameref{def:Analytic}.
  $f$ is analytic on the annulus $0 < \Modulus{z} < \infty$.
  This disk comprises the set $\ComplexNumbers - \lbrace z=0 \rbrace$.

  Now, we expand $f$ into a \nameref{thm:Laurent_Series}.
  We start by first expanding the $a_{n}$ components of the series using the definition of the \nameref{def:Complex_Power_Series}.
  \begin{align*}
    \sin(z) &= z - \frac{1}{3!} z^{3} + \frac{1}{5!} z^{5} - \cdots \\
    \sin \left( z^{3}\right) &= z^{3} - \frac{1}{3!} {(z^{3})}^{3} + \frac{1}{5!} {(z^{3})}^{5} - \cdots \\
    \intertext{Now, replace our simple $\sin$ with its corresponding power series.}
    f(z) &= \frac{z^{3} - \frac{1}{3!} {(z^{3})}^{3} + \frac{1}{5!} {(z^{3})}^{5} - \cdots}{z^{2}} \\
    &= z - \frac{1}{3!} z^{7} + \frac{1}{5!} z^{13} - \cdots
  \end{align*}

  Note the complete lack of negative exponents of $z$.
  This means that the \nameref{def:Principal_Part} of the \nameref{thm:Laurent_Series} has $0$ for all its $b_{1}, b_{2}, b_{3}, \ldots$ terms.

  $\therefore$ The \nameref{def:Singularity} $z=0$ is \nameref{def:Removable}.
\end{example}

\begin{definition}[Pole]\label{def:Pole}
  A \emph{pole} is the \nameref{def:Singularity} of a \nameref{def:Complex_Function} when the \nameref{thm:Laurent_Series} has a finite number of terms in its \nameref{def:Principal_Part}.

  This pole would be located at $z=a$.
  \begin{equation}\label{eq:Pole}
    f(z) = \frac{g(z)}{z-a}, \: g(a) \neq 0
  \end{equation}
\end{definition}

\begin{definition}[Order]\label{def:Pole_Order}
  The \emph{order} of a \nameref{def:Pole} is the highest negative power in the \nameref{def:Principal_Part} of a \nameref{thm:Laurent_Series} with a finite principal part.
\end{definition}

\begin{definition}[Residue]\label{def:Residue}
  The \emph{residue} of a \nameref{def:Complex_Function} is the coefficient of the first term of the \nameref{def:Principal_Part} of the \nameref{thm:Laurent_Series}.
  This means that the residue of a function $f$ with Laurent series $f(z) = \sum_{n=0}^{\infty} a_{n} {(z-a)}^{n} + \frac{b_{1}}{z-a} + \frac{b_{2}}{{(z-a)}^{2}} + \cdots$ is $b_{1}$.

  However, if a function as a finite number of terms in the \nameref{def:Principal_Part} of the \nameref{thm:Laurent_Series}, then there are formulas for finding the residues, based on the \nameref{def:Pole_Order} of the \nameref{def:Pole} at the \nameref{def:Singularity} $z=a$.
  \begin{equation}\label{eq:Residue}
    b_{1} = \Residue{f}{z=a} = \lim\limits_{z \to a} {(z-a)}^{n} f^{(n-1)}(z)
  \end{equation}

  The number of residues correspond to the number of \nameref{def:Pole}s in a \nameref{thm:Laurent_Series}.
  This is shown in \Cref{ex:Laurent Series Multiple Residues}.
\end{definition}

\begin{example}[Lecture 10, Example 4]{Laurent Series Pole Singularity}
  Given the function $f(z) = \frac{\sin \left( 3 z^{2} \right)}{z^{17}}$, find its \nameref{def:Principal_Part}, \nameref{def:Pole}s, \nameref{def:Pole_Order} of the pole, and the \nameref{def:Residue}?
  \tcblower{}
  The only \nameref{def:Singularity} here, is again $z=0$.
  $f$ is defined on the \nameref{def:Deleted} disk $0 < \Modulus{z} < \infty$.

  Now, we use the definition of the $\sin$ function's \nameref{def:Complex_Power_Series}, \Cref{eq:Complex_Power_Series-sin}.
  \begin{align*}
    \sin(z) &= z - \frac{1}{3!} z^{3} + \frac{1}{5!} z^{5} + \cdots \\
    \sin \left( 3z^{2} \right) &= 3z^{2} - \frac{1}{3!} { \left( 3z^{2} \right)}^{3} + \frac{1}{5!} {\left( 3z^{2} \right)}^{5} - \frac{1}{7!} { \left( 3z^{2} \right)}^{7} + \frac{1}{9!} { \left( 3z^{2} \right)}^{9} - \cdots \\
            &= 3z^{2} - \frac{3^{3}}{3!} z^{6} + \frac{3^{5}}{5!} z^{10} - \frac{3^{7}}{7!} z^{14} + \frac{3^{9}}{9!} z^{18} - \cdots \\
    \shortintertext{Now, plugging back into $f$.}
    f(z) &= \frac{3z^{2} - \frac{3^{3}}{3!} z^{6} + \frac{3^{5}}{5!} z^{10} - \frac{3^{7}}{7!} z^{14} + \frac{3^{9}}{9!} z^{18} - \cdots}{z^{17}} \\
            &= 3z^{-15} - \frac{3^{3}}{3!} z^{-11} + \frac{3^{5}}{5!} z^{-7} - \frac{3^{7}}{7!} z^{-3} + \frac{3^{9}}{9!} z^{1} - \cdots
  \end{align*}

  Therefore, the \nameref{def:Principal_Part} of $f$ is
  \begin{equation*}
    3z^{-15} - \frac{3^{3}}{3!} z^{-11} + \frac{3^{5}}{5!} z^{-7} - \frac{3^{7}}{7!} z^{-3} + \frac{3^{9}}{9!} z^{1}
  \end{equation*}

  Because the \nameref{def:Principal_Part} is finite, the \nameref{def:Singularity} $z=0$ is a \nameref{def:Pole}.
  In addition, the pole is of \nameref{def:Pole_Order} $15$.

  Lastly, the residue of $f$ is
  $b_{1}$, which in this case is $0$.
\end{example}

\subsubsection{Reverse Engineer Function from Laurent Series Properties}\label{subsubsec:Reverse_Engineer_Function_from_Laurent_Series_Properties}
Now that we know how to find the various properties of a \nameref{def:Complex_Function} with a \nameref{thm:Laurent_Series}, what if we are given the various properties of the function.
What does the function look like then?
This question is explored in \Cref{ex:Laurent Series Properties Find Function}.

\begin{example}[Lecture 10, Example 5]{Laurent Series Properties Find Function}
  Suppose $f(z)$ is \nameref{def:Analytic} on the \nameref{def:Annulus} $0 < \Modulus{z-a} < r$, and has a single \nameref{def:Pole} of \nameref{def:Pole_Order} 1.
  What does the \nameref{def:Complex_Function} look like?
  \tcblower{}
  We start with the general form of a \nameref{def:Complex_Function} expressed as a \nameref{thm:Laurent_Series}.
  \begin{equation*}
    f(z) = \sum_{n=0}^{\infty} a_{n} {(z-a)}^{n} + \frac{b_{1}}{z-a} + \frac{b_{2}}{{(z-a)}^{2}} + \cdots
  \end{equation*}

  Now, we remove terms and items that we do not need.
  Because we are told that the only \nameref{def:Pole} has \nameref{def:Pole_Order} of 1, the highest term in the $b_{n}$ portion has an exponent of $1$ on the $z$ term.
  This means we only go up to the $b_{1}$ term.
  Thus, our equation is:
  \begin{equation*}
    f(z) = \sum_{n=0}^{\infty} a_{n} {(z-a)}^{n} + \frac{b_{1}}{z-a}
  \end{equation*}
\end{example}

\Cref{ex:Laurent Series Properties Find Function} shows us something else that is will become important in \Cref{subsec:Cauchys_Residue_Theorem}.
We can perform some algebraic manipulations to have the numerator of some \nameref{def:Analytic} \nameref{def:Complex_Function} be another analytic function.
Take our solution from \Cref{ex:Laurent Series Properties Find Function}.
\begin{align*}
  f(z) &= \sum_{n=0}^{\infty} a_{n} {(z-a)}^{n} + \frac{b_{1}}{z-a} \\
       &= \frac{(z-a) \left( \sum_{n=0}^{\infty} a_{n} {(z-a)}^{n} \right) + b_{1}}{z-a} \\
       &= \frac{g(z)}{z-a}
\end{align*}

Thus, $g(z)$ is a sum of a power series with center $z=a$, meaning $g$ is \nameref{def:Analytic} on $\Modulus{z-a} < r$, where $g(a) = b_{1} \neq 0$.
If $g(a) = 0$, that means there would be some $z-a$ term that could cancel another.

\subsubsection{Multiple Poles}\label{subsubsec:Multiple_Poles}
Sometimes, it is tricky to determine the number of \nameref{def:Pole}s present in an equation.
This is explored in \Cref{ex:Laurent Series Non-Obvious Number of Poles}
\begin{example}[Lecture 10, Example 7]{Laurent Series Non-Obvious Number of Poles}
  Given the function $f(z) = \frac{z^{2} - 5z + 6}{z^{2} - 4}$, what are its \nameref{def:Pole}s?
  \tcblower{}
  Start by factoring the equation.
  \begin{align*}
    f(z) &= \frac{z^{2} - 5z + 6}{z^{2}- 4} \\
         &= \frac{(z-2) (z-3)}{(z-2) (z+2)} \\
         &= \frac{z-3}{z+2}
  \end{align*}

  Now, looking at this, we see it is obvious that there is only a single \nameref{def:Pole} of \nameref{def:Pole_Order} 1 (a \nameref{def:Simple_Pole} pole) at $z=-2$.
  However, $z=2$ \textbf{IS} a \nameref{def:Singularity} for $f(z)$, but it is \nameref{def:Removable}.
\end{example}

\begin{definition}[Simple]\label{def:Simple_Pole}
  A \emph{simple} \nameref{def:Pole} is one that is of \nameref{def:Pole_Order} 1.
\end{definition}

\begin{example}[Lecture 10, Example 8]{Laurent Series Multiple Residues}
  Given the function $f(z) = \frac{z+1}{(z-2) (z+3)}$, what are the \nameref{def:Residue}s of $f$?
  \tcblower{}
  We start by noting that there are 2 singularities, $z=2$ and $z=-3$.
  As these singularities \textbf{cannot} be canceled out, they are \nameref{def:Pole}s too.
  Both of these poles are \nameref{def:Simple_Pole}, as well.
  Because they are both of \nameref{def:Pole_Order} 1, we modify \Cref{eq:Residue} for that.

  Starting with the \nameref{def:Pole} $z=2$.
  \begin{align*}
    \Residue{f}{z=2} &= \lim\limits_{z \to 2} (z-2) f(z) \\
                     &= \lim\limits_{z \to 2} (z-2) \left( \frac{z+1}{(z-2) (z+3)} \right) \\
                     &= \lim\limits_{z \to 2} \frac{z+1}{z+3} \\
                     &= \frac{3}{5} \\
  \end{align*}

  Now, the \nameref{def:Pole} $z=-3$.
  \begin{align*}
    \Residue{f}{z=-3} &= \lim\limits_{z \to -3} \bigl( z- (-3) \bigr) f(z) \\
                      &= \lim\limits_{z \to -3} (z+3) \left( \frac{z+1}{(z-2)(z+3)} \right) \\
                      &= \lim\limits_{z \to -3} \frac{z+1}{z-2} \\
                      &= \frac{-2}{-5} \\
                      &= \frac{2}{5}
  \end{align*}

  Thus, our two \nameref{def:Residue}s are:
  \begin{align*}
    \Residue{f}{z=2} &= \frac{3}{5} \\
    \Residue{f}{z=-3} &= \frac{2}{5}
  \end{align*}
\end{example}

Lastly, another example of finding the \nameref{def:Residue}s of a \nameref{def:Complex_Function}, but this time a function with a \nameref{def:Pole} of \nameref{def:Pole_Order} 2.
This is demonstrated in \Cref{ex:Laurent Series Higher-Order Residues}

\begin{example}[Lecture 10, Example 9]{Laurent Series Higher-Order Residues}
  Given the function $f(z) = \frac{e^{z}}{{(z-2)}^{2} (z+3)}$, find its \nameref{def:Residue}s?
  \tcblower{}
  This has two singularities: $z=2$ and $z=-3$.
  In addition, these are both \nameref{def:Pole}s as well.
  The $z=2$ pole has \nameref{def:Pole_Order} 2, the other has order 1.

  Starting with the $z=2$ \nameref{def:Pole}:
  \begin{align*}
    \Residue{f}{z=2} &= \lim\limits_{z \to 2} \frac{d}{dz} \left\lbrace {(z-2)}^{2} \left( \frac{e^{z}}{{(z-2)}^{2} (z+3)} \right) \right\rbrace \\
                     &= \lim\limits_{z \to 2} \frac{d}{dz} \left\lbrace \frac{e^{z}}{z+3} \right\rbrace \\
                     &= \lim\limits_{z \to 2} \frac{e^{z}(z+3) - e^{z} (1)}{{(z+3)}^{2}} \\
                     &= \frac{5e^{z} - e^{1}}{5^{2}} \\
                     &= \frac{4e^{z}}{25}
  \end{align*}

  Now, the $z=-3$ pole:
  \begin{align*}
    \Residue{f}{z=-3} &= \lim\limits_{z \to -3} (z+3) \left( \frac{e^{z}}{{(z-2)}^{2} (z+3)} \right) \\
                      &= \lim\limits_{z \to -3} \frac{e^{z}}{{(z-2)}^{2}} \\
                      &= \frac{e^{-3}}{{(-3 - 2)}^{2}} \\
                      &= \frac{e^{-3}}{{(-5)}^{2}} \\
                      &= \frac{e^{-3}}{25}
  \end{align*}

  Thus, our two \nameref{def:Residue}s are:
  \begin{align*}
    \Residue{f}{z=2} &= \frac{4e^{z}}{25} \\
    \Residue{f}{z=-3} &= \frac{e^{-3}}{25}
  \end{align*}
\end{example}

\subsubsection{Zeros of a Function}\label{subsubsec:Zeros_of_Functions}
\begin{definition}[Zero]\label{def:Zero}
  Let $f$ be a \nameref{def:Complex_Function} defined on an \nameref{def:Open}, \nameref{def:Connected} set (domain), $\Omega$.
  Let $a \in \Omega$.
  Let $f \neq 0$, to avoid trivial cases.

  We say that $f$ has a \emph{zero} at $z=a$ if $f(a) = 0$.
\end{definition}

\begin{definition}[Simple Zero]\label{def:Simple_Zero}
  A \emph{simple zero} is a \nameref{def:Zero} with a \nameref{thm:Multiplicity} of $1$.
\end{definition}

\begin{example}[Lecture 11, Example 1]{Find Simple Zeros}
  Find all zeros of the function $f(z)$ below.
  \begin{equation*}
    f(z) = \sin(z)
  \end{equation*}
  \tcblower{}
  $f$ has a zero at $z=0, \pi, 2\pi, \ldots$, because when $z$ is set to any of those values, the function yields a value of $0$.
\end{example}

In addition to finding the location of a \nameref{def:Zero}, we also need to find the \nameref{thm:Multiplicity} of the zero.

\subsubsection{Multiplicity}\label{subsubsec:Multiplicity}
\begin{theorem}[Multiplicity]\label{thm:Multiplicity}
  $f \neq0$ has a \nameref{def:Zero} at $z=a$ if and only if there exists an \nameref{def:Analytic} \nameref{def:Complex_Function} $g$ on $\Omega$ and $n \geq 1$, $n \in \NaturalNumbers$ such that $f(z) = {(z-a)}^{n} g(z)$.

  The value of $n$ is the \emph{multiplicity} of the zero $z=a$.
\end{theorem}

\begin{example}[Lecture 11, Example 2]{Find Multiplicity by Observation}
  Given the \nameref{def:Complex_Function} $f(z)$, as defined below, what are the zeros of the function and their multiplicities?
  \begin{equation*}
    f(z) = {(z^{2}-4)}^{3} {(z^{2} - z - 6)}^{5}
  \end{equation*}
  \tcblower{}
  The first thing to check that $f$ is \nameref{def:Analytic} on some $\Omega$, where I assume $\Omega = \ComplexNumbers$.
  $f$ is analytic, so we can move on.

  For this polynomial, we can find all the zeros by factoring.
  \begin{align*}
    f(z) &= {(z^{2}-4)}^{3} {(z^{2} - z - 6)}^{5} \\
         &= {(z-2) (z+2)}^{3} {(z+2) (z-3)}^{5} \\
         &= {(z-2)}^{3} {(z+2)}^{3} {(z+2)}^{5} {(z-3)}^{5} \\
         &= {(z-2)}^{3} {(z+2)}^{8} {(z-3)}^{5}
  \end{align*}

  Therefore, the three zeros are:
  \begin{description}[noitemsep]
  \item $z=2$, with \nameref{thm:Multiplicity} $3$
  \item $z=-2$, with \nameref{thm:Multiplicity} $8$
  \item $z=3$, with \nameref{thm:Multiplicity} $5$
  \end{description}
\end{example}

Because we have only found the \nameref{thm:Multiplicity} by observation so far, we would prefer an algorithm to figure it out.
This would be particularly useful if we cannot factor the term.
The algorithm is illustrated by \Cref{ex:Multiplicity Algorithm}, and is codified in \Cref{algo:Multiplicity}.

\begin{example}[Lecture 11, Example 3]{Multiplicity Algorithm}
  Suppose $f$ has a zero of \nameref{thm:Multiplicity} $2$ at $z=a$.
  This means that the function is of the general form $f(z) = {(z-a)}^{2} g(z)$, where $g(a) \neq 0$.

  To confirm that the \nameref{thm:Multiplicity} is actually 2, we perform the derivation below.
  \begin{align*}
    \shortintertext{Start by taking the derivative of $f$.}
    f'(z) &= 2(z-a) g(z) + {(z-a)}^{2} g'(z) \\
    \shortintertext{Now, plug $z=a$ directly into $f'(z)$.}
    f'(a) &= 2(a-a) g(a) + {(a-a)}^{2} g'(a) \\
          &= 0 + 0 \\
          &= 0 \\
    \intertext{Because $f'(z) = 0$, we must repeat this procedure again.}
    f''(z) &= {(z-a)}^{2} g''(z) + 2(z-a) g(z) + 2(z-a) g'(z) + 2 g(z) \\
    \shortintertext{Plug $z=a$ into $f''(z)$.}
    f''(a) &= {(a-a)}^{2} g''(a) + 2(a-a) g(a) + 2(a-a) g'(a) + 2 g(a) \\
          &= 0 + 0 + 0 + 2 g(z)
  \end{align*}

  We stop at the point where $f^{(n)}(a) \neq 0$, and the value of $n$ is the \nameref{thm:Multiplicity} of the zero $z=a$.
\end{example}

%%% Local Variables:
%%% mode: latex
%%% TeX-master: "../../Math_333-MatrixAlg_ComplexVars-Reference_Sheet"
%%% End:
