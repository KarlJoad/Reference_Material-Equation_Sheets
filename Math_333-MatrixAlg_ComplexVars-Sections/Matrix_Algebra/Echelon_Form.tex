\subsection{Echelon Form}\label{subsec:Echelon_Form}
\begin{theorem}[Echelon Form]\label{thm:Echelon_Form}
  A \nameref{def:Matrix} $A =
  \begin{bmatrix}
    r_{1} \\
    r_{2} \\
    \vdots \\
    r_{n}
  \end{bmatrix}
  $ is said to be in \emph{Echelon form} if the first non-zero element of the $i$th row, $r_{i}$ where $i \geq 2$ appears on a column that is to the right from which the first non-zero element of the $r_{i-1}$ appears.

  Some examples of this are:
  \begin{equation*}
    \begin{bmatrix}
      0 & 1 & 2 & 3 & 4 & 5 \\
      0 & 0 & 5 & 6 & 0 & 0 \\
      0 & 0 & 0 & 0 & 7 & 0 \\
      0 & 0 & 0 & 0 & 0 & 1 \\
      0 & 0 & 0 & 0 & 0 & 0 \\
    \end{bmatrix}
  \end{equation*}
\end{theorem}


%%% Local Variables:
%%% mode: latex
%%% TeX-master: "../../Math_333-MatrixAlg_ComplexVars-Reference_Sheet"
%%% End:
