\subsection{Diagonalization}\label{subsec:Diagonalization}
To start off with, we need some background knowledge on what we mean by \nameref{def:Diagonalization}.

\begin{definition}[Diagonal Matrix]\label{def:Diagonal_Matrix}
  A \emph{diagonal matrix} is a \nameref{def:Matrix} whose only non-zero elements are on the main diagonal of the matrix.

  In general, this is seen as:
  \begin{equation}\label{eq:Diagonal_Matrix}
    A =
    \begin{pmatrix}
      a_{1,1} & 0 & 0 & \cdots \\
      0 & a_{2,2} & 0 & \cdots \\
      \vdots & \ddots & \ddots & \vdots \\
      0 & \cdots & \cdots & a_{n, n}
    \end{pmatrix}
  \end{equation}
\end{definition}

\begin{definition}[Diagonalizable]\label{def:Diagonalizable}
  Let the \nameref{def:Matrix} $A_{n \by n}$.
  We say $A$ is \emph{diagonalizable} to a \nameref{def:Diagonal_Matrix} $D$ if there exists an invertible matrix $P$ such that \Cref{eq:Diagonalizable} holds true.

  \begin{equation}\label{eq:Diagonalizable}
    P^{-1} A P = D
  \end{equation}
\end{definition}

\begin{definition}[Diagonalization]\label{def:Diagonalization}
  Let the matrix $A_{n \by n}$ be \nameref{def:Diagonalizable} by an invertible \nameref{def:Matrix} $P$ to form a \nameref{def:Diagonal_Matrix} $D$.
  $P$ is called a matrix that implements the \emph{diagonalization} shown in \Cref{eq:Diagonalization}.

  \begin{equation}\label{eq:Diagonalization}
    AP = PD
  \end{equation}
\end{definition}


%%% Local Variables:
%%% mode: latex
%%% TeX-master: "../../Math_333-MatrixAlg_ComplexVars-Reference_Sheet"
%%% End:
