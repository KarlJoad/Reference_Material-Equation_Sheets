\subsection{Inverse Matrices}\label{subsec:Inverse_Matrices}
\begin{definition}[Inverse Matrix]\label{def:Inverse_Matrix}
  Let $A_{n \by n}$ be an $n \by n$ \nameref{def:Matrix}.
  A matrix $B$ is called an inverse of $A$ if and only if
  \begin{equation}\label{eq:Inverse_Matrix}
    AB = I
  \end{equation}
\end{definition}

\begin{theorem}[Inverse Matrix Commutativity]\label{thm:Inverse_Matrix_Commutativity}
  If $B$ is an inverse of $A$, then $AB = BA = I$.
\end{theorem}

\begin{theorem}[Uniqueness of Inverse Matrix]\label{thm:Inverse_Matrix_Uniqueness}
  Let $A_{n \by n}$ be an $n \by n$ \nameref{def:Matrix}.

  Then, $A$ has \textbf{at most} one inverse, denoted $A^{-1}$.
\end{theorem}

%%% Local Variables:
%%% mode: latex
%%% TeX-master: "../../Math_333-MatrixAlg_ComplexVars-Reference_Sheet"
%%% End:
