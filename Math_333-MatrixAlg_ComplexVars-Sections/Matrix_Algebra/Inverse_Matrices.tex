\subsection{Inverse Matrices}\label{subsec:Inverse_Matrices}
\begin{definition}[Inverse Matrix]\label{def:Inverse_Matrix}
  Let $A_{n \by n}$ be an $n \by n$ \nameref{def:Matrix}.
  A matrix $B$ is called an inverse of $A$ if and only if
  \begin{equation}\label{eq:Inverse_Matrix}
    AB = I
  \end{equation}
\end{definition}

\begin{theorem}[Inverse Matrix Commutativity]\label{thm:Inverse_Matrix_Commutativity}
  If $B$ is an inverse of $A$, then $AB = BA = I$.
\end{theorem}

\begin{theorem}[Uniqueness of Inverse Matrix]\label{thm:Inverse_Matrix_Uniqueness}
  Let $A_{n \by n}$ be an $n \by n$ \nameref{def:Matrix}.

  Then, $A$ has \textbf{at most} one inverse, denoted $A^{-1}$.
\end{theorem}

\begin{example}[Lecture 13, Example 3]{Matrix with No Inverse}
  Let $A$ be the \nameref{def:Matrix} defined below.
  \begin{equation*}
    A =
    \begin{bmatrix}
      1 & 1 \\
      2 & 2
    \end{bmatrix}
  \end{equation*}
  Find a matrix $B$ such that $B = A^{-1}$?
  \tcblower{}
  Using the definition of an \nameref{def:Inverse_Matrix}, we can prove this with an example.
  \begin{align*}
    \begin{bmatrix}
      1 & 1 \\
      2 & 2
    \end{bmatrix}
          \begin{bmatrix}
            b_{11} & b_{12} \\
            b_{21} & b_{22}
          \end{bmatrix}
        &=
          \begin{bmatrix}
            1 & 0 \\
            0 & 1
          \end{bmatrix} \\
    &=
      \begin{bmatrix}
        b_{11} + b_{21} & b_{12} + b_{22} \\
        2b_{11} + 2b_{22} & 2b_{12} + 2b_{22}
      \end{bmatrix}
  \end{align*}

  We have a system of equations we can show that there is no \nameref{def:Inverse_Matrix}.

  \begin{align*}
    b_{11} + b_{21} &= 1 \\
    2b_{11} + 2b_{21} &= 0
  \end{align*}

  However, this system is in\nameref{def:Consistent}, meaning there is no solution for $b_{11}$ and $b_{21}$.

  Therefore, there does not exist an $A^{-1} = B$ such that $AB = I$.
\end{example}

\subsubsection{Inverse Matrices using Elementary Matrices}\label{subsubsec:Inverse_Matrices_using_Elementary_Matrices}
The \nameref{def:Inverse_Matrix} of any \nameref{def:Matrix} can be found by using \nameref{def:Elementary_Row_Op}s and the actual inverse can be found by using an \nameref{def:Elementary_Matrix}.

\begin{theorem}
  Let $A$ be a matrix with dimensions $n \by n$ ($A_{n \by n}$).

  Then, reducing $A$ to \nameref{thm:Echelon_Form} can be encoded with a series of elementary matrices.
  Mathematically, this implies
  \begin{equation*}
    E_{t} E_{t-1} \cdots E_{2} E_{1} A = I
  \end{equation*}

  This yields 2 cases:
  \begin{equation*}
    \begin{cases}
      \begin{amat}{3}
        \vdots & \cdots & \cdots & x \\
        \vdots & \cdots & \cdots & x \\
        0 & 0 & \cdots & 0 \\
      \end{amat} \\[3em]

      \begin{amat}{3}
        \vdots & \cdots & \cdots & x \\
        \vdots & \cdots & \cdots & x \\
        0 & 0 & \cdots & 1 \\
      \end{amat}
    \end{cases}
  \end{equation*}

  In the second case, we can actually continue past the \nameref{thm:Echelon_Form} and get the original \nameref{def:Matrix} to yield $I$.
  This is shown in \Cref{ex:Elementary Row Operations to find Inverse}
\end{theorem}

%%% Local Variables:
%%% mode: latex
%%% TeX-master: "../../Math_333-MatrixAlg_ComplexVars-Reference_Sheet"
%%% End:
