\subsection{Eigenvectors}\label{subsec:Eigenvectors}
\begin{definition}[Eigenvector]\label{def:Eigenvector}
  Let $A$ be an $n \by n$ matrix, denoted $A_{n \by n}$ and $X_{n \by 1}$ be a column vector of unknowns.
  $X_{n \by 1}$ is said to be an \emph{eigenvector} of $A$ if:

  \begin{equation}\label{eq:Eigenvector}
    AX = \lambda X
  \end{equation}

  \begin{propertylist}
  \item $X \neq 0$.\label{prop:Eigenvector_Nonzero}
  \item $\lambda$ is a scalar, called an \nameref{def:Eigenvalue}.\label{prop:Eigenvector_Value}.
  \end{propertylist}

  \begin{remark}[Uniqueness of Eigenvectors]
    There can be infinitely many \nameref{def:Eigenvector}s for a given \nameref{def:Eigenvalue}.
  \end{remark}
\end{definition}

\begin{example}[Lecture 17, Example 1]{Verify Vector is not an Eigenvector}
  Given $A$ and $Y$, is $Y$ an \nameref{def:Eigenvector} of $A$?
  \begin{align*}
    A &=
        \begin{pmatrix}
          2 & 3 \\
          6 & -1
        \end{pmatrix} &
                        Y &=
                            \begin{pmatrix}
                              2 \\
                              -1
                            \end{pmatrix}
  \end{align*}
  \tcblower{}
  First, we check that the two properties of an \nameref{def:Eigenvector} are satisfied.
  \Cref{prop:Eigenvector_Nonzero} is satisfied ($Y \neq 0$).

  Now, we attempt to find the \nameref{def:Eigenvalue} for this particular \nameref{def:Eigenvector}.
  \begin{align*}
    AY &=
         \begin{pmatrix}
           2 & 3 \\
           6 & -1
         \end{pmatrix}
               \begin{pmatrix}
                 2 \\
                 -1
               \end{pmatrix} \\
    &=
      \begin{pmatrix}
        1 \\
        13
      \end{pmatrix} \\
  \end{align*}

  The problem here is that there is \textbf{no} $\lambda$ such that the equation for an \nameref{def:Eigenvalue} is satisfied.
  \begin{equation*}
    \lambda
    \begin{pmatrix}
      2 \\
      -1
    \end{pmatrix}
    =
    \begin{pmatrix}
      1 \\
      13
    \end{pmatrix}
  \end{equation*}

  $\therefore$ $Y$ is not an \nameref{def:Eigenvector} of $A$.
\end{example}

\begin{example}[Lecture 17, Example 2]{Verify Vector is Eigenvector}
  Verify that $X$ is an \nameref{def:Eigenvector} of $A$?
  \begin{align*}
    A &=
        \begin{pmatrix}
          6 & -1 \\
          2 & 3
        \end{pmatrix} &
                        X &=
                            \begin{pmatrix}
                              1 \\
                              1
                            \end{pmatrix}
  \end{align*}
  \tcblower{}
  Verifying \Cref{prop:Eigenvector_Nonzero} is done by simple observation.

  Now, we need to attempt to find a $\lambda$ such that \Cref{eq:Eigenvector} is satisfied.
  \begin{align*}
    AX &=
         \begin{pmatrix}
           6 & -1 \\
           2 & 3
         \end{pmatrix}
               \begin{pmatrix}
                 1 \\
                 1
               \end{pmatrix} \\
    &=
      \begin{pmatrix}
        5 \\
        5
      \end{pmatrix} \\
    &= 5
      \begin{pmatrix}
        1 \\
        1
      \end{pmatrix} \\
    &= 5 X
  \end{align*}

  $\therefore$ $X$ \textit{is} an \nameref{def:Eigenvector} of $A$, with the \nameref{def:Eigenvalue} $\lambda = 5$.
\end{example}


\subsection{Eigenvalues}\label{subsec:Eigenvalues}

%%% Local Variables:
%%% mode: latex
%%% TeX-master: "../../Math_333-MatrixAlg_ComplexVars-Reference_Sheet"
%%% End:
