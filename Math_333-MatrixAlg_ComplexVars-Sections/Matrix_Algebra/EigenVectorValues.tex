\subsection{Eigenvectors}\label{subsec:Eigenvectors}
\begin{definition}[Eigenvector]\label{def:Eigenvector}
  Let $A$ be an $n \by n$ matrix, denoted $A_{n \by n}$ and $X_{n \by 1}$ be a column vector of unknowns.
  $X_{n \by 1}$ is said to be an \emph{eigenvector} of $A$ if:

  \begin{equation}\label{eq:Eigenvector}
    AX = \lambda X
  \end{equation}

  \begin{propertylist}
  \item $X \neq 0$.\label{prop:Eigenvector_Nonzero}
  \item $\lambda$ is a scalar, called an \nameref{def:Eigenvalue}.\label{prop:Eigenvector_Value}.
  \end{propertylist}

  \begin{remark}[Uniqueness of Eigenvectors]
    There can be infinitely many \nameref{def:Eigenvector}s for a given \nameref{def:Eigenvalue}.
  \end{remark}
\end{definition}

\begin{example}[Lecture 17, Example 1]{Verify Vector is not an Eigenvector}
  Given $A$ and $Y$, is $Y$ an \nameref{def:Eigenvector} of $A$?
  \begin{align*}
    A &=
        \begin{pmatrix}
          2 & 3 \\
          6 & -1
        \end{pmatrix} &
                        Y &=
                            \begin{pmatrix}
                              2 \\
                              -1
                            \end{pmatrix}
  \end{align*}
  \tcblower{}
  First, we check that the two properties of an \nameref{def:Eigenvector} are satisfied.
  \Cref{prop:Eigenvector_Nonzero} is satisfied ($Y \neq 0$).

  Now, we attempt to find the \nameref{def:Eigenvalue} for this particular \nameref{def:Eigenvector}.
  \begin{align*}
    AY &=
         \begin{pmatrix}
           2 & 3 \\
           6 & -1
         \end{pmatrix}
               \begin{pmatrix}
                 2 \\
                 -1
               \end{pmatrix} \\
    &=
      \begin{pmatrix}
        1 \\
        13
      \end{pmatrix} \\
  \end{align*}

  The problem here is that there is \textbf{no} $\lambda$ such that the equation for an \nameref{def:Eigenvalue} is satisfied.
  \begin{equation*}
    \lambda
    \begin{pmatrix}
      2 \\
      -1
    \end{pmatrix}
    =
    \begin{pmatrix}
      1 \\
      13
    \end{pmatrix}
  \end{equation*}

  $\therefore$ $Y$ is not an \nameref{def:Eigenvector} of $A$.
\end{example}

\begin{example}[Lecture 17, Example 2]{Verify Vector is Eigenvector}
  Verify that $X$ is an \nameref{def:Eigenvector} of $A$?
  \begin{align*}
    A &=
        \begin{pmatrix}
          6 & -1 \\
          2 & 3
        \end{pmatrix} &
                        X &=
                            \begin{pmatrix}
                              1 \\
                              1
                            \end{pmatrix}
  \end{align*}
  \tcblower{}
  Verifying \Cref{prop:Eigenvector_Nonzero} is done by simple observation.

  Now, we need to attempt to find a $\lambda$ such that \Cref{eq:Eigenvector} is satisfied.
  \begin{align*}
    AX &=
         \begin{pmatrix}
           6 & -1 \\
           2 & 3
         \end{pmatrix}
               \begin{pmatrix}
                 1 \\
                 1
               \end{pmatrix} \\
    &=
      \begin{pmatrix}
        5 \\
        5
      \end{pmatrix} \\
    &= 5
      \begin{pmatrix}
        1 \\
        1
      \end{pmatrix} \\
    &= 5 X
  \end{align*}

  $\therefore$ $X$ \textit{is} an \nameref{def:Eigenvector} of $A$, with the \nameref{def:Eigenvalue} $\lambda = 5$.
\end{example}

\begin{lemma}\label{lem:Matrix_Multiply_Zero_Vector}
  Let $B_{n \by n}$ be a \nameref{def:Matrix} with an \nameref{def:Inverse_Matrix} (it is invertible), meaning $B \neq 0$.
  Suppose $B X_{n \by 1} = 0$.

  Then, $X_{n \by 1} = 0$.

  \begin{remark*}
    This mirrors normal algebra, when $ab = 0$, and $a \neq 0$, then $b$ \textbf{must} be $0$.
  \end{remark*}
\end{lemma}

\begin{proof}[Proof of \Cref*{lem:Matrix_Multiply_Zero_Vector}]
  Suppose $BX = 0$.

  Then,
  \begin{align*}
    B^{-1} (BX) &= B^{-1} 0 \\
    \intertext{By the rules of \nameref{prop:Matrix_Associativity}.}
    (B^{-1} B) X &= 0 \\
    \intertext{By the definition of the \nameref{def:Multiplicative_Identity_Matrix}.}
    IX &= 0 \\
    X &= 0
  \end{align*}
\end{proof}

\begin{lemma}[Contrapositive of \Cref*{lem:Matrix_Multiply_Zero_Vector}]\label{lem:Contrapositive_Matrix_Multiply_Zero_Vector}
  If $BX = 0$ and $X \neq 0$, then $B$ has no inverse.
  Namely, this means that $\det B = 0$.
\end{lemma}

\subsection{Eigenvalues}\label{subsec:Eigenvalues}
\begin{definition}[Eigenvalue]\label{def:Eigenvalue}
  Let $A$ be an $n \by n$ matrix, denoted $A_{n \by n}$ and $X_{n \by 1}$ be a column vector of unknowns.
  If $X_{n \by 1}$ is an eigenvector of $A$, then
  \begin{equation}\label{eq:Eigenvalue}
    AX = \lambda X
  \end{equation}

  \begin{propertylist}
  \item $X \neq 0$.
  \item $\lambda$ is a scalar, called an \emph{Eigenvalue}.\label{prop:Eigenvalue}
  \end{propertylist}
\end{definition}

However, \Cref{eq:Eigenvalue} is difficult to work with when attempting to find the \nameref{def:Characteristic_Polynomial}.
So, we can perform some reductions that make it easier to use.
\begin{align*}
  AX &= \lambda X \\
  \shortintertext{Remember that $X \neq 0$.}
  AX - \lambda X &= 0 \\
  \intertext{We cannot factor $X$ as it is right now, because it multiplies two different things, a \nameref{def:Matrix} and a scalar.
  We can solve this by using the \nameref{def:Multiplicative_Identity_Matrix}.}
  AX - \lambda I X &= 0 \\
  (A - \lambda I) X &= 0 \\
  \intertext{Now, using \Cref{lem:Contrapositive_Matrix_Multiply_Zero_Vector}, $\det B = 0$.}
  \det(A-\lambda I) &= 0 \\
  \shortintertext{Where $lambda$ is an \nameref{def:Eigenvalue}.}
\end{align*}

\begin{definition}[Characteristic Polynomial]\label{def:Characteristic_Polynomial}
  Consider $\det(A_{n \by n}-\lambda I_{n \by n})$ where $A$ is indeterminant.
  This will form a polynomial of degree $n$ in $\lambda$, the \emph{characteristic polynomial}.

  If $\lambda$ \emph{is} an \nameref{def:Eigenvalue}, then $\det(A-\lambda I) = 0$.
\end{definition}

\begin{example}[Lecture 17, Example 4]{Find all Eigenvalues and Eigenvectors}
  Find all \nameref{def:Eigenvalue}s and \nameref{def:Eigenvector}s of $A$?
  \begin{equation*}
    A =
    \begin{pmatrix}
      2 & 3 & 6 \\
      0 & 4 & 4 \\
      0 & 0 & 2
    \end{pmatrix}
  \end{equation*}
  \tcblower{}
  Start by the \nameref{def:Characteristic_Polynomial} of $A$.
  \begin{align*}
    A &=
    \begin{pmatrix}
      2 & 3 & 6 \\
      0 & 4 & 4 \\
      0 & 0 & 2
    \end{pmatrix} \\
    A - \lambda I &=
                    \begin{pmatrix}
                      2 - \lambda & 3 & 6 \\
                      0 & 4 - \lambda & 4 \\
                      0 & 0 & 2 - \lambda \\
                    \end{pmatrix} \\
    \intertext{This \nameref{def:Matrix} is \nameref{def:Upper_Triangular}.}
    \det(A - \lambda I) &= (2 - \lambda) (4 - \lambda) (2 - \lambda)
  \end{align*}

  With $A$'s \nameref{def:Characteristic_Polynomial}, we can find its roots to determine the \nameref{def:Eigenvalue}s for $A$.
  As we can see, there are two possible eigenvalues:
  \begin{description}[noitemsep]
  \item[$\lambda = 2$] \nameref{def:Algebraic_Multiplicity} 2
  \item[$\lambda = 4$] \nameref{def:Algebraic_Multiplicity} 1, or simple.
  \end{description}

  Now we need to find all the possible \nameref{def:Eigenvector}s for these \nameref{def:Eigenvalue}s.
  Starting with $\lambda = 2$:
  \begin{align*}
    A - 4I &=
             \begin{pmatrix}
               0 & 3 & 6 \\
               0 & 2 & 4 \\
               0 & 0 & 0
             \end{pmatrix} \\
    (A-\lambda I) X &= 0 \\
    \begin{pmatrix}
      0 & 3 & 6 \\
      0 & 2 & 4 \\
      0 & 0 & 0
    \end{pmatrix}
              \begin{pmatrix}
                x_{1} \\
                x_{2} \\
                x_{3}
              \end{pmatrix}
    &=
      \begin{pmatrix}
        0 \\
        0 \\
        0
      \end{pmatrix}
  \end{align*}

  This yields a system of equations that we can solve.
  \begin{align*}
    3x_{2} + 6x_{3} &= 0 \\
    2x_{2} + 4x_{3} &= 0 \\
    \intertext{We can see that each equation is just a multiple of a simpler equation.}
    3 (x_{2} + 2x_{3}) &= 0 \\
    2 (x_{2} + 2x_{3}) &= 0 \\
    \shortintertext{Now, solving this system.}
    x_{2} &= -2x_{3}
  \end{align*}

  Now, we put it back into a solution matrix.
  \begin{align*}
    \begin{pmatrix}
      x_{1} \\
      x_{2} \\
      x_{3}
    \end{pmatrix} &=
                    \begin{pmatrix}
                      x_{1} \\
                      -2x_{3} \\
                      x_{3}
                    \end{pmatrix} \\
    \intertext{Now, because of superposition, we can say}
    &=
      \begin{pmatrix}
        x_{1} + 0x_{3} \\
        0x_{1} - 2x_{3} \\
        0x_{1} + x_{3}
      \end{pmatrix} \\
    &=
      \begin{pmatrix}
        x_{1} \\
        0 \\
        0
      \end{pmatrix} +
    \begin{pmatrix}
      0 \\
      -2x_{3} \\
      x_{3}
    \end{pmatrix} \\
    &= x_{1}
      \begin{pmatrix}
        1 \\
        0 \\
        0 \\
      \end{pmatrix} +
    x_{3}
    \begin{pmatrix}
      0 \\
      -2 \\
      1
    \end{pmatrix}
  \end{align*}
  Where both $x_{1}$ and $x_{3}$ cannot be $0$ at the same time.

  We need two matrices to find all \nameref{def:Eigenvector}s of the \nameref{def:Eigenvalue} $\lambda = 2$.
  Thus, the \nameref{def:Eigenvalue} $\lambda = 2$ has a \nameref{def:Geometric_Multiplicity} of 2.

  Similarly, we can solve for the \nameref{def:Eigenvalue} $\lambda = 4$, which will have \nameref{def:Geometric_Multiplicity} 1.
\end{example}


%%% Local Variables:
%%% mode: latex
%%% TeX-master: "../../Math_333-MatrixAlg_ComplexVars-Reference_Sheet"
%%% End:
