\subsection{Gaussian Elimination}\label{subsec:Gaussian_Elimination}
\begin{definition}[Gaussian Elimination]\label{def:Gaussian_Elimination}
  \emph{Gaussian elimination} is a method of solving a system of linear equations using matrices and \nameref{def:Elementary_Row_Op}s.
\end{definition}

It is easier to show how \nameref{def:Gaussian_Elimination} works through an example.

\begin{example}[Lecture 12, Example 3]{Perform Gaussian Elimination}
  Solve the system of linear equations below.
  \begin{align*}
    x - 2y + 3z &= 5 \\
    2x + y - z &= 8 \\
    3x - y + 2z &= 13
  \end{align*}
  \tcblower{}
  Start by converting the system of linear equations to an \nameref{def:Augmented_Matrix}.
  \begin{align*}
    \begin{array}{cccl}
      1x &- 2y &+ 3z &= 5 \\
      2x &+ 1y &- 1z &= 8 \\
      3x &- 1y &+ 2z &= 13 \\
    \end{array}
         &=
           \begin{amat}{3}
             1 & -2 & 3 & 5 \\
             2 & 1 & -1 & 8 \\
             3 & -1 & 2 & 13 \\
           \end{amat} \\
    \intertext{Use repeated \nameref{def:Elementary_Row_Op}s to make lower rows have zeros, converting to \nameref{thm:Echelon_Form}.}
         &\grstep{-2r_{1}+r_{2}}\begin{amat}{3}
           1 & -2 & 3 & 5 \\
           0 & 5 & -7 & -2 \\
           3 & -1 & 2 & 13 \\
         \end{amat} \\
         &\grstep{-3r_{1}+r_{3}}\begin{amat}{3}
           1 & -2 & 3 & 5 \\
           0 & 5 & -7 & -2 \\
           0 & 5 & -7 & -2 \\
         \end{amat} \\
         &\grstep{\frac{1}{5}r_{2}}\begin{amat}{3}
           1 & -2 & 3 & 5 \\
           0 & 1 & \frac{-7}{5} & \frac{-2}{5} \\
           0 & 5 & -7 & -2 \\
         \end{amat} \\
         &\grstep{-5r_{2}+r_{3}}\begin{amat}{3}
           1 & -2 & 3 & 5 \\
           0 & 1 & \frac{-7}{5} & \frac{-2}{5} \\
           0 & 0 & 0 & 0 \\
         \end{amat} \\
  \end{align*}

  Now, we can reconvert back into a system of linear equations.
  \begin{equation*}
    \begin{amat}{3}
      1 & -2 & 3 & 5 \\
      0 & 1 & \frac{-7}{5} & \frac{-2}{5} \\
      0 & 0 & 0 & 0 \\
    \end{amat} \\
    =
    \begin{array}{cccl}
      1x &- 2y &+ 3z &= 5 \\
      0x &+ 1y &- \frac{7}{5}z &= \frac{-2}{5} \\
      0x &+ 0y &+ 0z &= 0 \\
    \end{array}
  \end{equation*}

  We now have 3 unknowns, but only 2 usable equations.
  Therefore, we solve for two of the unknowns in terms of a third.
  \begin{align*}
    y &= \frac{-2}{5} + \frac{7}{5}z \\
    x &= \frac{21}{5} - \frac{z}{5}
  \end{align*}
\end{example}

\begin{definition}[Consistent]\label{def:Consistent}
  A \emph{consistent} system of linear equations is one which has a solution.
\end{definition}


%%% Local Variables:
%%% mode: latex
%%% TeX-master: "../../Math_333-MatrixAlg_ComplexVars-Reference_Sheet"
%%% End:
