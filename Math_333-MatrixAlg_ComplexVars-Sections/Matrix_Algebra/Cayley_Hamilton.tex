\subsection{Cayley-Hamilton Theorem}\label{subsec:Cayley-Hamilton_Theorem}
\begin{theorem}[Cayley-Hamilton Theorem]\label{thm:Cayley-Hamilton_Theorem}
  A square \nameref{def:Matrix} $A_{n \by n}$ satisfies its own \nameref{def:Characteristic_Polynomial}.
  \begin{remark*}
    ``Satisfies'' in this context means that the \nameref{def:Characteristic_Polynomial} will return $0$ when the variable $\lambda$ is replaced by a value.
  \end{remark*}
\end{theorem}

\begin{example}[Lecture 17, Example 5]{Verify the Cayley-Hamilton Theorem}
  Verify the \nameref{thm:Cayley-Hamilton_Theorem} for the \nameref{def:Matrix} $A$?
  \begin{equation*}
    A =
    \begin{pmatrix}
      6 & -1 \\
      2 & 3
    \end{pmatrix}
  \end{equation*}
  \tcblower{}
  We need to find $A$'s \nameref{def:Characteristic_Polynomial}.
  \begin{align*}
    A - \lambda I &=
                    \begin{pmatrix}
                      6 - \lambda & -1 \\
                      2 & 3 - \lambda
                    \end{pmatrix} \\
    \det(A - \lambda I) &= (6-\lambda) (3-\lambda) - (-1)(2) \\
                  &= \lambda^{2} - 9\lambda + 20
  \end{align*}

  The \nameref{thm:Cayley-Hamilton_Theorem} asserts:
  \begin{align*}
    A^{2} - 9 A + 20I &= 0 \\
    \begin{pmatrix}
      6 & -1 \\
      2 & 3
    \end{pmatrix}
          \begin{pmatrix}
            6 & -1 \\
            2 & 3
          \end{pmatrix} -
                9
                \begin{pmatrix}
                  6 & -1 \\
                  2 & 3
                \end{pmatrix} +
                      \begin{pmatrix}
                        20 & 0 \\
                        0 & 20
                      \end{pmatrix}
        &\qeq 0 \\
    \begin{pmatrix}
      34 & -9 \\
      18 & 7
    \end{pmatrix} -
           \begin{pmatrix}
             54 & -9 \\
             18 & 27
           \end{pmatrix} +
                  \begin{pmatrix}
                    20 & 0 \\
                    0 & 20
                  \end{pmatrix}
                        &\qeq
                          \begin{pmatrix}
                            0 & 0 \\
                            0 & 0
                          \end{pmatrix} \\
    \begin{pmatrix}
      -20 & 0 \\
      0 & -20
    \end{pmatrix} +
          \begin{pmatrix}
            20 & 0 \\
            0 & 20
          \end{pmatrix} &\qeq
                          \begin{pmatrix}
                            0 & 0 \\
                            0 & 0
                          \end{pmatrix} \\
    \begin{pmatrix}
      0 & 0 \\
      0 & 0
    \end{pmatrix} &\overset{\checkmark}{=}
                    \begin{pmatrix}
                      0 & 0 \\
                      0 & 0
                    \end{pmatrix}
  \end{align*}
\end{example}

\subsubsection{Express Higher Powers}\label{subsubsec:Cayley-Hamilton_Express_Higher_Powers}
The \nameref{thm:Cayley-Hamilton_Theorem} states that if $\lambda = A$, then the \nameref{def:Characteristic_Polynomial} will equal 0.
If that is the case, then we can solve for higher powers of the polynomial using repeated substitution.

It is easiest to show this with an example, \Cref{ex:Cayley-Hamilton Higher Powers}.

\begin{example}[Lecture 18, Example 1]{Cayley-Hamilton Higher Powers}
  Given the \nameref{def:Matrix} $A =
  \begin{pmatrix}
    2 & -2 \\
    1 & 5 \\
  \end{pmatrix}$, find all $A^{n}$?
  We start by finding the \nameref{def:Characteristic_Polynomial}.
  \begin{align*}
    \det (A - \lambda I) &= \det
                           \begin{pmatrix}
                             2-\lambda & -2 \\
                             1 & 5 - \lambda
                           \end{pmatrix} \\
                         &= (2-\lambda)(5 - \lambda) - (-2)(1) \\
                         &= \lambda^{2}- 7\lambda + 12
  \end{align*}

  If $A$ satistifies the \nameref{thm:Cayley-Hamilton_Theorem}, then $A^{2} - 7A + 12I = 0$ will hold true.
  In that case, we can express exponents greater than 1 like so:
  \begin{align*}
    A^{2} &= 7A - 12I \\
    A^{3} &= A A^{2} \\
          &= A (7A - 12I) \\
          &= 7A^{2} - 12A \\
          &= 7(7A-12I) - 12A \\
          &= 49A - 84I - 12A \\
          &= 37A - 84I
  \end{align*}

  This computation can be repeated for higher powers, and for equations whigh higher starting characteristic equation exponents.
\end{example}

If we generalize, we can see that any exponent power of a \nameref{def:Matrix} $A$ can be expressed as a linear combination of $A$ and $I$.
\begin{equation}\label{eq:Matrix_Constant_Time_Higher_Power}
  A^{n} = c_{n} A + d_{n} I
\end{equation}

Similar to how \Cref{eq:Matrix_Constant_Time_Higher_Power} exists, a similar equation exists for the \nameref{def:Eigenvalue}s of a given \nameref{def:Matrix}, seen in \Cref{eq:Eigenvalue_Constant_Time_Higher_Power}.
\begin{equation}\label{eq:Eigenvalue_Constant_Time_Higher_Power}
  \lambda^{n} = c_{n} \lambda + d_{n}
\end{equation}

If we solve for $c_{n}$ and $d_{n}$ in \Cref{eq:Eigenvalue_Constant_Time_Higher_Power}, using Cramer's Rule, we can find general forms of $c_{n}$ and $d_{n}$.

For the example in \Cref{ex:Cayley-Hamilton Higher Powers}, if we plug in $\lambda = 3$ and $\lambda = 4$, then we have two equationsto work with.
\begin{align*}
  3 c_{n} + d_{n} &= 3^{n} \\
  4c_{n} + d_{n} &= 3^{n}
\end{align*}

Using Cramer's Rule, we can solve for $c_{n}$ and $d_{n}$.
\begin{align*}
  c_{n} &= \frac{\det
          \begin{pmatrix}
            3^{n} & 1 \\
            4 ^{n} & 1
          \end{pmatrix}}{\det
                     \begin{pmatrix}
                       3 & 1 \\
                       3 & 1
                     \end{pmatrix}} \\
  d_{n} &= \frac{\det
          \begin{pmatrix}
            3 & 3^{n} \\
            4 & 4^{n}
          \end{pmatrix}}{\det
                \begin{pmatrix}
                  3 & 1 \\
                  4 & 1
                \end{pmatrix}
                      } \\
  \shortintertext{Now, solving for these scalars:}
  c_{n} &= 4^{n} - 3^{n} \\
  d_{n} &= 3 \bigl( 3^{n} \bigr) - 3 \bigl( 4^{n} \bigr)
\end{align*}

%%% Local Variables:
%%% mode: latex
%%% TeX-master: "../../Math_333-MatrixAlg_ComplexVars-Reference_Sheet"
%%% End:
