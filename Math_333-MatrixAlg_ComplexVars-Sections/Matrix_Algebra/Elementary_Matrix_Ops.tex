\subsection{Elementary Matrix Operations}\label{subsec:Elementary_Matrix_Ops}
\begin{definition}[Elementary Matrix]\label{def:Elementary_Matrix}
  An \emph{elementary matrix} is a \nameref{def:Matrix} obtained by performing an \nameref{def:Elementary_Row_Op} on the \nameref{def:Multiplicative_Identity_Matrix}.
  This allows us to encode the action we take on the identity matrix \textbf{into} the identity matrix, essentially saving it.

  \begin{remark}[Elementary Column Matrix]\label{rmk:Elementary_Column_Matrix}
    There exists a corresponding class of elementary matrices where \nameref{def:Elementary_Column_Op}s are encoded on $I$.
  \end{remark}
\end{definition}

\begin{blackbox}
  Given the \nameref{def:Multiplicative_Identity_Matrix}, $I$, a regular \nameref{def:Matrix} $A$, perform a row reduction by adding $r_{1}$, multiplied by $\lambda$, of $A$ to $r_{2}$.
  \begin{align*}
    I &= \begin{bmatrix}
      1 & 0 \\
      0 & 1
    \end{bmatrix} \\
    A &= \begin{bmatrix}
      a_{11} & a_{12} \\
      a_{21} & a_{22}
    \end{bmatrix}
  \end{align*}

  We perform the \nameref{def:Elementary_Row_Op} on the \nameref{def:Multiplicative_Identity_Matrix}.
  \begin{equation*}
    \begin{bmatrix}
      1 & 0 \\
      0 & 1
    \end{bmatrix}
    \grstep{\lambda r_{1}+r_{2}}
    \begin{bmatrix}
      1 & 0 \\
      \lambda & 1
    \end{bmatrix}
  \end{equation*}

  Now that we have an \nameref{def:Elementary_Matrix}, we can perform the action we saved in the elementary matrix on any matrix, in this case, $A$.
  \begin{equation*}
    \begin{bmatrix}
      1 & 0 \\
      \lambda & 1
    \end{bmatrix}
    \begin{bmatrix}
      a_{11} & a_{12} \\
      a_{21} & a_{22}
    \end{bmatrix}
    =
    \begin{bmatrix}
      a_{11} & a_{12} \\
      \lambda a_{11} + a_{21} & \lambda a_{12} + a_{22}
    \end{bmatrix}
  \end{equation*}

  We can see that this result is the exact same as if we have performed the \nameref{def:Elementary_Row_Op} on the matrix $A$ directly.
\end{blackbox}

Now, we can move onto an example.

\begin{example}[Lecture 13, Example 2]{Using Elementary Matrices}
  Given the matrix $A =
  \begin{bmatrix}
    1 & 2 & 3 \\
    2 & 1 & 5 \\
    3 & 3 & 8
  \end{bmatrix}$, find an \nameref{def:Elementary_Matrix}, $E$, such that $EA =
  \begin{bmatrix}
    1 & 2 & 3 \\
    2 & 1 & 5 \\
    0 & 0 & 0
  \end{bmatrix}$?
  \tcblower{}
  If we look at $A$ for a minute, we notice that $r_{3} = r_{1} + r_{2}$.
  This will actually help us ensure our answer is correct.
  \begin{align*}
    \begin{bmatrix}
      1 & 2 & 3 \\
      2 & 1 & 5 \\
      3 & 3 & 8
    \end{bmatrix}
        &\grstep{-r_{1}+r_{3}}
          \begin{bmatrix}
            1 & 2 & 3 \\
            2 & 1 & 5 \\
            2 & 1 & 5
          \end{bmatrix} \\
    \intertext{So, we can encode the \nameref{def:Elementary_Row_Op} in an \nameref{def:Elementary_Matrix}.}
    E_{1} &=
            \begin{bmatrix}
              1 & 0 & 0 \\
              0 & 1 & 0 \\
              -1 & 0 & 1 \\
            \end{bmatrix} \\
    \intertext{Now row operate on the last remaining row.}
    \begin{bmatrix}
            1 & 2 & 3 \\
            2 & 1 & 5 \\
            2 & 1 & 5
          \end{bmatrix}
        &\grstep{-r_{2}+r_{3}}
          \begin{bmatrix}
            1 & 2 & 3 \\
            2 & 1 & 5 \\
            0 & 0 & 0
          \end{bmatrix} \\
    \intertext{Encode this second operation as another \nameref{def:Elementary_Matrix}.}
    E_{2} &=
            \begin{bmatrix}
              1 & 0 & 0 \\
              0 & 1 & 0 \\
              0 & -1 & 1
            \end{bmatrix}
  \end{align*}

  That means we can describe the relationship as
  \begin{equation*}
    E_{2}E_{1}A =
    \begin{bmatrix}
      1 & 2 & 3 \\
      2 & 1 & 5 \\
      0 & 0 & 0
    \end{bmatrix}
  \end{equation*}

  Thus, $E = E_{2}E_{1}$.
  \begin{align*}
    E &=
        \begin{bmatrix}
          1 & 0 & 0 \\
          0 & 1 & 0 \\
          0 & -1 & 1
        \end{bmatrix}
                   \begin{bmatrix}
                     1 & 0 & 0 \\
                     0 & 1 & 0 \\
                     -1 & 0 & 1 \\
                   \end{bmatrix} \\
    \intertext{But, this multiplication sucks. But, because we encoded a \nameref{def:Elementary_Row_Op}, we can perform the row operation specified in $E_{2}$ on $E_{1}$.}
    &\grstep{-r_{2}+r_{3}} \begin{bmatrix}
                     1 & 0 & 0 \\
                     0 & 1 & 0 \\
                     -1 & 0 & 1 \\
                   \end{bmatrix} \\
      &=
        \begin{bmatrix}
          1 & 0 & 0 \\
          0 & 1 & 0 \\
          -1 & -1 & 1
        \end{bmatrix}
  \end{align*}

  Thus, we have found our final answer for $E$.
\end{example}

%%% Local Variables:
%%% mode: latex
%%% TeX-master: "../../Math_333-MatrixAlg_ComplexVars-Reference_Sheet"
%%% End:
