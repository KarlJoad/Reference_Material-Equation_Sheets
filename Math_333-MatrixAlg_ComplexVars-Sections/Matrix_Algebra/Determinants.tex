\subsection{Determinants}\label{subsec:Determinants}
Before calculating the \nameref{def:Determinant}, we need to have some additional vocabulary wto work with.

\begin{definition}[Minor]\label{def:Minor}
  The \emph{minor} $\Minor{i}{j}$ is the \nameref{def:Determinant} of the sub\nameref{def:Matrix} obtained by deleting the $i$th row and $j$th column.
\end{definition}

\begin{blackbox}
  Given the matrix $
  \begin{pmatrix}
    1 & -1 & 3 \\
    2 & 5 & -1 \\
    3 & 0 & 5
  \end{pmatrix}$, the \nameref{def:Minor} at the index $3,2$ is found by deleting the \nth{3} row and \nth{2} column and taking the \nameref{def:Determinant} of the resulting submatrix.
  Thus,
  \begin{align*}
    \Minor{3}{2} &= \det
                   \begin{pmatrix}
                     1 & 3 \\
                     2 & -1
                   \end{pmatrix} \\
    \intertext{Use the definition of the \nameref{def:Determinant} for a $2 \by 2$ \nameref{def:Matrix}.}
                 &= 1(-1) - 3(2) \\
                 &= -1 - 6 \\
                 &= -7
  \end{align*}
\end{blackbox}

\begin{definition}[Cofactor]\label{def:Cofactor}
  The \emph{cofactor} $\Cofactor{i}{j}$ is related to the \nameref{def:Minor}, $\Minor{i}{j}$ by the equation below.
  \begin{equation}\label{eq:Cofactor}
    \Cofactor{i}{j} = {(-1)}^{i + j} \Minor{i}{j}
  \end{equation}
\end{definition}

\begin{blackbox}
  Using the same \nameref{def:Matrix} from earlier, the \nameref{def:Cofactor} of of the element in the \nth{3} row and \nth{2} column is:
  \begin{align*}
    \Cofactor{3}{2} &= {(-1)}^{3+2} \Minor{3}{2} \\
                    &= (-1) (-7) \\
                    &= 7
  \end{align*}
\end{blackbox}

Now with these terms defined, we can define the general algorithm for the \nameref{def:Determinant}.

\begin{definition}[Determinant]\label{def:Determinant}
  The \emph{determinant} of a \nameref{def:Matrix} is a scalar value that is computed out of a \textbf{square} matrix and encodes certain properties of the transformation specified by the matrix.

  There are 2 equations in use for the determinant, depending on the size of the matrix.
  For a $2 \by 2$ matrix,
  \begin{equation}\label{eq:Determinant_2x2}
    \det
    \begin{pmatrix}
      a & b \\
      c & d
    \end{pmatrix}
    = ad - bc
  \end{equation}

  For a matrix larger than $2 \by 2$, we have many possible ways of finding the determinant.
  This is explored further in \Cref{subsubsec:Expand_Determinant}, with their equations given in \Cref{eq:Determinant_Expand_ith_Row} and \Cref{eq:Determinant_Expand_jth_Column}.
\end{definition}

There is something very critical to remember about the \nameref{def:Determinant}, and it applies to \Cref{thm:Determinant_Invertible_Matrix}.
\begin{theorem}\label{thm:Determinant_Invertible_Matrix}
  A \nameref{def:Matrix} $A$ is invertible (has an \nameref{def:Inverse_Matrix}) \textbf{if and only if} $\det A = 0$.
  If this is the case, then:
  \begin{equation}\label{eq:Inverse_by_Determinant}
    A^{-1} = \frac{1}{\det A} \Adjoint(A)
  \end{equation}

  Where $\Adjoint$ is the \nameref{def:Adjoint}.
\end{theorem}

\begin{definition}[Adjoint]\label{def:Adjoint}
  The \emph{adjoint} of a \nameref{def:Matrix} $A$ is the matrix of \nameref{def:Cofactor}s of $A$ \nameref{def:Transpose}, $\Transpose{A}$.
\end{definition}


%%% Local Variables:
%%% mode: latex
%%% TeX-master: "../../Math_333-MatrixAlg_ComplexVars-Reference_Sheet"
%%% End:
