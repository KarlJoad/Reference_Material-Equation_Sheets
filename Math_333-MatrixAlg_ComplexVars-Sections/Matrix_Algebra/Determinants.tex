\subsection{Determinants}\label{subsec:Determinants}
Before calculating the \nameref{def:Determinant}, we need to have some additional vocabulary wto work with.

\begin{definition}[Minor]\label{def:Minor}
  The \emph{minor} $\Minor{i}{j}$ is the \nameref{def:Determinant} of the sub\nameref{def:Matrix} obtained by deleting the $i$th row and $j$th column.
\end{definition}

\begin{blackbox}
  Given the matrix $
  \begin{pmatrix}
    1 & -1 & 3 \\
    2 & 5 & -1 \\
    3 & 0 & 5
  \end{pmatrix}$, the \nameref{def:Minor} at the index $3,2$ is found by deleting the \nth{3} row and \nth{2} column and taking the \nameref{def:Determinant} of the resulting submatrix.
  Thus,
  \begin{align*}
    \Minor{3}{2} &= \det
                   \begin{pmatrix}
                     1 & 3 \\
                     2 & -1
                   \end{pmatrix} \\
    \intertext{Use the definition of the \nameref{def:Determinant} for a $2 \by 2$ \nameref{def:Matrix}.}
                 &= 1(-1) - 3(2) \\
                 &= -1 - 6 \\
                 &= -7
  \end{align*}
\end{blackbox}

\begin{definition}[Cofactor]\label{def:Cofactor}
  The \emph{cofactor} $\Cofactor{i}{j}$ is related to the \nameref{def:Minor}, $\Minor{i}{j}$ by the equation below.
  \begin{equation}\label{eq:Cofactor}
    \Cofactor{i}{j} = {(-1)}^{i + j} \Minor{i}{j}
  \end{equation}
\end{definition}

\begin{blackbox}
  Using the same \nameref{def:Matrix} from earlier, the \nameref{def:Cofactor} of of the element in the \nth{3} row and \nth{2} column is:
  \begin{align*}
    \Cofactor{3}{2} &= {(-1)}^{3+2} \Minor{3}{2} \\
                    &= (-1) (-7) \\
                    &= 7
  \end{align*}
\end{blackbox}

Now with these terms defined, we can define the general algorithm for the \nameref{def:Determinant}.

\begin{definition}[Determinant]\label{def:Determinant}
  The \emph{determinant} of a \nameref{def:Matrix} is a scalar value that is computed out of a \textbf{square} matrix and encodes certain properties of the transformation specified by the matrix.

  There are 2 equations in use for the determinant, depending on the size of the matrix.
  For a $2 \by 2$ matrix,
  \begin{equation}\label{eq:Determinant_2x2}
    \det
    \begin{pmatrix}
      a & b \\
      c & d
    \end{pmatrix}
    = ad - bc
  \end{equation}

  For a matrix larger than $2 \by 2$, we have many possible ways of finding the determinant.
  This is explored further in \Cref{subsubsec:Expand_Determinant}, with their equations given in \Cref{eq:Determinant_Expand_ith_Row} and \Cref{eq:Determinant_Expand_jth_Column}.
\end{definition}

There is something very critical to remember about the \nameref{def:Determinant}, and it applies to \Cref{thm:Determinant_Invertible_Matrix}.
\begin{theorem}\label{thm:Determinant_Invertible_Matrix}
  A \nameref{def:Matrix} $A$ is invertible (has an \nameref{def:Inverse_Matrix}) \textbf{if and only if} $\det A = 0$.
  If this is the case, then:
  \begin{equation}\label{eq:Inverse_by_Determinant}
    A^{-1} = \frac{1}{\det A} \Adjoint(A)
  \end{equation}

  Where $\Adjoint$ is the \nameref{def:Adjoint}.
\end{theorem}

\begin{definition}[Adjoint]\label{def:Adjoint}
  The \emph{adjoint} of a \nameref{def:Matrix} $A$ is the matrix of \nameref{def:Cofactor}s of $A$ \nameref{def:Transpose}, $\Transpose{A}$.
\end{definition}

\subsubsection{Expanding the Determinant by Row or Column}\label{subsubsec:Expand_Determinant}
To calculate the \nameref{def:Determinant} of a \nameref{def:Matrix} larger than $2 \by 2$, you must ``expand the determinant.''

\begin{remark*}
  You can perform this operation on any row or column you wish, you will receive the same anser.
  However, to make your life easier, it is best to pick the row or column that contains the most zeros.
\end{remark*}

To expand the \nameref{def:Determinant} of a \nameref{def:Matrix} $A_{n \by n}$ by the $i$th row the following equation follows:
\begin{equation}\label{eq:Determinant_Expand_ith_Row}
  \det A = a_{i,1} \Cofactor{i}{1} + a_{i,2} \Cofactor{i}{2} + \cdots + a_{i,n} \Cofactor{i}{n}
\end{equation}

To expand the \nameref{def:Determinant} of a \nameref{def:Matrix} $A_{n \by n}$ by the $j$th column the following equation follows:
\begin{equation}\label{eq:Determinant_Expand_jth_Column}
  \det A = a_{1,j} \Cofactor{1}{j} + a_{2,j} \Cofactor{2}{j} + \cdots + a_{n,j} \Cofactor{n}{j}
\end{equation}

\begin{example}[Lecture 16, Example 3]{Find Determinant}
  Find the \nameref{def:Determinant} of the $3 \by 3$ \nameref{def:Matrix} $A$ below, by expanding the first column?
  \begin{equation*}
    A =
    \begin{pmatrix}
      1 & -1 & 3 \\
      2 & -4 & 5 \\
      0 & 1 & 5
    \end{pmatrix}
  \end{equation*}
  \tcblower{}
  Using \Cref{eq:Determinant_Expand_jth_Column}, we can solve this.
  \begin{align*}
    \det A &= a_{1,j} \Cofactor{1}{j} + a_{2,j} \Cofactor{2}{j} + \cdots + a_{n,j} \Cofactor{n}{j} \\
           &= a_{1,1} \Cofactor{1}{1} + a_{2,1} \Cofactor{2}{1} + a_{3,1} \Cofactor{3}{1} \\
           &= 1 \left( {(-1)}^{1+1} \Minor{1}{1} \right) + 2 \left( {(-1)}^{2+1} \Minor{2}{1} \right) + 0 \left( {(-1)}^{3+1} \Minor{3}{1} \right) \\
           &= 1 (1) (\det A) + 2 (-1) (\det A) \\
    &= \det
      \begin{pmatrix}
        -4 & 5 \\
        1 & 5
      \end{pmatrix}
            -2 \det
            \begin{pmatrix}
              -1 & 3 \\
              1 & 5
            \end{pmatrix} \\
    \intertext{Now that the matrices are $2 \by 2$, we can use \Cref{eq:Determinant_2x2}.}
           &= (-4(5) - 5(1)) - 2(-1(5) - 3(1)) \\
           &= -20 - 5 - 2(-8) \\
           &= -25 + 16 \\
           &= -9
  \end{align*}

  We could have solved this by using the first row, in which case our equation to solve would have looked like so:
  \begin{equation*}
    \det A = a_{1,1} \Cofactor{1}{1} + a_{1,2} \Cofactor{1}{2} + \cdots + a_{1,3} \Cofactor{1}{3}
  \end{equation*}

  If we wanted to expand by the second row, our equation would have looked like:
  \begin{equation*}
    \det A = a_{2,1} \Cofactor{2}{1} + a_{2,2} \Cofactor{2}{2} + \cdots + a_{2,3} \Cofactor{2}{3}
  \end{equation*}
\end{example}

From our example, it is obvious that we want to expand the determinant through a row or column with the maximum number of zeros.

\subsubsection{Create Zeros for Determinants}\label{subsubsec:Create_0s_Determinants}
The \nameref{def:Elementary_Row_Op}s/\nameref{def:Elementary_Column_Op}s that we have learned about \textbf{do} change the \nameref{def:Determinant} of a \nameref{def:Matrix} in predictable ways.

\begin{description}[noitemsep]
\item[Interchange Rows:] Sign of the \nameref{def:Determinant} flips.
\item[Scalar Multiply:] The \nameref{def:Determinant} scales by the same amount.
\item[Multiply Row and Add:] No change.
\end{description}

\begin{example}[Lecture 16, Example 3]{Create Zeros to Simplify Determinant}
  Find the \nameref{def:Determinant} of the \nameref{def:Matrix} $A$?
  \begin{equation*}
    A =
    \begin{pmatrix}
      1 & 2 & -1 & 3 \\
      2 & -1 & 0 & 1 \\
      1 & 1 & 2 & 3 \\
      -1 & 1 & 2 & 5
    \end{pmatrix}
  \end{equation*}
  \tcblower{}
  We want to create zeros in as many places as possible.
  Let's start by expanding by the first column.
  \begin{equation*}
    \begin{pmatrix}
      1 & 2 & -1 & 3 \\
      2 & -1 & 0 & 1 \\
      1 & 1 & 2 & 3 \\
      -1 & 1 & 2 & 5
    \end{pmatrix}
  \grstep{\substack{-2r_{1}+r_{2} \\ -r_{1}+r_{3} \\ r_{1}+r_{4}}}
    \begin{pmatrix}
      1 & 2 & -1 & 3 \\
      0 & -5 & 2 & -5 \\
      0 & -1 & 3 & 0 \\
      0 & 3 & 1 & 8
    \end{pmatrix}
  \end{equation*}

  With all these zeros in the first column now, we can apply \Cref{eq:Determinant_Expand_jth_Column}.
  \begin{align*}
    \det A &= 1 \Cofactor{1}{1} + 0 \Cofactor{2}{1} + 0 \Cofactor{3}{1} + 0 \Cofactor{4}{1} \\
           &= {(-1)}^{1+1} \Minor{1}{1} \\
           &= \Minor{1}{1} \\
    &= \det
      \begin{pmatrix}
        -5 & 2 & -5 \\
        -1 & 3 & 0 \\
        3 & 1 & 8
      \end{pmatrix} \\
    \shortintertext{Interchange $r_{1}$ and $r_{2}$, so flip the sign of the determinant.}
           &= - \det
             \begin{pmatrix}
               -1 & 3 & 0 \\
               -5 & 2 & -5 \\
               3 & 1 & 8
             \end{pmatrix} \\
    \shortintertext{Factor a $-1$ out from the first column.}
    &= \det
      \begin{pmatrix}
        1 & 3 & 0 \\
        5 & 2 & -5 \\
        -3 & 1 & 8
      \end{pmatrix}
  \end{align*}

  Now, we perform \nameref{def:Elementary_Column_Op}s to make the second column have as mnay zeros as possible.
  \begin{equation*}
    \begin{pmatrix}
        1 & 3 & 0 \\
        5 & 2 & -5 \\
        -3 & 1 & 8
      \end{pmatrix}
      \grstep{3c_{1}+c_{2}}
      \begin{pmatrix}
        1 & 0 & 0 \\
        5 & -13 & -5 \\
        -3 & 10 & 8
      \end{pmatrix}
    \end{equation*}

    With this, we can now find the determinant by expanding the first row, using \Cref{eq:Determinant_Expand_ith_Row}.
    \begin{align*}
      \det A &= \det
               \begin{pmatrix}
                 1 & 0 & 0 \\
                 5 & -13 & -5 \\
                 -3 & 10 & 8
               \end{pmatrix} \\
             &=  1 \Cofactor{1}{1} + 0 \Cofactor{1}{2} + 0 \Cofactor{1}{3} \\
             &= {(-1)}^{1+1} \Minor{1}{1} \\
             &= \Minor{1}{1} \\
      &= \det
        \begin{pmatrix}
          -13 & 5 \\
          10 & 8
        \end{pmatrix}
    \end{align*}

    Now that the matrix is a $2 \by 2$, we can find the determinant directly, we can simplify, etc.\@
    \begin{align*}
      \det A &= \det
               \begin{pmatrix}
          -13 & 5 \\
          10 & 8
        \end{pmatrix} \\
             &= (-13)(8) - (5)(10) \\
             &= -54
    \end{align*}
\end{example}

\paragraph{Interchange Rows}
Let $A =
\begin{pmatrix}
  a & b \\
  c & d
\end{pmatrix}
$.
Its \nameref{def:Determinant} is $\det A = ad - bc$.

If we interchange the two rows, we get $\tilde{A} =
\begin{pmatrix}
  c & d \\
  a & b
\end{pmatrix}$.
The determinant of this is $\det \tilde{A} = cb - ad = -\det A$.

\begin{corollary}[2 Identical Rows]\label{cor:2_Identical_Rows}
  Let there be a \nameref{def:Matrix} $A_{m \by n}$.
  Select 2 rows $r_{i}$ and $r_{j}$ such that $i \neq j$, where $r_{i} = r_{j}$, and interchange them.
  Because the rows were identical, the \nameref{def:Determinant} of the new row-equivalent matrix is the same as the original.

  The only way that can happen is if the additive identity is the solution, which is 0.

  $\therefore \det \tilde{A} = \det A = 0$.
\end{corollary}


%%% Local Variables:
%%% mode: latex
%%% TeX-master: "../../Math_333-MatrixAlg_ComplexVars-Reference_Sheet"
%%% End:
