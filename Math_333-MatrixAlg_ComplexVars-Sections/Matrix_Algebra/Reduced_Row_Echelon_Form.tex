\subsection{Reduced Row Echelon Form}\label{subsec:RREF}
Using what we learned in \Cref{subsec:Inverse_Matrices} and \Cref{subsubsec:Inverse_Matrices_using_Elementary_Matrices}, we can solve systems of linear equations using a \nameref{def:Matrix}'s inverse.

This can be movtivated with a general example.
\begin{blackbox}
  Consider the system shown below.
  \begin{align*}
    a_{11} x_{1} + a_{12} x_{2} &= b_{1} \\
    a_{21} x_{1} + a_{22} x_{2} &= b_{2}
  \end{align*}

  We can convert this set of equations with 2 equations and 2 unknowns to a single linear equation with a single unknown.
  \textbf{Equivalently} to the system, we can write
  \begin{align*}
    \begin{pmatrix}
      a_{11} & a_{12} \\
      a_{21} & a_{22}
    \end{pmatrix}
    \begin{pmatrix}
      x_{1} \\
      x_{2}
    \end{pmatrix} &=
    \begin{pmatrix}
      a_{11} x_{1} + a_{12} x_{2} \\
      a_{21} x_{1} + a_{22} x_{2}
    \end{pmatrix} \\
    &=
      \begin{pmatrix}
        b_{1} \\
        b_{2}
      \end{pmatrix}
  \end{align*}

  The column matrices (column vectors) are technically one unknown with multiple unknown components.

  Just like with regular algebra, we can use the inverse of the items we know to solve the equation.
  \begin{align*}
    \intertext{Let}
    A &=
        \begin{pmatrix}
          a_{11} & a_{12} \\
          a_{21} & a_{22}
        \end{pmatrix} \\
    X &=
        \begin{pmatrix}
          x_{1} \\
          x_{2}
        \end{pmatrix} \\
    B &=
        \begin{pmatrix}
          b_{1} \\
          b_{2}
        \end{pmatrix} \\
    AX &= B \\
    \shortintertext{We make the assumption $A$ is invertible.}
    A^{-1} (A X) &= A^{-1} B \\
    (A^{-1} A) X &= A^{-1} B \\
    I X &= A^{-1} B
  \end{align*}

  Therefore, we can solve the system using $A^{-1}$.
\end{blackbox}

There are 3 possible cases for a solution to a system:
\begin{enumerate}[noitemsep]
\item There is no solution. This happens when one equation contradicts another.
\item There are infinitely many solutions. (\Cref{ex:System with Infinitely Many Solutions}).
\item There is exactly one solution. (\Cref{ex:Solve General System Using Matrices}).
\end{enumerate}

Now, we can illustrate with some examples.

\begin{example}[Lecture 15, Example 2]{Solve General System Using Matrices}
  Solve the system of equations shown below for all $a$, $b$, $c$?
  \begin{align*}
    x - 2y + 3z &= a \\
    -x + 0y + 2z &= b \\
    x + y - z &= c
  \end{align*}
  \tcblower{}
  First, we convert to a matrix equation.
  \begin{align*}
    \begin{pmatrix}
      1 & -2 & -3 \\
      -1 & 0 & 2 \\
      1 & 1 & -1
    \end{pmatrix}
              \begin{pmatrix}
                x \\
                y \\
                z
              \end{pmatrix} &=
                              \begin{pmatrix}
                                a \\
                                b \\
                                c
                              \end{pmatrix} \\
    \intertext{We actually already know that $A$ has an inverse. We found it in \Cref{ex:Elementary Row Operations to find Inverse}.}
    \begin{pmatrix}
      x \\
      y \\
      z \\
    \end{pmatrix} &=
                    \begin{pmatrix}
                      2 & 5 & 4 \\
                      -1 & -2 & -1 \\
                      1 & 3 & 2
                    \end{pmatrix}
                              \begin{pmatrix}
                                a \\
                                b \\
                                c
                              \end{pmatrix} \\
    &=
      \begin{pmatrix}
        2a + 5b + 4c \\
        -a - 2b - c \\
        a + 3b + 2c
      \end{pmatrix} \\
        &=
          \begin{pmatrix}
            2a \\
            -a \\
            a
          \end{pmatrix} +
    \begin{pmatrix}
      5b \\
      -2b \\
      3b
    \end{pmatrix} +
    \begin{pmatrix}
      4c \\
      -c \\
      2c
    \end{pmatrix} \\
    &= a
      \begin{pmatrix}
        2 \\
        -1 \\
        1
      \end{pmatrix}
    + b
    \begin{pmatrix}
      5 \\
      -2 \\
      3
    \end{pmatrix}
    + c
    \begin{pmatrix}
      4 \\
      -1 \\
      2
    \end{pmatrix}
  \end{align*}

  Thus, we have a single unique solution based on the values for $a$, $b$, and $c$.
\end{example}


%%% Local Variables:
%%% mode: latex
%%% TeX-master: "../../Math_333-MatrixAlg_ComplexVars-Reference_Sheet"
%%% End:
