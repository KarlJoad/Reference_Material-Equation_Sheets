\subsection{Reduced Row Echelon Form}\label{subsec:RREF}
Using what we learned in \Cref{subsec:Inverse_Matrices} and \Cref{subsubsec:Inverse_Matrices_using_Elementary_Matrices}, we can solve systems of linear equations using a \nameref{def:Matrix}'s inverse.

This can be movtivated with a general example.
\begin{blackbox}
  Consider the system shown below.
  \begin{align*}
    a_{11} x_{1} + a_{12} x_{2} &= b_{1} \\
    a_{21} x_{1} + a_{22} x_{2} &= b_{2}
  \end{align*}

  We can convert this set of equations with 2 equations and 2 unknowns to a single linear equation with a single unknown.
  \textbf{Equivalently} to the system, we can write
  \begin{align*}
    \begin{pmatrix}
      a_{11} & a_{12} \\
      a_{21} & a_{22}
    \end{pmatrix}
    \begin{pmatrix}
      x_{1} \\
      x_{2}
    \end{pmatrix} &=
    \begin{pmatrix}
      a_{11} x_{1} + a_{12} x_{2} \\
      a_{21} x_{1} + a_{22} x_{2}
    \end{pmatrix} \\
    &=
      \begin{pmatrix}
        b_{1} \\
        b_{2}
      \end{pmatrix}
  \end{align*}

  The column matrices (column vectors) are technically one unknown with multiple unknown components.

  Just like with regular algebra, we can use the inverse of the items we know to solve the equation.
  \begin{align*}
    \intertext{Let}
    A &=
        \begin{pmatrix}
          a_{11} & a_{12} \\
          a_{21} & a_{22}
        \end{pmatrix} \\
    X &=
        \begin{pmatrix}
          x_{1} \\
          x_{2}
        \end{pmatrix} \\
    B &=
        \begin{pmatrix}
          b_{1} \\
          b_{2}
        \end{pmatrix} \\
    AX &= B \\
    \shortintertext{We make the assumption $A$ is invertible.}
    A^{-1} (A X) &= A^{-1} B \\
    (A^{-1} A) X &= A^{-1} B \\
    I X &= A^{-1} B
  \end{align*}

  Therefore, we can solve the system using $A^{-1}$.
\end{blackbox}


%%% Local Variables:
%%% mode: latex
%%% TeX-master: "../../Math_333-MatrixAlg_ComplexVars-Reference_Sheet"
%%% End:
