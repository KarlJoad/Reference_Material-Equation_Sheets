\subsection{Orthogonal Matrices}\label{subsec:Orthogonal_Matrices}
There is a particular class of matrices that is very helpful, nice to work with, and has an interesting property.
This is an \nameref{def:Orthogonal_Matrix}.

\begin{definition}[Orthogonal Matrix]\label{def:Orthogonal_Matrix}
  An \emph{orthogonal matrix} is a \nameref{def:Matrix} that is made up of \nameref{def:Orthonormal_Set}s.
  Such a matrix, has the property shown in \Cref{eq:Orthogonal_Matrix_Property}.

  \begin{equation}\label{eq:Orthogonal_Matrix_Property}
    \Transpose{A} = A^{-1}
  \end{equation}
\end{definition}

\begin{definition}[Orthonormal Set]\label{def:Orthonormal_Set}
  An \emph{orthonormal set} is one whose members have unit length (i.e.\ $1$), and whose members are \nameref{def:Linearly_Independent}.

  \begin{remark}
    This is a combination of an orthogonal set and a normal set.
    \begin{description}[noitemsep]
    \item[Orthogonal] The set's component elements are \nameref{def:Linearly_Independent}.
      For vectors (single-column/row matrices), this can be proven by performing the dot product operation on each vector and verifying that it is equal to 0.
    \item[Normal] The set's component elements have a magnitude (vector magnitude in this case) of $1$.
    \end{description}
  \end{remark}
\end{definition}

To show that an \nameref{def:Orthogonal_Matrix} \textbf{requires} \nameref{def:Orthonormal_Set}s, take a small example.

\begin{blackbox}
  Let $A$ be an \nameref{def:Orthogonal_Matrix}.
  Define $A_{n \by 1}$, as:
  \begin{equation*}
    A =
    \begin{pmatrix}
      r_{1} \\ r_{2} \\ \vdots \\ r_{n}
    \end{pmatrix}
  \end{equation*}

  Then,
  \begin{equation*}
    \Transpose{A} =
    \begin{pmatrix}
      r_{1} & r_{2} & \cdots & r_{n}
    \end{pmatrix}
  \end{equation*}

  By the definition of an \nameref{def:Orthogonal_Matrix} and \nameref{def:Inverse_Matrix}, we know that $A \Transpose{A} = I$.
  This means that when multiplying these two matrices together on the $i$th row and $j$th column, we have 2 cases:
  \begin{equation*}
    A_{i \by 1} \Transpose{A}_{1 \by j} =
    \begin{cases}
      1 & \text{ when } i = j \\
      0 & \text{ when } i \neq j
    \end{cases}
  \end{equation*}

  This means that:
  \begin{description}[noitemsep]
  \item When $i \neq j$, then $r_{i} \perp r_{j}$.
  \item When $i = j$, then $r_{i} \Transpose{r_{j}} = 1$, meaning each row has unit length.
  \end{description}
\end{blackbox}


%%% Local Variables:
%%% mode: latex
%%% TeX-master: "../../Math_333-MatrixAlg_ComplexVars-Reference_Sheet"
%%% End:
