\subsection{Orthogonal Matrices}\label{subsec:Orthogonal_Matrices}
There is a particular class of matrices that is very helpful, nice to work with, and has an interesting property.
This is an \nameref{def:Orthogonal_Matrix}.

\begin{definition}[Orthogonal Matrix]\label{def:Orthogonal_Matrix}
  An \emph{orthogonal matrix} is a \nameref{def:Matrix} that is made up of \nameref{def:Orthonormal_Set}s.
  Such a matrix, has the property shown in \Cref{eq:Orthogonal_Matrix_Property}.

  \begin{equation}\label{eq:Orthogonal_Matrix_Property}
    \Transpose{A} = A^{-1}
  \end{equation}
\end{definition}

\begin{definition}[Orthonormal Set]\label{def:Orthonormal_Set}
  An \emph{orthonormal set} is one whose members have unit length (i.e.\ $1$), and whose members are \nameref{def:Linearly_Independent}.

  \begin{remark}
    This is a combination of an orthogonal set and a normal set.
    \begin{description}[noitemsep]
    \item[Orthogonal] The set's component elements are \nameref{def:Linearly_Independent}.
      For vectors (single-column/row matrices), this can be proven by performing the dot product operation on each vector and verifying that it is equal to 0.
    \item[Normal] The set's component elements have a magnitude (vector magnitude in this case) of $1$.
    \end{description}
  \end{remark}
\end{definition}


%%% Local Variables:
%%% mode: latex
%%% TeX-master: "../../Math_333-MatrixAlg_ComplexVars-Reference_Sheet"
%%% End:
