\subsection{Systems of Equations}\label{subsec:Systems_Equations}
A \nameref{def:Matrix} can be used to represent a system of equations.

For example,
\begin{align*}
  x + 2y + 3z &= 5 \\
  2x - y + z &= 3 \\
  x - y + 5z &= 2
\end{align*}
can be represented with a set of matrices as follows
\begin{equation*}
  \begin{bmatrix}
    1 & 2 & 3 \\
    2 & -1 & 1 \\
    1 & -1 & 5 \\
  \end{bmatrix}
  \begin{bmatrix}
    x \\
    y \\
    z \\
  \end{bmatrix}
  =
  \begin{bmatrix}
    5 \\
    3 \\
    2 \\
  \end{bmatrix}
\end{equation*}

However, as we can see from the equations, and from our knowledge of how to solve these systems, we know that the coefficients are more important than the actual variables.
Remember, when solving these the traditional way, we attempt to cancel out the various variable terms \textbf{using} the coefficients.
So, we can use an \nameref{def:Augmented_Matrix} to show the relationship here.
\begin{equation*}
  \begin{amat}{3}
    1 & 2 & 3 & 5 \\
    2 & -1 & 1 & 3 \\
    1 & -1 & 5 & 2 \\
  \end{amat}
\end{equation*}

\begin{definition}[Augmented Matrix]\label{def:Augmented_Matrix}
  An \emph{augmented matrix} is one where \textbf{only} the coefficients of a system of linear equations are present.
  To denote the result of the equation, we use a vertical bar to separate the body of an equation with the equation's result.

  \begin{equation}\label{eq:Augmented_Matrix}
    \begin{amat}{2}
      1 & 2 & 3 \\
      4 & 5 & 6
    \end{amat}
  \end{equation}
\end{definition}

%%% Local Variables:
%%% mode: latex
%%% TeX-master: "../../Math_333-MatrixAlg_ComplexVars-Reference_Sheet"
%%% End:
