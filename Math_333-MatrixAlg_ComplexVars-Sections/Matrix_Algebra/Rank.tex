\subsection{Matrix Rank}\label{subsec:Matrix_Rank}
\begin{definition}[Rank]\label{def:Matrix_Rank}
  The \emph{rank} of a \nameref{def:Matrix} $A_{n \by m}$ is the largest number of \nameref{def:Linearly_Independent} ``chunks'' of a matrix.

  There are 2 kinds of rank: \nameref{rmk:Row_Rank} and \nameref{rmk:Column_Rank}.

  \begin{remark}[Row Rank]\label{rmk:Row_Rank}
    The \emph{rank} of a \nameref{def:Matrix} $A_{n \by m}$ is the largest number of \nameref{def:Linearly_Independent} rows of a matrix.
  \end{remark}

  \begin{remark}[Column Rank]\label{rmk:Column_Rank}
    The \emph{column rank} of a \nameref{def:Matrix} $A_{n \by m}$ is the largest number of \nameref{def:Linearly_Independent} columns of a matrix.
  \end{remark}

  \begin{remark}[Row Rank Column Rank Equivalence]
    \nameref{rmk:Row_Rank} can be proven to be the same as the \nameref{rmk:Column_Rank} for all matrices.
  \end{remark}
\end{definition}

\begin{theorem}
  Perform \nameref{def:Elementary_Row_Op}s/\nameref{def:Elementary_Column_Op}s to a \nameref{def:Matrix} to get it in \nameref{thm:Echelon_Form}.

  Then, the number of non-zero rows/columns is the row/column rank.
\end{theorem}

%%% Local Variables:
%%% mode: latex
%%% TeX-master: "../../Math_333-MatrixAlg_ComplexVars-Reference_Sheet"
%%% End:
