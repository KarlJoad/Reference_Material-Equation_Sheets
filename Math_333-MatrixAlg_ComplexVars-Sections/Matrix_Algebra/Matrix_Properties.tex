\subsection{Properties of Matrices}\label{subsec:Properties_Matrices}
A \nameref{def:Matrix} can be \nameref{def:Transpose}d.

\begin{definition}[Transpose]\label{def:Transpose}
  The \emph{transpose} of a \nameref{def:Matrix} $A$, denoted $\Transpose{A}$, is the matrix $A$ where all the columns became rows or vice-versa.

  \begin{align*}
    A &=
        \begin{pmatrix}
          1 & 2  & 3 \\
          0 & 5 & 0 \\
        \end{pmatrix} \\
    \Transpose{A} &=
                    \begin{pmatrix}
                      1 & 0 \\
                      2 & 5 \\
                      3 & 0
                    \end{pmatrix}
  \end{align*}
\end{definition}

\subsubsection{Properties of Matrix Addition}\label{subsubsec:Properties_Matrix_Addition}
\begin{propertylist}
\item \nameref{def:Matrix} Addition is commutative.\label{prop:Matrix_Add_Commutative}
  \begin{equation}\label{eq:Matrix_Add_Commutative}
    A+B = B+A
  \end{equation}

\item \nameref{def:Matrix} Addition is associative.\label{prop:Matrix_Add_Associative}
  \begin{equation}\label{eq:Matrix_Add_Associative}
    A+(B+C) = (A+B) + C
  \end{equation}

\item For any \nameref{def:Matrix} $A$, there is a unique matrix, $I$ or $O$, such that the equation below holds true.\label{prop:Matrix_Additive_Identity}
  \begin{equation}\label{eq:Matrix_Additive_Identity}
    A+O = A
  \end{equation}

\item For each \nameref{def:Matrix} $A$, there is a unique matrix $-A$ such that the equation below holds true.\label{prop:Matrix_Additive_Inverse}
  \begin{equation}\label{eq:Matrix_Additive_Inverse}
    A+(-A) = O
  \end{equation}

\item $A+B$ is a \nameref{def:Matrix} of the same dimensions as $A$ and $B$.\label{prop:Matrix_Addition_Closure}
\end{propertylist}

\subsubsection{Properties of Matrix Multiplication}\label{subsubsec:Properties_Matrix_Multiplication}
Matrix multiplication is defined as follows:
\begin{equation}\label{eq:Matrix_Multiplication}
  \begin{bmatrix}
    a_{11} & a_{12} & a_{13} & \cdots & a_{1n} \\
    a_{21} & a_{22} & \ddots & \ddots & a_{2n} \\
    \vdots & \ddots & \ddots & \ddots & a_{mn} \\
    a_{n\,1} & a_{n\,2} & \cdots & \cdots & a_{nn}
  \end{bmatrix}
\end{equation}

\begin{propertylist}
\item In general, \nameref{def:Matrix} multiplication is \textbf{NOT} commutative.\label{prop:Matrix_Not_Commutative}
  \begin{equation}\label{eq:Matrix_Mult_Not_Commutative}
    AB \neq BA
  \end{equation}

\item \nameref{def:Matrix} multiplication is associative.\label{prop:Matrix_Associativity}
  \begin{equation}\label{eq:Matrix_Mult_Associativity}
    (AB) C = A (BC)
  \end{equation}
\item \nameref{def:Matrix} multiplication is distributive.\label{prop:Matrix_Distributivity}
  \begin{equation}\label{eq:Matrix_Mult_Distributivity}
    \begin{aligned}
      A (B+C) &= AB + AC \\
      (B+C) A &= BA + CA
    \end{aligned}
  \end{equation}

\item There exists a \nameref{def:Matrix} $I$ that forms the \nameref{def:Multiplicative_Identity_Matrix} that satisfies the following rule:\label{prop:Mult_Identity_Matrix}
  \begin{equation}\label{eq:Mult_Identity_Matrix}
    \begin{aligned}
      AI &= A \\
      IA &= A
    \end{aligned}
  \end{equation}
  \begin{remark*}
    This is one of the two general cases where \nameref{def:Matrix} multiplication \textbf{IS} commutative.
  \end{remark*}

\item There exists a \nameref{def:Matrix} $O$ that forms the \nameref{def:Zero_Matrix}, satisfying the following rule:\label{prop:Mult_Zero_Matrix}
  \begin{equation}\label{eq:Mult_Zero_Matrix}
    \begin{aligned}
      AO &= O \\
      OA &= O
    \end{aligned}
  \end{equation}
  \begin{remark*}
    This is one of the two general cases where \nameref{def:Matrix} multiplication \textbf{IS} commutative.
  \end{remark*}

\item When performing multiplication, the \textbf{dimension property} states that the product of an $m \by n$ and an $n \by k$ matrix is an $m \by k$ \nameref{def:Matrix}.\label{prop:Mult_Dimension_Property}
\end{propertylist}

\begin{definition}[Identity Matrix]\label{def:Multiplicative_Identity_Matrix}
  The \emph{identity matrix}, $I$ is one that satisfies \Cref{prop:Mult_Identity_Matrix}.
  These follow the forms of the ones shown below.
  \begin{equation}\label{eq:Multiplicative_Identity_Matrix}
    \begin{aligned}
      I &=
      \begin{bmatrix}
        1 & 0 \\
        0 & 1 \\
      \end{bmatrix} \\
      &=
      \begin{bmatrix}
        1 & 0 & 0 \\
        0 & 1 & 0 \\
        0 & 0 & 1 \\
      \end{bmatrix}
    \end{aligned}
  \end{equation}
\end{definition}

\begin{definition}[Zero Matrix]\label{def:Zero_Matrix}
  The \emph{zero matrix}, $O$, is one that satisfies \Cref{prop:Mult_Zero_Matrix}.
  For \nameref{def:Matrix} multiplication, it is a matrix of all $0$.
  \begin{equation}
    \label{eq:3}
    \begin{aligned}
      O &=
      \begin{bmatrix}
        0 & 0 \\
        0 & 0 \\
      \end{bmatrix} \\
      &=
      \begin{bmatrix}
        0 & 0 & 0 \\
        0 & 0 & 0 \\
        0 & 0 & 0 \\
      \end{bmatrix}
    \end{aligned}
  \end{equation}
\end{definition}

%%% Local Variables:
%%% mode: latex
%%% TeX-master: "../../Math_333-MatrixAlg_ComplexVars-Reference_Sheet"
%%% End:
