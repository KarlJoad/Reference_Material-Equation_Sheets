\section{Complex Functions}\label{sec:Complex_Functions}
Complex functions, like their real-valued counterparts behave in much the same way.
\begin{equation}\label{eq:Complex_Function}
  f(z) = w
\end{equation}
\begin{description}[noitemsep]
\item $f$: The function or \nameref{def:Mapping} that corresponds the input to the output.
\item $z$: The input to the complex function/\nameref{def:Mapping}.
\item $w$: The output of the complex function/\nameref{def:Mapping}.
\end{description}

\begin{definition}[Mapping]\label{def:Mapping}
  A \emph{mapping} is synonym for a function in mathematics.
  The term comes from set theory, where the input set is mapped to an output set by some operations.
  The conventional way to denote a mapping is with the $\mapsto$ symbol.

  An example of a mapping is shown in \Cref{eq:Mapping}
  \begin{equation}\label{eq:Mapping}
    z \mapsto z^{2}
  \end{equation}
\end{definition}

A complex function can only accept and will only return values in \textbf{Cartesian} or \textbf{polar} form.
Because the output of a complex function is also a complex value, \Cref{eq:Output_Value_Function} makes sense.
\begin{equation}\label{eq:Output_Value_Function}
  f(z) = U(x, y) + iV(x, y)
\end{equation}

$U(x, y)$ and $V(x, y)$ can be as general as we want in $x$ and $y$.
This means both could be constants, both could be polynomials, one could be trascendental, and anything in between.

The functions $U(x, y)$ and $V(x, y)$ are functions that yield real-values $u, v$.
This means that $u, v$ can also be graphed on an \nameref{def:Argand_Plane}.
By our definition of $U(x, y)$ and $V(x, y)$, $U(x, y), V(x, y)$ are parametric functions.

\begin{example}[Lecture 4]{Find Output Functions}
  Given the \nameref{def:Mapping} $z \mapsto z^{2}$, where $z = x + iy$, find the output functions for each term $U(x, y)$ and $V(x, y)$?
  \tcblower{}
  I will choose to represent the mapping $z \mapsto z^{2}$ with the complex function $f(z) = z^{2}$.
  \begin{align*}
    z &\mapsto z^{2} \\
    f(z) &= z^{2} \\
    \shortintertext{Apply the definition of $z$.}
      &= {(x + iy)}^{2} \\
      &= x^{2} + 2xyi + i^{2}y^{2} \\
      &= x^{2} + 2xyi + (-1)y^{2} \\
      &= \left( x^{2} - y^{2} \right) + 2xyi \\
  \end{align*}

  By out definition of $U(x, y)$ and $V(x, y)$ in \Cref{eq:Output_Value_Function}, we can finish solving this.
  \begin{align*}
    f(z) &= U(x, y) + iV(x, y) \\
    f(z) &= \left( x^{2} - y^{2} \right) + 2xyi \\
    U(x, y) &= x^{2} - y^{2} \\
    V(x, y) &= 2xy
  \end{align*}

  Thus, our output functions are $U(x, y) = x^{2} - y^{2}$ and $V(x, y) = 2xy$.
\end{example}

\subsection{Graphing Complex Functions}\label{subsubsec:Graphing_Complex_Functions}
If we take a closer look at the complex function $f(z)$, we notice something that makes handling complex function difficult.
$f(z)$ is really a function of $x$ and $y$, because $z$ depends on those 2 real-valued parameters.
Thus, all our inputs lie on a 2-dimensional plane.

Now if we look at the output $w$, we also notice it is complex-valued, meaning it also depends on some $u$ and $v$, which are equal to the value of their functions $U(x, y)$ and $V(x, y)$.
This means that the output of the function $f(z)$ \textbf{also} lies on a 2-dimensional plane.
Meaning, the function is 4-dimensional.
The intersection of 2 planes in our 3-dimensional world never yields a point in the hyperplane, and thus, we cannot graph it.

Instead, we choose to graph the inputs and outputs separately, effectively showing the mapping and the way the \nameref{def:Pre_Image} is transformed into the \nameref{def:Image} instead.

\begin{definition}[Pre-Image]\label{def:Pre_Image}
  The \emph{pre-image} consists of all points on the input plane.
  In this case, the input plane is the $z$-plane, being constructed out of the orthogonal intersection of the $\Re$ and $\Im$ axes.
\end{definition}

\begin{definition}[Image]\label{def:Image}
  The \emph{image} consists of all points on the output plane.
  In this case, the input plane is the $w$-plane, being constructed out of the orthogonal intersection of the $U(x, y)$ axis on the horizontal and the $V(x, y)$ axis on the vertical.
\end{definition}

By graphing and viewing the input and the output simultaneously, we can see how the \nameref{def:Mapping} $f(z)$ distorts the input \nameref{def:Pre_Image} into the output \nameref{def:Image}.

If asked to graph the complex function/\nameref{def:Mapping}, you must find the expression for the output functions $U(x, y)$ and $V(x, y)$, and then graph the inputs $x, y$ against the outputs $U(x, y), V(x, y)$.

Because the output equations are in terms of, possibly 2, other variables, they are parametric equations.
If you are also asked to find the Cartesian form of the output equation, you must simplify the other terms away.

\begin{example}[Lecture 4]{Plot Simple Complex Function}
  Find the \nameref{def:Image} of the line $y=4$ on the map $f(z) = z^{2}$?
  Provide the Cartesian equation of the \nameref{def:Image} and the orientation of the image points as the \nameref{def:Pre_Image} points move $y=4$ from $-\infty$ towards $\infty$?
  \tcblower{}
  We found the parametric output functions $U(x, y)$ and $V(x, y)$ in \Cref{ex:Find Output Functions}, so we will use those here.

  How to perform this \nameref{def:Pre_Image} to \nameref{def:Image} plotting:
  \begin{enumerate}[noitemsep]
  \item Start by plotting $y=4$ in the $xy$-plane ($z$-plane).
  \item Then, start plugging values of $x, y=4$ into $U(x, y)$ and $V(x, y)$.
  \item Start with $x < 0$ and move towards $x>0$, as that will follow the orientation of the line provided in the question.
  \item Graph the \nameref{def:Image}'s results.
  \item Indicate the orientation of the image on its graph.
  \end{enumerate}

  To find the Cartesian form of the parametric output equations, we can start by eliminating $y$ from the parameters, as $y$ was specified to be a constant $y=4$.
  \begin{align*}
    U(x, y) &= x^{2} - y^{2} \\
    V(x, y) &= 2xy \\
    U(x, y=4) &= x^{2} - 4^{2} \\
    V(x, y=4) &= 2x (4) \\
    U(x) &= x^{2}- 16 \\
    V(x) &= 8x
  \end{align*}

  Now that $y$ has been eliminated, I will simplify $V(x)$ such that $x$ is in terms of $V$.
  \begin{align*}
    V(x) &= 8x \\
    x &= \frac{V}{8}
  \end{align*}

  Now that a value for $x$ has been found, we can plug that value back into $U(x)$, and simplify.
  \begin{align*}
    x &= \frac{V}{8} \\
    U \left( x=\frac{V}{8} \right) &= {\left( \frac{V}{8} \right)}^{2} - 16 \\
    U + 16 &= \frac{V^{2}}{64} \\
    V^{2} &= 64 (U + 16) \\
    V &= \sqrt{64 (U+16)}
  \end{align*}

  Thus, the Cartesian equation of the \nameref{def:Image} is a parabola whose defining equation is $V = \sqrt{64 (U+16)}$.

  \begin{remark*}
    I could have chosen to solve for $U$ in terms of $V$, but that would have required the addition of $\pm$ in many places due to the early application of the square root.
  \end{remark*}
\end{example}

\begin{example}[Lecture 4]{Plot Complex Trigonometric Function}
  Find the Cartesian equation of the \nameref{def:Image} of the line $x=4$ under the map $f(z) = \sin(z)$?
  Provide the Cartesian equation of the \nameref{def:Image}?
  \tcblower{}
  Using the \nameref{subsubsec:Complex_Angle_Sum_Difference_Identities} for $\sin$ (\Cref{eq:Sin_Angle_Sum_Difference}), we can put $\sin$ into Cartesian form and simplify.
  \begin{align*}
    \sin(x + iy) &= \bigl( \sin(x) \cosh(y) \bigr) + i \bigl( \cos(x) \sinh(y) \bigr) \\
    f(z = x + iy \vert x=4 ) &= \sin(z) \\
                 &= \bigl( \sin(4) \cosh(y) \bigr) + i \bigl( \cos(4) \sinh(y) \bigr) \\
    \shortintertext{Use the definition of the output functions.}
    U(y) &= \sin(4) \cosh(y) \\
    V(y) &= \cos(4) \sinh(y)
  \end{align*}

  How to perform this \nameref{def:Pre_Image} to \nameref{def:Image} plotting:
  \begin{enumerate}[noitemsep]
  \item Start by plotting $x=4$ in the $xy$-plane ($z$-plane).
  \item Then, start plugging values of $x=4, y$ into $U(x, y)$ and $V(x, y)$.
  \item Graph the \nameref{def:Image}'s results.
  \end{enumerate}

  If you notice, the \nameref{def:Image} that is created is a hyperbola.
  However, only one of the 2 arcs that is created is the correct one, as we lose information when we move between parametric and Cartesian forms.
  We can figure this out by looking at $U(x=4, y) = \sin(4) \cosh(y)$.
  \begin{enumerate}[noitemsep]
  \item We know $\sin(4) < 0$, as $4 > \pi$.
  \item In addition, $\cosh \nless 0$, by definition.
  \item Thus, if $(\sin(4) < 0) (\cosh \nless 0) = U(x=4, y) < 0$.
  \item Therefore, the only part of the hyperbola that should be kept is the one where $U < 0$.
  \end{enumerate}

  To get the Cartesian equation for this shape, we need to have a relation between $\cosh$ and $\sinh$; fortunately, we have one.
  The Pythagorean theorem for hyperbolic trigonometry would work perfectly, so we can substitue for $\cosh$ and $\sinh$.
  \begin{align*}
    \cosh^{2}(\theta) - \sinh^{2}(\theta) &= 1 \\
    \frac{U(y)}{\sin(4)} - \frac{V(y)}{\cos(4)} &= 1
  \end{align*}
\end{example}

%%% Local Variables:
%%% mode: latex
%%% TeX-master: "../../Math_333-MatrixAlg_ComplexVars-Reference_Sheet"
%%% End:


\subsection{Limits}\label{subsec:Limits}
Like in earlier maths, sometimes we are interested not in the exact point where a function has a value; instead, we are interested in how the function behaves as we approach that point.
If the function is a \nameref{def:Complex_Function}, i.e.\ ($\ComplexNumbers \mapsto \ComplexNumbers$), then we need to change our definition of a limit from the one we are familiar with to the one in \Cref{def:Complex_Limit}.

\begin{definition}[Limit]\label{def:Complex_Limit}
  The \emph{limit} of a \nameref{def:Complex_Function} behaves quite similarly to their purely-real counterparts.
  \begin{equation}\label{eq:Complex_Limit}
    \lim\limits_{\substack{z \to a \\ z \neq a}} f(z) = \ell
  \end{equation}
\end{definition}

To solve a problem involving a \nameref{def:Complex_Function} and a limit, you can perform the same steps as before.

\begin{example}[Lecture 5]{Solve Limit of Complex Function}
  If $f(z) = z^{2}$, then find the solution to this \nameref{def:Complex_Limit} of a \nameref{def:Complex_Function}?
  \begin{equation*}
    \lim\limits_{z \to 3i} f(z)
  \end{equation*}
  \tcblower{}
  \begin{align*}
    \lim\limits_{z \to 3i} f(z) &= \lim\limits_{z \to 3i} z^{2} \\
                                &= {(3i)}^{2} \\
                                &= 9 i^{2} = 9 (-1) \\
                                &= -9
  \end{align*}

  Thus, $\lim\limits_{z \to 3i} f(z) = -9$.
\end{example}

\begin{remark*}
  However, do note that the act of finding the \nameref{def:Complex_Limit} of a function approaching a point is distinctly different than finding the value of the function \textbf{at} that point.
\end{remark*}


%%% Local Variables:
%%% mode: latex
%%% TeX-master: "../../Math_333-MatrixAlg_ComplexVars-Reference_Sheet"
%%% End:



%%% Local Variables:
%%% mode: latex
%%% TeX-master: "../Math_333-MatrixAlg_ComplexVars-Reference_Sheet"
%%% End:
