\section{Complex Functions}\label{sec:Complex_Functions}
\begin{definition}[Complex Function]\label{def:Complex_Function}
  A \emph{complex function} is a like their purely real-valued brothers, but instead of mapping real number inputs to real number outputs ($\RealNumbers \mapsto \RealNumbers$), they map \nameref{def:Complex_Number} inputs to complex outputs ($\ComplexNumbers \mapsto \ComplexNumbers$).

  Complex functions, like their real-valued counterparts behave in much the same way.
  \begin{equation}\label{eq:Complex_Function}
    f(z) = w
  \end{equation}
  \begin{description}[noitemsep]
  \item $f$: The function or \nameref{def:Mapping} that corresponds the input to the output.
  \item $z$: The input to the complex function/\nameref{def:Mapping}.
  \item $w$: The output of the complex function/\nameref{def:Mapping}.
  \end{description}
\end{definition}

Sometimes a \nameref{def:Complex_Function}, like all other functions, is referred to as a \nameref{def:Mapping}.

\begin{definition}[Mapping]\label{def:Mapping}
  A \emph{mapping} is synonym for a function in mathematics.
  The term comes from set theory, where the input set is mapped to an output set by some operations.
  The conventional way to denote a mapping is with the $\mapsto$ symbol.

  An example of a mapping is shown in \Cref{eq:Mapping}
  \begin{equation}\label{eq:Mapping}
    z \mapsto z^{2}
  \end{equation}
\end{definition}

A complex function can only accept and will only return values in \textbf{Cartesian} or \textbf{polar} form.
Because the output of a complex function is also a complex value, \Cref{eq:Output_Value_Function} makes sense.
\begin{equation}\label{eq:Output_Value_Function}
  f(z) = U(x, y) + iV(x, y)
\end{equation}

$U(x, y)$ and $V(x, y)$ can be as general as we want in $x$ and $y$.
This means both could be constants, both could be polynomials, one could be transcendental, and anything in between.

The functions $U(x, y)$ and $V(x, y)$ are functions that yield real-values $u, v$.
This means that $u, v$ can also be graphed on an \nameref{def:Argand_Plane}.
By our definition of $U(x, y)$ and $V(x, y)$, $U(x, y), V(x, y)$ are parametric functions.

\begin{example}[Lecture 4]{Find Output Functions}
  Given the \nameref{def:Mapping} $z \mapsto z^{2}$, where $z = x + iy$, find the output functions for each term $U(x, y)$ and $V(x, y)$?
  \tcblower{}
  I will choose to represent the mapping $z \mapsto z^{2}$ with the complex function $f(z) = z^{2}$.
  \begin{align*}
    z &\mapsto z^{2} \\
    f(z) &= z^{2} \\
    \shortintertext{Apply the definition of $z$.}
      &= {(x + iy)}^{2} \\
      &= x^{2} + 2xyi + i^{2}y^{2} \\
      &= x^{2} + 2xyi + (-1)y^{2} \\
      &= \left( x^{2} - y^{2} \right) + 2xyi \\
  \end{align*}

  By out definition of $U(x, y)$ and $V(x, y)$ in \Cref{eq:Output_Value_Function}, we can finish solving this.
  \begin{align*}
    f(z) &= U(x, y) + iV(x, y) \\
    f(z) &= \left( x^{2} - y^{2} \right) + 2xyi \\
    U(x, y) &= x^{2} - y^{2} \\
    V(x, y) &= 2xy
  \end{align*}

  Thus, our output functions are $U(x, y) = x^{2} - y^{2}$ and $V(x, y) = 2xy$.
\end{example}

\subsection{Graphing Complex Functions}\label{subsubsec:Graphing_Complex_Functions}
If we take a closer look at the complex function $f(z)$, we notice something that makes handling complex function difficult.
$f(z)$ is really a function of $x$ and $y$, because $z$ depends on those 2 real-valued parameters.
Thus, all our inputs lie on a 2-dimensional plane.

Now if we look at the output $w$, we also notice it is complex-valued, meaning it also depends on some $u$ and $v$, which are equal to the value of their functions $U(x, y)$ and $V(x, y)$.
This means that the output of the function $f(z)$ \textbf{also} lies on a 2-dimensional plane.
Meaning, the function is 4-dimensional.
The intersection of 2 planes in our 3-dimensional world never yields a point in the hyperplane, and thus, we cannot graph it.

Instead, we choose to graph the inputs and outputs separately, effectively showing the mapping and the way the \nameref{def:Pre_Image} is transformed into the \nameref{def:Image} instead.

\begin{definition}[Pre-Image]\label{def:Pre_Image}
  The \emph{pre-image} consists of all points on the input plane.
  In this case, the input plane is the $z$-plane, being constructed out of the orthogonal intersection of the $\Re$ and $\Im$ axes.
\end{definition}

\begin{definition}[Image]\label{def:Image}
  The \emph{image} consists of all points on the output plane.
  In this case, the input plane is the $w$-plane, being constructed out of the orthogonal intersection of the $U(x, y)$ axis on the horizontal and the $V(x, y)$ axis on the vertical.
\end{definition}

By graphing and viewing the input and the output simultaneously, we can see how the \nameref{def:Mapping} $f(z)$ distorts the input \nameref{def:Pre_Image} into the output \nameref{def:Image}.

If asked to graph the complex function/\nameref{def:Mapping}, you must find the expression for the output functions $U(x, y)$ and $V(x, y)$, and then graph the inputs $x, y$ against the outputs $U(x, y), V(x, y)$.

Because the output equations are in terms of, possibly 2, other variables, they are parametric equations.
If you are also asked to find the Cartesian form of the output equation, you must simplify the other terms away.

\begin{example}[Lecture 4]{Plot Simple Complex Function}
  Find the \nameref{def:Image} of the line $y=4$ on the map $f(z) = z^{2}$?
  Provide the Cartesian equation of the \nameref{def:Image} and the orientation of the image points as the \nameref{def:Pre_Image} points move $y=4$ from $-\infty$ towards $\infty$?
  \tcblower{}
  We found the parametric output functions $U(x, y)$ and $V(x, y)$ in \Cref{ex:Find Output Functions}, so we will use those here.

  How to perform this \nameref{def:Pre_Image} to \nameref{def:Image} plotting:
  \begin{enumerate}[noitemsep]
  \item Start by plotting $y=4$ in the $xy$-plane ($z$-plane).
  \item Then, start plugging values of $x, y=4$ into $U(x, y)$ and $V(x, y)$.
  \item Start with $x < 0$ and move towards $x>0$, as that will follow the orientation of the line provided in the question.
  \item Graph the \nameref{def:Image}'s results.
  \item Indicate the orientation of the image on its graph.
  \end{enumerate}

  To find the Cartesian form of the parametric output equations, we can start by eliminating $y$ from the parameters, as $y$ was specified to be a constant $y=4$.
  \begin{align*}
    U(x, y) &= x^{2} - y^{2} \\
    V(x, y) &= 2xy \\
    U(x, y=4) &= x^{2} - 4^{2} \\
    V(x, y=4) &= 2x (4) \\
    U(x) &= x^{2}- 16 \\
    V(x) &= 8x
  \end{align*}

  Now that $y$ has been eliminated, I will simplify $V(x)$ such that $x$ is in terms of $V$.
  \begin{align*}
    V(x) &= 8x \\
    x &= \frac{V}{8}
  \end{align*}

  Now that a value for $x$ has been found, we can plug that value back into $U(x)$, and simplify.
  \begin{align*}
    x &= \frac{V}{8} \\
    U \left( x=\frac{V}{8} \right) &= {\left( \frac{V}{8} \right)}^{2} - 16 \\
    U + 16 &= \frac{V^{2}}{64} \\
    V^{2} &= 64 (U + 16) \\
    V &= \sqrt{64 (U+16)}
  \end{align*}

  Thus, the Cartesian equation of the \nameref{def:Image} is a parabola whose defining equation is $V = \sqrt{64 (U+16)}$.

  \begin{remark*}
    I could have chosen to solve for $U$ in terms of $V$, but that would have required the addition of $\pm$ in many places due to the early application of the square root.
  \end{remark*}
\end{example}

\begin{example}[Lecture 4]{Plot Complex Trigonometric Function}
  Find the Cartesian equation of the \nameref{def:Image} of the line $x=4$ under the map $f(z) = \sin(z)$?
  Provide the Cartesian equation of the \nameref{def:Image}?
  \tcblower{}
  Using the \nameref{subsubsec:Complex_Angle_Sum_Difference_Identities} for $\sin$ (\Cref{eq:Sin_Angle_Sum_Difference}), we can put $\sin$ into Cartesian form and simplify.
  \begin{align*}
    \sin(x + iy) &= \bigl( \sin(x) \cosh(y) \bigr) + i \bigl( \cos(x) \sinh(y) \bigr) \\
    f(z = x + iy \vert x=4 ) &= \sin(z) \\
                 &= \bigl( \sin(4) \cosh(y) \bigr) + i \bigl( \cos(4) \sinh(y) \bigr) \\
    \shortintertext{Use the definition of the output functions.}
    U(y) &= \sin(4) \cosh(y) \\
    V(y) &= \cos(4) \sinh(y)
  \end{align*}

  How to perform this \nameref{def:Pre_Image} to \nameref{def:Image} plotting:
  \begin{enumerate}[noitemsep]
  \item Start by plotting $x=4$ in the $xy$-plane ($z$-plane).
  \item Then, start plugging values of $x=4, y$ into $U(x, y)$ and $V(x, y)$.
  \item Graph the \nameref{def:Image}'s results.
  \end{enumerate}

  If you notice, the \nameref{def:Image} that is created is a hyperbola.
  However, only one of the 2 arcs that is created is the correct one, as we lose information when we move between parametric and Cartesian forms.
  We can figure this out by looking at $U(x=4, y) = \sin(4) \cosh(y)$.
  \begin{enumerate}[noitemsep]
  \item We know $\sin(4) < 0$, as $4 > \pi$.
  \item In addition, $\cosh \nless 0$, by definition.
  \item Thus, if $(\sin(4) < 0) (\cosh \nless 0) = U(x=4, y) < 0$.
  \item Therefore, the only part of the hyperbola that should be kept is the one where $U < 0$.
  \end{enumerate}

  To get the Cartesian equation for this shape, we need to have a relation between $\cosh$ and $\sinh$; fortunately, we have one.
  The Pythagorean theorem for hyperbolic trigonometry would work perfectly, so we can substitue for $\cosh$ and $\sinh$.
  \begin{align*}
    \cosh^{2}(\theta) - \sinh^{2}(\theta) &= 1 \\
    \frac{U(y)}{\sin(4)} - \frac{V(y)}{\cos(4)} &= 1
  \end{align*}
\end{example}

%%% Local Variables:
%%% mode: latex
%%% TeX-master: "../../Math_333-MatrixAlg_ComplexVars-Reference_Sheet"
%%% End:


\subsection{Limits}\label{subsec:Limits}
Like in earlier maths, sometimes we are interested not in the exact point where a function has a value; instead, we are interested in how the function behaves as we approach that point.
If the function is a \nameref{def:Complex_Function}, i.e.\ ($\ComplexNumbers \mapsto \ComplexNumbers$), then we need to change our definition of a limit from the one we are familiar with to the one in \Cref{def:Complex_Limit}.

\begin{definition}[Limit]\label{def:Complex_Limit}
  The \emph{limit} of a \nameref{def:Complex_Function} behaves quite similarly to their purely-real counterparts.
  \begin{equation}\label{eq:Complex_Limit}
    \lim\limits_{\substack{z \to a \\ z \neq a}} f(z) = \ell
  \end{equation}
\end{definition}

To solve a problem involving a \nameref{def:Complex_Function} and a limit, you can perform the same steps as before.

\begin{example}[Lecture 5]{Solve Limit of Complex Function}
  If $f(z) = z^{2}$, then find the solution to this \nameref{def:Complex_Limit} of a \nameref{def:Complex_Function}?
  \begin{equation*}
    \lim\limits_{z \to 3i} f(z)
  \end{equation*}
  \tcblower{}
  \begin{align*}
    \lim\limits_{z \to 3i} f(z) &= \lim\limits_{z \to 3i} z^{2} \\
                                &= {(3i)}^{2} \\
                                &= 9 i^{2} = 9 (-1) \\
                                &= -9
  \end{align*}

  Thus, $\lim\limits_{z \to 3i} f(z) = -9$.
\end{example}

\begin{remark*}
  However, do note that the act of finding the \nameref{def:Complex_Limit} of a function approaching a point is distinctly different than finding the value of the function \textbf{at} that point.
\end{remark*}


%%% Local Variables:
%%% mode: latex
%%% TeX-master: "../../Math_333-MatrixAlg_ComplexVars-Reference_Sheet"
%%% End:


\subsection{Derivatives}\label{subsec:Derivatives}
\begin{definition}[Derivative]\label{def:Complex_Derivative}
  The \emph{derivative} of a \nameref{def:Complex_Function}, like many other operations, must have its definitions slightly redefined to account for the extra dimension provided by the \nameref{def:Argand_Plane}.

  The \textbf{domain} of a derivative of a \nameref{def:Complex_Function} $f(z)$ requires that $f$ be defined in a \nameref{def:Neighborhood} of the point $a$.
  Once this is satisified, then \Cref{eq:Complex_Derivative} is used to find the actual value.
  \begin{equation}\label{eq:Complex_Derivative}
    f'(a) = \lim\limits_{z \to a} \frac{f(z) - f(a)}{z - a}
  \end{equation}

  \begin{remark}[Path Existence]\label{rmk:Complex_Derivative_Path_Existence}
    Because the \nameref{def:Complex_Derivative} of a \nameref{def:Complex_Function} is defined with a \nameref{def:Complex_Limit}, \textbf{ALL} paths must exist and have the same value.
  \end{remark}

  \begin{remark}[Uses]\label{rmk:Complex_Derivative_Uses}
    The definition of a \nameref{def:Complex_Derivative} is rarely used to calculate anything.
    Instead, it is used to prove the non-existence of a derivative of a complex function.
  \end{remark}
\end{definition}

\begin{definition}[Neighborhood]\label{def:Neighborhood}
  A \emph{neighborhood} is an open disk $\Modulus{z-a} < r$ for some $r$.
  The radius $r$ is unspecified, meaning that we must choose its value.
\end{definition}

\begin{example}[Lecture 5]{Derivative of Complex Function}
  Given $f(z)$, what is its \nameref{def:Complex_Derivative} at a point $a$, $f'(a)$?
  \begin{equation*}
    f(z) = z^{2}
  \end{equation*}
  \tcblower{}
  As this problem is simple, we can just apply \Cref{eq:Complex_Derivative} directly.
  \begin{align*}
    f'(a) &= \lim\limits_{z \to a} \frac{f(z) - f(a)}{z - a} \\
          &= \lim\limits_{z \to a} \frac{z^{2} - a^{2}}{z - a} \\
          &= \lim\limits_{z \to a} \frac{(z-a) (z+a)}{z - a} \\
          &= \lim\limits_{z \to a} (z+a) \\
          &= (a + a) \\
          &= 2a
  \end{align*}

  Thus, $f'(a) = 2a$.
\end{example}

\begin{example}[Lecture 5]{Non-Existence of Derivative}
  Given $f(z)$, $a = 2i$, show that $f'(2i)$ does not exist?
  \begin{equation*}
    f(z) = \Conjugate{z}
  \end{equation*}
  \tcblower{}
  We start by applying the definition of a \nameref{def:Complex_Derivative} of a \nameref{def:Complex_Function}.
  \begin{align*}
    f'(a) &= \lim\limits_{z \to 2i} \frac{f(z) - f(a)}{z - a} \\
          &= \lim\limits_{z \to 2i} \frac{\Conjugate{z} - \Conjugate{a}}{z - a} \\
    f'(2i) &= \lim\limits_{z \to 2i} \frac{\Conjugate{z} - (-2i)}{z - a} \\
          &= \lim\limits_{z \to 2i} \frac{\Conjugate{z} + 2i}{z - 2i} \\
  \end{align*}

  Now, we can fall back to the definition of the non-existence of a \nameref{def:Complex_Limit} of a \nameref{def:Complex_Function}.
  If the value of the function approaching the point $a$ by two different paths ($\Path_{1}, \Path_{2}$) have two different values, the limit does not exist.
  If we choose our paths to be:
  \begin{align*}
    \Path_{1} &\coloneqq x = 0 \\
    \Path_{2} &\coloneqq y = x + 2i
  \end{align*}

  Now solving the limit above using Path 1 ($\Path_{1}$):
  \begin{align*}
    f'(2i) &= \lim\limits_{\substack{z \to 2i \\ z \text{ on } \Path_{1}}} \frac{\Conjugate{z} + 2i}{z - 2i} \\
           &= \lim\limits_{\substack{x = 0 \\ y \to 2}} \frac{-yi + 2i}{yi - 2i} \\
           &= \lim\limits_{y \to 2} \frac{-i (y-2)}{i (y-2)} \\
           &= -1
  \end{align*}

  Now solve the above limit using Path 2 ($\Path_{2}$):
  \begin{align*}
    f'(2i) &= \lim\limits_{\substack{z \to 2i \\ z \text{ on } \Path_{2}}} \frac{\Conjugate{z} + 2i}{z - 2i} \\
           &= \lim\limits_{\substack{x \to 0 \\ y = 2 \\ z = x+2i}} \frac{\Conjugate{z}+2i}{z-2i} \\
           &= \lim\limits_{\substack{x \to 0 \\ y = 2}} \frac{x-2i+2i}{x+2i-2i} \\
           &= \lim\limits_{\substack{x \to 0}} \frac{x}{x} \\
           &= 1
  \end{align*}

  Now, like the definition of a \nameref{def:Complex_Limit} of a \nameref{def:Complex_Function} states:
  \begin{align*}
    \lim\limits_{\substack{z \to a \\ z \text{ on } \Path_{1}}} \frac{\Conjugate{z} + 2i}{z - 2i} &\neq \lim\limits_{\substack{z \to a \\ z \text{ on } \Path_{2}}} \frac{\Conjugate{z} + 2i}{z - 2i} \\
    \lim\limits_{z \to a} \frac{f(z) - f(a)}{z - a} &= \DNE
  \end{align*}

  Thus, because the backing \nameref{def:Complex_Limit} does not exist, the entire \nameref{def:Complex_Derivative} does not exist.
\end{example}

\subsubsection{Nicities}\label{subsubsec:Complex_Derivative_Nicities}
\nameref{def:Complex_Derivative}s of \nameref{def:Complex_Function}s obey the same rules as purely-real calculus.
This includes:
\begin{itemize}[noitemsep]
\item Product Rule
\item Quotient Rule
\item Chain Rule
\item etc.
\end{itemize}

\subsubsection{Cauchy-Riemann Equations}\label{subsubsec:Cauchy_Riemann_Equations}
Because we know the \nameref{def:Complex_Derivative} of a \nameref{def:Complex_Function} \textbf{can} exist, and that $f(z) = U(x, y) + i V(x, y)$, we need to know just how special $U(x, y)$ and $V(x, y)$ really are.

If $f$ has a derivative at $z = a$, then the Cauchy-Riemann Equations \textbf{MUST} also hold true for $f'(z)$ to exist.
In the Cartesian form, these equations are seen as \Cref{eq:Cauchy_Riemann_Equations-Cartesian}.
\begin{subequations}\label{eq:Cauchy_Riemann_Equations-Cartesian}
  \begin{equation}\label{subeq:Cauchy_Riemann_Equation-UxVy}
    \begin{aligned}
      \frac{\partial U}{\partial x} (a) &= \frac{\partial V}{\partial y} (a) \\
      U_{x} (a) &= V_{y} (a)
    \end{aligned}
  \end{equation}
  \begin{equation}\label{subeq:Cauchy_Riemann_Equation-UyVx}
    \begin{aligned}
      \frac{\partial U}{\partial y} (a) &= -\frac{\partial V}{\partial x} (a) \\
      U_{y} (a) &= -V_{x} (a)
    \end{aligned}
  \end{equation}
\end{subequations}

In the polar form, the Cauchy-Riemann Equations are seen as \Cref{eq:Cauchy_Riemann_Equations-Polar}:
\begin{subequations}\label{eq:Cauchy_Riemann_Equations-Polar}
  \begin{equation}\label{subeq:Cauchy_Riemann_Equations-UrVt}
    \begin{aligned}
      \frac{\partial U}{\partial r} (a) &= \frac{1}{r} \frac{\partial V}{\partial \theta} (a)\\
      U_{r} (a) &= \frac{1}{r} V_{\theta} (a)
    \end{aligned}
  \end{equation}
  \begin{equation}\label{subeq:Cauchy_Riemann_Equations-UtVr}
    \begin{aligned}
      \frac{\partial V}{\partial r} (a) &= \frac{-1}{r} \frac{\partial U}{\partial \theta} (a) \\
      V_{r} (a) &= \frac{-1}{r} V_{\theta} (a)
    \end{aligned}
  \end{equation}
\end{subequations}

\begin{example}[Lecture 5]{Existence of Derivative using Cauchy-Riemann Equations}
  Given the function $f(z)$, find its \nameref{def:Complex_Derivative}?
  \begin{equation*}
    f(z) = \Conjugate{z} = x - iy
  \end{equation*}
  \tcblower{}
  First, we identify the functions $U(x, y)$ and $V(x, y$).
  \begin{align*}
    U(x, y) &= x \\
    V(x, y) &= -y
  \end{align*}

  Now we start by checking \Cref{subeq:Cauchy_Riemann_Equation-UxVy}.
  \begin{align*}
    \frac{\partial U}{\partial x} &= 1 \\
    \frac{\partial V}{\partial y} &= -1 \\
    \frac{\partial U}{\partial x} &\neq \frac{\partial V}{\partial y} \\
    U_{x} &\neq V_{y}
  \end{align*}

  Because the \nameref{subsubsec:Cauchy_Riemann_Equations} are \textbf{not} satisified, $f(z) = \Conjugate{z}$ has \textbf{NO} \nameref{def:Complex_Derivative} at any point $z$.
\end{example}

\begin{theorem}\label{thm:Cauchy_Riemann_Complex_Function_Derivative}
  Suppose $f$ is a \nameref{def:Complex_Function} defined in the \nameref{def:Neighborhood} $\Modulus{z-a} < r$ for some $r$.
  Suppose the \nameref{subsubsec:Cauchy_Riemann_Equations} hold at a point, $a$, and that the 4 partial derivatives $U_{x}$, $U_{y}$, $V_{x}$, and $V_{y}$ exist and are \underline{continuous} at $z=a$.
  Then the \nameref{def:Complex_Derivative} of $f$ at $z=a$ is defined to be:
  \begin{equation}\label{eq:Complex_Function_Derivative-Cauchy_Riemann_Equation_Solution}
    \begin{aligned}
      f'(z) &= \frac{\partial U}{\partial x} + i \frac{\partial V}{\partial x} \\
      f'(a) &= \frac{\partial V}{\partial y}(a) - i \frac{\partial U}{\partial y}(a)
    \end{aligned}
  \end{equation}
\end{theorem}

\begin{theorem}\label{thm:Cauchy_Riemann_Complex_Function_Derivative-Polar}
  Suppose $f$ is a \nameref{def:Complex_Function} defined in the \nameref{def:Neighborhood} $\Modulus{z-a} < r$ for some $r$.
  Suppose the polar form of the \nameref{subsubsec:Cauchy_Riemann_Equations} (\Cref{eq:Cauchy_Riemann_Equations-Polar}) hold at a point, $a$, and that the 4 partial derivatives $U_{r}$, $U_{\theta}$, $V_{r}$, and $V_{\theta}$ exist and are \underline{continuous} at $z=a$.
  Then the \nameref{def:Complex_Derivative} of $f$ at $z=a$ is defined to be:
  \begin{equation}\label{eq:Complex_Function_Derivative-Cauchy_Riemann_Equation_Solution}
    \begin{aligned}
      f'(z) &= e^{-i \theta} \left( \frac{\partial U}{\partial r} + i \frac{\partial V}{\partial r} \right) \\
      f'(a) &= e^{-i \theta} \left( \frac{\partial V}{\partial \theta}(a) - i \frac{\partial U}{\partial \theta}(a) \right)
    \end{aligned}
  \end{equation}
\end{theorem}

\begin{remark*}
  If the \nameref{subsubsec:Cauchy_Riemann_Equations} fail to hold at $z = a$, then $f(z)$ fails to have a \nameref{def:Complex_Derivative} at $z = a$.
\end{remark*}

\begin{remark*}
  There \textit{are} functions where the \nameref{subsubsec:Cauchy_Riemann_Equations} hold at a point $a$, but the function does \textbf{NOT} have a \nameref{def:Complex_Derivative} at that point.
  However, these are rare pathological examples, so we will not discuss these functions.
\end{remark*}

\begin{example}[Lecture 5]{Differentiate Complex Function using Cauchy-Riemann Equations}
  Given the function $f(z)$, use the \nameref{subsubsec:Cauchy_Riemann_Equations} to find $f'(z)$?
  \begin{equation*}
    f(z) = z^{2} = \left( x^{2} - y^{2} \right) + 2xyi
  \end{equation*}
  \tcblower{}
  We identify $U(x, y)$ and $V(x, y)$ first.
  \begin{align*}
    U(x, y) &= x^{2} - y^{2} \\
    V(x, y) &= 2xyi
  \end{align*}

  Use \Cref{eq:Cauchy_Riemann_Equations-Cartesian}.
  \begin{align*}
    \frac{\partial U}{\partial x} &= 2x & \frac{\partial V}{\partial y} &= 2x \\
    \frac{\partial U}{\partial y} &= -2y & \frac{\partial V}{\partial x} &= 2y \\
  \end{align*}
  \begin{align*}
    \frac{\partial U}{\partial x} = 2&x = \frac{\partial V}{\partial y} \\
    \frac{\partial U}{\partial y} = -2&y = -\frac{\partial V}{\partial x} \\
  \end{align*}

  $U_{x}, V_{y}, U_{y}, V_{x}$ are polynomials.
  Hence, they are continuous.
  Hence $f$ has a derivative at all points.
  Because this function passes the requirements placed on $f(z)$ by the \nameref{subsubsec:Cauchy_Riemann_Equations}, we can find the \nameref{def:Complex_Derivative} of $f(z)$.
  \begin{align*}
    f'(z) &= \frac{\partial U}{\partial x} + i \frac{\partial V}{\partial x} \\
          &= 2x + 2yi
  \end{align*}

  Thus, $f'(z) = 2x + 2yi$.
\end{example}

\begin{example}[Lecture 5]{Differentiate Complex Trig using Cauchy-Riemann Equations}
  Given the function $f(z) = \cos(z)$, verify that $f'(z) = -\sin(z)$?
  \tcblower{}
  Start by identifying $U(x, y)$ and $V(x, y)$.
  \begin{align*}
    f(z) &= \cos(z) \\
    \cos(z) &= \cos(x + iy) \\
         &= \cos(x) \cosh(y) - i \sin(x) \sinh(y) \\
    U(x, y) &= \cos(x) \cosh(y) \\
    V(x, y) &= -\sin(x) \sinh(y) \\
  \end{align*}

  Now use \Cref{eq:Cauchy_Riemann_Equations-Cartesian}.
  \begin{align*}
    \frac{\partial U}{\partial x} &= -\sin(x) \cosh(y) & \frac{\partial V}{\partial y} &= -\sin(x) \cosh(y) \\
    \frac{\partial U}{\partial y} &= \cos(x) \sinh(y) & \frac{\partial V}{\partial x} &= -\cos(x) \sinh(y)
  \end{align*}
  \begin{align*}
    \frac{\partial U}{\partial x} = -\sin(x)& \cosh(y) = \frac{\partial V}{\partial y} \\
    \frac{\partial U}{\partial y} = \cos(x)&\sinh(y) = -\frac{\partial V}{\partial x}
  \end{align*}

  Thus, the \nameref{subsubsec:Cauchy_Riemann_Equations} are satisified.
  In addition, the 4 partial derivatives are continuous at all points.

  Therefore, $f(z) = \cos(z)$ \textbf{has} a derivative at all points.
  According to \Cref{eq:Complex_Function_Derivative-Cauchy_Riemann_Equation_Solution}, the solution is:
  \begin{align*}
    f'(z) &= \frac{\partial U}{\partial x} + i \frac{\partial V}{\partial x} \\
          &= -\sin(x) \cosh(y) + i \bigl( -\cos(x) \sinh(y) \bigr) \\
    \shortintertext{Factor out the negative.}
          &= - \bigl( \sin(x) \cosh(y) + i \cos(x) \sinh(y) \bigr) \\
    \shortintertext{By the definition of $\sin(z)$ in Cartesian form, we can simplify everything in the parentheses.}
          &= -\sin(x + iy) \\
          &= -\sin(z)
  \end{align*}
\end{example}


\end{definition}

\begin{definition}[Open Connected Set]\label{def:Open_Connected_Set}
  An \emph{open connected set} is a special type of set that we use to visualize on the \nameref{def:Argand_Plane}.
  \begin{description}[noitemsep]
  \item[Set:] As a set, it can be visualized as a blob in the \nameref{def:Argand_Plane}.
  \item[Open:] The set being open means that the boundary edge is \textbf{not} included with the set.
    This also means that for every point within the set, we can define a disk with a radius, where the \textbf{entire} disk is contained within the set.
  \item[Connected:] When a union of two points, with their disks, occurs inside this set, there is a \textbf{continuous} path between them that lies entirely within the set's boundaries.
  \end{description}
\end{definition}

\begin{theorem}
\begin{definition}[Analytic]\label{def:Analytic}
  Let $f$ be a \nameref{def:Complex_Function} ($f: \Omega \to \ComplexNumbers$) which has a \nameref{def:Complex_Derivative} at \textbf{all} points of an \nameref{def:Open_Connected_Set} $\Omega$.
  Then we say $f$ is \emph{analytic} in $\Omega$.

  This means that $f$ has a derivative at all points in the \nameref{def:Open_Connected_Set}.
\end{definition}

  Define a function $f$ like so:
  \begin{equation*}
    f(z) = U(x, y) + i V(x, y)
  \end{equation*}

  If the function $f$ is \nameref{def:Analytic} on $\Omega$, then $f$ has a derivative at all points in $\Omega$, meaning
  \begin{align*}
    \frac{\partial U}{\partial x} &= \frac{\partial V}{\partial y} \\
    \frac{\partial U}{\partial y} &= -\frac{\partial V}{\partial x}
  \end{align*}

  This means that the \nameref{subsubsec:Cauchy_Riemann_Equations} hold, and the partial derivatives are valid.

  Then, according to our work in \Cref{subsubsec:Special_U_V_Interdependency}, $U(x, y)$ and $V(x, y)$ are \nameref{def:Harmonic}.
\end{theorem}

\subsubsection{Specialties of $U$, $V$, and their Interdependency}\label{subsubsec:Special_U_V_Interdependency}
We start by using \Cref{eq:Cauchy_Riemann_Equations-Cartesian}.
\begin{align*}
  \frac{\partial U}{\partial x} &= \frac{\partial V}{\partial y} \\
  \frac{\partial U}{\partial y} &= -\frac{\partial V}{\partial x}
\end{align*}

Now, if we take the partial derivative of $U_{x} = V_{y}$ with respect to $y$
\begin{align*}
  \frac{\partial}{\partial y} \left( \frac{\partial U}{\partial x} \right) &= \frac{\partial}{\partial y} \left( \frac{\partial V}{\partial y} \right) \\
  \frac{\partial^{2} U}{\partial y \partial x} &= \frac{\partial^{2} V}{{\partial y}^{2}}
\end{align*}

If we take the partial derivative of $U_{y} = -V_{x}$ with respect to $x$
\begin{align*}
  \frac{\partial}{\partial x} \left( \frac{\partial U}{\partial y} \right) &= \frac{\partial}{\partial x} \left( -\frac{\partial V}{\partial x} \right) \\
  \frac{\partial^{2} U}{\partial x \partial y} &= -\frac{\partial^{2} V}{{\partial x}^{2}}
\end{align*}

For the class of functions we are concerned with, the order of differentiation does not matter, meaning $\frac{\partial^{2}}{\partial x \partial y} = \frac{\partial^{2}}{\partial y \partial x}$.

Now, if we add $\frac{\partial^{2} U}{\partial y \partial x} = \frac{\partial^{2} V}{{\partial y}^{2}}$ and $\frac{\partial^{2} U}{\partial x \partial y} = -\frac{\partial^{2} V}{{\partial x}^{2}}$, then:

\begin{subequations}\label{eq:U_V_Harmonic_Relation}
  \begin{equation}\label{subeq:U_V_Harmonic_Relation-V}
    \frac{\partial^{2} V}{{\partial y}^{2}} + \frac{\partial^{2} V}{{\partial x}^{2}} = 0
  \end{equation}
  \begin{equation}\label{subeq:U_V_Harmonic_Relation-U}
    \frac{\partial^{2} U}{{\partial x}^{2}} + \frac{\partial^{2} U}{{\partial y}^{2}} = 0
  \end{equation}
\end{subequations}

The class of equations that satisfy \Cref{eq:U_V_Harmonic_Relation} are called \nameref{def:Harmonic} functions.

\begin{definition}[Harmonic]\label{def:Harmonic}
  \emph{Harmonic} functions are real-valued functions that are twice continuously differentiable function $f: U \to R$, where $U$ is an open subset of $\RealNumbers^{n}$, that satisfies~\nameref{subsec:Laplaces_Equation} (\Cref{eq:Laplaces_Equation}).
\end{definition}

\begin{example}[Lecture 5]{Prove Function Terms are Harmonic}
  Given $f(z)$, verify $U(x, y)$ is \nameref{def:Harmonic}?
  \begin{equation*}
    f(z) = z^{3}
  \end{equation*}
  \tcblower{}
  We start by needing to find $U(x, y)$ and $V(x, y)$.
  \begin{align*}
    f(z) &= z^{3} \\
         &= z^{2} z \\
         &= \left( x^{2} - y^{2} + 2xyi \right) (x + iy) \\
         &= x^{3} - xy^{2} + ix^{2}y - iy^{3} - 2xy^{2} + 2x^{2}yi \\
         &= x^{3} - 3xy^{2} + i \left( 3x^{2}y - y^{3} \right) \\
    U(x, y) &= x^{3} - 3xy^{2} \\
    V(x, y) &= 3x^{2}y - y^{3}
  \end{align*}

  Now that we have $U(x, y)$, we can find the partial derivatives.
  \begin{align*}
    \frac{\partial U}{\partial x} &= 3x^{2} - 3y^{2} \\
    \frac{\partial^{2} U}{{\partial x}^{2}} &= 6x \\
    \frac{\partial U}{\partial y} &= -6xy \\
    \frac{\partial^{2} U}{{\partial y}^{2}} &= -6x \\
  \end{align*}

  Now, we use \Cref{subeq:U_V_Harmonic_Relation-U} to check the validity of the \nameref{def:Harmonic} relation.
  \begin{align*}
    \frac{\partial^{2} U}{{\partial x}^{2}} + \frac{\partial^{2} U}{{\partial y}^{2}} &= 0 \\
    6x + -6x &= 0 \\
    0 &\overset{\checkmark}{=} 0
  \end{align*}

  Thus, $U(x, y) = x^{3} - 3xy^{2}$ \textbf{is} \nameref{def:Harmonic}.
  Similarly, $V(x, y) = 3x^{2}y - y^{3}$ is also harmonic.
\end{example}
%%% Local Variables:
%%% mode: latex
%%% TeX-master: "../../Math_333-MatrixAlg_ComplexVars-Reference_Sheet"
%%% End:


%%% Local Variables:
%%% mode: latex
%%% TeX-master: "../Math_333-MatrixAlg_ComplexVars-Reference_Sheet"
%%% End:
