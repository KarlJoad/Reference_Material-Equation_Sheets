\section{Complex Functions}\label{sec:Complex_Functions}
Complex functions, like their real-valued counterparts behave in much the same way.
\begin{equation}\label{eq:Complex_Function}
  f(z) = w
\end{equation}
\begin{description}[noitemsep]
\item $f$: The function or \nameref{def:Mapping} that corresponds the input to the output.
\item $z$: The input to the complex function/\nameref{def:Mapping}.
\item $w$: The output of the complex function/\nameref{def:Mapping}.
\end{description}

\begin{definition}[Mapping]\label{def:Mapping}
  A \emph{mapping} is synonym for a function in mathematics.
  The term comes from set theory, where the input set is mapped to an output set by some operations.
  The conventional way to denote a mapping is with the $\mapsto$ symbol.

  An example of a mapping is shown in \Cref{eq:Mapping}
  \begin{equation}\label{eq:Mapping}
    z \mapsto z^{2}
  \end{equation}
\end{definition}

A complex function can only accept and will only return values in \textbf{Cartesian} or \textbf{polar} form.
Because the output of a complex function is also a complex value, \Cref{eq:Output_Value_Function} makes sense.
\begin{equation}\label{eq:Output_Value_Function}
  f(z) = U(x, y) + iV(x, y)
\end{equation}

$U(x, y)$ and $V(x, y)$ can be as general as we want in $x$ and $y$.
This means both could be constants, both could be polynomials, one could be trascendental, and anything in between.

The functions $U(x, y)$ and $V(x, y)$ are functions that yield real-values $u, v$.
This means that $u, v$ can also be graphed on an \nameref{def:Argand_Plane}.
By our definition of $U(x, y)$ and $V(x, y)$, $U(x, y), V(x, y)$ are parametric functions.

\begin{example}[Lecture 4]{Find Output Functions}
  Given the \nameref{def:Mapping} $z \mapsto z^{2}$, where $z = x + iy$, find the output functions for each term $U(x, y)$ and $V(x, y)$?
  \tcblower{}
  I will choose to represent the mapping $z \mapsto z^{2}$ with the complex function $f(z) = z^{2}$.
  \begin{align*}
    z &\mapsto z^{2} \\
    f(z) &= z^{2} \\
    \shortintertext{Apply the definition of $z$.}
      &= {(x + iy)}^{2} \\
      &= x^{2} + 2xyi + i^{2}y^{2} \\
      &= x^{2} + 2xyi + (-1)y^{2} \\
      &= \left( x^{2} - y^{2} \right) + 2xyi \\
  \end{align*}

  By out definition of $U(x, y)$ and $V(x, y)$ in \Cref{eq:Output_Value_Function}, we can finish solving this.
  \begin{align*}
    f(z) &= U(x, y) + iV(x, y) \\
    f(z) &= \left( x^{2} - y^{2} \right) + 2xyi \\
    U(x, y) &= x^{2} - y^{2} \\
    V(x, y) &= 2xy
  \end{align*}

  Thus, our output functions are $U(x, y) = x^{2} - y^{2}$ and $V(x, y) = 2xy$.
\end{example}




%%% Local Variables:
%%% mode: latex
%%% TeX-master: "../Math_333-MatrixAlg_ComplexVars-Reference_Sheet"
%%% End:
