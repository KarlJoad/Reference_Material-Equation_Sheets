\section{Complex Series}\label{sec:Complex_Series}
We start our discussion of complex series by talking about \nameref{def:Complex_Power_Series}.

\begin{definition}[Power Series]\label{def:Complex_Power_Series}
  A \emph{power series} is an infinite summation of terms that form infinitely long polynomials.
  These are defined around a center $a$, $a \in \ComplexNumbers$.
  \begin{equation}\label{eq:Complex_Power_Series}
    \sum_{n=0}^{\infty} a_{n} {(z-a)}^{n} = a_{0} + a_{1} (z-a) + a_{2} {(z-a)}^{2} + \cdots
  \end{equation}
\end{definition}

Because our \nameref{def:Complex_Power_Series} are infinite, we now have to ask about its convergence.
There is a theorem for this, \Cref{thm:Radius_of_Convergence}, called the \nameref{thm:Radius_of_Convergence}.

\begin{theorem}[Radius of Convergence]\label{thm:Radius_of_Convergence}
  There exists a number $R$, $0 \leq R \leq \infty$, such that $\forall z, \Modulus{z-a} < R$ the series is convergent, and $\forall z, \Modulus{z-a} > R$ is divergent.
  $forall z, \Modulus{z-a} = R$ does not have guaranteed divergence or convergence.
  This value $R$ is called the \emph{Radius of Convergence}, or the \emph{RoC}.
\end{theorem}


%%% Local Variables:
%%% mode: latex
%%% TeX-master: "../Math_333-MatrixAlg_ComplexVars-Reference_Sheet"
%%% End:
