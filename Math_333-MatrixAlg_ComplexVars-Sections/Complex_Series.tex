\section{Complex Series}\label{sec:Complex_Series}
We start our discussion of complex series by talking about \nameref{def:Complex_Power_Series}.

\begin{definition}[Power Series]\label{def:Complex_Power_Series}
  A \emph{power series} is an infinite summation of terms that form infinitely long polynomials.
  These are defined around a center $a$, $a \in \ComplexNumbers$.
  \begin{equation}\label{eq:Complex_Power_Series}
    \sum_{n=0}^{\infty} a_{n} {(z-a)}^{n} = a_{0} + a_{1} (z-a) + a_{2} {(z-a)}^{2} + \cdots
  \end{equation}
\end{definition}

Because our \nameref{def:Complex_Power_Series} are infinite, we now have to ask about its convergence.
There is a theorem for this, \Cref{thm:Region_of_Convergence}, called the \nameref{thm:Region_of_Convergence}.

\begin{theorem}[Region of Convergence]\label{thm:Region_of_Convergence}
  There exists a number $R$, $0 \leq R \leq \infty$, such that $\forall z, \Modulus{z-a} < R$ the series is convergent, and $\forall z, \Modulus{z-a} > R$ is divergent.
  $forall z, \Modulus{z-a} = R$ does not have guaranteed divergence or convergence.
  This value $R$ is called the \emph{Region of Convergence}, or the \emph{RoC}.
\end{theorem}

\begin{example}[Lecture 8, Example 3]{Finding the Region of Convergence}
  Compute the radius of convergence of the power series shown below.
  \begin{equation*}
    \sum_{n=0}^{\infty} \frac{3^{2n+5}}{2n+3} {(z-i)}^{7n+2}
  \end{equation*}
  \tcblower{}
  We can solve this by using d'Alembert's Ratio Test.
  This test states that for an infinite series to be convergent, $\lim_{n \to \infty} \Modulus{\frac{a_{n+1}}{a_{n}}} = C$, where $C$ is some constant value.

  Applying d'Alembert's Ratio test, we have
  \begin{align*}
    \lim_{n \to \infty} \Modulus{\frac{a_{n+1}}{a_{n}}} &= \lim_{n \to \infty} \Modulus{ \frac{\frac{3^{2(n+1) + 5}}{2(n+1) + 3} {(z-i)}^{7(n+1)+2}}{\frac{3^{2n+5}}{2n+3} {(z-i)}^{7n+2}}} \\
                                                        &= \lim_{n \to \infty} \Modulus{\frac{\frac{3^{2n+7}}{2n+5} {(z-i)}^{7n+9}}{\frac{3^{2n+5}}{2n+3} {(z-i)}^{7n+2}}} \\
    \shortintertext{We can cancel the values with exponents raised to the $n$.}
                                                        &= \lim_{n \to \infty} \frac{2n + 3}{2n + 5} 3^{2} \Modulus{{(z-i)}^{7}} \\
    \shortintertext{Evaluating the limit, we see that the fraction with $n$ will become $\frac{1}{1}$.}
                                                        &= 3^{2} \Modulus{{(z-i)}^{7}} \\
    \shortintertext{Using the Law of Moduli, we can simplify this.}
                                                        &= 3^{2} {\Modulus{z-i}}^{7}
  \end{align*}

  Now, we have to use \Cref{thm:Region_of_Convergence}.
  \begin{align*}
    \forall z, 3^{2} {\Modulus{z-i}}^{7} &< 1 \text{, Series is convergent} \\
    \forall z, 3^{2} {\Modulus{z-i}}^{7} &> 1 \text{, Series is divergent} \\
  \end{align*}

  Now, we move terms and exponents to the other side, leaving just $\Modulus{z-i}$ on the left.
  \begin{align*}
    \forall z, \Modulus{z-i} &< \frac{1}{3^{\frac{2}{7}}} \text{, Series is convergent} \\
    \forall z, \Modulus{z-i} &> \frac{1}{3^{\frac{2}{7}}} \text{, Series is divergent} \\
  \end{align*}
\end{example}

\begin{theorem}\label{thm:Power_Series-Analytic_Function}
  If the \nameref{def:Complex_Power_Series} $\sum_{n=0}^{\infty} a_{n} {(z-a)}^{n}$ has a \nameref{thm:Region_of_Convergence} $RoC > 0$, we can define $f(z)$ to be the summation of all the terms in the series, $f(z) = \sum_{n = 0}^{\infty} a_{n} {(z-a)}^{n}$, $\Modulus{z-a} < R$.

  Then, $f$ is \nameref{def:Analytic} in $\Modulus{z-a} < R$ and $f'(z)$ is term-by-term differentiable.
  \begin{align*}
    f(z) &= a_{0} + a_{1} (z-a) + a_{2} {(z-a)}^{2} + a_{3} {(z-a)}^{3} \cdots \\
    f'(z) &= 0 + a_{1} + 2 a_{2} (z-a) + 3 a_{3} {(z-a)}^{2} + \cdots \\
         &= \sum_{n=1}^{\infty} n a_{n} {(z-a)}^{n-1}, \, \forall z, \, \Modulus{z-a} < R
  \end{align*}
\end{theorem}

Likewise, the converse of \Cref{thm:Power_Series-Analytic_Function} holds true, seen in \Cref{thm:Analytic_Function-Power_Series}

\begin{theorem}\label{thm:Analytic_Function-Power_Series}
  Let $f$ be an \nameref{def:Analytic} \nameref{def:Complex_Function} on the disk $\Modulus{z-a} < R$.

  Then there exists a \nameref{def:Complex_Power_Series} with the same center $a$, such that
  \begin{equation*}
    f(z) = \sum_{n = 0}^{\infty} a_{n} {(z-a)}^{n}
  \end{equation*}
\end{theorem}

\subsection{Power Series of Common Functions}\label{subsec:Common_Function_Power_Series}
All of the common transcendental functions below are \nameref{def:Analytic} on the entire complex place, $\Modulus{z} < \infty$.

\begin{equation}\label{eq:Complex_Power_Series-e}
  \begin{aligned}
    e^{z} &= \sum_{n=0}^{\infty} \frac{z^{n}}{n!} \\
    & = 1 + \frac{1}{1!} z + \frac{1}{2!} z + \frac{1}{3!} z^{3} + \cdots
  \end{aligned}
\end{equation}

%%% Local Variables:
%%% mode: latex
%%% TeX-master: "../Math_333-MatrixAlg_ComplexVars-Reference_Sheet"
%%% End:
