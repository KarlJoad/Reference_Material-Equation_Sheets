\section{Complex Series}\label{sec:Complex_Series}
We start our discussion of complex series by talking about \nameref{def:Complex_Power_Series}.

\begin{definition}[Power Series]\label{def:Complex_Power_Series}
  A \emph{power series} is an infinite summation of terms that form infinitely long polynomials.
  These are defined around a center $a$, $a \in \ComplexNumbers$.
  \begin{equation}\label{eq:Complex_Power_Series}
    \sum_{n=0}^{\infty} a_{n} {(z-a)}^{n} = a_{0} + a_{1} (z-a) + a_{2} {(z-a)}^{2} + \cdots
  \end{equation}
\end{definition}

Because our \nameref{def:Complex_Power_Series} are infinite, we now have to ask about its convergence.
There is a theorem for this, \Cref{thm:Region_of_Convergence}, called the \nameref{thm:Region_of_Convergence}.

\begin{theorem}[Region of Convergence]\label{thm:Region_of_Convergence}
  There exists a number $R$, $0 \leq R \leq \infty$, such that $\forall z, \Modulus{z-a} < R$ the series is convergent, and $\forall z, \Modulus{z-a} > R$ is divergent.
  $forall z, \Modulus{z-a} = R$ does not have guaranteed divergence or convergence.
  This value $R$ is called the \emph{Region of Convergence}, or the \emph{RoC}.
\end{theorem}

\begin{example}[Lecture 8, Example 3]{Finding the Region of Convergence}
  Compute the radius of convergence of the power series shown below.
  \begin{equation*}
    \sum_{n=0}^{\infty} \frac{3^{2n+5}}{2n+3} {(z-i)}^{7n+2}
  \end{equation*}
  \tcblower{}
  We can solve this by using d'Alembert's Ratio Test.
  This test states that for an infinite series to be convergent, $\lim_{n \to \infty} \Modulus{\frac{a_{n+1}}{a_{n}}} = C$, where $C$ is some constant value.

  Applying d'Alembert's Ratio test, we have
  \begin{align*}
    \lim_{n \to \infty} \Modulus{\frac{a_{n+1}}{a_{n}}} &= \lim_{n \to \infty} \Modulus{ \frac{\frac{3^{2(n+1) + 5}}{2(n+1) + 3} {(z-i)}^{7(n+1)+2}}{\frac{3^{2n+5}}{2n+3} {(z-i)}^{7n+2}}} \\
                                                        &= \lim_{n \to \infty} \Modulus{\frac{\frac{3^{2n+7}}{2n+5} {(z-i)}^{7n+9}}{\frac{3^{2n+5}}{2n+3} {(z-i)}^{7n+2}}} \\
    \shortintertext{We can cancel the values with exponents raised to the $n$.}
                                                        &= \lim_{n \to \infty} \frac{2n + 3}{2n + 5} 3^{2} \Modulus{{(z-i)}^{7}} \\
    \shortintertext{Evaluating the limit, we see that the fraction with $n$ will become $\frac{1}{1}$.}
                                                        &= 3^{2} \Modulus{{(z-i)}^{7}} \\
    \shortintertext{Using the Law of Moduli, we can simplify this.}
                                                        &= 3^{2} {\Modulus{z-i}}^{7}
  \end{align*}

  Now, we have to use \Cref{thm:Region_of_Convergence}.
  \begin{align*}
    \forall z, 3^{2} {\Modulus{z-i}}^{7} &< 1 \text{, Series is convergent} \\
    \forall z, 3^{2} {\Modulus{z-i}}^{7} &> 1 \text{, Series is divergent} \\
  \end{align*}

  Now, we move terms and exponents to the other side, leaving just $\Modulus{z-i}$ on the left.
  \begin{align*}
    \forall z, \Modulus{z-i} &< \frac{1}{3^{\frac{2}{7}}} \text{, Series is convergent} \\
    \forall z, \Modulus{z-i} &> \frac{1}{3^{\frac{2}{7}}} \text{, Series is divergent} \\
  \end{align*}
\end{example}

\begin{theorem}\label{thm:Power_Series-Analytic_Function}
  If the \nameref{def:Complex_Power_Series} $\sum_{n=0}^{\infty} a_{n} {(z-a)}^{n}$ has a \nameref{thm:Region_of_Convergence} $RoC > 0$, we can define $f(z)$ to be the summation of all the terms in the series, $f(z) = \sum_{n = 0}^{\infty} a_{n} {(z-a)}^{n}$, $\Modulus{z-a} < R$.

  Then, $f$ is \nameref{def:Analytic} in $\Modulus{z-a} < R$ and $f'(z)$ is term-by-term differentiable.
  \begin{align*}
    f(z) &= a_{0} + a_{1} (z-a) + a_{2} {(z-a)}^{2} + a_{3} {(z-a)}^{3} \cdots \\
    f'(z) &= 0 + a_{1} + 2 a_{2} (z-a) + 3 a_{3} {(z-a)}^{2} + \cdots \\
         &= \sum_{n=1}^{\infty} n a_{n} {(z-a)}^{n-1}, \, \forall z, \, \Modulus{z-a} < R
  \end{align*}
\end{theorem}

Likewise, the converse of \Cref{thm:Power_Series-Analytic_Function} holds true, seen in \Cref{thm:Analytic_Function-Power_Series}

\begin{theorem}\label{thm:Analytic_Function-Power_Series}
  Let $f$ be an \nameref{def:Analytic} \nameref{def:Complex_Function} on the disk $\Modulus{z-a} < R$.

  Then there exists a \nameref{def:Complex_Power_Series} with the same center $a$, such that
  \begin{equation*}
    f(z) = \sum_{n = 0}^{\infty} a_{n} {(z-a)}^{n}
  \end{equation*}
\end{theorem}

\subsection{Power Series of Common Functions}\label{subsec:Common_Function_Power_Series}
All of the common transcendental functions below are \nameref{def:Analytic} on the entire complex place, $\Modulus{z} < \infty$.

\begin{equation}\label{eq:Complex_Power_Series-e}
  \begin{aligned}
    e^{z} &= \sum_{n=0}^{\infty} \frac{z^{n}}{n!} \\
    & = 1 + \frac{1}{1!} z + \frac{1}{2!} z + \frac{1}{3!} z^{3} + \cdots
  \end{aligned}
\end{equation}

\begin{equation}\label{eq:Complex_Power_Series-cos}
  \begin{aligned}
    \cos(z) &= \sum_{n=0}^{\infty} {(-1)}^{n} \frac{z^{2n}}{(2n)!} \\
    &= 1 - \frac{z^{2}}{2!} + \frac{z^{4}}{4!} - \frac{z^{6}}{6!} + \cdots
  \end{aligned}
\end{equation}

\begin{equation}\label{eq:Complex_Power_Series-sin}
  \begin{aligned}
    \sin(z) &= \sum_{n=0}^{\infty} {(-1)}^{n} \frac{z^{2n+1}}{(2n+1)!} \\
    &= z - \frac{z^{3}}{3!} + \frac{z^{5}}{5!} - \cdots
  \end{aligned}
\end{equation}

Now, some examples on how to use these series to solve problems.
\begin{example}[Lecture 9 Example 1]{Exponential not Centered on Origin}
  The mapping $z \to e^{z}$ is \nameref{def:Analytic} on $\Modulus{z-2} < \infty$.
  Determine its \nameref{def:Complex_Power_Series}?
  \tcblower{}
  We start by remembering that we already have a general power series for $e^{z}$, but this one is centered on $z = 0$.
  \begin{equation*}
    e^{z} = \sum_{n=0}^{\infty} \frac{z^{n}}{n!}
  \end{equation*}

  We can try just replacing $e^{z}$ weith $e^{z-2}$.
  \begin{align*}
    e^{z-2} &= \sum_{n=0}^{\infty} \frac{{(z-2)}^{n}}{n!} \\
    \intertext{The series above is technically correct, but not in the form we want.}
    e^{z} e^{-2} &= \sum_{n=0}^{\infty} \frac{{(z-2)}^{n}}{n!} \\
            &= e^{2} \sum_{n=0}^{\infty} \frac{{(z-2)}^{n}}{n!} \\
            &= \sum_{n=0}^{\infty} \frac{e^{2}}{n!} {(z-2)}^{n} \\
  \end{align*}
\end{example}

\begin{example}[Lecture 9, Example 2]{Power Series of Sinusoid with Angle Squared}
  Find the \nameref{def:Complex_Power_Series} for $z \to \sin(z^{2})$ for $\Modulus{z} < \infty$?
  \tcblower{}
  We already know what $\sin(z)$ is on $\Modulus{z} < \infty$, so we start there.
  \begin{equation*}
    \sin(z) = \sum_{n=0}^{\infty} {(-1)}^{n} \frac{z^{2n+1}}{(2n+1)!}
  \end{equation*}

  Now, we need to modify it to handle $z^{2}$.
  We can just replace all instances of $z$ with $z^{2}$.
  \begin{align*}
    \sin(z^{2}) &= \sum_{n=0}^{\infty} {(-1)}^{n} \frac{{\left( z^{2} \right)}^{2n+1}}{(2n+1)!} \\
                &= \sum_{n=0}^{\infty} {(-1)}^{n} \frac{z^{4n+2}}{(2n+1)!} \\
  \end{align*}
\end{example}

\begin{example}[Lecture 9, Example 3]{Power Series of Sinusoid not Centered on Origin}
  $z \to \sin(z)$ is \nameref{def:Analytic} on $\Modulus{z-2} < \infty$.
  Find the \nameref{def:Complex_Power_Series} of this function?
  \tcblower{}
  We start by taking what we already know, $\sin(z)$ on $\Modulus{z} < \infty$.
  \begin{equation*}
    \sin(z) = \sum_{n=0}^{\infty} {(-1)}^{n} \frac{z^{2n+1}}{(2n+1)!}
  \end{equation*}

  Now, we attempt to replace all instances of $z$ with $z-2$.
  \begin{align*}
    \sin(z) &= \sum_{n=0}^{\infty} {(-1)}^{n} \frac{z^{2n+1}}{(2n+1)!} \\
    \shortintertext{This is if $\sin$ were centered on $z=0$ though.}
            &= \sum_{n=0}^{\infty} {(-1)}^{n} \frac{{(z-2)}^{2n+1}}{(2n+1)!}
  \end{align*}

  But, that doesn't work, as we have no reasonable way to put this into our ``standard'' \nameref{def:Complex_Power_Series} form.
  Instead, start from the beginning again and say that $\sin(z) = \sin(z - 2 + 2)$.
  Now, we treat $\sin(z-2+2)$ as $\sin \bigl( (z-2) + 2 \bigr)$.
  We use the \nameref{subsec:Angle Sum and Difference Identities} to simplify this, namely \Cref{eq:Sin Angle Sum and Difference}.
  \begin{align*}
    \sin \bigl( (z-2) + 2 \bigr) &= \sin(z-2) \cos(2) + \cos(z-2) \sin(2) \\
    \shortintertext{Now we expand the sinusoids that have $z$ terms in them.}
                                 &= \cos(2) \left( \sum_{n=0}^{\infty} {(-1)}^{n} \frac{{(z-2)}^{2n+1}}{(2n+1)!} \right) + \sin(2) \left( \sum_{n=0}^{\infty} {(-1)}^{n} \frac{{(z-2)}^{2n}}{(2n)!} \right) \\
  \end{align*}

  Again, this is not in our ``standard'' form, but we can change it so it is, using 2 cases.
  \begin{equation*}
    \sin \bigl( (z-2) + 2 \bigr) = \sum_{n=0}^{\infty} a_{n} {(z-2)}^{n}
  \end{equation*}
  Where
  \begin{equation*}
    a_{n} =
    \begin{cases}
      \cos(2) {(-1)}^{\frac{n-1}{2}} \frac{{(z-2)}^{n}}{n!} & \text{If $n$ is odd} \\[5pt]

      \sin(2) {(-1)}^{\frac{n}{2}} \frac{{(z-2)}^{n}}{n!} & \text{If $n$ is even} \\
    \end{cases}
  \end{equation*}
\end{example}

One more useful series that we frequently use is the \nameref{def:Complex_Taylor_Series}.

\begin{definition}[Taylor Series]\label{def:Complex_Taylor_Series}
  The \emph{Taylor series} of a \nameref{def:Complex_Function} is created out of 2 equations.
  The first is \Cref{eq:Complex_Power_Series}, and the other is shown below.
  \begin{equation*}
    \begin{aligned}
      a_{n} &= \frac{1}{n!} \frac{d^{n}}{{dz}^{n}} f(a) \\
      &= \frac{f^{(n)}(a)}{n!}
    \end{aligned}
  \end{equation*}

  By replacing the $a_{n}$ terms, we end up with the general form of a Taylor series for a function $f$.
  \begin{equation}\label{eq:Complex_Taylor_Series}
    f(z) = \sum_{n=0}^{\infty} \frac{f^{(n)}(a)}{n!} {(z-a)}^{n}
  \end{equation}

  \begin{remark}[Maclaurin Series]\label{rmk:Complex_Maclaurin_Series}
    A \emph{Maclaurin series} is a special case of a \nameref{def:Complex_Taylor_Series}, where the function is centered on $a=0$.
  \end{remark}
\end{definition}

%%% Local Variables:
%%% mode: latex
%%% TeX-master: "../../Math_333-MatrixAlg_ComplexVars-Reference_Sheet"
%%% End:


\subsection{Power Series of Rational Functions}\label{subsec:Complex_Power_Series-Rational_Functions}
The key functions to know a \nameref{def:Complex_Power_Series} for are $\frac{1}{1-z}$ and $\frac{1}{1+z}$.
\begin{equation}\label{eq:Complex_Power_Series-1-z}
  \begin{aligned}
    \frac{1}{1-z} &= \sum_{n=0}^{\infty} z^{n} \\
    &= 1 + z + z^{2} + z^{3} + \cdots \, , \Modulus{z} < 1
  \end{aligned}
\end{equation}

\begin{equation}\label{eq:Complex_Power_Series-1+z}
  \begin{aligned}
    \frac{1}{1+z} &= \sum_{n=0}^{\infty} z^{n} \\
    &= 1 - z + z^{2} - z^{3} + \cdots \, , \Modulus{z} < 1
  \end{aligned}
\end{equation}

\begin{example}[Lecture 9, Example 4]{Power Series of Rational Function 1}
  Given the rational function $f(z) = \frac{1}{1-2z^{3}}$ is \nameref{def:Analytic}, find its \nameref{def:Complex_Power_Series}?
  \tcblower{}
  We start be determining the \nameref{thm:Region_of_Convergence}.

  If in the standard format, we have $\frac{1}{1-z}$ where $\Modulus{z} < 1$, we can follow that format.
  \begin{align*}
    \Modulus{2z^{3}} &< 1 \\
    \shortintertext{Property of Modulus, Constants can Factor out.}
    2 \Modulus{z^{3}} &< 1 \\
    \shortintertext{Use the Law of Moduli.}
    {\Modulus{z}}^{3} &< \frac{1}{2} \\
    \Modulus{z} &< \frac{1}{\sqrt[3]{2}}
  \end{align*}

  Now that we have our \nameref{thm:Region_of_Convergence}, which is a circle with center $a=0$ and $r = \frac{1}{\sqrt[3]{2}}$, there \textbf{MUST} be a \nameref{def:Complex_Power_Series} of $f(z)$.
  \begin{align*}
    \Modulus{z} &\Rightarrow \sum_{n=0}^{\infty} z^{n} \\
    \Modulus{2z^{3}} &\Rightarrow \sum_{n=0}^{\infty} {(2z^{3})}^{n} \\
                &= \sum_{n=0}^{\infty} 2^{n} z^{3n}
  \end{align*}
\end{example}

\begin{example}[Lecture 9, Example 4]{Power Series of Rational Function 2}
  Consider $z \to \frac{1}{2+3z}$ as a sum of \nameref{def:Complex_Power_Series} of a circle with center $z = 2 + 5i$.
  Find the power series of this function?
  \tcblower{}
  The function is \nameref{def:Analytic} everywhere exception $z = \frac{-3}{2}$.
  Thus, the \nameref{thm:Region_of_Convergence} is the distance from the center $a$ to the hole $z = \frac{-3}{2}$.

  We start by taking our function, and plugging the left-hand side of the circle, the modulus portion, into this function for $z$.
  \begin{align*}
    \frac{1}{2 + 3z} &= \frac{1}{2 + 3 \bigl( z - (2+5i) \bigr) + 3 (2+5i)} \\
    \intertext{The extra $3(2+5i)$ was needed be cancel out the added $-3(2+5i)$.
    This way, we are only adding by 0, which is allowed by algebra.}
                     &= \frac{1}{8 + 15i + 3 \bigl( z - (2+5i) \bigr)} \\
    \intertext{Now, we force our constants to match our standard equation, \Cref{eq:Complex_Power_Series-1+z}, by factoring out the constants.}
                     &= \frac{1}{(8 + 15i) \left( 1 + \frac{3}{8+15i} \bigl( z - (2+5i) \bigr) \right)} \\
    \intertext{To make our lives easier, we will create a new variable that will represent the variable expression in the denominator.}
    w &= \frac{3 \bigl( z - (2+5i) \bigr)}{8+15i} \\
    \frac{1}{2 + 3z} &= \frac{1}{(8+15i) (1+w)} \\
                     &= \frac{1}{8+15i} \left( \frac{1}{1+w} \right) \\
    \intertext{Now, we can use \Cref{eq:Complex_Power_Series-1+z} as our template.}
    &= \frac{1}{8+15i} \sum_{n=0}^{\infty} {(-1)}^{n} w^{n}, \: \Modulus{w} < 1
  \end{align*}

  Now, we have 2 derivations left to perform:
  \begin{enumerate}[noitemsep]
  \item We need to substitute the series back into $z$, and simplify it.
  \item We need to solve for the \nameref{thm:Region_of_Convergence} by substituting our $z$ expression back in for $w$.
  \end{enumerate}

  I will start with the \nameref{def:Complex_Power_Series}, and then complete the \nameref{thm:Region_of_Convergence}.
  \begin{align*}
    \frac{1}{2 + 3z} &= \frac{1}{8+15i} \sum_{n=0}^{\infty} {(-1)}^{n} w^{n} \\
                     &= \frac{1}{8+15i} \sum_{n=0}^{\infty} {(-1)}^{n} {\left( \frac{3 \bigl( z - (2+5i) \bigr)}{8+15i} \right)}^{n} \\
    \shortintertext{Pull the exponent into all the sub-terms.}
                     &= \frac{1}{8+15i} \sum_{n=0}^{\infty} {(-1)}^{n} \left( \frac{3^{n} {\bigl( z - (2+5i) \bigr)}^{n}}{{(8+15i)}^{n}} \right) \\
                     &= \frac{1}{8+15i} \sum_{n=0}^{\infty} \left( \frac{{(-1)}^{n} 3^{n} }{{(8+15i)}^{n}} \right) {\bigl( z - (2+5i) \bigr)}^{n} \\
                     &= \frac{1}{8+15i} \sum_{n=0}^{\infty} {\left( \frac{(-1)\, 3}{(8+15i)} \right)}^{n} {\bigl( z - (2+5i) \bigr)}^{n} \\
  \end{align*}

  With the \nameref{def:Complex_Power_Series} found, we now focus on the \nameref{thm:Region_of_Convergence}.
  \begin{align*}
    \Modulus{w} &< 1 \\
    \Modulus{\frac{3 \bigl( z - (2+5i) \bigr)}{8+15i}} &< 1 \\
    \shortintertext{Factor out the constants.}
    \Modulus{\frac{3}{8+15i}} \Modulus{z - (2+5i)} &< 1 \\
    \Modulus{z - (2+5i)} &< \Modulus{\frac{8+15i}{3}} \\
    \shortintertext{Use the Law of Moduli to separate the moduli in the fraction.}
    \Modulus{z - (2+5i)} &< \frac{\Modulus{8+15i}}{\Modulus{3}} \\
                &< \frac{\sqrt{64 + 225}}{3} \\
    &< \frac{\sqrt{289}}{3}
  \end{align*}

  Thus, our solution is:
  \begin{equation*}
    \frac{1}{2 + 3z} = \frac{1}{8+15i} \sum_{n=0}^{\infty} {\left( \frac{(-1)\, 3}{(8+15i)} \right)}^{n} {\bigl( z - (2+5i) \bigr)}^{n} , \: \Modulus{z - (2+5i)} < \frac{\sqrt{289}}{3}
  \end{equation*}
\end{example}


%%% Local Variables:
%%% mode: latex
%%% TeX-master: "../Math_333-MatrixAlg_ComplexVars-Reference_Sheet"
%%% End:
