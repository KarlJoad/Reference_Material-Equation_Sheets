\subsection{Diagonalization}\label{subsec:Complex_Matrix_Diagonalization}
\nameref{def:Diagonalization} of a \nameref{def:Complex_Matrix} behaves much the same way as the real-valued case.

For a \nameref{def:Complex_Matrix} $A$, a matrix that implements a \nameref{def:Diagonalization} is $P$, and this yields a completely real-valued matrix $D$.
This means:
\begin{equation*}
  P^{-1} A P = D
\end{equation*}

\begin{remark*}
  If $P$ is a \nameref{def:Hermitian_Matrix}, then $P^{-1} = \MatrixConjTrans{P}$.
\end{remark*}

%%% Local Variables:
%%% mode: latex
%%% TeX-master: "../../Math_333-MatrixAlg_ComplexVars-Reference_Sheet"
%%% End:
