\subsection{Hermitian Matrices}\label{subsec:Hermitian_Matrices}
\begin{definition}[Hermitian Matrix]\label{def:Hermitian_Matrix}
  A \emph{Hermitian matrix} is a complex square \nameref{def:Matrix} that is equal to its own \nameref{def:Conjugate_Transpose}.
  This makes Hermitian matrices the complex-valued complement to a \nameref{def:Real_Symmetric_Matrix}.
  Symbolically,

  \begin{equation}\label{eq:Hermitian_Matrix}
    \begin{aligned}
      \MatrixConjTrans{A} &= A^{-1}
    \end{aligned}
  \end{equation}

  For this to occur, the main diagonal must have \textbf{only} real values.
  The off-diagonals must be \nameref{def:Complex_Conjugate}s of each other.

  Hermitian matrices share the same 2 properties as real, symmetric matrices.
  \begin{propertylist}
  \item \nameref{def:Eigenvalue}s are entirely real-valued ($\RealNumbers$).\label{prop:Hermitian_Matrix-Real_Eigenvalues}
  \item \nameref{def:Eigenvector}s that correspond to \textbf{distinct} \nameref{def:Eigenvalue}s are orthogonal.\label{prop:Hermitian_Matrix-Orthogonal_Eigenvectors}
  \end{propertylist}
\end{definition}

Some examples of a \nameref{def:Hermitian_Matrix} are:
\begin{align*}
  A_{2 \by 2} &=
                \begin{bmatrix}
                  5 & 1 + 2i \\
                  1 - 2i & 6
                \end{bmatrix} \\
  B_{3 \by 3} &=
                \begin{bmatrix}
                  1 & 1 + 5i & -6i \\
                  1 - 5i & 2 & 1 + i \\
                  6i & 1 - i & 3
                \end{bmatrix}
\end{align*}

\begin{definition}[Conjugate Tranpose]\label{def:Conjugate_Transpose}
  The \emph{conjugate tranpose} of a \nameref{def:Complex_Matrix} is two matrix operations in succession.
  First, the element-wise conjugate is found, then the tranpose is taken of the whole matrix.

  \begin{equation}\label{eq:Conjugate_Tranpose}
    \MatrixConjTrans{B} = \Transpose{(\Conjugate{B})}
  \end{equation}
\end{definition}

\begin{example}[Lecture 22, Example 1]{Find Unitary Matrix Implementing Diagonalization}
  Find a \nameref{def:Unitary_Matrix} that implements a \nameref{def:Diagonalization} for the \nameref{def:Complex_Matrix} $A$?
  \begin{equation*}
    A =
    \begin{bmatrix}
      -1 & 2i \\
      -2i & 2
    \end{bmatrix}
  \end{equation*}
  \tcblower{}
  First, we start by finding the \nameref{def:Eigenvalue}s of $A$.
  We notice that $A$ is a \nameref{def:Hermitian_Matrix}, which means that there will be only real-valued eigenvalues.
  \begin{align*}
    A - \lambda I &=
                    \begin{bmatrix}
                      -1 - \lambda & 2i \\
                      -2i & 2 - \lambda
                    \end{bmatrix} \\
    \det(A - \lambda I) &= (-1 - \lambda) (2 - \lambda) - (2i) (-2i) \\
                  &= \lambda^{2} - \lambda - 2 - 4 \\
                  &= \lambda^{2} - \lambda - 6 \\
                  &= (\lambda - 3) (\lambda + 2) \\
  \end{align*}

  We get \nameref{def:Eigenvalue}s if and only if $\det(A - \lambda I) = 0$.
  So:
  \begin{description}[noitemsep]
  \item[$\lambda = 3$] Algebraic multiplicity 1.
  \item[$\lambda = -2$] Algebraic multiplicity 1.
  \end{description}

  Now, we find the \nameref{def:Eigenvector} that corresponds to the \nameref{def:Eigenvalue} $\lambda = 3$.
  \begin{align*}
    \begin{bmatrix}
      4 & 2i \\
      -2i & -1
    \end{bmatrix}
            \begin{bmatrix}
              x \\ y
            \end{bmatrix} &=
                            \begin{bmatrix}
                              0 \\ 0
                            \end{bmatrix} \\
    \intertext{Convert to a system of complex-valued linear equations.}
    -4x + 2iy &= 0 \\
    -2ix - y &= 0 \\
    \intertext{Solving the system.}
    y &= \frac{2x}{i} \\
    x &= x
  \end{align*}

  Plugging that all back in:
  \begin{align*}
    \begin{bmatrix}
      x \\ y
    \end{bmatrix} &=
                    \begin{bmatrix}
                      x \\ \frac{2x}{i}
                    \end{bmatrix} \\
    &= x
      \begin{bmatrix}
        1 \\ \frac{2}{i}
      \end{bmatrix} \\
    &= x
      \begin{bmatrix}
        1 \\ -2i
      \end{bmatrix}, \, x \neq 0
  \end{align*}

  Thus, the \nameref{def:Eigenvector} that corresponds to the \nameref{def:Eigenvalue} $\lambda = 3$ is:
  \begin{equation*}
    x
    \begin{bmatrix}
      1 \\ -2i
    \end{bmatrix}, \, x \neq 0
  \end{equation*}
  With geometric multiplicity 1.

  Now, we find the \nameref{def:Eigenvector} that corresponds to the \nameref{def:Eigenvalue} $\lambda = -2$.
  \begin{align*}
    \begin{bmatrix}
      1 & 2i \\
      -2i & 4
    \end{bmatrix}
            \begin{bmatrix}
              x \\ y
            \end{bmatrix} &=
                            \begin{bmatrix}
                              0 \\ 0
                            \end{bmatrix} \\
    \intertext{Convert to a system of complex-valued linear equations.}
    x + 2iy &= 0 \\
    -2ix + 4y &= 0 \\
    \intertext{Solving the system.}
    x &= -2iy \\
    y &= y
  \end{align*}

  Plugging that all back in:
  \begin{align*}
    \begin{bmatrix}
      x \\ y
    \end{bmatrix} &=
                    \begin{bmatrix}
                      -2iy \\ y
                    \end{bmatrix} \\
    &= y
      \begin{bmatrix}
        -2i \\ 1
      \end{bmatrix}, \, y \neq 0
  \end{align*}
  Geometric multiplicity 1.

  Because the given matrix was Hermitian, we know that these \nameref{def:Eigenvector}s are orthogonal.
  However, we will verify that property, then normalize them.
  \begin{align*}
    \InnerProd{
    \begin{bmatrix}
      1 \\ -2i
    \end{bmatrix}}{
    \begin{bmatrix}
      -2i \\ 1
    \end{bmatrix}} &= 1 (\Conjugate{-2i}) + -2i (\Conjugate{1}) \\
                   &= 2i + -2i \\
                   &= 0
  \end{align*}

  Now, normalize the \nameref{def:Eigenvector}s.
  \begin{align*}
    \Magnitude{v_{1}} &= \InnerProd{v_{1}}{v_{2}} & \Magnitude{v_{2}} &= \InnerProd{v_{2}}{v_{2}} \\
                      &= \sqrt{\InnerProd{
                        \begin{bmatrix}
                          1 \\ -2i
                        \end{bmatrix}}{
    \begin{bmatrix}
      1 \\ -2i
    \end{bmatrix}}} & &= \sqrt{\InnerProd{
                        \begin{bmatrix}
                          -2i \\ 1
                        \end{bmatrix}}{
    \begin{bmatrix}
      -2i \\ 1
    \end{bmatrix}}} \\
                      &= \sqrt{1+4} & &= \sqrt{4+1} \\
                      &= \sqrt{5} & &= \sqrt{5}
  \end{align*}

  Thus, the \nameref{def:Unitary_Matrix} that implements a \nameref{def:Diagonalization} of $A$ is $P$.
  \begin{align*}
    P &=
        \begin{bmatrix}
          \frac{1}{\sqrt{5}} & \frac{-2i}{\sqrt{5}} \\
          \frac{-2i}{\sqrt{5}} & \frac{1}{\sqrt{5}}
        \end{bmatrix} \\
    &= \frac{1}{\sqrt{5}}
      \begin{bmatrix}
        1 & -2i \\
        -2i & 1
      \end{bmatrix}
  \end{align*}
\end{example}

%%% Local Variables:
%%% mode: latex
%%% TeX-master: "../../Math_333-MatrixAlg_ComplexVars-Reference_Sheet"
%%% End:
