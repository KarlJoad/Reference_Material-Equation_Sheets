\subsection{Hermitian Matrices}\label{subsec:Hermitian_Matrices}
\begin{definition}[Hermitian Matrix]\label{def:Hermitian_Matrix}
  A \emph{Hermitian matrix} is a complex square \nameref{def:Matrix} that is equal to its own \nameref{def:Conjugate_Transpose}.
  This makes Hermitian matrices the complex-valued complement to a \nameref{def:Real_Symmetric_Matrix}.
  Symbolically,

  \begin{equation}\label{eq:Hermitian_Matrix}
    \begin{aligned}
      \MatrixConjTrans{A} &= A^{-1}
    \end{aligned}
  \end{equation}

  For this to occur, the main diagonal must have \textbf{only} real values.
  The off-diagonals must be \nameref{def:Complex_Conjugate}s of each other.

  Hermitian matrices share the same 2 properties as real, symmetric matrices.
  \begin{propertylist}
  \item \nameref{def:Eigenvalue}s are entirely real-valued ($\RealNumbers$).\label{prop:Hermitian_Matrix-Real_Eigenvalues}
  \item \nameref{def:Eigenvector}s that correspond to \textbf{distinct} \nameref{def:Eigenvalue}s are orthogonal.\label{prop:Hermitian_Matrix-Orthogonal_Eigenvectors}
  \end{propertylist}
\end{definition}

Some examples of a \nameref{def:Hermitian_Matrix} are:
\begin{align*}
  A_{2 \by 2} &=
                \begin{bmatrix}
                  5 & 1 + 2i \\
                  1 - 2i & 6
                \end{bmatrix} \\
  B_{3 \by 3} &=
                \begin{bmatrix}
                  1 & 1 + 5i & -6i \\
                  1 - 5i & 2 & 1 + i \\
                  6i & 1 - i & 3
                \end{bmatrix}
\end{align*}


%%% Local Variables:
%%% mode: latex
%%% TeX-master: "../../Math_333-MatrixAlg_ComplexVars-Reference_Sheet"
%%% End:
