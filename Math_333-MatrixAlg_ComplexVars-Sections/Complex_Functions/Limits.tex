\subsection{Limits}\label{subsec:Limits}
Like in earlier maths, sometimes we are interested not in the exact point where a function has a value; instead, we are interested in how the function behaves as we approach that point.
If the function is a \nameref{def:Complex_Function}, i.e.\ ($\ComplexNumbers \mapsto \ComplexNumbers$), then we need to change our definition of a limit from the one we are familiar with to the one in \Cref{def:Complex_Limit}.

\begin{definition}[Limit]\label{def:Complex_Limit}
  The \emph{limit} of a \nameref{def:Complex_Function} behaves quite similarly to their purely-real counterparts.
  \begin{equation}\label{eq:Complex_Limit}
    \lim\limits_{\substack{z \to a \\ z \neq a}} f(z) = \ell
  \end{equation}
\end{definition}

To solve a problem involving a \nameref{def:Complex_Function} and a limit, you can perform the same steps as before.

\begin{example}[Lecture 5]{Solve Limit of Complex Function}
  If $f(z) = z^{2}$, then find the solution to this \nameref{def:Complex_Limit} of a \nameref{def:Complex_Function}?
  \begin{equation*}
    \lim\limits_{z \to 3i} f(z)
  \end{equation*}
  \tcblower{}
  \begin{align*}
    \lim\limits_{z \to 3i} f(z) &= \lim\limits_{z \to 3i} z^{2} \\
                                &= {(3i)}^{2} \\
                                &= 9 i^{2} = 9 (-1) \\
                                &= -9
  \end{align*}

  Thus, $\lim\limits_{z \to 3i} f(z) = -9$.
\end{example}

\begin{remark*}
  However, do note that the act of finding the \nameref{def:Complex_Limit} of a function approaching a point is distinctly different than finding the value of the function \textbf{at} that point.
\end{remark*}

In addition to the properties and rules that traditional real-valued limits have, \nameref{def:Complex_Limit}s of \nameref{def:Complex_Function}s have additional properties, because they lie on a 2-dimensional plane.
This means that there are \textit{infinitely} many approachable directions to a point.

Thus, for a \nameref{def:Complex_Limit} of a \nameref{def:Complex_Function} to exist at a point $a$:
\begin{propertylist}
\item \textbf{ALL} path \nameref{def:Complex_Limit}s \textbf{MUST} exist.
\item \textbf{ALL} path \nameref{def:Complex_Limit}s \textbf{MUST} evaluate to the same value.
\end{propertylist}

\subsubsection{Limits of Complex Functions that Do Not Exist}\label{subsubsec:Complex_Limit_DNE}
For a \nameref{def:Complex_Limit} of a \nameref{def:Complex_Function} to \textbf{not} exist, as $z \to a$, then there are at least 2 paths $\Path_{1}, \Path_{2}$ that approach $a$ such that
\begin{equation}\label{eq:Complex_Limit_DNE}
  \lim\limits_{\substack{z \to a \\ z \text{ on } \Path_{1}}} f(z) \neq \lim\limits_{\substack{z \to a \\ z \text{ on } \Path_{2}}} f(z)
\end{equation}

\begin{example}[Lecture 5]{Limit of a Complex Function that DNE}
  Give the function $f(z)$ defined below, show that its \nameref{def:Complex_Limit} as $z \to 0$ Does Not Exist ($\DNE$)?
  \begin{equation*}
    f(z) = \frac{xy}{x^{2} + y^{2}}
  \end{equation*}
  \tcblower{}
  We start by referring to the definition of the lack of existence of a \nameref{def:Complex_Limit} of a \nameref{def:Complex_Function}.
  Namely, we must find at least 2 paths $\Path_{1}, \Path_{2}$ such that evaluating the limit will yield two different values.
  I choose:
  \begin{align*}
    \Path_{1} &\coloneqq y = 0 \\
    \Path_{2} &\coloneqq y = 2x
  \end{align*}

  Now we evaluate the expression below, using each path.
  \begin{equation*}
    \lim\limits_{\substack{z \to 0 \\ z \text{ on } \Path}} f(z)
  \end{equation*}

  Following Path 1, $\Path_{1}$:
  \begin{align*}
    \lim\limits_{\substack{z \to 0 \\ z \text{ on } \Path_{1}}} f(z) &= \lim\limits_{\substack{z \to 0 \\ z \text{ on } \Path_{1}}} \frac{xy}{x^{2}+y^{2}} \\
                                                                     &= \lim\limits_{\substack{x \to 0 \\ y = 0}} \frac{xy}{x^{2}+y^{2}} \\
                                                                     &= \lim\limits_{x \to 0} \frac{0}{x^{2}+0} \\
                                                                     &= 0
  \end{align*}

  Following Path 2, $\Path_{2}$:
  \begin{align*}
    \lim\limits_{\substack{z \to 0 \\ z \text{ on } \Path_{2}}} f(z) &= \lim\limits_{\substack{z \to 0 \\ z \text{ on } \Path_{2}}} \frac{xy}{x^{2}+y^{2}} \\
                                                                     &= \lim\limits_{\substack{x \to 0 \\ y = 2x}} \frac{xy}{x^{2}+y^{2}} \\
                                                                     &= \lim\limits_{x \to 0} \frac{2x^{2}}{5x^{2}} \\
                                                                     &= \frac{2}{5}
  \end{align*}

  Thus,
  \begin{align*}
    \lim\limits_{\substack{z \to 0 \\ z \text{ on } \Path_{1}}} f(z) &= 0 \\
    \lim\limits_{\substack{z \to 0 \\ z \text{ on } \Path_{2}}} f(z) &= \frac{2}{5} \\
    \lim\limits_{\substack{z \to 0 \\ z \text{ on } \Path_{1}}} f(z) &\neq \lim\limits_{\substack{z \to 0 \\ z \text{ on } \Path_{2}}} f(z)
  \end{align*}

  Because there exist two different paths that yield different results from the \nameref{def:Complex_Limit} of the provided function $f(z)$,
  \begin{equation*}
    \lim\limits_{z \to 0} f(z) = \DNE
  \end{equation*}
\end{example}


%%% Local Variables:
%%% mode: latex
%%% TeX-master: "../../Math_333-MatrixAlg_ComplexVars-Reference_Sheet"
%%% End:
