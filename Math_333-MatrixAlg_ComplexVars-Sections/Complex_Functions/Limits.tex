\subsection{Limits}\label{subsec:Limits}
Like in earlier maths, sometimes we are interested not in the exact point where a function has a value; instead, we are interested in how the function behaves as we approach that point.
If the function is a \nameref{def:Complex_Function}, i.e.\ ($\ComplexNumbers \mapsto \ComplexNumbers$), then we need to change our definition of a limit from the one we are familiar with to the one in \Cref{def:Complex_Limit}.

\begin{definition}[Limit]\label{def:Complex_Limit}
  The \emph{limit} of a \nameref{def:Complex_Function} behaves quite similarly to their purely-real counterparts.
  \begin{equation}\label{eq:Complex_Limit}
    \lim\limits_{\substack{z \to a \\ z \neq a}} f(z) = \ell
  \end{equation}
\end{definition}

To solve a problem involving a \nameref{def:Complex_Function} and a limit, you can perform the same steps as before.

\begin{example}[Lecture 5]{Solve Limit of Complex Function}
  If $f(z) = z^{2}$, then find the solution to this \nameref{def:Complex_Limit} of a \nameref{def:Complex_Function}?
  \begin{equation*}
    \lim\limits_{z \to 3i} f(z)
  \end{equation*}
  \tcblower{}
  \begin{align*}
    \lim\limits_{z \to 3i} f(z) &= \lim\limits_{z \to 3i} z^{2} \\
                                &= {(3i)}^{2} \\
                                &= 9 i^{2} = 9 (-1) \\
                                &= -9
  \end{align*}

  Thus, $\lim\limits_{z \to 3i} f(z) = -9$.
\end{example}


%%% Local Variables:
%%% mode: latex
%%% TeX-master: "../../Math_333-MatrixAlg_ComplexVars-Reference_Sheet"
%%% End:
