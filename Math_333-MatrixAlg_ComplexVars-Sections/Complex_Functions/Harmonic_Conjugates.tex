\subsection{Harmonic Conjugates}\label{subsec:Harmonic_Conjugates}
Now that we have shown and proven that a function can be \nameref{def:Harmonic} in \Cref{ex:Prove Function Terms are Harmonic}, we are curious to see that if one function is known, can we find the other one.

\begin{definition}[Simply Connected]\label{def:Simply_Connected}
  A \emph{simply} \nameref{def:Connected} set is a connected set $\Omega$ that does not have any holes in it.

  \begin{remark}[Multiply-Connected]\label{rmk:Multiply_Connected}
    If there is one hole, the region is \emph{doubly}-connected. \\
    If there are two holes, the region is \emph{triply}-connected.
  \end{remark}
\end{definition}

\begin{theorem}[Harmonic Conjugate]\label{thm:Harmonic_Conjugate}
  Let $\Omega$ be a \nameref{def:Simply_Connected}, \nameref{def:Connected}, and \nameref{def:Open} set.
  Let $U: \Omega \to \RealNumbers$, which is \nameref{def:Harmonic}.
  Then, there exists a function $V$, which is also harmonic, such that $f = U + iV$ is \nameref{def:Analytic}.
  Such a $V(x, y)$ would be called the harmonic conjugate of $U(x, y)$.
\end{theorem}

\begin{example}[Lecture 6, Example 1]{Verify Harmonicity and Find Harmonic Conjugates}
  Given $U(x, y) = x^{4} - 6x^{2}y^{2} + y^{4}$ is defined on the $\ComplexNumbers$ plane, verify that $U(x, y)$ is \nameref{def:Harmonic}?
  Find \textbf{all} \nameref{thm:Harmonic_Conjugate}s of $U(x, y)$?
  \tcblower{}
  First, we can verify that a function is \nameref{def:Harmonic} by checking that it satisfies \nameref{subsec:Laplaces_Equation}, (\Cref{eq:Laplaces_Equation}).
  \begin{equation*}
    \frac{\partial^{2} U}{{\partial x}^{2}} + \frac{\partial^{2} U}{{\partial y}^{2}} = 0
  \end{equation*}

  So, we start by taking the partial derivatives of $U(x, y)$.
  \begin{align*}
    \frac{\partial U}{\partial x} &= 4x^{3} - 12xy^{2} & \frac{\partial U}{\partial y} &= -12x^{2}y + 4y^{3} \\
    \frac{\partial^{2} U}{{\partial x}^{2}} &= 12x^{2} - 12y^{2} & \frac{\partial^{2} U}{{\partial y}^{2}} &= -12x^{2} + 12y^{2} \\
  \end{align*}

  Now, plugging these into \Cref{eq:Laplaces_Equation}:
  \begin{align*}
    0 &= \frac{\partial^{2} U}{{\partial x}^{2}} + \frac{\partial^{2} U}{{\partial y}^{2}} \\
      &= 12x^{2} - 12y^{2} + \left( -12x^{2} + 12y^{2} \right) \\
      &= \left( 12x^{2} - 12x^{2} \right) + \left( 12y^{2} - 12y^{2} \right) \\
      &\overset{\checkmark}{=} 0
  \end{align*}

  $\therefore U(x, y)$ satisfies \nameref{subsec:Laplaces_Equation}, and thus, is \nameref{def:Harmonic}. \\

  Now, we want $f = U + iV$ to be \nameref{def:Analytic}, so what is $V$?
  For $f$ to be analytic, the \nameref{subsubsec:Cauchy_Riemann_Equations} \textbf{must} hold.
  \begin{align*}
    \frac{\partial U}{\partial x} &= \frac{\partial V}{\partial y} \\
    \frac{\partial U}{\partial y} &= - \frac{\partial V}{\partial x} \\
    4x^{3} - 12xy^{2} &= V_{y} \\
    12x^{2}y - 4y^{3} &= V_{x}
  \end{align*}

  Now, we want to find just $V$, so we integrate $V_{y}$ with respect to $y$.
  \begin{align*}
    V(x, y) &= \int V_{y} \partial y \\
            &= \int 4x^{3} - 12xy^{2} \partial y \\
            &= 4x^{3}y - 4xy^{3} + C(x) \\
  \end{align*}

  Now, we need to solve for $C(x)$, a potential constant within the $x$ domain.
  \begin{align*}
    \frac{\partial}{\partial x} V(x, y) &= 12x^{2}y - 4y^{3} + \frac{dC}{dx} \\
    \intertext{Using the value for $V_{x}$ we found earlier, we can set each of these equal to each other.}
    12x^{2}y - 4y^{3} &= 12x^{2}y - 4y^{3} + \frac{dC}{dx} \\
    \frac{dC}{dx} &= 0
  \end{align*}

  Because we said $C(x)$ was a function solely in $x$, if $\frac{dC}{dx} = 0$, then $C(x)$ \textbf{MUST} be equal to only a constant.
  Thus, $C(x) = C$.
  So, all possible \nameref{thm:Harmonic_Conjugate}s of $U(x, y)$ are represented by a complete $V(x, y)$, shown below.
  \begin{equation*}
    V(x, y) = 4x^{3}y - 4xy^{3} + C
  \end{equation*}

  Therefore, $f$, according to our definition, is:
  \begin{align*}
    f &= U(x, y) + iV(x, y) \\
      &= x^{4} - 6x^{2}y^{2} + y^{2} + i \left( 4x^{3}y - 4xy^{3} + C \right) \\
      &= x^{4} + 4x^{3}yi - 6x^{2}y^{2} - 4xy^{3}i + y^{4} \\
      &= {(x + iy)}^{4} + iC
  \end{align*}
\end{example}

%%% Local Variables:
%%% mode: latex
%%% TeX-master: "../../Math_333-MatrixAlg_ComplexVars-Reference_Sheet"
%%% End:
