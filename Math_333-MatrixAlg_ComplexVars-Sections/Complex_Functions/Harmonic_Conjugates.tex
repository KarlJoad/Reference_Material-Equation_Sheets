\subsection{Harmonic Conjugates}\label{subsec:Harmonic_Conjugates}
Now that we have shown and proven that a function can be \nameref{def:Harmonic} in \Cref{ex:Prove Function Terms are Harmonic}, we are curious to see that if one function is known, can we find the other one.

\begin{definition}[Simply]\label{def:Simply}
  A \emph{simply} \nameref{def:Connected} set is a connected set $\Omega$ that does not have any holes in it.

  \begin{remark}[Multiply-Connected]\label{rmk:Multiply_Connected}
    If there is one hole, the region is \emph{doubly}-connected. \\
    If there are two holes, the region is \emph{triply}-connected.
  \end{remark}
\end{definition}

\begin{theorem}[Harmonic Conjugate]\label{thm:Harmonic_Conjugate}
  Let $\Omega$ be a \nameref{def:Simply} \nameref{def:Connected} \nameref{def:Open} set.
  Let $U: \Omega \to \RealNumbers$, which is \nameref{def:Harmonic}.
  Then, there exists a function $V$, which is also harmonic, such that $f = U + iV$ is \nameref{def:Analytic}.
  Such a $V(x, y)$ would be called the harmonic conjugate of $U(x, y)$.
\end{theorem}


%%% Local Variables:
%%% mode: latex
%%% TeX-master: "../../Math_333-MatrixAlg_ComplexVars-Reference_Sheet"
%%% End:
