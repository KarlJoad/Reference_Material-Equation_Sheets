\subsection{Derivatives}\label{subsec:Derivatives}
\begin{definition}[Derivative]\label{def:Complex_Derivative}
  The \emph{derivative} of a \nameref{def:Complex_Function}, like many other operations, must have its definitions slightly redefined to account for the extra dimension provided by the \nameref{def:Argand_Plane}.

  The \textbf{domain} of a derivative of a \nameref{def:Complex_Function} $f(z)$ requires that $f$ be defined in a \nameref{def:Neighborhood} of the point $a$.
  Once this is satisified, then \Cref{eq:Complex_Derivative} is used to find the actual value.
  \begin{equation}\label{eq:Complex_Derivative}
    f'(a) = \lim\limits_{z \to a} \frac{f(z) - f(a)}{z - a}
  \end{equation}

  \begin{remark}[Path Existence]\label{rmk:Complex_Derivative_Path_Existence}
    Because the \nameref{def:Complex_Derivative} of a \nameref{def:Complex_Function} is defined with a \nameref{def:Complex_Limit}, \textbf{ALL} paths must exist and have the same value.
  \end{remark}

  \begin{remark}[Uses]\label{rmk:Complex_Derivative_Uses}
    The definition of a \nameref{def:Complex_Derivative} is rarely used to calculate anything.
    Instead, it is used to prove the non-existence of a derivative of a complex function.
  \end{remark}
\end{definition}

\begin{definition}[Neighborhood]\label{def:Neighborhood}
  A \emph{neighborhood} is an open disk $\Modulus{z-a} < r$ for some $r$.
  The radius $r$ is unspecified, meaning that we must choose its value.
\end{definition}

\begin{example}[Lecture 5]{Derivative of Complex Function}
  Given $f(z)$, what is its \nameref{def:Complex_Derivative} at a point $a$, $f'(a)$?
  \begin{equation*}
    f(z) = z^{2}
  \end{equation*}
  \tcblower{}
  As this problem is simple, we can just apply \Cref{eq:Complex_Derivative} directly.
  \begin{align*}
    f'(a) &= \lim\limits_{z \to a} \frac{f(z) - f(a)}{z - a} \\
          &= \lim\limits_{z \to a} \frac{z^{2} - a^{2}}{z - a} \\
          &= \lim\limits_{z \to a} \frac{(z-a) (z+a)}{z - a} \\
          &= \lim\limits_{z \to a} (z+a) \\
          &= (a + a) \\
          &= 2a
  \end{align*}

  Thus, $f'(a) = 2a$.
\end{example}

\begin{example}[Lecture 5]{Non-Existence of Derivative}
  Given $f(z)$, $a = 2i$, show that $f'(2i)$ does not exist?
  \begin{equation*}
    f(z) = \Conjugate{z}
  \end{equation*}
  \tcblower{}
  We start by applying the definition of a \nameref{def:Complex_Derivative} of a \nameref{def:Complex_Function}.
  \begin{align*}
    f'(a) &= \lim\limits_{z \to 2i} \frac{f(z) - f(a)}{z - a} \\
          &= \lim\limits_{z \to 2i} \frac{\Conjugate{z} - \Conjugate{a}}{z - a} \\
    f'(2i) &= \lim\limits_{z \to 2i} \frac{\Conjugate{z} - (-2i)}{z - a} \\w
          &= \lim\limits_{z \to 2i} \frac{\Conjugate{z} + 2i}{z - 2i} \\
  \end{align*}

  Now, we can fall back to the definition of the non-existence of a \nameref{def:Complex_Limit} of a \nameref{def:Complex_Function}.
  If the value of the function approaching the point $a$ by two different paths ($\Path_{1}, \Path_{2}$) have two different values, the limit does not exist.
  If we choose our paths to be:
  \begin{align*}
    \Path_{1} &\coloneqq x = 0 \\
    \Path_{2} &\coloneqq y = x + 2i
  \end{align*}

  Now solving the limit above using Path 1 ($\Path_{1}$):
  \begin{align*}
    f'(2i) &= \lim\limits_{\substack{z \to 2i \\ z \text{ on } \Path_{1}}} \frac{\Conjugate{z} + 2i}{z - 2i} \\
           &= \lim\limits_{\substack{x = 0 \\ y \to 2}} \frac{-yi + 2i}{yi - 2i} \\
           &= \lim\limits_{y \to 2} \frac{-i (y-2)}{i (y-2)} \\
           &= -1
  \end{align*}

  Now solve the above limit using Path 2 ($\Path_{2}$):
  \begin{align*}
    f'(2i) &= \lim\limits_{\substack{z \to 2i \\ z \text{ on } \Path_{2}}} \frac{\Conjugate{z} + 2i}{z - 2i} \\
           &= \lim\limits_{\substack{x \to 0 \\ y = 2}} \frac{x-2i+2i}{x+2i-2i} \\
           &= \lim\limits_{\substack{x \to 0}} \frac{x}{x} \\
           &= 1
  \end{align*}

  Now, like the definition of a \nameref{def:Complex_Limit} of a \nameref{def:Complex_Function} states:
  \begin{align*}
    \lim\limits_{\substack{z \to a \\ z \text{ on } \Path_{1}}} \frac{\Conjugate{z} + 2i}{z - 2i} &\neq \lim\limits_{\substack{z \to a \\ z \text{ on } \Path_{2}}} \frac{\Conjugate{z} + 2i}{z - 2i} \\
    \lim\limits_{z \to a} \frac{f(z) - f(a)}{z - a} &= \DNE
  \end{align*}

  Thus, because the backing \nameref{def:Complex_Limit} does not exist, the entire \nameref{def:Complex_Derivative} does not exist.
\end{example}

\subsubsection{Nicities}\label{subsubsec:Complex_Derivative_Nicities}
\nameref{def:Complex_Derivative}s of \nameref{def:Complex_Function}s obey the same rules as purely-real calculus.
This includes:
\begin{itemize}[noitemsep]
\item Product Rule
\item Quotient Rule
\item Chain Rule
\item etc.
\end{itemize}

\subsubsection{Cauchy-Riemann Equations}\label{subsubsec:Cauchy_Riemann_Equations}
Because we know the \nameref{def:Complex_Derivative} of a \nameref{def:Complex_Function} \textbf{can} exist, and that $f(z) = U(x, y) + i V(x, y)$, we need to know just how special $U(x, y)$ and $V(x, y)$ really are.

%%% Local Variables:
%%% mode: latex
%%% TeX-master: "../../Math_333-MatrixAlg_ComplexVars-Reference_Sheet"
%%% End:
