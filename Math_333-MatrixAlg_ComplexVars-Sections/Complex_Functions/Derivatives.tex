\subsection{Derivatives}\label{subsec:Derivatives}
\begin{definition}[Derivative]\label{def:Complex_Derivative}
  The \emph{derivative} of a \nameref{def:Complex_Function}, like many other operations, must have its definitions slightly redefined to account for the extra dimension provided by the \nameref{def:Argand_Plane}.

  The \textbf{domain} of a derivative of a \nameref{def:Complex_Function} $f(z)$ requires that $f$ be defined in a \nameref{def:Neighborhood} of the point $a$.
  Once this is satisified, then \Cref{eq:Complex_Derivative} is used to find the actual value.
  \begin{equation}\label{eq:Complex_Derivative}
    f'(a) = \lim\limits_{z \to a} \frac{f(z) - f(a)}{z - a}
  \end{equation}

  \begin{remark}[Path Existence]\label{rmk:Complex_Derivative_Path_Existence}
    Because the \nameref{def:Complex_Derivative} of a \nameref{def:Complex_Function} is defined with a \nameref{def:Complex_Limit}, \textbf{ALL} paths must exist and have the same value.
  \end{remark}

  \begin{remark}[Uses]\label{rmk:Complex_Derivative_Uses}
    The definition of a \nameref{def:Complex_Derivative} is rarely used to calculate anything.
    Instead, it is used to prove the non-existence of a derivative of a complex function.
  \end{remark}
\end{definition}

\begin{definition}[Neighborhood]\label{def:Neighborhood}
  A \emph{neighborhood} is an open disk $\Modulus{z-a} < r$ for some $r$.
  The radius $r$ is unspecified, meaning that we must choose its value.
\end{definition}

\begin{example}[Lecture 5]{Derivative of Complex Function}
  Given $f(z)$, what is its \nameref{def:Complex_Derivative} at a point $a$, $f'(a)$?
  \begin{equation*}
    f(z) = z^{2}
  \end{equation*}
  \tcblower{}
  As this problem is simple, we can just apply \Cref{eq:Complex_Derivative} directly.
  \begin{align*}
    f'(a) &= \lim\limits_{z \to a} \frac{f(z) - f(a)}{z - a} \\
          &= \lim\limits_{z \to a} \frac{z^{2} - a^{2}}{z - a} \\
          &= \lim\limits_{z \to a} \frac{(z-a) (z+a)}{z - a} \\
          &= \lim\limits_{z \to a} (z+a) \\
          &= (a + a) \\
          &= 2a
  \end{align*}

  Thus, $f'(a) = 2a$.
\end{example}

%%% Local Variables:
%%% mode: latex
%%% TeX-master: "../../Math_333-MatrixAlg_ComplexVars-Reference_Sheet"
%%% End:
