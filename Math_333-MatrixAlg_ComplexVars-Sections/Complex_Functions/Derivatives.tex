\subsection{Derivatives}\label{subsec:Derivatives}
\begin{definition}[Derivative]\label{def:Complex_Derivative}
  The \emph{derivative} of a \nameref{def:Complex_Function}, like many other operations, must have its definitions slightly redefined to account for the extra dimension provided by the \nameref{def:Argand_Plane}.

  The \textbf{domain} of a derivative of a \nameref{def:Complex_Function} $f(z)$ requires that $f$ be defined in a \nameref{def:Neighborhood} of the point $a$.
  Once this is satisified, then \Cref{eq:Complex_Derivative} is used to find the actual value.
  \begin{equation}\label{eq:Complex_Derivative}
    f'(a) = \lim\limits_{z \to a} \frac{f(z) - f(a)}{z - a}
  \end{equation}

  \begin{remark}[Path Existence]\label{rmk:Complex_Derivative_Path_Existence}
    Because the \nameref{def:Complex_Derivative} of a \nameref{def:Complex_Function} is defined with a \nameref{def:Complex_Limit}, \textbf{ALL} paths must exist and have the same value.
  \end{remark}

  \begin{remark}[Uses]\label{rmk:Complex_Derivative_Uses}
    The definition of a \nameref{def:Complex_Derivative} is rarely used to calculate anything.
    Instead, it is used to prove the non-existence of a derivative of a complex function.
  \end{remark}
\end{definition}

\begin{definition}[Neighborhood]\label{def:Neighborhood}
  A \emph{neighborhood} is an open disk $\Modulus{z-a} < r$ for some $r$.
  The radius $r$ is unspecified, meaning that we must choose its value.
\end{definition}

\begin{example}[Lecture 5]{Derivative of Complex Function}
  Given $f(z)$, what is its \nameref{def:Complex_Derivative} at a point $a$, $f'(a)$?
  \begin{equation*}
    f(z) = z^{2}
  \end{equation*}
  \tcblower{}
  As this problem is simple, we can just apply \Cref{eq:Complex_Derivative} directly.
  \begin{align*}
    f'(a) &= \lim\limits_{z \to a} \frac{f(z) - f(a)}{z - a} \\
          &= \lim\limits_{z \to a} \frac{z^{2} - a^{2}}{z - a} \\
          &= \lim\limits_{z \to a} \frac{(z-a) (z+a)}{z - a} \\
          &= \lim\limits_{z \to a} (z+a) \\
          &= (a + a) \\
          &= 2a
  \end{align*}

  Thus, $f'(a) = 2a$.
\end{example}

\begin{example}[Lecture 5]{Non-Existence of Derivative}
  Given $f(z)$, $a = 2i$, show that $f'(2i)$ does not exist?
  \begin{equation*}
    f(z) = \Conjugate{z}
  \end{equation*}
  \tcblower{}
  We start by applying the definition of a \nameref{def:Complex_Derivative} of a \nameref{def:Complex_Function}.
  \begin{align*}
    f'(a) &= \lim\limits_{z \to 2i} \frac{f(z) - f(a)}{z - a} \\
          &= \lim\limits_{z \to 2i} \frac{\Conjugate{z} - \Conjugate{a}}{z - a} \\
    f'(2i) &= \lim\limits_{z \to 2i} \frac{\Conjugate{z} - (-2i)}{z - a} \\
          &= \lim\limits_{z \to 2i} \frac{\Conjugate{z} + 2i}{z - 2i} \\
  \end{align*}

  Now, we can fall back to the definition of the non-existence of a \nameref{def:Complex_Limit} of a \nameref{def:Complex_Function}.
  If the value of the function approaching the point $a$ by two different paths ($\Path_{1}, \Path_{2}$) have two different values, the limit does not exist.
  If we choose our paths to be:
  \begin{align*}
    \Path_{1} &\coloneqq x = 0 \\
    \Path_{2} &\coloneqq y = x + 2i
  \end{align*}

  Now solving the limit above using Path 1 ($\Path_{1}$):
  \begin{align*}
    f'(2i) &= \lim\limits_{\substack{z \to 2i \\ z \text{ on } \Path_{1}}} \frac{\Conjugate{z} + 2i}{z - 2i} \\
           &= \lim\limits_{\substack{x = 0 \\ y \to 2}} \frac{-yi + 2i}{yi - 2i} \\
           &= \lim\limits_{y \to 2} \frac{-i (y-2)}{i (y-2)} \\
           &= -1
  \end{align*}

  Now solve the above limit using Path 2 ($\Path_{2}$):
  \begin{align*}
    f'(2i) &= \lim\limits_{\substack{z \to 2i \\ z \text{ on } \Path_{2}}} \frac{\Conjugate{z} + 2i}{z - 2i} \\
           &= \lim\limits_{\substack{x \to 0 \\ y = 2 \\ z = x+2i}} \frac{\Conjugate{z}+2i}{z-2i} \\
           &= \lim\limits_{\substack{x \to 0 \\ y = 2}} \frac{x-2i+2i}{x+2i-2i} \\
           &= \lim\limits_{\substack{x \to 0}} \frac{x}{x} \\
           &= 1
  \end{align*}

  Now, like the definition of a \nameref{def:Complex_Limit} of a \nameref{def:Complex_Function} states:
  \begin{align*}
    \lim\limits_{\substack{z \to a \\ z \text{ on } \Path_{1}}} \frac{\Conjugate{z} + 2i}{z - 2i} &\neq \lim\limits_{\substack{z \to a \\ z \text{ on } \Path_{2}}} \frac{\Conjugate{z} + 2i}{z - 2i} \\
    \lim\limits_{z \to a} \frac{f(z) - f(a)}{z - a} &= \DNE
  \end{align*}

  Thus, because the backing \nameref{def:Complex_Limit} does not exist, the entire \nameref{def:Complex_Derivative} does not exist.
\end{example}

\subsubsection{Nicities}\label{subsubsec:Complex_Derivative_Nicities}
\nameref{def:Complex_Derivative}s of \nameref{def:Complex_Function}s obey the same rules as purely-real calculus.
This includes:
\begin{itemize}[noitemsep]
\item Product Rule
\item Quotient Rule
\item Chain Rule
\item etc.
\end{itemize}

\subsubsection{Cauchy-Riemann Equations}\label{subsubsec:Cauchy_Riemann_Equations}
Because we know the \nameref{def:Complex_Derivative} of a \nameref{def:Complex_Function} \textbf{can} exist, and that $f(z) = U(x, y) + i V(x, y)$, we need to know just how special $U(x, y)$ and $V(x, y)$ really are.

If $f$ has a derivative at $z = a$, then the Cauchy-Riemann Equations \textbf{MUST} also hold true for $f'(z)$ to exist.
In the Cartesian form, these equations are seen as \Cref{eq:Cauchy_Riemann_Equations-Cartesian}.
\begin{subequations}\label{eq:Cauchy_Riemann_Equations-Cartesian}
  \begin{equation}\label{subeq:Cauchy_Riemann_Equation-UxVy}
    \begin{aligned}
      \frac{\partial U}{\partial x} (a) &= \frac{\partial V}{\partial y} (a) \\
      U_{x} (a) &= V_{y} (a)
    \end{aligned}
  \end{equation}
  \begin{equation}\label{subeq:Cauchy_Riemann_Equation-UyVx}
    \begin{aligned}
      \frac{\partial U}{\partial y} (a) &= -\frac{\partial V}{\partial x} (a) \\
      U_{y} (a) &= -V_{x} (a)
    \end{aligned}
  \end{equation}
\end{subequations}

In the polar form, the Cauchy-Riemann Equations are seen as \Cref{eq:Cauchy_Riemann_Equations-Polar}:
\begin{subequations}\label{eq:Cauchy_Riemann_Equations-Polar}
  \begin{equation}\label{subeq:Cauchy_Riemann_Equations-UrVt}
    \begin{aligned}
      \frac{\partial U}{\partial r} (a) &= \frac{1}{r} \frac{\partial V}{\partial \theta} (a)\\
      U_{r} (a) &= \frac{1}{r} V_{\theta} (a)
    \end{aligned}
  \end{equation}
  \begin{equation}\label{subeq:Cauchy_Riemann_Equations-UtVr}
    \begin{aligned}
      \frac{\partial V}{\partial r} (a) &= \frac{-1}{r} \frac{\partial U}{\partial \theta} (a) \\
      V_{r} (a) &= \frac{-1}{r} V_{\theta} (a)
    \end{aligned}
  \end{equation}
\end{subequations}

\begin{example}[Lecture 5]{Existence of Derivative using Cauchy-Riemann Equations}
  Given the function $f(z)$, find its \nameref{def:Complex_Derivative}?
  \begin{equation*}
    f(z) = \Conjugate{z} = x - iy
  \end{equation*}
  \tcblower{}
  First, we identify the functions $U(x, y)$ and $V(x, y$).
  \begin{align*}
    U(x, y) &= x \\
    V(x, y) &= -y
  \end{align*}

  Now we start by checking \Cref{subeq:Cauchy_Riemann_Equation-UxVy}.
  \begin{align*}
    \frac{\partial U}{\partial x} &= 1 \\
    \frac{\partial V}{\partial y} &= -1 \\
    \frac{\partial U}{\partial x} &\neq \frac{\partial V}{\partial y} \\
    U_{x} &\neq V_{y}
  \end{align*}

  Because the \nameref{subsubsec:Cauchy_Riemann_Equations} are \textbf{not} satisified, $f(z) = \Conjugate{z}$ has \textbf{NO} \nameref{def:Complex_Derivative} at any point $z$.
\end{example}

\begin{theorem}\label{thm:Cauchy_Riemann_Complex_Function_Derivative}
  Suppose $f$ is a \nameref{def:Complex_Function} defined in the \nameref{def:Neighborhood} $\Modulus{z-a} < r$ for some $r$.
  Suppose the \nameref{subsubsec:Cauchy_Riemann_Equations} hold at a point, $a$, and that the 4 partial derivatives $U_{x}$, $U_{y}$, $V_{x}$, and $V_{y}$ exist and are \underline{continuous} at $z=a$.
  Then the \nameref{def:Complex_Derivative} of $f$ at $z=a$ is defined to be:
  \begin{equation}\label{eq:Complex_Function_Derivative-Cauchy_Riemann_Equation_Solution}
    \begin{aligned}
      f'(z) &= \frac{\partial U}{\partial x} + i \frac{\partial V}{\partial x} \\
      f'(a) &= \frac{\partial V}{\partial y}(a) - i \frac{\partial U}{\partial y}(a)
    \end{aligned}
  \end{equation}
\end{theorem}

\begin{remark*}
  If the \nameref{subsubsec:Cauchy_Riemann_Equations} fail to hold at $z = a$, then $f(z)$ fails to have a \nameref{def:Complex_Derivative} at $z = a$.
\end{remark*}

\begin{remark*}
  There \textit{are} functions where the \nameref{subsubsec:Cauchy_Riemann_Equations} hold at a point $a$, but the function does \textbf{NOT} have a \nameref{def:Complex_Derivative} at that point.
  However, these are rare pathological examples, so we will not discuss these functions.
\end{remark*}

\begin{example}[Lecture 5]{Differentiate Complex Function using Cauchy-Riemann Equations}
  Given the function $f(z)$, use the \nameref{subsubsec:Cauchy_Riemann_Equations} to find $f'(z)$?
  \begin{equation*}
    f(z) = z^{2} = \left( x^{2} - y^{2} \right) + 2xyi
  \end{equation*}
  \tcblower{}
  We identify $U(x, y)$ and $V(x, y)$ first.
  \begin{align*}
    U(x, y) &= x^{2} - y^{2} \\
    V(x, y) &= 2xyi
  \end{align*}

  Use \Cref{eq:Cauchy_Riemann_Equations-Cartesian}.
  \begin{align*}
    \frac{\partial U}{\partial x} &= 2x & \frac{\partial V}{\partial y} &= 2x \\
    \frac{\partial U}{\partial y} &= -2y & \frac{\partial V}{\partial x} &= 2y \\
  \end{align*}
  \begin{align*}
    \frac{\partial U}{\partial x} = 2&x = \frac{\partial V}{\partial y} \\
    \frac{\partial U}{\partial y} = -2&y = -\frac{\partial V}{\partial x} \\
  \end{align*}

  $U_{x}, V_{y}, U_{y}, V_{x}$ are polynomials.
  Hence, they are continuous.
  Hence $f$ has a derivative at all points.
  Because this function passes the requirements placed on $f(z)$ by the \nameref{subsubsec:Cauchy_Riemann_Equations}, we can find the \nameref{def:Complex_Derivative} of $f(z)$.
  \begin{align*}
    f'(z) &= \frac{\partial U}{\partial x} + i \frac{\partial V}{\partial x} \\
          &= 2x + 2yi
  \end{align*}

  Thus, $f'(z) = 2x + 2yi$.
\end{example}

\begin{example}[Lecture 5]{Differentiate Complex Trig using Cauchy-Riemann Equations}
  Given the function $f(z) = \cos(z)$, verify that $f'(z) = -\sin(z)$?
  \tcblower{}
  Start by identifying $U(x, y)$ and $V(x, y)$.
  \begin{align*}
    f(z) &= \cos(z) \\
    \cos(z) &= \cos(x + iy) \\
         &= \cos(x) \cosh(y) - i \sin(x) \sinh(y) \\
    U(x, y) &= \cos(x) \cosh(y) \\
    V(x, y) &= -\sin(x) \sinh(y) \\
  \end{align*}

  Now use \Cref{eq:Cauchy_Riemann_Equations-Cartesian}.
  \begin{align*}
    \frac{\partial U}{\partial x} &= -\sin(x) \cosh(y) & \frac{\partial V}{\partial y} &= -\sin(x) \cosh(y) \\
    \frac{\partial U}{\partial y} &= \cos(x) \sinh(y) & \frac{\partial V}{\partial x} &= -\cos(x) \sinh(y)
  \end{align*}
  \begin{align*}
    \frac{\partial U}{\partial x} = -\sin(x)& \cosh(y) = \frac{\partial V}{\partial y} \\
    \frac{\partial U}{\partial y} = \cos(x)&\sinh(y) = -\frac{\partial V}{\partial x}
  \end{align*}

  Thus, the \nameref{subsubsec:Cauchy_Riemann_Equations} are satisified.
  In addition, the 4 partial derivatives are continuous at all points.

  Therefore, $f(z) = \cos(z)$ \textbf{has} a derivative at all points.
  According to \Cref{eq:Complex_Function_Derivative-Cauchy_Riemann_Equation_Solution}, the solution is:
  \begin{align*}
    f'(z) &= \frac{\partial U}{\partial x} + i \frac{\partial V}{\partial x} \\
          &= -\sin(x) \cosh(y) + i \bigl( -\cos(x) \sinh(y) \bigr) \\
    \shortintertext{Factor out the negative.}
          &= - \bigl( \sin(x) \cosh(y) + i \cos(x) \sinh(y) \bigr) \\
    \shortintertext{By the definition of $\sin(z)$ in Cartesian form, we can simplify everything in the parentheses.}
          &= -\sin(x + iy) \\
          &= -\sin(z)
  \end{align*}
\end{example}

\begin{definition}[Analytic]\label{def:Analytic}
  Let $f$ be a \nameref{def:Complex_Function} which has a \nameref{def:Complex_Derivative} at all points of an \nameref{def:Open_Connected_Set} $\Omega$.
  Then we say $f$ is \emph{analytic} in $\Omega$.

  This means that $f$ has a derivative at all points in the \nameref{def:Open_Connected_Set}.
\end{definition}

\begin{definition}[Open Connected Set]\label{def:Open_Connected_Set}
  An \emph{open connected set} is a special type of set that we use to visualize on the \nameref{def:Argand_Plane}.
  \begin{description}[noitemsep]
  \item[Set:] As a set, it can be visualized as a blob in the \nameref{def:Argand_Plane}.
  \item[Open:] The set being open means that the boundary edge is \textbf{not} included with the set.
    This also means that for every point within the set, we can define a disk with a radius, where the \textbf{entire} disk is contained within the set.
  \item[Connected:] When a union of two points, with their disks, occurs inside this set, there is a \textbf{continuous} path between them that lies entirely within the set's boundaries.
  \end{description}
\end{definition}

\begin{theorem}
  Define a function $f$ like so:
  \begin{equation*}
    f(z) = U(x, y) + i V(x, y)
  \end{equation*}

  If the function $f$ is \nameref{def:Analytic} on $\Omega$, then $f$ has a derivative at all points in $\Omega$, meaning
  \begin{align*}
    \frac{\partial U}{\partial x} &= \frac{\partial V}{\partial y} \\
    \frac{\partial U}{\partial y} &= -\frac{\partial V}{\partial x}
  \end{align*}

  This means that the \nameref{subsubsec:Cauchy_Riemann_Equations} hold, and the partial derivatives are valid.

  Then, according to our work in \Cref{subsubsec:Special_U_V_Interdependency}, $U(x, y)$ and $V(x, y)$ are \nameref{def:Harmonic}.
\end{theorem}

\subsubsection{Specialties of $U$, $V$, and their Interdependency}\label{subsubsec:Special_U_V_Interdependency}
%%% Local Variables:
%%% mode: latex
%%% TeX-master: "../../Math_333-MatrixAlg_ComplexVars-Reference_Sheet"
%%% End:
