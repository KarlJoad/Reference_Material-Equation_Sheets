\subsection{Paths}\label{subsec:Paths}
In this course, \nameref{def:Path}s, curve, and contour are synonymous.
However, in general, they are not.

\begin{definition}[Path]\label{def:Path}
  A \emph{path} is static \nameref{def:Image} of a line on a plane.
  However, when we say we are interested in the path of an image, we are not really talking about the static image we see, but rather \textbf{HOW} we got that image, what function generated it, etc.
\end{definition}

In general, if we are given a \nameref{def:Path}, we can find the any point on the path by parameterizing the beginning and end of the path.
This is the function $z: [a, b] \to \ComplexNumbers$ where $z = z(t)$ where $a \leq t \leq b$.

\begin{example}[Lecture 6, Example 2]{Parameterize a Path}
  Given a line segment $[0, 2+3i]$ in the $\ComplexNumbers$ plane, what is the parameterized function that creates \textbf{every} point for the segment?
  \tcblower{}
  Technically, there are infinitely many solutions.
  Just a few are presented below:
  \begin{align*}
    z(t) &= (2 + 3i)t \:\: 0 \leq t \leq 1 \\
    z(t) &= (2 + 3i)2t \:\: 0 \leq t \leq \frac{1}{2} \\
    z(t) &= (2 + 3i)(1-t) \:\: 0 \leq t \leq 1 \\
    \shortintertext{If $z = x + iy$:}
    y &= \frac{3}{2} x \:\: 0 \leq x \leq 2
  \end{align*}
\end{example}

The general form for \nameref{def:Parameterizing_Path} a \nameref{def:Path} is shown in \Cref{eq:Parameterizing_Path}.

\begin{definition}[Parameterizing]\label{def:Parameterizing_Path}
  The act of \emph{parameterizing} a \nameref{def:Path} is the act of finding a parametric equation that completely describes the \nameref{def:Path} in the \nameref{def:Image}.
  The general equation for a path defined from $[a, b]$ where $a, b \in \ComplexNumbers$ is shown below.
  \begin{equation}\label{eq:Parameterizing_Path}
    z(t) = a(1-t) + b(t) \:\: 0 \leq t \leq 1
  \end{equation}
\end{definition}

\begin{example}[Lecture 6, Example 4]{Parameterize a Circle}
  Given the circle centered at point $a$, where $a \in \ComplexNumbers$ with a radius $r > 0$, find a parameterized equation?
  \tcblower{}
  Start by stating the definition of a circle in the complex plane.
  \begin{equation*}
    \Modulus{z-a} = r
  \end{equation*}

  By thinking about this a little, we can see that we can generate every point on the circle's edge by keeping $r$ constant and varying only $\theta = \arg(z)$, the argument of the circle.
  Now, to simplify things, we will also write some of the subsequent equations in polar form.
  \begin{align*}
    z - a &= r \bigl( \cos(\theta) + i \sin(\theta) \bigr) \\
    \shortintertext{\Cref{eq:Euler Complex} can simplify this further.}
          &= a + r e^{i \theta} \:\: 0 \leq \theta \leq 2\pi
  \end{align*}

  Thus, we have parameterized the equation for a circle, in 2 forms: \Cref{subeq:Circle_Parameterized_CCD} and \Cref{subeq:Circle_Parameterized_CD}.
Parameterization of a circle in Counter Clockwise Direction (CCD)
\begin{subequations}\label{eq:Circles_Parameterized}
  \begin{equation}\label{subeq:Circle_Parameterized_CCD}
    z = a + r e^{i \theta} \:\: 0 \leq \theta \leq 2\pi
  \end{equation}
  \begin{equation}\label{subeq:Circle_Parameterized_CD}
    z = a + r e^{-i \theta} \:\: 0 \leq \theta \leq 2\pi
  \end{equation}
\end{subequations}
\end{example}

%%% Local Variables:
%%% mode: latex
%%% TeX-master: "../../Math_333-MatrixAlg_ComplexVars-Reference_Sheet"
%%% End:
