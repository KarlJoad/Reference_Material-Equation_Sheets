\subsection{Integration}\label{subsec:Integration}
The integration method that we will present in this section is \nameref{def:Path}-agnostic.
This means that regardless of how you go about \nameref{def:Parameterizing_Path} the \nameref{def:Path}, the integral returns the same result.

The functions we will be integrating are complex-valued functions.
We will be integrating these functions along one of their many \nameref{def:Path}s.
These paths are typically denoted with either $C$, $\Gamma$, or $\gamma$.
I will choose $\Path$ in this document, to match up with the use of $\Path$ in \nameref{def:Complex_Limit}s.

\begin{definition}[Integration on Paths]\label{def:Integration_on_Paths}
  Let $f$ be a continuous function on the image of the \nameref{def:Path}.
  We denote this integral similarly to how we denote it regularly.
  \begin{equation}\label{eq:Integration_on_Paths}
    \begin{aligned}
      \int_{\Path} f &= \int_{\Path} f(z) dz \\
      &= \int_{a}^{b} f\bigl( z(t) \bigr) z'(t) dt
    \end{aligned}
  \end{equation}
\end{definition}

\begin{example}[Lecture 7]{Calculate the Integral on any Path}
  Given the line segment $\ell = [-4+5i, 3-2i]$, and an arbitrary point $z$ defined by the function $z(t)$.
  Evaluate $\int_{\ell} \Real{z} dz$?
  \tcblower{}
  Start by parameterizing the path.
  I will use the general form of finding a parameterized function for $z(t)$ from \Cref{eq:Parameterizing_Path}.
  \begin{align*}
    z(t) &= a(1-t) + bt \:\: 0 \leq t \leq 1 \\
         &= (-4+5i)(1-t) + (3-2i)t \:\: 0 \leq t \leq 1
  \end{align*}

  Now, we can evaluate the integral.
  \begin{align*}
    \int_{\ell} \Real{z} dz &= \int_{0}^{1} \bigl( -4(1-t) + 3t \bigr) z'(t) dt \\
                            &= \int_{0}^{1} (7t - 4) z'(t) dt \\
                            &= \int_{0}^{1} (7t - 4) \bigl( (-4+5i)(-t) + (3-2i) \bigr) dt \\
                            &= \int_{0}^{1} (7t - 4) (7 - 7i) dt \\
                            &= (7-7i) \int_{0}^{1} (7t-4) dt \\
                            &= (7-7i) \left( \Bigl( \frac{7}{2}t^{2}-4t \Bigr) \vert_{t=0}^{t=1} \right) \\
                            &= \frac{-7}{2}(1-i)
  \end{align*}
\end{example}

\begin{theorem}[Complex Antiderivative]\label{thm:Complex_Antiderivative}
  Let $\Omega$ be a \nameref{def:Simply} \nameref{def:Connected} region and $f: \Omega \to \ComplexNumbers$ is an \nameref{def:Analytic} \nameref{def:Complex_Function} on $\Omega$.
  Then, there exists another \nameref{def:Analytic} function ($\exists F: \Omega \to \ComplexNumbers$) such that the derivative of the new function is equal to the original function.
  \begin{equation*}
    F' = f
  \end{equation*}

  Thus, $F$ is an antiderivatives of $f$.
  \begin{equation}\label{eq:Complex_Antiderivative}
    \int_{\Path} f = F(B) - F(A)
  \end{equation}

  \begin{remark*}[Use]
    This is mainly used for evaluating integrals.
  \end{remark*}
\end{theorem}

If we look at \Cref{eq:Complex_Antiderivative}, we notice that the right-hand side ($F(B) - F(A)$) is independent of the \nameref{def:Path}.


%%% Local Variables:
%%% mode: latex
%%% TeX-master: "../../Math_333-MatrixAlg_ComplexVars-Reference_Sheet"
%%% End:
