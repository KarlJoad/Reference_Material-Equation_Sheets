\subsection{Integration}\label{subsec:Integration}
\subsubsection{Integration on Paths}\label{subsubsec:Integration_on_Paths}
The integration method that we will present in this section is \nameref{def:Path}-agnostic.
This means that regardless of how you go about \nameref{def:Parameterizing_Path} the \nameref{def:Path}, the integral returns the same result.

The functions we will be integrating are complex-valued functions.
We will be integrating these functions along one of their many \nameref{def:Path}s.
These paths are typically denoted with either $C$, $\Gamma$, or $\gamma$.
I will choose $\Path$ in this document, to match up with the use of $\Path$ in \nameref{def:Complex_Limit}s.

\begin{definition}[Integration on Paths]\label{def:Integration_on_Paths}
  Let $f$ be a continuous function on the image of the \nameref{def:Path}.
  We denote this integral similarly to how we denote it regularly.
  \begin{equation}\label{eq:Integration_on_Paths}
    \begin{aligned}
      \int_{\Path} f &= \int_{\Path} f(z) dz \\
      &\coloneqq \int_{a}^{b} f\bigl( z(t) \bigr) z'(t) dt
    \end{aligned}
  \end{equation}

  \begin{remark}
    You can \textbf{always} perform an integration by parameterizing the path and integrating.
    However, this can become very complex sometimes.
  \end{remark}
\end{definition}

\begin{example}[Lecture 7]{Calculate the Integral on any Path}
  Given the line segment $\ell = [-4+5i, 3-2i]$, and an arbitrary point $z$ defined by the function $z(t)$.
  Evaluate $\int_{\ell} \Real{z} dz$?
  \tcblower{}
  Start by parameterizing the path.
  I will use the general form of finding a parameterized function for $z(t)$ from \Cref{eq:Parameterizing_Path}.
  \begin{align*}
    z(t) &= a(1-t) + bt \:\: 0 \leq t \leq 1 \\
         &= (-4+5i)(1-t) + (3-2i)t \:\: 0 \leq t \leq 1
  \end{align*}

  Now, we can evaluate the integral.
  \begin{align*}
    \int_{\ell} \Real{z} dz &= \int_{0}^{1} \bigl( -4(1-t) + 3t \bigr) z'(t) dt \\
                            &= \int_{0}^{1} (7t - 4) z'(t) dt \\
                            &= \int_{0}^{1} (7t - 4) \bigl( (-4+5i)(-t) + (3-2i) \bigr) dt \\
                            &= \int_{0}^{1} (7t - 4) (7 - 7i) dt \\
                            &= (7-7i) \int_{0}^{1} (7t-4) dt \\
                            &= (7-7i) \left( {\left[ \frac{7}{2}t^{2}-4t \right]}_{t=0}^{t=1} \right) \\
                            &= \frac{-7}{2}(1-i)
  \end{align*}
\end{example}

\subsubsection{Antiderivatives}\label{subsubsec:Antiderivatives}
\begin{theorem}[Complex Antiderivative]\label{thm:Complex_Antiderivative}
  Let $\Omega$ be a \nameref{def:Simply_Connected}, \nameref{def:Connected}, and \nameref{def:Open} region.
  Let $f$ be an \nameref{def:Analytic} \nameref{def:Complex_Function} defined on $\Omega$, $f: \Omega \to \ComplexNumbers$ on $\Omega$.

  Then, there exists another \nameref{def:Analytic} function ($\exists F: \Omega \to \ComplexNumbers$) such that the derivative of the new function is equal to the original function.
  \begin{equation*}
    F' = f
  \end{equation*}

  Thus, $F$ is \textbf{an} antiderivatives of $f$.
  \begin{equation}\label{eq:Complex_Antiderivative}
    \int_{\Path} f = F(B) - F(A)
  \end{equation}

  \begin{remark*}[Use]
    This is mainly used for evaluating integrals.
  \end{remark*}
\end{theorem}

\subsubsection{Using Antiderivatives}\label{subsubsec:Using_Antiderivatives}
To use antiderivatives to evaluate integrals, we use the steps below:
\begin{enumerate}[noitemsep]
\item Have a \nameref{def:Simply_Connected}, \nameref{def:Open}, \nameref{def:Connected} region, $\Omega$.
\item Show that $C$ is a \nameref{def:Path}/curve fully inside the region $\Omega$.
\item Show that $f$ is \nameref{def:Analytic} on $\Omega$.
  \begin{itemize}[noitemsep]
  \item This is the big condition and the hard one to prove.
  \end{itemize}
\item Evaluate $\int_{C} f = {[F]}_{A}^{B} = F(B) - F(A)$.
  \begin{itemize}[noitemsep]
  \item Notice that the right-hand side ($F(B) - F(A)$) is \textbf{independent} of the \nameref{def:Path}.
  \item Only the starting and end points matter.
  \end{itemize}
\end{enumerate}

\begin{example}[Lecture 7]{Antiderivatives}
  Evaluate the integral $\int_{\Path} z \sin \left( z^{2} \right) dz$ where $\Path$ is equal to the union of the line segment that stretches from the origin to $(1, 2)$, \textbf{with} the lower half of the circle whose center is at $a = 3+2i$ with radius $r = 2$.
  \tcblower{}
  Using \Cref{thm:Complex_Antiderivative}, we know that we don't have to actually evaluate this integral.
  Instead, we can find the beginning and end points of the \nameref{def:Path}, and just take the difference of those.
  So, $A = 0+0i$, and $B = 5+2i$.
  $z \sin \left( z^{2} \right)$ is \nameref{def:Analytic} on $\ComplexNumbers$.
  Therefore, we are guaranteed the derivative's existence.
  This also means that \textbf{an} antiderivative exists.
  \begin{equation*}
    F(z) = \frac{-1}{2} \cos \left( z^{2} \right)
  \end{equation*}

  Remember that there would be a constant term at the end of the antiderivative, but because we will be using this function and subtracting it from another value, the constants would just cancel each other out.

  Using \Cref{eq:Complex_Antiderivative}, we can now solve this directly.
  \begin{align*}
    \int_{\Path} f &= \int_{\Path} z \sin \left( z^{2} \right) dz \\
                   &= F(B) - F(A) \\
                   &= \frac{-1}{2} \cos \left( {(5+2i)}^{2} \right) - \frac{-1}{2} \cos \left( {(0+0i)}^{2} \right) \\
                   &= \frac{-1}{2} \left( \cos \left( {(5+2i)}^{2} \right) - \cos(0) \right) \\
                   &= \frac{-1}{2} \left( \cos \left( {(5+2i)}^{2} \right) \right) + \frac{1}{2} \\
  \end{align*}

  The last step would be to use the \nameref{subsec:Angle Sum and Difference Identities} to simplify the remaining $\cos$ into a normal, Cartesian, form.
\end{example}

\subsubsection{Which to Method to Integrate with?}
You choose your integration method based on whether or not the function you are integrating is \nameref{def:Analytic}.
Refer to \Cref{subsubsec:Analytic_NonAnalytic} if you need a short list of \nameref{def:Complex_Function}s that usually are/aren't analytic.

\begin{itemize}[noitemsep]
\item If the function \textbf{IS} \nameref{def:Analytic}, then you can use \nameref{subsubsec:Antiderivatives}.
\item If the function is \textbf{NOT} \nameref{def:Analytic}, then you must use \nameref{def:Integration_on_Paths}.
\end{itemize}

\begin{example}[Lecture 7, Problem 5]{Choose Method of Integration}
  Given a \nameref{def:Path} $C$ being the unit circle in the counter-clockwise direction, evaluate the expression below?
  \begin{equation*}
    \int_{C} \frac{1}{z} dz
  \end{equation*}
  \tcblower{}
  In this case, our $\Omega = \ComplexNumbers$.
  If we start by looking at our given expression to integrate, $\frac{1}{z}$, we quickly notice that this is \textbf{NOT} \nameref{def:Analytic}, because there is a hole at $z = 0$, which conflicts with $\Omega$'s requirement to be \nameref{def:Simply_Connected}.

  This leads to us \textbf{NOT} being able to use \nameref{subsubsec:Antiderivatives}.
  So, we have to perform an \nameref{def:Integration_on_Paths}.

  We start by \nameref{def:Parameterizing_Path} the \nameref{def:Path}.
  Using \Cref{subeq:Circle_Parameterized_CCD}, we can easily get the parameterization.
  \begin{align*}
    z(\theta) &= a + re^{i \theta} \:\: 0 \leq \theta \leq 2 \pi \\
    r &= 1 \\
    a &= 0 \\
    z(\theta) &= e^{i \theta}
  \end{align*}

  Using our result for $z(\theta)$, we can find $z'(\theta)$.
  \begin{align*}
    z(\theta) &= e^{i \theta} \\
    z'(\theta) &= i e^{i \theta}
  \end{align*}

  Now, we integrate according to \Cref{eq:Integration_on_Paths}.
  \begin{align*}
    \int_{C} \frac{1}{z} dz &= \int_{0}^{2 \pi} \frac{1}{z(\theta)} z'(\theta) d\theta \\
                            &= \int_{0}^{2 \pi} \frac{1}{e^{i \theta}} \left( i e^{i \theta} \right) d\theta \\
                            &= \int_{0}^{2 \pi} i d\theta \\
                            &= i \int_{0}^{2 \pi} d\theta \\
                            &= i {[\theta]}_{\theta = 0}^{\theta = 2 \pi} \\
                            &= 2 \pi i
  \end{align*}
\end{example}

\subsubsection{Cauchy's Integral Theorem}\label{subsubsec:Cauchys_Integral_Theorem}
\begin{theorem}[Cauchy's Integral Theorem]\label{thm:Cauchys_Integral_Theorem}
  Let $\Omega$ be a \nameref{def:Simply_Connected} and \nameref{def:Open} set.
  Let $f$ be an \nameref{def:Analytic} function in $\Omega$.
  Let $C$ be a curve/\nameref{def:Path} \textbf{INSIDE} $\Omega$, described in a counter-clockwise direction, be \nameref{def:Closed_Path} and a \nameref{def:Simple_Path}.
  Let $a$ be a point in the interior of $C$.

  Then, the below equation holds true.
  \begin{equation}\label{eq:Cauchys_Integral_Theorem}
    \int_{C} \frac{f(z)}{{(z-a)}^{n}} dz = 2 \pi i \left( \frac{1}{(n-1)!} \right) \frac{ d^{n-1} f(a)}{{dz}^{n-1}}
  \end{equation}
\end{theorem}

\begin{definition}[Closed]\label{def:Closed_Path}
  A \emph{closed} path is one whose starting point is the same as its ending point.
\end{definition}

\begin{definition}[Simple]\label{def:Simple_Path}
  A \emph{simple} path is one that doesn't cross itself.
  For example, the figure $8$ is an example of a curve that is \textbf{not} simple.
\end{definition}

\begin{example}[Lecture 7, Problem 6]{Use Cauchy's Integral Theorem}
  Let a curve $C$ be the rectangle with points $(4, 1), (-4, 1), (-4, -1), (4, -1)$.
  Use \nameref{thm:Cauchys_Integral_Theorem} to solve the below expression?
  \begin{equation*}
    \int_{C} \frac{\sin(z)}{3z-4 (z-5)} dz
  \end{equation*}
  \tcblower{}
  In this case, our $\Omega = \ComplexNumbers$.
  In addition, from inspection of \Cref{eq:Cauchys_Integral_Theorem}, we see that the denominator \textbf{must} be in the form of $z-a$.
  However, we are given the product of $3z-4 (z-5)$.

  But, we can ``eliminate'' one of them, and treat it as part of $f(z)$ by remembering the fourth assumption of \Cref{thm:Cauchys_Integral_Theorem}, that $a$ must be \textbf{inside} $C$.
  If we look at the value $z-5$, we notice that the value for $a = 5$, which is outside $C$.
  So, we ``bring up'' that expression and have it all divided by $3z-4$.
  \begin{align*}
    \int_{C} \frac{\sin(z)}{3z-4 (z-5)} dz &= \int_{C} \frac{\frac{\sin(z)}{z-5}}{3z-4} dz \\
    f(z) &= \frac{\sin(z)}{z-5} \\
    \intertext{$f(z)$ is \nameref{def:Analytic} \textbf{inside} $C$. However, it is non-analytic at $z = 5$.}
    \intertext{From \Cref{eq:Cauchys_Integral_Theorem}, we see that the denominator $z-a$ \textbf{must} be in that form. So, we can factor the $3$ out.}
    \int_{C} \frac{\frac{\sin(z)}{z-5}}{3z-4} dz &= \frac{1}{3} \int_{C} \frac{\frac{\sin(z)}{z-5}}{z-\frac{4}{3}} dz \\
    \shortintertext{Use \Cref{eq:Cauchys_Integral_Theorem} now.}
                                           &= \frac{1}{3} \left( 2 \pi i \left( \frac{1}{(1-1)!} \right) \right) \frac{d^{1-1} f(a)}{{dz}^{1-1}} \\
                                           &= \frac{1}{3} \left( 2 \pi i \left( \frac{1}{0!} \right) \right) f(a) \\
                                           &= \frac{1}{3} \bigl( 2 \pi i (1) \bigr) \left( \frac{\sin \left( \frac{4}{3} \right)}{\frac{4}{3} - 5} \right) \\
    &= \frac{-2 \pi i}{11} \sin \left( \frac{4}{3} \right)
  \end{align*}
\end{example}

\paragraph{Generalizing Cauchy's Integral Theorem}\label{par:Generalize_Cauchy_Integral_Theorem}
We want to remove requirements on $C$ to be \nameref{def:Simple_Path} from \nameref{thm:Cauchys_Integral_Theorem}.
This means that a non-\nameref{def:Simple_Path} is one with at least two loops, meaning there is no single uniform direction the path travels.
However, if we were to break the path into the union of two simple paths, then we could use \nameref{thm:Cauchys_Integral_Theorem}.
In fact, this is exactly what we will do!

\begin{remark*}
  If you remember back to Calculus of a single variable, we had something similar that we used frequently.
  We were allowed to break a single integral into two separate ones that were added together, which allowed us to evaluate many different expressions.
  Likewise, we were also able to flib the integral's bounds and change teh direction of integration, which is also used here.
\end{remark*}

%%% Local Variables:
%%% mode: latex
%%% TeX-master: "../../Math_333-MatrixAlg_ComplexVars-Reference_Sheet"
%%% End:
