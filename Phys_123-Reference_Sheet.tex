\documentclass[10pt,letterpaper,final,twoside,notitlepage]{article}
\usepackage[margin=.5in]{geometry}
\usepackage[utf8]{inputenc}
\usepackage[english]{babel}
\usepackage{amsmath}
\usepackage{amsfonts}
\usepackage{amssymb}
\usepackage{amsthm} % Gives us plain, definition, and remark to use in \theoremstyle{style}
\usepackage{graphicx}

\usepackage{hyperref} % Generate hyperlinks to referenced items
\usepackage[noabbrev,nameinlink]{cleveref} % Fancy cross-references in the document everywhere
\usepackage{nameref} % Can make references by name to places
\usepackage{subcaption} % Allows for multiple figures in one Figure environment
\usepackage{siunitx} % Gives us ways to typeset units for stuff
\usepackage{enumitem} % Provides [noitemsep, nolistsep] for more compact lists
\usepackage{chngcntr} % Allows us to tamper with the counter a little more
\usepackage{empheq} % Allow boxing of equations in special math environments
\usepackage{tcolorbox} % Allows us to create boxes of various types for examples
\usepackage{tikz} % Allows us to create TikZ and PGF Pictures
\usetikzlibrary{trees}
%\usepackage{ctable} % Greater control over tables and how they look

\graphicspath{{./Drawings/Phys_123}} % Uncomment this to use pictures in this document
\counterwithin{equation}{section} % Uncomment to number eqns with sec nums too

\theoremstyle{plain}
\newtheorem{theorem}{Theorem}
\counterwithin{theorem}{section}

\theoremstyle{definition}
\newtheorem{definition}{Defn}
\newtheorem{corollary}{Corollary}[section]

\theoremstyle{remark}
\newtheorem{remark}{Remark}[definition]
\newtheorem*{remark*}{Remark}
%\counterwithin{definition}{subsection} % Uncomment to have definitions use section.subsection numbering

% Create a special list that handles properties. It can be broken and restarted
\newlist{propertylist}{enumerate}{1} % {Name}{Template}{Max Depth}
\setlist[propertylist, 1]{label=\textbf{(\roman*)}, noitemsep, nolistsep} % Set options

% Create a special list that handles enumerate starting with lower letters. Breakable/Restartable.
\newlist{boldalphlist}{enumerate}{1} % {Name}{Template}{Max Depth}
\setlist[boldalphlist, 1]{label=\textbf{(\alph*)}, noitemsep, nolistsep} % Set options

% Create a tcolorbox for examples
% Argument #1 is optional, given by [], that is the textbook's problem number
% Argument #2 is mandatory, given by {}, that is the title for the example
\newtcolorbox[auto counter,
number within=section,
number format=\arabic,
crefname={example}{examples}, % Define reference format for cref (No Capitals)
Crefname={Example}{Examples}, % Reference format for cleveref (With Capitals)
]{example}[2][]{ % The [2][] Means the first argument is optional
	width=\textwidth,
	title={Example \thetcbcounter: #2. #1},
	fonttitle=\bfseries,
	label={ex:#2},
	nameref=#2,
	colbacktitle=white!100!black,
	coltitle=black!100!white,
	colback=white!100!black,
	upperbox=visible,
	lowerbox=visible,
	sharp corners=all
}

% Redefine the 'end of proof' symbol to be a black square, not blank
\renewcommand\qedsymbol{$\blacksquare$} % Change proofs to have black square at end

\DeclareSIUnit\mile{mi}
\DeclareSIUnit{\mph}{mph}

\DeclareMathOperator{\cross}{\times}
\DeclareMathOperator{\RealNums}{\mathbb{R}}
\DeclareMathOperator*{\argmax}{argmax} % Thin Space and subscripts are UNDER in display

\begin{titlepage}
  \title{Phys 123: Classical Mechanics - Reference Sheet}
  \author{Karl Hallsby}
  \date{Last Edited: \today}
\end{titlepage}

\begin{document}
\pagenumbering{gobble}
\maketitle
\pagenumbering{roman} % i, ii, iii on beginning pages, that don't have content
\tableofcontents
\clearpage
\pagenumbering{arabic} % 1,2,3 on content pages

\section{Vectors} \label{sec:Vectors}
\begin{definition}[Vector] \label{def:Vector}
  A \emph{vector} is a way to show both magnitude of displacement and direction of displacement.
  Vectors are drawn as rays.
  \begin{remark}
    \nameref{def:Vector}s and \nameref{def:Scalar}s may seem similar, but are different.
  \end{remark}
\end{definition}

\begin{definition}[Scalar] \label{def:Scalar}
  A \emph{scalar} is a way to show \emph{\textbf{ONLY}} the magnitude of a displacement, without any direction information.
\end{definition}

\subsection{Vector Properties} \label{subsec:Vector Properties}
\begin{propertylist}
  \item $\vec{A} + \vec{B} = \vec{C}$
  \item $\vec{0} = \langle 0, 0, 0, \ldots, 0 \rangle$
  \item $\vec{A} + \vec{0} = \vec{A}$
  \item $\vec{A} + -\vec{A} = \vec{0}$
  \item $\left( \vec{A} + \vec{B} \right) + \vec{C} = \vec{A} + \left( \vec{B} + \vec{C} \right)$
  \item $\vec{A} + \vec{B} = \vec{B} + \vec{A}$
  \item Magnitude of vector: $\lVert \vec{A} \rVert = \sqrt{A_{x}^{2} + A_{y}^{2} + A_{z}^{2}}$
\end{propertylist}
  
\subsubsection{Getting Components} \label{subsubsec:Vector Components}
Getting the components of a vector involves solving the imaginary pythagorean triangle around the vector.

For a 2-dimensional vector, $\vec{V}$, you have the components $\langle V_{x}, V_{y} \rangle$.
You find their values with this equation:
\begin{equation} \label{eq:Vector Components}
  \begin{aligned}
    V_{x} &= V \cos \theta \\
    V_{y} &= V \sin \theta
  \end{aligned}
\end{equation}

\subsubsection{3D Unit Vectors} \label{subsubsec:3D Unit Vectors}
3-dimensional vectors shouldn't be any too crazy by this point.
They are just another variable that can be thrown around in the vector.
However, the three \nameref{subsubsec:3D Unit Vectors} are special.
You can also use these to describe any lower-dimensional vector as well.
\begin{equation}
  \begin{aligned}
    \hat{\imath} &= \langle 1, 0, 0 \rangle \\
    \hat{\jmath} &= \langle 0, 1, 0 \rangle \\
    \hat{k} &= \langle 0, 0, 1 \rangle
  \end{aligned}
\end{equation}

\subsubsection{Addition} \label{subsubsec:Vector Addition}
Vectors are additive, and are done from head-to-tail.
This means that
\begin{equation} \label{eq:Vector Addition}
  \vec{A} + \vec{B} = \vec{C}  
\end{equation}

This means that in 3-dimensional vectors, they are added like this:
\begin{equation} \label{eq:3D Vector Addition}
  \begin{aligned}
    \vec{A} &= \langle A_{x}, A_{y}, A_{z} \rangle \\
    \vec{B} &= \langle B_{x}, B_{y}, B_{z} \rangle \\
    \vec{A} + \vec{B} &= \langle A_{x}+B_{x}, A_{y}+B_{y}, A_{z}+B_{z} \rangle
  \end{aligned}
\end{equation}

\subsubsection{Scalar Multiplication} \label{subsubsec:Scalar Vector Multiplication}
When applying multiplication between a scalar and a vector, you perform \nameref{subsubsec:Scalar Vector Multiplication}.
\begin{equation} \label{eq:Scalar Vector Multiplication}
  2 \times \vec{V} = 2 \langle V_{x}, V_{y} \rangle = \langle 2V_{x}, 2V_{y}, 2V_{z} \rangle
\end{equation}

This means that you do \emph{\textbf{NOT}} modify the direction of the vector, you only change its magnitude.

\subsubsection{Scalar (Dot) Product} \label{subsubsec:Dot Product}
The \nameref{subsubsec:Dot Product} is the first of two ways to multiply 2 vectors.
The other is the \nameref{subsubsec:Cross Product}.
There are 2 ways to calculate the \nameref{subsubsec:Dot Product}.

The first involves using the magnitudes of each vector and multiplying those by the cosine of the angle between them.
\begin{equation} \label{eq:Dot Product Magnitudes}
  \vec{A} \cdot \vec{B} = \lVert \vec{A} \rVert \lVert \vec{B} \rVert \cos \left( \theta \right)
\end{equation}

The second is done by adding the product of each component of each vector.
\begin{equation} \label{eq:Dot Product Components}
  \vec{A} \cdot \vec{B} = A_{x}B_{x} + A_{y}B_{y} + A_{z}B_{z}
\end{equation}

\begin{remark*}
  This means that when you apply the \nameref{subsubsec:Dot Product} to 2 vectors, you return a \nameref{def:Scalar}.
\end{remark*}

\paragraph{Properties of \nameref{subsubsec:Dot Product}} \label{par:Dot Product Properties}
\begin{propertylist}
  \item $( \vec{A} )^{2} = \vec{A} \cdot \vec{A}$
  \item $\vec{A} \cdot \vec{B} = \vec{B} \cdot \vec{A}$
\end{propertylist}

\subsubsection{Vector (Cross) Product} \label{subsubsec:Cross Product}
The \nameref{subsubsec:Cross Product} is the second of two ways to multiply 2 vectors.
The other is the \nameref{subsubsec:Dot Product}.
There are 2 ways to calculate the \nameref{subsubsec:Cross Product}.

The first involves using the magnitudes of each vector and multiplying those by the sine of the angle between them.
\begin{equation} \label{eq:Cross Product Magnitudes}
  \vec{A} \cross \vec{B} = \lVert \vec{A} \rVert \lVert \vec{B} \rVert \sin \theta
\end{equation}

The second is done by taking the determinant of a $2 \times 2$ or $3 \times 3$ matrix.
\begin{equation} \label{eq:Cross Product Determinant}
  \begin{aligned}
    \vec{A} \cross \vec{B}
    &= \det \begin{bmatrix}
      \hat{\imath} & \hat{\jmath} & \hat{k} \\
      A_{x} & A_{y} & A_{z} \\
      B_{x} & B_{y} & B_{z}
    \end{bmatrix} \\
    &= \begin{vmatrix}
      \hat{\imath} & \hat{\jmath} & \hat{k} \\
      A_{x} & A_{y} & A_{z} \\
      B_{x} & B_{y} & B_{z}
    \end{vmatrix} \\
    &= \left( A_{y}B_{z}-A_{z}B_{y} \right) \hat{\imath} - \left( A_{x}B_{z}-A_{z}B_{x} \right) \hat{\jmath} + \left( A_{x}B_{y} - A_{y}B_{x} \right) \hat{k} \\
    &= \langle A_{y}B_{z}-A_{z}B_{y}, - \left( A_{x}B_{z}+A_{z}B_{x} \right), A_{x}B_{y}-A_{y}B_{x} \rangle
  \end{aligned}
\end{equation}

\begin{remark*}
  This means that when you apply the \nameref{subsubsec:Cross Product} to 2 vectors, you return a \nameref{def:Vector}.
\end{remark*}

\paragraph{Properties of \nameref{subsubsec:Cross Product}} \label{par:Cross Product Properties}
\begin{propertylist}
  \item $\vec{A} \cross \vec{A} = \vec{0}$
  \item $\vec{A} \cross \vec{B} = - \left( \vec{B} \cross \vec{A} \right)$
  \item $\vec{A} \cross \left( \vec{B} \cross \vec{C} \right) = \vec{B} \left( \vec{A} \cdot \vec{C}\right) - \vec{C} \left( \vec{A} \cdot \vec{B} \right)$
  \item $\vec{A} \cdot \left( \vec{B} \cross \vec{C} \right) = \vec{C} \left( \vec{A} \cross \vec{B} \right) = \vec{B} \cdot \left( \vec{C} \cross \vec{A} \right)$
\end{propertylist}
 % Section 1

\section{Kinematics} \label{sec:Kinematics}
\begin{definition}[Kinematics]
  \emph{Kinematics} is a way to describe macroscopic motion with equations.
  This includes anything moving, falling, thrown, shot, launched, etc.
  This forms the fundamental basis for all of classical mechanics.
\end{definition}

\subsection{1-D Kinematics} \label{subsec:1-D Kinematics}
\begin{definition}[1-D Displacement] \label{def:1-D Displacement}
  \emph{One dimensional displacement} is calculated based on the change in position of the `thing.'
  \begin{equation} \label{eq:1-D Displacement}
    s = x_{2} - x_{1}
  \end{equation}
  \begin{remark}
    \emph{Displacement is different than path!}
    Displacement is the change in position of an object.
    Path is the length of the path takes between its starting and end point.
  \end{remark}
\end{definition}

\begin{definition}[1-D Velocity] \label{def:1-D Velocity}
  \emph{One dimensional velocity} is calculated as the displacement per unit time.
  There is instantaneous velocity and average velocity.
  Average velocity is calculated with \Cref{eq:1-D Average Velocity}.
  \begin{equation} \label{eq:1-D Average Velocity}
    v = \frac{\Delta x}{\Delta t} = \frac{x_{2}-x_{1}}{t_{2}-t_{1}}
  \end{equation}
  Instantaneous velocity is calculated by reducing the time interval $\Delta t$ to 0.
  This can be summarized in \Cref{eq:1-D Instantaneous Velocity}.
  \begin{equation} \label{eq:1-D Instantaneous Velocity}
    \begin{aligned}
      v &= \lim\limits_{\Delta t \rightarrow 0} \frac{\Delta x}{\Delta t} \\
      &= \frac{dx}{dt}
    \end{aligned}
  \end{equation}
\end{definition}

\begin{definition}[Acceleration] \label{def:1-D Acceleration}
  \emph{One dimesional acceleration} is the change in velocity over time.
  Again, there is average acceleration and instantaneous acceleration.
  Average acceleration is calculated with \Cref{eq:1-D Average Acceleration}
  \begin{equation} \label{eq:1-D Average Acceleration}
    a = \frac{\Delta v}{\Delta t} = \frac{v_{2} - v_{1}}{t_{2} - t_{1}}
  \end{equation}
  Instantaneous acceleration is calculated by reducing the time interval $\Delta t$ to 0.
  This can be summarized by \Cref{eq:1-D Instantaneous Acceleration}.
  \begin{equation} \label{eq:1-D Instantaneous Acceleration}
    \begin{aligned}
      a &= \lim\limits_{\Delta t \rightarrow 0} \frac{\Delta v}{\Delta t} \\
      &= \frac{dv}{dt} = \frac{d^{2}x}{dt^{2}}
    \end{aligned}
  \end{equation}
\end{definition}

\subsection{Multi-Dimensional Kinematics} \label{Multi-D Kinematics}
Because we can represent a two-dimensional and three-dimensional space in sets, and movement through this space as their respectively dimensioned vectors, we can construct multi-dimensional problems with multi-dimensional vectors!
This is a massive simplification, because instead of solving for one equation with three variables, we can solve three equations for one variable each!!

\textbf{For the following definitions, I have assumed that we are in a 3-dimensional space $(x, y, z)$.}

\begin{definition}[Multi-Dimensional Position] \label{Multi-D Position}
  \emph{Position} in multiple dimensions is done by referring to each of the consituent dimensions.

  \begin{equation} \label{eq:Multi-D Position}
    \vec{s} = \left( x(t), y(t), z(t) \right)
  \end{equation}
\end{definition}

\begin{definition}[Multi-Dimensional Displacement] \label{Multi-D Displacement}
  \emph{Displacement} in multiple dimensions can be broken down into several \nameref{def:1-D Displacement}s.
  Since \nameref{def:1-D Displacement} is calculated as the differnce between the start and end position, the same is true for th emulti-dimensional case.

  \begin{equation} \label{eq:Multi-D Displacement}
    \begin{aligned}
      \vec{r} &= \Delta \vec{s} = \vec{s}_{2} - \vec{s}_{1} \\
      &= \langle x_{2}(t)-x_{1}(t), y_{2}(t)-y_{1}(t), z_{2}(t)-z_{1}(t) \\
      &= \langle r_{x}(t), r_{y}(t), r_{z}(t)
    \end{aligned}
  \end{equation}
\end{definition}

\begin{definition}[Multi-Dimensional Velocity] \label{Multi-D Velocity}
  \emph{Velocity} in multiple dimensions is described in much the same way as \nameref{def:1-D Velocity}.

  \begin{equation} \label{eq:Multi-D Velocity}
    \begin{aligned}
      \vec{v} &= \frac{d \vec{r}}{dt} \\
      &= \biggl \langle \frac{d r_{x}(t)}{dt}, \frac{d r_{y}(t)}{dt}, \frac{d r_{z}(t)}{dt} \biggr \rangle \\
      &= \bigl \langle r_{x}'(t), r_{y}'(t), r_{z}'(t) \bigr \rangle
    \end{aligned}
  \end{equation}
\end{definition}

\begin{definition}[Multi-Dimensional Acceleration] \label{Multi-D Acceleration}
  \emph{Acceleration} in multiple dimensions is described in much the same way as \nameref{def:1-D Acceleration}.

  \begin{equation} \label{eq:Multi-D Acceleration}
    \begin{aligned}
      \vec{a} &= \frac{d \vec{v}}{dt} = \frac{d^{2} \vec{r}}{dt} \\
      &= \biggl \langle \frac{d v_{x}(t)}{dt}, \frac{d v_{y}(t)}{dt}, \frac{d v_{z}(t)}{dt} \biggr \rangle = \biggl \langle \frac{d^{2} r_{x}(t)}{dt}, \frac{d^{2} r_{y}(t)}{dt}, \frac{d^{2} r_{z}(t)}{dt} \biggr \rangle \\
      &= \bigl \langle v_{x}'(t), v_{y}'(t), v_{z}'(t) \bigr \rangle = \bigl \langle r_{x}''(t), r_{y}''(t), r_{z}''(t) \bigr \rangle
    \end{aligned}
  \end{equation}
\end{definition}

\subsection{Projectile Motion} \label{subsec:Projectile Motion}
\begin{definition}[Projectile] \label{def:Projectile}
  A \emph{projectile} is any body given an initial velocity that then follows a path determined by gravity and air resistance.
  \begin{remark}
    However, for most of our calculations, we will neglect air resistance.
    Air resistance can be a difficult thing to calcultate for, especially in the variable cases that we will have.
  \end{remark}
\end{definition}

There are a few things to keep in mind with projectiles in motion.
\begin{enumerate}
  \item Origin is where the projectile starts from
  \item The x-axis is the \emph{distance} that the projectile travels. This is its displacement.
  \item The y-axis is the \emph{height} that the projectile travels.
  \item The end point (landing point) is the only thing that may change on the x-axis.
  \item The acceleration vector is as follows: $\langle 0, -g \rangle$.
  \item \emph{Trajectory} depends on $\vec{v}_{0}$ and $\vec{a}$ \emph{ONLY}.
  \item The two components of the projectile's initial velocity are \emph{independent} ($v_{0,x}, v_{0,y}$).
\end{enumerate}

\subsubsection{Projectile Motion Equations} \label{subsubsec:Projectile Motion Eqns}
The following equations are used to solve for various questions that could be asked about projectile motion.

\paragraph{Initial Velocity Components} \label{par:Projectile Initial Velocity Components}
\begin{equation} \label{eq:Projectile Initial Velocity Components}
  \begin{aligned}
    v_{0,x} &= v_{0} \cos ( \theta ) & v_{0,y} &= v_{0} \sin ( \theta ) \\
  \end{aligned}
\end{equation}

\paragraph{Velocity Components} \label{par:Projectile Velocity Components}
\begin{equation} \label{eq:Projectile Velocity Components}
  \begin{aligned}
    v_{x} &= v_{0,x} \cos ( \theta ) & v_{y} &= v_{0,y} \sin ( \theta ) -gt \\
  \end{aligned}
\end{equation}

\paragraph{Projectile Position} \label{par:Projectile Position}
\begin{equation} \label{eq:Projectile Position}
  \begin{aligned}
    x &= v_{0} t \cos ( \theta ) & y &= v_{0} t \sin ( \theta ) - \frac{1}{2} gt^{2} \\
  \end{aligned}
\end{equation}

\paragraph{Projectile Time} \label{par:Projectile Time}
\begin{equation} \label{eq:Projectile Time}
  \begin{aligned}
    t &= \frac{x}{v_{0} \cos ( \theta )} & t &= \frac{v_{0} \sin ( \theta )}{g}
  \end{aligned}
\end{equation}

\paragraph{Projectile Range} \label{par:Projectile Range}
\begin{equation} \label{eq:Projectile Range}
  R = \frac{v_{0}}{g} \cos ( \theta ) \sin ( \theta )
\end{equation}

\paragraph{Projectile Maximum Range} \label{par:Projectile Maximum Range}
\begin{equation} \label{eq:Projectile Maximum Range}
  R_{\text{Max}} = \frac{v_{0}^{2}}{g}
\end{equation}
This means that the $\theta$ in \Cref{eq:Projectile Range} is \ang{45}.

\paragraph{Projectile Height} \label{par:Projectile Height}
\begin{equation} \label{eq:Projectile Height}
  h = \frac{v_{0}^{2}}{2g} \sin^{2} ( \theta )
\end{equation}

\paragraph{Projectile Maximum Height} \label{par:Projectile Maximum Height}
\begin{equation}
  h = \frac{v_{0}^{2}}{2g}
\end{equation}
The lack of $\sin^{2} ( \theta )$ from \Cref{eq:Projectile Height} means that there is \textbf{\emph{NO}} $y$-component to the velocity, meaning the projectile is at its instant of maximum height.  % Section 2

\section{Uniform Circular Motion} \label{sec:Uniform Circular Motion}
\begin{definition}[Uniform Circular Motion] \label{def:Uniform Circular Motion}
  \emph{Uniform circular motion} is when an object of moving in a perpetual circular motion.
  There is no outside source of acceleration changing the state of the system.
  \begin{remark}
    This does \emph{not} happen in real life.
    However, it is useful for modelling things under ideal conditions that do happen in real life.
  \end{remark}
\end{definition}

\subsection{Angular Part of Circular Motion} \label{subsec:Angular Circular Motion}
During \nameref{def:Uniform Circular Motion}, your terminology changes a little bit.
\begin{definition}[Angular Position] \label{def:Angular Position}
  \emph{Angular position} is determined with radians around a circle.
  It is denoted with \[ \vec{\theta} \]
\end{definition}

\begin{definition}[Angular Velocity] \label{def:Angular Velocity}
  \emph{Angular velocity} is orthogonal to the flat 2-D plane that the object is traveling in.
  It is the dervative of the \nameref{def:Angular Position}.

  \begin{equation} \label{eq:Angular Velocity}
    \begin{aligned}
      \vec{\omega} &= \frac{d \theta}{dt} \\
      &= \langle 0, 0, \omega \rangle \\
    \end{aligned}
  \end{equation}
\end{definition}

\begin{definition}[Angular Acceleration] \label{def:Angular Acceleration}
  \emph{Angular acceleration} is the derivative of the \nameref{def:Angular Velocity}.

  \begin{equation} \label{eq:Angular Acceleration}
    \vec{\alpha} = \frac{d \vec{\omega}}{dt} = \frac{d^{2} \vec{\theta}}{dt}
  \end{equation}

  \begin{remark}
    Note that under \nameref{def:Uniform Circular Motion}, by its very definition, there cannot be any acceleration on the object.
    Therefore, when an object is in uniform circular motion, $\vec{\alpha} = 0$.

    However, when the object is \textbf{\emph{NOT}} in \nameref{def:Uniform Circular Motion} the object is undergoing \nameref{def:Linear Acceleration}.
  \end{remark}
\end{definition}

\subsection{Linear Part of Circular Motion} \label{subsec:Linear Circular Motion}
\begin{definition}[Linear Position] \label{def:Linear Position}
  \emph{Linear position} relates the position of an object from the cartesian coordinate plane to the polar. 
  This means that:

  \begin{equation} \label{eq:Linear Position}
    \begin{aligned}
      x &= r \cos ( \theta ) & y &= r \sin ( \theta )
    \end{aligned}
  \end{equation}
\end{definition}

\begin{definition}[Linear Velocity] \label{def:Linear Velocity}
  \emph{Linear velocity} relates the velocity of an object in a line to its \nameref{def:Angular Velocity}.

  \begin{equation} \label{eq:Linear Velocity}
    \begin{aligned}
      \vec{v} &= \frac{d \vec{r}}{dt} \\
      \frac{d \vec{r}}{dt} &= \biggl \langle \frac{dx}{dt}, \frac{dy}{dt}, 0 \biggr \rangle = \biggl \langle \frac{d}{dt} r \cos ( \theta ), \frac{d}{dt} r \sin ( \theta ), 0 \biggr \rangle \\
      &= \Bigl \langle -r \sin ( \theta ) \omega, r \cos ( \theta ) \omega, 0 \Bigr \rangle \\
      &= \vec{\omega} \cross \vec{r}
    \end{aligned}
  \end{equation}
\end{definition}

\begin{definition}[Linear Acceleration] \label{def:Linear Acceleration}
  \emph{Linear acceleration} is the derivative of \nameref{def:Linear Velocity}.
  It relates the acceleration of an object in a line is relative to its \nameref{def:Angular Acceleration}

  \begin{equation} \label{eq:Linear Acceleration}
    \begin{aligned}
      \frac{d \vec{a}}{dt} &= \frac{d \vec{v}}{dt} \\
      &= \langle -r \omega^{2} \cos ( \theta ), -r \omega^{2} \sin ( \theta ), 0 \rangle = \omega^{2} \langle -r \cos ( \theta ), -r \sin ( \theta ), 0 \rangle \left( \right) \\
      &= - \omega^{2} \vec{r}
    \end{aligned}
  \end{equation}
\end{definition}

\subsection{Relation Between the \nameref{subsec:Angular Circular Motion} and the \nameref{subsec:Linear Circular Motion}} \label{subsec:Relations Circular Motion}
There are a few equations that relate both the \nameref{subsec:Angular Circular Motion} and the \nameref{subsec:Linear Circular Motion}.
\paragraph{Velocity and Angular Velocity} \label{par:Relate Velocity Angular Velocity}
\begin{equation} \label{eq:Relate Velocity Angular Velocity}
  v = \omega r
\end{equation}

\paragraph{Acceleration and Angular Acceleration} \label{par:Relate Acceleration Angular Acceleration}
\begin{equation}
  \begin{aligned}
    a &= \omega^{2} r \\
    &= \frac{v^{2}}{r} \\
  \end{aligned}
\end{equation}
 % Section 3

\section{Reference Frames}\label{sec:Reference Frames}
\begin{definition}[Reference Frames]\label{def:Reference Frames}
  An \emph{inertial reference frame} is a frame for the world.
  It is easiest to think of of an inertial reference frame with an example.
  For instance, when you're in a car going 40 \si{\mph} and you see someone going 45, you can only tell they're going 5 \si{\mph} faster than you.
\end{definition}

\begin{definition}[Galileo Transformation/Relativity Principle]\label{def:Galileo Transformation}
  This is a transformation that happens when you're calculating in one \nameref{def:Reference Frames} and the event is happening in a different \nameref{def:Reference Frames}.

  \begin{equation}\label{eq:Galileo Transformation}
    \begin{aligned}
      \frac{d \vec{R}}{dt} &= \frac{d \vec{r}}{dt} + \frac{d \vec{r}'}{dt} \\
      \vec{V} &= \vec{v} + \vec{v}'
    \end{aligned}
  \end{equation}
\end{definition}
 % Section 4

\section{Newton's Laws} \label{sec:Newtons Laws}
There are 3 fundamental laws of classical mechanics.

\begin{enumerate}[noitemsep, nolistsep]
  \item An object in motion/at reast stays as such, unless acted upon by an outide force(s).
  \item Force is equal to the change in momentum ($p=mv$) per change in time. For a constant mass, force equals mass times acceleration ($\vec{F} = m \vec{a}$).
  \item For every force there is an equal and opposite force.
\end{enumerate}

 % Section 5

%====================================APPENDIX====================================
\appendix
\counterwithin{equation}{section}
\counterwithin{definition}{subsection}

\clearpage
\subsection{Physical Constants} \label{app:Physical Constants}
	\begin{table}[h!]
		\centering
		\begin{tabular}{|c|c|c|}
			\hline
			\textbf{Constant Name} & \textbf{Variable Letter} & \textbf{Value} \\ \hline
			Boltzmann Constant & $R$ & $8.314 \si{\joule / \mole~\kelvin}$ \\ \hline
			Universal Gravitational & $G$ & $6.67408 \times 10^{-11} \si{\meter^{3}~\kilogram^{-1}~\second^{-2}}$ \\ \hline
			Planck's Constant & $h$ & $6.62607004 \times 10^{-34} \si{\meter \kilogram / \second}$ \\ \hline
			Speed of Light & $c$ & $299792458 \si{\meter / \second}$ \\ \hline
			Mass of Earth & $m_{Earth}$ & $5.972 \times 10^{24} \si{\kilogram}$ \\ \hline
			Diameter of Earth & $d_{Earth}$ & $12742 \si{\kilo\meter}$ \\ \hline
		\end{tabular}
	\end{table}

\clearpage
\subsection{Trigonometry} \label{app:Trig}
	\subsubsection{Trigonometric Formulas} \label{subsubsec:Trig Formulas}
		\begin{equation} \label{eq:Sin plus Sin with diff Angles}
			\sin \left( \alpha \right) + \sin \left( \beta \right) = 2 \sin \left( \frac{\alpha + \beta}{2} \right) \cos\left( \frac{\alpha - \beta}{2} \right)  
		\end{equation}
		\begin{equation} \label{eq:Cosine-Sine Product}
			\cos \left( \theta \right) \sin \left( \theta \right) = \frac{1}{2} \sin \left( 2 \theta \right)
		\end{equation}
	
	\subsubsection{Euler Equivalents of Trigonometric Functions} \label{subsubsec:Euler Equivalents}
		\begin{equation} \label{eq:Euler Sin}
			\sin \left( x \right) = \frac{e^{\imath x} + e^{-\imath x}}{2}
		\end{equation}
		\begin{equation} \label{eq:Euler Cos}
			\cos \left( x \right) = \frac{e^{\imath x} - e^{-\imath x}}{2 \imath}
		\end{equation}
		\begin{equation} \label{eq:Euler Sinh}
			\sinh \left( x \right) = \frac{e^{x} - e^{-x}}{2}
		\end{equation}
		\begin{equation} \label{eq:Euler Cosh}
			\cosh \left( x \right) = \frac{e^{x} + e^{-x}}{2}
		\end{equation}

\clearpage
\section{Calculus}\label{app:Calculus}
\subsection{L'Hopital's Rule}\label{subsec:LHopitals_Rule}
L'Hopital's Rule can be used to simplify and solve expressions regarding limits that yield irreconcialable results.
\begin{lemma}[L'Hopital's Rule]\label{lemma:LHopitals_Rule}
  If the equation
  \begin{equation*}
    \lim\limits_{x \rightarrow a} \frac{f(x)}{g(x)} =
    \begin{cases}
      \frac{0}{0} \\
      \frac{\infty}{\infty} \\
    \end{cases}
  \end{equation*}
  then \Cref{eq:LHopitals_Rule} holds.
  \begin{equation}\label{eq:LHopitals_Rule}
    \lim\limits_{x \rightarrow a} \frac{f(x)}{g(x)} = \lim\limits_{x \rightarrow a} \frac{f'(x)}{g'(x)}
  \end{equation}
\end{lemma}

\subsection{Fundamental Theorems of Calculus}\label{subsec:Fundamental Theorem of Calculus}
\begin{definition}[First Fundamental Theorem of Calculus]\label{def:1st Fundamental Theorem of Calculus}
  The \emph{first fundamental theorem of calculus} states that, if $f$ is continuous on the closed interval $\left[ a,b \right]$ and $F$ is the indefinite integral of $f$ on $\left[ a,b \right]$, then

  \begin{equation}\label{eq:1st Fundamental Theorem of Calculus}
    \int_{a}^{b}f \left( x \right) dx = F \left( b \right) - F \left( a \right)
  \end{equation}
\end{definition}

\begin{definition}[Second Fundamental Theorem of Calculus]\label{def:2nd Fundamental Theorem of Calculus}
  The \emph{second fundamental theorem of calculus} holds for $f$ a continuous function on an open interval $I$ and $a$ any point in $I$, and states that if $F$ is defined by

  \begin{equation*}
    F \left( x \right) = \int_{a}^{x} f \left( t \right) dt,
  \end{equation*}
  then
  \begin{equation}\label{eq:2nd Fundamental Theorem of Calculus}
    \begin{aligned}
      \frac{d}{dx} \int_{a}^{x} f \left( t \right) dt &= f \left( x \right) \\
      F' \left( x \right) &= f \left( x \right) \\
    \end{aligned}
  \end{equation}
\end{definition}

\begin{definition}[argmax]\label{def:argmax}
  The arguments to the \emph{argmax} function are to be maximized by using their derivatives.
  You must take the derivative of the function, find critical points, then determine if that critical point is a global maxima.
  This is denoted as
  \begin{equation*}\label{eq:argmax}
    \argmax_{x}
  \end{equation*}
\end{definition}

\subsection{Rules of Calculus}\label{subsec:Rules of Calculus}
\subsubsection{Chain Rule}\label{subsubsec:Chain Rule}
\begin{definition}[Chain Rule]\label{def:Chain Rule}
  The \emph{chain rule} is a way to differentiate a function that has 2 functions multiplied together.

  If
  \begin{equation*}
    f(x) = g(x) \cdot h(x)
  \end{equation*}
  then,
  \begin{equation}\label{eq:Chain Rule}
    \begin{aligned}
      f'(x) &= g'(x) \cdot h(x) + g(x) \cdot h'(x) \\
      \frac{df(x)}{dx} &= \frac{dg(x)}{dx} \cdot g(x) + g(x) \cdot \frac{dh(x)}{dx} \\
    \end{aligned}
  \end{equation}
\end{definition}

\subsection{Useful Integrals}\label{subsec:Useful_Integrals}
\begin{equation}\label{eq:Cosine_Indefinite_Integral}
  \int \cos(x) \; dx = \sin(x)
\end{equation}

\begin{equation}\label{eq:Sine_Indefinite_Integral}
  \int \sin(x) \; dx = -\cos(x)
\end{equation}

\begin{equation}\label{eq:x_Cosine_Indefinite_Integral}
  \int x \cos(x) \; dx = \cos(x) + x \sin(x)
\end{equation}
\Cref{eq:x_Cosine_Indefinite_Integral} simplified with Integration by Parts.

\begin{equation}\label{eq:x_Sine_Indefinite_Integral}
  \int x \sin(x) \; dx = \sin(x) - x \cos(x)
\end{equation}
\Cref{eq:x_Sine_Indefinite_Integral} simplified with Integration by Parts.

\begin{equation}\label{eq:x_Squared_Cosine_Indefinite_Integral}
  \int x^{2} \cos(x) \; dx = 2x \cos(x) + (x^{2} - 2) \sin(x)
\end{equation}
\Cref{eq:x_Squared_Cosine_Indefinite_Integral} simplified by using Integration by Parts twice.

\begin{equation}\label{eq:x_Squared_Sine_Indefinite_Integral}
  \int x^{2} \sin(x) \; dx = 2x \sin(x) - (x^{2} - 2) \cos(x)
\end{equation}
\Cref{eq:x_Squared_Sine_Indefinite_Integral} simplified by using Integration by Parts twice.

\begin{equation}\label{eq:Exponential_Cosine_Indefinite_Integral}
  \int e^{\alpha x} \cos(\beta x) \; dx = \frac{e^{\alpha x} \bigl( \alpha \cos(\beta x) + \beta \sin(\beta x) \bigr)}{\alpha^{2} + \beta^{2}} + C
\end{equation}

\begin{equation}\label{eq:Exponential_Sine_Indefinite_Integral}
  \int e^{\alpha x} \sin(\beta x) \; dx = \frac{e^{\alpha x} \bigl( \alpha \sin(\beta x) - \beta \cos(\beta x) \bigr)}{\alpha^{2}+\beta^{2}} + C
\end{equation}

\begin{equation}\label{eq:Exponential_Indefinite_Integral}
  \int e^{\alpha x} \; dx = \frac{e^{\alpha x}}{\alpha}
\end{equation}

\begin{equation}\label{eq:x_Exponential_Indefinite_Integral}
  \int x e^{\alpha x} \; dx = e^{\alpha x} \left( \frac{x}{\alpha} - \frac{1}{\alpha^{2}} \right)
\end{equation}
\Cref{eq:x_Exponential_Indefinite_Integral} simplified with Integration by Parts.

\begin{equation}\label{eq:Inverse_x_Indefinite_Integral}
  \int \frac{dx}{\alpha + \beta x} = \int \frac{1}{\alpha + \beta x} \; dx = \frac{1}{\beta} \ln (\alpha + \beta x)
\end{equation}

\begin{equation}\label{eq:Inverse_x_Squared_Indefinite_Integral}
  \int \frac{dx}{\alpha^{2} + \beta^{2} x^{2}} = \int \frac{1}{\alpha^{2} + \beta^{2} x^{2}} \; dx = \frac{1}{\alpha \beta} \arctan \left( \frac{\beta x}{\alpha} \right)
\end{equation}

\begin{equation}\label{eq:a_Exponential_Indefinite_Integral}
  \int \alpha^{x} \; dx = \frac{\alpha^{x}}{\ln(\alpha)}
\end{equation}

\begin{equation}\label{eq:a_Exponential_Derivative}
  \frac{d}{dx} \alpha^{x} = \frac{d\alpha^{x}}{dx} = \alpha^{x} \ln(x)
\end{equation}

\subsection{Leibnitz's Rule}\label{subsec:Leibnitzs_Rule}
\begin{lemma}[Leibnitz's Rule]\label{lemma:Leibnitzs_Rule}
  Given
  \begin{equation*}
    g(t) = \int_{a(t)}^{b(t)} f(x, t) \, dx
  \end{equation*}
  with $a(t)$ and $b(t)$ differentiable in $t$ and $\frac{\partial f(x, t)}{\partial t}$ continuous in both $t$ and $x$, then
  \begin{equation}\label{eq:Leibnitzs_Rule}
    \frac{d}{dt} g(t) = \frac{d g(t)}{dt} = \int_{a(t)}^{b(t)} \frac{\partial f(x, t)}{\partial t} \, dx + f \bigl[ b(t), t \bigr] \, \frac{d b(t)}{dt} - f \bigl[ a(t), t \bigr] \, \frac{d a(t)}{dt}
  \end{equation}
\end{lemma}



\clearpage
\section{Complex Numbers}\label{sec:Complex_Numbers}
\begin{definition}[Complex Number]\label{def:Complex_Number}
  A \emph{complex number} is a hyper real number system.
  This means that two real numbers, $a, b \in \RealNumbers$, are used to construct the set of complex numbers, denoted $\ComplexNumbers$.

  A complex number is written, in Cartesian form, as shown in \Cref{eq:Complex_Number} below.
  \begin{equation}\label{eq:Complex_Number}
    z = a \pm ib
  \end{equation}
  where
  \begin{equation}\label{eq:Imaginary_Value}
    i = \sqrt{-1}
  \end{equation}

  \begin{remark*}[$i$ vs. $j$ for Imaginary Numbers]
    Complex numbers are generally denoted with either $i$ or $j$.
    Electrical engineering regularly makes use of $j$ as the imaginary value.
    This is because alternating current $i$ is already taken, so $j$ is used as the imaginary value instad.
  \end{remark*}
\end{definition}

\subsection{Parts of a Complex Number}\label{subsec:Complex_Number_Parts}
A \nameref{def:Complex_Number} is made of up 2 parts:
\begin{enumerate}[noitemsep]
\item \nameref{def:Real_Part}
\item \nameref{def:Imaginary_Part}
\end{enumerate}

\begin{definition}[Real Part]\label{def:Real_Part}
  The \emph{real part} of an imaginary number, denoted with the $\Re$ operator, is the portion of the \nameref{def:Complex_Number} with no part of the imaginary value $i$ present.

  If $z = x + iy$, then
  \begin{equation}\label{eq:Real_Part}
    \Real{z} = x
  \end{equation}

  \begin{remark}[Alternative Notation]\label{rmk:Real_Part_Alternative_Notation}
    The \nameref{def:Real_Part} of a number sometimes uses a slightly different symbol for denoting the operation.
    It is:
    \begin{equation*}
      \mathfrak{Re}
    \end{equation*}
  \end{remark}
\end{definition}

\begin{definition}[Imaginary Part]\label{def:Imaginary_Part}
  The \emph{imaginary part} of an imaginary number, denoted with the $\Im$ operator, is the portion of the \nameref{def:Complex_Number} where the imaginary value $i$ is present.

  If $z = x + iy$, then
  \begin{equation}\label{eq:Imaginary_Part}
    \Imag{z} = y
  \end{equation}

  \begin{remark}[Alternative Notation]\label{rmk:Imaginary_Part_Alternative_Notation}
    The \nameref{def:Imaginary_Part} of a number sometimes uses a slightly different symbol for denoting the operation.
    It is:
    \begin{equation*}
      \mathfrak{Im}
    \end{equation*}
  \end{remark}
\end{definition}

\subsection{Binary Operations}\label{subsec:Binary_Operations}

%%% Local Variables:
%%% mode: latex
%%% TeX-master: shared
%%% End:


\subsection{Complex Conjugates}\label{app:Complex_Conjugates}
\begin{definition}[Complex Conjugate]\label{def:Complex_Conjugate}
  The conjugate of a complex number is called its \emph{complex conjugate}.
  The complex conjugate of a complex number is the number with an equal real part and an imaginary part equal in magnitude but opposite in sign.
  If we have a complex number as shown below,
  \begin{equation*}
    z = a \pm bi
  \end{equation*}

  then, the conjugate is denoted and calculated as shown below.
  \begin{equation}\label{eq:Complex_Conjugates}
    \Conjugate{z} = a \mp bi
  \end{equation}
\end{definition}

The \nameref{def:Complex_Conjugate} can also be denoted with an asterisk ($*$).
This is generally done for complex functions, rather than single variables.
\begin{equation}\label{eq:Complex_Conjugates_Asterisk}
  z^{*} = \Conjugate{z}
\end{equation}

%%% Local Variables:
%%% mode: latex
%%% TeX-master: shared
%%% End:


\subsection{Geometry of Complex Numbers}\label{subsec:Geometry_Complex_Numbers}
So far, we have viewed \nameref{def:Complex_Number}s only algebraically.
However, we can also view them geometrically as points on a 2 dimensional \nameref{def:Argand_Plane}.

\begin{definition}[Argand Plane]\label{def:Argand_Plane}
  An \emph{Argane Plane} is a standard two dimensional plane whose points are all elements of the complex numbers, $z \in \ComplexNumbers$.
  This is taken from Descarte's definition of a completely real plane.

  The Argand plane contains 2 lines that form the axes, that indicate the real component and the imaginary component of the complex number specified.
\end{definition}

A \nameref{def:Complex_Number} can be viewed as a point in the \nameref{def:Argand_Plane}, where the \nameref{def:Real_Part} is the ``$x$''-component and the \nameref{def:Imaginary_Part} is the ``$y$''-component.

By plotting this, you see that we form a right triangle, so we can find the hypotenuse of that triangle.
This hypotenuse is the distance the point $p$ is from the origin, refered to as the \nameref{def:Complex_Number_Modulus}.
\begin{remark*}
  When working with \nameref{def:Complex_Number}s geometrically, we refer to the points, where they are defined like so:
  \begin{equation*}
    z = x + iy = p(x, y)
  \end{equation*}

  Note that $p$ is \textbf{not} a function of $x$ and $y$.
  Those are the values that inform us \textbf{where} $p$ is located on the \nameref{def:Argand_Plane}.
\end{remark*}

\subsubsection{Modulus of a Complex Number}\label{subsubsec:Complex_Number_Modulus}
\begin{definition}[Modulus]\label{def:Complex_Number_Modulus}
  The \emph{modulus} of a \nameref{def:Complex_Number} is the distance from the origin to the complex point $p$.
  This is based off the Pythagorean Theorem.
  \begin{equation}\label{eq:Complex_Number_Modulus}
    \begin{aligned}
      {\lvert z \rvert}^{2} = x^{2} + y^{2} &= z \Conjugate{z} \\
      \lvert z \rvert &= \sqrt{x^{2} + y^{2}}
    \end{aligned}
  \end{equation}
\end{definition}

\begin{propertylist}
\item The \emph{Law of Moduli} states that $\lvert z w \rvert = \lvert z \rvert \lvert w \rvert$.\label{prop:Law_of_Moduli}.
\end{propertylist}

We can prove \Cref{prop:Law_of_Moduli} using an algebraic identity.
\begin{proof}[Prove \Cref*{prop:Law_of_Moduli}]
  Let $z$ and $w$ be complex numbers ($z, w \in \ComplexNumbers$).
  We are asked to prove
  \begin{equation*}
    \lvert z w \rvert = \lvert z \rvert \lvert w \rvert
  \end{equation*}

  But, it is actually easier to prove
  \begin{equation*}
    {\lvert z w \rvert}^{2} = {\lvert z \rvert}^{2} {\lvert w \rvert}^{2}
  \end{equation*}

  We start by simplifying the ${\lvert z w \rvert}^{2}$ equation above.
  \begin{align*}
    {\lvert z w \rvert}^{2} &= {\lvert z \rvert}^{2} {\lvert w \rvert}^{2} \\
    \intertext{Using the definition of the \nameref{def:Complex_Number_Modulus} of a \nameref{def:Complex_Number} in \Cref{eq:Complex_Number_Modulus}, we can expand the modulus.}
                            &= (z w) (\Conjugate{z w}) \\
    \intertext{Using \Cref{prop:Complex_Conjugate_Split} for multiplication allows us to do the next step.}
                            &= (z w) (\Conjugate{z} \Conjugate{w}) \\
    \intertext{Using Multiplicative Associativity and Multiplicative Commutativity, we can simplify this further.}
                            &= (z \Conjugate{z}) (w \Conjugate{w}) \\
                            &= {\lvert z \rvert}^{2} {\lvert w \rvert}^{2}
  \end{align*}

  Note how we never needed to define $z$ or $w$, so this is as general a result as possible.
\end{proof}

\paragraph{Algebraic Effects of the Modulus' \Cref*{prop:Law_of_Moduli}}\label{par:Law_of_Moduli-Algebraic_Effects}
For this section, let $z = x_{1} + iy_{1}$ and $w = x_{2} + iy_{2}$.
Now,
\begin{align*}
  z w &= (x_{1}x_{2} - y_{1}y_{2}) + i(x_{1}y_{2} + x_{2}y_{1}) \\
  {\lvert z w \rvert}^{2} &= {(x_{1}x_{2} - y_{1}y_{2})}^{2} + {(x_{1}y_{2} + x_{2}y_{1})}^{2} \\
      &= \left( x_{1}^{2} + x_{2}^{2} \right) \left( x_{2}^{2} + y_{2}^{2} \right) \\
      &= {\lvert z \rvert}^{2} {\lvert w \rvert}^{2}
\end{align*}

However, the Law of Moduli (\Cref{prop:Law_of_Moduli}) does \textbf{not} hold for a hyper complex number system one that uses 2 or more imaginaries, i.e.\ $z = a + iy + jz$.
But, the Law of Moduli (\Cref{prop:Law_of_Moduli}) \textbf{does} hold for hyper complex number system that uses 3 imaginaries, $a = z + iy + jz + k \ell$.

\paragraph{Conceptual Effects of the Modulus' \Cref*{prop:Law_of_Moduli}}\label{par:Law_of_Moduli-Conceptual_Effects}
We are interested in seeing if $\lvert z w \rvert = (x_{1}^{2} + y_{1}^{2})(x_{2}^{2}+y_{2}^{2})$ can be extended to more complex terms (3 terms in the complex number).

However, Langrange proved that the equation below \textbf{always} holds.
Note that the $z$ below has no relation to the $z$ above.
\begin{equation*}
  (x_{1} + y_{1} + z_{1}) \neq X^{2} + Y^{2} + Z^{2}
\end{equation*}

%%% Local Variables:
%%% mode: latex
%%% TeX-master: shared
%%% End:


\subsection{Circles and Complex Numbers}\label{subsec:Circles_Complex_Numbers}
We need to define both a center and a radius, just like with regular purely real values.
\Cref{eq:Circles_Complex_Numbers} defines the relation required for a circle using \nameref{def:Complex_Number}s.
\begin{equation}\label{eq:Circles_Complex_Numbers}
  \lvert z - a \rvert = r
\end{equation}

\begin{example}[Lecture 2, Example 1]{Convert to Circle}
  Given the expression below, find the location of the center of the circle and the radius of the circle?
  \begin{equation*}
    \lvert 5 iz + 10 \rvert = 7
  \end{equation*}
  \tcblower{}
  This is just a matter of simplification and moving terms around.
  \begin{align*}
    \lvert 5 iz + 10 \rvert &= 7 \\
    \lvert 5i (z + \frac{10}{5i}) \rvert &= 7 \\
    \lvert 5i (z + \frac{2}{i}) \rvert &= 7 \\
    \lvert 5i (z + \frac{2}{i} \frac{-i}{-i}) \rvert &= 7 \\
    \lvert 5i (z - 2i) \rvert &= 7 \\
    \intertext{Now using the Law of Moduli (\Cref{prop:Law_of_Moduli}) $\lvert a b \rvert = \lvert a \rvert \lvert b \rvert$, we can simplify out the extra imaginary term.}
    \lvert 5i \rvert \lvert z-2i \rvert &= 7 \\
    5 \lvert z - 2i \rvert &= 7 \\
    \lvert z - 2i \rvert = \frac{7}{5}
  \end{align*}

  Thus, the circle formed by the equation $\lvert 5 iz + 10 \rvert = 7$ is actually $\lvert z - 2i \rvert = \frac{7}{5}$, with a center at $a = 2i$ and a radius of $\frac{7}{5}$.
\end{example}

\subsubsection{Annulus}\label{subsubsec:Annulus}
\begin{definition}[Annulus]\label{def:Annulus}
  An \emph{annulus} is a region that is bounded by 2 concentric circles.
  This takes the form of \Cref{eq:Annulus}.
  \begin{equation}\label{eq:Annulus}
    r_{1} \leq \lvert z - a \rvert \leq r_{2}
  \end{equation}

  In \Cref{eq:Annulus}, each of the $\leq$ symbols could also be replaced with $<$.
  This leads to 3 different possibilities for the annulus:
  \begin{enumerate}[noitemsep]
  \item If both inequality symbols are $\leq$, then it is a \textbf{Closed Annulus}.
  \item If both inequality symbols are $<$, then it is an \textbf{Open Annulus}.
  \item If \textbf{only one} inequality symbol $<$ and the other $\leq$, then it is not an \textbf{Open Annulus}.
  \end{enumerate}
\end{definition}


%%% Local Variables:
%%% mode: latex
%%% TeX-master: shared
%%% End:



%%% Local Variables:
%%% mode: latex
%%% TeX-master: shared
%%% End:


\end{document}