\documentclass[10pt,letterpaper,final,twoside,notitlepage]{article}
\usepackage[margin=.5in]{geometry}
\usepackage[utf8]{inputenc}
\usepackage[english]{babel}
\usepackage{amsmath}
\usepackage{amsfonts}
\usepackage{amssymb}
\usepackage{amsthm} % Gives us plain, definition, and remark to use in \theoremstyle{style}
\usepackage{graphicx}

\usepackage{hyperref} % Generate hyperlinks to referenced items
\usepackage[noabbrev,nameinlink]{cleveref} % Fancy cross-references in the document everywhere
\usepackage{nameref} % Can make references by name to places
\usepackage{subcaption} % Allows for multiple figures in one Figure environment
\usepackage{siunitx} % Gives us ways to typeset units for stuff
\usepackage{enumitem} % Provides [noitemsep, nolistsep] for more compact lists
\usepackage{chngcntr} % Allows us to tamper with the counter a little more
\usepackage{empheq} % Allow boxing of equations in special math environments
\usepackage{tcolorbox} % Allows us to create boxes of various types for examples
\usepackage{tikz} % Allows us to create TikZ and PGF Pictures
\usetikzlibrary{trees}
%\usepackage{ctable} % Greater control over tables and how they look

\graphicspath{{./Drawings/Phys_123}} % Uncomment this to use pictures in this document
\counterwithin{equation}{section} % Uncomment to number eqns with sec nums too

\theoremstyle{plain}
\newtheorem{theorem}{Theorem}
\counterwithin{theorem}{section}

\theoremstyle{definition}
\newtheorem{definition}{Defn}
\newtheorem{corollary}{Corollary}[section]

\theoremstyle{remark}
\newtheorem{remark}{Remark}[definition]
\newtheorem*{remark*}{Remark}
%\counterwithin{definition}{subsection} % Uncomment to have definitions use section.subsection numbering

% Create a special list that handles properties. It can be broken and restarted
\newlist{propertylist}{enumerate}{1} % {Name}{Template}{Max Depth}
\setlist[propertylist, 1]{label=\textbf{(\roman*)}, noitemsep, nolistsep} % Set options

% Create a special list that handles enumerate starting with lower letters. Breakable/Restartable.
\newlist{boldalphlist}{enumerate}{1} % {Name}{Template}{Max Depth}
\setlist[boldalphlist, 1]{label=\textbf{(\alph*)}, noitemsep, nolistsep} % Set options

% Create a tcolorbox for examples
% Argument #1 is optional, given by [], that is the textbook's problem number
% Argument #2 is mandatory, given by {}, that is the title for the example
\newtcolorbox[auto counter,
number within=section,
number format=\arabic,
crefname={example}{examples}, % Define reference format for cref (No Capitals)
Crefname={Example}{Examples}, % Reference format for cleveref (With Capitals)
]{example}[2][]{ % The [2][] Means the first argument is optional
	width=\textwidth,
	title={Example \thetcbcounter: #2. #1},
	fonttitle=\bfseries,
	label={ex:#2},
	nameref=#2,
	colbacktitle=white!100!black,
	coltitle=black!100!white,
	colback=white!100!black,
	upperbox=visible,
	lowerbox=visible,
	sharp corners=all
}

% Redefine the 'end of proof' symbol to be a black square, not blank
\renewcommand\qedsymbol{$\blacksquare$} % Change proofs to have black square at end

\DeclareSIUnit\mile{mi}
\DeclareSIUnit{\mph}{mph}

\DeclareMathOperator{\cross}{\times}
\DeclareMathOperator{\RealNums}{\mathbb{R}}
\DeclareMathOperator*{\argmax}{argmax} % Thin Space and subscripts are UNDER in display

\begin{titlepage}
  \title{Phys 123: Classical Mechanics - Reference Sheet}
  \author{Karl Hallsby}
  \date{Last Edited: \today}
\end{titlepage}

\begin{document}
\pagenumbering{gobble}
\maketitle
\pagenumbering{roman} % i, ii, iii on beginning pages, that don't have content
\tableofcontents
\clearpage
\pagenumbering{arabic} % 1,2,3 on content pages

\section{Vectors}\label{sec:Vectors}
\begin{definition}[Vector]\label{def:Vector}
  A \emph{vector} is a way to show both magnitude of displacement and direction of displacement.
  Vectors are drawn as rays.
  \begin{remark}
    \nameref{def:Vector}s and \nameref{def:Scalar}s may seem similar, but are different.
  \end{remark}
\end{definition}

\begin{definition}[Scalar]\label{def:Scalar}
  A \emph{scalar} is a way to show \emph{\textbf{ONLY}} the magnitude of a displacement, without any direction information.
\end{definition}

\subsection{Vector Properties}\label{subsec:Vector Properties}
\begin{propertylist}
  \item $\vec{A} + \vec{B} = \vec{C}$
  \item $\vec{0} = \langle 0, 0, 0, \ldots, 0 \rangle$
  \item $\vec{A} + \vec{0} = \vec{A}$
  \item $\vec{A} + -\vec{A} = \vec{0}$
  \item $\left( \vec{A} + \vec{B} \right) + \vec{C} = \vec{A} + \left( \vec{B} + \vec{C} \right)$
  \item $\vec{A} + \vec{B} = \vec{B} + \vec{A}$
  \item Magnitude of vector: $\lVert \vec{A} \rVert = \sqrt{A_{x}^{2} + A_{y}^{2} + A_{z}^{2}}$
\end{propertylist}

\subsubsection{Getting Components}\label{subsubsec:Vector Components}
Getting the components of a vector involves solving the imaginary pythagorean triangle around the vector.

For a 2-dimensional vector, $\vec{V}$, you have the components $\langle V_{x}, V_{y} \rangle$.
You find their values with this equation:
\begin{equation}\label{eq:Vector Components}
  \begin{aligned}
    V_{x} &= V \cos \theta \\
    V_{y} &= V \sin \theta
  \end{aligned}
\end{equation}

\subsubsection{3D Unit Vectors}\label{subsubsec:3D Unit Vectors}
3-dimensional vectors shouldn't be any too crazy by this point.
They are just another variable that can be thrown around in the vector.
However, the three \nameref{subsubsec:3D Unit Vectors} are special.
You can also use these to describe any lower-dimensional vector as well.
\begin{equation}
  \begin{aligned}
    \hat{\imath} &= \langle 1, 0, 0 \rangle \\
    \hat{\jmath} &= \langle 0, 1, 0 \rangle \\
    \hat{k} &= \langle 0, 0, 1 \rangle
  \end{aligned}
\end{equation}

\subsubsection{Addition}\label{subsubsec:Vector Addition}
Vectors are additive, and are done from head-to-tail.
This means that
\begin{equation}\label{eq:Vector Addition}
  \vec{A} + \vec{B} = \vec{C}
\end{equation}

This means that in 3-dimensional vectors, they are added like this:
\begin{equation}\label{eq:3D Vector Addition}
  \begin{aligned}
    \vec{A} &= \langle A_{x}, A_{y}, A_{z} \rangle \\
    \vec{B} &= \langle B_{x}, B_{y}, B_{z} \rangle \\
    \vec{A} + \vec{B} &= \langle A_{x}+B_{x}, A_{y}+B_{y}, A_{z}+B_{z} \rangle
  \end{aligned}
\end{equation}

\subsubsection{Scalar Multiplication}\label{subsubsec:Scalar Vector Multiplication}
When applying multiplication between a scalar and a vector, you perform \nameref{subsubsec:Scalar Vector Multiplication}.
\begin{equation}\label{eq:Scalar Vector Multiplication}
  2 \times \vec{V} = 2 \langle V_{x}, V_{y} \rangle = \langle 2V_{x}, 2V_{y}, 2V_{z} \rangle
\end{equation}

This means that you do \emph{\textbf{NOT}} modify the direction of the vector, you only change its magnitude.

\subsubsection{Scalar (Dot) Product}\label{subsubsec:Dot Product}
The \nameref{subsubsec:Dot Product} is the first of two ways to multiply 2 vectors.
The other is the \nameref{subsubsec:Cross Product}.
There are 2 ways to calculate the \nameref{subsubsec:Dot Product}.

The first involves using the magnitudes of each vector and multiplying those by the cosine of the angle between them.
\begin{equation}\label{eq:Dot Product Magnitudes}
  \vec{A} \cdot \vec{B} = \lVert \vec{A} \rVert \lVert \vec{B} \rVert \cos \left( \theta \right)
\end{equation}

The second is done by adding the product of each component of each vector.
\begin{equation}\label{eq:Dot Product Components}
  \vec{A} \cdot \vec{B} = A_{x}B_{x} + A_{y}B_{y} + A_{z}B_{z}
\end{equation}

\begin{remark*}
  This means that when you apply the \nameref{subsubsec:Dot Product} to 2 vectors, you return a \nameref{def:Scalar}.
\end{remark*}

\paragraph{Properties of \nameref{subsubsec:Dot Product}}\label{par:Dot Product Properties}
\begin{propertylist}
  \item $( \vec{A} )^{2} = \vec{A} \cdot \vec{A}$
  \item $\vec{A} \cdot \vec{B} = \vec{B} \cdot \vec{A}$
\end{propertylist}

\subsubsection{Vector (Cross) Product}\label{subsubsec:Cross Product}
The \nameref{subsubsec:Cross Product} is the second of two ways to multiply 2 vectors.
The other is the \nameref{subsubsec:Dot Product}.
There are 2 ways to calculate the \nameref{subsubsec:Cross Product}.

The first involves using the magnitudes of each vector and multiplying those by the sine of the angle between them.
\begin{equation}\label{eq:Cross Product Magnitudes}
  \vec{A} \cross \vec{B} = \lVert \vec{A} \rVert \lVert \vec{B} \rVert \sin \theta
\end{equation}

The second is done by taking the determinant of a $2 \times 2$ or $3 \times 3$ matrix.
\begin{equation}\label{eq:Cross Product Determinant}
  \begin{aligned}
    \vec{A} \cross \vec{B}
    &= \det \begin{bmatrix}
      \hat{\imath} & \hat{\jmath} & \hat{k} \\
      A_{x} & A_{y} & A_{z} \\
      B_{x} & B_{y} & B_{z}
    \end{bmatrix} \\
    &= \begin{vmatrix}
      \hat{\imath} & \hat{\jmath} & \hat{k} \\
      A_{x} & A_{y} & A_{z} \\
      B_{x} & B_{y} & B_{z}
    \end{vmatrix} \\
    &= \left( A_{y}B_{z}-A_{z}B_{y} \right) \hat{\imath} - \left( A_{x}B_{z}-A_{z}B_{x} \right) \hat{\jmath} + \left( A_{x}B_{y} - A_{y}B_{x} \right) \hat{k} \\
    &= \bigl\langle A_{y}B_{z}-A_{z}B_{y}, - \left( A_{x}B_{z}+A_{z}B_{x} \right), A_{x}B_{y}-A_{y}B_{x} \bigr\rangle
  \end{aligned}
\end{equation}

\begin{remark*}
  This means that when you apply the \nameref{subsubsec:Cross Product} to 2 vectors, you return a \nameref{def:Vector}.
\end{remark*}

\paragraph{Properties of \nameref{subsubsec:Cross Product}}\label{par:Cross Product Properties}
\begin{propertylist}
  \item $\vec{A} \cross \vec{A} = \vec{0}$
  \item $\vec{A} \cross \vec{B} = - \left( \vec{B} \cross \vec{A} \right)$
  \item $\vec{A} \cross \left( \vec{B} \cross \vec{C} \right) = \vec{B} \left( \vec{A} \cdot \vec{C}\right) - \vec{C} \left( \vec{A} \cdot \vec{B} \right)$
  \item $\vec{A} \cdot \left( \vec{B} \cross \vec{C} \right) = \vec{C} \left( \vec{A} \cross \vec{B} \right) = \vec{B} \cdot \left( \vec{C} \cross \vec{A} \right)$
\end{propertylist}
 % Section 1

\section{Kinematics}\label{sec:Kinematics}
\begin{definition}[Kinematics]\label{def:Kinematics}
  \emph{Kinematics} is a way to describe macroscopic motion with equations.
  This includes anything moving, falling, thrown, shot, launched, etc.
  This forms the fundamental basis for all of classical mechanics.
\end{definition}

\subsection{1-D Kinematics}\label{subsec:1-D Kinematics}
\begin{definition}[1-D Displacement]\label{def:1-D Displacement}
  \emph{One dimensional displacement} is calculated based on the change in position of the `thing.'
  \begin{equation}\label{eq:1-D Displacement}
    s = x_{2} - x_{1}
  \end{equation}
  \begin{remark}
    \emph{Displacement is different than path!}
    Displacement is the change in position of an object.
    Path is the length of the path takes between its starting and end point.
  \end{remark}
\end{definition}

\begin{definition}[1-D Velocity]\label{def:1-D Velocity}
  \emph{One dimensional velocity} is calculated as the displacement per unit time.
  There is instantaneous velocity and average velocity.
  Average velocity is calculated with \Cref{eq:1-D Average Velocity}.
  \begin{equation}\label{eq:1-D Average Velocity}
    v = \frac{\Delta x}{\Delta t} = \frac{x_{2}-x_{1}}{t_{2}-t_{1}}
  \end{equation}
  Instantaneous velocity is calculated by reducing the time interval $\Delta t$ to 0.
  This can be summarized in \Cref{eq:1-D Instantaneous Velocity}.
  \begin{equation}\label{eq:1-D Instantaneous Velocity}
    \begin{aligned}
      v &= \lim\limits_{\Delta t \rightarrow 0} \frac{\Delta x}{\Delta t} \\
      &= \frac{dx}{dt}
    \end{aligned}
  \end{equation}
\end{definition}

\begin{definition}[Acceleration]\label{def:1-D Acceleration}
  \emph{One dimesional acceleration} is the change in velocity over time.
  Again, there is average acceleration and instantaneous acceleration.
  Average acceleration is calculated with \Cref{eq:1-D Average Acceleration}
  \begin{equation}\label{eq:1-D Average Acceleration}
    a = \frac{\Delta v}{\Delta t} = \frac{v_{2} - v_{1}}{t_{2} - t_{1}}
  \end{equation}
  Instantaneous acceleration is calculated by reducing the time interval $\Delta t$ to 0.
  This can be summarized by \Cref{eq:1-D Instantaneous Acceleration}.
  \begin{equation}\label{eq:1-D Instantaneous Acceleration}
    \begin{aligned}
      a &= \lim\limits_{\Delta t \rightarrow 0} \frac{\Delta v}{\Delta t} \\
      &= \frac{dv}{dt} = \frac{d^{2}x}{dt^{2}}
    \end{aligned}
  \end{equation}
\end{definition}

\subsection{Multi-Dimensional Kinematics}\label{Multi-D Kinematics}
Because we can represent a two-dimensional and three-dimensional space in sets, and movement through this space as their respectively dimensioned vectors, we can construct multi-dimensional problems with multi-dimensional vectors!
This is a massive simplification, because instead of solving for one equation with three variables, we can solve three equations for one variable each!!

\textbf{For the following definitions, I have assumed that we are in a 3-dimensional space $(x, y, z)$.}

\begin{definition}[Multi-Dimensional Position]\label{Multi-D Position}
  \emph{Position} in multiple dimensions is done by referring to each of the consituent dimensions.

  \begin{equation}\label{eq:Multi-D Position}
    \vec{s} = \left( x(t), y(t), z(t) \right)
  \end{equation}
\end{definition}

\begin{definition}[Multi-Dimensional Displacement]\label{Multi-D Displacement}
  \emph{Displacement} in multiple dimensions can be broken down into several \nameref{def:1-D Displacement}s.
  Since \nameref{def:1-D Displacement} is calculated as the differnce between the start and end position, the same is true for th emulti-dimensional case.

  \begin{equation}\label{eq:Multi-D Displacement}
    \begin{aligned}
      \vec{r} &= \Delta \vec{s} = \vec{s}_{2} - \vec{s}_{1} \\
      &= \langle x_{2}(t)-x_{1}(t), y_{2}(t)-y_{1}(t), z_{2}(t)-z_{1}(t) \\
      &= \langle r_{x}(t), r_{y}(t), r_{z}(t)
    \end{aligned}
  \end{equation}
\end{definition}

\begin{definition}[Multi-Dimensional Velocity]\label{Multi-D Velocity}
  \emph{Velocity} in multiple dimensions is described in much the same way as \nameref{def:1-D Velocity}.

  \begin{equation}\label{eq:Multi-D Velocity}
    \begin{aligned}
      \vec{v} &= \frac{d \vec{r}}{dt} \\
      &= \biggl \langle \frac{d r_{x}(t)}{dt}, \frac{d r_{y}(t)}{dt}, \frac{d r_{z}(t)}{dt} \biggr \rangle \\
      &= \bigl \langle r_{x}'(t), r_{y}'(t), r_{z}'(t) \bigr \rangle
    \end{aligned}
  \end{equation}
\end{definition}

\begin{definition}[Multi-Dimensional Acceleration]\label{Multi-D Acceleration}
  \emph{Acceleration} in multiple dimensions is described in much the same way as \nameref{def:1-D Acceleration}.

  \begin{equation}\label{eq:Multi-D Acceleration}
    \begin{aligned}
      \vec{a} &= \frac{d \vec{v}}{dt} = \frac{d^{2} \vec{r}}{dt} \\
      &= \biggl \langle \frac{d v_{x}(t)}{dt}, \frac{d v_{y}(t)}{dt}, \frac{d v_{z}(t)}{dt} \biggr \rangle = \biggl \langle \frac{d^{2} r_{x}(t)}{dt}, \frac{d^{2} r_{y}(t)}{dt}, \frac{d^{2} r_{z}(t)}{dt} \biggr \rangle \\
      &= \bigl \langle v_{x}'(t), v_{y}'(t), v_{z}'(t) \bigr \rangle = \bigl \langle r_{x}''(t), r_{y}''(t), r_{z}''(t) \bigr \rangle
    \end{aligned}
  \end{equation}
\end{definition}

\subsection{Projectile Motion}\label{subsec:Projectile Motion}
\begin{definition}[Projectile]\label{def:Projectile}
  A \emph{projectile} is any body given an initial velocity that then follows a path determined by gravity and air resistance.
  \begin{remark}
    However, for most of our calculations, we will neglect air resistance.
    Air resistance can be a difficult thing to calcultate for, especially in the variable cases that we will have.
  \end{remark}
\end{definition}

There are a few things to keep in mind with projectiles in motion.
\begin{enumerate}
  \item Origin is where the projectile starts from
  \item The x-axis is the \emph{distance} that the projectile travels. This is its displacement.
  \item The y-axis is the \emph{height} that the projectile travels.
  \item The end point (landing point) is the only thing that may change on the x-axis.
  \item The acceleration vector is as follows: $\langle 0, -g \rangle$.
  \item \emph{Trajectory} depends on $\vec{v}_{0}$ and $\vec{a}$ \emph{ONLY}.
  \item The two components of the projectile's initial velocity are \emph{independent} ($v_{0,x}, v_{0,y}$).
\end{enumerate}

\subsubsection{Projectile Motion Equations}\label{subsubsec:Projectile Motion Eqns}
The following equations are used to solve for various questions that could be asked about projectile motion.

\paragraph{Initial Velocity Components}\label{par:Projectile Initial Velocity Components}
\begin{equation}\label{eq:Projectile Initial Velocity Components}
  \begin{aligned}
    v_{0,x} &= v_{0} \cos ( \theta ) & v_{0,y} &= v_{0} \sin ( \theta ) \\
  \end{aligned}
\end{equation}

\paragraph{Velocity Components}\label{par:Projectile Velocity Components}
\begin{equation}\label{eq:Projectile Velocity Components}
  \begin{aligned}
    v_{x} &= v_{0,x} \cos ( \theta ) & v_{y} &= v_{0,y} \sin ( \theta ) -gt \\
  \end{aligned}
\end{equation}

\paragraph{Projectile Position}\label{par:Projectile Position}
\begin{equation}\label{eq:Projectile Position}
  \begin{aligned}
    x &= v_{0} t \cos ( \theta ) & y &= v_{0} t \sin ( \theta ) - \frac{1}{2} gt^{2} \\
  \end{aligned}
\end{equation}

\paragraph{Projectile Time}\label{par:Projectile Time}
\begin{equation}\label{eq:Projectile Time}
  \begin{aligned}
    t &= \frac{x}{v_{0} \cos ( \theta )} & t &= \frac{v_{0} \sin ( \theta )}{g}
  \end{aligned}
\end{equation}

\paragraph{Projectile Range}\label{par:Projectile Range}
\begin{equation}\label{eq:Projectile Range}
  R = \frac{v_{0}}{g} \cos ( \theta ) \sin ( \theta )
\end{equation}

\paragraph{Projectile Maximum Range}\label{par:Projectile Maximum Range}
\begin{equation}\label{eq:Projectile Maximum Range}
  R_{\text{Max}} = \frac{v_{0}^{2}}{g}
\end{equation}
This means that the $\theta$ in \Cref{eq:Projectile Range} is \ang{45}.

\paragraph{Projectile Height}\label{par:Projectile Height}
\begin{equation}\label{eq:Projectile Height}
  h = \frac{v_{0}^{2}}{2g} \sin^{2} ( \theta )
\end{equation}

\paragraph{Projectile Maximum Height}\label{par:Projectile Maximum Height}
\begin{equation}
  h = \frac{v_{0}^{2}}{2g}
\end{equation}
The lack of $\sin^{2} ( \theta )$ from \Cref{eq:Projectile Height} means that there is \textbf{\emph{NO}} $y$-component to the velocity, meaning the projectile is at its instant of maximum height.

%%% Local Variables:
%%% mode: latex
%%% TeX-master: "../Phys_123-Reference_Sheet"
%%% End:
  % Section 2

\section{Uniform Circular Motion} \label{sec:Uniform Circular Motion}
\begin{definition}[Uniform Circular Motion] \label{def:Uniform Circular Motion}
  \emph{Uniform circular motion} is when an object of moving in a perpetual circular motion.
  There is no outside source of acceleration changing the state of the system.
  \begin{remark}
    This does \emph{not} happen in real life.
    However, it is useful for modelling things under ideal conditions that do happen in real life.
  \end{remark}
\end{definition}
 % Section 3

\section{Reference Frames}\label{sec:Reference Frames}
\begin{definition}[Reference Frames]\label{def:Reference Frames}
  An \emph{inertial reference frame} is a frame for the world.
  It is easiest to think of of an inertial reference frame with an example.
  For instance, when you're in a car going 40 \si{\mph} and you see someone going 45, you can only tell they're going 5 \si{\mph} faster than you.
\end{definition}

\begin{definition}[Galileo Transformation/Relativity Principle]\label{def:Galileo Transformation}
  This is a transformation that happens when you're calculating in one \nameref{def:Reference Frames} and the event is happening in a different \nameref{def:Reference Frames}.

  \begin{equation}\label{eq:Galileo Transformation}
    \begin{aligned}
      \frac{d \vec{R}}{dt} &= \frac{d \vec{r}}{dt} + \frac{d \vec{r'}}{dt} \\
      \vec{V} &= \vec{v} + \vec{v'}
    \end{aligned}
  \end{equation}
\end{definition}

%%% Local Variables:
%%% mode: latex
%%% TeX-master: "../Phys_123-Mechanics-Reference_Sheet"
%%% End:
 % Section 4

\section{Newton's Laws} \label{sec:Newtons Laws}
There are 3 fundamental laws of classical mechanics.

\begin{enumerate}[noitemsep, nolistsep]
  \item An object in motion/at reast stays as such, unless acted upon by an outide force(s).
  \item Force is equal to the change in momentum ($p=mv$) per change in time. For a constant mass, force equals mass times acceleration ($\vec{F} = m \vec{a}$).
  \item For every force there is an equal and opposite force.
\end{enumerate}

There are a few forces that are fundamental in our universe.

\begin{itemize}[noitemsep, nolistsep]
  \item Gravitational Force
  \item Magnetic Force
  \item Electric Force
  \item Frictional Force
  \item Normal Force
  \item Tension Force
\end{itemize}

\begin{remark*}
  \textbf{\emph{REMEMBER}} that forces are vectors!!
\end{remark*}

To solve any force problem, you should construct a free-body diagram.

\subsection{Newton's Laws in 1-D} \label{subsec:Newtons Laws 1-D}
\begin{example}[]{Atwood Machine}
  Find the acceleration that is the result of dropping one side of the machine?
  Neglect any friction that may occur because of the pulley.

  \tcblower

  \begin{align*}
    m_{2} \vec{a}_{2} &= m_{2} \vec{g} + \vec{T}_{2} \\
    m_{1} \vec{a}_{1} &= m_{1} \vec{g} + \vec{T}_{1}
  \end{align*}

  One thing to remember is that both masses will have the same acceleration, because the pulley has no friction.
  So because $\vec{a}_{2} = \vec{a}_{1}$, we can solve for a single $\vec{a}$ variable.

  \begin{align*}
    m_{2} \vec{a} &= m_{2} \vec{g} + \vec{T}_{2} \\
    m_{1} \vec{a} &= m_{1} \vec{g} + \vec{T}_{1} \\
    \vec{a} \left( m_{2} + m_{1} \right) &= \vec{g} \left( m_{2} - m_{1} \right) \\
    \vec{a} &= \frac{g \left( m_{2} - m_{1} \right)}{m_{2}+m_{1}}
  \end{align*}
\end{example}

\subsection{Newton's Laws in Multi-D} \label{subsec:Newtons Laws Multi-D}
For a multi-dimensional situation, you break the free-body problem's vectors down into their components.

\begin{example}[]{Sliding Block}
  If a block of mass $m$ slides down a ramp of angle $\theta$, what is the coefficient of static and kinetic friction?

  \tcblower

\end{example}
 % Section 5

\section{Dynamics of Circular Motion} \label{sec:Dynamics Circular Motion} % Section 6

\section{Springs} \label{sec:Springs}

\subsection{Hooke's Law} \label{subsec:Hookes Law} % Section 7

\section{Energy} \label{sec:Energy}

\subsection{Kinetic Energy} \label{subsec:Kinetic Energy}

\subsection{Potential Energy} \label{subsec:Potential Energy}

\subsection{Conservation of Energy} \label{subsec:Conservation of Energy} % Section 8

\section{Systems of Particles} \label{sec:Systems of Particles}

\subsection{Center of Mass} \label{subsec:Center of Mass} % Section 9

\section{Gravitation} \label{sec:Gravitation}
\begin{definition}[Gravity] \label{def:Gravity}
  \emph{Gravity} is one of the 4 fundamental forces in the universe.
  It is also one of the least understood.
  However, it it the thing that is keeping us ``glued'' to this plant.
\end{definition}
 % Section ??

\section{Statics}\label{sec:Statics}
\begin{definition}[Statics]\label{def:Statics}
  \emph{Statics} is the case when an object is at rest.
  It is when the sum of all forces in $x$, $y$, and $z$ axes is equal to 0.

  \begin{itemize}[noitemsep, nolistsep]
    \item \nameref{def:1-D Acceleration} is 0, which means $\vec{F}_{\text{Net}} = \vec{0}$. This means that there is no linear motion.
    \item \nameref{def:Angular Acceleration} is 0, which means $\vec{\tau}_{\text{Net}} = \vec{0}$. This means that there is no rotational motion.
    \item The Center of Gravity is the \nameref{def:Center of Mass}.
  \end{itemize}

  To solve any \nameref{sec:Statics} problem, you simply have to solve for \Cref{eq:Statics}.
  The net force and net \nameref{def:Torque} both must sum to 0.

  \begin{equation}\label{eq:Statics}
    \begin{aligned}
      \sum\limits_{i=1}^{n} \vec{F}_{i} &= \vec{0} \\
      \sum\limits_{i=1}^{n} \vec{\tau}_{i} &= \vec{0}
    \end{aligned}
  \end{equation}
\end{definition}
 % Section ??

\section{The 4 Fundamental Interactions in the Universe}\label{sec:Fundamental Interactions}
There are 4 fundamental interactions present in our universe, as we know it right now.
\begin{enumerate}[noitemsep, nolistsep]
  \item \nameref{def:Gravity}
  \item Electromagnetic
  \item ``Weak'' (Neutrinos and the like)
  \item ``Strong'' (Holds atomic nucleus together)
\end{enumerate}

%%% Local Variables:
%%% mode: latex
%%% TeX-master: "../Phys_123-Reference_Sheet"
%%% End: % Section ?? (Extra Section)

%====================================APPENDIX====================================
\appendix
\counterwithin{equation}{section}
\counterwithin{definition}{subsection}

\clearpage
\subsection{Physical Constants} \label{app:Physical Constants}
	\begin{table}[h!]
		\centering
		\begin{tabular}{|c|c|c|}
			\hline
			\textbf{Constant Name} & \textbf{Variable Letter} & \textbf{Value} \\ \hline
			Boltzmann Constant & $R$ & $8.314 \si{\joule / \mole~\kelvin}$ \\ \hline
			Universal Gravitational & $G$ & $6.67408 \times 10^{-11} \si{\meter^{3}~\kilogram^{-1}~\second^{-2}}$ \\ \hline
			Planck's Constant & $h$ & $6.62607004 \times 10^{-34} \si{\meter \kilogram / \second}$ \\ \hline
			Speed of Light & $c$ & $299792458 \si{\meter / \second}$ \\ \hline
			Mass of Earth & $m_{Earth}$ & $5.972 \times 10^{24} \si{\kilogram}$ \\ \hline
			Diameter of Earth & $d_{Earth}$ & $12742 \si{\kilo\meter}$ \\ \hline
		\end{tabular}
	\end{table}

\clearpage
\subsection{Trigonometry} \label{app:Trig}
	\subsubsection{Trigonometric Formulas} \label{subsubsec:Trig Formulas}
		\begin{equation} \label{eq:Sin plus Sin with diff Angles}
			\sin \left( \alpha \right) + \sin \left( \beta \right) = 2 \sin \left( \frac{\alpha + \beta}{2} \right) \cos\left( \frac{\alpha - \beta}{2} \right)  
		\end{equation}
		\begin{equation} \label{eq:Cosine-Sine Product}
			\cos \left( \theta \right) \sin \left( \theta \right) = \frac{1}{2} \sin \left( 2 \theta \right)
		\end{equation}

\clearpage
\subsection{Calculus} \label{app:Calculus}
	\subsubsection{Fundamental Theorems of Calculus} \label{subsubsec:Fundamental Theorem of Calculus}
		\begin{definition}[First Fundamental Theorem of Calculus] \label{def:1st Fundamental Theorem of Calculus}
			The \emph{first fundamental theorem of calculus} states that, if $f$ is continuous on the closed interval $\left[ a,b \right]$ and $F$ is the indefinite integral of $f$ on $\left[ a,b \right]$, then 
			\begin{equation} \label{eq:1st Fundamental Theorem of Calculus}
				\int_{a}^{b}f \left( x \right) dx = F \left( b \right) - F \left( a \right)
			\end{equation}
		\end{definition}
		\begin{definition}[Second Fundamental Theorem of Calculus] \label{def:2nd Fundamental Theorem of Calculus}
			The \emph{second fundamental theorem of calculus} holds for $f$ a continuous function on an open interval $I$ and $a$ any point in $I$, and states that if $F$ is defined by
			\begin{equation*}
				F \left( x \right) = \int_{a}^{x} f \left( t \right) dt,
			\end{equation*}
			then
			\begin{equation} \label{eq:2nd Fundamental Theorem of Calculus}
				\begin{aligned}
					\frac{d}{dx} \int_{a}^{x} f \left( t \right) dt &= f \left( x \right) \\
					F' \left( x \right) &= f \left( x \right) \\
				\end{aligned}
			\end{equation}
		\end{definition}

\clearpage
\section{Complex Numbers}
	\begin{equation} \label{eq:Exponential to Rectangular}
		A e^{-ix} = A \left[ \cos \left( x \right) + i\sin \left( x \right) \right]
	\end{equation}


\end{document}