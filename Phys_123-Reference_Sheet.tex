\documentclass[10pt,letterpaper,final,twoside,notitlepage]{article}
\usepackage{referencesheet}

%\graphicspath{{./Drawings/Phys_123}} % Uncomment this to use pictures in this document
\counterwithin{equation}{section} % Uncomment this to number equations with section numbers too

\author{Karl Hallsby}
\title{Phys 123 Equations}

\begin{document}
\pagenumbering{roman} % i, ii, iii on beginning pages, that don't have content
\tableofcontents
\clearpage
\pagenumbering{arabic} % 1,2,3 on content pages

\section{General Information} \label{sec:General Info}
\subsection{Vectors} \label{subsec:Vectors}

\section{Kinematics} \label{sec:Kinematics}
\begin{definition}[Kinematics]
  \emph{Kinematics} is a way to describe macroscopic motion with equations.
  This includes anything moving, falling, thrown, shot, launched, etc.
  This forms the fundamental basis for all of classical mechanics.
\end{definition}

\subsection{1-D Kinematics} \label{subsec:1D Kinematics}
\begin{definition}[1-D Displacement] \label{def:1D Displacement}
  \emph{One dimensional displacement} is calculated based on the change in position of the `thing.'
  \begin{equation} \label{eq:1D Displacement}
    s = x_{2} - x_{1}
  \end{equation}
  \begin{remark}
    \emph{Displacement is different than path!}
    Displacement is the change in position of an object.
    Path is the length of the path takes between its starting and end point.
  \end{remark}
\end{definition}

\begin{definition}[1-D Velocity] \label{def:1D Velocity}
  \emph{One dimensional velocity} is calculated as the displacement per unit time.
  There is instantaneous velocity and average velocity.
  Average velocity is calculated with \Cref{eq:1D Average Velocity}.
  \begin{equation} \label{eq:1D Average Velocity}
    v = \frac{\Delta x}{\Delta t} = \frac{x_{2}-x_{1}}{t_{2}-t_{1}}
  \end{equation}
  Instantaneous velocity is calculated by reducing the time interval $\Delta t$ to 0.
  This can be summarized in \Cref{eq:1D Instantaneous Velocity}.
  \begin{equation} \label{eq:1D Instantaneous Velocity}
    v = \lim\limits_{\Delta t \rightarrow 0} \frac{\Delta x}{\Delta t} = \frac{dx}{dt}
  \end{equation}
\end{definition}

\begin{definition}[Acceleration] \label{def:1D Acceleration}
  \emph{One dimesional acceleration} is the change in velocity over time.
  Again, there is average acceleration and instantaneous acceleration.
  Average acceleration is calculated with \Cref{eq:1D Average Acceleration}
  \begin{equation} \label{eq:1D Average Acceleration}
    a = \frac{\Delta v}{\Delta t} = \frac{v_{2} - v_{1}}{t_{2} - t_{1}}
  \end{equation}
  Instantaneous acceleration is calculated by reducing the time interval $\Delta t$ to 0.
  This can be summarized by \Cref{eq:1D Instantaneous Acceleration}.
  \begin{equation} \label{eq:1D Instantaneous Acceleration}
    a = \lim\limits_{\Delta t \rightarrow 0} \frac{\Delta v}{\Delta t} = \frac{dv}{dt} = \frac{d^{2}x}{dt^{2}}
  \end{equation}
\end{definition}
    

\end{document}