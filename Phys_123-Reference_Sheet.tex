\documentclass[10pt,letterpaper,final,twoside,notitlepage]{article}
\usepackage[margin=.5in]{geometry}
\usepackage[utf8]{inputenc}
\usepackage[english]{babel}
\usepackage{amsmath}
\usepackage{amsfonts}
\usepackage{amssymb}
\usepackage{amsthm} % Gives us plain, definition, and remark to use in \theoremstyle{style}
\usepackage{graphicx}

\usepackage{hyperref} % Generate hyperlinks to referenced items
\usepackage[noabbrev,nameinlink]{cleveref} % Fancy cross-references in the document everywhere
\usepackage{nameref} % Can make references by name to places
\usepackage{subcaption} % Allows for multiple figures in one Figure environment
\usepackage{siunitx} % Gives us ways to typeset units for stuff
\usepackage{enumitem} % Provides [noitemsep, nolistsep] for more compact lists
\usepackage{chngcntr} % Allows us to tamper with the counter a little more
\usepackage{empheq} % Allow boxing of equations in special math environments
\usepackage{tcolorbox} % Allows us to create boxes of various types for examples
\usepackage{tikz} % Allows us to create TikZ and PGF Pictures
\usetikzlibrary{trees}
%\usepackage{ctable} % Greater control over tables and how they look

\graphicspath{{./Drawings/Phys_123}} % Uncomment this to use pictures in this document
\counterwithin{equation}{section} % Uncomment to number eqns with sec nums too

\theoremstyle{plain}
\newtheorem{theorem}{Theorem}
\counterwithin{theorem}{section}

\theoremstyle{definition}
\newtheorem{definition}{Defn}
\newtheorem{corollary}{Corollary}[section]

\theoremstyle{remark}
\newtheorem{remark}{Remark}[definition]
\newtheorem*{remark*}{Remark}
%\counterwithin{definition}{subsection} % Uncomment to have definitions use section.subsection numbering

% Create a special list that handles properties. It can be broken and restarted
\newlist{propertylist}{enumerate}{1} % {Name}{Template}{Max Depth}
\setlist[propertylist, 1]{label=\textbf{(\roman*)}, noitemsep, nolistsep} % Set options

% Create a special list that handles enumerate starting with lower letters. Breakable/Restartable.
\newlist{boldalphlist}{enumerate}{1} % {Name}{Template}{Max Depth}
\setlist[boldalphlist, 1]{label=\textbf{(\alph*)}, noitemsep, nolistsep} % Set options

% Create a tcolorbox for examples
% Argument #1 is optional, given by [], that is the textbook's problem number
% Argument #2 is mandatory, given by {}, that is the title for the example
\newtcolorbox[auto counter,
number within=section,
number format=\arabic,
crefname={example}{examples}, % Define reference format for cref (No Capitals)
Crefname={Example}{Examples}, % Reference format for cleveref (With Capitals)
]{example}[2][]{ % The [2][] Means the first argument is optional
	width=\textwidth,
	title={Example \thetcbcounter: #2. #1},
	fonttitle=\bfseries,
	label={ex:#2},
	nameref=#2,
	colbacktitle=white!100!black,
	coltitle=black!100!white,
	colback=white!100!black,
	upperbox=visible,
	lowerbox=visible,
	sharp corners=all
}

% Redefine the 'end of proof' symbol to be a black square, not blank
\renewcommand\qedsymbol{$\blacksquare$} % Change proofs to have black square at end

\DeclareMathOperator{\RealNums}{\mathbb{R}}
\DeclareMathOperator*{\argmax}{argmax} % Thin Space and subscripts are UNDER in display

\begin{titlepage}
  \title{Phys 123: Classical Mechanics - Reference Sheet}
  \author{Karl Hallsby}
  \date{Last Edited: \today}
\end{titlepage}

\begin{document}
\pagenumbering{gobble}
\maketitle
\pagenumbering{roman} % i, ii, iii on beginning pages, that don't have content
\tableofcontents
\clearpage
\pagenumbering{arabic} % 1,2,3 on content pages

\section{General Information} \label{sec:General Info}
\subsection{Vectors} \label{subsec:Vectors}

\section{Kinematics} \label{sec:Kinematics}
\begin{definition}[Kinematics]
  \emph{Kinematics} is a way to describe macroscopic motion with equations.
  This includes anything moving, falling, thrown, shot, launched, etc.
  This forms the fundamental basis for all of classical mechanics.
\end{definition}

\subsection{1-D Kinematics} \label{subsec:1D Kinematics}
\begin{definition}[1-D Displacement] \label{def:1D Displacement}
  \emph{One dimensional displacement} is calculated based on the change in position of the `thing.'
  \begin{equation} \label{eq:1D Displacement}
    s = x_{2} - x_{1}
  \end{equation}
  \begin{remark}
    \emph{Displacement is different than path!}
    Displacement is the change in position of an object.
    Path is the length of the path takes between its starting and end point.
  \end{remark}
\end{definition}

\begin{definition}[1-D Velocity] \label{def:1D Velocity}
  \emph{One dimensional velocity} is calculated as the displacement per unit time.
  There is instantaneous velocity and average velocity.
  Average velocity is calculated with \Cref{eq:1D Average Velocity}.
  \begin{equation} \label{eq:1D Average Velocity}
    v = \frac{\Delta x}{\Delta t} = \frac{x_{2}-x_{1}}{t_{2}-t_{1}}
  \end{equation}
  Instantaneous velocity is calculated by reducing the time interval $\Delta t$ to 0.
  This can be summarized in \Cref{eq:1D Instantaneous Velocity}.
  \begin{equation} \label{eq:1D Instantaneous Velocity}
    v = \lim\limits_{\Delta t \rightarrow 0} \frac{\Delta x}{\Delta t} = \frac{dx}{dt}
  \end{equation}
\end{definition}

\begin{definition}[Acceleration] \label{def:1D Acceleration}
  \emph{One dimesional acceleration} is the change in velocity over time.
  Again, there is average acceleration and instantaneous acceleration.
  Average acceleration is calculated with \Cref{eq:1D Average Acceleration}
  \begin{equation} \label{eq:1D Average Acceleration}
    a = \frac{\Delta v}{\Delta t} = \frac{v_{2} - v_{1}}{t_{2} - t_{1}}
  \end{equation}
  Instantaneous acceleration is calculated by reducing the time interval $\Delta t$ to 0.
  This can be summarized by \Cref{eq:1D Instantaneous Acceleration}.
  \begin{equation} \label{eq:1D Instantaneous Acceleration}
    a = \lim\limits_{\Delta t \rightarrow 0} \frac{\Delta v}{\Delta t} = \frac{dv}{dt} = \frac{d^{2}x}{dt^{2}}
  \end{equation}
\end{definition}

%====================================APPENDIX====================================
\appendix
\counterwithin{equation}{section}
\counterwithin{definition}{subsection}

\clearpage
\subsection{Physical Constants} \label{app:Physical Constants}
	\begin{table}[h!]
		\centering
		\begin{tabular}{|c|c|c|}
			\hline
			\textbf{Constant Name} & \textbf{Variable Letter} & \textbf{Value} \\ \hline
			Boltzmann Constant & $R$ & $8.314 \si{\joule / \mole~\kelvin}$ \\ \hline
			Universal Gravitational & $G$ & $6.67408 \times 10^{-11} \si{\meter^{3}~\kilogram^{-1}~\second^{-2}}$ \\ \hline
			Planck's Constant & $h$ & $6.62607004 \times 10^{-34} \si{\meter \kilogram / \second}$ \\ \hline
			Speed of Light & $c$ & $299792458 \si{\meter / \second}$ \\ \hline
			Mass of Earth & $m_{Earth}$ & $5.972 \times 10^{24} \si{\kilogram}$ \\ \hline
			Diameter of Earth & $d_{Earth}$ & $12742 \si{\kilo\meter}$ \\ \hline
		\end{tabular}
	\end{table}

\clearpage
\subsection{Trigonometry} \label{app:Trig}
	\subsubsection{Trigonometric Formulas} \label{subsubsec:Trig Formulas}
		\begin{equation} \label{eq:Sin plus Sin with diff Angles}
			\sin \left( \alpha \right) + \sin \left( \beta \right) = 2 \sin \left( \frac{\alpha + \beta}{2} \right) \cos\left( \frac{\alpha - \beta}{2} \right)  
		\end{equation}
		\begin{equation} \label{eq:Cosine-Sine Product}
			\cos \left( \theta \right) \sin \left( \theta \right) = \frac{1}{2} \sin \left( 2 \theta \right)
		\end{equation}

\clearpage
\subsection{Calculus} \label{app:Calculus}
	\subsubsection{Fundamental Theorems of Calculus} \label{subsubsec:Fundamental Theorem of Calculus}
		\begin{definition}[First Fundamental Theorem of Calculus] \label{def:1st Fundamental Theorem of Calculus}
			The \emph{first fundamental theorem of calculus} states that, if $f$ is continuous on the closed interval $\left[ a,b \right]$ and $F$ is the indefinite integral of $f$ on $\left[ a,b \right]$, then 
			\begin{equation} \label{eq:1st Fundamental Theorem of Calculus}
				\int_{a}^{b}f \left( x \right) dx = F \left( b \right) - F \left( a \right)
			\end{equation}
		\end{definition}
		\begin{definition}[Second Fundamental Theorem of Calculus] \label{def:2nd Fundamental Theorem of Calculus}
			The \emph{second fundamental theorem of calculus} holds for $f$ a continuous function on an open interval $I$ and $a$ any point in $I$, and states that if $F$ is defined by
			\begin{equation*}
				F \left( x \right) = \int_{a}^{x} f \left( t \right) dt,
			\end{equation*}
			then
			\begin{equation} \label{eq:2nd Fundamental Theorem of Calculus}
				\begin{aligned}
					\frac{d}{dx} \int_{a}^{x} f \left( t \right) dt &= f \left( x \right) \\
					F' \left( x \right) &= f \left( x \right) \\
				\end{aligned}
			\end{equation}
		\end{definition}

\clearpage
\section{Complex Numbers}
	\begin{equation} \label{eq:Exponential to Rectangular}
		A e^{-ix} = A \left[ \cos \left( x \right) + i\sin \left( x \right) \right]
	\end{equation}


\end{document}