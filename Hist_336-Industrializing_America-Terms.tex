\documentclass[10pt,letterpaper,final,twoside,notitlepage]{article}
\usepackage[margin=.5in]{geometry} % 1/2 inch margins on all pages
\usepackage[utf8]{inputenc} % Define the input encoding
\usepackage[USenglish]{babel} % Define language used
\usepackage{amsmath,amsfonts,amssymb}
\usepackage{amsthm} % Gives us plain, definition, and remark to use in \theoremstyle{style}
\usepackage{mathtools} % Allow for text and math in align* environment.
\usepackage{thmtools}
\usepackage{thm-restate}
\usepackage{graphicx}

\usepackage[
backend=biber,
style=alphabetic,
citestyle=authoryear]{biblatex} % Must include citation somewhere in document to print bibliography
\usepackage{hyperref} % Generate hyperlinks to referenced items
\usepackage[nottoc]{tocbibind} % Prints the Reference/Bibliography in TOC as well
\usepackage[noabbrev,nameinlink]{cleveref} % Fancy cross-references in the document everywhere
\usepackage{nameref} % Can make references by name to places
\usepackage{caption} % Allows for greater control over captions in figure, algorithm, table, etc. environments
\usepackage{subcaption} % Allows for multiple figures in one Figure environment
\usepackage[binary-units=true]{siunitx} % Gives us ways to typeset units for stuff
\usepackage{csquotes} % Context-sensitive quotation facilities
\usepackage{enumitem} % Provides [noitemsep, nolistsep] for more compact lists
\usepackage{chngcntr} % Allows us to tamper with the counter a little more
\usepackage{empheq} % Allow boxing of equations in special math environments
\usepackage[x11names]{xcolor} % Gives access to coloring text in environments or just text, MUST be before tikz
\usepackage{tcolorbox} % Allows us to create boxes of various types for examples
\usepackage{tikz} % Allows us to create TikZ and PGF Pictures
\usepackage{ctable} % Greater control over tables and how they look
\usepackage{diagbox} % Allow us to have shared diagonal cells in tables
\usepackage{multirow} % Allow us to have a single cell in a table span multiple rows
\usepackage{titling} % Put document information throughout the document programmatically
\usepackage[linesnumbered,ruled,vlined]{algorithm2e} % Allows us to write algorithms in a nice style.

\counterwithin{figure}{section}
\counterwithin{table}{section}
\counterwithin{equation}{section}
\counterwithin{algocf}{section}
\crefname{algocf}{algorithm}{algorithms}
\Crefname{algocf}{Algorithm}{Algorithms}
\setcounter{secnumdepth}{4}
\setcounter{tocdepth}{4} % Include \paragraph in toc
\crefname{paragraph}{paragraph}{paragraphs}
\Crefname{paragraph}{Paragraph}{Paragraphs}

% Create a theorem environment
\theoremstyle{plain}
\newtheorem{theorem}{Theorem}[section]
% Create a numbered theorem-like environment for lemmas
\newtheorem{lemma}{Lemma}[theorem]

% Create a definition environment
\theoremstyle{definition}
\newtheorem{definition}{Defn}
\newtheorem{corollary}{Corollary}[section]
% \begin{definition}[Term] \label{def:}
%   Make sure the term is emphasized with \emph{term}.
%   This ensures that if \emph is changed, it shows up everywhere
% \end{definition}

% Create a numbered remark environment numbered based on definition
% NOTE: This version of remark MUST go inside a definition environment
\theoremstyle{remark}
\newtheorem{remark}{Remark}[definition]
%\counterwithin{definition}{subsection} % Uncomment to have definitions use section.subsection numbering

% Create an unnumbered remark environment for general use
% NOTE: This version of remark has NO restrictions on placement
\newtheorem*{remark*}{Remark}

% Create a special list that handles properties. It can be broken and restarted
\newlist{propertylist}{enumerate}{1} % {Name}{Template}{Max Depth}
% [newlistname, LevelsToApplyTo]{formatting options}
\setlist[propertylist, 1]{label=\textbf{(\roman*)}, ref=\textbf{(\roman*)}, noitemsep, nolistsep}
\crefname{propertylisti}{property}{properties}
\Crefname{propertylisti}{Property}{Properties}

% Create a special list that handles enumerate starting with lower letters. Breakable/Restartable.
\newlist{boldalphlist}{enumerate}{1} % {Name}{Template}{Max Depth}
% [newlistname, LevelsToApplyTo]{formatting options}
\setlist[boldalphlist, 1]{label=\textbf{(\alph*)}, ref=\alph*, noitemsep, nolistsep} % Set options

\newlist{nocrefenumerate}{enumerate}{1} % {Name}{Template}{Max Depth}
% [newlistname, LevelsToApplyTo]{formatting options}
\setlist[nocrefenumerate, 1]{label=(\arabic*), ref=(\arabic*), noitemsep, nolistsep}

% Create a list that allows for deeper nesting of numbers. By default enumerate only allows depth=4.
\newlist{nestednums}{enumerate}{6}
% [newlistname, LevelsToApplyTo]{formatting options}
\setlist[nestednums]{noitemsep, label*=\arabic*.}

\tcbuselibrary{breakable} % Allow tcolorboxes to be broken across pages
% Create a tcolorbox for examples
% /begin{example}[extra name]{NAME}
% Create a tcolorbox for examples
% Argument #1 is optional, given by [], that is the textbook's problem number
% Argument #2 is mandatory, given by {}, that is the title for the example
% Avoid putting special characters, (), [], {}, ",", etc. in the title.
\newtcolorbox[auto counter,
number within=section,
number format=\arabic,
crefname={example}{examples}, % Define reference format for cref (No Capitals)
Crefname={Example}{Examples}, % Reference format for cleveref (With Capitals)
]{example}[2][]{ % The [2][] Means the first argument is optional
  width=\textwidth,
  title={Example \thetcbcounter: #2. #1}, % Parentheses and commas are not well supported
  fonttitle=\bfseries,
  label={ex:#2},
  nameref=#2,
  colbacktitle=white!100!black,
  coltitle=black!100!white,
  colback=white!100!black,
  upperbox=visible,
  lowerbox=visible,
  sharp corners=all,
  breakable
}

% Create a tcolorbox for general use
\newtcolorbox[% auto counter,
% number within=section,
% number format=\arabic,
% crefname={example}{examples}, % Define reference format for cref (No Capitals)
% Crefname={Example}{Examples}, % Reference format for cleveref (With Capitals)
]{blackbox}{
  width=\textwidth,
  % title={},
  fonttitle=\bfseries,
  % label={},
  % nameref=,
  colbacktitle=white!100!black,
  coltitle=black!100!white,
  colback=white!100!black,
  upperbox=visible,
  lowerbox=visible,
  sharp corners=all
}

% Redefine the 'end of proof' symbol to be a black square, not blank
\renewcommand{\qedsymbol}{$\blacksquare$} % Change proofs to have black square at end

% Common Mathematical Stuff
\newcommand{\Abs}[1]{\ensuremath{\lvert #1 \rvert}}
\newcommand{\DNE}{\ensuremath{\mathrm{DNE}}} % Used when limit of function Does Not Exist

% Complex Numbers functions
\renewcommand{\Re}{\operatorname{Re}} % Redefine to use the command, but not the fraktur version
\renewcommand{\Im}{\operatorname{Im}} % Redefine to use the command, but not the fraktur version
\newcommand{\Real}[1]{\ensuremath{\Re \lbrace #1 \rbrace}}
\newcommand{\Imag}[1]{\ensuremath{\Im \lbrace #1 \rbrace}}
\newcommand{\Conjugate}[1]{\ensuremath{\overline{#1}}}
\newcommand{\Modulus}[1]{\ensuremath{\lvert #1 \rvert}}
\DeclareMathOperator{\PrincipalArg}{\ensuremath{Arg}}

% Math Operators that are useful to abstract the written math away to one spot
% Number Sets
\DeclareMathOperator{\RealNumbers}{\ensuremath{\mathbb{R}}}
\DeclareMathOperator{\AllIntegers}{\ensuremath{\mathbb{Z}}}
\DeclareMathOperator{\PositiveInts}{\ensuremath{\mathbb{Z}^{+}}}
\DeclareMathOperator{\NegativeInts}{\ensuremath{\mathbb{Z}^{-}}}
\DeclareMathOperator{\NaturalNumbers}{\ensuremath{\mathbb{N}}}
\DeclareMathOperator{\ComplexNumbers}{\ensuremath{\mathbb{C}}}
\DeclareMathOperator{\RationalNumbers}{\ensuremath{\mathbb{Q}}}

% Calculus operators
\DeclareMathOperator*{\argmax}{argmax} % Thin Space and subscripts are UNDER in display

% Signal and System Functions
\DeclareMathOperator{\UnitStep}{\mathcal{U}}
\DeclareMathOperator{\sinc}{sinc} % sinc(x) = (sin(pi x)/(pi x))

% Transformations
\DeclareMathOperator{\Lapl}{\mathcal{L}} % Declare a Laplace symbol to be used

% Logical Operators
\DeclareMathOperator{\XOR}{\oplus}

% x86 CPU Registers
\newcommand{\rbpRegister}{\texttt{\%rbp}}
\newcommand{\rspRegister}{\texttt{\%rsp}}
\newcommand{\ripRegister}{\texttt{\%rip}}
\newcommand{\raxRegister}{\texttt{\%rax}}
\newcommand{\rbxRegister}{\texttt{\%rbx}}

%%% Local Variables:
%%% mode: latex
%%% TeX-master: shared
%%% End:


% These packages are more specific to certain documents, but will be availabe in the template
% \usepackage{esint} % Provides us with more types of integral symbols to use
% \usepackage[outputdir=./TeX_Output]{minted} % Allow us to nicely typeset 300+ programming languages
% \crefname{lstlisting}{listing}{listings}
% \Crefname{lstlisting}{Listing}{Listings}
% This document must be compiled with the -shell-escape flag if the packages above are uncommented

% \graphicspath{{./Drawings/Course/}} % Uncomment this to use pictures in this document
% \addbibresource{./Bibliographies/CourseNum-Name.bib}
% % Define English Imperial units.
% Length
\DeclareSIUnit\inch{in}
\DeclareSIUnit\in{in}

\DeclareSIUnit\feet{ft}
\DeclareSIUnit\ft{ft}

\DeclareSIUnit\yard{yd}
\DeclareSIUnit\yd{yd}

\DeclareSIUnit\mile{mi}
\DeclareSIUnit\mi{mi}

% Volume
\DeclareSIUnit\fluidOunce{fl oz}
\DeclareSIUnit\floz{fl oz}

\DeclareSIUnit\pint{pt}
\DeclareSIUnit\pt{pt}

\DeclareSIUnit\quart{qt}
\DeclareSIUnit\qt{qt}

\DeclareSIUnit\gallon{gal}
\DeclareSIUnit\gal{gal}

% Mass
\DeclareSIUnit\grain{gr}
\DeclareSIUnit\gr{gr}

\DeclareSIUnit\ounce{oz}
\DeclareSIUnit\oz{oz}

\DeclareSIUnit\pound{lb}
\DeclareSIUnit\lb{lb}

\DeclareSIUnit\poundMass{lbm}
\DeclareSIUnit\lbm{lbm}

\DeclareSIUnit\ton{t}
\DeclareSIUnit\slug{slug}

% Temperature
\DeclareSIUnit\rankine{R}
\DeclareSIUnit[number-unit-product={}]\degreeF{\degree{}F}
\DeclareSIUnit[number-unit-product={}]\dF{\degree{}F}
\DeclareSIUnit[number-unit-product={}]\degF{\degree{}F}

\DeclareSIUnit[number-unit-product={}]\degreeR{\degree{}R}
\DeclareSIUnit[number-unit-product={}]\dR{\degree{}R}
\DeclareSIUnit[number-unit-product={}]\degR{\degree{}R}
% \DeclareSIUnit[number-unit-product={}]\Rankine{\degree{}R}
% \DeclareSIUnit[number-unit-product={}]\rankine{\degree{}R}
\DeclareSIUnit[number-unit-product={}]\degreeRankine{\degree{}R}

% Pressure
\DeclareSIUnit\bar{bar}
\DeclareSIUnit\atm{atm}
\DeclareSIUnit\psia{psia}
\DeclareSIUnit\psig{psig}
\DeclareSIUnit\psi{psi}

% Energy
\DeclareSIUnit\btu{btu}
\DeclareSIUnit\BTU{BTU}

%Force
\DeclareSIUnit\poundForce{lbf}
\DeclareSIUnit\lbf{lbf}

% volumetric flow
\DeclareSIUnit\cfm{cfm}
\DeclareSIUnit\CFM{CFM}

% Moles
\DeclareSIUnit\lbmol{lbmol}

%%% Local Variables:
%%% mode: latex
%%% TeX-master: shared
%%% End:


% Math Operators that are useful to abstract the written math away to one spot
% These are supposed to be document-specific mathematical operators that will make life easier
% Many fundamental operators are defined in Reference_Sheet_Preamble.tex

\let\emph\relax
\DeclareTextFontCommand{\emph}{\large\em}

\begin{titlepage}
  \title{Hist 336: Industrializing America --- Terms Sheet \\ Illinois Institute of Technology}
  \author{Karl Hallsby}
  \date{Last Edited: \today} % We want to inform people when this document was last edited
\end{titlepage}

\begin{document}
\pagenumbering{gobble}
\maketitle
\pagenumbering{arabic} % 1,2,3 on content pages

\begin{description}
\item[Virginia Plan] The Virginia Plan was the plan that created a strong federal government.
  Its terms:
  \begin{itemize}[noitemsep]
  \item Get rid of Articles of Confederation, as they were too weak.
  \item 2 House Legislature, where this Congress can pass laws over the states.
    \textbf{BOTH} houses would be proportional to the state's population.
    This would mean Virginia would get the largest representation in \textbf{BOTH} houses, as it has the greatest population.
    \begin{enumerate}[noitemsep]
    \item Upper House --- Senate
    \item Lower House --- House of Representatives
    \end{enumerate}
  \item Executive branch, with one official in charge of the entire system.
  \item Judicial system evaluates the legality of Legislative and Executive branches.
  \end{itemize}

\item[New Jersey Plan] The New Jersey Plan was a plan to create a slightly stronger federal government, but still keep a majority of the power in the states' governments.
  Its terms:
  \begin{itemize}[noitemsep]
  \item Give the national government the power to tax the entire nation.
  \item Federal government would regulate the trade of the nation.
  \item Keep the rest of the Articles of Confederation.
  \item Every state will have the same number of delegates in the national Legislature.
  \end{itemize}

\item[3/5 Clause] This clause states that slaves count as $\frac{3}{5}$ of a white person for taxation and representation purposes, although the possible taxes were never levied.
  The clause was needed to get the Southern vote for the new Consitution, which kept a vast majority of the Virginia Plan, but changed the upper legislative house, the Senate, to be constant for each state, namely at 2 senators per state.

  This means that even though the North and South had roughly equal populations, the South would dominate the population-based house.
\item[Electoral College] The Electoral College is created because the writers of the Constitution believe that if the common man votes, they will elect an idiot.
  Thus, the People actually vote for electors, and the electors will vote for the People.

  The elector is \textbf{NOT} bound to vote for the person they said they would.
  However, today, this is not the case, but only for the \underline{first} ballot.

  Each state gets a number of electors equal to the number of representatives they have in the House and the number of senators they have.

  This means you can win the popular vote, but lose the electoral vote, meaning you lose the election.

\item[Checks and Balances] These are ways for each branch to check the others' powers.
  These help balance each of these branches out.

  \begin{description}[noitemsep]
  \item[Executive Branch] The Executive Branch is the one that is supposed to enforce the laws and make sure they can function correctly.
    \begin{itemize}[noitemsep]
    \item The President can appoint people to places, but the Legislative Branch (Senate) must approve it.
    \item The President is allowed to make treaties, but the Senate must ratify the treaty into law.
    \item The President is the Commander-in-Chief of the armed forces, but only Congress can declare war.
    \end{itemize}
  \item[Legislative Branch] The one that makes the laws.
    \begin{itemize}[noitemsep]
    \item Congress can make a law, but the president can veto it.
    \item Congress can impeach (indict) the president.
    \item Congress can kick out the president.
    \end{itemize}
  \item[Judicial Branch] The Judicial Branch is the one that runs the judicial system, including the courts.
    \begin{itemize}[noitemsep]
    \item The Judicial Branch can declare laws unconstitutional, but the Justices in this branch must be approved by Congress.
    \end{itemize}
  \end{description}

\item[Federalists] These were the people who wanted the new Consitution to be passed.
  Typically, members of this faction were better off Americans, mainly business owners and commercial farmers.
  They were capable of spending more money for pamphlets and literature.
  They were highly organized.

  To make sure that the Constitution passed, they appeased the worried voters by promising to make a Bill of Rights full of Consitutional Amendments as soon as the Constitution is passed.
  The Constitution was sent out for Ratification on September 17, 1787, and passed in 1788.

\item[Anti-Federalists] These were the people who were highly-afraid of a strong Federal government.
  These were the normal Americans who didn't want a government to tell them what to do.
  They believed that a strong government that was controlled by the rich (which it was) would only help the rich and screw over the little people.

\item[Subsistence Economy] The subsistence economy makes up a majority of people in the United States at this time.
  They were typically poor farmers who could only grow enough to live, and maybe \textbf{trade} for other, small, things.

\item[Bill of Rights] A set of Constitutional Amendments that were promised by the Federalists to the Anti-Federalists.
  These provided several freedoms, rights, protections, and prohibitions for people that were not provided for in the Constitution.
  \begin{description}[noitemsep]
  \item[Freedoms] These are things you are granted by being a citizen of this nation.
    These can be removed due to legal changes or if you become an enemy combatant.
    \begin{enumerate}[noitemsep]
    \item Freedom of Speech
    \item Freedom of Religion
    \item Freedom of Press
    \end{enumerate}
  \item[Rights] These are things you by existing and being human.
    These can be removed due to legal changes or if you become an enemy combatant.
    \begin{enumerate}[noitemsep]
    \item Right to bear arms and raise a militia.
      \begin{itemize}[noitemsep]
      \item This was created because people were scared of a strong central government.
      \end{itemize}
    \item Right to assemble.
    \item Right to petition the government.
    \item Right to trial by jury.
    \end{enumerate}
  \item[Protections] These can be removed due to legal changes or if you become an enemy combatant.
    \begin{enumerate}[noitemsep]
    \item Protection from excessive bail.
    \item Protection against illegal search and seizure.
    \item Protection against self-incrimination and double-jeopardy.
    \end{enumerate}
  \item[Prohibition] These can be removed only if you become an enemy combatant.
    \begin{enumerate}[noitemsep]
    \item Cruel and unusual punishment.
    \item Cannot quarter troops in your house.
    \end{enumerate}
  \end{description}

\item[Tariff] A tariff is, essentially, an import tax on a certain set of goods.
  These were levied against foreign-made goods to fund the new federal government.
  Tariffs are good for helping American businesses, but they hurt American buyers.
  Therefore, the subsistence economy (a majority of people in the US) hates this, but the commercial economy loves it.

\item[Alexander Hamilton] Is the man who architected the strong federal government we have today.
  He hated the subsistence economy and its members to the point that he raised an excise tax of 25\% on the production of whiskey.
  He used this tax to pay down the federal debt.

  Hamilton believed that the government should help the rich, allowing the rich to kick-start the economy.
  To achieve this, he conceived of the National Bank, that would coordinate and compete with the few state banks as there were at the time.
  This allowed the federal government to control the production of US currency, as the federal government cannot print money right now.
  This bank will hold all federal government money.
  Southerners were particularly against this.

  He stated that the Constitution doesn't say anything about the government \textit{not} chartering a National Bank, so the federal government \textit{can} charter such a bank.
  This interpretation of the Constitution means that Hamilton was a Broad Constructionist, a believer in Implied Powers.

\item[Thomas Jefferson] Jefferson \textbf{hated} the idea of a National Bank, although he never said so.
  Jefferson was an strong support of Strct Construction, meaning all the powers that the federal government had were specified in the Constitution.

\item[Strict Construction] A Strict Construction view of the Constitution states that the Constitution provides all the rules and powers that the federal government is allowed to have, and that \textbf{any} extension of these powers requires amendments to the Constitution.

\item[Broad Construction / Implied Powers] Broad Construction believers believe that the Constitution was meant to be the end-goal, but some powers that the government requires were not explicitly specified.
  Thus, the Constitution is up to some interpretation, allowing the government to expand its powers in slightly unique ways.

\item[Federalists / Democratic Republicans] After President Washington's first term, the US started dividing into 2 political factions:
  \begin{description}[noitemsep]
  \item[Federalists] These are still the same group from before.
    They believed in a strong federal government, tended to take Broad Constructionist views on the Constitution, tended to be wealthier, and overall, be from the North.
  \item[Democratic Republicans] These are the remnants of the Anti-Federalists.
    These are typically poorer people, namely of the subsistence economy.
    Most of the members of this faction were from the South or the West.

    Namely, they just want to be left alone.
  \end{description}

\item[John Adams] Adams was Washington's Vice President throughout Washington's two terms.
  Adams was a Federalist, and thus, the Democratic Republicans \textbf{hated} him.

  Adams was a poor democratic politician, as he was bad at making compromises with others.
  However, he \textbf{always} put the country's interests above his own party's, which puts him at odds with his own political party.

  Most of his presidential terms, he was stuck dealing with foreign affairs.
  Namely, at this point, Britian and France were at war again.
  At this point, France was stealing our trading ships, which is technically an act of war.

  To grow the army and navy, to allow the US to protect its own interests, a direct tax on Housing, Land, and Slaves was passed.
  However, like all throughout history, Americans \textbf{HATE} taxes.
  In addition, the Alien and Sedition Acts were passed, namely to target poor immigrant workers.

\item[Alien and Sedition Acts] These are a set of discriminatory acts that target poor immigrant workers from Europe.
  \begin{description}[noitemsep]
  \item[Alien Enemies Act] States that the President can deport foreign national from the contry.
  \item[Alien Friends Act] States that the president can deport any foreign national if the President suspects them of treason, even if we are \textbf{not} at war with that country.
  \item[Naturalization Act] Change the Constitution's 5 year waiting time to become a Naturalized citizen to 14 years.
  \item[Sedition Act] It is criminal to conspire against the government.
    In addition, it is a federal crime to say or print anything false, scandalous, or malicious about the government or any government official.
  \end{description}

\item[Yeoman Farmer] These are members of the Subsistence Economy.
  These farmers only grow what they need, and are completely independent of anyone else, including any government.
  These people would form the ``Country of Liberty''.

\item[Louisiana Purchase] Jefferson purchases this land from France, for \$15 million.
  It gives us everything west of the Mississippi River to the Rocky Mountains, and everything from Oklahoma to Canada.
  This also allows us to trade through New Orleans, which was perfect for Jefferson's country of Yeoman farmers.

  Jefferson skirts his strict constructionist view of the Consitution by saying that the Constituion \textbf{doesn't} say that the government \textit{cannot} buy land.

\item[Impressment] June 1807, British ships attack \textbf{US ships} in \textbf{US territorial waters}.
  In addition, the British kill 3 Americans and \emph{impress} the remaining men into Britain's naval service.

  In this case, this is not only impressment, but also kidnapping.
  The People press Jefferson for war, but he really doesn't want it.
  In addition, because of his Democratic Republican views, he has greatly shrunk the government, to the point where we do not have the military forces required to handle this problem.

\item[Embargo Act of 1807] This embargo was placed because of Britain's impressment of US saiors during Britain's and France's war.
  This embargo stated that the US would not trade with anyone until either France or Great Britian leave the US, and its ships, alone.

  However, this ends up causing significant economic damage to the US, but not to anyone else.
  This greatly hurts the Federalist political faction.

\item[James Madison] Madison was Jefferson's Secretary of State.
  One of the first things he did when he was inaugerated in 1809, was to remove the Embargo Acto fo 1807.
  He replaced the embargo with a ban on trading with only Great Britain and France, the US can trade with anyone else.

\item[Tecumseh] Brother of Tenkswatawa.
  He was the military leader of the Native people who had a spiritual reawakening, led by his brother.
  He helped establish Prophetstown in 1808.

\item[Tenkswatawa / The Prophet] Brother of Tecumseh.
  He was the religious/spiritual leader of the Native people as he led them through a spiritual reawakening.
  He taught that the Natives must go back to their original ways, and must use non-violent resistance.
  He helped establish the town of Prophetstown in 1808.

  However, in November, 1811, General Harrison attacks these Natives in Prophetstown.
  Tecumseh was not in the town, so Tenkswatawa doesn't stop his kinsfolk from violently defending against their attackers.
  So, the Natives turn to the British for help in dealing with the Americans.

\item[War of 1812] Starts June 1 1812.
  After the British impressment of American naval troops, the People were clamoring for war.
  In addition, there was a minor faction in Congress that also advocated for a war with Britain, called the \emph{Warhawks}.
  From a factional stand-point, no Federalist voted in favor of the war.

  People felt like Britain was still treating us like a colony.
  However, due to the Democratic Republicans, there were no new taxes to expand the military, so Britain had 3 times the army the US did.
  To add, the US only had 16 naval vessels, and 7 of them were too old to sail.

  Because of the US's declaration of war, Britain sent 200 ships and blockaded the Eastern seaboard, stopping all trade from that side of the nation.
  In response, the US tries to grab the only thing it can from Britain at this time, Canada.
  The US attacks Canada 3 times, and each time the US fails, mainly due to dissent and division.
  This was mainly due to Federalist state's state militias not crossing the border into Canada to attack.

  To top it off, the North felt like slaveowners started this war, and Federalist states threaten to secede the Union.
  This massively damaged the Federalist reputation, and made them look like traitors.

  The only reason the US didn't lose was because Britain was more focused on dealing with France.
  So, there were relatively few British troops in the US.\@
  After 1814, Britain ended the war with France.
  In August 1814, Britain managed to burn down Washington D.C.\@

  At the end of 1814, there was a peace conference in Gent (Modern-day Belgium).
  Britain just wanted the wars to end, and was willing to be generous.
  Essentially, this treat put things back to the way they were.
  In January 1815, the \emph{Treaty of Gent} was signed.

  However, due to the time required for the propagation of information in this time, In January 1815, Andrew Jackson attacks New Orleans.
  The Battle of New Orleans was wildly successful for the American troops, as 2000 British were either killed or wounded, while only 21 Americans were killed/wounded.
  Because of this, Andrew Jackson becomes a national hero.

  In the end, what people perceived was that the US defeated a super powerful foe 2 times, which caused a large shift in national identity.
  This brought around the time where what it means to be American was defined.
  This led to a huge increase in Nationalism, where people saw each other and themselves as American.
  This pride also kills the Federalist Party.

\item[Blockade] A blockade is when the seaboard, with ports, is blocked off from the outside contact by sea.
  In the War of 1812, Britain blockaded the Eastern Seaboard after the US's declaration of war.

\item[Nationalism] Nationalism is the love of your people.
  Patriotism is a love of your country.
  The US is one of the few countries where Nationalism and Patriotism do not mean the same thing.

  After the War of 1812, people are identifying as American, rather than with their states.
  People are proud to be Americans, and are willing to call themselves that.

\item[The Era of Good Feelings] This was a period of post-war unity in the US.\@
  Due to the collapse of the Federalist Party, only the Democratic Republicans remained.
  Americans, in general, were happy and liked this new-found unity.

  There was also a post-war economic boom, which came around after 2 years of war and 5 years of embargo.
  Large tariffs were put in place, to protect and foster American businesses.
  The government used this money to build canals, such as the Erie Canal.

\item[James Monroe] Runs for President in 1816 and 1820.
  He makes the US \textbf{the} dominant power in the Western hemisphere.
  He forces Spain to give the US Florida, and deals with the British to get the Pacific NorthWest.

  In December 1823, he issues the Monroe doctrine.

\item[Monroe Doctrine] The Monroe Doctrine was a declaration by James Monroe about how foreign affairs will be handled in the Western Hemisphere.
  It stated:
  \begin{itemize}[noitemsep]
  \item The US will not tolerate \textsc{any} European intervention in the Western Hemisphere.
  \item In exchange the US will stay out of Europe and its affairs.
  \item European nations can keep their colonies.
  \item This was issues because many South American colonies were starting to gain their independence, and Monroe wanted to protect these nations.
  \end{itemize}

  However, the US has no practical way to enforce this, but Britain enforces it for us.
  Britain has its sugar colonies, and they don't want other European powers down there.

\item[Sovereign] At this point in time, this is th efirst time in US history where the nation acts like a sovereign nation.
  This means the US is free from outside influence, nobody is telling us what to do.

\item[Henry Clay] Clay is a legislator from Kentucky during the end of the Period of Good Feelings.
  During this time, Missouri wanted to enter the Union as a state, a slave state.
  However, this would upset the equal balance already present in the US, of 11 free and 11 slave states.
  The North wanted to protect its interests and saw the expansion of slavery, and the South wanted another slave state.

  So, there was a complete divide in Congress, where the Senate blocked Missouri from becoming a state.
  So, Henry Clay came up with the Missouri Compromise.

\item[Missouri Compromise] A compromise made in Congress about how Missouri can enter the Union, with stipulations about how states should be going forward.
  This chops off a portion of Massachussets and makes it a new state called Maine.
  Missouri is allowed into the Union as a slave state.
  The rest of the Louisiana Territory will only be free states, from now on.

\item[William Crawford] Crawford was President Monroe's Secretary of State.
  He was a strong state's rights advocate.
  In the 1824 election, he ran against Andrew Jackson, John Quincy Adams, and Henry Clay.

\item[John Quincy Adams] Adams was President Monroe's Secretary of the Treasury.
  He was a believer in a strong Federal government, and would essentially be a Federalist.
  In the 1824 election, he ran against William Crawford, Andrew Jackson, and Henry Clay.

\item[Andrew Jackson] Jackson, after the Battle of New Orleans, was a national hero and a friend of the people.
  In the 1824 election, he ran against William Crawford, John Quincy Adams, and Henry Clay.
  Jackson won 43\% of the popular vote.

  He left the Democratic Republican party and formed a new one, the Democratic Party.
  These were the left-overs from the \textbf{original} Democratic Republican party, meaning members were typically from the subsistence economy, and largely favored state's rights.

  To handle this, the Democratic Republicans change their name to the National Republican party.
  These were the \textbf{original} Federalists, and wanted to help business with a strong federal government.

  In 1828, Jackson ran again, making this the first presidential election that resembles modern-day elections.
  There were national campaigns, party-controlled media, meetings/rallies, literature, buttons, parties, meet-ups, etc.\@
  This was also the first election that was not about just issues, but also personality and the person.
  This is called the first campaign of image-making.
  To do all of this, Jackson takes no public stance on any issue.
  Jackson ends up having 3 major issues during his presidency, but also does a few other things.
  \begin{description}[noitemsep]
  \item[Spoils System] After Jackson's presidential election victory in 1828, he started using the Spoils system.
    The spoils system is based off the phrase, ``To the Victor go the spoils''.
    Jackson put his supporters, many of whom were untrained in their fields, into government positions.

  \item[Indian Removal] Jackson was strongly against having Natives in the US, and allowed this to happen.
    There were 125000 Native Americans in the South, east of the Mississippi River.
    They held land, by treaty.
    The Cherokee were the largest Native population in Georgia, but did everything the white way.

    Georgia stated that all land \textbf{not} held by whites was the state's land, and was available for white people to take.
    Several other statesd followed suit.

  \item[Cherokee] The largest Native American tribe in the US at the time.
    They currently inhabited the Southern US, and lived their life the way the white people taught them to.
    After the Indian Removal Act of 1830 was passed, the Cherokee sued the US government, and the trial went all the way to the Supreme Court.
    The court voted in-favor of the Cherokee, but Jackson doesn't care and continues anyways.

  \item[Indian Removal Act of 1830] To get the Natives out of the Southeast US, Jackson pressures Congress to pass this law, which required that the Cherokee must move to Oklahoma from Georgia.
    Of the 125000 Natives in this area before this forced migration, almost a quarter of them die on the journey.

  \item[Trail of Tears] The forced migration that was brought by the Indian Removal Act of 1830 is called the Trail of Tears.
    During the travel from Georgia to Oklahoma, over mountains, and through the vast plains of west of the Appalachians, over 1/4 of the members of the tribes died.

  \item[Nullification] Is an extreme version of State's rights.
    This idea allowed a state to nullify a federal tariff, if the state did not like it.
    This happened because the federal government had no way to collect the tariff directly, and relied on the states to do it for them.
    For the most part, this was relatively unused, but there was one set of tariffs in particular that this \textbf{was} used on, the Tariff of Abominations.

    This also helps form the basis for the secessionism present in the South, along with \textit{The South Carolina Exposition and Protest}.

  \item[Tariff of Abominations] This was a tariff put in place in 1816, and was raised several times.
    At the time when nullification was a popular idea, this tariff reached 50\%.
    The tariff was on practically anything the US had to import, including:
    \begin{itemize}[noitemsep]
    \item Cloth
    \item Salt
    \item Steel
    \item Iron
    \item Paper
    \item Glass
    \end{itemize}

    The South \textbf{HATE} this tariff, because the subsistence economy bears the brunt of this tariff, and the South tended to be a big believer in state's rights.
    Out of the entire South, South Carolina was the most fervent hater of this set of tariffs.
    This state was super conservative, super wealthy, losing population, and had a majority black-American population (This was the only state in the Union with a population distributio n like this).

  \item[Force Act] Jackson passes this law in January 1833.
    This law was passed in response to Calhoun's \textit{The South Carolina Exposition and Protest}.
    Ths law allows the president to call the army out to protect federal law and keep the union together.
    To compromise with the voter-base, Jackson also cuts tariffs in half, to ensure this law passes.

  \item[Bank Crisis] Jackson \textbf{HATES} the National Bank, because farmers hate banks.
    In 1836, the National Bank is due for its regular 20-year recharter.
    So, Jackson wants to kill the national bank.

    After Jackson's win in the 1832 presidential election, Jackson removes all the government money from the national bank.
    However, he goes through several Secretaries of the Treasury before he finds someone willing to actually take the money out.
    The one who does this is Roger Taney.

    Once the money is taken out, the money is redistributed and put in pet banks.
    These were banks that Jackson/Democrats liked.
    Then, these banks could lend this money out.
    However, they flooded the economy, causing mass-inflation and an economic collapse in 1837.
    This economic collapse is called the Panic of 1837.

  \item[Nicholas Biddle] Biddle is the president of the National bank.
    He wheels-and-deals with Jackson to keep the national bank out of the 1832 presidential election, so Jackson will support the bank when re-chartering comes along.

  \item[Roger Taney] Taney (pronounced Taw-Ney) is the Secretary of the Treasury that actually removes the money from the National bank.
  \end{description}

\item[Second Great Awakening] This was a religious and spiritual awakening the US.\@
  It opened up religion to the masses, and encouraged more people to go to Church.
  In addition, two new denominations of Christianity became popular:
  \begin{itemize}[noitemsep]
  \item Methodists
  \item Baptists
  \end{itemize}

  In these denominations, the congregation pick their clergy.
  This was important because the people were starting to gain more control in the political affairs in of their state and the nation, and they were also gaining something in their religion.

  At this time, preachers were also wandering the country, preaching.
  They would hold Revival Meetings, where they would essentially perform as televangelists.
  These meetings were a form of entertainment for the people.
  Eventually, this leads to the rise of Evangelical Christianity, which requires:
  \begin{itemize}[noitemsep]
  \item You \textbf{have} to have a conversion experience.
  \item Spend personal time with God and accept Jesus as your personal lord and savior.
  \item A very literal interpretation of the Bible.
  \item Very emotional and active ceremonies/worship.
  \end{itemize}

  Evangelical Christianity required you to \textbf{choose} to accept Jesus/God and made it up to you to save yourself.
  This made this form of Christianity a very selfish, but democratic religion.

\item[John C. Calhoun] Calhoun is a senator from South Carolina.
  He anonymously publishes \textit{The South Carolina Exposition and Protest}, which expands the ideas of Nullification from just taxes/tariffs to entire laws and legislation.

\item[\textit{The South Carolina Exposition and Protest}] This work is published, anonymously, by John C. Calhoun.
  This work states:
  \begin{itemize}[noitemsep]
  \item If a state doesn't like a federal law, they can nullify it.
    \begin{itemize}[noitemsep]
    \item Technically, this is illegal under the Constitution.
    \end{itemize}
  \item States must vote to nullify a federal law.
  \item The only thing that cannot be nullified is the Consitution, and any Amendments made to it.
  \item If a law becomes a Constitutional Amendment, and the state still doesn't like it, then the state may secede the Union.
  \end{itemize}

\item[Martin Van Buren] Van Buren is Jackson's successor, and really doesn't do much.
  However, he is the one that must deal with Jackson's creation of the Panic of 1837.

\item[Panic of 1837] The Panic of 1837 was an economic depression caused by the mass-lending of money from Jackson's pet banks using government money.
  At the height of the Panic, there was over 20\% unemployment.
  The Panic lasts until 1843.

\item[Whigs] This political party was formed in 1836 as a response to Jackson's doings.
  The core portion of this group was mainly made of:
  \begin{itemize}[noitemsep]
  \item The wealthy
  \item The business people
  \item The merchants
  \item Factory owners
  \item Democrats that don't agree with Jackson's doings.
  \end{itemize}

  Members of this party favored:
  \begin{itemize}[noitemsep]
  \item A US National Bank
  \item Want paper money
  \item Want protective tariffs for US businesses
  \item Want subsidized transportation and related infrastructure (Canals, Railroads, etc.)
  \item Anti-alcohol laws
  \item Sunday Laws, stating that no work was to happen on Sundays
  \item Anti-immigration laws, even though they were the ones who would benefit most from these people.
  \item Public education
  \item Government-run asylums, hospitals, etc.
  \end{itemize}

  This was the party of morality.
  At this time, women were thought to be more moral, so the Whigs started including women in the political process.
  Usually, women formed auxiliary groups that encouraged men to vote and encouraged other women to get men to vote.

\item[William Henry Harrison] In the 1840 presidential election, Harrison was the Whig nominee.
  He was the hero of 1812, and the one who burned Prophetstown.
  He was super-rich, had a mansion, but showed himself to the people as ``a regular Joe''.

  Harrison wins the election, and Congress is also Whig-controlled.
  This is the ifrst time in 40 eyars that the commercial economy party has controlled politics.

  However, Harrison is the oldest man elected president by this time, and dies 30 days after taking office, making Tyler the president for the next 4 years.

\item[John Tyler] Tyler was selected as Harrison's vice president for the 1840 presidential campaign.
  Tyler is also a Whig, but was a former Democrat.

  After Harrison's death, Tyler becomes the president.
  As president, he blocks everything the Whigs try to do, and for this, he gets kicked out of the Whig party.

  Tyler's big goal is to bring Texas into the Union as a state, which was being blocked by Northerners.
  Because Texas is south of the line the Missouri Compromise laid down, Texas was a slave state.

  Right before Tyler leaves office, he gets to bring Texas into the Union.

\item[James Knox Polk] Polk was the 1844 presidential election winner.
  He wanted to annex everything he could.

\item[Benevolent Empire] The Benevolent Empire was a religious idea that started being passed around in the 1840s.
  It believed in American Exceptionalism, and focused on making the US even better.
  They formed many groups to try to make American society more moral.

  Eventually, this and the 2nd Great Awakening will help form the African Methodist Episcopal Church.

\item[African Methodist Episcopal Church] This was a Christian denomination and organization for free African-Americans in the US.\@
  It was formed out of the Benevolent Empire's idea of American Exceptionalism, and the 2nd Great Awakening's revival of emotional spiritual practices.

\item[Temperance Movement] The Temperance Movement was a moral reform movement that attempted to reduce, and eventually eliminate, alcohol in the United States.

\item[American Temperance Society] The American Temperance Society was the organization that actually pushed the Temperance Movement through American society.
  They focused on demonizing alcohol, making it a sin.
  Because of the 2nd Great Awakening, sinning could be avoided by willpower and believing in God.

  At this time, alcohol was cheaper to buy and safer to drink than water.
  So, the Society blamed everything bad or wrong with society on alcohol.

  There was opposition to this group, mainly from the idea of a Puritan conspiracy.
  Poor people and subsistence farmers tended to hate this group.
  The farmers hated it because they couldn't sell their excess grains for alcohol, and the poor hated it because now they had no escape form reality.
  In addition, immigrants hate it because Europe has different sensibilities about alcohol than the US.\@
  Lastly, there is large amounts of opposition because women are leading this movement.

  Over time, this became successful, mostly in the middle-class.
  Women join this group in huge numbers.
  The society managed to reduce the number of 100-proof shots taken per day by everyone over the age of 14 from 5 to 1.5.

\item[Church of Jesus Christ of Latter-Day Saints] This denomination of Christianity was formed, mainly as a response to the increasing role in public and political life women were playing in the US.\@

  The church offers a tight-knit community with which the lose-rs of this social change can band together and help each other out.
  This is a religious communal society, which gives religious certainty and a feeling of support and commitment.

  This denomination is based on the idea of male authority, where women cannot attain God's salvation without obedience and subservience to men.
  In addition, once the Mormons move to Nauvoo, Illinois, the more radical ideas of Mormonism come into shape.
  These include:
  \begin{itemize}[noitemsep]
  \item Baptism of the dead.
  \item Eternal marriage, stating that ``till death do us part'' is not actually the end of the marriage.
  \item Polygamy, meaning one man has lots of wives.
  \item If you are a good Mormon, you become another planet's god after your physical body's death.
  \end{itemize}

  For this, among several other reasons, the Mormons are not seen as Christian.

  \begin{description}[noitemsep]
  \item[Joseph Smith] Smith is the founder of the Church of LDS.\@
    He is from a New England family that is losing during this social change.
    He believes that religion is currently controlled by the wealthy, and hates that women are taking public roles in the US.\@
    He also believes that the religious individualism from the 2nd Great Awakening is problematic.

  \item[Book of Mormon] Smith is ``told'' by God, through a cow, to find the Golden Tablets, which form the Book of Mormon.
    These were written by one of the 12 tribes of Israel who ended up in North America before the Native-Americans killed them.

  \item[Nauvoo, Illinois] The Mormon church moves from New England to Illinois to set up their community.
    This is where Smith gets the more ``extreme'' Mormon ideas.
    In 1844, Smith is murdered, and the Mormons move from Nauvoo to Utah, to what will become Salt Lake City.
    Brigham Young is the one who actually leads this migration.
  \end{description}

\item[Utopian Reform] Utopian reforms wanted to step away from modern American society and form something completely new.
  They wanted to avoid the capitalist and individualism problem completely by forming a utopia.
  If they formed a functioning utopia, then we it could entice Americans over.
  These were mainly seen as communal efforts.
  There were several kinds of utopian reforms, religious, secular, and philosophical being the biggest ones.
  \begin{description}[noitemsep]
  \item[Shakers] Christian utopia religion formed by Ann Lee.
    This questioned the form of God, stating that God can present as both male and female.
    The last presentation of God as a male was Jesus, and the next would be a woman.
    They believed that the next (or possibly present) presentation of God would be Ann Lee, their founder.

    This religion was formed in Britain, and moved across the Atlantic.
    They believed that modern society was harsh, that private property caused greed, and that monogomous marriage made women slaves.
    This meant that there was no marriage, and that there was no sex, meaning no procreation of the Shaker community would allow the continuation of the religion.

  \item[Ann Lee] Founder of the Shaker religion.
    Believed to be the next incarnation of God, akin to Jesus.

  \item[Oneida Community] Christian utopia religion formed by John Humphry Noyes.
    Like the Shakers, they believed that private property caused greed and that monogomous marriage made women slaves.
    However, they believed that everyone in their community was married to each other.
    This meant that there was no single husband or wife, instead this was a complex marriage web, where anyone could have sex with anyone.

    Children were raised communally.
    However, the religion starts dying out because the age at which women become initiated to the complex marriage started decreasing.

  \item[John Humphry Noyes] Founder of the Oneida Community.

  \item[Robert Owen] Owen was a Scottish industrialist that buys a textile mill and reforms it.
    He takes many ideas from Marxist socialism about people taking care of each other.
    He asks ``How can we expect the poor to be good given the conditions they face?''
    Thus, he teaches his workers and their children, he gives access to free public education, builds better housing, among other things.

    To further his goal, he buys New Harmony, Indiana and tries to set up a utopian society.
    He divides people into either factory-workers or farmers.
    The factory will make goods needed for farming, and the farmers grow what is needed for the factory workers to live.
    However, this fails due to greed on both sides.

  \item[Transcendentalism] A philosophical form of utopianism where reality can only be truly experienced by feeling the world emotionally.
    Believers in this doctrine want to become the best you can be.

  \item[Separate Spheres] A philosophical ideology that the modern world is divided into separate spheres, each inhabited by their respective genders and gender roles.
    Men deal with the world outside of the home, and women deal with stuff inside the home.
  \end{description}

\item[Female Moral Reform Society] An anti-prostitution reform society.
  They believe that women become prostitutes because they have no other choice.
  They also believe that this is \textbf{NOT} the woman's fault, but men's.
  The group helps women leave their life of prostitution.

  However, they see another marginalized group, the slaves, and work to free them, before freeing their sisters in prostitution.

\item[American Colonization Society] A gradualist anti-slavery movement formed in 1817.
  Some notable members include James Madison, James Monroe, and Abraham Lincoln.

  This groups purchases slaves and then sends them back to Africa.
  This was a highly racist group, believing that Africans/black people shouldn't be in the US.\@
  The group lasts through the Civil War, but achieves very little.
  This is because slavery is a profitable business, making this a logistically and monetarily infeasible plan.

  This group galvanizes free blacks in America, who start their own anti-slavery societies.
  They start their own newspapers, spread the word of slavery.
  The largest thing that happens is that they start the formation of the Immediatist/Abolitionist groups.

\item[Abolition] An immediatist anti-slavery movement formed in 1832.
  The group believes that equality is for all people.
  They have 2 core tenents.
  \begin{enumerate}[noitemsep]
  \item Slavery is morally wrong.
    Because it is morally wrong, you must support a complete and immediate end to slavery.
  \item Complete legal equality for freed slaves.
  \end{enumerate}

  Neither of these ideas are popular.

  To spread their message and tenents, the American Anti-Slavery Society is formed.
  \begin{description}[noitemsep]
  \item[William Lloyd Garrison] Garrison, from New England, formed the Abolitionist movement in 1832.
    He was quite radical, believing that \textbf{ALL} people should be equal.
    This included men and women, regardless of skin color or economic background.

  \item[American Anti-Slavery Society] Started in 1833, this society publishes books, pamphlets, holds rallies and awakening meetings to spread the word of abolitionism.
    By 1840, there were 200,000 members in the whole country.
    This brings the concept of religion commenting on slavery based on moral grounds.

    Congress places a gag rule on the organization about its anti-slavery message.

  \item[Angelina and Sarah Grimke] These sisters gave abolitionist speeches to both men and women.
    They were not heeded much, as women lecturing to men was not a thing at this time.
    They were called whores, among other names.
    However, Garrison supported them.

  \item[Lucretia Mott] During the 1840 World Anti-Slavery Society's meeting, they were barred entrance, because she was a woman.
    She was a co-founder of the Women's Rights Society, with Elizabeth Stanton.
    In 1848, the first Women's Rights Convention was held in Seneca Falls.
    This convention was dealt with women's equality and the right to vote for women.
  \item[Elizabeth Cady Stanton] During the 1840 World Anti-Slavery Society's meeting, they were barred entrance, because she was a woman.
    She was a co-founder of the Women's Rights Society, with Lucretia Mott.
    In 1848, the first Women's Rights Convention was held in Seneca Falls.
    This convention was dealt with women's equality and the right to vote for women.
  \end{description}

\item[Political Anti-Slavery] This was a much larger group in the US, with many more supporters.
  The idea was to politicize slavery and make it an issue.
  Do achieve anything political support was needed.

\item[Liberty Party] A third political party during this time.
  It was a political party that explicitly dealt with the condemnation of slavery and racial discrimination.
  Although they never managed to get a president in office, they forced the other parties to tackle the topic of slavery.

  The Whig party picks this up, because they are the ``Moral Party''.
  They push the idea of Slave Power, feeding the conspiracy theories.

\item[Slave Power] This is the belief that Southerners control the government with Northern lackeys, and will spread slavery everywhere in the Union.

\item[Upper South] The Upper South consisted of states like Maryland, Virginia, Missouri, North Carolina, Arkansas, Delaware, etc.\@
  These states do not have the right conditions to grow plantation crops, namely cotton.
  Therefore, slavery as a societal institution is not required.
  So, less than half the population owns any slaves.

  They are also experiencing an economic depression right now, because growing only tobacco exhausted the soil.
  Eventually the started rotating crops, and started growing wheat, corn, and other grains.
  They are also, slowly, turning to manufacturing.

\item[Lower South/Deep South] This comprised everything else in the South, including Georgia, Florida, South Carolina, Alabama, Louisiana, and Texas.
  Here, the number one crop was cotton, which \textbf{REQUIRED} slaves to tend to it and harvest it.
  This allowed the US to supply the world with cotton.
  Slaves were also needed to grow the corn that grew in the other half of the growing year.

  The plantation was a commercial farm with at least 20 slaves.
  Overall, less than 5\% of Southern families actually lived on a plantation, but this population controlled a majority of the South.

  This was \textbf{incredibly} rural, with very few cities.
  Overall, the plantations were self-sufficient, so there was little need for urbanization.

  There was a shortage of skilled labor in the South, which led some slaves to taking up these positions, much to the chagrin of white people and slaveowners.

\item[Chattel] Slaves were usually treated as chattel, which meant that they were personal property of their owners.
  This meant that the slaves could be sold like any other good or service a person may own or perform.

\item[Slave Codes] These were sets of legal codes that only applied to slaves.
  They were very restrictive of giving the slaves anything, as the more given to them, the more aware the slaves would be to their station.
  Some common entries in these codes were:
  \begin{itemize}[noitemsep]
  \item No Property ownership
  \item No guns
  \item No alcohol
  \item No marriage
  \item Cannot testify against a white person
  \item Were not allowed to be taught to read or write.
    \begin{itemize}[noitemsep]
    \item This allowed the white population to control their slave populations much better.
    \end{itemize}
  \item No legal protections available
  \item One slaveowner cannot kill another owner's slaves, as that would be damage of property.
  \item Whipping as a punishment was allowed.
  \item Slave execution was allowed.
  \end{itemize}

\item[Task System] Here, the slave was given a task, and when the task was completed the slave was done for the day.
  The work completed would be inspected by the person who assigned the task.
  Typically, this was used for slaves that worked inside the house.

\item[Gang System] Here, the slave was supervised by the overseer, who would monitor and conduct the slave's work all day.
  A majority of slaves operated under this system.
  Typically, this was used for slaves that worked in the fields.

\item[Rebellion/Run Away/Resistance] In a bid for slaves to control their own destiny, thesewere the three alternatives you had.
  \begin{enumerate}[noitemsep]
  \item Rebellion involved getting many slaves together to fight against their owners.
    \begin{description}[noitemsep]
    \item[Gabriel Prosser’s Rebellion] This was a slave rebellion in 1800 that was turned in by another slave.
    \item[Denmark Vesey’s Conspiracy] Slave rebellion in 1822, where another slave turned in the rebellion.
    \item[Nat Turner’s Rebellion] This was unique, as Turner was seen as a model slave.
      He killed his owner and his family.
      His rebellion collapses in 2 days, but Turner hides out for 2 months.
      This scares whites, because they couldn't even trust their model slaves anymore.
    \end{description}
  \item Running away meant using whatever social network you had or could find and escape to the North.
    Some slaves got very creative about making their escape too.
    \begin{description}[noitemsep]
    \item[Underground Railroad] This was a set of safehouses and people that moved slaves from Georgia to Illinois.
      Whites held the properties that the slaves were hidden on, but other African-Americans did most of the work.
    \end{description}
  \item Small acts of defiance that allowed some autonomy and self-determination.
  \end{enumerate}

\item[Popular Sovereignty] During this time, the expansion of the railroad system allowed the West to settled very quickly.
  Again, the topic of allowing states into the Union as either free or slave states became a problem.
  With popular sovereignty, the territory would decide for itself.
  However, there is not specific time when this decision was mandated to happen.

\item[Wilmot Proviso] This was a war-funding bill that tried to pass, but stated that all territory won during the Mexican-American War would be free territory.
  The territory won after this was included New Mexico, Arizona, and Southern California.

  It was up for a vote 50 separate times in Congress.
  The Proviso continuously passes in the House, but continuously dies in the Senate, because of the equal split of free and slave state senators.
  Eventually, it is passed, as it is stuck onto the end of a war-funding bill.

  This proviso convinces the South that if the North takes the Senate, then slavery will disappear, even though that would not be the case.

\item[Lewis Cass] Cass was a Democrat that was a proponent of Popular Sovereignty.
  He was a presidential candidate in the 1846 election.

\item[Zachary Taylor] Taylor was a Whig and a general from Louisiana.
  He was a slaveowner and plantation owner, with over 100 slaves, but supported popular sovereignty.
  He wins the 1846 presidential election.

\item[Compromise of 1850] When California wants to become a state in 1849, as a free state, this causes problems.
  Taylow allows them into the Union, because they decided to be a free state through popular sovereignty.

  This scared the South so much that a compromise needed to be made.
  The compromise stated that California would come in as a free state, the New York slave trade would end, and that escaped slaves that fled north could be reclaimed.
  Taylor was going to veto this compromise, but dies before he can do so.

\item[Millard Fillmoer] Taylor's vice president would sign this, although he splits some of them up first, and each section passes separately.

\item[Fugitive Slave Act] This act allowed a Southern slaveholder to come North, identify their escaped slave, have a judge hear their claim, and reclaim their slaves.
  This also meant that the government must arrest slaves before this trial with a judge.

  Many slaves now had to flee to Canada, rather than just the Northern US.\@
  In addition, this scares free blacks, making 20,000 such people flee to Canada.

\item[Harriet Beecher Stowe] A white woman who visited a Southern plantation for several hours.
  She gave ``first hand experience'' of a slave plantation.
  She wrote \textit{Uncle Tom's Cabin}.

\item[\textit{Uncle Tom’s Cabin}] A book that was adapted to a play that had a huge influence on the North's impression of slavery.
  Because this was a play, even the common person, who might not be able to read or write, could learn and hear the story.

\item[Franklin Pierce] Democrat who won the 1852 presidential election.
  He was part of the Young America movement.
  Really all he did was distract the US from the topic of slavery, but does so poorly.

\item[Young America Movement] This movement is another that handled American Exceptionalism.
  Here, the younger Americans want to spread America and its ideals to the rest of the world.
  They believe that US life is the best in the world, and it should be spread to Central and South America.

  He also expands the American Southwest by purchase.

\item[Stephen Douglas]
\item[Kansas-Nebraska Act]
\item[Charles Sumner]
\item[Preston Brooks]
\item[Know Nothing Party]
\item[Republican Party]
\item[James Buchanan]
\item[John C. Fremont]
\item[Millard Fillmore]
\item[Abraham Lincoln]
\item[Dred Scott]
\end{description}

% To make this print, you must include a citation somewhere in the document
\clearpage
\printbibliography{}
\end{document}

%%% Local Variables:
%%% mode: latex
%%% TeX-master: t
%%% End:
