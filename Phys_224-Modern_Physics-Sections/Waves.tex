\section{Waves} \label{sec:Waves}
{\Large Usually of form $y = y_{m} \sin \left( kx \pm \omega t \right)$}
There are two types of waves:
	\begin{enumerate}[noitemsep, nolistsep]
		\item Transverse Waves - Waves where displacement from equilibrium is orthogonal to direction of propagation
		\begin{itemize}[noitemsep, nolistsep] % Examples of Transverse Waves
			\item String Waves
			\item Electromagnetic Waves
		\end{itemize}
		\item Longitudinal Waves - Waves where displacement from equilibrium is parallel to direction of propagation
		\begin{itemize}[noitemsep, nolistsep] % Examples of Longitudinal Waves
			\item Pressure Waves
			\item Sound Waves (Which are a type of pressure wave)
		\end{itemize}
	\end{enumerate}

	\begin{itemize}[noitemsep]
		\item $y_{m}$ - Amplitude, $m$
		\item $k$ - Angular Wave Number, $rad/m$
			\begin{itemize}[noitemsep, nolistsep]
				\item $k = \frac{2 \pi}{\lambda}$
				\item $\lambda$ is wavelength, $m$
			\end{itemize}
		\item $\omega$ - Angular Frequency, $rad/s$
			\begin{itemize}[noitemsep, nolistsep]
				\item $\omega = 2 \pi f$
				\item $f$ is frequency, $Hz$
				\item Sign of this goes the opposite the direction the wave is going
					\begin{enumerate}[noitemsep]
						\item Wave going in positive direction $\left( + \right)$, then the sign should be negative $\left( - \right)$
						\item Wave going in negative direction $\left( - \right)$, then the sign should be positive $\left( + \right)$
					\end{enumerate}
			\end{itemize}
		\item $v = \lambda f$, Wave Velocity, $m/s$
			\begin{itemize}[noitemsep]
				\item $v = \frac{\omega}{2 \pi} * \frac{2 \pi}{k} = \frac{\omega}{k}$
				\item This can be proven with the angular portion of any wave (inside the parentheses of trig function)
				\begin{align*} % Prove why k = \lambda / 2 \pi
					kx-\omega t &= \text{Constant} \\
					\frac{d}{dt} \left[ kx-\omega t \right] &= \frac{d}{dt} \left[ \text{Constant} \right] \\
					k \frac{dx}{dt} - \omega \frac{dt}{dt} &= 0 \\
					kv - \omega &= 0 \\
					kv &= \omega \\
					v &= \frac{\omega}{k} \\
				\end{align*}
				Since $\omega = 2 \pi f$ , then $k = \frac{\lambda}{2 \pi}$ 
			\end{itemize}
	\end{itemize}

	\subsection{Wave Interference} \label{subsec:Wave Interference}
	Waves are nice, and they just sum when they interfere. Let: 
	\begin{align}
		y_{1} \left( x, t \right) &= y \sin \left( kx - \omega t \right) \\
		y_{2} \left( x, t \right) &= y \sin \left( kx + \omega t + \varphi \right) \\
		Y \left( x, t \right) &= y\left[ \sin \left( kx - \omega t \right) + \sin \left( kx + \omega t + \varphi \right) \right] \label{eq:Sum of 2 Waves}
	\end{align}
	You can usually use Equation~\ref{eq:Sin plus Sin with diff Angles} to simplify Equation~\ref{eq:Sum of 2 Waves}.

		\subsubsection{Constructive/Destructive Interference} \label{subsubsec:Constructive/Destructive Interference}
		\begin{align*}		
			\phi &= 2 \pi \frac{Path Length Diff}{\lambda} = 2 \pi \frac{\Delta \text{PLD}}{\lambda} \\
			n &= \frac{\phi}{2\pi} = 4 \frac{Path Length Diff}{\lambda}\\
		\end{align*}
		\begin{itemize}[noitemsep]
			\item $Path Length Diff$ - Is the Difference in path lengths that the waves must travel
			\item $\phi$ - Angular Location of points of Complete Constructive/Destructive Interference
			\item $n$ - Number of locations where there is Complete Constructive/Destructive Interference
		\end{itemize}

	\subsection{Standing Waves} \label{subsec:Standing Waves}
	This is actually the superposition of 2 waves, traveling in opposite directions, on a medium that is fixed at both ends, i.e. a taut string held by a wall.
		\subsubsection*{Location of Nodes and Antinodes} \label{subsubsec:Node/Antinode Location}
		\begin{itemize}[noitemsep]
			\item Nodes - $x = n \frac{\lambda}{2}$ for $n = 0, 1, 2, \ldots$
			\begin{itemize}[noitemsep, nolistsep]
				\item Always at closed ends of tubes
			\end{itemize}
			\item Antinodes - $x = \left( n + \frac{1}{2} \right) \frac{\lambda}{2}$ for $n = 0, 1, 2, \ldots$
			\begin{itemize}[noitemsep, nolistsep]
				\item Always at open ends of tubes
			\end{itemize}
		\end{itemize}

		\subsubsection{Resonant Frequencies/Harmonics} \label{subsubsec:Resonant Frequencies/Harmonics}
		These can also be called harmonics. There is a resonant frequency for every number of nodes/antinodes on tthe standing wave. \newline
		$f = \frac{v}{\lambda} = n \frac{v}{2L}$
		\begin{itemize}[noitemsep]
			\item $L$ is the length of the medium (The String).
			\item $\lambda$ is the wavelength of the wave formed.
		\end{itemize}
		This can be extended to find the base resonant frequency, if you know how many node levels are between the two resonant frequencies given, i.e. they say that the \textbf{NEXT} frequency, means $n+1$. \newline
		$f_{n+m}-f_{n} = \left( n+m \right) \frac{v}{2L} - n \frac{v}{2L} = m \frac{v}{2L}$
		
	\subsection{Reflecting Sound} \label{subsec:Reflecting Sound}
		\begin{itemize}[noitemsep]
			\item $D = \left( n + 1 \right)d = vt$
			\begin{itemize}[noitemsep,nolistsep]
				\item $n$ is the number of reflections that occurred
				\item $n+1$ is used when we want the distance the wave covers
			\end{itemize}
		\end{itemize}
	
	\subsection{Sound in Different Mediums} \label{subsec:Sound in Different Mediums}
	Frequency is a property of a wave, and \textbf{CANNOT BE ALTERED.} This means that:
	\begin{align*}
		v &= \lambda f \\
		v_{Sound, Material 1} &= \lambda_{Material 1}f_{Unique, Material 1} \\
		v_{Sound, Material 2} &= \lambda_{Material 2}f_{Unique, Material 2} \\
		f_{Unique, Material 1} &= f_{Unique, Material 2} \\
		\frac{v_{Sound, Material 1}}{\lambda_{Material 1}} &= \frac{v_{Sound, Material 2}}{\lambda_{Material 2}} \\
	\end{align*}
	
	\subsection{Doppler Effect} \label{subsec:Doppler Effect}
	{\Large $f' = f \frac{v \pm v_{D}}{v \pm v_{S}}$}
	\begin{itemize}[noitemsep]
		\item Moving \textit{\textbf{TOWARDS}} each other: Frequency Increase
		\item Moving \textit{\textbf{AWAY}} from each other: Frequency Decrease
		\item $f$ - Initial Frequency, \si{\hertz}
		\item $v$ - Sound Speed, \si{\meter / \second}
		\item $v_{D}$ - Detector Speed, \si{\meter / \second}
		\item $v_{S}$ - Source Speed, \si{\meter / \second}
		\item \textbf{For Numerator}:
			\begin{itemize}[noitemsep,nolistsep]
				\item If detector is moving towards the source, $+$
				\item If detector is moving away from the source, $-$
			\end{itemize}
		\item \textbf{For Denominator}:
			\begin{itemize}[noitemsep,nolistsep]
				\item If source is moving away from detector, $+$
				\item If source is moving towards the detector, $-$
			\end{itemize}
		\end{itemize}
