\section{Light}\label{sec:Light}
\begin{definition}[Photon]
  A \emph{photon} is an electric field (wave) propagating through another electric field.
\end{definition}
The various $\theta$ used in Equations~\eqref{eq:Reflected Light},~\eqref{eq:Refracted Light} are measured from the surface normal.

\emph{Chromatic Dispersion} is the breaking up of polychromatic light by spectra.
Think of Pink Floyd's \textit{Dark Side of the Moon} album cover.
This happens because shorter wavelength, higher frequency light has a slightly higher index of refraction.
\begin{figure}[h!]
  \centering
  \includegraphics[scale=0.5]{./Polychromatic_Light_Indices_Refraction.png}
  \caption{Indices of Refraction for Polychromatic Light}
  \label{fig:Indices of Refraction for Polychromatic Light}
\end{figure}

\subsection{Reflection}\label{subsec:Reflection}
\begin{equation}\label{eq:Reflected Light}
  \theta_{reflected} = \theta_{incident}
\end{equation}

\subsection{Refraction}\label{subsec:Refraction}
\begin{definition}[Snell's Law]\label{def:Snell's Law}
  \begin{equation}\label{eq:Refracted Light}
    n_{refract} \sin \left( \theta_{refract} \right) = n_{incident} \sin \left( \theta_{incident} \right)
  \end{equation}
\end{definition}
\begin{definition}[Index of Refraction]\label{def:Index of Refraction}
  \begin{equation}\label{eq:Index of Refraction}
    \begin{aligned}
      n_{i} &= \frac{c}{v_{i}} \\
      \lambda &= \frac{\lambda_{0}}{n} \\
    \end{aligned}
  \end{equation}
\end{definition}
The total number of waves that pass through a material is given in Equation~\eqref{eq:Number of Waves}.
\begin{equation}\label{eq:Number of Waves}
  \begin{aligned}
    N &= \frac{L}{\lambda_{n}} \\
    &= \frac{L}{\frac{\lambda_{0}}{n}}
    N &= \frac{nL}{\lambda_{0}} \\
  \end{aligned}
\end{equation}

\subsection{Total Internal Reflection}\label{subsec:Total Internal Reflection}
\begin{definition}[Total Internal Reflection]\label{def:Total Internal Reflection}
  Total Internal Reflection occurs when the \emph{refracted light's angle is $\frac{\pi}{2}$}.
  \begin{equation}\label{eq:Total Internal Reflection}
    n_{refract} \sin \left( \theta_{refract} \right) = n_{incident} \sin \left( \theta_{incident} \right) \text{, where } \theta_{refract} = \frac{\pi}{2}
  \end{equation}
\end{definition}

\subsection{Interference and Diffraction}\label{subsec:Interference and Diffraction}
\begin{definition}[Huygen's Principle]\label{def:Huygen's Principle}
  Any point on a plane wavefront can be treated as a source of outgoing spherical waves.
  \begin{note}
    This is a mathematical construct/model.
  \end{note}
\end{definition}
\begin{definition}[Phase Difference]\label{def:Phase Difference}
  Waves from the same source, but measured in such a way that there is a Path Length Difference (PLD) between them will have a \emph{phase difference}.
  \begin{equation}\label{eq:Phase Difference}
    \varphi = \frac{2 \pi}{\lambda} \left( \PLD \right)
  \end{equation}
\end{definition}
All \nameref{subsec:Interference and Diffraction} equations used from here on out are based on Equation~\eqref{eq:Phase Difference}.
\begin{definition}[Intensity]\label{def:Intensity}
  \emph{Intensity} of an electromagnetic wave (light) is proportional to the square of the amplitude of the electric field.
  \begin{equation}\label{eq:Intensity}
    \begin{aligned}
      \Intensity &= E^{2} = \sin^{2} \left( \alpha \right) \\
      \Intensity_{0} &= \alpha^{2} \\
    \end{aligned}
  \end{equation}
\end{definition}

\subsubsection{Single Slit Interference}\label{subsubsec:Single Slit Interference}
\begin{definition}[Single Slit Interference]\label{def:Single Slit Interference}
  When waves propagate through a single slit, there is a Probability Distribution Function that describes where intensity of light is greatest and smallest.
  The location of the \emph{\textbf{minima}} are given in Equation~\eqref{eq:Minima of Single Slit Interference}.
  This phenomena occurs because \nameref{def:Huygen's Principle} says a wave has infinite sources of light that radiate spherically.
  \begin{equation}\label{eq:Minima of Single Slit Interference}
    m \lambda = a \sin \left( \theta_{m} \right)
  \end{equation}
  \begin{itemize}
  \item $m$ is the number of minima from the central major distribution of intensity.
    \begin{enumerate}[noitemsep, nolistsep]
    \item First minima from central intensity point means $m=1$
    \item Second minima from central intensity point means $m=2$
    \item etc.
    \end{enumerate}
  \item $\lambda$ is the wavelength of the incoming wave
  \item $a$ is the width of the single slit
  \item $\theta_{m}$ is the angle formed by the diffracted light from the normal
  \end{itemize}
  Thus, the equation for the \nameref{eq:Phase Difference of Single Slit Interference} is given in Equation~\eqref{eq:Phase Difference of Single Slit Interference}.
  \begin{equation}\label{eq:Phase Difference of Single Slit Interference}
    \varphi_{\text{Single Slit}} = \frac{2 \pi}{\lambda} a \sin \left( \theta \right) \si{\radian}
  \end{equation}
  There is a connection between the angle $\theta$ that locates a point on the screen with the light intensity $I\left( \theta \right)$ at that point.
  \begin{equation}\label{eq:Alpha for Single Slit Interference}
    \begin{aligned}
      \alpha &= \frac{1}{2} \phi \\
      \alpha &= m \pi \\
      \alpha &= \frac{\pi}{\lambda} a\sin \left( \theta \right) \\
    \end{aligned}
  \end{equation}
\end{definition}

\subsubsection{Double Slit Diffraction}\label{subsubsec:Double Slit Diffraction}
\begin{definition}[Double Slit Diffraction]\label{def:Double Slit Diffraction}
  The \emph{\textbf{maxima}} of the interference is given in Equation.
  \begin{equation}\label{eq:Maxima of Double Slit Diffraction}
    n \lambda = d \sin \left( \theta_{n} \right)
  \end{equation}
  \begin{itemize}[noitemsep, nolistsep]
  \item $n$ is the number of maxima from the central major distribution of intensity.
    \begin{enumerate}[noitemsep, nolistsep]
    \item Central maxima means $n=0$
    \item First maxima means $n=\pm 1$
    \item etc.
    \end{enumerate}
  \item $d$ is the distance between the slits
  \item $\lambda$ is the incoming wave's wavelength
  \item $\theta_{n}$ is the phase difference from the normal by the diffracted light
  \end{itemize}
  Thus, the equation for the \nameref{eq:Phase Difference of Double Slit Refraction} is given in Equation~\eqref{eq:Phase Difference of Double Slit Refraction}.
  \begin{equation}\label{eq:Phase Difference of Double Slit Refraction}
    \varphi_{\text{Double Slit}} = \frac{2 \pi}{\lambda} d \sin \left( \theta \right) \si{\radian}
  \end{equation}
\end{definition}

%%% Local Variables:
%%% mode: latex
%%% TeX-master: "../Phys_224-Modern_Physics-Reference_Sheet"
%%% End:
