\section{Entropy}\label{sec:Entropy}
There is a heavy relationship between this and the \nameref{def:2nd Law of Thermo}.
\begin{definition}[Entropy]\label{def:Entropy}
  \emph{Entropy} is a measure of the number of available states/configurations.
  \emph{Entropy} can be thought of as a measure of the probability of an energy distribution in materials.
  Entropy states that energy will want ot end in its most probable state, where it is most event distributed.
  The \emph{units for entropy} are \si{\joule / \kelvin}.
  \begin{equation}\label{eq:Entropy}
    S = k_{B} \ln \left( \Omega \right)
  \end{equation}
  \begin{note}
    \textbf{Entropy is NOT disorder.}
  \end{note}
\end{definition}
\begin{definition}[Change in Entropy]\label{def:Change in Entropy}
  \emph{Change in entropy} is the change in the number of available states/configurations for energy in a system.
  \begin{equation}\label{eq:Change in Entropy}
    \begin{aligned}
      \Delta S &= S_{f} - S_{i} \\
      &= \int_{a}^{b} \frac{dQ \left( T \right)}{T} \\
      &= nR \ln \left( \frac{V_{f}}{V_{i}} \right) \\
      &= mC \ln \left( \frac{T_{f}}{T_{i}} \right) \\
    \end{aligned}
  \end{equation}
  \begin{note}
    Because $\Delta S$ is based off of heat,
    \begin{equation}
      \Delta S_{Total} = \sum_{states} \Delta S
    \end{equation}
  \end{note}
\end{definition}

\subsection{Engines}\label{subsec:Engines}
Engines are inherently cyclical.
\begin{definition}[Efficiency]\label{def:Efficiency}
  \emph{Efficiency} is a measure of how effect an engine is in turning the heat taken in and turning it into work out.
  \begin{equation}\label{eq:Efficiency}
    \begin{aligned}
      \text{Efficiency} &= \frac{W_{Out}}{Q_{In}} \\
      \text{Eff} &= \frac{Q_{In}-Q_{Out}}{Q_{In}} \\
      \text{Eff} &= \frac{T_{Hot}-T_{Cold}}{T_{Hot}} \\
    \end{aligned}
  \end{equation}
\end{definition}

\subsubsection{Stirling Engine}\label{subsubsec:Stirling Engine}
Based on the \nameref{eq:Ideal Gas Law}.
You can find \nameref{def:Efficiency} with Equation~\eqref{eq:Efficiency}.

\subsubsection{Carnot Engine}\label{subsubsec:Carnot Engine}
Uses Isothermal expansion and adiabatic expansion of gases to achieve work.
The total work done is the area of the Carnot figure.
You can find \nameref{def:Efficiency} with Equation~\eqref{eq:Efficiency}.

\begin{definition}[Coefficient of Performance]\label{def:Coefficient of Performance}
  \emph{Coefficient of performance} is a measurement of the amount of energy a Carnot engine moves in comparison to its cold reservoir.
  \begin{equation}\label{eq:Coefficient of Performance}
    \begin{aligned}
      K_{C} &= \frac{T_{L}}{T_{H} - T_{L}} \\
      K_{C} &= \frac{\lvert Q_{L} \rvert}{\lvert Q_{H} \rvert - \lvert Q_{L} \rvert} \\
      K &= \frac{\lvert Q_{L} \rvert}{\lvert W \rvert} \\
    \end{aligned}
  \end{equation}
\end{definition}