\documentclass[10pt,letterpaper,final,twoside,notitlepage]{article}
\usepackage[margin=.5in]{geometry} % 1/2 inch margins on all pages
\usepackage[utf8]{inputenc} % Define the input encoding
\usepackage[USenglish]{babel} % Define language used
\usepackage{amsmath}
\usepackage{amsfonts}
\usepackage{amssymb}
\usepackage{amsthm} % Gives us plain, definition, and remark to use in \theoremstyle{style}
\usepackage{graphicx}

\usepackage{hyperref} % Generate hyperlinks to referenced items
\usepackage[noabbrev,nameinlink]{cleveref} % Fancy cross-references in the document everywhere
\usepackage{nameref} % Can make references by name to places
\usepackage{subcaption} % Allows for multiple figures in one Figure environment
\usepackage{siunitx} % Gives us ways to typeset units for stuff
\usepackage{enumitem} % Provides [noitemsep, nolistsep] for more compact lists
\usepackage{chngcntr} % Allows us to tamper with the counter a little more
\usepackage{empheq} % Allow boxing of equations in special math environments
\usepackage{tcolorbox} % Allows us to create boxes of various types for examples
\usepackage{tikz} % Allows us to create TikZ and PGF Pictures
\usepackage{ctable} % Greater control over tables and how they look
% \usepackage{minted} % Allow us to nicely typeset 300+ programming languages

\counterwithin{equation}{section}
\setcounter{secnumdepth}{4}
\setcounter{tocdepth}{4} % Include \paragraph in toc

% Create a theorem environment
\theoremstyle{plain}
\newtheorem{theorem}{Theorem}

% Create a definition environment
\theoremstyle{definition}
\newtheorem{definition}{Defn}
\newtheorem{corollary}{Corollary}[section]
% \begin{definition}[Term] \label{def:}
% 		Make sure the term is emphasized with \emph{term}.
%		This ensures that if \emph is changed, it shows up everywhere
% \end{definition}

% Create a numbered remark environment numbered based on definition
% NOTE: This version of remark MUST go inside a definition environment
\theoremstyle{remark}
\newtheorem{remark}{Remark}[definition]
%\counterwithin{definition}{subsection} % Uncomment to have definitions use section.subsection numbering

% Create an unnumbered remark environment for general use
% NOTE: This version of remark has NO restrictions on placement
\newtheorem*{remark*}{Remark}

% Create a special list that handles properties. It can be broken and restarted
\newlist{propertylist}{enumerate}{1} % {Name}{Template}{Max Depth}
\setlist[propertylist, 1]{label=\textbf{(\roman*)}, noitemsep, nolistsep} % Set options

% Create a special list that handles enumerate starting with lower letters. Breakable/Restartable.
\newlist{boldalphlist}{enumerate}{1} % {Name}{Template}{Max Depth}
\setlist[boldalphlist, 1]{label=\textbf{(\alph*)}, ref=\alph*, noitemsep, nolistsep} % Set options

% Create a tcolorbox for examples
\newtcolorbox[auto counter,
number within=section,
number format=\arabic,
crefname={example}{examples}, % Define reference format for cref (No Capitals)
Crefname={Example}{Examples}, % Reference format for cleveref (With Capitals)
]{example}[2][]{
  width=\textwidth,
  title={Example \thetcbcounter: #2. #1},
  fonttitle=\bfseries,
  label={ex:#2},
  nameref=#2,
  colbacktitle=white!100!black,
  coltitle=black!100!white,
  colback=white!100!black,
  upperbox=visible,
  lowerbox=visible,
  sharp corners=all
}

% Create a tcolorbox for general use
\newtcolorbox[% auto counter,
% number within=section,
% number format=\arabic,
% crefname={example}{examples}, % Define reference format for cref (No Capitals)
% Crefname={Example}{Examples}, % Reference format for cleveref (With Capitals)
]{blackbox}{
  width=\textwidth,
  % title={},
  fonttitle=\bfseries,
  % label={},
  % nameref=,
  colbacktitle=white!100!black,
  coltitle=black!100!white,
  colback=white!100!black,
  upperbox=visible,
  lowerbox=visible,
  sharp corners=all
}

% Redefine the 'end of proof' symbol to be a black square, not blank
\renewcommand\qedsymbol{$\blacksquare$} % Change proofs to have black square at end

% Math Operators that are useful to abstract the written math away to one spot
\DeclareMathOperator{\RealNums}{\mathbb{R}}
\DeclareMathOperator{\AllIntegers}{\mathbb{Z}}
\DeclareMathOperator{\PositiveInts}{\mathbb{Z}^{+}}
\DeclareMathOperator{\NegativeInts}{\mathbb{Z}^{-}}
\DeclareMathOperator{\NaturalNums}{\mathbb{N}}
\DeclareMathOperator{\TimeDelay}{\mathrm{TD}_{k}}
\DeclareMathOperator{\FoldTime}{\mathrm{FD}}
\DeclareMathOperator{\SignalOperator}{\mathcal{T}}
\DeclareMathOperator*{\argmax}{argmax} % Thin Space and subscripts are UNDER in display

\begin{titlepage}
  \title{EITF75: Systems and Signals - Reference Sheet}
  \author{Karl Hallsby}
  \date{Last Edited: \today} % We want to inform people when this document was last edited
\end{titlepage}

\begin{document}
\pagenumbering{gobble}
\maketitle
\pagenumbering{roman} % i, ii, iii on beginning pages, that don't have content
\tableofcontents
\clearpage
\pagenumbering{arabic} % 1,2,3 on content pages

\section{Sinusoids}\label{sec:Sinusoids}
There are several ways to characterize \nameref{sec:Sinusoids}.
The first is by dimension:
\begin{enumerate}[noitemsep]
\item Multidimensional/Multichannel Signals
\item Monodimensional/Monochannel Signals
\end{enumerate}

You can also classify sinusoids by their independent variable (usually time) and the values they take.
\begin{enumerate}[noitemsep]
\item \nameref{def:Continuous-Time Signals} or Analog Signals
\item \nameref{def:Discrete-Time Signals}
\end{enumerate}

\begin{definition}[Continuous-Time Signals]\label{def:Continuous-Time Signals}
  \emph{Continuous-time signals} or \emph{Analog signals} are defined for every value of time and they take on values in the continuous interval $(a,b)$, where $a$ can be $-\infty$ and $b$ can be $\infty$.
  Mathematically, these signals can be described by functions of a continuous variable.

  For example,
  \begin{equation*}
    x_{1}(t) = \cos \pi t \text{, } x_{2}(t) = e^{-\lvert t \rvert} \text{, } -\infty < t < \infty
  \end{equation*}
\end{definition}

\begin{definition}[Discrete-Time Signals]\label{def:Discrete-Time Signals}
  \emph{Discrete-time signals} are defined only at certain specifed values of time.
  These time instants \textbf{\emph{need not}} be equidistant, but in practice, they are usually taken at equally speced intervals for computation convenience and mathematical tractability.

  For example,
  \begin{equation*}
    x(t_{n}) = e^{-\lvert t_{n} \rvert} \text{, } n=0, \pm 1, \pm 2, \ldots
  \end{equation*}

  A \nameref{def:Discrete-Time Signals} can be represented mathematically by a sequence of real or complex numbers.
  \begin{remark}
    To emphasize the discrete-time nature of the signal, we shall denote the signal as $x(n)$, rather than $x(t)$.
  \end{remark}
  \begin{remark}
    If the time instants $t_{n}$ are equally spaced (i.e., $t_{n}=nT$), the notation $x(nT)$ is also used.
  \end{remark}
\end{definition}

\subsection{Continuous-Time Signals}\label{subsec:Continuous-Time Signals}
\subsubsection{Frequency in Continuous-Time Signals}\label{subsubsec:Frequency in Continuous-Time Signals}
A simple harmonic oscillation is mathematically described by \Cref{eq:Simple Harmonic Oscillation}.

\begin{equation}\label{eq:Simple Harmonic Oscillation}
  x_{a}(t) = A \cos \left( \Omega t + \theta \right) \text{, } -\infty < t < \infty
\end{equation}

\begin{remark*}
  The subscript $a$ is used with $x(t)$ to denote an analog signal.
\end{remark*}

This signal is completely characterized by three parameters:
\begin{enumerate}[noitemsep]
\item $A$, the \emph{amplitude} of the sinusoid
\item $\Omega$, the \emph{frequency} in radians per second (\si{\radian / \second})
\item $\theta$, the \emph{phase} in radians.
\end{enumerate}

Instead of $\Omega$, the frequency $F$ in cycles per second or hertz (\si{\hertz}) is used.
\begin{equation}\label{eq:Continuous Angular Frequency to Frequency}
  \Omega = 2 \pi F
\end{equation}

Plugging~\eqref{eq:Continuous Angular Frequency to Frequency} into~\eqref{eq:Simple Harmonic Oscillation}, yields
\begin{equation}\label{eq:Frequency Harmonic Oscillation}
  x_{a}(t) = A \cos \left( 2 \pi Ft + \theta \right) \text{, } -\infty < t < \infty
\end{equation}

\subsubsection{Properties of Continuous-Time Sinusoidal Signals}\label{subsubsec:Properties Continuous-Time Sinusoids}
The analog sinusoidal signal in \cref{eq:Frequency Harmonic Oscillation} is characterized by the following properties:
\begin{propertylist}
\item For every fixed value of the frequency $F$, $x_{a}(t)$ is periodic.
  \begin{equation*}
    x_{a}(t+T_{p}) = x_{a}(t)
  \end{equation*}
  where $T_{p} = \frac{1}{F}$ is the fundamental period.
\item Continuous-time sinusoidal signals with distinct (different) frequencies are themselves distinct.
\item Increasing the frequency $F$ results in an increase in the rate of oscillation of the signal, in the sense that more periods are included in the given time interval.
\end{propertylist}

\subsection{Discrete-Time Signals}\label{subsec:Discrete-Time Signals}
\subsubsection{Frequency in Discrete-Time Signals}\label{subsubsec:Frequency in Discrete-Time Signals}
A discrete-time sinusoidal signal may be expressed as
\begin{equation}\label{eq:Discrete Time Sinusoid}
  x(n) = A \cos \left( \omega n + \theta \right) \text{, } n \in \mathbb{Z} \text{, } -\infty < n < \infty
\end{equation}

The signal is characterized by these parameters:
\begin{enumerate}[noitemsep]
\item $n$, the sample number. MUST be an integer.
\item $A$, the \emph{amplitude} of the sinusoid
\item $\omega$, the \emph{angular frequency} in radians per sample
\item $\theta$, is the \emph{phase}, in radians.
\end{enumerate}

Instead of $\omega$, we use the frequency variable $f$ defined by
\begin{equation}\label{eq:Discrete Angular Frequency to Frequency}
  \omega \equiv 2 \pi f
\end{equation}

Using~\eqref{eq:Discrete Time Sinusoid} and~\eqref{eq:Discrete Angular Frequency to Frequency} yields
\begin{equation}\label{eq:Discrete Frequency Sinusoid}
  x(n) = A \cos \left( 2 \pi fn + \theta \right) \text{, } n \in \mathbb{Z} \text{, } -\infty < n < \infty
\end{equation}

\subsubsection{Properties of Discrete-Time Sinusoidal Signals}\label{subsubsec:Properties Discrete-Time Sinusoids}
\begin{propertylist}
\item A discrete-time sinusoid is periodic \textbf{\emph{ONLY}} if its frequency is a rational number.
\item Discrete-time sinusoids whose frequencies are separated by an integer multiple of $2\pi$ are identical.
\item The highest rate of oscillation in a discrete-time sinusoid is attained when $\omega = \pm \pi$ or, equivalently, $f= \pm \frac{1}{2}$.
\end{propertylist}

%%% Local Variables:
%%% mode: latex
%%% TeX-master: "../EITF75-Systems_and_Signals-Reference_Sheet"
%%% End:


\section{Discrete-Time Systems}\label{sec:Discrete-Time Systems}
As discussed in \Cref{subsec:Discrete-Time Signals}, $x(n)$ is a function of an independent variable that is an integer.
It is important to note that a discrete-time signal is \emph{not defined} at instants between the samples.
Also, if $n$ is not an integer, $x(n)$ is not defined.

Besides graphical representation of a discrete-time system, there are 3 ways to represent a discrete-time signal.
\begin{enumerate}[noitemsep]
\item \nameref{subsubsec:Functional Representation}
\item \nameref{subsubsec:Tabular Representation}
\item \nameref{subsubsec:Sequence Representation}
\end{enumerate}

\subsection{Representing Discrete-Time Systems}\label{subsec:Representing Discrete-Time Systems}
\subsubsection{Functional Representation}\label{subsubsec:Functional Representation}
This representation of a discrete-time system is done as a mathematical function.
\begin{equation}\label{eq:Functional Representation}
  x(n) = \begin{cases}
    1 ,& \text{for } n = 1,3 \\
    4 ,& \text{for } n = 2 \\
    0 ,& \text{elsewhere}
  \end{cases}
\end{equation}

\subsubsection{Tabular Representation}\label{subsubsec:Tabular Representation}
This representation of a discrete-time sysem is done as a table of corresponding values.
\begin{table}[h!]
  \centering
  \begin{tabular}{c|cccccccccc}
    $n$ & $\ldots$ & -2 & -1 & 0 & 1 & 2 & 3 & 4 & 5 & $\ldots$ \\ \midrule
    $x(n)$ & $\ldots$ & 0 & 0 & 0 & 1 & 4 & 1 & 0 & 0 & $\ldots$
  \end{tabular}
\end{table}

\subsubsection{Sequence Representation}\label{subsubsec:Sequence Representation}
There are 2 methods of representation for this.
The first includes all values for $-\infty < n < \infty$.
In all cases, $n=0$ is marked in the sequence, somehow.
I will do this with an underline.
\begin{equation}\label{eq:Infinite Sequence Representation}
  x(n) = \lbrace \ldots, 0, \underline{0}, 1, 4, 1, 0, 0, \ldots \rbrace
\end{equation}

The second only works if all $x(n)$ values for $n < 0$ are 0.
\begin{equation}\label{eq:Zero Sequence Representation}
  x(n) = \lbrace \underline{0}, 1, 4, 1, 0, 0, \ldots \rbrace
\end{equation}

A finite-duration sequence can be represented as
\begin{equation}\label{eq:Finite Sequence Representation}
  x(n) = \lbrace 3, -1, \underline{-2}, 5, 0, 4, -1 \rbrace
\end{equation}
This is identified as a seven-point sequence.

A finite-duration sequence where $x(n)=0$ for all $n<0$ is represented as
\begin{equation}\label{eq:Zero Finite Sequence Representation}
  x(n) = \lbrace \underline{0}, 1, 4, 1 \rbrace
\end{equation}
This is identified as a four-point sequence.

\subsection{Elementary Discrete-Time Signals}\label{subsec:Elementary Discrete-Time Signals}
The following signals are basic signals that appear often and play an important role in signal processing.

\subsubsection{Unit Impulse Signal}\label{subsubsec:Unit Impulse Signal}
\begin{definition}[Unit Impulse Signal]\label{def:Unit Impulse Signal}
  The \emph{unit impulse signal} or \emph{unit sample sequence} is denoted as $\delta(n)$ and is defined as
  \begin{equation}\label{eq:Unit Impulse Signal}
    \delta(n) \equiv = \begin{cases}
      1, & \text{for } n = 0 \\
      0, & \text{for } n \neq 0
    \end{cases}
  \end{equation}

  This function is a signal that is zero everywhere, except at $n=0$, where its value is $1$.

  \begin{remark}
    This signal is different that the analog signal $\delta (t)$, which is also called a unit impulse, and is defined to be 0 everywhere except $t=0$.
    The discrete unit impulse sequence is much less mathematically complicated.
  \end{remark}
\end{definition}

\subsubsection{Unit Step Signal}\label{subsubsec:Unit Step Signal}
\begin{definition}[Unit Step Signal]\label{def:Unit Step Signal}
  The \emph{unit step signal} is denoted as $u(n)$ and is defined as
  \begin{equation}\label{eq:Unit Step Signal}
    u(n) \equiv \begin{cases}
      1, & \text{for } n \geq 0 \\
      0, & \text{for } n < 0
    \end{cases}
  \end{equation}
\end{definition}

\subsubsection{Unit Ramp Signal}\label{subsubsec:Unit Ramp Signal}
\begin{definition}[Unit Ramp Signal]\label{def:Unit Ramp Signal}
  The \emph{unit ramp signal} is denoted as $u_{r}(n)$ and is defined as
  \begin{equation}\label{eq:Unit Ramp Signal}
    u_{r}(n) \equiv \begin{cases}
      n, & \text{for } n \geq 0 \\
      0, & \text{for } n < 0
    \end{cases}
  \end{equation}
\end{definition}

\subsubsection{Exponential Signal}\label{subsubsec:Exponential Signal}
\begin{definition}[Exponential Signal]\label{def:Exponential Signal}
  The \emph{exponential signal} is a sequence of the form
  \begin{equation}\label{eq:Exponential Signal}
    x(n) = a^{n} \>\> \text{for all } n
  \end{equation}

  If $a$ is real, then $x(n)$ is a real signal.
  When $a$ is complex valued ($a \equiv b \pm c j$), it can be expressed as
   \begin{equation}\label{eq:Complex Exponential Signal}
     \begin{aligned}
       x(n) &= r^{n} e^{j \theta n} \\
       &= r^{n} \left( \cos \theta n + j \sin \theta n \right)
     \end{aligned}
   \end{equation}
  This can be expressed by graphing the real and imaginary parts
  \begin{equation}\label{eq:Real Imaginary Complex Exponential Signal}
    \begin{aligned}
      x_{R}(n) &\equiv r^{n} \cos \theta n \\
      x_{I}(n) &\equiv r^{n} j \sin \theta n
    \end{aligned}
  \end{equation}
  or by graphing the amplitude function and phase function.
  \begin{equation}\label{eq:Amplitude Phase Complex Exponential Signal}
    \begin{aligned}
      \lvert x(n) \rvert &= A(n) \equiv r^{n} \\
      \angle x(n) &= \phi(n) \equiv \theta n
    \end{aligned}
  \end{equation}
\end{definition}

\subsection{Classification of Discrete-Time Signals}\label{subsec:Classification Discrete-Time Signals}
\subsubsection{Energy Signal}\label{subsubsec:Energy Signal}
\subsubsection{Power Signal}\label{subsubsec:Power Signal}
\subsubsection{Periodic and Aperiodic Signals}\label{subsubsec:Periodic Aperiodic Signals}
\subsubsection{Symmetric and Antisymmetric Signals}\label{subsubsec:Symmetric and Antisymmetric Signals}
%%% Local Variables:
%%% mode: latex
%%% TeX-master: "../EITF75-Systems_and_Signals-Reference_Sheet"
%%% End:


\section{Convolutions}\label{sec:Convolutions}
\begin{definition}[Convolution]\label{def:Convolution}
  The \emph{convolution} operator.

  \begin{equation}\label{eq:Convolution}
    y(t) = \sum\limits_{k=-\infty}^{\infty} x(k) * h(n-k)
  \end{equation}
\end{definition}

If there is a relaxed \nameref{def:Linear_Time-Invariant} system to an input $x(n)$, then the output can be found by computing the \nameref{def:Convolution} of the input with the sample response on the system.
This results in the equation shown below.
\begin{equation}\label{eq:LTI_System_Convolution}
  y(n) = x(n) * h(n)
\end{equation}

\subsection{Properties of the Convolution}\label{subsec:Convolution_Properties}
\begin{table}[h!]
  \centering
  \begin{tabular}{cc}
    \toprule
    \nameref{subsubsec:Convolution_Property-Identity} & $y(n) = x(n) * \delta(n) = x(n)$ \\
    \nameref{subsubsec:Convolution_Property-Shifting} & $x(n) * \delta(n-k) = y(n-k) = x(n-k)$ \\
    \nameref{subsubsec:Convolution_Property-Commutative} & $x(n) * h(n) = h(n) * x(n)$ \\
    \nameref{subsubsec:Convolution_Property-Associative} & $\left[ x(n) * h_{1}(n) \right] * h_{2}(n) = x(n) * \left[ h_{1}(n) * h_{2}(n) \right]$\\
    \nameref{subsubsec:Convolution_Property-Distributive} & $x(n) * \left[ h_{1}(n) + h_{2}(n) \right] = x(n) * h_{1}(n) + x(n) * h_{2}(n)$ \\
    \bottomrule
  \end{tabular}
  \caption{\nameref{subsec:Convolution_Properties}}
  \label{tab:Convolution_Properties}
\end{table}

\subsubsection{Identity Property}\label{subsubsec:Convolution_Property-Identity}
\begin{definition}[Identity Property]\label{def:Convolution_Property-Identity}
  The \nameref{def:Unit Impulse Signal} is the identity element for the \nameref{def:Convolution}.
  \begin{equation}\label{eq:Convolution_Property-Identity}
    y(n) = x(n) * \delta(n) = x(n)
  \end{equation}
\end{definition}

\subsubsection{Shifting Property}\label{subsubsec:Convolution_Property-Shifting}
\begin{definition}[Shifting Property]\label{def:Convolution_Property-Shifting}
  Since the $\delta(n)$ function is the Identity function, if we shift $\delta(n)$ by $k$, the convolution sequence is also shifted by $k$.
  \begin{equation}\label{eq:Convolution_Property-Shifting}
    x(n) * \delta(n-k) = y(n-k) = x(n-k)
  \end{equation}
\end{definition}

\subsubsection{Commutative Law}\label{subsubsec:Convolution_Property-Commutative}
\begin{definition}[Commutative Law for Convolutions]\label{def:Convolution_Property-Commutative}
  The \emph{commutative law for \nameref{def:Convolution}s} is just like many other operations.
  \begin{equation}\label{eq:Convolution_Property-Commutative}
    x(n) * h(n) = h(n) * x(n)
  \end{equation}
\end{definition}

\subsubsection{Associative Law}\label{subsubsec:Convolution_Property-Associative}
\begin{definition}[Associative Law for Convolutions]\label{def:Convolution_Property-Associative}
  The \emph{associative law for \nameref{def:Convolution}s} is just like many other operations.
  \begin{equation}\label{eq:Convolution_Property-Associative}
    \left[ x(n) * h_{1}(n) \right] * h_{2}(n) = x(n) * \left[ h_{1}(n) * h_{2}(n) \right]
  \end{equation}
\end{definition}

\subsubsection{Distributive Law}\label{subsubsec:Convolution_Property-Distributive}
\begin{definition}[Distributive Law for Convolutions]\label{def:Convolution_Property-Distributive}
  The \emph{distributive law for \nameref{def:Convolution}s} is just like many other operations.
  \begin{equation}\label{eq:Convolution_Property-Distributive}
    x(n) * \left[ h_{1}(n) + h_{2}(n) \right] = x(n) * h_{1}(n) + x(n) * h_{2}(n)
  \end{equation}
\end{definition}

%%% Local Variables:
%%% mode: latex
%%% TeX-master: "../EITF75-Systems_and_Signals-Reference_Sheet"
%%% End:


%====================================APPENDIX====================================
\appendix
\counterwithin{definition}{subsection}

\clearpage
\subsection{Trigonometry} \label{app:Trig}
	\subsubsection{Trigonometric Formulas} \label{subsubsec:Trig Formulas}
		\begin{equation} \label{eq:Sin plus Sin with diff Angles}
			\sin \left( \alpha \right) + \sin \left( \beta \right) = 2 \sin \left( \frac{\alpha + \beta}{2} \right) \cos\left( \frac{\alpha - \beta}{2} \right)  
		\end{equation}
		\begin{equation} \label{eq:Cosine-Sine Product}
			\cos \left( \theta \right) \sin \left( \theta \right) = \frac{1}{2} \sin \left( 2 \theta \right)
		\end{equation}

\clearpage
\subsection{Calculus} \label{app:Calculus}
	\subsubsection{Fundamental Theorems of Calculus} \label{subsubsec:Fundamental Theorem of Calculus}
		\begin{definition}[First Fundamental Theorem of Calculus] \label{def:1st Fundamental Theorem of Calculus}
			The \emph{first fundamental theorem of calculus} states that, if $f$ is continuous on the closed interval $\left[ a,b \right]$ and $F$ is the indefinite integral of $f$ on $\left[ a,b \right]$, then 
			\begin{equation} \label{eq:1st Fundamental Theorem of Calculus}
				\int_{a}^{b}f \left( x \right) dx = F \left( b \right) - F \left( a \right)
			\end{equation}
		\end{definition}
		\begin{definition}[Second Fundamental Theorem of Calculus] \label{def:2nd Fundamental Theorem of Calculus}
			The \emph{second fundamental theorem of calculus} holds for $f$ a continuous function on an open interval $I$ and $a$ any point in $I$, and states that if $F$ is defined by
			\begin{equation*}
				F \left( x \right) = \int_{a}^{x} f \left( t \right) dt,
			\end{equation*}
			then
			\begin{equation} \label{eq:2nd Fundamental Theorem of Calculus}
				\begin{aligned}
					\frac{d}{dx} \int_{a}^{x} f \left( t \right) dt &= f \left( x \right) \\
					F' \left( x \right) &= f \left( x \right) \\
				\end{aligned}
			\end{equation}
		\end{definition}

\end{document}
%%% Local Variables:
%%% mode: latex
%%% TeX-master: t
%%% End:
