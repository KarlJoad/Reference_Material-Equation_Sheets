\documentclass[10pt,letterpaper,final,twoside,notitlepage]{article}
\usepackage[margin=.5in]{geometry} % 1/2 inch margins on all pages
\usepackage[utf8]{inputenc} % Define the input encoding
\usepackage[USenglish]{babel} % Define language used
\usepackage{amsmath}
\usepackage{amsfonts}
\usepackage{amssymb}
\usepackage{amsthm} % Gives us plain, definition, and remark to use in \theoremstyle{style}
\usepackage{graphicx}

\usepackage{hyperref} % Generate hyperlinks to referenced items
\usepackage[noabbrev,nameinlink]{cleveref} % Fancy cross-references in the document everywhere
\usepackage{nameref} % Can make references by name to places
\usepackage{subcaption} % Allows for multiple figures in one Figure environment
\usepackage{siunitx} % Gives us ways to typeset units for stuff
\usepackage{enumitem} % Provides [noitemsep, nolistsep] for more compact lists
\usepackage{chngcntr} % Allows us to tamper with the counter a little more
\usepackage{empheq} % Allow boxing of equations in special math environments
\usepackage{tcolorbox} % Allows us to create boxes of various types for examples
\usepackage{tikz} % Allows us to create TikZ and PGF Pictures
\usepackage{ctable} % Greater control over tables and how they look
% \usepackage{minted} % Allow us to nicely typeset 300+ programming languages

\counterwithin{equation}{section}
\setcounter{secnumdepth}{4}
\setcounter{tocdepth}{4} % Include \paragraph in toc

% Create a theorem environment
\theoremstyle{plain}
\newtheorem{theorem}{Theorem}

% Create a definition environment
\theoremstyle{definition}
\newtheorem{definition}{Defn}
\newtheorem{corollary}{Corollary}[section]
% \begin{definition}[Term] \label{def:}
% 		Make sure the term is emphasized with \emph{term}.
%		This ensures that if \emph is changed, it shows up everywhere
% \end{definition}

% Create a numbered remark environment numbered based on definition
% NOTE: This version of remark MUST go inside a definition environment
\theoremstyle{remark}
\newtheorem{remark}{Remark}[definition]
%\counterwithin{definition}{subsection} % Uncomment to have definitions use section.subsection numbering

% Create an unnumbered remark environment for general use
% NOTE: This version of remark has NO restrictions on placement
\newtheorem*{remark*}{Remark}

% Create a special list that handles properties. It can be broken and restarted
\newlist{propertylist}{enumerate}{1} % {Name}{Template}{Max Depth}
\setlist[propertylist, 1]{label=\textbf{(\roman*)}, noitemsep, nolistsep} % Set options

% Create a special list that handles enumerate starting with lower letters. Breakable/Restartable.
\newlist{boldalphlist}{enumerate}{1} % {Name}{Template}{Max Depth}
\setlist[boldalphlist, 1]{label=\textbf{(\alph*)}, ref=\alph*, noitemsep, nolistsep} % Set options

% Create a tcolorbox for examples
\newtcolorbox[auto counter,
number within=section,
number format=\arabic,
crefname={example}{examples}, % Define reference format for cref (No Capitals)
Crefname={Example}{Examples}, % Reference format for cleveref (With Capitals)
]{example}[2][]{
  width=\textwidth,
  title={Example \thetcbcounter: #2. #1},
  fonttitle=\bfseries,
  label={ex:#2},
  nameref=#2,
  colbacktitle=white!100!black,
  coltitle=black!100!white,
  colback=white!100!black,
  upperbox=visible,
  lowerbox=visible,
  sharp corners=all
}

% Create a tcolorbox for general use
\newtcolorbox[% auto counter,
% number within=section,
% number format=\arabic,
% crefname={example}{examples}, % Define reference format for cref (No Capitals)
% Crefname={Example}{Examples}, % Reference format for cleveref (With Capitals)
]{blackbox}{
  width=\textwidth,
  % title={},
  fonttitle=\bfseries,
  % label={},
  % nameref=,
  colbacktitle=white!100!black,
  coltitle=black!100!white,
  colback=white!100!black,
  upperbox=visible,
  lowerbox=visible,
  sharp corners=all
}

% Redefine the 'end of proof' symbol to be a black square, not blank
\renewcommand\qedsymbol{$\blacksquare$} % Change proofs to have black square at end

% Math Operators that are useful to abstract the written math away to one spot
\DeclareMathOperator{\RealNums}{\mathbb{R}}
\DeclareMathOperator{\AllIntegers}{\mathbb{Z}}
\DeclareMathOperator{\PositiveInts}{\mathbb{Z}^{+}}
\DeclareMathOperator{\NegativeInts}{\mathbb{Z}^{-}}
\DeclareMathOperator{\NaturalNums}{\mathbb{N}}
\DeclareMathOperator*{\argmax}{argmax} % Thin Space and subscripts are UNDER in display

\begin{titlepage}
  \title{EITF75: Systems and Signals - Reference Sheet}
  \author{Karl Hallsby}
  \date{Last Edited: \today} % We want to inform people when this document was last edited
\end{titlepage}

\begin{document}
\pagenumbering{gobble}
\maketitle
\pagenumbering{roman} % i, ii, iii on beginning pages, that don't have content
\tableofcontents
\clearpage
\pagenumbering{arabic} % 1,2,3 on content pages

\section{Sinusoids}\label{sec:Sinusoids}
There are several ways to characterize \nameref{sec:Sinusoids}.
The first is by dimension:
\begin{enumerate}[noitemsep]
\item Multidimensional/Multichannel Signals
\item Monodimensional/Monochannel Signals
\end{enumerate}

You can also classify sinusoids by their independent variable (usually time) and the values they take.
\begin{enumerate}[noitemsep]
\item \nameref{def:Continuous-Time Signals} or Analog Signals
\item \nameref{def:Discrete-Time Signals}
\item There is a third way to classify sinusoids and their signals: \nameref{subsec:Digital Signals}
\end{enumerate}

\begin{definition}[Continuous-Time Signals]\label{def:Continuous-Time Signals}
  \emph{Continuous-time signals} or \emph{Analog signals} are defined for every value of time and they take on values in the continuous interval $(a,b)$, where $a$ can be $-\infty$ and $b$ can be $\infty$.
  Mathematically, these signals can be described by functions of a continuous variable.

  For example,
  \begin{equation*}
    x_{1}(t) = \cos \pi t \text{, } x_{2}(t) = e^{-\lvert t \rvert} \text{, } -\infty < t < \infty
  \end{equation*}
\end{definition}

\begin{definition}[Discrete-Time Signals]\label{def:Discrete-Time Signals}
  \emph{Discrete-time signals} are defined only at certain specifed values of time.
  These time instants \textbf{\emph{need not}} be equidistant, but in practice, they are usually taken at equally speced intervals for computation convenience and mathematical tractability.

  For example,
  \begin{equation*}
    x(t_{n}) = e^{-\lvert t_{n} \rvert} \text{, } n=0, \pm 1, \pm 2, \ldots
  \end{equation*}

  A \nameref{def:Discrete-Time Signals} can be represented mathematically by a sequence of real or complex numbers.
  \begin{remark}
    To emphasize the discrete-time nature of the signal, we shall denote the signal as $x(n)$, rather than $x(t)$.
  \end{remark}
  \begin{remark}
    If the time instants $t_{n}$ are equally spaced (i.e., $t_{n}=nT$), the notation $x(nT)$ is also used.
  \end{remark}
\end{definition}

\subsection{Continuous-Time Signals}\label{subsec:Continuous-Time Signals}
\subsubsection{Frequency in Continuous-Time Signals}\label{subsubsec:Frequency in Continuous-Time Signals}
A simple harmonic oscillation is mathematically described by \Cref{eq:Simple Harmonic Oscillation}.

\begin{equation}\label{eq:Simple Harmonic Oscillation}
  x_{a}(t) = A \cos \left( \Omega t + \theta \right) \text{, } -\infty < t < \infty
\end{equation}

\begin{remark*}
  The subscript $a$ is used with $x(t)$ to denote an analog signal.
\end{remark*}

This signal is completely characterized by three parameters:
\begin{enumerate}[noitemsep]
\item $A$, the \emph{amplitude} of the sinusoid
\item $\Omega$, the \emph{frequency} in radians per second (\si{\radian / \second})
\item $\theta$, the \emph{phase} in radians.
\end{enumerate}

Instead of $\Omega$, the frequency $F$ in cycles per second or hertz (\si{\hertz}) is used.
\begin{equation}\label{eq:Continuous Angular Frequency to Frequency}
  \Omega = 2 \pi F
\end{equation}

Plugging~\eqref{eq:Continuous Angular Frequency to Frequency} into~\eqref{eq:Simple Harmonic Oscillation}, yields
\begin{equation}\label{eq:Frequency Harmonic Oscillation}
  x_{a}(t) = A \cos \left( 2 \pi Ft + \theta \right) \text{, } -\infty < t < \infty
\end{equation}

\subsubsection{Properties of Continuous-Time Sinusoidal Signals}\label{subsubsec:Properties Continuous-Time Sinusoids}
The analog sinusoidal signal in \cref{eq:Frequency Harmonic Oscillation} is characterized by the following properties:
\begin{propertylist}
\item For every fixed value of the frequency $F$, $x_{a}(t)$ is periodic.
  \begin{equation*}
    x_{a}(t+T_{p}) = x_{a}(t)
  \end{equation*}
  where $T_{p} = \frac{1}{F}$ is the fundamental period.
\item Continuous-time sinusoidal signals with distinct (different) frequencies are themselves distinct.
\item Increasing the frequency $F$ results in an increase in the rate of oscillation of the signal, in the sense that more periods are included in the given time interval.
\end{propertylist}

\subsection{Discrete-Time Signals}\label{subsec:Discrete-Time Signals}
\subsubsection{Frequency in Discrete-Time Signals}\label{subsubsec:Frequency in Discrete-Time Signals}
A discrete-time sinusoidal signal may be expressed as
\begin{equation}\label{eq:Discrete Time Sinusoid}
  x(n) = A \cos \left( \omega n + \theta \right) \text{, } n \in \AllIntegers \text{, } -\infty < n < \infty
\end{equation}

The signal is characterized by these parameters:
\begin{enumerate}[noitemsep]
\item $n$, the sample number. MUST be an integer.
\item $A$, the \emph{amplitude} of the sinusoid
\item $\omega$, the \emph{angular frequency} in radians per sample
\item $\theta$, is the \emph{phase}, in radians.
\end{enumerate}

Instead of $\omega$, we use the frequency variable $f$ defined by
\begin{equation}\label{eq:Discrete Angular Frequency to Frequency}
  \omega \equiv 2 \pi f
\end{equation}

Using~\eqref{eq:Discrete Time Sinusoid} and~\eqref{eq:Discrete Angular Frequency to Frequency} yields
\begin{equation}\label{eq:Discrete Frequency Sinusoid}
  x(n) = A \cos \left( 2 \pi fn + \theta \right) \text{, } n \in \AllIntegers \text{, } -\infty < n < \infty
\end{equation}

\subsubsection{Properties of Discrete-Time Sinusoidal Signals}\label{subsubsec:Properties Discrete-Time Sinusoids}
\begin{propertylist}
\item A discrete-time sinusoid is periodic \textbf{\emph{ONLY}} if its frequency is a rational number.
\item Discrete-time sinusoids whose frequencies are separated by an integer multiple of $2\pi$ are identical. This leads us to the idea of a \nameref{def:Frequency Alias}.\label{prop:Discrete-Time Integer Multiple}
\item The highest rate of oscillation in a discrete-time sinusoid is attained when $\omega = \pm \pi$ or, equivalently, $f= \pm \frac{1}{2}$.\label{prop:Discrete-Time Frequency Limit}
\end{propertylist}

\subsubsection{Frequency Aliases}\label{subsubsec:Frequency Aliases}
The concept of a \nameref{def:Frequency Alias} is drawn from the idea that discrete-time sinusoids whose frequencies are separated by an integer mutliple of $2\pi$ are identical and that frequencies $\lvert f \rvert > \frac{1}{2}$ are identical.
(Properties~\ref{prop:Discrete-Time Integer Multiple} and~\ref{prop:Discrete-Time Frequency Limit})

\begin{definition}[Frequency Alias]\label{def:Frequency Alias}
  A \emph{frequency alias} is a sinusoid having a frequency $\lvert \omega \rvert > \pi$ or $\lvert f \rvert > \frac{1}{2}$.
  This is because this sinusoid is \emph{indistinguishable} (\emph{identical}) to one with  frequency $\lvert \omega \rvert < \pi$ or $\lvert f \rvert < \frac{1}{2}$.
  \newline
  \begin{blackbox}
    A \emph{frequency alias} is a sequence resulting from the
    following assertion based on the sinusoid
    $\cos(\omega_{0}n + \theta)$.

    It follows that
    \begin{equation*}
      \cos \left[ \left( \omega_{0} + 2\pi \right)n + \theta \right] = \cos \left( \omega_{0}n + 2\pi n + \theta \right) = \cos (\omega_{0}n + \theta
    \end{equation*}

    As a result, all sinusoidal sequences
    \begin{equation*}
      x_{k}(n) = A \cos (\omega_{k}n + \theta) , \> k=0, 1, 2, \ldots
    \end{equation*}

    where
    \begin{equation*}
      \omega k = \omega_{0} + 2k \pi , \> -\pi \leq \omega_{0} \leq \pi
    \end{equation*}

    are \emph{indistinguishable} (i.e., \emph{identical}). \newline
  \end{blackbox}

  Because of this, we regard frequencies in the range of $-\pi \leq \omega \leq \pi$ or $-\frac{1}{2} \leq f \leq \frac{1}{2}$ as unique, and all frequencies that fall outside of these ranges as aliases.

  \begin{remark}
    It should be noted that there is a difference between discrete-time sinusoids and continuous-time sinusoids here.
    Continuous-time sinusoids have distinct signals for $\Omega$ or $F$ in the entire range $-\infty < \Omega < \infty$ or $-\infty < F < \infty$.
  \end{remark}
\end{definition}

\subsection{Sampling Rates and Sampling Frequency}\label{subsec:Sampling Rates and Frequency}
\subsubsection{Nyquist Rate}\label{subsubsec:Nyquist Rate}
\subsubsection{Nyquist Frequency}\label{subsubsec:Nyquist Frequency}

\subsection{Digital Signals}\label{subsec:Digital Signals}
\begin{definition}[Digital Signals]\label{def:Digital Signals}
  \emph{Digital signals} are a subset of \nameref{def:Discrete-Time Signals}.
  In this case, not only are the values being measured occuring at fixed points in time, the values themselves can only take certain, fixed values.
\end{definition}

\subsubsection{Quantization}\label{subsubsec:Quantization}
\paragraph{Quantization Levels}\label{par:Quantization Levels}
\paragraph{Bit Requirements}\label{par:Quantization Bit Requirements}
\paragraph{Bit Rate}\label{par:Quantization Bit Rate}
%%% Local Variables:
%%% mode: latex
%%% TeX-master: "../EITF75-Systems_and_Signals-Reference_Sheet"
%%% End:


\section{Discrete-Time Systems}\label{sec:Discrete-Time Systems}
As discussed in \Cref{subsec:Discrete-Time Signals}, $x(n)$ is a function of an independent variable that is an integer.
It is important to note that a discrete-time signal is \emph{not defined} at instants between the samples.
Also, if $n$ is not an integer, $x(n)$ is not defined.

Besides graphical representation of a discrete-time system, there are 3 ways to represent a discrete-time signal.
\begin{enumerate}[noitemsep]
\item \nameref{subsubsec:Functional Representation}
\item \nameref{subsubsec:Tabular Representation}
\item \nameref{subsubsec:Sequence Representation}
\end{enumerate}

\subsection{Representing Discrete-Time Systems}\label{subsec:Representing Discrete-Time Systems}
\subsubsection{Functional Representation}\label{subsubsec:Functional Representation}
This representation of a discrete-time system is done as a mathematical function.
\begin{equation}\label{eq:Functional Representation}
  x(n) = \begin{cases}
    1 ,& \text{for } n = 1,3 \\
    4 ,& \text{for } n = 2 \\
    0 ,& \text{elsewhere}
  \end{cases}
\end{equation}

\subsubsection{Tabular Representation}\label{subsubsec:Tabular Representation}
This representation of a discrete-time sysem is done as a table of corresponding values.
\begin{table}[h!]
  \centering
  \begin{tabular}{c|cccccccccc}
    $n$ & $\ldots$ & -2 & -1 & 0 & 1 & 2 & 3 & 4 & 5 & $\ldots$ \\ \midrule
    $x(n)$ & $\ldots$ & 0 & 0 & 0 & 1 & 4 & 1 & 0 & 0 & $\ldots$
  \end{tabular}
\end{table}

\subsubsection{Sequence Representation}\label{subsubsec:Sequence Representation}
There are 2 methods of representation for this.
The first includes all values for $-\infty < n < \infty$.
In all cases, $n=0$ is marked in the sequence, somehow.
I will do this with an underline.
\begin{equation}\label{eq:Infinite Sequence Representation}
  x(n) = \lbrace \ldots, 0, \underline{0}, 1, 4, 1, 0, 0, \ldots \rbrace
\end{equation}

The second only works if all $x(n)$ values for $n < 0$ are 0.
\begin{equation}\label{eq:Zero Sequence Representation}
  x(n) = \lbrace \underline{0}, 1, 4, 1, 0, 0, \ldots \rbrace
\end{equation}

A finite-duration sequence can be represented as
\begin{equation}\label{eq:Finite Sequence Representation}
  x(n) = \lbrace 3, -1, \underline{-2}, 5, 0, 4, -1 \rbrace
\end{equation}
This is identified as a seven-point sequence.

A finite-duration sequence where $x(n)=0$ for all $n<0$ is represented as
\begin{equation}\label{eq:Zero Finite Sequence Representation}
  x(n) = \lbrace \underline{0}, 1, 4, 1 \rbrace
\end{equation}
This is identified as a four-point sequence.

\subsection{Elementary Discrete-Time Signals}\label{subsec:Elementary Discrete-Time Signals}
The following signals are basic signals that appear often and play an important role in signal processing.

\subsubsection{Unit Impulse Signal}\label{subsubsec:Unit Impulse Signal}
\begin{definition}[Unit Impulse Signal]\label{def:Unit Impulse Signal}
  The \emph{unit impulse signal} or \emph{unit sample sequence} is denoted as $\delta(n)$ and is defined as
  \begin{equation}\label{eq:Unit Impulse Signal}
    \delta(n) \equiv = \begin{cases}
      1, & \text{for } n = 0 \\
      0, & \text{for } n \neq 0
    \end{cases}
  \end{equation}

  This function is a signal that is zero everywhere, except at $n=0$, where its value is $1$.

  \begin{remark}
    This signal is different that the analog signal $\delta (t)$, which is also called a unit impulse, and is defined to be 0 everywhere except $t=0$.
    The discrete unit impulse sequence is much less mathematically complicated.
  \end{remark}
\end{definition}

\subsubsection{Unit Step Signal}\label{subsubsec:Unit Step Signal}
\begin{definition}[Unit Step Signal]\label{def:Unit Step Signal}
  The \emph{unit step signal} is denoted as $u(n)$ and is defined as
  \begin{equation}\label{eq:Unit Step Signal}
    u(n) \equiv \begin{cases}
      1, & \text{for } n \geq 0 \\
      0, & \text{for } n < 0
    \end{cases}
  \end{equation}
\end{definition}

\subsubsection{Unit Ramp Signal}\label{subsubsec:Unit Ramp Signal}
\begin{definition}[Unit Ramp Signal]\label{def:Unit Ramp Signal}
  The \emph{unit ramp signal} is denoted as $u_{r}(n)$ and is defined as
  \begin{equation}\label{eq:Unit Ramp Signal}
    u_{r}(n) \equiv \begin{cases}
      n, & \text{for } n \geq 0 \\
      0, & \text{for } n < 0
    \end{cases}
  \end{equation}
\end{definition}

\subsubsection{Exponential Signal}\label{subsubsec:Exponential Signal}
\begin{definition}[Exponential Signal]\label{def:Exponential Signal}
  The \emph{exponential signal} is a sequence of the form
  \begin{equation}\label{eq:Exponential Signal}
    x(n) = a^{n} \>\> \text{for all } n
  \end{equation}

  If $a$ is real, then $x(n)$ is a real signal.
  When $a$ is complex valued ($a \equiv b \pm c j$), it can be expressed as
   \begin{equation}\label{eq:Complex Exponential Signal}
     \begin{aligned}
       x(n) &= r^{n} e^{j \theta n} \\
       &= r^{n} \left( \cos \theta n + j \sin \theta n \right)
     \end{aligned}
   \end{equation}
  This can be expressed by graphing the real and imaginary parts
  \begin{equation}\label{eq:Real Imaginary Complex Exponential Signal}
    \begin{aligned}
      x_{R}(n) &\equiv r^{n} \cos \theta n \\
      x_{I}(n) &\equiv r^{n} j \sin \theta n
    \end{aligned}
  \end{equation}
  or by graphing the amplitude function and phase function.
  \begin{equation}\label{eq:Amplitude Phase Complex Exponential Signal}
    \begin{aligned}
      \lvert x(n) \rvert &= A(n) \equiv r^{n} \\
      \angle x(n) &= \phi(n) \equiv \theta n
    \end{aligned}
  \end{equation}
\end{definition}

\subsection{Classification of Discrete-Time Signals}\label{subsec:Classification Discrete-Time Signals}
In order to apply some mathematical methods to discrete-time signals, we must characterize these signals.

\subsubsection{Energy Signal}\label{subsubsec:Energy Signal}
\begin{definition}[Energy Signal]\label{def:Energy Signal}
  The energy $E$ of a signal $x(n)$ is defined as
  \begin{equation}\label{eq:Energy Signal}
    E \equiv \sum_{n=-\infty}^{\infty} \lvert x(n) \rvert ^{2}
  \end{equation}
  The energy of a signal can be finite or infinite. If $E$ is finite ($0 < E < \infty$), then $x(n)$ is called an \emph{energy signal}.
\end{definition}

\subsubsection{Power Signal}\label{subsubsec:Power Signal}
\begin{definition}[Power Signal]\label{def:Power Signal}
  The average power of a discrete time signal $x(n)$ is defined as
  \begin{equation}\label{eq:Power Signal}
    P = \lim_{N \rightarrow \infty} \frac{1}{2N+1} \sum_{n=-N}^{N} \lvert x(n) \rvert ^{2}
  \end{equation}

  This means that there are 2 potential outcomes:
  \begin{enumerate}[noitemsep]
  \item If $E$ is finite, $P=0$
  \item If $E$ is infinite, $P$ may be either finite or infinite
  \end{enumerate}

  If $P$ is finite and nonzero, the signal is called a \emph{power signal}.
\end{definition}

\subsubsection{Periodic and Aperiodic Signals}\label{subsubsec:Periodic Aperiodic Signals}
A signal $x(n)$ is periodic with period $N$ ($N>0$) if and only if
\begin{equation}\label{eq:Periodic Signal}
  x(n+N) = x(n) \text{for all } n
\end{equation}

The smallest value of $N$ for which~\eqref{eq:Periodic Signal} holds is called the fundamental period.
If there is no value of $N$ that satisfies~\eqref{eq:Periodic Signal}, the signal is called \emph{nonperiodic} or \emph{aperiodic}.

\subsubsection{Symmetric and Antisymmetric Signals}\label{subsubsec:Symmetric and Antisymmetric Signals}
A real-valued signal $x(n)$ is called \emph{symmetric} or \emph{even} if
\begin{equation}\label{eq:Symmetric Signal}
  x(n) = x(-n)
\end{equation}

On the other hand, a signal $x(n)$ is called \emph{antisymmetric} or \emph{odd} if
\begin{equation}\label{eq:Asymmetric Signal}
  x(n) = -x(-n)
\end{equation}

\subsection{Discrete-Time Signal Manipulations}\label{subsec:Discrete-Time Signal Manipulations}
\subsubsection{Transformation of the Independent Variable (Time)}\label{subsubsec:Transform Independent Variable}
It is important to note that \nameref{par:Shifting in Time} and \nameref{par:Folding} are not commutative.
For example,
\begin{equation}\label{eq:Delay then Fold}
  \TimeDelay \lbrace \FoldTime \left[ x(n) \right] \rbrace = \TimeDelay \left[ x(-n) \right] = x(-n + k)
\end{equation}
whereas
\begin{equation}\label{eq:Fold then Delay}
  \FoldTime \lbrace \TimeDelay \left[ x(n) \right] \rbrace = \FoldTime \left[ x(n-k) \right] = x(-n-k)
\end{equation}

\paragraph{Shifting in Time}\label{par:Shifting in Time}
A signal $x(n)$ may be shifted in time by replacing the independent variable $n$ by $n-k$, where $k$ is an integer.
If $k$ is a positive integer, the time shift results in a delay of the signal by $k$ units of time (moves left).
If $k$ is a negative integer, the time shift results in an advance of the signal by $\lvert k \rvert$ units of time (moves right).

This could be denoted by
\begin{equation}\label{eq:Shifting in Time}
  \TimeDelay \left[ x(n) \right] = x(n-k)
\end{equation}

You cannot advance a signal that is being generated in real-time.
Because that would involve signal samples that haven't been generated yet.
So, you can only advance a signal that is stored on something.
However, you can always introduce a delay to a signal.

\paragraph{Folding}\label{par:Folding}
Another useful modification of the time base is to replace $n$ with $-n$.
The result is a \emph{folding} or \emph{reflection} of the original signal around $n=0$.

This could be denoted by
\begin{equation}\label{eq:Folding}
  \FoldTime \left[ x(n) \right] = x(-n)
\end{equation}

\subsubsection{Addition, Multiplication, and Scaling}\label{subsubsec:Addition Multiplication and Scaling}
Amplitude modifications include \nameref{par:Amplitude Addition}, \nameref{par:Amplitude Multiplication}, and \nameref{par:Amplitude Scaling}.

\paragraph{Addition}\label{par:Amplitude Addition}
The \emph{sum} of 2 signals $x_{1}(n)$ and $x_{2}(n)$ is a signal $y(n)$ whose value at any instant is equal to the sum of the values of these two signals at that instant.
\begin{equation}\label{eq:Amplitude Addition}
  y(n) = x_{1}(n) + x_{2}(n), \> -\infty < n < \infty
\end{equation}

\paragraph{Multiplication}\label{par:Amplitude Multiplication}
The \emph{product} of two signals $x_{1}(n)$ and $x_{2}(n)$ is a signal $y(n)$ whose value at any instant is equal to the product of the values of these two signals at that instant.
\begin{equation}\label{eq:Amplitude Multiplication}
  y(n) = x_{1}(n) x_{2}(n), \> -\infty < n < \infty
\end{equation}

\paragraph{Amplitude Scaling}\label{par:Amplitude Scaling}
\emph{Amplitude scaling} of a signal by a constant $A$ is accomplished by multiplying every signal smaple by $A$.
Consequently, we obtain
\begin{equation}\label{eq:Amplitude Scaling}
  y(n) = A x(n), \> -\infty < n < \infty
\end{equation}

%%% Local Variables:
%%% mode: latex
%%% TeX-master: "../EITF75-Systems_and_Signals-Reference_Sheet"
%%% End:


\section{Convolutions}\label{sec:Convolutions}
\begin{definition}[Convolution]\label{def:Convolution}
  The \emph{convolution} operator.

  \begin{equation}\label{eq:Convolution}
    y(t) = \sum\limits_{k=-\infty}^{\infty} x(k) * h(n-k)
  \end{equation}
\end{definition}

%%% Local Variables:
%%% mode: latex
%%% TeX-master: "../EITF75-Systems_and_Signals-Reference_Sheet"
%%% End:


%====================================APPENDIX====================================
\appendix
\counterwithin{definition}{subsection}

\clearpage
\subsection{Trigonometry} \label{app:Trig}
	\subsubsection{Trigonometric Formulas} \label{subsubsec:Trig Formulas}
		\begin{equation} \label{eq:Sin plus Sin with diff Angles}
			\sin \left( \alpha \right) + \sin \left( \beta \right) = 2 \sin \left( \frac{\alpha + \beta}{2} \right) \cos\left( \frac{\alpha - \beta}{2} \right)  
		\end{equation}
		\begin{equation} \label{eq:Cosine-Sine Product}
			\cos \left( \theta \right) \sin \left( \theta \right) = \frac{1}{2} \sin \left( 2 \theta \right)
		\end{equation}
	
	\subsubsection{Euler Equivalents of Trigonometric Functions} \label{subsubsec:Euler Equivalents}
		\begin{equation} \label{eq:Euler Sin}
			\sin \left( x \right) = \frac{e^{\imath x} + e^{-\imath x}}{2}
		\end{equation}
		\begin{equation} \label{eq:Euler Cos}
			\cos \left( x \right) = \frac{e^{\imath x} - e^{-\imath x}}{2 \imath}
		\end{equation}
		\begin{equation} \label{eq:Euler Sinh}
			\sinh \left( x \right) = \frac{e^{x} - e^{-x}}{2}
		\end{equation}
		\begin{equation} \label{eq:Euler Cosh}
			\cosh \left( x \right) = \frac{e^{x} + e^{-x}}{2}
		\end{equation}

\clearpage
\section{Calculus}\label{app:Calculus}
\subsection{L'Hopital's Rule}\label{subsec:LHopitals_Rule}
L'Hopital's Rule can be used to simplify and solve expressions regarding limits that yield irreconcialable results.
\begin{lemma}[L'Hopital's Rule]\label{lemma:LHopitals_Rule}
  If the equation
  \begin{equation*}
    \lim\limits_{x \rightarrow a} \frac{f(x)}{g(x)} =
    \begin{cases}
      \frac{0}{0} \\
      \frac{\infty}{\infty} \\
    \end{cases}
  \end{equation*}
  then \Cref{eq:LHopitals_Rule} holds.
  \begin{equation}\label{eq:LHopitals_Rule}
    \lim\limits_{x \rightarrow a} \frac{f(x)}{g(x)} = \lim\limits_{x \rightarrow a} \frac{f'(x)}{g'(x)}
  \end{equation}
\end{lemma}

\subsection{Fundamental Theorems of Calculus}\label{subsec:Fundamental Theorem of Calculus}
\begin{definition}[First Fundamental Theorem of Calculus]\label{def:1st Fundamental Theorem of Calculus}
  The \emph{first fundamental theorem of calculus} states that, if $f$ is continuous on the closed interval $\left[ a,b \right]$ and $F$ is the indefinite integral of $f$ on $\left[ a,b \right]$, then

  \begin{equation}\label{eq:1st Fundamental Theorem of Calculus}
    \int_{a}^{b}f \left( x \right) dx = F \left( b \right) - F \left( a \right)
  \end{equation}
\end{definition}

\begin{definition}[Second Fundamental Theorem of Calculus]\label{def:2nd Fundamental Theorem of Calculus}
  The \emph{second fundamental theorem of calculus} holds for $f$ a continuous function on an open interval $I$ and $a$ any point in $I$, and states that if $F$ is defined by

  \begin{equation*}
    F \left( x \right) = \int_{a}^{x} f \left( t \right) dt,
  \end{equation*}
  then
  \begin{equation}\label{eq:2nd Fundamental Theorem of Calculus}
    \begin{aligned}
      \frac{d}{dx} \int_{a}^{x} f \left( t \right) dt &= f \left( x \right) \\
      F' \left( x \right) &= f \left( x \right) \\
    \end{aligned}
  \end{equation}
\end{definition}

\begin{definition}[argmax]\label{def:argmax}
  The arguments to the \emph{argmax} function are to be maximized by using their derivatives.
  You must take the derivative of the function, find critical points, then determine if that critical point is a global maxima.
  This is denoted as
  \begin{equation*}\label{eq:argmax}
    \argmax_{x}
  \end{equation*}
\end{definition}

\subsection{Rules of Calculus}\label{subsec:Rules of Calculus}
\subsubsection{Chain Rule}\label{subsubsec:Chain Rule}
\begin{definition}[Chain Rule]\label{def:Chain Rule}
  The \emph{chain rule} is a way to differentiate a function that has 2 functions multiplied together.

  If
  \begin{equation*}
    f(x) = g(x) \cdot h(x)
  \end{equation*}
  then,
  \begin{equation}\label{eq:Chain Rule}
    \begin{aligned}
      f'(x) &= g'(x) \cdot h(x) + g(x) \cdot h'(x) \\
      \frac{df(x)}{dx} &= \frac{dg(x)}{dx} \cdot g(x) + g(x) \cdot \frac{dh(x)}{dx} \\
    \end{aligned}
  \end{equation}
\end{definition}

\subsection{Useful Integrals}\label{subsec:Useful_Integrals}
\begin{equation}\label{eq:Cosine_Indefinite_Integral}
  \int \cos(x) \; dx = \sin(x)
\end{equation}

\begin{equation}\label{eq:Sine_Indefinite_Integral}
  \int \sin(x) \; dx = -\cos(x)
\end{equation}

\begin{equation}\label{eq:x_Cosine_Indefinite_Integral}
  \int x \cos(x) \; dx = \cos(x) + x \sin(x)
\end{equation}
\Cref{eq:x_Cosine_Indefinite_Integral} simplified with Integration by Parts.

\begin{equation}\label{eq:x_Sine_Indefinite_Integral}
  \int x \sin(x) \; dx = \sin(x) - x \cos(x)
\end{equation}
\Cref{eq:x_Sine_Indefinite_Integral} simplified with Integration by Parts.

\begin{equation}\label{eq:x_Squared_Cosine_Indefinite_Integral}
  \int x^{2} \cos(x) \; dx = 2x \cos(x) + (x^{2} - 2) \sin(x)
\end{equation}
\Cref{eq:x_Squared_Cosine_Indefinite_Integral} simplified by using Integration by Parts twice.

\begin{equation}\label{eq:x_Squared_Sine_Indefinite_Integral}
  \int x^{2} \sin(x) \; dx = 2x \sin(x) - (x^{2} - 2) \cos(x)
\end{equation}
\Cref{eq:x_Squared_Sine_Indefinite_Integral} simplified by using Integration by Parts twice.

\begin{equation}\label{eq:Exponential_Cosine_Indefinite_Integral}
  \int e^{\alpha x} \cos(\beta x) \; dx = \frac{e^{\alpha x} \bigl( \alpha \cos(\beta x) + \beta \sin(\beta x) \bigr)}{\alpha^{2} + \beta^{2}} + C
\end{equation}

\begin{equation}\label{eq:Exponential_Sine_Indefinite_Integral}
  \int e^{\alpha x} \sin(\beta x) \; dx = \frac{e^{\alpha x} \bigl( \alpha \sin(\beta x) - \beta \cos(\beta x) \bigr)}{\alpha^{2}+\beta^{2}} + C
\end{equation}

\begin{equation}\label{eq:Exponential_Indefinite_Integral}
  \int e^{\alpha x} \; dx = \frac{e^{\alpha x}}{\alpha}
\end{equation}

\begin{equation}\label{eq:x_Exponential_Indefinite_Integral}
  \int x e^{\alpha x} \; dx = e^{\alpha x} \left( \frac{x}{\alpha} - \frac{1}{\alpha^{2}} \right)
\end{equation}
\Cref{eq:x_Exponential_Indefinite_Integral} simplified with Integration by Parts.

\begin{equation}\label{eq:Inverse_x_Indefinite_Integral}
  \int \frac{dx}{\alpha + \beta x} = \int \frac{1}{\alpha + \beta x} \; dx = \frac{1}{\beta} \ln (\alpha + \beta x)
\end{equation}

\begin{equation}\label{eq:Inverse_x_Squared_Indefinite_Integral}
  \int \frac{dx}{\alpha^{2} + \beta^{2} x^{2}} = \int \frac{1}{\alpha^{2} + \beta^{2} x^{2}} \; dx = \frac{1}{\alpha \beta} \arctan \left( \frac{\beta x}{\alpha} \right)
\end{equation}

\begin{equation}\label{eq:a_Exponential_Indefinite_Integral}
  \int \alpha^{x} \; dx = \frac{\alpha^{x}}{\ln(\alpha)}
\end{equation}

\begin{equation}\label{eq:a_Exponential_Derivative}
  \frac{d}{dx} \alpha^{x} = \frac{d\alpha^{x}}{dx} = \alpha^{x} \ln(x)
\end{equation}

\subsection{Leibnitz's Rule}\label{subsec:Leibnitzs_Rule}
\begin{lemma}[Leibnitz's Rule]\label{lemma:Leibnitzs_Rule}
  Given
  \begin{equation*}
    g(t) = \int_{a(t)}^{b(t)} f(x, t) \, dx
  \end{equation*}
  with $a(t)$ and $b(t)$ differentiable in $t$ and $\frac{\partial f(x, t)}{\partial t}$ continuous in both $t$ and $x$, then
  \begin{equation}\label{eq:Leibnitzs_Rule}
    \frac{d}{dt} g(t) = \frac{d g(t)}{dt} = \int_{a(t)}^{b(t)} \frac{\partial f(x, t)}{\partial t} \, dx + f \bigl[ b(t), t \bigr] \, \frac{d b(t)}{dt} - f \bigl[ a(t), t \bigr] \, \frac{d a(t)}{dt}
  \end{equation}
\end{lemma}



\end{document}
%%% Local Variables:
%%% mode: latex
%%% TeX-master: t
%%% End:
