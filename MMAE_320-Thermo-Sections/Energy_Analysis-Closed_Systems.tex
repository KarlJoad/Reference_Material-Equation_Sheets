\section{Energy Analysis of \nameref*{def:Closed_System}s}\label{def:Energy_Analysis_Closed_Systems}
Gases can perform work, meaning they can expend \nameref{def:Energy}.

For example, in a piston, there is a contained gas, which can press one side up and out.
Remember the equation for \nameref{def:Work}, \Cref{eq:Work}.
But, because we are dealing with gas pressures, we rewrite \Cref{eq:Work}.
\begin{align*}
  \Work &= \Pressure \Area \Distance \\
  \Delta \Work &= \Pressure \Area \Delta \Distance \\
        &= \Pressure \Delta \Volume \\
  \intertext{We can express the change in work using differentials too, so we can integrate.}
  d \Work &= \Pressure d\Volume
\end{align*}

This leads to \Cref{eq:Moving_Boundary_Work}
\begin{equation}\label{eq:Moving_Boundary_Work}
\Work = \int_{\text{State 1}}^{\text{State 2}} \Pressure d\Volume
\end{equation}

\Cref{eq:Moving_Boundary_Work} has two possibilities:

\begin{description}[noitemsep]
\item[Volume is Constant] $\Work = 0$
\item[Pressure is Constant] $\Work = \Pressure (\Volume_{2} - \Volume_{1})$
\end{description}

Now, we may be interested in the total energy required to change or create a system.
This is \nameref{def:Enthalpy}.
For example, \nameref{def:Enthalpy_Vaporization} is the amount of energy required to convert a system from a \nameref{def:Saturated_Liquid} to a \nameref{def:Saturated_Vapor}.

\begin{definition}[Enthalpy]\label{def:Enthalpy}
  Another way to think of this, is \emph{enthalpy} is the \nameref{def:Energy} required to create a new \nameref{def:System}.
  This can only be done if the current system performs work on its boundary, i.e.\ moves the boundary.

  Enthalpy is typically the total \nameref{def:Heat} content of a \nameref{def:System}.
  It is equal to the \nameref{def:Internal_Energy} of the system plus the product of \nameref{def:Pressure} and volume.
  \begin{equation}\label{eq:Enthalpy}
    \Enthalpy = \InternalEnergy + \Pressure \Volume \:\: \si{\joule}
  \end{equation}
\end{definition}

However, \nameref{def:Enthalpy} is an \nameref{def:Extensive_Property}, but we would prefer to work with an \nameref{def:Intensive_Property}.
Thus, we more frequently use \nameref{def:Specific_Enthalpy}.

\begin{definition}[Specific Enthalpy]\label{def:Specific_Enthalpy}
  \emph{Specific enthalpy} is also commonly referred to as just \emph{enthalpy}, as context makes clear which one is being discussed.
  Specific enthalpy is the total \nameref{def:Heat} content of a \nameref{def:System} per unit mass of that system.
  Specific Enthalpy is defined in \Cref{eq:Specific_Enthalpy}.
  \begin{equation}\label{eq:Specific_Enthalpy}
    \SpecificEnthalpy = \SpecificInternalEnergy + \Pressure \SpecificVolume \:\: \si{\joule\per\gram}
  \end{equation}
\end{definition}


%%% Local Variables:
%%% mode: latex
%%% TeX-master: "../MMAE_320-Thermo-Reference_Sheet"
%%% End:
