\subsection{Continuity Equation}\label{subsec:Continuity_Equation}
We perform this analysis when we are interested in the matter flow of a \nameref{def:System}.
We still need to have \nameref{def:Law_Conservation_Energy}.
The important thing to remember here is that there is no change of mass of the control volume, meaning there is \textbf{no} storage of the mass in the system.
This means we have \emph{steady flow}.
This also means we could have an unsteady flow, but we deal with that later in this section.

Now, to motivate the \nameref{def:Continuity_Equation}, imagine an expandable tank.
If there are $n$ flows inwards and $m$ flows outwards, we can construct an equation that looks like this.
\begin{align*}
  (\FlowRate{\Mass_{In,1}} + \FlowRate{\Mass_{In,2}}) - \FlowRate{\Mass_{Out}} &= {\left[ \frac{dm}{dt} \right]}_{\text{Control Volume}} \\
  \sum\limits_{\text{Inlets}} \FlowRate{\Mass} - \sum\limits_{\text{Outlets}} \FlowRate{\Mass} &= \FlowRate{\Mass}_{\text{Control Volume}}
\end{align*}


%%% Local Variables:
%%% mode: latex
%%% TeX-master: "../../MMAE_320-Thermo-Reference_Sheet"
%%% End:
