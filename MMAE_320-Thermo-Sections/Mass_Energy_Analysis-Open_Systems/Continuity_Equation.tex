\subsection{Continuity Equation}\label{subsec:Continuity_Equation}
We perform this analysis when we are interested in the matter flow of a \nameref{def:System}.
We still need to have \nameref{def:Law_Conservation_Energy}.
The important thing to remember here is that there is no change of mass of the control volume, meaning there is \textbf{no} storage of the mass in the system.
This means we have \emph{steady flow}.
This also means we could have an unsteady flow, but we deal with that later in this section.

Now, to motivate the \nameref{def:Continuity_Equation}, imagine an expandable tank.
If there are $n$ flows inwards and $m$ flows outwards, we can construct an equation that looks like this.
\begin{align*}
  (\FlowRate{\Mass_{In,1}} + \FlowRate{\Mass_{In,2}}) - \FlowRate{\Mass_{Out}} &= {\left[ \frac{dm}{dt} \right]}_{\text{Control Volume}} \\
  \sum\limits_{\text{Inlets}} \FlowRate{\Mass} - \sum\limits_{\text{Outlets}} \FlowRate{\Mass} &= \FlowRate{\Mass}_{\text{Control Volume}}
\end{align*}

\begin{definition}[Continuity Equation]\label{def:Continuity_Equation}
  The \emph{continuity equation} states the relationship between the mass flow into a point in a \nameref{def:System} and its output.
  Succinctly, it can be stated:
  \begin{equation}\label{eq:Continuity_Equation}
    \begin{aligned}
      \sum\limits_{\text{Inlets}} \FlowRate{\Mass} - \sum\limits_{\text{Outlets}} \FlowRate{\Mass} &= \FlowRate{\Mass}_{\text{Control Volume}} \\
      \sum\limits_{\text{Inlets}} \FlowRate{\Volume} - \sum\limits_{\text{Outlets}} \FlowRate{\Volume} &= \FlowRate{\Volume}_{\text{Control Volume}}
    \end{aligned}
  \end{equation}

  \begin{description}[noitemsep]
  \item $\FlowRate{\Mass}_{\text{Control Volume}}$ is the storage possible in a vessel, in mass.
    Likewise, $\FlowRate{\Volume}_{\text{Control Volume}}$ is the possible storage of a vessel, in volume.
  \end{description}

  \begin{remark}[Alternate Definitions of Mass Flowrate]\label{rmk:Alternative_Mass_Flowrate_Defns}
    It will be useful to remember alternative versions of mass flowrate for this particular section.
    Namely,
    \begin{align*}
      \FlowRate{\Mass}_{1} &= \FlowRate{\Mass}_{2} \\
      \Density_{1} \Velocity_{1} \Area_{1} &= \Density_{2} \Velocity_{2} \Area_{2}
    \end{align*}
  \end{remark}
\end{definition}

%%% Local Variables:
%%% mode: latex
%%% TeX-master: "../../MMAE_320-Thermo-Reference_Sheet"
%%% End:
