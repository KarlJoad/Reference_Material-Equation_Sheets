\subsection{Continuity Equation}\label{subsec:Continuity_Equation}
We perform this analysis when we are interested in the matter flow of a \nameref{def:System}.
We still need to have \nameref{def:Law_Conservation_Energy}.
The important thing to remember here is that there is no change of mass of the control volume, meaning there is \textbf{no} storage of the mass in the system.
This means we have \emph{steady flow}.
This also means we could have an unsteady flow, but we deal with that later in this section.

Now, to motivate the \nameref{def:Continuity_Equation}, imagine an expandable tank.
If there are $n$ flows inwards and $m$ flows outwards, we can construct an equation that looks like this.
\begin{align*}
  (\FlowRate{\Mass_{In,1}} + \FlowRate{\Mass_{In,2}}) - \FlowRate{\Mass_{Out}} &= {\left[ \frac{dm}{dt} \right]}_{\text{Control Volume}} \\
  \sum\limits_{\text{Inlets}} \FlowRate{\Mass} - \sum\limits_{\text{Outlets}} \FlowRate{\Mass} &= \FlowRate{\Mass}_{\text{Control Volume}}
\end{align*}

\begin{definition}[Continuity Equation]\label{def:Continuity_Equation}
  The \emph{continuity equation} states the relationship between the mass flow into a point in a \nameref{def:System} and its output.
  Succinctly, it can be stated:
  \begin{equation}\label{eq:Continuity_Equation}
    \begin{aligned}
      \sum\limits_{\text{Inlets}} \FlowRate{\Mass} - \sum\limits_{\text{Outlets}} \FlowRate{\Mass} &= \FlowRate{\Mass}_{\text{Control Volume}} \\
      \sum\limits_{\text{Inlets}} \FlowRate{\Volume} - \sum\limits_{\text{Outlets}} \FlowRate{\Volume} &= \FlowRate{\Volume}_{\text{Control Volume}}
    \end{aligned}
  \end{equation}

  \begin{description}[noitemsep]
  \item $\FlowRate{\Mass}_{\text{Control Volume}}$ is the storage possible in a vessel, in mass.
    Likewise, $\FlowRate{\Volume}_{\text{Control Volume}}$ is the possible storage of a vessel, in volume.
  \end{description}

  \begin{remark}[Alternate Definitions of Mass Flowrate]\label{rmk:Alternative_Mass_Flowrate_Defns}
    It will be useful to remember alternative versions of mass flowrate for this particular section.
    Namely,
    \begin{align*}
      \FlowRate{\Mass}_{1} &= \FlowRate{\Mass}_{2} \\
      \Density_{1} \Velocity_{1} \Area_{1} &= \Density_{2} \Velocity_{2} \Area_{2}
    \end{align*}
  \end{remark}
\end{definition}

There are 2 situations that can arise from the \nameref{def:Continuity_Equation}.
\begin{enumerate}[noitemsep]
\item Rigid control volumes, $\FlowRate{\Mass}_{\text{Control Volume}} = 0$.
\item Non-rigid control volumes, $\FlowRate{\Mass}_{\text{Control Volume}} \neq 0$.
\end{enumerate}

\begin{example}{Use Continuity Equation}
  A nozzle is attached to a \SI{20}{\gallon} bucket to fill it.
  The inner diameter of the hose is \SI{1}{\inch} and reduces to \SI{0.5}{\inch} at the nozzle's exit.
  If the average velocity of the water in the hose is \SI{8}{\feet\per\second} determine the volume and mass flow rates, $\FlowRate{\Volume}$ and $\FlowRate{\Mass}$ of the water through the hose and how long it will take to fill the bucket?
  Also determine the average velocity at the tip of the nozzle, $\Velocity_{Out}$?
  \tcblower{}
  \textbf{Concepts:} \\
  A nozzle has 1 inlet and 1 outlet.
  The diameter of the nozzle at the inlet is $\Diameter_{In} = \SI{1}{\inch}$, and the outlet $\Diameter_{Out} = \SI{0.5}{\inch}$. \\
  There is no temperature change in the hose, so the density of the water is constant, $\Change{\Density_{Water}} = 0$. \\
  Because the host does not have any storage, $\FlowRate{\Mass}_{\text{Control Volume}} = 0$.
  Therefore, $\FlowRate{\Mass}_{In} = \FlowRate{\Mass}_{Out}$ and $\FlowRate{\Volume}_{In} = \FlowRate{\Volume}_{Out}$.

  \textbf{Explore:} \\
  $\Density_{\text{Water}} = \SI{62.3}{\lbm\per\feet\cubed}$, as \SI{32}{\degreeF}. \\
  The bucket has a capacity of \SI{20}{\gallon}, which means $20 = \FlowRate{\Volume} \times \text{Time to fill}$.

  \textbf{Plan:} \\
  Solve for $\FlowRate{\Volume}$, $\FlowRate{\Mass}$, $\Volume_{2}$.

  \textbf{Solve:} \\
  Start with a definition of volume flowrate.
  \begin{align*}
    \FlowRate{\Volume}_{In} &= \Area_{In} \Velocity_{In} \\
                            &= \frac{\pi \Diameter_{In}^{2}}{4} \SI{8}{\feet\per\second} \\
                            &= \SI{0.0436}{\cubic\feet\per\second}
  \end{align*}

  Now, we can find the mass flowrate through the inlet of the hose.
  \begin{align*}
    \FlowRate{\Mass} &= \Density_{Water} \FlowRate{\Volume} \\
                     &= \SI{62.3}{\lbm\per\feet\cubed} (\SI{0.0436}{\cubic\feet\per\second}) \\
                     &= \SI{2.72}{\lbm\per\second}
  \end{align*}

  Because the flowrates at the inlet and outlet of the hose are the same, we can reuse these values to solve for the time to fill the bucket.
  \begin{align*}
    \SI{20}{\gallon} \frac{\SI{1}{\cubic\feet}}{\SI{7.48}{\gallon}} &= \FlowRate{\Volume} \Time \\
    \Time &= \SI{61.3}{\second}
  \end{align*}

  Lastly, we need to solve for the velocity of the flowing water at the tip of the nozzle.
  We can use the relation between mass flowrates, shown in \Cref{rmk:Alternative_Mass_Flowrate_Defns}.
  \begin{align*}
    \Density_{Water} \Area_{In} \Velocity_{In} &= \Density_{Water} \Area_{Out} \Velocity_{Out} \\
    \Area_{In} \Velocity_{In} &= \Area_{Out} \Velocity_{Out} \\
    \Velocity_{Out} &= \frac{\Area_{In} \Velocity_{In}}{\Area_{Out}} \\
                                               &= \frac{\frac{\pi \Diameter_{In}^{2}}{4} \Velocity_{In}}{\frac{\pi \Diameter_{Out}^{2}}{4}} \\
    \Velocity_{Out} &= \SI{32}{\feet\per\second}
  \end{align*}

  \textbf{Validate:} \\
  We can validate that our answer for volume flowrate makes sense by multiplying the outlet volume flowrate by the outlet velocity and the outlet cross-sectional area.
  \begin{align*}
    \FlowRate{\Volume}_{Out} &= \Velocity_{Out} \times \Area_{Out} \\
                             &= \SI{32}{\feet\per\second} \frac{\pi \Diameter_{Out}^{2}}{4} \\
  \end{align*}

  If we look at this closely, we notice that when the diameter of the outlet was halved, the total area was quartered.
  Correspondingly, there was a four times increase in the velocity of the water at the outlet.

  \textbf{Generalize:} \\
  Constant mass flow and constant fluid density reuislts in a change at the outlet based on the change in area.
\end{example}

%%% Local Variables:
%%% mode: latex
%%% TeX-master: "../../MMAE_320-Thermo-Reference_Sheet"
%%% End:
