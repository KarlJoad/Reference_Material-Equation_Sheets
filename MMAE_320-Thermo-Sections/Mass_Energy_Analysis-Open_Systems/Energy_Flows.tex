\subsection{Energy Flows}\label{subsec:Energy_Flows}
\begin{table}[h!tbp]
  \centering
  \begin{tabular}{ccccc}
    \toprule
    & \nameref{def:Internal_Energy} & Kinetic Energy & Potential Energy & \nameref{def:Flow_Energy} \\
    \midrule
    \nameref{def:Specific_Energy}, $\SpecificEnergy$ & $\SpecificInternalEnergy$ & $\frac{\Velocity^{2}}{2}$ & $\Gravity \Distance$ & $\frac{\Pressure}{\Density}$ \\
    \nameref{def:Energy}, $\Energy$ & $\InternalEnergy$ & $\frac{\Mass \Velocity^{2}}{2}$ & $\Mass \Gravity \Distance$ & $\Pressure \Volume$ \\
    \bottomrule
  \end{tabular}
  \caption{Types of Energy for Mass Flow}
  \label{tab:Energy_Flows}
\end{table}

We can often connect the \nameref{def:Internal_Energy} and the \nameref{def:Flow_Energy} of a process and call that \nameref{def:Enthalpy}, $\SpecificEnthalpy$ or $\Enthalpy$ for a particular control volume.
\begin{align*}
  \SpecificEnthalpy &= \SpecificInternalEnergy + \frac{\Pressure}{\Density} \\
  \Enthalpy &= \InternalEnergy + \FlowEnergy
\end{align*}

And we know that power is the time-rate of change of energy.
\begin{align*}
  \Power &= \FlowRate{\Energy} \\
  &= \FlowRate{\Mass} \SpecificEnergy
\end{align*}

Just like with mass flows, if a control volume has not storage capacity, then it also has no energy.
Thus, if there is no storage, and the \nameref{def:Law_Conservation_Energy} is abided by, then:
\begin{align*}
  \FlowRate{\Mass}_{In} - \FlowRate{\Mass}_{Out} &= \FlowRate{\Mass}_{\text{Control Volume}} \\
  \FlowRate{\Energy}_{In} - \FlowRate{\Energy}_{Out} &= \FlowRate{\Energy}_{\text{Control Volume}} \\
  \FlowRate{\Mass}_{In} &= \FlowRate{\Mass}_{Out} \\
  \FlowRate{\Energy}_{In} &= \FlowRate{\Energy}_{Out}
\end{align*}

Looking at the equations received, it is obvious that we can apply any kind of \nameref{def:Energy} that may be present in the system to the above equations and solve for any term we may be interested in.
If we now expand our \nameref{def:Continuity_Equation} to energies, we get \Cref{eq:Energy_Continuity_Equation}.

\begin{equation}\label{eq:Energy_Continuity_Equation}
  \begin{aligned}
    \FlowRate{\Heat}_{In} + \FlowRate{\Work}_{In} + \sum\limits_{\text{In}} \FlowRate{\Mass} \SpecificEnergy &= \FlowRate{\Heat}_{Out} + \FlowRate{\Work}_{Out} + \sum\limits_{\text{Out}} \FlowRate{\Mass} \SpecificEnergy \\
    \FlowRate{\Heat}_{In} + \FlowRate{\Work}_{In} + \sum\limits_{\text{In}} \FlowRate{\Mass} (\SpecificEnthalpy + \frac{\Velocity^{2}}{2} + \Gravity \Distance) &= \FlowRate{\Heat}_{Out} + \FlowRate{\Work}_{Out} + \sum\limits_{\text{Out}} \FlowRate{\Mass} (\SpecificEnthalpy + \frac{\Velocity^{2}}{2} + \Gravity \Distance)
  \end{aligned}
\end{equation}

If we have a steady flow, then \Cref{eq:Energy_Continuity_Equation} can be simplified a bit.
\begin{equation}\label{eq:Steady_Flow_Energy_Continuity_Equation}
  \begin{aligned}
    \Change{\FlowRate{\Heat}} - \Change{\FlowRate{\Work}} &= \FlowRate{\Mass} \left( \Change{\SpecificEnthalpy} + \Change{\KineticEnergy} + \Change{\PotentialEnergy} \right) \\
    &= \FlowRate{\Mass} \left( (\SpecificEnthalpy_{2} - \SpecificEnthalpy_{1}) + \frac{\Velocity_{2}^{2} - \Velocity_{1}^{2}}{2} + \Gravity (\Distance_{2} - \Distance_{1}) \right)
  \end{aligned}
\end{equation}

The work in \Cref{eq:Steady_Flow_Energy_Continuity_Equation} is shaft work.
This could be a propeller turning the fluid, or the fluid doing work on the shaft to achieve something.

\subsubsection{Nozzles and Diffusers}\label{subsubsec:Nozzles_Diffusers}
There are some general properties about nozzles and diffusers that are useful to remember.
\begin{itemize}[noitemsep]
\item For nozzles, $\Velocity_{2} \gg \Velocity_{1}$.
  The energy gained from the change in kinetic energy is changed from the \nameref{def:Flow_Energy}, because $\Pressure_{2} < \Pressure_{1}$.
\item For diffusers, $\Velocity_{2} \ll \Velocity_{1}$.
  The energy lost from the change in kinetic energy is given to the \nameref{def:Flow_Energy}, because $\Pressure_{2} > \Pressure_{1}$.
\item In general, $\Change{\FlowRate{\Heat}} \approx 0$.
  There is very little \nameref{def:Heat_Transfer}.
\item In general, $\Change{\FlowRate{\Work}} \approx 0$.
  There are no propellers or outside shafts.
\item In general, $\FlowRate{\KineticEnergy} \neq 0$.
  The kinetic energy changes dramatically.
\item In general, $\SpecificEnthalpy \neq 0$.
  There is a large volume change.
\end{itemize}

\begin{example}[Textbook Problem 6.31]{Behavior of Nozzles}
  Given a nozzle that pressures air in an \nameref{def:Adiabatic} \nameref{def:Process} at an initial pressure of $\Pressure_{1} = \SI{300}{\kilo\pascal}$, initial temperature of $\Temp_{1} = \SI{200}{\degreeCelsius}$, an inlet velocity of $\Velocity_{1} = \SI{30}{\meter\per\second}$, and an inlet cross-sectional area of $\Area_{1} = \SI{80}{\centi\meter\squared}$.
  The outlet pressure is $\Pressure_{2} = \SI{100}{\kilo\pascal}$ and a velocity of $\Velocity_{2} = \SI{180}{\meter\per\second}$.
  Find the mass flow rate, $\FlowRate{\Mass}$, the temperature of the output air, $\Temp_{2}$, and the cross-sectional area of the outlet, $\Area_{2}$?
  \tcblower{}
  \textbf{Concepts:} \\
  A nozzle has a much greater outlet velocity compared to its inlet velocity, $\Velocity_{Out} \gg \Velocity_{In}$, and $\Pressure_{In} > \Pressure_{Out}$. \\
  Air is an ideal gas and is compressible.
  So, we might have to deal with different densities of the air at the inlet and outlet, $\Density_{1}$ may not necessarily be equal to $\Density_{2}$. \\
  This system seems to be under steady flow conditions, so $\FlowRate{\Mass}_{In} = \FlowRate{\Mass}_{Out}$. \\
  Due to the \nameref{def:Law_Conservation_Energy}, we can say $\FlowRate{\Energy}_{In} = \FlowRate{\Energy}_{Out}$.
  Namely, there is no energy storage in this system.

  \textbf{Explore:} \\
  \begin{align*}
    \FlowRate{\Mass}_{In} &= \FlowRate{\Mass}_{Out} \\
    {(\Density \Velocity \Area)}_{In} &= {(\Density \Velocity \Area)}_{Out} \\
  \end{align*}

  Now, we can find the density of the inlet air by using the tables for it.
  But, to find the density of the air at the outlet, we can use \Cref{eq:Ideal_Gas_Equation}, and take the inverse of the \nameref{def:Specific_Volume}.
  \begin{align*}
    \FlowRate{\Mass}_{In} &= \Density_{2} \Velocity_{2} \Area_{2} \\
                          &= \frac{1}{\SpecificVolume_{2}} \Velocity_{2} \Area_{2} \\
    \intertext{Find the \nameref{def:Specific_Volume} using \Cref{eq:Ideal_Gas_Equation}.}
    \Pressure \SpecificVolume &= \GasConstant \Temp \\
    \SpecificVolume &= \frac{\GasConstant \Temp}{\Pressure} \\
    \FlowRate{\Mass}_{In} &= \frac{\Pressure}{\GasConstant \Temp} \Velocity_{2} \Area_{2} \\
  \end{align*}

  We also have several other key pieces of information to solve this.
  \begin{align*}
    \Change{\FlowRate{\Heat}} &= 0 \\
    \Change{\FlowRate{\Work}} &= 0 \\
    \Change{\PotentialEnergy} &= 0 \\
  \end{align*}

  Therefore, only kinetic energy and enthalpy change, $\Change{\KineticEnergy}$ and $\Change{\SpecificEnthalpy}$, respectively.
  We can calculate the change in \nameref{def:Specific_Enthalpy} using a slightly modified version of \Cref{eq:Change_Internal_Energy_Specific_Heat_Constant_Pressure}.
  \begin{align*}
    \Change{\Enthalpy} &= \SpecificHeatPressure \Mass \Change{\Temp} \\
    \Change{\SpecificEnthalpy} &= \frac{\Change{\Enthalpy}}{\Mass} \\
                       &= \frac{\SpecificHeatPressure \Mass \Change{\Temp}}{\Mass} \\
                       &= \SpecificHeatPressure \Change{\Temp}
  \end{align*}

  We use $\SpecificHeatPressure$, \nameref{def:Specific_Heat} under constant pressure, because once the air leaves the nozzle, it is in an environment with constant pressure.

  \textbf{Plan:} \\
  Solve for $\SpecificVolume_{2}$ using the \Cref{eq:Ideal_Gas_Equation}. \\
  Solve for $\Area_{2}$ using the relation $\FlowRate{\Mass} = \frac{\Area_{1} \Velocity_{1}}{\Velocity_{2}}$. \\
  Use our energy conservation relation to solve for $\Temp_{2}$ using $\Change{\Temp}$.

  \textbf{Solve:} \\
  \begin{align*}
    \SpecificVolume_{1} &= \frac{\GasConstant \Temp_{1}}{\Pressure} \\
                        &= \SI{0.4525}{\meter\cubed\per\kilo\gram}
  \end{align*}

  Now that we have the initial \nameref{def:Specific_Volume}, we can find the mass flowrate.
  \begin{align*}
    \FlowRate{\Mass}_{1} &= \frac{\Area_{1} \Velocity_{1}}{\SpecificVolume_{1}} \\
                         &= \SI{0.5304}{\kilo\gram\per\second}
  \end{align*}

  Now, we can solve for our energy conservation equation, \Cref{eq:Energy_Continuity_Equation} to find the final temperature, $\Temp_{2}$.
  \begin{align*}
    \Change{\FlowRate{\Heat}} - \Change{\FlowRate{\Work}} &= \FlowRate{\Mass} \left( \Change{\SpecificEnthalpy} - \frac{\Change{\KineticEnergy}}{\Mass} \right) \\
    0 - 0 &= \FlowRate{\Mass} \left(\SpecificHeatPressure \Change{\Temp} + \frac{\Velocity_{2}^{2} - \Velocity_{1}^{2}}{2} \right) \\
    \intertext{Refer to the textbook's Table A.2 for $\SpecificHeatPressure$. Some interpolation is needed.}
    0 &= \SI{0.5304}{\kilo\gram\per\second} \left( \SI{1.02}{\kilo\joule\per\kilo\gram} (\Temp_{2} - (200 + 273.15)\si{\kelvin}) - \frac{\SI{180}{\meter\per\second} - \SI{30}{\meter\per\second}}{2} \right) \\
    \intertext{Now, rearrange the equation such that you solve for $\Temp_{2}$.}
    \Temp_{2} &= \SI{184.25}{\degreeCelsius}
  \end{align*}

  Now, we can solve for the final \nameref{def:Specific_Volume}.
  \begin{align*}
    \SpecificVolume_{2} &= \frac{\GasConstant \Temp_{2}}{\Pressure_{2}} \\
                        &= \SI{1.312}{\meter\cubed\per\kilo\gram} \\
    \intertext{Plugging that into the mass flow equation we have, to find the outlet cross-sectional area, $\Area_{2}$.}
    \Change{\Mass}_{1} &= \Change{\Mass}_{2} \\
    \SI{0.5304}{\kilo\gram\per\second} &= \frac{\Area_{2} \Velocity_{2}}{\SpecificVolume_{2}} \\
    \Area_{2} &= \SI{0.00386}{\meter\squared} \\
                        &= \SI{38.6}{\centi\meter\squared}
  \end{align*}

  \textbf{Validate:} \\
  The area decreased, so it definitely is a nozzle.
  In particular, all the other factors help us confirm that.
  The pressure decreased, the enthalpy decreased, the velocity increased, and the cross-sectional area decreased.
  Thus, this is definitely a nozzle.
  \begin{center}
    \begin{tabular}{c|c}
      \toprule
      In & Out \\
      \midrule
      Pressure High & Pressure Low \\
      Velocity Low & Velocity High \\
      Start Temperature & Temperature Decreased \\
      Start Enthalpy & Enthalpy Decreased \\
      \bottomrule
    \end{tabular}
  \end{center}

  \textbf{Generalize:} \\
  In a nozzle, velocity and pressure change occurs at the expense of the \nameref{def:Enthalpy}.
\end{example}

\begin{example}[Textbook Problem 6.35]{Steam Nozzle}
  A nozzle is handling steam that enters at $\Pressure_{1} = \SI{5}{\mega\pascal}$ and $\Temp_{1} = \SI{400}{\degreeCelsius}$ steadily at $\Velocity_{1} = \SI{80}{\meter\per\second}$.
  The steam leaves at $\Pressure_{2} = \SI{2}{\mega\pascal}$ and $\Temp_{2} = \SI{300}{\degreeCelsius}$.
  The inlet of the nozzle is $\Area_{1} = \SI{50}{\centi\meter\squared}$.
  Heat is being lost at a rate of $Q_{Out} = \SI{120}{\kilo\joule\per\second}$.
  Find the mass flowrate of the steam $\FlowRate{\Mass}$, the exit velocity of the steam $\Velocity_{2}$, and the cross-sectional area of the nozzle's outlet $\Area_{2}$?
  \tcblower{}
  \textbf{Concepts:} \\
  The \textbf{nozzle} has steady flow, with \textit{steam}. \\
  A nozzle has no storage capacity, so $\FlowRate{\Mass}_{In} = \FlowRate{\Mass}_{Out}$. \\
  Steam is \emph{not} an ideal gas, so we have to use the Tables in the back of the textbook.

  \textbf{Explore:} \\
  Tables A.4, A.5, and A.6 will be of use.
  Initially, the steam is superheated, so we start by using Table A.6.

  Expanding the definition of mass flowrate, we get the relation:
  \begin{equation*}
    \FlowRate{\Mass} = \Density \Area \Velocity
  \end{equation*}

  It is also useful to remember
  \begin{equation*}
    \Density = \frac{1}{\SpecificVolume}
  \end{equation*}

  We assume that $\FlowRate{\Heat}_{In} = 0$, because we are not told otherwise. \\
  A nozzle is a rigid body, so it does not undergo any pressure changes.
  Thus, $\FlowRate{\Work}_{In} = \FlowRate{\Work}_{Out}$, meaning $\Change{\FlowRate{\Work}} = 0$. \\
  Again, $\Change{\PotentialEnergy} = 0$, because there is no change in height.

  \textbf{Plan:} \\
  Find the \nameref{def:Specific_Volume} for each state ($\SpecificVolume_{1}$ and $\SpecificVolume_{2}$), and their respective \nameref{def:Enthalpy} ($\SpecificEnthalpy_{1}$ and $\SpecificEnthalpy_{2}$). \\
  Use these values, along with the ones given in the problem statement to solve \Cref{eq:Steady_Flow_Energy_Continuity_Equation}.

  \textbf{Solve:} \\
  From Table A.6 at $\Pressure_{1} = \SI{5}{\mega\pascal}$ and $\Temp_{1} = \SI{400}{\degreeCelsius}$:
  \begin{align*}
    \SpecificVolume_{1} &= \SI{0.05784}{\meter\cubed\per\kilo\gram} \\
    \SpecificEnthalpy_{1} &= \SI{3196.7}{\kilo\joule\per\kilo\gram}
  \end{align*}

  Likewise, for the output, at $\Pressure_{2} = \SI{2}{\mega\pascal}$ and $\Temp_{2} = \SI{300}{\degreeCelsius}$:
  \begin{align*}
    \SpecificVolume_{2} &= \SI{0.12551}{\meter\cubed\per\kilo\gram} \\
    \SpecificEnthalpy_{2} &= \SI{3024.2}{\kilo\joule\per\kilo\gram}
  \end{align*}

  First, we find the mass flowrate through the nozzle, $\FlowRate{\Mass}$.
  \begin{align*}
    \FlowRate{\Mass} &= \frac{\Velocity_{1} \Area_{1}}{\SpecificVolume_{1}} \\
                     &= \SI{6.92}{\kilo\gram\per\second} \\
  \end{align*}

  Now, solve \Cref{eq:Steady_Flow_Energy_Continuity_Equation}, to find the final velocity $\Velocity_{2}$:
  \begin{align*}
    \FlowRate{\Heat}_{In} - \FlowRate{\Heat}_{Out} + \Change{\FlowRate{\Work}} &= \FlowRate{\Mass} \left( \Change{\SpecificEnthalpy} + \frac{\Change{\KineticEnergy}}{\Mass} \right) \\
    \intertext{The \nameref{def:Flow_Energy} is handled by the use of \nameref{def:Specific_Enthalpy} in this problem.}
    0 - \FlowRate{\Heat}_{Out} + 0 &= \FlowRate{\Mass} \left( \SpecificEnthalpy_{2} - \SpecificEnthalpy_{1} + \frac{\Velocity_{2}^{2} - \Velocity_{1}^{2}}{2} \right) \\
    -\FlowRate{\Heat}_{Out} &= \FlowRate{\Mass} \left( \SpecificEnthalpy_{2} - \SpecificEnthalpy_{1} + \frac{\Velocity_{2}^{2} - \Velocity_{1}^{2}}{2} \right) \\
    \FlowRate{\Mass} (\SpecificEnthalpy_{1} - \SpecificEnthalpy_{2}) - \FlowRate{\Heat}_{Out} &= \FlowRate{\Mass} \left( \frac{\Velocity_{2}^{2}}{2} - \frac{\Velocity_{1}^{2}}{2} \right) \\
    \FlowRate{\Mass} \left( \SpecificEnthalpy_{1} - \SpecificEnthalpy_{2} + \frac{\Velocity_{1}^{2}}{2} \right) - \FlowRate{\Heat}_{Out} &= \FlowRate{\Mass} \frac{\Velocity_{2}^{2}}{2} \\
    \Velocity_{2}^{2} &= \SI{562}{\meter\per\second}
  \end{align*}

  Lastly, we find the outlet's cross-sectional area, $\Area_{2}$.
  \begin{align*}
    \FlowRate{\Mass} &= \frac{\Velocity_{2} \Area_{2}}{\SpecificVolume_{2}} \\
    \Area_{2} &= \FlowRate{\Mass} \left( \frac{\SpecificVolume_{2}}{\Velocity_{2}} \right) \\
                     &= \SI{0.00154}{\meter\squared} \\
  \end{align*}

  \textbf{Validate:} \\
  \begin{center}
    \begin{tabular}{c|c}
      \toprule
      In & Out \\
      \midrule
      Pressure High & Pressure Low \\
      Velocity Low & Velocity High \\
      Start Temperature & Temperature Decreased \\
      Start Enthalpy & Enthalpy Decreased \\
      \bottomrule
    \end{tabular}
  \end{center}

  \textbf{Generalize:} \\
  The reduction in \nameref{def:Enthalpy} allowed the steam to gain its velocity, increasing the steam's kinetic energy.
  The temperature also decreased, because the enthalpy decreased.

  From the energy balance:
  \begin{align*}
    \FlowRate{\Heat}_{Out} &= \Change{\SpecificEnthalpy} + \Change{\KineticEnergy} \\
    \SI{-120}{\kilo\joule\per\second} &= \SI{-1194}{\kilo\joule\per\second} + \SI{1073}{\kilo\joule\per\second}
  \end{align*}
\end{example}

\subsubsection{Turbines and Compressors}\label{subsubsec:Turbines_Compressors}
\begin{definition}[Turbine]\label{def:Turbine}
  A \emph{turbine} takes some of the \nameref{def:Energy} in the fluid and transfers it to a shaft.
  The mass flowrate is constant.
  This means that there is a flowrate of \nameref{def:Work} out of the \nameref{def:System}.
\end{definition}

A \nameref{def:Turbine} is the opposite of a \nameref{def:Compressor}.

\begin{definition}[Compressor]\label{def:Compressor}
  A \emph{compressor} increases the \nameref{def:Energy} of a fluid by performing \nameref{def:Work} with the shaft.
  The mass flowrate is constant.
  This means that there is a flowrate of \nameref{def:Work} into the \nameref{def:System}.
\end{definition}

\begin{table}[h!tbp]
  \centering
  \begin{tabular}{cc}
    \toprule
    \nameref{def:Turbine}s & \nameref{def:Compressor}s \\
    \midrule
    $\FlowRate{\Work}_{In} = 0$ & $\FlowRate{\Work}_{In} \neq 0$ \\
    $\FlowRate{\Work}_{Out} \neq 0$ & $\FlowRate{\Work}_{Out} = 0$ \\
    $\FlowRate{\Heat}_{In} \approx 0$ & $\FlowRate{\Heat}_{In} \neq 0$ \\
    $\FlowRate{\Heat}_{Out} \approx 0$ & $\FlowRate{\Heat}_{Out} \neq 0$ \\
    $\Change{\PotentialEnergy} = 0$ & $\Change{\PotentialEnergy} = 0$ \\
    $\Change{\KineticEnergy} \neq 0$ & $\Change{\KineticEnergy} \approx 0$ \\
    $\Change{\Enthalpy} \neq 0$ & $\Change{\Enthalpy} \neq 0$ \\
    \bottomrule
  \end{tabular}
  \caption{Properties of \nameref*{def:Turbine}s and \nameref*{def:Compressor}s}
  \label{tab:Turbine_Compressor_Properties}
\end{table}

\begin{example}[Textbook Problem 6.52]{Adiabatic Turbines}
  Steam flows steadily through an \nameref{def:Adiabatic} \nameref{def:Turbine}.
  The inlet conditions of the steam are $\Pressure_{In} = \SI{10}{\mega\pascal}$, $\Temp_{In} = \SI{450}{\degreeCelsius}$, and $\Velocity_{In} = \SI{80}{\meter\per\second}$.
  The outlet conditions are $\Pressure_{Out} = \SI{10}{\kilo\pascal}$, $\Quality = 0.92$, and $\Velocity_{Out} = \SI{50}{\meter\per\second}$.
  The mass flowrate of the steam is $\FlowRate{\Mass} = \SI{12}{\kilo\gram\per\second}$.
  Find the change in kinetic energy $\Change{\KineticEnergy}$, the power output $\FlowRate{\Work}_{Out}$, and the inlet cross-sectional area $\Area_{In}$?
  \tcblower{}
  \textbf{Concepts:} \\
  This is a \nameref{def:Turbine} with a steady flow.
  So, we use the properties in \Cref{tab:Turbine_Compressor_Properties}. \\
  $\FlowRate{\Work}_{Out}$ is at the shaft.
  This is gathered at the expense of the mass flow's energy, both kinetic and enthalpy.
  Thus, $\Change{\KineticEnergy} \neq 0$ and $\Change{\SpecificEnthalpy} \neq 0$. \\
  The fluid at the outlet is a \nameref{def:Saturated_Mixture}. \\
  $\FlowRate{\Mass}_{In} = \FlowRate{\Mass}_{Out}$. \\
  $\FlowRate{\Energy}_{In} = \FlowRate{\Energy}_{Out}$.

  \textbf{Explore:} \\
  For kinetic energy, we need $\FlowRate{\Mass}$ and the velocities at the inlet and outlet. \\
  To solve for the output power, we need the \nameref{def:Specific_Enthalpy} for the steam at the states at the inlet and the outlet.
  To find these, we can use Tables A.4, A.5, and A.6 from the textbook. \\
  Due to the \nameref{def:Law_Conservation_Energy}, and the steady mass flow, we get \Cref{eq:Steady_Flow_Energy_Continuity_Equation}.
  \begin{align*}
    \Change{\FlowRate{\Heat}} + \Change{\FlowRate{\Work}} &= \sum\limits_{\text{Inlet}} \FlowRate{\Mass} \SpecificEnergy - \sum\limits_{\text{Outlet}} \FlowRate{\Mass} \SpecificEnergy \\
    (\FlowRate{\Heat}_{In} - \FlowRate{\Heat}_{Out}) + (\FlowRate{\Work}_{In} - \FlowRate{\Work}_{Out}) &= \sum\limits_{\text{Inlet}} \FlowRate{\Mass} \SpecificEnergy - \sum\limits_{\text{Outlet}} \FlowRate{\Mass} \SpecificEnergy \\
    (0 - 0) + (0 - \FlowRate{\Work}_{Out}) &= \sum\limits_{\text{Inlet}} \FlowRate{\Mass} \SpecificEnergy - \sum\limits_{\text{Outlet}} \FlowRate{\Mass} \SpecificEnergy \\
  \end{align*}

  Lastly, at the outlet, we know that the steam is a \nameref{def:Saturated_Mixture}, so we need to use the \nameref{def:Quality} to determine the \nameref{def:Specific_Enthalpy} of the steam.
  \begin{equation*}
    \SpecificEnthalpy = \Fluid{\SpecificEnthalpy} + \Quality(\Vapor{\SpecificEnthalpy} - \Fluid{\SpecificEnthalpy})
  \end{equation*}

  \textbf{Plan:} \\
  Solve for the change in kinetic energy $\Change{\KineticEnergy}$. \\
  Solve the energy balance equation we have to find the power output, $\FlowRate{\Work}_{Out}$. \\
  Using the mass flowrate and the inlet velocity, find the inlet's cross-sectional area, $\Area_{In}$.
  This can be found using the equations in \Cref{rmk:Alternative_Mass_Flowrate_Defns}.

  \textbf{Solve:} \\
  Find the inlet's information from Table A.6 in the textbook.
  At $\Pressure_{In} = \SI{10}{\mega\pascal}$, $\Temp_{In} = \SI{450}{\degreeCelsius}$
  \begin{align*}
    \SpecificVolume_{In} &= \SI{0.02978}{\meter\cubed\per\kilo\gram} \\
    \SpecificEnthalpy_{In} &= \SI{3242}{\kilo\joule\per\kilo\gram}
  \end{align*}

  Find the outlet's information from Table A.5 in the textbook.
  At $\Pressure_{Out} = \SI{10}{\kilo\pascal}$, $\Quality = 0.92$
  \begin{align*}
    \SaturatedTemp &= \SI{45.81}{\degreeCelsius} \\
    \Fluid{\SpecificEnthalpy} &= \SI{191.81}{\kilo\joule\per\kilo\gram} \\
    \Vapor{\SpecificEnthalpy} &= \SI{2583.9}{\kilo\joule\per\kilo\gram} \\
    \SpecificEnthalpy &= \Fluid{\SpecificEnthalpy} + \Quality (\Vapor{\SpecificEnthalpy} - \Fluid{\SpecificEnthalpy}) \\
    &= \SI{2392}{\kilo\joule\per\kilo\gram}
  \end{align*}

  Now that we have some background information, we will solve for the change in kinetic energy.
  \begin{align*}
    \Change{\KineticEnergy} &= \frac{\KineticEnergy_{Out} - \KineticEnergy_{In}}{\Mass} \\
                            &= \frac{\Velocity_{Out}^{2} - \Velocity_{In}^{2}}{2} \\
                            &= \frac{50^{2} - 80^{2}}{2} \left( \frac{\SI{1}{\kilo\joule\per\kilo\gram}}{\SI{1000}{\meter\squared\per\second\squared}} \right) \\
    &= \SI{-1.95}{\kilo\joule\per\kilo\gram}
  \end{align*}

  Now, we solve our energy balance equation for the power output.
  \begin{align*}
    -\FlowRate{\Work}_{Out} &= \FlowRate{\Mass} (\Change{\SpecificEnthalpy} + \Change{\KineticEnergy}) \\
    \FlowRate{\Work}_{Out} &= -\FlowRate{\Mass} \bigl( (\SpecificEnthalpy_{Out} - \SpecificEnthalpy_{In}) + \Change{\KineticEnergy} \bigr) \\
                            &= \FlowRate{\Mass} \bigl( (\SpecificEnthalpy_{In} - \SpecificEnthalpy_{Out}) - \Change{\KineticEnergy} \bigr) \\
                            &= \SI{12}{\kilo\gram\per\second} \bigl( (3242 - 2392)\si{\kilo\joule\per\kilo\gram} - (-\SI{1.95}{\kilo\joule\per\kilo\gram}) \bigr) \\
                            &= \SI{10.2}{\mega\joule\per\second} \\
                            &= \SI{10.2}{\mega\watt}
  \end{align*}

  Lastly, find the inlet's cross-sectional area:
  \begin{align*}
    \FlowRate{\Mass} \SpecificVolume_{In} &= \Area_{In} \Velocity_{In} \\
    \Area_{In} &= \SI{44.67}{\centi\meter\squared}
  \end{align*}

  \textbf{Validate:} \\
  We need to double check that we actually \textit{have} an energy balance.
  \begin{align*}
    \FlowRate{\Work}_{Out} &= \SI{10.22}{\mega\watt} \\
    &= \SI{10.2}{\mega\watt} + \SI{0.027}{\mega\watt}
  \end{align*}

  The values for properties of the states were from the textbook's tables.\\
  For \nameref{def:Turbine} operation, $\Change{\SpecificEnthalpy}$, $\Change{\KineticEnergy}$, $\Change{\Pressure}$, and $\Change{\Temp}$.

  \textbf{Generalize:} \\
  Most of the work comes from $\Change{\SpecificEnthalpy}$, very little comes from $\Change{\KineticEnergy}$. \\
  If this was not \nameref{def:Adiabatic}, then $\Change{\Heat}$ would be very small, like $\Change{\KineticEnergy}$. \\
  The mass flowrate as a large effect on the power generated.
\end{example}

\begin{example}[Textbook Problem 6.58]{Compressor}
  Helium is to be compressed from $\Pressure_{In} = \SI{120}{\kilo\pascal}$ and $\Temp_{In} = \SI{310}{\kelvin}$ to $\Pressure_{Out} = \SI{700}{\kilo\pascal}$ and $\Temp_{Out} = \SI{430}{\kelvin}$.
  A heat loss of $\FlowRate{\Heat}_{Out} = \SI{20}{\kilo\joule\per\kilo\gram}$ occurs during the compression process.
  Neglecting kinetic energy changes, determine the power required to complete this $\FlowRate{\Work}_{In}$ for $\FlowRate{\Mass} = \SI{90}{\kilo\gram\per\minute}$?
  \tcblower{}
  \textbf{Concepts:} \\
  Helium is an ideal gas, we can use \Cref{eq:Ideal_Gas_Equation}. \\
  A compressor will increase the energy of a fluid by increasing its pressure and enthalpy. \\
  We are told that the helium is flowing steadily because $\FlowRate{\Mass}$ is provided for us. \\
  We can find the change in enthalpy by using the \nameref{def:Specific_Heat} under constant pressure.
  We are told enough to say that our energy balance equation in this problem is \Cref{eq:Steady_Flow_Energy_Continuity_Equation}.
  \begin{equation*}
    \Change{\FlowRate{\Heat}} + \Change{\FlowRate{\Work}} = \FlowRate{\Mass} (\Change{\SpecificEnthalpy} + \Change{\KineticEnergy} + \Change{\PotentialEnergy})
  \end{equation*}

  \textbf{Explore:} \\
  We are told to neglect kinetic energy changes, so $\Change{\KineticEnergy} = 0$. \\
  We also know that there is no change in potential energy, because the height doesn't change, $\Change{\PotentialEnergy} = 0$. \\
  We can find the change in enthalpy by using the \nameref{def:Specific_Heat} under constant pressure, $\Change{\SpecificEnthalpy} = \SpecificHeatPressure \Change{\Temp}$. \\
  We are not told anything about any heat in or work out, so assume those are zero, $\FlowRate{\Heat}_{In} = 0$, $\FlowRate{\Work}_{Out} = 0$.
  Using all of this,
  \begin{equation*}
    -\FlowRate{\Heat}_{Out} + \FlowRate{\Work}_{In} = \FlowRate{\Mass} (\SpecificHeatPressure \Change{\Temp})
  \end{equation*}

  \textbf{Plan:} \\
  Find the \nameref{def:Specific_Heat} under constant pressure from Table A.3. \\
  Solve for $\FlowRate{\Work}_{In}$.

  \textbf{Solve:} \\
  From the textbook,
  \begin{equation*}
    \SpecificHeatPressure = \SI{5.1426}{\kilo\joule\per\kilo\gram\kelvin}
  \end{equation*}
  \begin{align*}
    -\SI{20}{\kilo\joule\per\kilo\gram} + \FlowRate{\Work}_{In} &= \SI{90}{\kilo\gram\per\minute} \left( \frac{\SI{60}{\second}}{\SI{1}{\minute}} \right) \bigl( 5.1426 (450 - 310) \bigr) \\
    \FlowRate{\Work}_{In} &= \SI{965}{\kilo\joule\per\second} \\
                                                                &= \SI{965}{\kilo\watt}
  \end{align*}
\end{example}

\subsubsection{Throttling Valves}\label{subsubsec:Throttling_Valves}
\begin{definition}[Throttling Valve]\label{def:Throttling_Valve}
  \emph{Throttling valve}s are flow restriction devices that produce large pressure and temperature drops.
  This is achieved with either an actual valve or a highly porous material.
\end{definition}

\begin{table}[h!tbp]
  \centering
  \begin{tabular}{c}
    \toprule
    \nameref{def:Throttling_Valve} \\
    \midrule
    $\Change{\FlowRate{\Heat}} = 0$ \\
    $\Change{\FlowRate{\Work}} = 0$ \\
    $\Change{\KineticEnergy} = 0$ \\
    $\Change{\PotentialEnergy} = 0$ \\
    $\SpecificEnthalpy_{In} = \SpecificEnthalpy_{Out}$ \\
    \bottomrule
  \end{tabular}
  \caption{Properties of \nameref*{def:Throttling_Valve}s}
  \label{tab:Throttling_Valve_Properties}
\end{table}

\begin{definition}[Isenthalpic]\label{def:Isenthalpic}
  A \nameref{def:Process} that is \emph{isenthalpic}, just like a process that is \nameref{def:Isothermal}, is one where the \nameref{def:Enthalpy} (or \nameref{def:Specific_Enthalpy}) is constant.
\end{definition}

Because the \nameref{def:Specific_Enthalpy} remains constant for a \nameref{def:Throttling_Valve}, we can expand the definition of \nameref{def:Specific_Enthalpy} to solve.
\begin{equation}\label{eq:Throttling_Valve}
  {(\SpecificInternalEnergy + \Pressure \Volume)}_{In} = {(\SpecificInternalEnergy + \Pressure \Volume)}_{Out}
\end{equation}

\begin{example}[Textbook Problem 6.68]{Throttling Valve}
  An \nameref{def:Adiabatic} capilliary tube is used in a refrigeration system to drop the pressure from the condenser level to the evaporator level.
  R-134a enters the capilliary tube as a \nameref{def:Saturated_Liquid} at $\Temp_{In} = \SI{50}{\degreeCelsius}$ and leaves at $\Temp_{Out} = \SI{-12}{\degreeCelsius}$.
  Determine the \nameref{def:Quality} of the refrigerant at the inlet of the evaporator?
  \tcblower{}
  \textbf{Concepts:} \\
  \nameref{def:Throttling_Valve}s are \nameref{def:Isenthalpic}, their \nameref{def:Enthalpy} remains constant.
  We also know \Cref{eq:Throttling_Valve}.
  \begin{equation*}
    {(\SpecificInternalEnergy + \Pressure \Volume)}_{In} = {(\SpecificInternalEnergy + \Pressure \Volume)}_{Out}
  \end{equation*}

  We assume that this operates under steady flow conditions, so $\FlowRate{\Mass}_{In} = \FlowRate{\Mass}_{Out}$. \\
  We are told this is \nameref{def:Adiabatic}, so $\Change{\Heat} = 0$. \\
  There are no moving parts, so $\Change{\Work} = 0$.

  \textbf{Explore:} \\
  We can find the \nameref{def:Specific_Enthalpy} for each state from the tables in the back of the textbook. \\
  It is implied that the refrigerant at the outlet is a \nameref{def:Saturated_Mixture}, which has a quality.
  \begin{equation*}
    \SpecificEnthalpy = \Fluid{\SpecificEnthalpy} + \Quality (\Vapor{\SpecificEnthalpy} - \Fluid{\SpecificEnthalpy})
  \end{equation*}

  \textbf{Plan:} \\
  Find \nameref{def:Specific_Enthalpy} values from Table A.11 from the textbook. \\
  Solve for $\SpecificEnthalpy_{Out}$. \\
  Then solve for \nameref{def:Quality}, $\Quality$.

  \textbf{Solve:} \\
  From Table A.11 at $\Temp_{In} = \SI{50}{\degreeCelsius}$, we know that this fluid is a \nameref{def:Saturated_Liquid}, so we can use $\Fluid{\SpecificEnthalpy}$ directly.
  However, we must \nameref{def:Interpolate} the results.
  \begin{equation*}
    \SpecificEnthalpy_{In} = \Fluid{\SpecificEnthalpy} = \SI{123.45}{\kilo\joule\per\kilo\gram}
  \end{equation*}

  From Table A.11 at $\Temp_{In} = \SI{-12}{\degreeCelsius}$.
  \begin{align*}
    \Fluid{\SpecificEnthalpy} &= \SI{35.92}{\kilo\joule\per\kilo\gram} \\
    \FluidVapor{\SpecificEnthalpy} &= \SI{207.38}{\kilo\joule\per\kilo\gram}
  \end{align*}

  Now, we plug these values into our \nameref{def:Isenthalpic} equation.
  \begin{align*}
    \SpecificEnthalpy_{In} &= \SpecificEnthalpy_{Out} \\
    \SpecificEnthalpy_{Out} &= \Fluid{\SpecificEnthalpy} + \Quality (\FluidVapor{\SpecificEnthalpy}) \\
    \SpecificEnthalpy_{In} &= \Fluid{\SpecificEnthalpy} + \Quality (\FluidVapor{\SpecificEnthalpy}) \\
    123.45 &= 35.92 + \Quality (207.38) \\
    \Quality &= 0.422
  \end{align*}

  \textbf{Validate:} \\
  We know at the inlet that
  \begin{align*}
    \SpecificEnthalpy &= \SpecificInternalEnergy + \Pressure \Volume \\
    123 &= 122 + \Pressure \Volume \\
    \Pressure \Volume &\approx 1
  \end{align*}

  Then, at the outlet, we see
  \begin{align*}
    \SpecificEnthalpy &= \SpecificInternalEnergy + \Pressure \Volume \\
    \shortintertext{Remember that a \nameref{def:Throttling_Valve} is \nameref{def:Isenthalpic}.}
    123 &= 36 + \Pressure \Volume \\
    \Pressure \Volume &\approx 87
  \end{align*}

  Because the \nameref{def:Internal_Energy} decreased, the temperature of the fluid decreased as well.

  \textbf{Generalize:} \\
  A \nameref{def:Throttling_Valve} is similar to a nozzle or a diffuser, because they re-apportion the \nameref{def:Energy} in the mass flow.
\end{example}

\subsubsection{Mixing Chambers}\label{subsubsec:Mixing_Chambers}
\begin{definition}[Mixing Chamber]\label{def:Mixing_Chamber}
  \emph{Mixing chamber}s are used whenever two different fluid streams together that have the same $\Density$.
  Each of the different streams comes in through a different inlet.
  There is a steady flow in from each of the inlets, so
  \begin{equation*}
    \FlowRate{\Mass}_{Out} = \FlowRate{\Mass}_{In, 1} + \FlowRate{\Mass}_{In, 2}
  \end{equation*}

  In general, we assume that the fluids entering the chamber are the same, preventing possible chemical reactions.
  For example, we do not mix water and alcohol.
\end{definition}

\begin{table}[h!tbp]
  \centering
  \begin{tabular}{c}
    \toprule
    \nameref{def:Mixing_Chamber} \\
    \midrule
    $\Change{\Heat} = 0$ \\
    $\Change{\Work} = 0$ \\
    $\Change{\KineticEnergy} = 0$ \\
    $\Change{\PotentialEnergy} = 0$ \\
    $\FlowRate{\Energy}_{In} = \FlowRate{\Energy}_{Out}$ \\
    \bottomrule
  \end{tabular}
  \caption{\nameref*{def:Mixing_Chamber} Properties}
  \label{tab:Mixing_Chamber_Properties}
\end{table}

%%% Local Variables:
%%% mode: latex
%%% TeX-master: "../../MMAE_320-Thermo-Reference_Sheet"
%%% End:
