\subsection{Energy Flows}\label{subsec:Energy_Flows}
\begin{table}[h!tbp]
  \centering
  \begin{tabular}{ccccc}
    \toprule
    & \nameref{def:Internal_Energy} & Kinetic Energy & Potential Energy & \nameref{def:Flow_Energy} \\
    \midrule
    \nameref{def:Specific_Energy}, $\SpecificEnergy$ & $\SpecificInternalEnergy$ & $\frac{\Velocity^{2}}{2}$ & $\Gravity \Distance$ & $\frac{\Pressure}{\Density}$ \\
    \nameref{def:Energy}, $\Energy$ & $\InternalEnergy$ & $\frac{\Mass \Velocity^{2}}{2}$ & $\Mass \Gravity \Distance$ & $\Pressure \Volume$ \\
    \bottomrule
  \end{tabular}
  \caption{Types of Energy for Mass Flow}
  \label{tab:Energy_Flows}
\end{table}

We can often connect the \nameref{def:Internal_Energy} and the \nameref{def:Flow_Energy} of a process and call that \nameref{def:Enthalpy}, $\SpecificEnthalpy$ or $\Enthalpy$ for a particular control volume.
\begin{align*}
  \SpecificEnthalpy &= \SpecificInternalEnergy + \frac{\Pressure}{\Density} \\
  \Enthalpy &= \InternalEnergy + \FlowEnergy
\end{align*}

And we know that power is the time-rate of change of energy.
\begin{align*}
  \Power &= \FlowRate{\Energy} \\
  &= \FlowRate{\Mass} \SpecificEnergy
\end{align*}

Just like with mass flows, if a control volume has not storage capacity, then it also has no energy.
Thus, if there is no storage, and the \nameref{def:Law_Conservation_Energy} is abided by, then:
\begin{align*}
  \FlowRate{\Mass}_{In} - \FlowRate{\Mass}_{Out} &= \FlowRate{\Mass}_{\text{Control Volume}} \\
  \FlowRate{\Energy}_{In} - \FlowRate{\Energy}_{Out} &= \FlowRate{\Energy}_{\text{Control Volume}} \\
  \FlowRate{\Mass}_{In} &= \FlowRate{\Mass}_{Out} \\
  \FlowRate{\Energy}_{In} &= \FlowRate{\Energy}_{Out}
\end{align*}

Looking at the equations received, it is obvious that we can apply any kind of \nameref{def:Energy} that may be present in the system to the above equations and solve for any term we may be interested in.
If we now expand our \nameref{def:Continuity_Equation} to energies, we get \Cref{eq:Energy_Continuity_Equation}.

\begin{equation}\label{eq:Energy_Continuity_Equation}
  \begin{aligned}
    \FlowRate{\Heat}_{In} + \FlowRate{\Work}_{In} + \sum\limits_{\text{In}} \FlowRate{\Mass} \SpecificEnergy &= \FlowRate{\Heat}_{Out} + \FlowRate{\Work}_{Out} + \sum\limits_{\text{Out}} \FlowRate{\Mass} \SpecificEnergy \\
    \FlowRate{\Heat}_{In} + \FlowRate{\Work}_{In} + \sum\limits_{\text{In}} \FlowRate{\Mass} (\SpecificEnthalpy + \frac{\Velocity^{2}}{2} + \Gravity \Distance) &= \FlowRate{\Heat}_{Out} + \FlowRate{\Work}_{Out} + \sum\limits_{\text{Out}} \FlowRate{\Mass} (\SpecificEnthalpy + \frac{\Velocity^{2}}{2} + \Gravity \Distance)
  \end{aligned}
\end{equation}

If we have a steady flow, then \Cref{eq:Energy_Continuity_Equation} can be simplified a bit.
\begin{equation}\label{eq:Steady_Flow_Energy_Continuity_Equation}
  \begin{aligned}
    \Change{\FlowRate{\Heat}} - \Change{\FlowRate{\Work}} &= \FlowRate{\Mass} \left( \Change{\SpecificEnthalpy} + \Change{\KineticEnergy} + \Change{\PotentialEnergy} \right) \\
    &= \FlowRate{\Mass} \left( (\SpecificEnthalpy_{2} - \SpecificEnthalpy_{1}) + \frac{\Velocity_{2}^{2} - \Velocity_{1}^{2}}{2} + \Gravity (\Distance_{2} - \Distance_{1}) \right)
  \end{aligned}
\end{equation}


%%% Local Variables:
%%% mode: latex
%%% TeX-master: "../../MMAE_320-Thermo-Reference_Sheet"
%%% End:
