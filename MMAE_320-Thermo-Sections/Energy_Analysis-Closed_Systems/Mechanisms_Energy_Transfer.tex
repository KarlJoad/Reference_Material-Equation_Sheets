\subsection{Mechanisms of Energy Transfer}\label{subsec:Mechanisms_Energy_Transfer}
From our discussion of changes in energy in \Cref{subsec:Divisions_of_Energy}, we know that there are 3 main mechanisms of energy transfer:
\begin{enumerate}[noitemsep]
\item \nameref{def:Heat_Transfer}
\item \nameref{def:Work}
\item Mass Flow
\end{enumerate}

In addition, according to the \nameref{def:Law_Conservation_Energy}, we know that the total amount of energy must be conserved throughout a process.
This leads to 2 separate types of processes occurring.
\begin{description}[noitemsep]
\item[Cyclical] A cyclic process is one where the beginning and end states are identical.
  This also means that $\Heat_{In} = \Work_{Out}$ and $\Heat_{Out} = \Work_{In}$, making $\Change{\Energy} = 0$, and $\Energy_{In} = \Energy_{Out}$.
\item[Regular] A regular process is the general way to deal with processes, and cyclic processes are ones that are special cases of a regular process.
  In this case, there is a change in energy $\Change{\Energy} = (\Heat_{In} - \Heat_{Out}) + (\Work_{In} - \Work_{Out})$.
\end{description}

Now, we can generalize \Cref{eq:Change_Total_Energy} even more, to allow for the changing of systems.
We start with an equation that represents the energy of the entire \nameref{def:System}, \Cref{eq:System_Energy}.

\begin{equation}\label{eq:System_Energy}
  \InternalEnergy + \KineticEnergy + \PotentialEnergy + \FlowEnergy = \Heat + \Work
\end{equation}

Now, if we want to discuss the change in the total energy of the system, we can use \Cref{eq:Change_System_Energy}.

\begin{equation}\label{eq:Change_System_Energy}
  \begin{aligned}
    \Change{\InternalEnergy} + \Change{\KineticEnergy} + \Change{\PotentialEnergy} + \Change{\FlowEnergy} &= \Change{\Heat} + \Change{\Work} \\
    &= (\Heat_{In} - \Heat_{Out}) + (\Work_{In} - \Work_{Out})
  \end{aligned}
\end{equation}

The implications of \Cref{eq:Change_System_Energy} are listed below.
\begin{itemize}[noitemsep]
\item For stationary systems, $\Change{\KineticEnergy} = 0$, and $\Change{\PotentialEnergy} = 0$.
\item For stationary and \nameref{def:Closed_System}s (when there is \textbf{no} mass transfer) $\Change{\KineticEnergy} = 0$, $\Change{\PotentialEnergy} = 0$, $\Change{\FlowEnergy} = 0$.
  Therefore, $\Change{\InternalEnergy} = \Change{\Heat} + \Change{\Work}$.
\end{itemize}

\begin{example}[Textbook Problem 5.29]{Enthalpy of Saturated Water}
  Saturated water vapor at $\Temp = \SI{200}{\degreeCelsius}$ and is \nameref{def:Isothermal}ly condensed into a \nameref{def:Saturated_Liquid} in a piston-cylinder device.
  Calculate the \nameref{def:Heat_Transfer} and \nameref{def:Work} done during this process in \si{\kilo\joule\per\kilo\gram}?
  \tcblower{}
  \textbf{Concepts:} \\
  A piston-cylinder means that the pressure is constant throughout the duration of a \nameref{def:Process}. \\
  An \nameref{def:Isothermal} process is one in which the temperature is constant. \\
  The water is moving from a \nameref{def:Saturated_Vapor} to a \nameref{def:Saturated_Liquid}. \\
  The total energy is conserved throughout the \nameref{def:Process}

  \textbf{Explore:} \\
  From our understanding of \nameref{def:Saturated_Mixture}s, we know that we should use Table A.4 from the textbook's appendix.
  This is both a stationary system and a \nameref{def:Closed_System}, so $\Change{\KineticEnergy} = 0$, $\Change{\PotentialEnergy} = 0$, and $\Change{\FlowEnergy} = 0$.
  Therefore, the only terms in \Cref{eq:Change_System_Energy} we need to care about are
  \begin{equation*}
    \Change{\InternalEnergy} = \Change{\Heat} + \Change{\Work}
  \end{equation*}

  To simplify this further, we can assume that at the starting state, there was no \nameref{def:Heat} going in and no \nameref{def:Work} being done.
  Therefore, $\Heat_{In} = 0$ and $\Work_{In} = 0$.
  So, the equation above simplifes down to
  \begin{align*}
    \InternalEnergy_{2} - \InternalEnergy_{1} &= -\Heat_{Out} + -\Work_{Out} \\
                                              &= -\Heat_{Out} - \Pressure(\Volume_{2} - \Volume_{1})
  \end{align*}

  The difference in \nameref{def:Internal_Energy} is actually recorded in the textbook's Table A.4 as $\InternalEnergy_{fg}$.

  \textbf{Plan:} \\
  Solve for the above equation, using Table A.4 in the textbook to find the value of the change in \nameref{def:Internal_Energy} of the system.

  \textbf{Solve:} \\
  From Table A.4, we know that at $\Temp = \SI{200}{\degreeCelsius}$, $\SaturatedPressure = \SI{1555}{\kilo\pascal}$.
  We also know $\SaturatedFluidVol = \SI{0.001157}{\meter\cubed\per\kilo\gram}$ and $\SaturatedVaporVol = \SI{0.12721}{\meter\cubed\per\kilo\gram}$.
  Lastly, we can also find $\SaturatedVaporSpecificIntEnergy = \SI{2594}{\kilo\joule\per\kilo\gram}$ and $\SaturatedFluidSpecificIntEnergy = \SI{850.46}{\kilo\joule\per\kilo\gram}$.

  \begin{align*}
    \Work_{Out} &= 1555 (0.001157 - 0.12621) \\
                &= \SI{-196}{\kilo\pascal\meter\cubed\per\kilo\gram} \\
                &= \SI{-196}{\kilo\joule\per\kilo\gram} \\
    \SaturatedFluidSpecificIntEnergy - \SaturatedVaporIntEnergy &= 850 - 2594 \\
                &= \SI{-1744}{\kilo\joule\per\kilo\gram} \\
    \SaturatedFluidSpecificIntEnergy - \SaturatedVaporIntEnergy &= -\Heat_{Out} - \Work_{Out} \\
    \SI{-1744}{\kilo\joule\per\kilo\gram} &= -\Heat_{Out} - \SI{-196}{\kilo\joule\per\kilo\gram} \\
    \Heat_{Out} &= \SI{1940}{\kilo\joule\per\kilo\gram}
  \end{align*}

  Thus, the vapor is cooling and condensing in the piston-cylinder, which makes sense in this case.

  \textbf{Validate:} \\
  We can validate this by performing the same calculations using the change in \nameref{def:Specific_Enthalpy} instead.
  \begin{align*}
    \Change{\SpecificEnthalpy} &= -\Heat_{Out} \\
    \SaturatedFluidSpecificEnthalpy - \SaturatedVaporSpecificEnthalpy &= -\Heat_{Out} \\
    \intertext{The change in specific enthalpy can be found in Table A.4 again.}
    (852.3 - 2792) \si{\kilo\joule\per\kilo\gram} &= -\Heat_{Out} \\
    \Heat_{Out} &= \SI{1939.7}{\kilo\joule\per\kilo\gram} \\
    \shortintertext{Now, taking significant figures.}
    \Heat_{Out} &= \SI{1940}{\kilo\joule\per\kilo\gram}
  \end{align*}

  \textbf{Generalize:} \\
  Any time the system is changing states, we can use \nameref{def:Enthalpy} and/or \nameref{def:Specific_Enthalpy}.
  This helps us avoid these longer calculations using all the other forms of \nameref{def:Energy} in the \nameref{def:System}.
\end{example}


%%% Local Variables:
%%% mode: latex
%%% TeX-master: "../../MMAE_320-Thermo-Reference_Sheet"
%%% End:
