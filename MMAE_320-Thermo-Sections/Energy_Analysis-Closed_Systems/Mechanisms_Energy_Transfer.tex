\subsection{Mechanisms of Energy Transfer}\label{subsec:Mechanisms_Energy_Transfer}
From our discussion of changes in energy in \Cref{subsec:Divisions_of_Energy}, we know that there are 3 main mechanisms of energy transfer:
\begin{enumerate}[noitemsep]
\item \nameref{def:Heat_Transfer}
\item \nameref{def:Work}
\item Mass Flow
\end{enumerate}

In addition, according to the \nameref{def:Law_Conservation_Energy}, we know that the total amount of energy must be conserved throughout a process.
This leads to 2 separate types of processes occurring.
\begin{description}[noitemsep]
\item[Cyclical] A cyclic process is one where the beginning and end states are identical.
  This also means that $\Heat_{In} = \Work_{Out}$ and $\Heat_{Out} = \Work_{In}$, making $\Change{\Energy} = 0$, and $\Energy_{In} = \Energy_{Out}$.
\item[Regular] A regular process is the general way to deal with processes, and cyclic processes are ones that are special cases of a regular process.
  In this case, there is a change in energy $\Change{\Energy} = (\Heat_{In} - \Heat_{Out}) + (\Work_{In} - \Work_{Out})$.
\end{description}

Now, we can generalize \Cref{eq:Change_Total_Energy} even more, to allow for the changing of systems.
We start with an equation that represents the energy of the entire \nameref{def:System}, \Cref{eq:System_Energy}.

\begin{equation}\label{eq:System_Energy}
  \InternalEnergy + \KineticEnergy + \PotentialEnergy + \FlowEnergy = \Heat + \Work
\end{equation}

Now, if we want to discuss the change in the total energy of the system, we can use \Cref{eq:Change_System_Energy}.

\begin{equation}\label{eq:Change_System_Energy}
  \begin{aligned}
    \Change{\InternalEnergy} + \Change{\KineticEnergy} + \Change{\PotentialEnergy} + \Change{\FlowEnergy} &= \Change{\Heat} + \Change{\Work} \\
    &= (\Heat_{In} - \Heat_{Out}) + (\Work_{In} - \Work_{Out})
  \end{aligned}
\end{equation}


%%% Local Variables:
%%% mode: latex
%%% TeX-master: "../../MMAE_320-Thermo-Reference_Sheet"
%%% End:
