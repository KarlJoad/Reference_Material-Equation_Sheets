\subsection{Incompressible Substances}\label{subsubsec:Incompressible_Substances}
In general, because there is little that the substance \textit{can} do, because it is incompressible, it leads to the \nameref{def:Specific_Heat}s being equal to each other.
\begin{equation*}
  \SpecificHeatPressure = \SpecificHeatVolume
\end{equation*}

\begin{example}[Textbook Problem 5.66]{Using Specific Heat}
  A $\SI{4}{\meter} \times \SI{5}{\meter} \times \SI{6}{\meter}$ room is to be heated by a baseboard resistance heater.
  This heater must raise the temperature from \SI{7}{\degreeCelsius} to \SI{23}{\degreeCelsius} within 15 minutes.
  Assuming no heat losses in the room, and an absolute pressure of $\Pressure = \SI{100}{\kilo\pascal}$ determine the required power of the resistance heater assuming a constant \nameref{def:Specific_Heat} at room temperature?
  \tcblower{}
  \textbf{Concepts:} \\
  Room has air at roughly \SI{1}{\atm}. \\
  This is ideally a \nameref{def:Closed_System}, but if the room is leaky, then it is an \nameref{def:Open_System}. \\
  Air can be treated as an ideal gas. \\
  The resistance heater is inside the room and no mention is made of temperature outside the room, meaning there is no heat gain nor is there any heat loss.

  \textbf{Explore:} \\
  We could use different \nameref{def:Specific_Heat}s depending on the system types.
  However, we can assume that the problem meant this system to be a \nameref{def:Closed_System}.
  This implies a constant volume, so $\SpecificHeatVolume$ should be used.
  We need Table A.2 in the textbook's appendix.
  Under the implication of constant volume, for air, $\SpecificHeatVolume = \SI{0.718}{\kilo\joule\per\kilo\gram\kelvin}$.
  However, for air under constant pressure, an \nameref{def:Open_System}, $\SpecificHeatPressure = \SI{1.005}{\kilo\joule\per\kilo\gram\kelvin}$.

  Now, as a discussion of the energy balance of this system, we use the equation below.
  \begin{equation*}
    \Change{\InternalEnergy} = \Change{\Heat} + \Change{\Work}
  \end{equation*}

  We were tole that $\Change{\Heat} = 0$ because there is no heat loss in the room, and likewise, there are no heat gains.
  Therefore,
  \begin{equation*}
    \Change{\InternalEnergy} = \Work_{In} - \Work_{Out}
  \end{equation*}

  If the room were leaky, and this were an \nameref{def:Open_System}, then $\Work_{Out} \neq 0$.
  We know that $\Work_{In}$ is the work done on the room due to the heater.

  In addition, we know that we can calculate the change in \nameref{def:Internal_Energy} by using the \nameref{def:Specific_Heat}.
  \begin{align*}
    \Change{\InternalEnergy} &= \SpecificHeatVolume \Mass \Change{\Temp} \\
    &= \FlowRate{\Work} \Change{\Time}
  \end{align*}

  \textbf{Plan:} \\
  Find all the mass in the system.
  Use $\SpecificHeatVolume$ to solve for this.

  \textbf{Solve:} \\
  Start by finding the mass of the air in the room.
  \begin{align*}
    \Mass &= \frac{\Pressure \Volume}{\GasConstant \Temp} \\
    &= \SI{149}{\kilo\gram}
  \end{align*}

  Now, move onto finding the change in \nameref{def:Internal_Energy}.
  \begin{align*}
    \Change{\InternalEnergy} &= \SpecificHeatVolume \Mass \Change{\Temp} \\
                             &= 0.718 (149) (23 - 7) \\
                             &= \SI{1711}{\kilo\joule}
  \end{align*}

  Now plug that into our equation about the work done multiplied by the time.
  \begin{align*}
    \Change{\InternalEnergy} &= \FlowRate{\Work} \Change{\Time} \\
    \SI{1711}{\kilo\joule} &= \FlowRate{\Work} \left( \SI{15}{\minute} \frac{\SI{60}{\second}}{\SI{1}{\minute}} \right) \\
    \FlowRate{\Work} &= \SI{1.9}{\kilo\watt}
  \end{align*}
\end{example}

%%% Local Variables:
%%% mode: latex
%%% TeX-master: "../../MMAE_320-Thermo-Reference_Sheet"
%%% End:
