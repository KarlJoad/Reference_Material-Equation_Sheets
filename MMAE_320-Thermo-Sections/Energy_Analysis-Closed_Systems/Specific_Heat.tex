\subsection{Heat and Specific Heat for Energy Transfer}\label{subsec:Heat_Specific_Heat_Energy_Transfer}
\begin{definition}[Specific Heat]\label{def:Specific_Heat}
  \emph{Specific heat} is the \nameref{def:Energy} required to raise the termperature of a unit mass of a material by one degree.
  Specific heat is denoted with $\SpecificHeat$.
  This is a \textbf{linear} term between temperature and energy.
  Like any other \nameref{def:Specific_Property}, specific heat is also an \nameref{def:Intensive_Property}.

  \begin{remark}[Possible Materials to Apply to]
    This \nameref{def:Intensive_Property} only works for the following materials, given they \textbf{do not undergo phase changes}:
    \begin{itemize}[noitemsep]
    \item Noble Gases and Air
    \item Liquids
    \item Solids
    \end{itemize}
  \end{remark}

  \begin{remark}[Other Notable Properties]
    \nameref{def:Specific_Heat} is \textbf{not} always linear over large temperature changes or when under very high pressures or temperatures.
  \end{remark}
\end{definition}

There are 2 cases of using \nameref{def:Specific_Heat}:
\begin{enumerate}[noitemsep]
\item \nameref{subsubsec:Specific_Heat_Constant_Volume}
\item \nameref{subsubsec:Specific_Heat_Constant_Pressure}
\end{enumerate}

Both of these cases are \textbf{NOT} the same, meaning $\SpecificHeatVolume \neq \SpecificHeatPressure$.
{\large\textbf{However}}, there \textbf{is} a relation between $\SpecificHeatVolume$ and $\SpecificHeatPressure$, shown in \Cref{eq:Specific_Heat_Constant_Volume_Constant_Pressure_Relation}.

\begin{equation}\label{eq:Specific_Heat_Constant_Volume_Constant_Pressure_Relation}
  \SpecificHeatPressure = \SpecificHeatVolume + \GasConstant
\end{equation}

\subsubsection{Constant Volume}\label{subsubsec:Specific_Heat_Constant_Volume}
If the material in question has a \nameref{def:Specific_Heat} and undergoes a process at constant volume, then:
\begin{equation*}
  \SpecificHeatVolume \equiv {\left[ \frac{\partial \InternalEnergy}{\partial \Temp} \right]}_{\Volume}
\end{equation*}

The equation above is approximately linear for smaller changes in both temperature and volume, which yields \Cref{eq:Change_Internal_Energy_Specific_Heat_Constant_Volume}

\begin{equation}\label{eq:Change_Internal_Energy_Specific_Heat_Constant_Volume}
  \Change{\InternalEnergy} = \SpecificHeat_{V} \Mass \Change{\Temp}
\end{equation}

\subsubsection{Constant Pressure}\label{subsubsec:Specific_Heat_Constant_Pressure}
If the material in question has a \nameref{def:Specific_Heat} and undergoes a process at constant pressure, then:
\begin{equation*}
  \SpecificHeatPressure \equiv {\left[ \frac{\partial \Enthalpy}{\partial \Temp} \right]}_{\Pressure}
\end{equation*}

The equation above is approximately linear for smaller changes in both temperature and volume, which yields \Cref{eq:Change_Internal_Energy_Specific_Heat_Constant_Pressure}

\begin{equation}\label{eq:Change_Internal_Energy_Specific_Heat_Constant_Pressure}
  \Change{\Enthalpy} = \SpecificHeatPressure \Mass \Change{\Temp}
\end{equation}

\begin{example}[Textbook Problem 5.79]{Energy Required by Heater}
  A mass of $\Mass = \SI{15}{\kilo\gram}$ of air in a piston-cylinder device is heated from $\Temp_{1} = \SI{25}{\degreeCelsius}$ to $\Temp_{2} = \SI{77}{\degreeCelsius}$ by passing electric current through a resistance heater inside the cylinder.
  The \nameref{def:Pressure} inside the cylinder is held constant at $\Pressure = \SI{300}{\kilo\pascal}$ throughout the \nameref{def:Process}.
  The \nameref{def:Heat} loss in the cylidner is $\Heat = \SI{60}{\kilo\joule}$.
  Determine the electric energy required by the electric heater in \si{\kilo\watt\hour}?
  \tcblower{}
  \textbf{Concepts:} \\
  Because this is a piston-cylinder, the \nameref{def:Pressure} is constant. \\
  This has no limitation on the volume of the air inside the piston-cylinder.
  Because the heating is greater than the heat loss, the air will expand. \\
  This is a \nameref{def:Closed_System}, so there is a heat transfer and it is possible to do work. \\
  The electric output of the heater is measured in \si{\kilo\watt\hour}, which is power multiplied by time.
  Thus, $\SI{1}{\kilo\watt\hour} = \SI{3600}{\kilo\joule}$. \\
  The energy balance for this problem is summarized by $\Change{\InternalEnergy} = \Change{\Heat} + \Change{\Work}$. \\
  \nameref{def:Enthalpy} exists in a system only when a new \nameref{def:System} is being created.
  There was work done by the air as it expanded on the piston, and it moved the air outside of the piston too, so we can use \nameref{def:Enthalpy} here.

  \textbf{Explore:} \\
  We start by looking at $\Change{\Heat}$.
  Because there is no heat going into the system from the surroundings, $\Heat_{In} = 0$.
  However, because the system is losing heat to the surroundings, $\Heat_{Out} = \SI{60}{\kilo\joule}$.

  Now, looking at $\Change{\Work}$.
  The electric heater is doing work on the air, by heating it up, $\Work_{In}$ is equal to the work done by the electric heater.
  Then, because the air is expanding, it is also performing work on the piston-cylinder to move the piston.
  Thus, we determine this is flow energy, where the pressure is constant, so $\Work_{Out} = \Pressure \Volume$.

  Lastly, because we are dealing with a system that \emph{is} changing, we can use \nameref{def:Enthalpy} instead.
  In $\Change{\Enthalpy} = -\Heat_{Out} + \Work_{In}$, the $\Work_{Out}$ is actually implicitly added to the $\Change{\Enthalpy}$ side, so it doesn't need to be recorded twice.
  This is because $\Enthalpy = \InternalEnergy + \Pressure \Volume$, from \Cref{eq:Enthalpy}.

  \begin{align*}
    \Change{\InternalEnergy} &= -\Heat_{Out} + \Work_{In} - \Work_{Out} \\
                             &= -\Heat_{Out} + \Work_{In} - \Pressure \Change{\Volume} \\
    \Change{\InternalEnergy} + \Pressure \Change{\Volume} &= -\Heat_{Out} + \Work_{In} \\
    \Change{\Enthalpy} &= -\Heat_{Out} + \Work_{In} \\
  \end{align*}

  \textbf{Plan:} \\
  Solve $\Change{\Enthalpy} = -\Heat_{Out} + \Work_{In}$. \\
  Find the states from tables in the back of the textbook, namely Table A.21.
  We need to keep in mine that the tables will have the \nameref{def:Specific_Enthalpy}, so we must change it into just \nameref{def:Enthalpy} with $\Enthalpy = \Mass \SpecificEnthalpy$.

  \textbf{Solve:} \\
  From Table A.21 in the textbook: at $\Temp_{1} = \SI{25}{\degreeCelsius}$, $\SpecificEnthalpy_{1} = \SI{298.18}{\kilo\joule\per\kilo\gram}$.
  Likewise, at $\Temp_{2} = \SI{77}{\degreeCelsius}$, $\SpecificEnthalpy_{2} = \SI{350.5}{\kilo\joule\per\kilo\gram}$.

  Now that we have every term in our equation to use for solving, we solve it.
  \begin{align*}
    \Change{\Enthalpy} &= -\Heat_{Out} + \Work_{In} \\
    \Mass \Change{\SpecificEnthalpy} &= -60 + \Work_{In} \\
    \Mass (\SpecificEnthalpy_{2} - \SpecificEnthalpy_{1}) + 60 &= \Work_{In} \\
    \Work_{In} &= 15 (350.5 - 298.18) + 60 \\
                       &= \SI{844.8}{\kilo\joule}
  \end{align*}

  Now, we have to divide by $\SI{1}{\kilo\watt\hour} = \SI{3600}{\kilo\joule}$.
  \begin{align*}
    \Work_{In} &= \SI{844.8}{\kilo\joule} \\
               &= \frac{844.8}{3600} \\
               &= \SI{0.235}{\kilo\watt\hour}
  \end{align*}

  \textbf{Validate:} \\
  We could do this with \nameref{def:Specific_Heat} as well.
  \begin{align*}
    \Change{\InternalEnergy} &= \SpecificHeatVolume \Mass \Change{\Temp} \\
    \Change{\Enthalpy} &= \SpecificHeatPressure \Mass \Change{\Temp}
  \end{align*}

  But, because this \nameref{def:Process} was done under constant \nameref{def:Pressure}, we need to use the second one.
  $\SpecificHeatPressure = \SI{1.0065}{\kilo\joule\per\kilo\gram\kelvin}$ is taken from Table A.2, and was averaged over the $\SpecificHeatPressure$ values for the range $\SI{0}{\degreeCelsius}$ and $\SI{100}{\degreeCelsius}$.
  Because of the way $\SpecificHeatPressure$ is defined, we must convert our temperatures into Kelvin too, which is simple.
  Once done, we can solve this equation.
  \begin{align*}
    \Change{\Enthalpy} &= \SpecificHeatPressure \Mass \Change{\Temp} \\
                       &= 1.0065 (\SI{15}{\kilo\gram}) (350 - 298) \si{\kelvin} \\
                       &= \SI{785.07}{\kilo\joule}
  \end{align*}

  Now, if we account for the $\SI{60}{\kilo\joule}$ of heat loss, then $785.07 + 60 = \SI{845.07}{\kilo\joule}$, which is about the same.

  \textbf{Generalize:} \\
  We can use the state of the \nameref{def:System} and the energy balance equation or the \nameref{def:Specific_Heat} to find the total \nameref{def:Energy} of the system.
\end{example}

%%% Local Variables:
%%% mode: latex
%%% TeX-master: "../../MMAE_320-Thermo-Reference_Sheet"
%%% End:
