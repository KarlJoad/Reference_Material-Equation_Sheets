\subsection{Heat and Specific Heat for Energy Transfer}\label{subsec:Heat_Specific_Heat_Energy_Transfer}
\begin{definition}[Specific Heat]\label{def:Specific_Heat}
  \emph{Specific heat} is the \nameref{def:Energy} required to raise the termperature of a unit mass of a material by one degree.
  Specific heat is denoted with $\SpecificHeat$.
  This is a \textbf{linear} term between temperature and energy.
  Like any other \nameref{def:Specific_Property}, specific heat is also an \nameref{def:Intensive_Property}.

  \begin{remark}[Possible Materials to Apply to]
    This \nameref{def:Intensive_Property} only works for the following materials, given they \textbf{do not undergo phase changes}:
    \begin{itemize}[noitemsep]
    \item Noble Gases and Air
    \item Liquids
    \item Solids
    \end{itemize}
  \end{remark}

  \begin{remark}[Other Notable Properties]
    \nameref{def:Specific_Heat} is \textbf{not} always linear over large temperature changes or when under very high pressures or temperatures.
  \end{remark}
\end{definition}

There are 2 cases of using \nameref{def:Specific_Heat}:
\begin{enumerate}[noitemsep]
\item \nameref{subsubsec:Specific_Heat_Constant_Volume}
\item \nameref{subsubsec:Specific_Heat_Constant_Pressure}
\end{enumerate}

Both of these cases are \textbf{NOT} the same, meaning $\SpecificHeatVolume \neq \SpecificHeatPressure$.
{\large\textbf{However}}, there \textbf{is} a relation between $\SpecificHeatVolume$ and $\SpecificHeatPressure$, shown in \Cref{eq:Specific_Heat_Constant_Volume_Constant_Pressure_Relation}.

\begin{equation}\label{eq:Specific_Heat_Constant_Volume_Constant_Pressure_Relation}
  \SpecificHeatPressure = \SpecificHeatVolume + \GasConstant
\end{equation}

\subsubsection{Constant Volume}\label{subsubsec:Specific_Heat_Constant_Volume}
If the material in question has a \nameref{def:Specific_Heat} and undergoes a process at constant volume, then:
\begin{equation*}
  \SpecificHeatVolume \equiv {\left[ \frac{\partial \InternalEnergy}{\partial \Temp} \right]}_{\Volume}
\end{equation*}

The equation above is approximately linear for smaller changes in both temperature and volume, which yields \Cref{eq:Change_Internal_Energy_Specific_Heat_Constant_Volume}

\begin{equation}\label{eq:Change_Internal_Energy_Specific_Heat_Constant_Volume}
  \Change{\InternalEnergy} = \SpecificHeat_{V} \Mass \Change{\Temp}
\end{equation}

\subsubsection{Constant Pressure}\label{subsubsec:Specific_Heat_Constant_Pressure}
If the material in question has a \nameref{def:Specific_Heat} and undergoes a process at constant pressure, then:
\begin{equation*}
  \SpecificHeatPressure \equiv {\left[ \frac{\partial \Enthalpy}{\partial \Temp} \right]}_{\Pressure}
\end{equation*}

The equation above is approximately linear for smaller changes in both temperature and volume, which yields \Cref{eq:Change_Internal_Energy_Specific_Heat_Constant_Pressure}

\begin{equation}\label{eq:Change_Internal_Energy_Specific_Heat_Constant_Pressure}
  \Change{\Enthalpy} = \SpecificHeatPressure \Mass \Change{\Temp}
\end{equation}


%%% Local Variables:
%%% mode: latex
%%% TeX-master: "../../MMAE_320-Thermo-Reference_Sheet"
%%% End:
