\section{Introduction}\label{sec:Introduction}
\begin{definition}[Energy]\label{def:Energy}
  \emph{Energy} is the stuff in the universe that can cause changes in material states.
  Some examples are:
  \begin{itemize}[noitemsep]
  \item Thermal Energy
  \item Electrical Energy
  \item Mechanical Energy
  \end{itemize}
\end{definition}

One fundamental thing about \nameref{def:Energy} that was discovered was the \nameref{def:Law_Conservation_Energy}.
\begin{definition}[Law of Conservation of Energy]\label{def:Law_Conservation_Energy}
  The \emph{law of conservation of energy} states that energy cannot be created or destroyed; it can only change form.
  This means that the total amount of energy in an interaction is constant.

  Expressed as an equation, this is recognized as
  \begin{equation}\label{eq:Law_Conservation_Energy}
    \sum_{i \in I} E_{i} = C
  \end{equation}
  where
  \begin{description}[noitemsep]
  \item $I$: The set of all interactions of interest.
  \item $E$: The energy of this particular interaction, $i$.
  \item $C$: The total energy of the system, a constant.
  \end{description}
\end{definition}

The \nameref{def:Law_Conservation_Energy} leads to \Cref{eq:Energy_Change}, shown below.
\begin{equation}\label{eq:Energy_Change}
  E_{in} - E_{out} = \Delta E
\end{equation}

In a typical use, we are concerned with \nameref{def:Heat} as the form of \nameref{def:Energy}.
\begin{definition}[Heat]\label{def:Heat}
  \emph{Heat} is the form of \nameref{def:Energy} that can be transferred from one system to another as a result of temperature difference.
\end{definition}

\begin{definition}[Thermodynamics]\label{def:Thermodynamics}
  \emph{Thermodynamics} is the science of equilibrium states and changes between these states.
\end{definition}


%%% Local Variables:
%%% mode: latex
%%% TeX-master: "../MMAE_320-Thermo-Reference_Sheet"
%%% End:
