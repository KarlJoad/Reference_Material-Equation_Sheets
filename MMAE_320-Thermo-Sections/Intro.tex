\section{Introduction}\label{sec:Introduction}
\begin{definition}[Energy]\label{def:Energy}
  \emph{Energy} is the stuff in the universe that can cause changes in material states.
  A change in energy is the ability to do \nameref{def:Work}.
  Energy can never be negative, but changes in energy can.

  Some examples are:
  \begin{itemize}[noitemsep]
  \item Thermal Energy
  \item Electrical Energy
  \item Mechanical Energy
  \end{itemize}
\end{definition}

One fundamental thing about \nameref{def:Energy} that was discovered was the \nameref{def:Law_Conservation_Energy}.
\begin{definition}[Law of Conservation of Energy]\label{def:Law_Conservation_Energy}
  The \emph{law of conservation of energy} states that energy cannot be created or destroyed; it can only change form.
  This means that the total amount of energy in an interaction is constant.

  Expressed as an equation, this is recognized as
  \begin{equation}\label{eq:Law_Conservation_Energy}
    \sum_{i \in I} E_{i} = C
  \end{equation}
  where
  \begin{description}[noitemsep]
  \item $I$: The set of all interactions of interest.
  \item $E$: The energy of this particular interaction, $i$.
  \item $C$: The total energy of the system, a constant.
  \end{description}
\end{definition}

\nameref{def:Law_Conservation_Energy} closely mirrors the other major fundamental physical law, the \nameref{def:Law_Conservation_Mass}.

\begin{definition}[Law of Conservation of Mass]\label{def:Law_Conservation_Mass}
  The \emph{law of conservation of mass} states that a \nameref{def:Process} cannot create nor destroy any matter; it can only change form/state.
\end{definition}

The \nameref{def:Law_Conservation_Energy} leads to \Cref{eq:Energy_Change}, shown below.
\begin{equation}\label{eq:Energy_Change}
  E_{in} - E_{out} = \Delta E
\end{equation}

In a typical use, we are concerned with \nameref{def:Heat} as the form of \nameref{def:Energy}.
\begin{definition}[Heat]\label{def:Heat}
  \emph{Heat} is the form of \nameref{def:Energy} that can be transferred from one \nameref{def:System} to another as a result of temperature difference.
  The variable letter for the energy in a system is $\Heat$.
\end{definition}

\begin{definition}[Thermodynamics]\label{def:Thermodynamics}
  \emph{Thermodynamics} is the science of equilibrium states and changes between these states.
\end{definition}

Pure thermodynamic analysis will show the toal amount of \nameref{def:Energy} moving through the system, but not the rate at which it occurs.
This is where \nameref{def:Heat_Transfer} comes in.

\begin{definition}[Heat Transfer]\label{def:Heat_Transfer}
  \emph{Heat transfer} is the science that deals with the determination of the \textit{rates} of \nameref{def:Heat}-based \nameref{def:Energy} transfers.
  This science deals with systems that lack a thermal equilibrium, meaning it cannot be based on only principles of \nameref{def:Thermodynamics}.
\end{definition}

For every flow of energy in a system, there \textbf{MUST} be a \nameref{def:Driving_Force}.
\begin{definition}[Driving Force]\label{def:Driving_Force}
  \emph{Driving force} are the conditions of the system for the transfer of \nameref{def:Energy} to occur.
  There are any number of driving forces in the universe, some are listed here:
  \begin{itemize}[noitemsep]
  \item \nameref{def:Heat_Transfer} requires a temperature difference.
  \item Electric current flow requires a voltage difference.
  \item Fluid flow requires a pressure difference.
  \end{itemize}
\end{definition}

For \nameref{def:Heat_Transfer}, the \nameref{def:Temperature_Gradient} determines the rate ofthe transfer.

\begin{definition}[Temperature Gradient]\label{def:Temperature_Gradient}
  \emph{Temperature gradient} is the temperature difference per unit length, or the rate of change of temperature.

  \begin{remark}[Extension]
    The use of gradients extends to other \nameref{def:Driving_Force}s as well.
  \end{remark}
\end{definition}

\subsection{Fluid Mechanics}\label{subsec:Fluid_Mechanics}
\begin{definition}[Stress]\label{def:Stress}
  \emph{Stress} is defined as force per unit area.
  \begin{equation}\label{eq:Stress}
    \tau = \frac{F}{A}
  \end{equation}
\end{definition}

Obeying Newton's third law, typically, there is a normal force per unit area, called the \textbf{normal stress}.
In a fluid at rest, the normal stress is called \textbf{pressure}.

\subsection{Units/Dimensions}\label{subsec:Units_Dimensions}
\subsubsection{SI System}\label{subsubsec:SI_System}
There are 7 fundamental dimensions and units in the SI system, shown in \Cref{tab:7_Fundamental_Dimensions}.
\begin{table}[h!tbp]
  \centering
  \begin{tabular}{ll}
    \toprule
    Dimension & Unit \\
    \midrule
    Length & meter (\si{\meter{}}) \\
    Mass & kilogram (\si{\kilo{} \gram{}}) \\
    Time & second (\si{\second{}}) \\
    Temperature & kelvin (\si{\kelvin{}}) \\
    Electric Current & ampere (\si{\ampere{}}) \\
    Amount of Light & candela (\si{\candela{}}) \\
    Amount of Matter & mole (\si{\mole{}}) \\
    \bottomrule
  \end{tabular}
  \caption{The 7 Fundamental Dimensions and Units}
  \label{tab:7_Fundamental_Dimensions}
\end{table}

\subsubsection{English System}\label{subsubsec:English_System}
The English system uses a very different set of units to describe these dimensions.

\paragraph{Mass}\label{par:English_Mass}
The English unit of mass is the pound-mass.
\begin{equation*}
  \si{\lbm}
\end{equation*}

\paragraph{Force}\label{par:English_Force}
The English unit of mass is the pound-force.
\begin{equation*}
  \SI{1}{\lbf} = \SI[per-mode=symbol]{32.174}{\lbm\feet\per\second\squared}
\end{equation*}

\paragraph{Energy}\label{par:English_Energy}
The English unit of energy is the British Thermal Unit (BTU).
\SI{1}{\btu} raises the temperature of \SI{1}{\lbm} of water at \SI{68}{\degreeF} by \SI{1}{\degreeF}.

\paragraph{Work}\label{par:English_Work}
\begin{definition}[Work]\label{def:Work}
  \emph{Work} is defined to be force multiplied by the path distance the force was applied over.
  \begin{equation}\label{eq:Work}
    W = \vec{F} d
  \end{equation}
\end{definition}

The English unit of \nameref{def:Work} (energy per time) is also the watt (\si{\watt}).

\paragraph{Power}\label{par:English_Power}
\begin{definition}[Power]\label{def:Power}
  \emph{Power} is defined as energy per unit time.
  \begin{equation}\label{eq:Power}
    \Power = \frac{\Energy}{t}
  \end{equation}
\end{definition}

The unit for power is typically the horsepower.

%%% Local Variables:
%%% mode: latex
%%% TeX-master: "../MMAE_320-Thermo-Reference_Sheet"
%%% End:
