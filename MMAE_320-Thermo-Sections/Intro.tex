\section{Introduction}\label{sec:Introduction}
\begin{definition}[Energy]\label{def:Energy}
  \emph{Energy} is the stuff in the universe that can cause changes in material states.
  Some examples are:
  \begin{itemize}[noitemsep]
  \item Thermal Energy
  \item Electrical Energy
  \item Mechanical Energy
  \end{itemize}
\end{definition}

One fundamental thing about \nameref{def:Energy} that was discovered was the \nameref{def:Law_Conservation_Energy}.
\begin{definition}[Law of Conservation of Energy]\label{def:Law_Conservation_Energy}
  The \emph{law of conservation of energy} states that energy cannot be created or destroyed; it can only change form.
  This means that the total amount of energy in an interaction is constant.

  Expressed as an equation, this is recognized as
  \begin{equation}\label{eq:Law_Conservation_Energy}
    \sum_{i \in I} E_{i} = C
  \end{equation}
  where
  \begin{description}[noitemsep]
  \item $I$: The set of all interactions of interest.
  \item $E$: The energy of this particular interaction, $i$.
  \item $C$: The total energy of the system, a constant.
  \end{description}
\end{definition}

The \nameref{def:Law_Conservation_Energy} leads to \Cref{eq:Energy_Change}, shown below.
\begin{equation}\label{eq:Energy_Change}
  E_{in} - E_{out} = \Delta E
\end{equation}

In a typical use, we are concerned with \nameref{def:Heat} as the form of \nameref{def:Energy}.
\begin{definition}[Heat]\label{def:Heat}
  \emph{Heat} is the form of \nameref{def:Energy} that can be transferred from one system to another as a result of temperature difference.
\end{definition}

\begin{definition}[Thermodynamics]\label{def:Thermodynamics}
  \emph{Thermodynamics} is the science of equilibrium states and changes between these states.
\end{definition}

Pure thermodynamic analysis will show the toal amount of \nameref{def:Energy} moving through the system, but not the rate at which it occurs.
This is where \nameref{def:Heat_Transfer} comes in.

\begin{definition}[Heat Transfer]\label{def:Heat_Transfer}
  \emph{Heat transfer} is the science that deals with the determination of the \textit{rates} of \nameref{def:Heat}-based \nameref{def:Energy} transfers.
  This science deals with systems that lack a thermal equilibrium, meaning it cannot be based on only principles of \nameref{def:Thermodynamics}.
\end{definition}

For every flow of energy in a system, there \textbf{MUST} be a \nameref{def:Driving_Force}.
\begin{definition}[Driving Force]\label{def:Driving_Force}
  \emph{Driving force} are the conditions of the system for the transfer of \nameref{def:Energy} to occur.
  There are any number of driving forces in the universe, some are listed here:
  \begin{itemize}[noitemsep]
  \item \nameref{def:Heat_Transfer} requires a temperature difference.
  \item Electric current flow requires a voltage difference.
  \item Fluid flow requires a pressure difference.
  \end{itemize}
\end{definition}

For \nameref{def:Heat_Transfer}, the \nameref{def:Temperature_Gradient} determines the rate ofthe transfer.

\begin{definition}[Temperature Gradient]\label{def:Temperature_Gradient}
  \emph{Temperature gradient} is the temperature difference per unit length, or the rate of change of temperature.

  \begin{remark}[Extension]
    The use of gradients extends to other \nameref{def:Driving_Force}s as well.
  \end{remark}
\end{definition}


%%% Local Variables:
%%% mode: latex
%%% TeX-master: "../MMAE_320-Thermo-Reference_Sheet"
%%% End:
