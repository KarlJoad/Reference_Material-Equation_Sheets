\section{Mixtures}\label{sec:Mixtures}
In the real world, we come across \nameref{def:Mixture}s all the time.
They do not have the same properties as their \nameref{def:Pure_Substance} counterparts.

\begin{definition}[Mixture]\label{def:Mixture}
  A \emph{mixture} is a substance or set of substances that contain at least more than one different \nameref{def:Saturated_Liquid} and \nameref{def:Saturated_Vapor} at a time.
\end{definition}

When a \nameref{def:Saturated_Liquid} is boiling to become a \nameref{def:Saturated_Vapor}, (Points $A$ and $B$ in \Cref{fig:Phase_Change}), all of the liquid \textbf{must} turn to gas first.
This \nameref{def:Process} has a \nameref{def:Quality}.

\begin{definition}[Quality]\label{def:Quality}
  \emph{Quality} is the ratio of a \nameref{def:Mixture} that is a \nameref{def:Saturated_Vapor} to \nameref{def:Saturated_Liquid}.

  \begin{equation}\label{eq:Quality}
    \begin{aligned}
      \text{Quality} &\equiv \frac{\text{Mass of \nameref{def:Saturated_Vapor}}}{\text{Total mass of \nameref{def:Mixture}}} \\
      \Quality &= \frac{\Mass_{Vapor}}{\Mass_{Liquid} + \Mass_{Vapor}}
    \end{aligned}
  \end{equation}

  \Cref{eq:Quality} is always valued $0 \leq \Quality \leq 1$.
  This is like \nameref{subsec:Energy_Efficiency}, $\Efficiency$.
\end{definition}


%%% Local Variables:
%%% mode: latex
%%% TeX-master: "../MMAE_320-Thermo-Reference_Sheet"
%%% End:
