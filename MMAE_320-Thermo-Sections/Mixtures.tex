\section{Mixtures}\label{sec:Mixtures}
In the real world, we come across \nameref{def:Mixture}s all the time.
They do not have the same properties as their \nameref{def:Pure_Substance} counterparts.

\begin{definition}[Mixture]\label{def:Mixture}
  A \emph{mixture} is a substance or set of substances that contain at least more than one different \nameref{def:Saturated_Liquid} and \nameref{def:Saturated_Vapor} at a time.
\end{definition}

When a \nameref{def:Saturated_Liquid} is boiling to become a \nameref{def:Saturated_Vapor}, (Points $A$ and $B$ in \Cref{fig:Phase_Change}), all of the liquid \textbf{must} turn to gas first.
This \nameref{def:Process} has a \nameref{def:Quality}.

\begin{definition}[Quality]\label{def:Quality}
  \emph{Quality} is the ratio of a \nameref{def:Mixture} that is a \nameref{def:Saturated_Vapor} to \nameref{def:Saturated_Liquid}.

  \begin{equation}\label{eq:Quality}
    \begin{aligned}
      \text{Quality} &\equiv \frac{\text{Mass of \nameref{def:Saturated_Vapor}}}{\text{Total mass of \nameref{def:Mixture}}} \\
      \Quality &= \frac{\Mass_{Vapor}}{\Mass_{Liquid} + \Mass_{Vapor}}
    \end{aligned}
  \end{equation}

  \Cref{eq:Quality} is always valued $0 \leq \Quality \leq 1$.
  This is like \nameref{subsec:Energy_Efficiency}, $\Efficiency$.
\end{definition}

\begin{definition}[Average Specific Volume]\label{def:Average_Specific_Volume}
  \emph{Average specfic volume} defines the total volume of a system when it is existing in 2 different phases.
  If we are interested in the average specific volume of the \nameref{def:Mixture}, we have \Cref{eq:Average_Specific_Volume}.
  \begin{equation}\label{eq:Average_Specific_Volume}
    \AvgSpecificVol = \frac{\SaturatedVaporVol}{\SaturatedFluidVol}
  \end{equation}

  \begin{description}[noitemsep]
  \item $\AvgSpecificVol$: The Average Specific Volume at any point in time during a \nameref{def:Process}.
  \item $\SaturatedVaporVol$: The \nameref{def:Saturated_Liquid} volume.
  \item $\SaturatedFluidVol$: The \nameref{def:Saturated_Vapor} volume.
  \end{description}

  \begin{equation}\label{eq:Average_Specific_Volume_Using_Quality}
    \begin{aligned}
      \Quality &= \frac{\Volume_{Gas}}{\Volume_{Total}} \\
      \AvgSpecificVol &= (1 - \Quality) \SaturatedFluidVol + \Quality \SaturatedVaporVol \\
      &= \SaturatedFluidVol + \Quality (\SaturatedVaporVol - \SaturatedFluidVol)
    \end{aligned}
  \end{equation}
\end{definition}

From \Cref{eq:Average_Specific_Volume}, it is clear that to get the total volume of \textbf{one phase of the system}, we use \Cref{eq:Single_Phase_Average_Total_Volume}
\begin{equation}\label{eq:Single_Phase_Average_Total_Volume}
  \AvgSpecificVol \Mass_{Phase} = \Volume_{Phase}
\end{equation}

Now, \Cref{eq:Single_Phase_Average_Total_Volume} only finds the volume for a single phase in the system, but because volume is additive, we can use \Cref{eq:Average_Total_Volume}.
\begin{equation}\label{eq:Average_Total_Volume}
  \AvgSpecificVol \Mass_{Total} = \Mass_{Liquid} \SaturatedFluidVol + \Mass_{Gas} \SaturatedVaporVol
\end{equation}

In fact, \Cref{eq:Average_Total_Volume} can be re-expressed using \nameref{def:Quality}, which gives us \Cref{eq:Average_Specific_Volume_Using_Quality}.
\begin{align*}
  \AvgSpecificVol \Mass_{Total} &= \Mass_{Liquid} \SaturatedFluidVol + \Mass_{Gas} \SaturatedVaporVol \\
  \AvgSpecificVol \Mass_{Total} &= (\Mass_{Total} - \Mass_{Gas}) \SaturatedFluidVol + \Mass_{Gas} \SaturatedVaporVol \\
  \AvgSpecificVol &= \frac{1}{\Mass_{Total}} \bigl( (\Mass_{Total} - \Mass_{Gas}) \SaturatedFluidVol + \Mass_{Gas} \SaturatedVaporVol \bigr) \\
                                &= \left( 1 - \frac{\Mass_{Gas}}{\Mass_{Total}} \right) \SaturatedFluidVol + \frac{\Mass_{Gas}}{\Mass_{Total}} \SaturatedVaporVol \\
  \Quality &= \frac{\Mass_{Gas}}{\Mass_{Total}} \\
  \AvgSpecificVol &= (1 - \Quality) \SaturatedFluidVol + \Quality \SaturatedVaporVol \\
  &= \SaturatedFluidVol + \Quality (\SaturatedVaporVol - \SaturatedFluidVol)
\end{align*}

The quantities $\SaturatedFluidVol$ and $\SaturatedVaporVol$ \textbf{are fixed} once pressure and temperature have been established.

\begin{example}{Change in Mixture Properties}
  Suppose there is a pot of water with $\Mass = \SI{3}{\kilo\gram}$ of water at $\Pressure = \SI{101}{\kilo\pascal}$, and temperature $\Temp = \SI{100}{\degreeCelsius}$.
  The pot is sitting at $\Temp$ to start with and continues to heat until all water is evaporated.
  \begin{itemize}[noitemsep]
  \item What is the volume of the water at the start?
  \item What is the \nameref{def:Quality} of the \nameref{def:Mixture} at the start?
  \item If $\Quality=\frac{1}{2}$, what is the volume of the resulting system?
  \item If $\Quality=1$, what is the volume of the resulting system?
  \end{itemize}
  \tcblower{}
  Using the equations we have already, we know
  \begin{equation*}
    \Volume = \SaturatedFluidVol \Mass_{Total}
  \end{equation*}
  From Table A.4 in the textbook, we know $\SaturatedFluidVol = \SI{0.001043}{\meter\cubed\per\kilo\gram}$.
  Thus, we can just plug that value into the equation and solve for $\Volume$.
  \begin{align*}
    \Volume &= \SaturatedFluidVol \Mass_{Total} \\
    \intertext{At this pressure and temperature, the liquid water is not becoming a vapor due to boiling, so $\Mass_{Total} = \Mass_{Liquid}$}
            &= \SI{0.001043}{\meter\cubed\per\kilo\gram} (\SI{3}{\kilo\gram}) \\
            &= \SI{0.003129}{\meter\cubed} \\
            &= \SI{3.129}{\liter}
  \end{align*}

  Now to solve for the ``initial'' \nameref{def:Quality}.
  Because the water has \textbf{\textit{JUST}} reached its \nameref{def:Saturated_Liquid} state, the water has \textbf{begun} boiling, but not yet started to vaporize.
  Thus,
  \begin{equation*}
    \Quality = 0
  \end{equation*}

  We are given the \nameref{def:Quality} of the system, $\Quality=\frac{1}{2}$, meaning we have converted half the \nameref{def:Saturated_Liquid} to vapor.
  We can easily find the mass of the vapor, then find its volume.
  \begin{align*}
    \Quality &= \frac{1}{2} \\
    \intertext{Remember, due to the Law of Conservation of Mass, \textbf{all} \SI{3}{\kilo\gram} of the water remains \textbf{IN} the system!}
      &= \frac{\Mass_{Gas}}{\Mass_{Total}} = \frac{\Mass_{Gas}}{\SI{3}{\kilo\gram}} \\
    \frac{1}{2} &= \frac{\Mass_{Gas}}{\SI{3}{\kilo\gram}} \\
    \Mass_{Gas} &= \SI{1.5}{\kilo\gram} \\
  \end{align*}

  Now that we know the mass of the vaporized liquid, we can use \Cref{eq:Average_Specific_Volume_Using_Quality} to solve for the total volume.
  \begin{align*}
    \AvgSpecificVol &= \SaturatedFluidVol + \Quality(\SaturatedVaporVol - \SaturatedFluidVol) \\
    \intertext{The value for $\SaturatedFluidVol$ remains constant throughout this entire \nameref{def:Process}, so we can reuse that value. The value for $\SaturatedVaporVol$ is found in Table A.4 from the textbook.}
                    &= \SI{0.001043}{\meter\cubed\per\kilo\gram} + \frac{1}{2} (\SI{1.6720}{\meter\cubed\per\kilo\gram} - \SI{0.001043}{\meter\cubed\per\kilo\gram}) \\
                    &= \SI{0.8365}{\meter\cubed\per\kilo\gram}
  \end{align*}

  Now that we have the \nameref{def:Average_Specific_Volume} of the system, we can find the total volume of the system.
  \begin{align*}
    \Volume_{Total} &= \AvgSpecificVol \Mass_{Total} \\
                    &= \SI{0.8365}{\meter\cubed\per\kilo\gram} (\SI{3}{\kilo\gram}) \\
                    &= \SI{2509}{\liter}
  \end{align*}

  Remember that $\Volume_{Total}$ contains the volume for \textbf{both} the steam \textbf{and} the water.
  If we are curious about the separate values for each, then we can solve for the individual terms in \Cref{eq:Single_Phase_Average_Total_Volume}.
  \begin{align*}
    \Volume_{Water} &= \Mass_{Water} \SaturatedFluidVol \\
                    &= \SI{1.5}{\kilo\gram} (\SI{0.001043}{\meter\cubed\per\kilo\gram}) \\
                    &= \SI{1.56}{\liter} \\
    \Volume_{Steam} &= \Mass_{Steam} \SaturatedVaporVol \\
                    &= \SI{1.5}{\kilo\gram} (\SI{1.6720}{\meter\cubed\per\kilo\gram}) \\
                    &= \SI{2508}{\liter}
  \end{align*}

  Thus,
  \begin{align*}
    \Volume_{Water} &=  \SI{1.56}{\liter} \\
    \Volume_{Steam} &= \SI{2508}{\liter}
  \end{align*}

  For this last part, $\Quality = 1$, meaning that \textbf{all} the water has become steam.
  Thus, $\Mass_{Water} = \SI{0}{\kilo\gram}$ and $\Mass_{Steam} = \SI{3}{\kilo\gram}$.
  Because we also know $\SaturatedVaporVol = \SI{1.6720}{\meter\cubed\per\kilo\gram}$, we can easily just solve for \Cref{eq:Single_Phase_Average_Total_Volume} as the total volume.
  \begin{align*}
    \Volume_{Total} &= \Volume_{Steam} \\
    \Volume_{Steam} &= \SaturatedVaporVol \Mass_{Steam} \\
                    &= (\SI{1.6720}{\meter\cubed\per\kilo\gram}) (\SI{3}{\kilo\gram}) \\
                    &= \SI{5016}{\liter}
  \end{align*}

  So, after the water has completely become steam, the total volume of the system is $\Volume_{Total} = \SI{5016}{\liter}$.
\end{example}

%%% Local Variables:
%%% mode: latex
%%% TeX-master: "../MMAE_320-Thermo-Reference_Sheet"
%%% End:
