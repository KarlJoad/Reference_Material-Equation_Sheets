\section{Mixtures}\label{sec:Mixtures}
In the real world, we come across \nameref{def:Mixture}s all the time.
They do not have the same properties as their \nameref{def:Pure_Substance} counterparts.

\begin{definition}[Mixture]\label{def:Mixture}
  A \emph{mixture} is a substance or set of substances that contain at least more than one different \nameref{def:Saturated_Liquid} and \nameref{def:Saturated_Vapor} at a time.
\end{definition}

When a \nameref{def:Saturated_Liquid} is boiling to become a \nameref{def:Saturated_Vapor}, (Points $A$ and $B$ in \Cref{fig:Phase_Change}), all of the liquid \textbf{must} turn to gas first.
This \nameref{def:Process} has a \nameref{def:Quality}.

\begin{definition}[Quality]\label{def:Quality}
  \emph{Quality} is the ratio of a \nameref{def:Mixture} that is a \nameref{def:Saturated_Vapor} to \nameref{def:Saturated_Liquid}.

  \begin{equation}\label{eq:Quality}
    \begin{aligned}
      \text{Quality} &\equiv \frac{\text{Mass of \nameref{def:Saturated_Vapor}}}{\text{Total mass of \nameref{def:Mixture}}} \\
      \Quality &= \frac{\Mass_{Vapor}}{\Mass_{Liquid} + \Mass_{Vapor}}
    \end{aligned}
  \end{equation}

  \Cref{eq:Quality} is always valued $0 \leq \Quality \leq 1$.
  This is like \nameref{subsec:Energy_Efficiency}, $\Efficiency$.
\end{definition}

\begin{definition}[Average Specific Volume]\label{def:Average_Specific_Volume}
  \emph{Average specfic volume} defines the total volume of a system when it is existing in 2 different phases.
  If we are interested in the average specific volume of the \nameref{def:Mixture}, we have \Cref{eq:Average_Specific_Volume}.
  \begin{equation}\label{eq:Average_Specific_Volume}
    \AvgSpecificVol = \frac{\SaturatedVaporVol}{\SaturatedFluidVol}
  \end{equation}

  \begin{description}[noitemsep]
  \item $\AvgSpecificVol$: The Average Specific Volume at any point in time during a \nameref{def:Process}.
  \item $\SaturatedVaporVol$: The \nameref{def:Saturated_Liquid} volume.
  \item $\SaturatedFluidVol$: The \nameref{def:Saturated_Vapor} volume.
  \end{description}

  \begin{equation}\label{eq:Average_Specific_Volume_Using_Quality}
    \begin{aligned}
      \Quality &= \frac{\Volume_{Gas}}{\Volume_{Total}} \\
      \AvgSpecificVol &= (1 - \Quality) \SaturatedFluidVol + \Quality \SaturatedVaporVol \\
      &= \SaturatedFluidVol + \Quality (\SaturatedVaporVol - \SaturatedFluidVol)
    \end{aligned}
  \end{equation}
\end{definition}

From \Cref{eq:Average_Specific_Volume}, it is clear that to get the total volume of \textbf{one phase of the system}, we use \Cref{eq:Single_Phase_Average_Total_Volume}
\begin{equation}\label{eq:Single_Phase_Average_Total_Volume}
  \AvgSpecificVol \Mass_{Phase} = \Volume_{Phase}
\end{equation}

Now, \Cref{eq:Single_Phase_Average_Total_Volume} only finds the volume for a single phase in the system, but because volume is additive, we can use \Cref{eq:Average_Total_Volume}.
\begin{equation}\label{eq:Average_Total_Volume}
  \AvgSpecificVol \Mass_{Total} = \Mass_{Liquid} \SaturatedFluidVol + \Mass_{Gas} \SaturatedVaporVol
\end{equation}


%%% Local Variables:
%%% mode: latex
%%% TeX-master: "../MMAE_320-Thermo-Reference_Sheet"
%%% End:
