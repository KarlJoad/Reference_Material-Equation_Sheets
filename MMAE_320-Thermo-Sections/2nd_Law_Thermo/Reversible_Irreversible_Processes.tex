\subsection{Reversible and Irreversible Processes}\label{subsec:Reversible_Irreversible_Processes}
\begin{definition}[Reversible Process]\label{def:Reversible_Process}
  A \emph{reversible process} is a thermodynamic \nameref{def:Process} that can reversed without affecting the system surroundings.
  This means the system is always in equilibrium.
\end{definition}

\begin{definition}[Irreversible Process]\label{def:Irreversible_Process}
  An \emph{irreversible process} is a thermodynamic \nameref{def:Process} where the state of the \nameref{def:System} or surroundings has been altered in a non-reversible way.
  This means that whatever the process is, some part of it is lost and cannot be recovered.
\end{definition}

Some things that make a \nameref{def:Process} irreversible are:
\begin{itemize}[noitemsep]
\item Friction
\item \nameref{def:Heat_Transfer}
\end{itemize}

\begin{blackbox}
  \nameref{def:Irreversible_Process}es are what we encounter in the real world.
  \nameref{def:Reversible_Process}es are idealized situations that we use to simplify \nameref{def:System}.
\end{blackbox}

Since most everyday \nameref{def:Process}es are \textbf{not} \nameref{def:Reversible_Process}es, then the ideal process is the \nameref{def:Reversible_Process} and reflects the \textbf{mot} efficient process possible.
This means that $\Efficiency = 1 - \frac{\Heat_{Out}}{\Heat_{In}} < 1$ for all realistic \nameref{def:Process}es.

%%% Local Variables:
%%% mode: latex
%%% TeX-master: "../../MMAE_320-Thermo-Reference_Sheet"
%%% End:
