\subsection{Isentropic Efficiencies}\label{subsec:Isentropic_Efficiencies}
Consider a \nameref{fig:Heat_Engine}, like the one shown in \Cref{fig:Heat_Engine}.
\begin{itemize}[noitemsep]
\item If each sub-\nameref{def:Process} is \nameref{def:Isentropic}, then on a $\Temp-\Entropy$ graph, then the process is perfectly vertical.
\item If any part is even slightly irreversible, then the ending entropy is further to the right compared ot any previous part of the \nameref{def:Cycle}.
\end{itemize}

If we are talking in a non-ideal case, then we can say:
\begin{equation*}
  \Efficiency = \frac{\Work_{Out}}{\Heat_{In}}
\end{equation*}

\subsubsection{Turbine}\label{subsubsec:Turbine_Isentropic_Efficiency}
If we deal with the actual work out and ideal heat in, we say the \nameref{def:Isentropic} efficiency is:
\begin{align*}
  \Isentropic{\Efficiency} &= \frac{\Work_{Out, \text{Actual}}}{\Heat_{In, \text{Ideal}}} \\
                           &= \frac{\FlowRate{\Mass}(\SpecificEnthalpy_{In} - \SpecificEnthalpy_{Out, \text{Actual}})}{\FlowRate{\Mass} (\SpecificEnthalpy_{In} - \SpecificEnthalpy_{Out, \text{Ideal}})} \\
                           &= \frac{\SpecificEnthalpy_{In} - \SpecificEnthalpy_{Out, \text{Actual}}}{\SpecificEnthalpy_{In} - \SpecificEnthalpy_{Out, \text{Ideal}}}
\end{align*}

For a \nameref{def:Turbine},
\begin{equation}\label{eq:Isentropic_Efficiency-Turbine}
  \Isentropic{\Efficiency} = \frac{\SpecificEnthalpy_{In} - \SpecificEnthalpy_{Out, \text{Actual}}}{\SpecificEnthalpy_{In} - \SpecificEnthalpy_{Out, \text{Ideal}}}
\end{equation}

\begin{example}{Turbine Entropy}
  Given an \nameref{def:Adiabatic} \nameref{def:Turbine} that uses steam where the inlet steam is at $\Pressure_{In} = \SI{1000}{\psia}$ and $\Temp_{In} = \SI{800}{\degreeF}$.
  The outlet is $\Pressure_{Out} = \SI{400}{\psia}$.
  Find $\frac{\FlowRate{\Work_{Out}}}{\FlowRate{\Mass}}$?
  \tcblower{}
  \textbf{Concepts:} \\
  This is a \nameref{def:Turbine}, so it has constant flow.
  Meaning $\FlowRate{\Mass}_{In} = \FlowRate{\Mass}_{Out}$. \\
  This is an \nameref{def:Adiabatic} \nameref{def:Turbine}, which means the \nameref{def:Process} is a \nameref{def:Reversible_Process}, and is an \nameref{def:Isentropic} process.
  Meaning $\SpecificEntropy_{In} = \SpecificEntropy_{Out}$. \\
  Lastly, energy is conserved, so $\FlowRate{\Energy}_{In} = \FlowRate{\Energy}_{Out}$.

  \textbf{Explore:} \\
  The change in energy is due to the \nameref{def:Enthalpy} changes in the steam. \\
  The \nameref{def:Specific_Entropy} remains unchanged throughout the process, so we can use these values from Tables. \\
  Tables A.4E --- A.7E will help us solve this.

  \textbf{Plan:} \\
  Find the steam's initial state properties, $\SpecificEnthalpy_{In}$ and $\SpecificEntropy_{In}$. \\
  Use $\SpecificEntropy_{In} = \SpecificEntropy_{Out}$ to help find $\SpecificEnthalpy_{Out}$. \\
  Use the energy balance equation for a \nameref{def:Turbine} $\FlowRate{\Work}_{Out} = \FlowRate{\Mass}(\SpecificEnthalpy_{Out} - \SpecificEnthalpy_{In})$ and solve for $\frac{\FlowRate{\Work_{Out}}}{\FlowRate{\Mass}}$.

  \textbf{Solve:} \\
  The steam enters the \nameref{def:Turbine} as a \nameref{def:Superheated_Vapor}, so we use Table A.6E.
  From Table A.6E at $\Pressure_{In} = \SI{1000}{\psia}$ and $\Temp_{In} = \SI{800}{\degreeF}$, we see:
  \begin{description}[noitemsep]
  \item $\SpecificEnthalpy_{In} = \SI{1389.0}{\btu\per\lbm}$
  \item $\SpecificEntropy_{In} = \SI{1.5670}{\btu\per\lbm\per\rankine}$
  \end{description}

  Now, for the outlet state.
  Start by looking at Table A.5E at $\Pressure_{Out} = \SI{400}{\psia}$.
  If we look at the \nameref{def:Specific_Entropy} entries, we see that the highest value is when the steam is a \nameref{def:Saturated_Vapor}, with $\Vapor{\SpecificEntropy} = \SI{1.4852}{\btu\per\lbm\per\rankine}$.
  However, because $\SpecificEntropy_{In} = \SpecificEntropy_{Out}$, we see that this is not enough.
  So, that means the steam is still a \nameref{def:Superheated_Vapor} at the outlet. \\
  From Table A.6E, at $\Pressure_{Out} = \SI{400}{\psia}$, we look for $\SpecificEntropy_{Out} = \SI{1.5670}{\btu\per\lbm\per\rankine}$.
  The closest value to that $\SpecificEntropy$ is at $\Temp_{Out} = \SI{550}{\degreeF}$, with $\SpecificEnthalpy_{Out} = \SI{1277.3}{\btu\per\lbm}$.
  (In reality, we should \nameref{def:Interpolate} our values, but this is close enough for this example.)

  Now, solving the energy balance equation for $\frac{\FlowRate{\Work}_{Out}}{\FlowRate{\Mass}}$.
  \begin{align*}
    - \FlowRate{\Work}_{Out} &= \FlowRate{\Mass} (\SpecificEnthalpy_{Out} - \SpecificEnthalpy_{In}) \\
    \frac{-\FlowRate{\Work}_{Out}}{\FlowRate{\Mass}} &= \SpecificEnthalpy_{Out} - \SpecificEnthalpy_{In} \\
    \frac{\FlowRate{\Work}_{Out}}{\FlowRate{\Mass}} &= \SpecificEnthalpy_{In} - \SpecificEnthalpy_{Out} \\
                           &= \SI{1389.0}{\btu\per\lbm} - \SI{1277.3}{\btu\per\lbm} \\
    &= \SI{111.7}{\btu\per\lbm}
  \end{align*}

  \textbf{Generalize:} \\
  We can use \nameref{def:Specific_Entropy} to find the state of the working fluid, just like the \nameref{def:Specific_Enthalpy}, temperature, or pressure.
\end{example}

\begin{example}[Textbook Example 8.14]{Isentropic Efficiency of a Steam Turbine}
  Steam enters an \nameref{def:Adiabatic} \nameref{def:Turbine} steadily at $\Pressure_{In} = \SI{3}{\mega\pascal}$ and $\Temp_{In} = \SI{400}{\degreeCelsius}$ and leaves at $\Pressure_{Out} = \SI{50}{\kilo\pascal}$ and $\Temp_{Out} = \SI{100}{\degreeCelsius}$.
  If the power output of the turbine is $\FlowRate{\Work}_{Out}= \SI{2}{\mega\watt}$, determine the isentropic efficiency of the turbine and the mass flowrate of the steam flowing through the turbine?
  \tcblower{}
  \textbf{Concepts:} \\
  We are given the information for both the inlet and the outlet.
  These are the actual values of the turbine.
  If we want the idealized values, we need to interpret the inlet data a little bit. \\
  For a turbine be \nameref{def:Isentropic}, it needs to be \nameref{def:Adiabatic}, which we are told it is.
  This means that $\SpecificEntropy_{In} = \SpecificEntropy_{Out}$. \\
  The isentropic efficiency of a turbine is seen in \Cref{eq:Isentropic_Efficiency-Turbine}.

  \textbf{Explore:} \\
  Because this is steam, Tables A.4 through A.7 will be of use. \\
  The idealized values for the ideal case can only be found by treating this system as \nameref{def:Adiabatic} and ensuring that $\SpecificEntropy_{In} = \SpecificEntropy_{Out}$.

  \textbf{Plan:} \\
  Use the tables to find the actual inlet and outlet specific enthalpies. \\
  Use the idea that $\SpecificEntropy_{In} = \SpecificEntropy_{Out}$ to find the ideal outlet specific enthalpy. \\
  Solve \Cref{eq:Isentropic_Efficiency-Turbine} for $\Isentropic{\Efficiency}$.
\end{example}

\begin{example}[Textbook Problem 8.128]{Power Output of Isentropically Efficienct Steam Turbine}
  Steam at $\Pressure_{In} = \SI{3}{\mega\pascal}$ and $\Temp_{In} = \SI{400}{\degreeCelsius}$ is expanded to $\Pressure_{Out} = \SI{30}{\kilo\pascal}$ in an \nameref{def:Adiabatic} \nameref{def:Turbine} with an \nameref{def:Isentropic} efficiency of $\Isentropic{\Efficiency} = 0.92$.
  Determine the power produced by this turbine ($\FlowRate{\Work}_{Out}$) in \si{\kilo\watt} when the mass flowrate is $\FlowRate{\Mass} = \SI{2}{\kilo\gram\per\second}$?
  \tcblower{}
  \textbf{Concepts:} \\
  This is a \nameref{def:Turbine}, with \nameref{def:Adiabatic} steady flow.
  A turbine gets all its energy from the change in \nameref{def:Enthalpy}, $\Change{\SpecificEnthalpy}$. \\
  The equation for \nameref{def:Isentropic} efficiency is:
  \begin{equation*}
    \Isentropic{\Efficiency} = \frac{\text{Actual Work Out}}{\text{Isentropic Work Out}}
  \end{equation*}

  \textbf{Explore:} \\
  To solve for the work out, we need to know the maximum work out that we could gather.
  This only occurs if the \nameref{def:Turbine} is perfectly \nameref{def:Isentropic}, meaning $\Entropy_{In} = \Entropy_{Out}$.

  If we expand the isentropic efficiency equation, then we get:
  \begin{equation*}
    \Isentropic{\Efficiency} = \frac{\SpecificEnthalpy_{In} - \SpecificEnthalpy_{Out, \text{Actual}}}{\SpecificEnthalpy_{In} - \SpecificEnthalpy_{Out, \text{Ideal}}}
  \end{equation*}

  The energy balance for this turbine is:
  \begin{equation*}
    \FlowRate{\Work}_{Out, \text{Actual}} = \FlowRate{\Mass} (\SpecificEnthalpy_{In} - \SpecificEnthalpy_{Out})
  \end{equation*}

  This is a steam turbine, so Tables A.4 through A.7 will be of use.

  \textbf{Plan:} \\
  Find properties of the initial state, $\SpecificEnthalpy_{In}$, $\SpecificEntropy_{In}$. \\
  Use our assumption of isentropic to find $\SpecificEntropy_{Out, \text{Ideal}}$.

  \textbf{Solve:} \\
  We see that from the inlet fluid's pressure and temperature, it is a \nameref{def:Superheated_Vapor}, so we use Table A.6.
  From Table A.6 at $\Pressure_{In} = \SI{3}{\mega\pascal}$ and $\Temp_{In} = \SI{400}{\degreeCelsius}$, we have:
  \begin{align*}
    \SpecificEnthalpy_{In} &= \SI{3231}{\kilo\joule\per\kilo\gram} \\
    \SpecificEntropy_{In} &= \SI{6.923}{\kilo\joule\per\kilo\gram\per\kelvin}
  \end{align*}

  If we assume that the \nameref{def:Turbine} is perfectly \nameref{def:Isentropic}, then we notice that the steam has become a \nameref{def:Saturated_Mixture}, so we use Table A.5 to find its properties.
  \begin{align*}
    \Quality &= \frac{\SpecificEntropy_{In} - \Fluid{\SpecificEntropy}}{\Vapor{\SpecificEntropy} - \Fluid{\SpecificEntropy}} \\
             &= \frac{6.923 - 0.9441}{7.7675 - 0.9441} \\
    \Quality &= 0.8763
  \end{align*}

  Now, we find the ideal $\SpecificEnthalpy_{Out, \text{Ideal}}$.
  \begin{align*}
    \SpecificEnthalpy_{Out, \text{Ideal}} &= \Fluid{\SpecificEnthalpy} + \Quality (\Vapor{\SpecificEnthalpy} - \Fluid{\SpecificEnthalpy}) \\
                                          &= 289.27 + 0.8763 (2624.6 - 289.27) \\
                                          &= \SI{2335.72}{\kilo\joule\per\kilo\gram}
  \end{align*}

  So, the \nameref{def:Isentropic} work out is:
  \begin{align*}
    \FlowRate{\Work}_{Out, \text{Ideal}} &= \FlowRate{\Mass} (\SpecificEnthalpy_{In} - \SpecificEnthalpy_{Out, \text{Ideal}}) \\
                                         &= 2 (3231 - 2335.72) \\
                                         &= \SI{1790.56}{\kilo\joule\per\second}
  \end{align*}

  Now, we plug that into our isentropic efficiency.
  \begin{align*}
    \Isentropic{\Efficiency} &= \frac{\FlowRate{\Work}_{Out, \text{Actual}}}{\FlowRate{\Work}_{Out, \text{Ideal}}} \\
    \shortintertext{Rearranging the equation.}
    \FlowRate{\Work}_{Out, \text{Actual}} &= \Isentropic{\Efficiency} \FlowRate{\Work}_{Out, \text{Ideal}} \\
                             &= 0.92 (1790.56) \\
                             &= \SI{1647.32}{\kilo\joule\per\second}
  \end{align*}

  \textbf{Validate:} \\

  \textbf{Generalize:} \\

\end{example}

\subsubsection{Compressor}\label{subsubsec:Compressor_Isentropic_Efficiency}
For a \nameref{def:Compressor}, we have:
\begin{equation}\label{eq:Isentropic_Efficiency-Compressor}
  \begin{aligned}
    \Isentropic{\Efficiency} &= \frac{\text{\nameref{def:Isentropic} Work In}}{\text{Actual Work In}} \\
    &= \frac{\SpecificEnthalpy_{Out, \text{Ideal}} - \SpecificEnthalpy_{In}}{\SpecificEnthalpy_{Out, \text{Actual}} - \SpecificEnthalpy_{In}}
  \end{aligned}
\end{equation}

\begin{example}[Textbook Problem 8.24]{Isentropic Compressor}
  Air is compressed by a $\FlowRate{\Work}_{In} = \SI{12}{\kilo\watt}$ \nameref{def:Compressor}.
  The air temperature is maintained at $\Temp = \SI{25}{\degreeCelsius}$ during the process.
  Then, as a result of \nameref{def:Heat_Transfer}, the surrounding air, which is at $\Temp = \SI{10}{\degreeCelsius}$.
  Determine the rate of entropy change of the air?
  \tcblower{}
  \textbf{Concepts:} \\
  \nameref{def:Compressor}s typically increase the air's temperature.
  Because there is no temperature change of the air, that means the whole compressor is being cooled, thus a $\FlowRate{\Heat}_{Out} \neq 0$. \\
  We find the $\FlowRate{\Entropy}$, but we aren't given a mass flowrate, $\FlowRate{\Mass}$. \\
  We should also treat air as an ideal gas.

  \textbf{Explore:} \\
  We can ignore $\PotentialEnergy$ and $\KineticEnergy$.
  The energy balance for a compressor is:
  \begin{equation*}
    \FlowRate{\Work}_{In} - \FlowRate{\Heat}_{Out} = \FlowRate{\Mass} (\SpecificEnthalpy_{Out} - \SpecificEnthalpy_{In})
  \end{equation*}

  Inside the \nameref{def:Compressor}, it is \nameref{def:Isothermal}.
  Therefore, the \nameref{def:Process} \textbf{INSIDE} the compressor is a \nameref{def:Reversible_Process}, so it is \nameref{def:Isentropic}. \\
  Using the energy balance $\FlowRate{\Energy}_{In} = \FlowRate{\Energy}_{Out}$, we can say:
  \begin{align*}
    \FlowRate{\Energy}_{In} &= \FlowRate{\Energy}_{Out} \\
    \FlowRate{\Work}_{In} + \FlowRate{\Mass} \SpecificEnthalpy_{In} &= \FlowRate{\Heat}_{Out} + \FlowRate{\Mass} \SpecificEnthalpy_{Out} \\
    \intertext{Because $\SpecificEnthalpy \propto \Temp$, substitute for temperatures.
    We can do this because of the specific enthalpy-specific heat equations for ideal gases.}
    \FlowRate{\Work}_{In} + \FlowRate{\Mass} \Temp_{In} &= \FlowRate{\Heat}_{Out} + \FlowRate{\Mass} \Temp_{In} \\
    \intertext{And, the temperatures are the same, so they cancel out.}
    \FlowRate{\Work}_{In} &= \FlowRate{\Heat}_{Out}
  \end{align*}

  Lastly, we can find the \nameref{def:Entropy} flowrate.
  \begin{align*}
    \Entropy &= \frac{\Heat}{\Temp} \\
    \FlowRate{\Entropy} &= \frac{\FlowRate{\Heat}}{\Temp} \\
    \intertext{Because the heat is flowing outwards, there is actually a negative.}
    \FlowRate{\Entropy} &= \frac{-\FlowRate{\Heat}}{\Temp}
  \end{align*}

  \textbf{Plan:} \\
  Solve for the entropy flowrate, $\FlowRate{\Entropy}$.

  \textbf{Solve:} \\
  \begin{align*}
    \FlowRate{\Entropy} &= \frac{-\FlowRate{\Heat}}{\Temp} \\
                        &= \frac{\SI{-12}{\kilo\watt}}{(25 + 273.15)\si{\kelvin}} \\
                        &= \SI{-0.0403}{\kilo\watt\per\kelvin}
  \end{align*}

  \textbf{Validate:} \\
  Even though the \nameref{def:Entropy} inside the system decreased, the overall entropy of the entire system (compressor and surroundings) increased by a similar amount.
  \begin{align*}
    \FlowRate{\Entropy} &= \frac{\FlowRate{\Heat}}{\Temp} \\
                        &= \frac{\SI{12}{\kilo\watt}}{(10 + 273.15)\si{\kelvin}} \\
                        &= \SI{0.0424}{\kilo\watt\per\kelvin}
  \end{align*}

  \textbf{Generalize:} \\
  If we add the entropy gained by the surroundings and the entropy lost by the \nameref{def:Compressor} together, we get a non-zero result.
  This aligns with what we understand about \nameref{def:Entropy} and \nameref{def:Irreversible_Process}es and the fact that we assumed this process was actually a \nameref{def:Reversible_Process}.
\end{example}

\begin{example}[Textbook Problem 8.130]{Isentropic Efficiency of Compressor}
  An \nameref{def:Adiabatic} steady-flow device compresses argon at $\Pressure_{In} = \SI{200}{\kilo\pascal}$ and $\Temp_{In} = \SI{27}{\degreeCelsius}$ to $\Pressure_{Out} = \SI{2}{\mega\pascal}$.
  If the argon leaves this \nameref{def:Compressor} at $\Temp_{Out} = \SI{550}{\degreeCelsius}$ what is the isentropic efficiency, $\Isentropic{\Efficiency}$ of the compressor?
  \tcblower{}
  \textbf{Concepts:} \\
  Because the working fluid is argon, it is an ideal gas.
  The equations we found for ideal gases are really just approximations.
  We have $\SpecificHeatPressure$, $\SpecificHeatVolume$, and $k$. \\
  Since we need to assume that the system is \nameref{def:Isentropic} for some of the calculations, we have some equations that we can use for an ideal gas (\Cref{subsec:Ideal_Gases_Entropy_Changes}).

  \textbf{Explore:} \\
  The efficiency of a compressor is:
  \begin{equation*}
    \Isentropic{\Efficiency} = \frac{\FlowRate{\Work}_{In, \text{Ideal}}}{\FlowRate{\Work}_{Out, \text{Actual}}}
  \end{equation*}

  The energy balance for a compressor is:
  \begin{equation*}
    \FlowRate{\Work}_{In} = \FlowRate{\Mass} (\SpecificEnthalpy_{Out} - \SpecificEnthalpy_{In})
  \end{equation*}

  If we substitute that into the isentropic efficiency equation:
  \begin{align*}
    \Isentropic{\Efficiency} &= \frac{\FlowRate{\Work}_{In, \text{Ideal}}}{\FlowRate{\Work}_{Out, \text{Actual}}} \\
                             &= \frac{\FlowRate{\Mass} (\SpecificEnthalpy_{Out, \text{Ideal}} - \SpecificEnthalpy_{In})}{\FlowRate{\Mass} (\SpecificEnthalpy_{Out, \text{Actual}} - \SpecificEnthalpy_{In})} \\
    \shortintertext{We can cancel out like terms.}
                             &= \frac{\Change{\SpecificEnthalpy}_{\text{Ideal}}}{\Change{\SpecificEnthalpy}_{\text{Actual}}} \\
    \intertext{Because this is an ideal gas, we can substitute in \Cref{eq:Change_Internal_Energy_Specific_Heat_Constant_Pressure}.}
                             &= \frac{\SpecificHeatPressure \Change{\Temp}_{\text{Ideal}}}{\SpecificHeatPressure \Change{\Temp}_{\text{Actual}}} \\
                             &\approx \frac{\Change{\Temp}_{\text{Ideal}}}{\Change{\Temp}_{\text{Actual}}} \\
    &= \frac{\Temp_{Out, \text{Ideal}} - \Temp_{In}}{\Temp_{Out, \text{Actual}} - \Temp_{In}}
  \end{align*}

  To find the \nameref{def:Isentropic} output fluid temperature, we have \Crefrange{subeq:Isentropic_Ideal_Gas-Pressure_Volume}{subeq:Isentropic_Ideal_Gas-Temperature_Pressure}.
  \begin{align*}
    \frac{\Temp_{2}}{\Temp_{1}} &= {\left( \frac{\Pressure_{2}}{\Pressure_{1}} \right)}^{\frac{k-1}{k}} \\
    \Temp_{Out, \text{Ideal}} &= \Temp_{In} {\left( \frac{\Pressure_{2}}{\Pressure_{1}} \right)}^{\frac{k-1}{k}}
  \end{align*}
  where $k = \frac{\SpecificHeatPressure}{\SpecificHeatVolume}$.

  Table A.2 will have these values.

  \textbf{Plan:} \\
  Solve for the \nameref{def:Isentropic} output fluid temperature.

  \textbf{Solve:} \\
  From Table A.2:
  \begin{align*}
    \SpecificHeatPressure &= \SI{0.5203}{\kilo\joule\per\kilo\gram\per\kelvin} \\
    \SpecificHeatVolume &= \SI{0.3122}{\kilo\joule\per\kilo\gram\per\kelvin} \\
    k &= 1.667
  \end{align*}

  Solving for the ideal outlet fluid temperature:
  \begin{align*}
    \Temp_{Out, \text{Ideal}} &= \Temp_{In} {\left( \frac{\Pressure_{2}}{\Pressure_{1}} \right)}^{\frac{k-1}{k}} \\
                              &= (27 + 273.15) {\left( \frac{\SI{2000}{\kilo\pascal}}{\SI{200}{\kilo\pascal}} \right)}^{\frac{1.667 - 1}{1.667}} \\
    \Temp_{Out, \text{Ideal}} &= \SI{754.151}{\kelvin} \\
                              &= \SI{481.001}{\degreeCelsius}
  \end{align*}

  Now for the isentropic efficiency:
  \begin{align*}
    \Isentropic{\Efficiency} &= \frac{\Temp_{Out, \text{Ideal}} - \Temp_{In}}{\Temp_{Out, \text{Actual}} - \Temp_{In}} \\
                             &= \frac{481.001 - 27}{550 - 27} \\
    \Isentropic{\Efficiency} &= 0.868071
  \end{align*}

  \textbf{Validate:} \\

  \textbf{Generalize:} \\

\end{example}

\subsubsection{Pump}\label{subsubsec:Pump_Isentropic_Efficiency}
Liquids are much more dense than gases.
This means that the previously-used energy balance equation is not quite the same.
Before, we had:
\begin{align*}
  \FlowRate{\Energy} &= \FlowRate{\Mass} \SpecificEnergy \\
                     &= \FlowRate{\Mass} (\FlowEnergy + \KineticEnergy + \PotentialEnergy) \\
                     &= \FlowRate{\Mass} \left( \frac{\Change{\Pressure}}{\Density} + \frac{\Change{\Velocity^{2}}}{2} + \Gravity \Change{\Distance} \right) \\
  \Change{\FlowRate{\Heat}} - \Change{\FlowRate{\Work}} &= \FlowRate{\Mass} \left( (\SpecificEnthalpy_{Out} - \SpecificEnthalpy_{In}) + \frac{\Velocity_{Out}^{2} - \Velocity_{In}^{2}}{2} + \Gravity (\Distance_{Out} - \Distance_{In}) \right)
\end{align*}

If we remember the \nameref{subsec:Clausius_Inequality} for a \nameref{def:Reversible_Process}.
\begin{align*}
  \Change{\Entropy} &= 0 \\
  \int\limits_{1}^{2} \Temp d\Entropy = \int\limits_{1}^{2} d\Heat &= \Change{\Heat} \\
  \Temp \Change{\Entropy} &= \Change{\Heat}
\end{align*}

Now, we connect it to both \nameref{def:Closed_System}s and \nameref{def:Open_System}s.
\begin{align*}
  \Heat - \Work &= \InternalEnergy \\
  \shortintertext{Now, looking at the individual definitions for each expression.}
  \Heat &= \int \Temp d\Entropy \\
  \Work &= \int \Pressure d\Volume \\
  \Enthalpy &= \InternalEnergy + \Pressure \Volume \\
  d\Enthalpy &= d\InternalEnergy + \Pressure d\Volume + \Volume d\Pressure \\
  \shortintertext{Take a differential of each term in the equation.}
  d\Heat - d\Work = d\InternalEnergy \\
  \Temp d\Entropy - \Pressure d\Volume = d\InternalEnergy
  \shortintertext{Substitute in for the internal energy.}
  \Temp d\Entropy &= d\Enthalpy - \Volume d\Pressure \\
  \int \Temp d\Entropy &= \int d\Enthalpy - \int \Volume d\Pressure
\end{align*}

In fact, we have created an important equation, called Gibbs Equation, shown in \Cref{eq:Gibbs_Equation}.
\begin{equation}\label{eq:Gibbs_Equation}
  \Temp d\Entropy = d\Enthalpy - \Volume d\Pressure
\end{equation}

Using these results, we can substitute back into the original energy balance for heat.
\begin{align*}
  \Change{\Heat} - \Change{\Work} &= \Mass (\SpecificEnthalpy_{Out} - \SpecificEnthalpy_{In}) + \left( \frac{\Velocity_{Out}^{2} - \Velocity_{In}^{2}}{2} \right) + \Gravity (\Distance_{Out} - \Distance_{In}) \\
  \shortintertext{Drop the mass.}
  \Change{q} - \Change{w} &= (\SpecificEnthalpy_{Out} - \SpecificEnthalpy_{In}) + \left( \frac{\Velocity_{Out}^{2} - \Velocity_{In}^{2}}{2} \right) + \Gravity (\Distance_{Out} - \Distance_{In}) \\
  \shortintertext{Remember,}
  \Heat &= (\SpecificEnthalpy_{Out} - \SpecificEnthalpy_{In}) - \int \Volume d\Pressure \\
  \shortintertext{So,}
  \Work &= -\int \Volume d\Pressure - \Change{\KineticEnergy} - \Change{\PotentialEnergy} \\
  \intertext{For a pump, we tend to ignore the kinetic and potential energy.
  Because it is difficult for liquids to change in pressure, we tend to treat them as if it is constant.}
  \Work &= -\Volume (\Pressure_{Out}- \Pressure_{In}) \\
  w &= -\SpecificVolume (\Pressure_{Out}- \Pressure_{In})
\end{align*}

Using all of this information, we can now find the isentropic efficiency of a pump.
\begin{equation}\label{eq:Isentropic_Efficiency-Pump}
  \begin{aligned}
    \Isentropic{\Efficiency} &= \frac{\SpecificVolume (\Pressure_{Out, \text{Ideal}} - \Pressure_{In})}{\SpecificEnthalpy_{Out, \text{Actual}} - \SpecificEnthalpy_{In}} \\
    &= \frac{\SpecificEnthalpy_{Out, \text{Ideal}} - \SpecificEnthalpy_{In}}{\SpecificEnthalpy_{Out, \text{Actual}} - \SpecificEnthalpy_{In}}
  \end{aligned}
\end{equation}

\subsubsection{Nozzle}\label{subsubsec:Nozzle_Isentropic_Efficiency}
Because the major energy present in a nozzle is the change in kinetic energy.
Thus, \Cref{eq:Isentropic_Efficiency-Nozzle} is natural.

\begin{equation}\label{eq:Isentropic_Efficiency-Nozzle}
  \begin{aligned}
    \Isentropic{\Efficiency} &= \frac{\Change{\KineticEnergy}_{\text{Actual}}}{\Change{\KineticEnergy}_{\text{Ideal}}} \\
    &= \frac{\Velocity_{Out, \text{Actual}}^{2} - \Velocity_{In}^{2}}{\Velocity_{Out, \text{Ideal}}^{2} - \Velocity_{In}^{2}}
  \end{aligned}
\end{equation}

%%% Local Variables:
%%% mode: latex
%%% TeX-master: "../../MMAE_320-Thermo-Reference_Sheet"
%%% End:
