\subsection{Ideal Gas Equation}\label{subsec:Ideal_Gas_Equation}
The Ideal Gas Equation is used when:
\begin{itemize}[noitemsep]
\item We have dry air.
\item Air is \textbf{not} good under high pressures, like in turbines.
\item Not good on for liquid water.
\item Can occasionally be used for steam, when the pressure is $\Pressure \leq \SI{10}{\kilo\pascal}$.
\item Not good for steam at high temperatures or pressures.
\item Good for the noble gases, and hydrogen.
\end{itemize}

\begin{equation}\label{eq:Ideal_Gas_Equation}
  \Pressure \SpecificVolume = \GasConstant \Temp
\end{equation}
where
\begin{description}[noitemsep]
\item $\Pressure$: The \textbf{absolute} pressure.
\item $\SpecificVolume$: The \nameref{def:Specific_Volume}.
\item $\GasConstant$: Gas constant, specified in terms of gases and units.
\item $\Temp$: \textbf{Absolute} temperature (\si{\kelvin} or \si{\rankine}).
\end{description}

There is an alternative to \Cref{eq:Ideal_Gas_Equation}, \Cref{eq:Ideal_Gas_Equation-Mass}.
\begin{equation}\label{eq:Ideal_Gas_Equation-Mass}
  \Pressure \Volume = \Mass \GasConstant \Temp
\end{equation}

\begin{example}[Problem 4.79]{Ideal Gas in Tire}
  The air in a tire has its temperature and pressure measured, as the pressure inside the tire depends on the temperature of the air in the tire.
  Initially, at $\Temp_{i} = \SI{25}{\degreeCelsius}$ the pressure is $\Pressure_{i, \text{Gage}} = \SI{210}{\kilo\pascal}$.
  The volume of the tire is $\Volume = \SI{0.025}{\meter\cubed}$.
  Determine the air pressure in the tire when the tmeperature is $\Temp_{f} = \SI{50}{\degreeCelsius}$?
  Also determine the amount of air that must be let out to restore the original pressure at the new temperature, assuming $\Pressure_{atm} = \SI{100}{\kilo\pascal}$?
  \tcblower{}
  \textbf{Concepts:} \\
  \begin{itemize}[noitemsep]
  \item The tire warms up when driven.
    \begin{equation*}
      \Temp_{i} = \SI{25}{\degreeCelsius} \to \Temp_{f} = \SI{50}{\degreeCelsius}
    \end{equation*}
  \item Air is an ideal gas.
  \item The pressure of the outside air is $\Pressure_{atm} = \SI{100}{\kilo\pascal}$
  \item We are given an initial volume, but no final volume.
  \item We are given a gage pressure, which is the atmospheric pressure plus the pressure in the tire.
    \begin{equation*}
      \Pressure_{\text{gage}} = \Pressure_{abs} - \Pressure_{atm}
    \end{equation*}
  \end{itemize}

  \textbf{Explore:} \\
  The \nameref{subsec:Ideal_Gas_Equation} needs both $\Pressure_{abs}$ and $\Temp_{abs}$.
  We should use \Cref{eq:Ideal_Gas_Equation-Mass}, because we need to know how much air to let out.
  \begin{equation*}
    \Pressure \Volume = \Mass \GasConstant \Temp
  \end{equation*}
  The volume is either unchanged, or changes very little.
  \begin{equation*}
    {\left( \frac{\Pressure \Volume}{\Temp} \right)}_{i} = {\left( \frac{\Pressure \Volume}{\Temp} \right)}_{f}
  \end{equation*}
\end{example}

%%% Local Variables:
%%% mode: latex
%%% TeX-master: "../../MMAE_320-Thermo-Reference_Sheet"
%%% End:
