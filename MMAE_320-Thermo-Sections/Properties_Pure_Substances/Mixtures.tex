\subsection{Mixtures}\label{subsec:Mixtures}
In the real world, we come across \nameref{def:Mixture}s all the time.
They do not have the same properties as their \nameref{def:Pure_Substance} counterparts.

\begin{definition}[Mixture]\label{def:Mixture}
  A \emph{mixture} is a substance or set of substances that contain at least more than one different \nameref{def:Saturated_Liquid} and \nameref{def:Saturated_Vapor} at a time.
\end{definition}

When a \nameref{def:Saturated_Liquid} is boiling to become a \nameref{def:Saturated_Vapor}, (Points $A$ and $B$ in \Cref{fig:Phase_Change}), all of the liquid \textbf{must} turn to gas first.
This \nameref{def:Process} has a \nameref{def:Quality}.

\begin{definition}[Quality]\label{def:Quality}
  \emph{Quality} is the ratio of a \nameref{def:Mixture} that is a \nameref{def:Saturated_Vapor} to \nameref{def:Saturated_Liquid}.

  \begin{equation}\label{eq:Quality}
    \begin{aligned}
      \text{Quality} &\equiv \frac{\text{Mass of \nameref{def:Saturated_Vapor}}}{\text{Total mass of \nameref{def:Mixture}}} \\
      \Quality &= \frac{\Mass_{Vapor}}{\Mass_{Liquid} + \Mass_{Vapor}}
    \end{aligned}
  \end{equation}

  \Cref{eq:Quality} is always valued $0 \leq \Quality \leq 1$.
  This is like \nameref{subsec:Energy_Efficiency}, $\Efficiency$.
\end{definition}

\begin{definition}[Specific Volume]\label{def:Specific_Volume}
  \emph{Specific volume} is the volume per unit mass of a substance.
  This is typically well-defined for \nameref{def:Saturated_Mixture}s, because the system is in 2 different phases.
  If we are interested in the specific volume of the \nameref{def:Mixture}, we have \Cref{eq:Specific_Volume}.
  \begin{equation}\label{eq:Specific_Volume}
    \SpecificVolume = \frac{\SaturatedVaporVol}{\SaturatedFluidVol}
  \end{equation}

  \begin{description}[noitemsep]
  \item $\SpecificVolume$: The Specific Volume at any point in time during a \nameref{def:Process}.
  \item $\SaturatedVaporVol$: The \nameref{def:Saturated_Liquid} volume.
  \item $\SaturatedFluidVol$: The \nameref{def:Saturated_Vapor} volume.
  \end{description}

  \begin{equation}\label{eq:Specific_Volume_Using_Quality}
    \begin{aligned}
      \Quality &= \frac{\Volume_{Gas}}{\Volume_{Total}} \\
      \SpecificVolume &= (1 - \Quality) \SaturatedFluidVol + \Quality \SaturatedVaporVol \\
      &= \SaturatedFluidVol + \Quality (\SaturatedVaporVol - \SaturatedFluidVol)
    \end{aligned}
  \end{equation}
\end{definition}

From \Cref{eq:Specific_Volume}, it is clear that to get the total volume of \textbf{one phase of the system}, we use \Cref{eq:Single_Phase_Total_Volume}
\begin{equation}\label{eq:Single_Phase_Total_Volume}
  \SpecificVolume \Mass_{Phase} = \Volume_{Phase}
\end{equation}

Now, \Cref{eq:Single_Phase_Total_Volume} only finds the volume for a single phase in the system, but because volume is additive, we can use \Cref{eq:Total_Volume}.
\begin{equation}\label{eq:Total_Volume}
  \SpecificVolume \Mass_{Total} = \Mass_{Liquid} \SaturatedFluidVol + \Mass_{Gas} \SaturatedVaporVol
\end{equation}

If using \nameref{def:Mixture}s, we can define \nameref{def:Specific_Enthalpy} using \Cref{eq:Mixture_Specific_Enthalpy}.
\begin{equation}\label{eq:Mixture_Specific_Enthalpy}
  \SpecificEnthalpy = \SpecificEnthalpy_{Liquid} + \Quality \SpecificEnthalpy_{Gas-Liquid}
\end{equation}
where $\SpecificEnthalpy_{Gas-Liquid} = \SpecificEnthalpy_{Gas} - \SpecificEnthalpy_{Liquid}$

If using \nameref{def:Mixture}s, we can define \nameref{def:Specific_Energy} using \Cref{eq:Mixture_Internal_Energy}.
\begin{equation}\label{eq:Mixture_Internal_Energy}
  \SpecificInternalEnergy = \SpecificInternalEnergy_{Liquid} + \Quality \SpecificInternalEnergy_{Gas-Liquid}
\end{equation}
where $\SpecificInternalEnergy_{Gas-Liquid} = \SpecificInternalEnergy_{Gas} - \SpecificInternalEnergy_{Liquid}$

In fact, \Cref{eq:Total_Volume} can be re-expressed using \nameref{def:Quality}, which gives us \Cref{eq:Specific_Volume_Using_Quality}.
\begin{align*}
  \SpecificVolume \Mass_{Total} &= \Mass_{Liquid} \SaturatedFluidVol + \Mass_{Gas} \SaturatedVaporVol \\
  \SpecificVolume \Mass_{Total} &= (\Mass_{Total} - \Mass_{Gas}) \SaturatedFluidVol + \Mass_{Gas} \SaturatedVaporVol \\
  \SpecificVolume &= \frac{1}{\Mass_{Total}} \bigl( (\Mass_{Total} - \Mass_{Gas}) \SaturatedFluidVol + \Mass_{Gas} \SaturatedVaporVol \bigr) \\
                                &= \left( 1 - \frac{\Mass_{Gas}}{\Mass_{Total}} \right) \SaturatedFluidVol + \frac{\Mass_{Gas}}{\Mass_{Total}} \SaturatedVaporVol \\
  \Quality &= \frac{\Mass_{Gas}}{\Mass_{Total}} \\
  \SpecificVolume &= (1 - \Quality) \SaturatedFluidVol + \Quality \SaturatedVaporVol \\
  &= \SaturatedFluidVol + \Quality (\SaturatedVaporVol - \SaturatedFluidVol)
\end{align*}

The quantities $\SaturatedFluidVol$ and $\SaturatedVaporVol$ \textbf{are fixed} once pressure and temperature have been established.

\begin{example}{Change in Mixture Properties}
  Suppose there is a pot of water with $\Mass = \SI{3}{\kilo\gram}$ of water at $\Pressure = \SI{101}{\kilo\pascal}$, and temperature $\Temp = \SI{100}{\degreeCelsius}$.
  The pot is sitting at $\Temp$ to start with and continues to heat until all water is evaporated.
  \begin{itemize}[noitemsep]
  \item What is the volume of the water at the start?
  \item What is the \nameref{def:Quality} of the \nameref{def:Mixture} at the start?
  \item If $\Quality=\frac{1}{2}$, what is the volume of the resulting system?
  \item If $\Quality=1$, what is the volume of the resulting system?
  \end{itemize}
  \tcblower{}
  Using the equations we have already, we know
  \begin{equation*}
    \Volume = \SaturatedFluidVol \Mass_{Total}
  \end{equation*}
  From Table A.4 in the textbook, we know $\SaturatedFluidVol = \SI{0.001043}{\meter\cubed\per\kilo\gram}$.
  Thus, we can just plug that value into the equation and solve for $\Volume$.
  \begin{align*}
    \Volume &= \SaturatedFluidVol \Mass_{Total} \\
    \intertext{At this pressure and temperature, the liquid water is not becoming a vapor due to boiling, so $\Mass_{Total} = \Mass_{Liquid}$}
            &= \SI{0.001043}{\meter\cubed\per\kilo\gram} (\SI{3}{\kilo\gram}) \\
            &= \SI{0.003129}{\meter\cubed} \\
            &= \SI{3.129}{\liter}
  \end{align*}

  Now to solve for the ``initial'' \nameref{def:Quality}.
  Because the water has \textbf{\textit{JUST}} reached its \nameref{def:Saturated_Liquid} state, the water has \textbf{begun} boiling, but not yet started to vaporize.
  Thus,
  \begin{equation*}
    \Quality = 0
  \end{equation*}

  We are given the \nameref{def:Quality} of the system, $\Quality=\frac{1}{2}$, meaning we have converted half the \nameref{def:Saturated_Liquid} to vapor.
  We can easily find the mass of the vapor, then find its volume.
  \begin{align*}
    \Quality &= \frac{1}{2} \\
    \intertext{Remember, due to the Law of Conservation of Mass, \textbf{all} \SI{3}{\kilo\gram} of the water remains \textbf{IN} the system!}
      &= \frac{\Mass_{Gas}}{\Mass_{Total}} = \frac{\Mass_{Gas}}{\SI{3}{\kilo\gram}} \\
    \frac{1}{2} &= \frac{\Mass_{Gas}}{\SI{3}{\kilo\gram}} \\
    \Mass_{Gas} &= \SI{1.5}{\kilo\gram} \\
  \end{align*}

  Now that we know the mass of the vaporized liquid, we can use \Cref{eq:Specific_Volume_Using_Quality} to solve for the total volume.
  \begin{align*}
    \SpecificVolume &= \SaturatedFluidVol + \Quality(\SaturatedVaporVol - \SaturatedFluidVol) \\
    \intertext{The value for $\SaturatedFluidVol$ remains constant throughout this entire \nameref{def:Process}, so we can reuse that value. The value for $\SaturatedVaporVol$ is found in Table A.4 from the textbook.}
                    &= \SI{0.001043}{\meter\cubed\per\kilo\gram} + \frac{1}{2} (\SI{1.6720}{\meter\cubed\per\kilo\gram} - \SI{0.001043}{\meter\cubed\per\kilo\gram}) \\
                    &= \SI{0.8365}{\meter\cubed\per\kilo\gram}
  \end{align*}

  Now that we have the \nameref{def:Specific_Volume} of the system, we can find the total volume of the system.
  \begin{align*}
    \Volume_{Total} &= \SpecificVolume \Mass_{Total} \\
                    &= \SI{0.8365}{\meter\cubed\per\kilo\gram} (\SI{3}{\kilo\gram}) \\
                    &= \SI{2509}{\liter}
  \end{align*}

  Remember that $\Volume_{Total}$ contains the volume for \textbf{both} the steam \textbf{and} the water.
  If we are curious about the separate values for each, then we can solve for the individual terms in \Cref{eq:Single_Phase_Total_Volume}.
  \begin{align*}
    \Volume_{Water} &= \Mass_{Water} \SaturatedFluidVol \\
                    &= \SI{1.5}{\kilo\gram} (\SI{0.001043}{\meter\cubed\per\kilo\gram}) \\
                    &= \SI{1.56}{\liter} \\
    \Volume_{Steam} &= \Mass_{Steam} \SaturatedVaporVol \\
                    &= \SI{1.5}{\kilo\gram} (\SI{1.6720}{\meter\cubed\per\kilo\gram}) \\
                    &= \SI{2508}{\liter}
  \end{align*}

  Thus,
  \begin{align*}
    \Volume_{Water} &=  \SI{1.56}{\liter} \\
    \Volume_{Steam} &= \SI{2508}{\liter}
  \end{align*}

  For this last part, $\Quality = 1$, meaning that \textbf{all} the water has become steam.
  Thus, $\Mass_{Water} = \SI{0}{\kilo\gram}$ and $\Mass_{Steam} = \SI{3}{\kilo\gram}$.
  Because we also know $\SaturatedVaporVol = \SI{1.6720}{\meter\cubed\per\kilo\gram}$, we can easily just solve for \Cref{eq:Single_Phase_Total_Volume} as the total volume.
  \begin{align*}
    \Volume_{Total} &= \Volume_{Steam} \\
    \Volume_{Steam} &= \SaturatedVaporVol \Mass_{Steam} \\
                    &= (\SI{1.6720}{\meter\cubed\per\kilo\gram}) (\SI{3}{\kilo\gram}) \\
                    &= \SI{5016}{\liter}
  \end{align*}

  So, after the water has completely become steam, the total volume of the system is $\Volume_{Total} = \SI{5016}{\liter}$.
\end{example}

\begin{example}[Problem 4.23]{What's My State}
  Given certain intensive properties of the \nameref{def:Pure_Substance}, water, fill in the table?
  \begin{enumerate}[noitemsep]
  \item $\Temp = \SI{50}{\degreeCelsius}$ and a volume of $\Volume = \SI{4.16}{\meter\cubed\per\kilo\gram}$.
  \item $\Pressure = \SI{200}{\kilo\pascal}$ and the water is in the \nameref{def:Saturated_Vapor} phase.
  \item $\Temp = \SI{250}{\degreeCelsius}$, $\Pressure = \SI{400}{\kilo\pascal}$, and $\Volume = \SI{0.595}{\meter\cubed\per\kilo\gram}$.
  \item $\Temp = \SI{110}{\degreeCelsius}$, $\Pressure = \SI{600}{\kilo\pascal}$.
  \end{enumerate}
  \tcblower{}
  \begin{center}
    \begin{tabular}{ccccc}
      \toprule
      & Temperature $\si{\degreeCelsius}$ & Pressure $\si{\kilo\pascal}$ & Volume $\si{\meter\cubed\per\kilo\gram}$ & Phase \\
      \midrule
      1 & 50 & 12.352 & 4.16 & \nameref{def:Saturated_Mixture} \\
      2 & 120.21 & 200 & 0.88578 & \nameref{def:Saturated_Vapor} \\
      3 & 250 & 400 & 0.595 & \nameref{def:Superheated_Vapor}/Steam \\
      4 & 110 & 600 & 0.001052 & \nameref{def:Compressed_Liquid} \\
      \bottomrule
    \end{tabular}
  \end{center}

  \begin{enumerate}[noitemsep]
  \item Phase 1:
    \begin{enumerate}[noitemsep]
    \item We need to look at Table A.4 for phase 1's saturated pressure, and find it $\SaturatedPressure$.
    \item Now, looking at the values of $\SaturatedFluidVol$ and $\SaturatedVaporVol$ for water at $\Temp = \SI{50}{\degreeCelsius}$, we see the value we're given is somewhere between those.
      Thus, this is a mixture of a \nameref{def:Saturated_Liquid} and a \nameref{def:Saturated_Vapor}, a \nameref{def:Saturated_Mixture}.
    \end{enumerate}
  \item Phase 2:
    \begin{enumerate}[noitemsep]
    \item We were told a pressure, so we should use Table A.5.
    \item Look in Table A.5 and find $\Pressure = \SI{200}{\kilo\pascal}$, and get the saturation temperature, $\SaturatedTemp$.
    \item They told us it was a \nameref{def:Saturated_Vapor}, so we know the value is somewhere on the saturated vapor line, meaning the volume is the same as $\SaturatedVaporVol$.
    \end{enumerate}
  \item Phase 3:
    \begin{enumerate}[noitemsep]
    \item Because we are given both pressure and temperature, either Table A.4 or Table A.5 is applicable.
    \item However, neither table seems to make much sense, as on the temperature table, the pressure is too low, and on the pressure table, the temperature is too low.
    \item On both tables, our volume is significantly greater than even the $\SaturatedVaporVol$.
    \item Because $\Volume > \SaturatedVaporVol$, we go to Table A.6 for superheated water.
    \item Using Table A.6, we see that $\SaturatedVaporVol$ in the table matches the volume given to us.
    \item This means the water is now in its superheated vapor phase.
    \end{enumerate}
  \item Phase 4:
    \begin{enumerate}[noitemsep]
    \item You can start by using Table A.4 or Table A.5, but I will start with Table A.4.
    \item Find \SI{110}{\degreeCelsius} in the table.
    \item Once there, you'll notice their $\SaturatedPressure$ is \textbf{much} lower than what we were given, so we move to the next table.
    \item So, we move to Table A.5, and the temperature in the table is lower than what we were given, so we move onto the next table.
    \item This means that the temperature is lower than is needed to start boiling the water at \SI{600}{\kilo\pascal} and the pressure is too high for the water to start boiling at \SI{110}{\degreeCelsius}.
    \item This means that it must be a \nameref{def:Compressed_Liquid}.
    \item But, we said to treat \nameref{def:Compressed_Liquid}s as \nameref{def:Saturated_Liquid} at temperature $\Temp$, so use Table A.4 instead.
    \end{enumerate}
  \end{enumerate}
\end{example}

\begin{example}[Problem 4.112]{Changing State of R-134a}
  Given $\Mass = \SI{1}{\kilo\gram}$ of R-134a fills a rigid container of volume $\Volume = \SI{0.1450}{\meter\cubed}$ at an initial temperature of $\Temp_{i} = \SI{-40}{\degreeCelsius}$, at a final pressure of $\Pressure_{f} = \SI{200}{\kilo\pascal}$.
  If the container is heated, determine the inital pressure ($\Pressure_{i}$) and the final temperature ($\Temp_{f}$)?
  \tcblower{}
  \textbf{Concepts:} \\
  The vessel has rigid walls, meaning the vessel's volume will remain constant.
  The \nameref{def:Law_Conservation_Mass} states that the gas's mass will remain constant throughout the \nameref{def:Process}. \\
  \begin{equation*}
    \SpecificVolume = \frac{\Volume}{\Mass}
  \end{equation*}

  Finding a state requires knowing 2 of 3 \nameref{def:Intensive_Property}, ($\Pressure$, $\Temp$, $\SpecificVolume$).

  \textbf{Explore:} \\
  Because both mass and volume are constant, the \nameref{def:Specific_Volume} remains constant.
  Values for R-134a are present in Tables A.11 and A.12 in the textbook, which mirror Tables A.4 and A.5, respectively.

  \textbf{Plan:} \\
  Using $\Pressure$ or $\Temp$ and $\SpecificVolume$, solve for the state.

  \textbf{Solve:} \\
  We use Table A.11 in the textbook, because we are given a temperature.
  So, we know:
  \begin{center}
    \begin{tabular}{cccc}
      \toprule
      $\Temp$ & $\SaturatedPressure$ & $\SaturatedFluidVol$ & $\SaturatedVaporVol$ \\
      \midrule
      \SI{-40}{\degreeCelsius} & \SI{51.25}{\kilo\pascal} & \SI{0.0007053}{\meter\cubed\per\kilo\gram} & \SI{0.36064}{\meter\cubed\per\kilo\gram} \\
      \bottomrule
    \end{tabular}
  \end{center}

  Now, we can also find $\SpecificVolume$:
  \begin{align*}
    \SpecificVolume &= \frac{\Volume}{\Mass} \\
                    &= \frac{\SI{0.1450}{\meter\cubed}}{\SI{1}{\kilo\gram}} \\
                    &= \SI{0.1450}{\meter\cubed\per\kilo\gram}
  \end{align*}

  Now, we notice that $\SpecificVolume$ is somewhere between $\SaturatedFluidVol$ and $\SaturatedVaporVol$, meaning the R-134a is in its \nameref{def:Saturated_Mixture} phase right now.

  We use Table A.12 in the textbook, because we are given a pressure for the final state.
  So, we know:
  \begin{center}
    \begin{tabular}{cccc}
      \toprule
      $\Pressure$ & $\SaturatedTemp$ & $\SaturatedFluidVol$ & $\SaturatedVaporVol$ \\
      \midrule
      \SI{200}{\kilo\pascal} & \SI{-10.09}{\degreeCelsius} & \SI{0.0007381}{\meter\cubed\per\kilo\gram} & \SI{0.099951}{\meter\cubed\per\kilo\gram} \\
      \bottomrule
    \end{tabular}
  \end{center}

  Because we know $\SpecificVolume = \SI{0.1450}{\meter\cubed\per\kilo\gram}$, and we have the values for $\SaturatedFluidVol$ and $\SaturatedVaporVol$, we can tell that the gas is not in a superheated vapor state.
  Because the gas is superheated, we need to use Table A.13 instead.

  That table says that a gas with $\SpecificVolume = \SI{0.14504}{\meter\cubed\per\kilo\gram}$ is at $\SaturatedTemp = \SI{90}{\degreeCelsius}$.

  Thus,
  \begin{align*}
    \Pressure_{i} &= \SI{51.25}{\kilo\pascal} \\
    \Temp_{f} &= \SI{90}{\degreeCelsius}
  \end{align*}
\end{example}

\begin{example}[Problem 4.113]{Changing State of Water English}
  Given $\Mass = \SI{1}{\lbm}$ of water at an initial temperature $\Temp_{i} = \SI{400}{\degreeF}$ fills a weighted piston cylinder device with movable lid of volume $\Volume_{i} = \SI{2.649}{\feet\cubed}$.
  If the vessel is cooled to $\Temp_{f} = \SI{100}{\degreeF}$, determine the final pressure ($\Pressure_{f}$) and the final volume of the container $\Volume_{f}$?
  \tcblower{}
  \textbf{Concepts:} \\
  Weighted piston cylinder devices ensure that the pressure exerted by the fluid remains constant, although we don't know what it is.
  The mass of the water in this problem remained unchanged throughout the \nameref{def:Process} as well.
  We know the initial volume, initial temperature and the final temperature.
  Because we were given each temperature, we know the fluid in the vessel is being cooled.

  \textbf{Explore:} \\
  We want to get 2 of those 3 those \nameref{def:Intensive_Property} ($\Pressure$, $\Temp$, or $\SpecificVolume$).
   We can find the $\SpecificVolume_{i}$ for the initial point in time, because we know the volume of the vessel initially and the mass is constant.
  We also know the initial temperature, so we have all the \nameref{def:Intensive_Property}s that we need to solve for the initial state.

  \textbf{Plan:} \\
  Use Table A.4E to find $\Pressure$.
  Because the pressure is constant, we can use $\Pressure$ and the final temperature $\Temp_{f}$ to solve for the final state's $\Volume_{f}$.

  \textbf{Solve:} \\
  Fome Table A.4E, we can find and know
  \begin{center}
    \begin{tabular}{cccc}
      \toprule
      $\Temp$ & $\SaturatedPressure$ & $\SaturatedFluidVol$ & $\SaturatedVaporVol$ \\
      \midrule
      \SI{400}{\degreeF} & \SI{247}{\psia} & \SI{0.01864}{\feet\cubed\per\lbm} & \SI{1.864}{\feet\cubed\per\lbm} \\
      \bottomrule
    \end{tabular}
  \end{center}

  We also know that the \nameref{def:Specific_Volume} of the water is:
  \begin{align*}
    \SpecificVolume &= \frac{\Volume}{\Mass} \\
                    &= \frac{\SI{2.649}{\feet\cubed}}{\SI{1}{\lbm}} \\
                    &= \SI{2.649}{\feet\cubed\per\lbm} \\
  \end{align*}

  Because $\SpecificVolume > \SaturatedVaporVol$, we know that the fluid is a superheated vapor, steam.

  Now, because the fluid is a superheated vapor, we need to use Table A.6E to find the pressure that the water/steam is under.
  Using the temperature and the \nameref{def:Specific_Volume} of the water, we look those values up in Table A.6E, we see that $\Pressure = \SI{180}{\psia}$.
  The saturation pressure $\SaturatedTemp = \SI{373}{\degreeF}$, which means that the final temperature $\Temp_{f}$ is likely a liquid.

  Because the water is likely a liquid at its final temperature, we use Table A.4E and look up the value of water at \SI{100}{\degreeF}.
  \begin{center}
    \begin{tabular}{cccc}
      \toprule
      $\Temp$ & $\SaturatedPressure$ & $\SaturatedFluidVol$ & $\SaturatedVaporVol$ \\
      \midrule
      \SI{100}{\degreeF} & \SI{0.95052}{\psia} & \SI{0.01613}{\feet\cubed\per\lbm} & \SI{349.83}{\feet\cubed\per\lbm} \\
      \bottomrule
    \end{tabular}
  \end{center}

  If we use the pressure we got from Table A.6E $\Pressure = \SI{180}{\psia}$, and look that value up in Table A.5E, we see that $\SaturatedTemp = \SI{373.07}{\degreeF}$ to even boil the water.
  This likely means that the water is a \nameref{def:Compressed_Liquid}, and because of our rule, $\SpecificVolume_{f} = \SI{0.01613}{\feet\cubed\per\lbm}$.
  Therefore, by multiplying $\SpecificVolume_{f}$ by the mass of the water, we can find the volume it is consuming.
  \begin{align*}
    \Volume_{f} &= \SpecificVolume_{f} m \\
                &= \SI{0.01613}{\feet\cubed\per\lbm} (\SI{1}{\lbm}) \\
                &= \SI{0.01613}{\feet\cubed}
  \end{align*}
\end{example}

%%% Local Variables:
%%% mode: latex
%%% TeX-master: "../../MMAE_320-Thermo-Reference_Sheet"
%%% End:
