\section{Energy, and Energy Transfer}\label{sec:Energy_Energy_Transfer}
\begin{definition}[Macroscopic Energy Form]\label{def:Macroscopic_Energy_Form}
  \emph{Macroscopic energy form}s are typically ones that have to deal with objects on a macroscopic level.
  These energies are:
  \begin{enumerate}[noitemsep]
  \item Kinetic
  \item Potential
  \end{enumerate}

  \begin{remark}
    This definition is included because the textbook makes use of it.
  \end{remark}
\end{definition}

\begin{definition}[Microscopic Energy Form]\label{def:Microscopic_Energy_Form}
  \emph{Microscopic energy forms} are energies that act on non-macroscopic levels.
  Namely, they affect their systems on microscopic levels.
  These energies include:
  \begin{enumerate}[noitemsep]
  \item Sensible
    \begin{itemize}[noitemsep]
    \item Heat
    \item Kinetic energy of molecules
    \end{itemize}
  \item Latent
    \begin{itemize}[noitemsep]
    \item Phase Changes
    \end{itemize}
  \item Chemical
    \begin{itemize}[noitemsep]
    \item Combustion
    \end{itemize}
  \item Nuclear
  \end{enumerate}

  \begin{remark}
    This definition is included because the textbook makes use of it.
  \end{remark}
\end{definition}

\begin{definition}[Internal Energy]\label{def:Internal_Energy}
  \emph{Internal energy} is equivalent to \nameref{def:Microscopic_Energy_Form}s.
  It means the \nameref{def:Energy} that the object in question inherently has at that point in time.
\end{definition}

\subsection{Energy Quality}\label{subsec:Energy_Quality}
Energy has quality!
\begin{itemize}[noitemsep]
\item \nameref{def:Macroscopic_Energy_Form}
  \begin{itemize}[noitemsep]
  \item Structured
  \item Moves as a single unit
  \end{itemize}
\item \nameref{def:Microscopic_Energy_Form}
  \begin{itemize}[noitemsep]
  \item \textbf{Not} structures
  \item Does \textbf{not} move as a single unit
  \end{itemize}
\end{itemize}

These differences mean that we measure the efficiency of each type of energy form differently.

\subsection{Energy and Flows}\label{subsec:Energy_and_Flows}
When moving a fluid through a pipe, we can find the amount of work done by the fluid flowing, called teh \nameref{def:Flow_Energy}.
\begin{align*}
  \Pressure &= \frac{\Force}{\text{Area}} \\
  \Volume_{\mathrm{Cylinder}} &= \ell \cdot \text{Area} \\
  \Work &= \Force \cdot \text{Distance} \\
\end{align*}

If we substitute for the common terms in the formula for work, then we end up with \Cref{eq:Flow_Energy}.


%%% Local Variables:
%%% mode: latex
%%% TeX-master: "../MMAE_320-Thermo-Reference_Sheet"
%%% End:
