\section{Energy, and Energy Transfer}\label{sec:Energy_Energy_Transfer}
\begin{definition}[Macroscopic Energy Form]\label{def:Macroscopic_Energy_Form}
  \emph{Macroscopic energy form}s are typically ones that have to deal with objects on a macroscopic level.
  These energies are:
  \begin{enumerate}[noitemsep]
  \item Kinetic
  \item Potential
  \end{enumerate}

  \begin{remark}
    This definition is included because the textbook makes use of it.
  \end{remark}
\end{definition}

\begin{definition}[Microscopic Energy Form]\label{def:Microscopic_Energy_Form}
  \emph{Microscopic energy forms} are energies that act on non-macroscopic levels.
  Namely, they affect their systems on microscopic levels.
  These energies include:
  \begin{enumerate}[noitemsep]
  \item Sensible
    \begin{itemize}[noitemsep]
    \item Heat
    \item Kinetic energy of molecules
    \end{itemize}
  \item Latent
    \begin{itemize}[noitemsep]
    \item Phase Changes
    \end{itemize}
  \item Chemical
    \begin{itemize}[noitemsep]
    \item Combustion
    \end{itemize}
  \item Nuclear
  \end{enumerate}

  \begin{remark}
    This definition is included because the textbook makes use of it.
  \end{remark}
\end{definition}

\begin{definition}[Internal Energy]\label{def:Internal_Energy}
  \emph{Internal energy} is equivalent to \nameref{def:Microscopic_Energy_Form}s.
  It means the \nameref{def:Energy} that the object in question inherently has at that point in time.
\end{definition}

\subsection{Energy Quality}\label{subsec:Energy_Quality}
Energy has quality!
\begin{itemize}[noitemsep]
\item \nameref{def:Macroscopic_Energy_Form}
  \begin{itemize}[noitemsep]
  \item Structured
  \item Moves as a single unit
  \end{itemize}
\item \nameref{def:Microscopic_Energy_Form}
  \begin{itemize}[noitemsep]
  \item \textbf{Not} structures
  \item Does \textbf{not} move as a single unit
  \end{itemize}
\end{itemize}

These differences mean that we measure the efficiency of each type of energy form differently.

\subsection{Energy and Flows}\label{subsec:Energy_and_Flows}
When moving a fluid through a pipe, we can find the amount of work done by the fluid flowing, called teh \nameref{def:Flow_Energy}.
\begin{align*}
  \Pressure &= \frac{\Force}{\text{Area}} \\
  \Volume_{\mathrm{Cylinder}} &= \ell \cdot \text{Area} \\
  \Work &= \Force \cdot \text{Distance} \\
\end{align*}

If we substitute for the common terms in the formula for work, then we end up with \Cref{eq:Flow_Energy}.

\begin{definition}[Flow Energy]\label{def:Flow_Energy}
  \emph{Flow energy} is the energy that a fluid flowing through a long, straight pipe has.

  \begin{equation}\label{eq:Flow_Energy}
    \begin{aligned}
      \Work &= \Pressure \Volume \\
      \FlowEnergy &= \Pressure \Volume \\
    \end{aligned}
  \end{equation}

  \begin{remark}[Energy Form]
    Typically, \nameref{def:Flow_Energy} is categorized with the \nameref{def:Macroscopic_Energy_Form}s, because it behaves more like those and can be nearly as efficient as them.
    This is true even though this is technically an application of microscopic energies.
    This is because we are not worried about the internal energy of the fluid in the pipe, but are instead interested in the mechanical movement of it.
  \end{remark}
\end{definition}

\subsection{Divisions of Energy}\label{subsec:Divisions_of_Energy}
We are always interested in the change in energy that occurs due to something.
This is seen as \Cref{eq:Change_Total_Energy}.

\begin{equation}\label{eq:Change_Total_Energy}
  \begin{aligned}
    \Change{\TotalEnergy} &= \Change{\InternalEnergy} + \Change{\KineticEnergy} + \Change{\PotentialEnergy} + \Change{\FlowEnergy} \\
    &= \frac{\InternalEnergy_{2} - \InternalEnergy_{1}}{\Mass} + \frac{v_{2}^{2} - v_{1}^{2}}{2} + \Gravity (h_{2} - h_{1}) + \frac{\Pressure_{2} - \Pressure_{1}}{\Density} \\
  \end{aligned}
\end{equation}
\begin{itemize}[noitemsep]
\item The internal energy cannot be completely converted into work.
\item Mechanical energy is typically defined to be these types of energies.
  These can be completely converted into work by an ideal machine.
  \begin{itemize}[noitemsep]
  \item Kinetic energy ($\KineticEnergy$)
  \item Potential energy ($\PotentialEnergy$)
  \item Flow energy ($\FlowEnergy$)
  \end{itemize}
\end{itemize}

We are also interested in the \nameref{def:Specific_Energy} of the system.
\begin{definition}[Specific Energy]\label{def:Specific_Energy}
  \emph{Specific energy} is an \nameref{def:Intensive_Property} of a system.
  It is the total energy of a system divided by the total mass of the system.
  \begin{equation}\label{eq:Specific_Energy}
    \begin{aligned}
      \SpecificEnergy &= \frac{\TotalEnergy}{\Mass} \\
      &= \frac{\InternalEnergy}{\Mass} + \frac{1}{2} v^{2} + \Gravity h + \frac{\Pressure \Volume}{\Mass} \\
      &= \frac{\InternalEnergy}{\Mass} + \frac{1}{2} v^{2} + \Gravity h + \frac{\Pressure}{\Density} \\
    \end{aligned}
  \end{equation}
\end{definition}

The \nameref{def:Specific_Energy} of a system can be used to find the change in energy per unit time, or the \nameref{def:Power}.
\begin{equation}\label{eq:Energy_Flow_Rate}
  \begin{aligned}
    \Change{\FlowRate{\TotalEnergy}} &= \FlowRate{\Mass} \Change{\SpecificEnergy} \\
    \Power &= \FlowRate{\Mass} \Change{\SpecificEnergy} \\
  \end{aligned}
\end{equation}
where $\FlowRate{\Mass}$ is the mass flow rate, seen by \Cref{eq:Mass_Flow_Rate}

\begin{equation}\label{eq:Mass_Flow_Rate}
  \FlowRate{\Mass} = \frac{\Mass}{\Time}
\end{equation}


%%% Local Variables:
%%% mode: latex
%%% TeX-master: "../MMAE_320-Thermo-Reference_Sheet"
%%% End:
