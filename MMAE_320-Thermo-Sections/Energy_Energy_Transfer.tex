\section{Energy, and Energy Transfer}\label{sec:Energy_Energy_Transfer}
\begin{definition}[Macroscopic Energy Form]\label{def:Macroscopic_Energy_Form}
  \emph{Macroscopic energy form}s are typically ones that have to deal with objects on a macroscopic level.
  These energies are:
  \begin{enumerate}[noitemsep]
  \item Kinetic
  \item Potential
  \end{enumerate}

  \begin{remark}
    This definition is included because the textbook makes use of it.
  \end{remark}
\end{definition}

\begin{definition}[Microscopic Energy Form]\label{def:Microscopic_Energy_Form}
  \emph{Microscopic energy forms} are energies that act on non-macroscopic levels.
  Namely, they affect their systems on microscopic levels.
  These energies include:
  \begin{enumerate}[noitemsep]
  \item Sensible
    \begin{itemize}[noitemsep]
    \item Heat
    \item Kinetic energy of molecules
    \end{itemize}
  \item Latent
    \begin{itemize}[noitemsep]
    \item Phase Changes
    \end{itemize}
  \item Chemical
    \begin{itemize}[noitemsep]
    \item Combustion
    \end{itemize}
  \item Nuclear
  \end{enumerate}

  \begin{remark}
    This definition is included because the textbook makes use of it.
  \end{remark}
\end{definition}

\begin{definition}[Internal Energy]\label{def:Internal_Energy}
  \emph{Internal energy} is equivalent to \nameref{def:Microscopic_Energy_Form}s.
  It means the \nameref{def:Energy} that the object in question inherently has at that point in time.
\end{definition}

\subsection{Energy Quality}\label{subsec:Energy_Quality}
Energy has quality!
\begin{itemize}[noitemsep]
\item \nameref{def:Macroscopic_Energy_Form}
  \begin{itemize}[noitemsep]
  \item Structured
  \item Moves as a single unit
  \end{itemize}
\item \nameref{def:Microscopic_Energy_Form}
  \begin{itemize}[noitemsep]
  \item \textbf{Not} structures
  \item Does \textbf{not} move as a single unit
  \end{itemize}
\end{itemize}

These differences mean that we measure the efficiency of each type of energy form differently.

\subsection{Energy and Flows}\label{subsec:Energy_and_Flows}
When moving a fluid through a pipe, we can find the amount of work done by the fluid flowing, called teh \nameref{def:Flow_Energy}.
\begin{align*}
  \Pressure &= \frac{\Force}{\text{Area}} \\
  \Volume_{\mathrm{Cylinder}} &= \ell \cdot \text{Area} \\
  \Work &= \Force \cdot \text{Distance} \\
\end{align*}

If we substitute for the common terms in the formula for work, then we end up with \Cref{eq:Flow_Energy}.

\begin{definition}[Flow Energy]\label{def:Flow_Energy}
  \emph{Flow energy} is the energy that a fluid flowing through a long, straight pipe has.

  \begin{equation}\label{eq:Flow_Energy}
    \begin{aligned}
      \Work &= \Pressure \Volume \\
      \FlowEnergy &= \Pressure \Volume \\
    \end{aligned}
  \end{equation}

  \begin{remark}[Energy Form]
    Typically, \nameref{def:Flow_Energy} is categorized with the \nameref{def:Macroscopic_Energy_Form}s, because it behaves more like those and can be nearly as efficient as them.
    This is true even though this is technically an application of microscopic energies.
    This is because we are not worried about the internal energy of the fluid in the pipe, but are instead interested in the mechanical movement of it.
  \end{remark}
\end{definition}

\subsection{Divisions of Energy}\label{subsec:Divisions_of_Energy}
We are always interested in the change in energy that occurs due to something.
This is seen as \Cref{eq:Change_Total_Energy}.

\begin{equation}\label{eq:Change_Total_Energy}
  \begin{aligned}
    \Change{\TotalEnergy} &= \Change{\InternalEnergy} + \Change{\KineticEnergy} + \Change{\PotentialEnergy} + \Change{\FlowEnergy} \\
    &= \frac{\InternalEnergy_{2} - \InternalEnergy_{1}}{\Mass} + \frac{v_{2}^{2} - v_{1}^{2}}{2} + \Gravity (h_{2} - h_{1}) + \frac{\Pressure_{2} - \Pressure_{1}}{\Density} \\
  \end{aligned}
\end{equation}
\begin{itemize}[noitemsep]
\item The internal energy cannot be completely converted into work.
\item Mechanical energy is typically defined to be these types of energies.
  These can be completely converted into work by an ideal machine.
  \begin{itemize}[noitemsep]
  \item Kinetic energy ($\KineticEnergy$)
  \item Potential energy ($\PotentialEnergy$)
  \item Flow energy ($\FlowEnergy$)
  \end{itemize}
\end{itemize}

We are also interested in the \nameref{def:Specific_Energy} of the system.
\begin{definition}[Specific Energy]\label{def:Specific_Energy}
  \emph{Specific energy} is an \nameref{def:Intensive_Property} of a system.
  It is the total energy of a system divided by the total mass of the system.
  \begin{equation}\label{eq:Specific_Energy}
    \begin{aligned}
      \SpecificEnergy &= \frac{\TotalEnergy}{\Mass} \\
      &= \frac{\InternalEnergy}{\Mass} + \frac{1}{2} v^{2} + \Gravity h + \frac{\Pressure \Volume}{\Mass} \\
      &= \frac{\InternalEnergy}{\Mass} + \frac{1}{2} v^{2} + \Gravity h + \frac{\Pressure}{\Density} \\
    \end{aligned}
  \end{equation}
\end{definition}

The \nameref{def:Specific_Energy} of a system can be used to find the change in energy per unit time, or the \nameref{def:Power}.
\begin{equation}\label{eq:Energy_Flow_Rate}
  \begin{aligned}
    \Change{\FlowRate{\TotalEnergy}} &= \FlowRate{\Mass} \Change{\SpecificEnergy} \\
    \Power &= \FlowRate{\Mass} \Change{\SpecificEnergy} \\
  \end{aligned}
\end{equation}
where $\FlowRate{\Mass}$ is the mass flow rate, seen by \Cref{eq:Mass_Flow_Rate}

\begin{equation}\label{eq:Mass_Flow_Rate}
  \FlowRate{\Mass} = \frac{\Mass}{\Time}
\end{equation}

\subsection{Energy Flow and Systems}\label{subsec:Energy_Flow_Systems}
These systems have \nameref{def:Energy} interactions that cross the boundary of a \nameref{def:System}.
This only has 2 options:
\begin{enumerate}[noitemsep]
\item \nameref{def:Heat} transfer into or out of the \nameref{def:System}.
\item \nameref{def:Work} done on the system by the surroundings.
\end{enumerate}

\begin{example}[Lecture 4, Problem 3.12]{Power of Water}
  Consider a river flowing towards a lake at an average velocity of $v_{\mathrm{River}} = \SI{3}{\meter\per\second}$ and a volume flow rate of $\FlowRate{\Volume} = \SI{500}{\cubic\meter\per\second}$.
  The \SI{500}{\cubic\meter\per\second} is at a location \SI{90}{\meter} above the lake's surface.
  Determine the total mechanical energy in the river water per unit mass and the power generation potential of the entire river?
  \tcblower{}
  The question is really asking us to find the \nameref{def:Specific_Energy} and the power generation.

  \textbf{Assumption:} Assume the flow is constant, that we are in steady flow.

  \textbf{Concepts and Explore:}\\
  There is no state change, likely meaning $\Change{\InternalEnergy} = 0$.
  The velocity of the water within the lake should be $v_{\mathrm{Lake}} = 0$ ($\Change{\KineticEnergy} \neq 0$).
  There is a potential energy change, $\Change{\PotentialEnergy} \neq 0$.
  There is a flow, but the only pressure involved is $\Pressure_{atm}$, and the height difference is so small that the change in pressure is negligible.

  \begin{equation*}
    \Power = \FlowRate{\Mass} \Change{\SpecificEnergy}
  \end{equation*}

  To find $\FlowRate{\Mass}$, we can multiply the density of water with the volume flow rate of the river, to find the mass flow rate.
  \begin{equation*}
    \FlowRate{\Mass} = \Density_{H_{2}O} \FlowRate{\Volume}
  \end{equation*}

  \textbf{Plan:}
  \begin{enumerate}[noitemsep]
  \item Solve for $\Change{\SpecificEnergy}$ using $\Change{\KineticEnergy} + \Change{\PotentialEnergy}$.
  \item Solve for mass flow rate, $\FlowRate{\Mass} = \Density_{H_{2}O} \FlowRate{\Volume}$.
  \item Solve for $\Power = \FlowRate{\Mass} \Change{\SpecificEnergy}$.
  \end{enumerate}

  \textbf{Solve:}
  \begin{align*}
    \Change{\SpecificEnergy} &= \Change{\KineticEnergy} + \Change{\PotentialEnergy} \\
                             &= \frac{{(\SI{3}{\meter\per\second})}^{2} + {(\SI{0}{\meter\per\second})}^{2}}{2} + \SI{9.81}{\meter\per\second\squared} (\SI{90}{\meter} - \SI{0}{\meter}) \\
    &= \SI{4.5}{\meter\squared\per\second\squared} + \SI{882}{\meter\squared\per\second\squared} = \SI{886.5}{\joule\per\kilo\gram}
  \end{align*}

  \begin{align*}
    \FlowRate{\Mass} &= \Density \FlowRate{\Volume} \\
                     &= \SI{1000}{\kilo\gram\per\meter\cubed} (\SI{500}{\meter\cubed\per\second}) \\
                     &= \SI{500000}{\kilo\gram\per\second}
  \end{align*}

  \begin{align*}
    \Power &= \FlowRate{\Mass} \Change{\SpecificEnergy} \\
           &= \SI{886.5}{\joule\per\kilogram} (\SI{500000}{\kilo\gram\per\second}) \\
           &= \SI{444000000}{\watt} \\
           &= \SI{444}{\mega\watt}
  \end{align*}

  \textbf{Generalize:}
  Most of the energy in this problem came from the water falling in height.
  Overall, the pressure change and the velocity of the water made very little impact on the total energy in the system, in comparison to the change in potential energy.
\end{example}

\begin{example}[Lecture 4, Problem 3.14]{Mechanical Energy of Air}
  Wind is blowing steadily at $\Velocity = \SI{10}{\meter\per\second}$.
  Determine the mechanical energy of the air per unit mass and the power generation potential of a wind turbine with $d = \SI{60}{\meter}$ diameter blades at that location?
  Take $\Density_{\mathrm{air}} = \SI{1.25}{\kilo\gram\per\meter\cubed}$.
  \tcblower{}
  \textbf{Concepts and Explore:}
  \begin{itemize}[noitemsep]
  \item The air is flowing steadily.
  \item There is no state change, so $\Change{\InternalEnergy} = 0$.
  \item There is a change in the blades' velocity, $\Change{\KineticEnergy} \neq 0$.
  \item There is no change in the potential energy of the air, so $\Change{\PotentialEnergy} = 0$.
  \item There is no pressure change on the different sides of the turbine, so $\Change{\FlowEnergy} = 0$.
  \end{itemize}

  \begin{align*}
    \Power &= \FlowRate{\Mass} \Change{\SpecificEnergy} \\
    \FlowRate{\Mass} &= \Density_{\mathrm{air}} \FlowRate{\Volume} \\
    \FlowRate{\Volume} &= \Velocity \left( \pi {\left( \frac{d}{2} \right)}^{2} \right)
  \end{align*}

  \textbf{Plan:}
  \begin{enumerate}[noitemsep]
  \item Solve for $\Change{\SpecificEnergy}$ using $\KineticEnergy$ only.
  \item Simplify $\FlowRate{\Mass}$.
  \item Solve for $\Power = \FlowRate{\Mass} \Change{\SpecificEnergy}$.
  \end{enumerate}

  \textbf{Solve:}
  \begin{align*}
    \Change{\SpecificEnergy} &= \frac{1}{2} {(\SI{10}{\meter\per\second})}^{2} \\
                             &= \SI{50}{\meter\squared\per\second\squared} \\
                             &= \SI{50}{\joule\per\kilo\gram}
  \end{align*}

  \begin{align*}
    \FlowRate{\Mass} &= \Density_{\mathrm{air}} \Velocity \left( \pi {\left( \frac{d}{2} \right)}^{2} \right) \\
                     &= \SI{1.25}{\kilo\gram\per\cubic\meter} (\SI{10}{\meter\per\second}) \left( \frac{\pi 60^{2}}{4} \right) \\
                     &= \SI{35343}{\kilo\gram\per\second}
  \end{align*}

  \begin{align*}
    \Power &= \FlowRate{\Mass} \Change{\SpecificEnergy} \\
           &= \SI{35343}{\kilo\gram\per\second} (\SI{50}{\joule\per\kilo\gram}) \\
           &= \SI{1767150}{\joule\per\second} \\
           &= \SI{1767}{\kilo\watt}
  \end{align*}

  \textbf{Validate:}
  Since we had a straightforward application of the equations, it is likely that the answers make sense.
  In addition, we did perform some dimensional analysis to figure out the way the units should be put together, which further reinforces the likelihood of this solution being the right one.

  \textbf{Generalize:}
  Like \Cref{ex:Power of Water}, the value we received only holds true when the air is flowing steadily.
  If steady flow were not happening, then the received value will fluctuate.

  Most of the power generated through this turbine is done because of the area of the blades that the air is moving through.
  This happens despite the very low density of air in this problem.

  When looking through the equations, the velocity the air is moving becomes cubed, meaning a small change in velocity will have drastic changes in the power generated.
\end{example}

\subsection{Adiabatic Processes}\label{subsec:Adiabatic_Processes}
\nameref{def:Heat} is a form of energy transfer.
However, there are two types of processes:
\begin{enumerate}[noitemsep]
\item Processes that have \textbf{no} heat transfer (\nameref{def:Adiabatic}), $\Heat = 0$.
\item Processes that \textbf{have} heat transfer (Non-\nameref{def:Adiabatic}), $\Heat \neq 0$.
\end{enumerate}

\begin{definition}[Adiabatic]\label{def:Adiabatic}
  A process is \emph{adiabatic} when there is \textbf{no} transfer of \nameref{def:Heat} whatsoever.
  This can be achieved by a system that is heavily insulated, preventing a temperature difference from causing a heat transfer.
  Adiabatic processes yield the expression below:
  \begin{equation}\label{eq:Adiabatic}
    \Heat = 0
  \end{equation}

  \begin{remark}
    An \nameref{def:Adiabatic} process does \textbf{not} mean that there cannot be work.
    A process can be both adiabatic and have work done on it, meaning there is still a change in energy in the system.
  \end{remark}
\end{definition}

\begin{definition}[1st Law of Thermodynamics]\label{def:1_Law_Thermo}
  The \emph{1st law of thermodynamics} states that the total energy of a system \textbf{cannot} be created or destroyed during a \nameref{def:Process}; it can only change \textbf{forms}.

  Symbolically, this is represented as \Cref{eq:1_Law_Thermo}.
  \begin{equation}\label{eq:1_Law_Thermo}
    \Change{\Energy} = 0
  \end{equation}
\end{definition}

%%% Local Variables:
%%% mode: latex
%%% TeX-master: "../MMAE_320-Thermo-Reference_Sheet"
%%% End:
