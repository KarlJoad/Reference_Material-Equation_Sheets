\subsection{Processes}\label{subsec:Processes}
\begin{definition}[Process]\label{def:Process}
  A \emph{process} is the change in a \nameref{def:System} from one \nameref{def:Equilibrium} state to another.
  The \textbf{path} of the process is the set of \nameref{def:Quasi-equilibrium} processes.

  It is easiest to deal with paths where if one of the variables is changed just a small amount, the change becomes linear.
  This is similar to the concept of a derivative, because if we sample at a small/short enough rate, the changes introduced to the \nameref{def:System} will be linear.
\end{definition}

\begin{definition}[Quasi-equilibrium]\label{def:Quasi-equilibrium}
  A \emph{quasi-equilibrium} process is one that is sufficiently slow so that the \nameref{def:System} can adjust itself internally so that properties change constantly \textbf{throughout} the system.
\end{definition}

Lastly, if the \nameref{def:Process} is a \nameref{def:Cycle}, then the initial and final \nameref{def:State}s are the same.
\begin{definition}[Cycle]\label{def:Cycle}
  A \emph{cycle} is a \nameref{def:Process} that returns to its initital state at the end of the process.
\end{definition}


%%% Local Variables:
%%% mode: latex
%%% TeX-master: "../../MMAE_320-Thermo-Reference_Sheet"
%%% End:
