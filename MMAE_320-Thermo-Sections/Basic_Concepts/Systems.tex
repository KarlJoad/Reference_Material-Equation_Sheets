\subsection{Systems}\label{subsec:Systems}
\begin{definition}[System]\label{def:System}
  A \emph{system} is defined as a quantity of matter or a region in space chosen for study.
\end{definition}

The mass or region outside the \nameref{def:System} is the \textbf{surroundings}.
The surface that separates the \nameref{def:System} from the surroundings is the \textbf{boundary}.

\subsubsection{Type of Systems}\label{subsubsec:Types_Systems}
There are 2 types of systems:
\begin{enumerate}[noitemsep]
\item \nameref{def:Open_System}
\item \nameref{def:Closed_System}
\end{enumerate}

\begin{definition}[Open System]\label{def:Open_System}
  An \emph{open system} is a \nameref{def:System} in which the mass of the system is \textbf{not} constant.
  Thus, mass and \nameref{def:Energy} can flow from the system to the surroundings.
  Energy can flow by either \nameref{def:Work}, or it can be \nameref{def:Heat}.

  \begin{remark}[Control Volume]\label{rmk:Control_Volume}
    Sometimes an \nameref{def:Open_System} is called a \emph{control volume}, because the volume of the system is constant.
  \end{remark}

  \begin{remark}
    Flow through these devices is typically easier by selecting the region based on volume, rather than mass.
  \end{remark}
\end{definition}

\begin{definition}[Closed System]\label{def:Closed_System}
  A \emph{closed system} is a \nameref{def:System} where the mass of the system \textbf{is} constant, but \nameref{def:Energy} can move between the system and the surroundings.
  This type of system can either be \nameref{def:Adiabatic} or non-adiabatic.
  For each case:
  \begin{description}[noitemsep]
  \item[\nameref{def:Adiabatic}] There is an \nameref{def:Energy} transfer by \nameref{def:Heat} \textbf{and} \nameref{def:Work}.
  \item[Non-adiabatic] The energy transfer is done by \nameref{def:Work} \textbf{only}.
  \end{description}

  \begin{remark}[Control Mass]\label{rmk:Control_Mass}
    Sometimes a \nameref{def:Closed_System} is called a \emph{control mass}, because mass is constant.
  \end{remark}
\end{definition}

There is also a third type of system, the \nameref{def:Isolated_System}, which is a special case of the \nameref{def:Closed_System}.
\begin{definition}[Isolated System]\label{def:Isolated_System}
  An \emph{isolated system} is a \nameref{def:System} where \textbf{neither} mass or \nameref{def:Energy} can move from the system to the surroundings.
  This system has the properties of being \nameref{def:Adiabatic} \textbf{and} there is no \nameref{def:Work} done.
\end{definition}

\subsubsection{Properties of Systems}\label{subsubsec:Properties_Systems}
\begin{definition}[Property]\label{def:Property}
  A \emph{property} of a \nameref{def:System} is a characteristic of the system.
\end{definition}

Some common properties are:
\begin{itemize}[noitemsep]
\item $\Pressure$, Pressure
\item $\Temp$, Temperature
\item $\Volume$, Volume
\item $\Mass$, Mass
\end{itemize}

There are 2 types of properties:
\begin{enumerate}[noitemsep]
\item \nameref{def:Intensive_Property}
\item \nameref{def:Extensive_Property}
\end{enumerate}

\begin{definition}[Intensive Property]\label{def:Intensive_Property}
  \emph{Intensive properties} are properties that are \textbf{independent} of the mass of the system.
  These include:
  \begin{itemize}[noitemsep]
  \item $\Pressure$, Pressure
  \item $\Temp$, Temperature
  \item $\Density$, Density
  \end{itemize}
\end{definition}

\begin{definition}[Extensive Property]\label{def:Extensive_Property}
  \emph{Extensive properties} are properties that are \textbf{dependent} on the size or extent of the system.
  These include:
  \begin{itemize}[noitemsep]
  \item Total mass
  \item Total volume
  \item Total momentum
  \end{itemize}
\end{definition}

Some extensive properties can be used to form a \nameref{def:Specific_Property}.
\begin{definition}[Specific Property]\label{def:Specific_Property}
  A \emph{specific property} is an \nameref{def:Extensive_Property} per unit mass.
  Some examples are:
  \begin{itemize}[noitemsep]
  \item Specific Volume $\SpecificVolume = \frac{\Volume}{m}$
  \item Specific Total Energy $\SpecificEnergy = \frac{E}{m}$
  \end{itemize}
\end{definition}

\subsubsection{Generalizations}\label{subsubsec:Generalizations}
We make some generalizations about the fluids we are working with to make calculations easier.
For example, we treat fluids as a \textbf{continuum}, rather than the true atomic nature of the substance.
This allows us to:
\begin{itemize}[noitemsep]
\item Treat properties as point functions
\item Assume properties vary continually in space with no discontinuities
\end{itemize}

This assumption is valid so long as the size of the \nameref{def:System} is relatively large compared to the space between the component molecules.

%%% Local Variables:
%%% mode: latex
%%% TeX-master: "../../MMAE_320-Thermo-Reference_Sheet"
%%% End:
