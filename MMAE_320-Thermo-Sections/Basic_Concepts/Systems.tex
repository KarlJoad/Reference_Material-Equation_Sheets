\subsection{Systems}\label{subsec:Systems}
\begin{definition}[System]\label{def:System}
  A \emph{system} is defined as a quantity of matter or a region in space chosen for study.
\end{definition}

The mass or region outside the \nameref{def:System} is the \textbf{surroundings}.
The surface that separates the \nameref{def:System} from the surroundings is the \textbf{boundary}.

\subsubsection{Type of Systems}\label{subsubsec:Types_Systems}
There are 2 types of systems:
\begin{enumerate}[noitemsep]
\item \nameref{def:Open_System}
\item \nameref{def:Closed_System}
\end{enumerate}

\begin{definition}[Open System]\label{def:Open_System}
  An \emph{open system} is a \nameref{def:System} in which the mass of the system is \textbf{not} constant.
  Thus, mass and \nameref{def:Energy} can flow from the system to the surroundings.

  \begin{remark}[Control Volume]\label{rmk:Control_Volume}
    Sometimes an \nameref{def:Open_System} is called a \emph{control volume}, because the volume of the system is constant.
  \end{remark}

  \begin{remark}
    Flow through these devices is typically easier by selecting the region based on volume, rather than mass.
  \end{remark}
\end{definition}


%%% Local Variables:
%%% mode: latex
%%% TeX-master: "../../MMAE_320-Thermo-Reference_Sheet"
%%% End:
