\subsection{Pressure}\label{subsec:Pressure}
\begin{definition}[Pressure]\label{def:Pressure}
  \emph{Pressure} is the normal \textbf{scalar} force exerted per unit area.
  Because it is a scalar, it has no dependency on the direction of the normal force.
  It can be expressed using \nameref{def:Density}, gravity, and depth/length.
  \begin{equation}\label{eq:Pressure}
    \Pressure = \Density \Gravity d
  \end{equation}

  The SI unit of pressure is the pascal:
  \begin{equation}\label{eq:Pressure_Pascal}
    \si{\pascal} = \si[per-mode=fraction]{\newton\per\meter\squared}
  \end{equation}

  The bar is also used:
  \begin{equation*}
    \SI{1}{\bar} = \SI{100}{\kilo\pascal}
  \end{equation*}

  The English unit of pressure is the pound-force per square inch.
  \begin{equation}\label{eq:Pressure_Bar}
    \si{\lbf\per\inch\squared} = \si{\psi}
  \end{equation}
\end{definition}

We exist, roughly, at one atmosphere of pressure, or $\SI{1}{\atm}$.
The equivalents of this are:
\begin{align*}
  \SI{1}{\atm} &= \SI{101325}{\kilo\pascal} \\
  \SI{1}{\atm} &= \SI{14.696}{\psi}
\end{align*}

Using this, we can measure pressure, typically with either \nameref{def:Gage_Pressure} or \nameref{def:Vacuum_Pressure}.
\begin{definition}[Gage Pressure]\label{def:Gage_Pressure}
  \emph{Gage pressure} is the \nameref{def:Pressure} that a gage reads after being calibrated to some standard pressure, typically one atmosphere.
  Thus, the actual pressure of something read by the gage is represented by \Cref{eq:Gage_Pressure} below.
  \begin{equation}\label{eq:Gage_Pressure}
    \Pressure_{\mathrm{abs}} = \Pressure_{\mathrm{atm}} + \Pressure_{\mathrm{gage}}
  \end{equation}
\end{definition}

\begin{definition}[Vacuum Pressure]\label{def:Vacuum_Pressure}
  \emph{Vacuum pressure} is the \nameref{def:Pressure} that a gage reads after being calibrated to some standard pressure, typically one atmosphere.
  Thus, the actual pressure of something read by the gage is represented by \Cref{eq:Vacuum_Pressure} below.
  \begin{equation}\label{eq:Vacuum_Pressure}
    \Pressure_{\mathrm{abs}} = \Pressure_{\mathrm{atm}} - \Pressure_{\mathrm{vacuum}}
  \end{equation}
\end{definition}

\subsubsection{Pressure Inside a Fluid}\label{subsubsec:Pressure_Inside_Fluid}
Pressure inside a fluid increases linearly with depth.
This is seen in \Cref{eq:Pressure_in_Fluid}, where $\Delta z$ is the distance between the two measured points.

\begin{equation}\label{eq:Pressure_in_Fluid}
  \begin{aligned}
    \Pressure_{\mathrm{below}} &= \Pressure_{\mathrm{above}} + \Density \Gravity \lvert \Delta z \rvert \\
    &= \Pressure_{\mathrm{above}} + \SpecificWeight \lvert \Delta z \rvert \\
  \end{aligned}
\end{equation}

%%% Local Variables:
%%% mode: latex
%%% TeX-master: "../../MMAE_320-Thermo-Reference_Sheet"
%%% End:
