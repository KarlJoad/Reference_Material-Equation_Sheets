\subsection{Pressure}\label{subsec:Pressure}
\begin{definition}[Pressure]\label{def:Pressure}
  \emph{Pressure} is the normal \textbf{scalar} force exerted per unit area.
  Because it is a scalar, it has no dependency on the direction of the normal force.
  It can be expressed using \nameref{def:Density}, gravity, and depth/length.
  \begin{equation}\label{eq:Pressure}
    \Pressure = \Density \Gravity d
  \end{equation}

  The SI unit of pressure is the pascal:
  \begin{equation}\label{eq:Pressure_Pascal}
    \si{\pascal} = \si[per-mode=fraction]{\newton\per\meter\squared}
  \end{equation}

  The bar is also used:
  \begin{equation*}
    \SI{1}{\bar} = \SI{100}{\kilo\pascal}
  \end{equation*}

  The English unit of pressure is the pound-force per square inch.
  \begin{equation}\label{eq:Pressure_Bar}
    \si[per-mode=symbol]{\lbf\per\inch\squared} = \si{\psi}
  \end{equation}
\end{definition}


%%% Local Variables:
%%% mode: latex
%%% TeX-master: "../../MMAE_320-Thermo-Reference_Sheet"
%%% End:
