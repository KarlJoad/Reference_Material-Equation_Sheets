\subsection{Temperature}\label{subsec:Temperature}
\begin{definition}[0th Law of Thermodynamics]\label{def:0_Law_Thermo}
  The \emph{0th law of thermodynamics} states that if two bodies are in thermal \nameref{def:Equilibrium} with a third body, they are also in thermal equilibrium with each other.
\end{definition}

\subsubsection{Subjective Temperature Scales}\label{subsubsec:Subjective_Temp_Scales}
There are 2 subjective temperature scales:
\begin{enumerate}[noitemsep]
\item Celsius (\si{\degreeCelsius})
\item Fahrenheit (\si{\degreeF})
\end{enumerate}

These are based off of physical temperatures at certain points on Earth regarding water freezing and water boiling.
These are useful for regular use, but not as well-suited for thermodynamic use.

\begin{equation}\label{eq:Fahrenheit_Celsius}
  \frac{5}{9} (\Temp(\si{\degreeF}) - 32) = \Temp(\si{\degreeCelsius})
\end{equation}

\subsubsection{Objective Temperature Scales}\label{subsubsec:Objective_Temp_Scales}
Objective temperature scales are ones that are based off of physical, universal constants.
There are 2 objective temperature scales:
\begin{enumerate}[noitemsep]
\item Kelvin (\si{\kelvin})
\item Rankine (\si{\rankine})
\end{enumerate}

Both of these have placed $0$ at the point where all molecular motion stops.

The equations to convertion between Celsius and Kelvin is shown in \Cref{eq:Celsius_Kelvin}.
\begin{equation}\label{eq:Celsius_Kelvin}
  \Temp(\si{\kelvin{}}) = \Temp(\si{\degreeCelsius}) + 273.15
\end{equation}

The equations to convertion between Fahrenheit and Rankine is shown in \Cref{eq:Fahrenheit_Rankine}.
\begin{equation}\label{eq:Fahrenheit_Rankine}
  \Temp(\si{\rankine{}}) = \Temp(\si{\degreeF{}}) + 459.67
\end{equation}

%%% Local Variables:
%%% mode: latex
%%% TeX-master: "../../MMAE_320-Thermo-Reference_Sheet"
%%% End:
