\subsection{Density}\label{subsec:Density}
\begin{definition}[Density]\label{def:Density}
  \emph{Density} is defined as mass per unit volume.

  \begin{equation}\label{eq:Density}
    \Density = \frac{\Mass}{\Volume} \:\: \si{\kilo\gram\per\meter\cubed}
  \end{equation}
\end{definition}

\begin{definition}[Specific Volume]\label{def:Specific_Volume}
  \emph{Specific volume} is the reciprocal of \nameref{def:Density}, and is the amount of volume per unit mass.

  \begin{equation}\label{eq:Specific_Volume}
    \SpecificVolume = \frac{\Volume}{\Mass} = \frac{1}{\Density}
  \end{equation}
\end{definition}

For substances that lack a uniform mass and volume, density can also be realized by \Cref{eq:Differential_Density}.
\begin{equation}\label{eq:Differential_Density}
  \Density = \frac{d \Mass}{d \Volume}
\end{equation}

Sometimes the density of a substance is given relative to another substance, usually water.
This is \nameref{def:Specific_Gravity}.
\begin{definition}[Specific Gravity]\label{def:Specific_Gravity}
  \emph{Specific gravity} is the \nameref{def:Density} of a substance as a ratio to another substance, usually water.
  This is expressed as
  \begin{equation}\label{eq:Specific_Gravity}
    \SpecificGravity = \frac{\Density}{\Density_{\mathrm{H_{2}O}}}
  \end{equation}
\end{definition}

\begin{definition}[Specific Weight]\label{def:Specific_Weight}
  \emph{Specific weight} is the weight of a unit volume of a substance.
  It is expressed as:
  \begin{equation}\label{eq:Specific_Weight}
    \SpecificWeight = \Density \Gravity \:\: \si{\newton\per\meter\cubed}
  \end{equation}
\end{definition}

%%% Local Variables:
%%% mode: latex
%%% TeX-master: "../../MMAE_320-Thermo-Reference_Sheet"
%%% End:
