\subsection{States}\label{subsec:States}
\begin{definition}[State]\label{def:State}
  The \emph{state} of the system includes all properties that can be measured or calculated which can completely describe the condition of the system.
\end{definition}

In any given \nameref{def:State}, all the properties of a system will have fixed values.
If any single property changes, all the others will change in accordance.
We are typically interested in \nameref{def:Equilibrium} states.

\begin{definition}[Equilibrium]\label{def:Equilibrium}
  \emph{Equilibrium} is a \nameref{def:State} of a \nameref{def:System} which does not change when isolated from it surroundings.
  A system will only leave equilibrium when disturbed by an outside \nameref{def:Energy}, or an unbalanced \nameref{def:Driving_Force}.
\end{definition}

There are several different types of \nameref{def:Equilibrium}:
\begin{description}[noitemsep]
\item[Thermal Equilibrium] If the temperature throughout the entire \nameref{def:System} is constant.
\item[Mechanical Equilibrium] If there is no change in pressure at any point of the \nameref{def:System} with regards to time.
  However, pressure can vary within the system, but for most of our concerns, this isn't a problem.
\item[Phase Equilibrium] If there is more than one phase of matter, this type of equilibrium is reached with each phase reaches its equilibrium level and stays there.
\item[Chemical Equilibrium] If the chemical composition of a \nameref{def:System} does not change with time, i.e.\ no chemical reactions could occur.
\end{description}


%%% Local Variables:
%%% mode: latex
%%% TeX-master: "../../MMAE_320-Thermo-Reference_Sheet"
%%% End:
