\section{The Second Law of Thermodynamics}\label{sec:2nd_Law_Thermo}
\begin{definition}[\nth{2} Law of Thermodynamics]\label{def:2nd_Law_Thermo}
  The \emph{\nth{2} law of thermodynamics} states that the total \nameref{def:Entropy} of an \nameref{def:Isolated_System} can \textbf{never} decrease over time, and is constant if and only if all \nameref{def:Process}es are reversible.
  Isolated systems spontaneously evolve towards thermodynamic equilibrium, the state with maximum entropy.

  In terms of total \nameref{def:Energy} in a system, this law states that the total energy of the system is minimized at all times during a \nameref{def:Process}.

  \begin{remark}
    We do not discuss \nameref{def:Entropy} in this section.
    Rather, we discuss it in great detail in \Cref{sec:Entropy}.
  \end{remark}
\end{definition}

In more common terms, the \nameref{def:2nd_Law_Thermo} explains why some \nameref{def:Process}es only move in one direction.
For example, water going over a waterfall is \textit{technically} symmetric.
Meaning, that there is a chance the water will actually go back \textbf{up} the cliff.
However, the \nameref{def:2nd_Law_Thermo} explains that this reversal is so thermodynamically unfavorable that it will never happen.

Another example is a container of water that is greater than that of its surroundings.
For instance, the water is $\Temp = \SI{90}{\degreeF}$ and the surroundings are at $\Temp = \SI{20}{\degreeF}$.
The \nameref{def:2nd_Law_Thermo} explains why the heat always flows \textbf{out} of the water and to the surroundings, rather than the other way around.

This means that \textbf{energy is dispersed}.
This spread (dispersion) of energy is \nameref{def:Entropy}.

\begin{blackbox}
  \nameref{def:Energy} naturally flows from being concentrated in one place to another such that the dispersal of energy is maximized.
  This limits the \nameref{def:Work} that can be performed.
\end{blackbox}

%%% Local Variables:
%%% mode: latex
%%% TeX-master: "../MMAE_320-Thermo-Reference_Sheet"
%%% End:
