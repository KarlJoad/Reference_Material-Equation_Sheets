\section{The Second Law of Thermodynamics}\label{sec:2nd_Law_Thermo}
\begin{definition}[\nth{2} Law of Thermodynamics]\label{def:2nd_Law_Thermo}
  The \emph{\nth{2} law of thermodynamics} states that the total \nameref{def:Entropy} of an \nameref{def:Isolated_System} can \textbf{never} decrease over time, and is constant if and only if all \nameref{def:Process}es are reversible.
  Isolated systems spontaneously evolve towards thermodynamic equilibrium, the state with maximum entropy.

  In terms of total \nameref{def:Energy} in a system, this law states that the total energy of the system is minimized at all times during a \nameref{def:Process}.

  \begin{remark}
    We do not discuss \nameref{def:Entropy} in this section.
    Rather, we discuss it in great detail in \Cref{sec:Entropy}.
  \end{remark}
\end{definition}

In more common terms, the \nameref{def:2nd_Law_Thermo} explains why some \nameref{def:Process}es only move in one direction.
For example, water going over a waterfall is \textit{technically} symmetric.
Meaning, that there is a chance the water will actually go back \textbf{up} the cliff.
However, the \nameref{def:2nd_Law_Thermo} explains that this reversal is so thermodynamically unfavorable that it will never happen.

Another example is a container of water that is greater than that of its surroundings.
For instance, the water is $\Temp = \SI{90}{\degreeF}$ and the surroundings are at $\Temp = \SI{20}{\degreeF}$.
The \nameref{def:2nd_Law_Thermo} explains why the heat always flows \textbf{out} of the water and to the surroundings, rather than the other way around.

This means that \textbf{energy is dispersed}.
This spread (dispersion) of energy is \nameref{def:Entropy}.

\begin{blackbox}
  \nameref{def:Energy} naturally flows from being concentrated in one place to another such that the dispersal of energy is maximized.
  This limits the \nameref{def:Work} that can be performed.
\end{blackbox}

\begin{definition}[Kelvin-Planck \nth{2} Law of Thermodynamics]\label{def:Kelvin_Planck-2nd_Law_Thermo}
  There are a variety of definitions of the \nth{2} law of thermodynamics.
  The one in \Cref{def:2nd_Law_Thermo} is just one of them.
  The \emph{Kelvin-Planck \nth{2} law of thermodynamics} states that it is impossible for any device that operates on a \nameref{def:Cycle} to receive heat form a single reservoir and use to to produce \nameref{def:Work}.

  \begin{remark}[Thermal Reservoir]\label{rmk:Thermal_Reservoir}
    A thermal source with a constant temperature.
    For example, the Earth, a large boiler, the air, a lake, or river are considered thermal reservoirs.
    More concretely, this means that if \nameref{def:Heat} is removed, there is no temperature change.
  \end{remark}
\end{definition}

\begin{definition}[Cycle]\label{def:Cycle}
  A \emph{cycle} is a \nameref{def:Process} whose ending state is the same as its starting state, allowing the process to continue again.

  \begin{remark}[Cyclic]\label{rmk:Cyclic}
    A \emph{cyclic} \nameref{def:Process} is one that behaves as a \nameref{def:Cycle}.
  \end{remark}
\end{definition}

\subsection{Thermodynamic Cycles}\label{subsec:Thermodynamic_Cycles}
Cycles absorb work, limiting the amount of energy that can be extracted.
You cannot perfectly convert \nameref{def:Heat} into \nameref{def:Work}.
This is because the cycle itself requires some amount of work to even occur.

%%% Local Variables:
%%% mode: latex
%%% TeX-master: "../MMAE_320-Thermo-Reference_Sheet"
%%% End:
