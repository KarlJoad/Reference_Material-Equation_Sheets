\documentclass[10pt,letterpaper,final,twoside,notitlepage]{article}
\usepackage[margin=.5in]{geometry}
\usepackage[utf8]{inputenc}
\usepackage[english]{babel}
\usepackage{amsmath}
\usepackage{amsfonts}
\usepackage{amssymb}
\usepackage{amsthm} % Gives us plain, definition, and remark to use in \theoremstyle{style}
\usepackage{graphicx}

\usepackage{hyperref} % Generate hyperlinks to referenced items
\usepackage{nameref} % Can make references by name to places
\usepackage{ctable} % Greater control over tables and how they look
\usepackage{subcaption} % Allows for multiple figures in one Figure environment
\usepackage{textcomp} % Forcibly loads a package that gensymb relies on
\usepackage{gensymb} % Gives access to some characters, for example, the degree symbol
\usepackage{enumitem} % Provides [noitemsep, nolistsep] for more compact lists
\usepackage{chngcntr} % Allows us to tamper with the counter a little more
\usepackage{empheq} % Allow boxing of equations in special math environments

%\graphicspath{{./Drawings/Class_Name_and_Number}} % Uncomment this to use pictures in this document
%\numberwithin{equation}{section} % Uncomment this to number equations with section numbers too

\theoremstyle{plain}
\newtheorem{theorem}{Theorem}

\theoremstyle{definition}
\newtheorem{definition}{Defn}
\newtheorem{corollary}{Corollary}[section]

\theoremstyle{remark}
\newtheorem{remark}{Remark}[definition]
\newtheorem*{remark*}{Remark}
%\counterwithin{definition}{subsection} % Uncomment to have definitions use section.subsection numbering

\newcounter{example}[section]
\newenvironment{example} % Define an example environment
	{\refstepcounter{example} \begin{center}\begin{tabular}{|p{0.9\textwidth}|} \hline Example~\theexample.  \\ }
	{\\ \hline \end{tabular} \end{center}}

\renewcommand\qedsymbol{$\blacksquare$} % Change proofs to have black square at end

\DeclareMathOperator{\RealNums}{\mathbb{R}}

\author{Karl Hallsby}
\title{Math 251 Differential Equations}

\begin{document}
\section{Introduction} \label{sec:Introduction}
	\begin{example}
		Test example.
	\end{example}
	\subsection{Definitions and Terminology} \label{subsec:Definitions and Terminology}
		\begin{definition}[Differential Equation] \label{def:Differential Equation}
			A \emph{differential equation (DE)} is an equation with 1 or more derivatives.
			\begin{remark}
				The highest differential determines the order of the differential equation.
				This means that the differential equation below is of order 2.
				\begin{align*} 
					y'' + y &= 0 \\
					\frac{d^{2}y}{dx^{2}} + y &= 0 \\
				\end{align*} 
			\end{remark}
		\end{definition}
		\begin{definition}[Initial Value Problem] \label{def:Initial Value Problem}
			A differential equation with one or more initial conditions is called an \emph{initial value problem (IVP)}.
			\begin{remark}
				To solve an initial value problem, you must have the same number of initial conditions as the order of the differential equation.
			\end{remark}
			\begin{remark}[Existence of Unique Solution]
				$R$ is a rectangular region on the xy-plane $a \leq x \leq b$, $c \leq y \leq d$ that contains $\left( x_{0}, y_{0} \right)$ interior.
				If $f \left( x,y \right)$ and $\frac{df}{dy}$ are continuous on $R$, then an interval exists $I_{0}$ such that $\left( x_{0}-h, x_{0}+h \right)$ where $h>0$, on the interval $\left[ a,b \right]$, and a unique function $y \left( x \right)$, defined on $I_{0}$ that is a solution of the initial value problem.
			\end{remark}
		\end{definition}
	
	\subsection{Separable Differential Equation}
		\begin{definition}[Separable]
			A \emph{separable} differential equation allows you to move various elements around to solve the equation.
			For example,
			\begin{align*}
				\frac{dP}{dt} &= kP \\
				\frac{1}{P} dP &= k dt \\
				\ln \left( P \right) &= kt + C \\
				P &= Ce^{kt}
			\end{align*}
			\begin{remark}
				These are used extensively in modelling phenomena with differential equations.
				These include: \nameref{subsubsec:Population Growth}, \nameref{subsubsec:Radioactive Decay}, \nameref{subsubsec:Newton Law of Cooling/Heating}, and \nameref{subsubsec:Spread of Disease}.
			\end{remark}
		\end{definition}
		
	\subsection{Modeling with Differential Equations} \label{subsec:Modeling with DEs}
		\subsubsection{Population Growth} \label{subsubsec:Population Growth}
			\begin{definition}[Population Growth]
				\emph{Population growth} can be modelled with a separable differential equation. Namely,
				\begin{equation} \label{eq:Population Growth}
					\frac{dP}{dt} = kP
				\end{equation}
				\begin{remark}[Population Growth Parameters] \label{rmk:Population Growth Parameters}
					The parameters for the \nameref{eq:Population Growth}~equation are given below.
					\begin{itemize}[noitemsep, nolistsep]
						\item $k>0$
						\item $P>0$
					\end{itemize}
				\end{remark}
			\end{definition}
		\subsubsection{Radioactive Decay} \label{subsubsec:Radioactive Decay}
			\begin{definition}[Radioactive Decay] \label{def:Radioactive Decay}
				\emph{Radioactive decay} is the process that some particularly heave atoms undergo.
				\begin{definition}[Half-Life] \label{subdef:Half-Life}
					The \emph{half-life} is the usual reported metric, and is defined as the amount of time required for an element to half its mass through \nameref{def:Radioactive Decay}.
				\end{definition}
				\begin{equation} \label{eq:Radioactive Decay}
					\frac{1}{2} A_{0} = A_{0} e^{kt}
				\end{equation}
				\begin{remark}[Radioactive Decay Parameters] \label{rmk:Radioactive Decay Parameters}
					The parameters for the \nameref{eq:Radioactive Decay}~equation are given below.
					\begin{itemize}[noitemsep, nolistsep]
						\item $k<0$
						\item $A>0$
					\end{itemize}
				\end{remark}
			\end{definition}
		\subsubsection{Newton's Law of Cooling/Heating} \label{subsubsec:Newton Law of Cooling/Heating}
			\begin{definition}[Newton's Law of Cooling/Heating] \label{def:Newton Law of Cooling/Heating}
				\emph{Newton's Law of Cooling/Heating} is the same equation, but some of the parameters change.
				This equation is defined as:
				\begin{equation} \label{eq:Newton Law of Cooling/Heating}
					\frac{dT}{dt} = k \left( T-T_{m} \right)
				\end{equation}
				\begin{remark} \label{rmk:Newton Law of Cooling/Heating Parameters}
					The parameters for the \nameref{eq:Newton Law of Cooling/Heating}~equation are given below.
					\begin{itemize}[noitemsep, nolistsep]
						\item $\frac{dT}{dt}$; The rate of change of temperature in the object per unit time.
						\item $k<0$; The cooling constant and is unique to every object.
						\item $T$; The starting temperature.
						\item $T_{m}$; The temperature of the surrounding medium.
					\end{itemize}
				\end{remark}
			\end{definition}
		\subsubsection{Spread of Disease} \label{subsubsec:Spread of Disease}
			\begin{definition}[Spread of Disease] \label{def:Spread of Disease}
				This is used to model the spread of something throughout a society or group of people.
				\begin{equation} \label{eq:Spread of Disease}
					\frac{dx}{dt} = kxy
				\end{equation}
				\begin{remark}
					The parameters for the \nameref{eq:Spread of Disease}~equation are given below.
					\begin{itemize}[noitemsep, nolistsep]
						\item $\frac{dx}{dt}$; Change in the number of infected per unit time.
						\item $k<0$; Transmission Constant
						\item$x$; Number of Infected
						\item $y$; Number of non-infected, $y$ is really a function of $x$
						\begin{itemize}[noitemsep, nolistsep]
							\item $y = n+1-x$
						\end{itemize}
					\end{itemize}
				\end{remark}
			\end{definition}

%====================================APPENDIX====================================
\clearpage
\appendix
\counterwithin{equation}{section}

\section{Reference Material} \label{sec:Reference Material}
\subsection{Trigonometry} \label{app:Trig}
	\subsubsection{Trigonometric Formulas} \label{subsubsec:Trig Formulas}
		\begin{equation} \label{eq:Sin plus Sin with diff Angles}
			\sin \left( \alpha \right) + \sin \left( \beta \right) = 2 \sin \left( \frac{\alpha + \beta}{2} \right) \cos\left( \frac{\alpha - \beta}{2} \right)  
		\end{equation}
		\begin{equation} \label{eq:Cosine-Sine Product}
			\cos \left( \theta \right) \sin \left( \theta \right) = \frac{1}{2} \sin \left( 2 \theta \right)
		\end{equation}
	
	\subsubsection{Euler Equivalents of Trigonometric Functions} \label{subsubsec:Euler Equivalents}
		\begin{equation} \label{eq:Euler Sin}
			\sin \left( x \right) = \frac{e^{\imath x} + e^{-\imath x}}{2}
		\end{equation}
		\begin{equation} \label{eq:Euler Cos}
			\cos \left( x \right) = \frac{e^{\imath x} - e^{-\imath x}}{2 \imath}
		\end{equation}
		\begin{equation} \label{eq:Euler Sinh}
			\sinh \left( x \right) = \frac{e^{x} - e^{-x}}{2}
		\end{equation}
		\begin{equation} \label{eq:Euler Cosh}
			\cosh \left( x \right) = \frac{e^{x} + e^{-x}}{2}
		\end{equation}
\section{Calculus}\label{app:Calculus}
\subsection{L'Hopital's Rule}\label{subsec:LHopitals_Rule}
L'Hopital's Rule can be used to simplify and solve expressions regarding limits that yield irreconcialable results.
\begin{lemma}[L'Hopital's Rule]\label{lemma:LHopitals_Rule}
  If the equation
  \begin{equation*}
    \lim\limits_{x \rightarrow a} \frac{f(x)}{g(x)} =
    \begin{cases}
      \frac{0}{0} \\
      \frac{\infty}{\infty} \\
    \end{cases}
  \end{equation*}
  then \Cref{eq:LHopitals_Rule} holds.
  \begin{equation}\label{eq:LHopitals_Rule}
    \lim\limits_{x \rightarrow a} \frac{f(x)}{g(x)} = \lim\limits_{x \rightarrow a} \frac{f'(x)}{g'(x)}
  \end{equation}
\end{lemma}

\subsection{Fundamental Theorems of Calculus}\label{subsec:Fundamental Theorem of Calculus}
\begin{definition}[First Fundamental Theorem of Calculus]\label{def:1st Fundamental Theorem of Calculus}
  The \emph{first fundamental theorem of calculus} states that, if $f$ is continuous on the closed interval $\left[ a,b \right]$ and $F$ is the indefinite integral of $f$ on $\left[ a,b \right]$, then

  \begin{equation}\label{eq:1st Fundamental Theorem of Calculus}
    \int_{a}^{b}f \left( x \right) dx = F \left( b \right) - F \left( a \right)
  \end{equation}
\end{definition}

\begin{definition}[Second Fundamental Theorem of Calculus]\label{def:2nd Fundamental Theorem of Calculus}
  The \emph{second fundamental theorem of calculus} holds for $f$ a continuous function on an open interval $I$ and $a$ any point in $I$, and states that if $F$ is defined by

  \begin{equation*}
    F \left( x \right) = \int_{a}^{x} f \left( t \right) dt,
  \end{equation*}
  then
  \begin{equation}\label{eq:2nd Fundamental Theorem of Calculus}
    \begin{aligned}
      \frac{d}{dx} \int_{a}^{x} f \left( t \right) dt &= f \left( x \right) \\
      F' \left( x \right) &= f \left( x \right) \\
    \end{aligned}
  \end{equation}
\end{definition}

\begin{definition}[argmax]\label{def:argmax}
  The arguments to the \emph{argmax} function are to be maximized by using their derivatives.
  You must take the derivative of the function, find critical points, then determine if that critical point is a global maxima.
  This is denoted as
  \begin{equation*}\label{eq:argmax}
    \argmax_{x}
  \end{equation*}
\end{definition}

\subsection{Rules of Calculus}\label{subsec:Rules of Calculus}
\subsubsection{Chain Rule}\label{subsubsec:Chain Rule}
\begin{definition}[Chain Rule]\label{def:Chain Rule}
  The \emph{chain rule} is a way to differentiate a function that has 2 functions multiplied together.

  If
  \begin{equation*}
    f(x) = g(x) \cdot h(x)
  \end{equation*}
  then,
  \begin{equation}\label{eq:Chain Rule}
    \begin{aligned}
      f'(x) &= g'(x) \cdot h(x) + g(x) \cdot h'(x) \\
      \frac{df(x)}{dx} &= \frac{dg(x)}{dx} \cdot g(x) + g(x) \cdot \frac{dh(x)}{dx} \\
    \end{aligned}
  \end{equation}
\end{definition}

\subsection{Useful Integrals}\label{subsec:Useful_Integrals}
\begin{equation}\label{eq:Cosine_Indefinite_Integral}
  \int \cos(x) \; dx = \sin(x)
\end{equation}

\begin{equation}\label{eq:Sine_Indefinite_Integral}
  \int \sin(x) \; dx = -\cos(x)
\end{equation}

\begin{equation}\label{eq:x_Cosine_Indefinite_Integral}
  \int x \cos(x) \; dx = \cos(x) + x \sin(x)
\end{equation}
\Cref{eq:x_Cosine_Indefinite_Integral} simplified with Integration by Parts.

\begin{equation}\label{eq:x_Sine_Indefinite_Integral}
  \int x \sin(x) \; dx = \sin(x) - x \cos(x)
\end{equation}
\Cref{eq:x_Sine_Indefinite_Integral} simplified with Integration by Parts.

\begin{equation}\label{eq:x_Squared_Cosine_Indefinite_Integral}
  \int x^{2} \cos(x) \; dx = 2x \cos(x) + (x^{2} - 2) \sin(x)
\end{equation}
\Cref{eq:x_Squared_Cosine_Indefinite_Integral} simplified by using Integration by Parts twice.

\begin{equation}\label{eq:x_Squared_Sine_Indefinite_Integral}
  \int x^{2} \sin(x) \; dx = 2x \sin(x) - (x^{2} - 2) \cos(x)
\end{equation}
\Cref{eq:x_Squared_Sine_Indefinite_Integral} simplified by using Integration by Parts twice.

\begin{equation}\label{eq:Exponential_Cosine_Indefinite_Integral}
  \int e^{\alpha x} \cos(\beta x) \; dx = \frac{e^{\alpha x} \bigl( \alpha \cos(\beta x) + \beta \sin(\beta x) \bigr)}{\alpha^{2} + \beta^{2}} + C
\end{equation}

\begin{equation}\label{eq:Exponential_Sine_Indefinite_Integral}
  \int e^{\alpha x} \sin(\beta x) \; dx = \frac{e^{\alpha x} \bigl( \alpha \sin(\beta x) - \beta \cos(\beta x) \bigr)}{\alpha^{2}+\beta^{2}} + C
\end{equation}

\begin{equation}\label{eq:Exponential_Indefinite_Integral}
  \int e^{\alpha x} \; dx = \frac{e^{\alpha x}}{\alpha}
\end{equation}

\begin{equation}\label{eq:x_Exponential_Indefinite_Integral}
  \int x e^{\alpha x} \; dx = e^{\alpha x} \left( \frac{x}{\alpha} - \frac{1}{\alpha^{2}} \right)
\end{equation}
\Cref{eq:x_Exponential_Indefinite_Integral} simplified with Integration by Parts.

\begin{equation}\label{eq:Inverse_x_Indefinite_Integral}
  \int \frac{dx}{\alpha + \beta x} = \int \frac{1}{\alpha + \beta x} \; dx = \frac{1}{\beta} \ln (\alpha + \beta x)
\end{equation}

\begin{equation}\label{eq:Inverse_x_Squared_Indefinite_Integral}
  \int \frac{dx}{\alpha^{2} + \beta^{2} x^{2}} = \int \frac{1}{\alpha^{2} + \beta^{2} x^{2}} \; dx = \frac{1}{\alpha \beta} \arctan \left( \frac{\beta x}{\alpha} \right)
\end{equation}

\begin{equation}\label{eq:a_Exponential_Indefinite_Integral}
  \int \alpha^{x} \; dx = \frac{\alpha^{x}}{\ln(\alpha)}
\end{equation}

\begin{equation}\label{eq:a_Exponential_Derivative}
  \frac{d}{dx} \alpha^{x} = \frac{d\alpha^{x}}{dx} = \alpha^{x} \ln(x)
\end{equation}

\subsection{Leibnitz's Rule}\label{subsec:Leibnitzs_Rule}
\begin{lemma}[Leibnitz's Rule]\label{lemma:Leibnitzs_Rule}
  Given
  \begin{equation*}
    g(t) = \int_{a(t)}^{b(t)} f(x, t) \, dx
  \end{equation*}
  with $a(t)$ and $b(t)$ differentiable in $t$ and $\frac{\partial f(x, t)}{\partial t}$ continuous in both $t$ and $x$, then
  \begin{equation}\label{eq:Leibnitzs_Rule}
    \frac{d}{dt} g(t) = \frac{d g(t)}{dt} = \int_{a(t)}^{b(t)} \frac{\partial f(x, t)}{\partial t} \, dx + f \bigl[ b(t), t \bigr] \, \frac{d b(t)}{dt} - f \bigl[ a(t), t \bigr] \, \frac{d a(t)}{dt}
  \end{equation}
\end{lemma}


\end{document}