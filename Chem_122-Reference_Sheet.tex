\documentclass[10pt,letterpaper,final,twoside,notitlepage]{article}
\usepackage[margin=.5in]{geometry}
\usepackage[utf8]{inputenc}
\usepackage[english]{babel}
\usepackage{amsmath}
\usepackage{amsfonts}
\usepackage{amssymb}
\usepackage{amsthm} % Gives us plain, definition, and remark to use in \theoremstyle{style}
\usepackage{graphicx}

\usepackage{hyperref} % Generate hyperlinks to referenced items
\usepackage[noabbrev,nameinlink]{cleveref} % Fancy cross-references in the document everywhere
\usepackage{nameref} % Can make references by name to places
\usepackage{subcaption} % Allows for multiple figures in one Figure environment
\usepackage{siunitx} % Gives us ways to typeset units for stuff
\usepackage{enumitem} % Provides [noitemsep, nolistsep] for more compact lists
\usepackage{chngcntr} % Allows us to tamper with the counter a little more
\usepackage{empheq} % Allow boxing of equations in special math environments
\usepackage{tcolorbox} % Allows us to create boxes of various types for examples
\usepackage{tikz} % Allows us to create TikZ and PGF Pictures
\usetikzlibrary{trees}
%\usepackage{ctable} % Greater control over tables and how they look

% Topic Specific Packages
\usepackage{chemfig} % TikZ dependency. Used to draw chemical stuff

\graphicspath{{./Drawings/Chem_122}} % Uncomment this to use pictures in this document
\counterwithin{equation}{section} % Uncomment to number eqns with sec nums too

\theoremstyle{plain}
\newtheorem{theorem}{Theorem}
\counterwithin{theorem}{section}

\theoremstyle{definition}
\newtheorem{definition}{Defn}
\newtheorem{corollary}{Corollary}[section]

\theoremstyle{remark}
\newtheorem{remark}{Remark}[definition]
\newtheorem*{remark*}{Remark}
%\counterwithin{definition}{subsection} % Uncomment to have definitions use section.subsection numbering

% Create a special list that handles properties. It can be broken and restarted
\newlist{propertylist}{enumerate}{1} % {Name}{Template}{Max Depth}
\setlist[propertylist, 1]{label=\textbf{(\roman*)}, noitemsep, nolistsep} % Set options

% Create a special list that handles enumerate starting with lower letters. Breakable/Restartable.
\newlist{boldalphlist}{enumerate}{1} % {Name}{Template}{Max Depth}
\setlist[boldalphlist, 1]{label=\textbf{(\alph*)}, noitemsep, nolistsep} % Set options

% Create a tcolorbox for examples
% Argument #1 is optional, given by [], that is the textbook's problem number
% Argument #2 is mandatory, given by {}, that is the title for the example
\newtcolorbox[auto counter,
number within=section,
number format=\arabic,
crefname={example}{examples}, % Define reference format for cref (No Capitals)
Crefname={Example}{Examples}, % Reference format for cleveref (With Capitals)
]{example}[2][]{ % The [2][] Means the first argument is optional
	width=\textwidth,
	title={Example \thetcbcounter: #2. #1},
	fonttitle=\bfseries,
	label={ex:#2},
	nameref=#2,
	colbacktitle=white!100!black,
	coltitle=black!100!white,
	colback=white!100!black,
	upperbox=visible,
	lowerbox=visible,
	sharp corners=all
}

% Redefine the 'end of proof' symbol to be a black square, not blank
\renewcommand\qedsymbol{$\blacksquare$} % Change proofs to have black square at end

\DeclareMathOperator{\RealNums}{\mathbb{R}}
\DeclareMathOperator*{\argmax}{argmax} % Thin Space and subscripts are UNDER in display

\author{Karl Hallsby}
\title{Math 252 Differential Equations}

\begin{document}
\pagenumbering{roman} % i, ii, iii on beginning pages, that don't have content
\tableofcontents
\clearpage
\pagenumbering{arabic} % 1, 2, 3 on content pages

\section{Introduction} \label{sec:Introduction}
\subsection{Basic Chemistry Things} \label{subsec:Basic Chemistry Things}
\begin{definition}[Chemistry] \label{def:Chemistry}
  \emph{Chemistry} is the study of \nameref{def:Matter} and its changes.
  \begin{remark}
    We tend to use the macroscopic world to visualize the microscopic world.
  \end{remark}
\end{definition}

\begin{definition}[Matter] \label{def:Matter}
  \emph{Matter} is ``stuff'' that has both mass and volume.
\end{definition}

\begin{definition}[Scientific Method] \label{def:Scientific Method}
  The \emph{scientific method} is a systematic approach to research that utilizes qualitative or quantitative measurements.
\end{definition}

\begin{definition}[Hypothesis] \label{def:Hypothesis}
  A \emph{hypothesis} is a tentative explanation that will be tested using the \nameref{def:Scientific Method}.
\end{definition}

\begin{definition}[Law] \label{def:Law}
  A \emph{law} is a statement of a relationship between phenomena that is alwasy the same, under the same conditions.
  \begin{remark}
    These tend to be drawn from large amounts of data.
  \end{remark}
\end{definition}

\begin{example}[]{Law 1}
  Chlorine (Cl) is a highly reactive gas.
\end{example}

\begin{example}[]{Law 2}
  Matter is neither created nor destroyed.
\end{example}

\begin{definition}[Theory] \label{def:Theory}
  A \emph{theory} is a unifying principle that explains a body of facts based on facts and laws.
  These are constantly tested for validity.
  \begin{remark}
    A \nameref{def:Hypothesis} can turn into a \nameref{def:Theory} with enough experimentation and acceptance.
  \end{remark}
\end{definition}

\begin{example}[]{Theory 1}
  Reactivity of elements depends on the element's electron $\left( e^{-} \right)$ configuration.
\end{example}

\begin{example}[]{Theory 2}
  All matter is made up of tiny, indestructible particles, called atoms.
\end{example}

\subsection{Matter} \label{subsec:Matter}
As \Cref{def:Matter} said, matter must have both volume and mass.
Matter can have several \nameref{def:Matter State}s.

\begin{definition}[Matter State] \label{def:Matter State}
  A \emph{matter state} or \emph{state of matter} is just the configuration of atoms in a particular material.
  There are 3 common states:
  \begin{enumerate}[noitemsep, nolistsep]
  \item Solid
  \item Liquid
  \item Gas
  \end{enumerate}
\end{definition}

But, \nameref{def:Matter} can be categorized in different ways as well.
The Matter Tree is one way to categorize them.
\begin{figure}[h!]
  \input{./Drawings/Chem_122/Matter_Tree.tikz}
  \caption{Matter Tree}
  \label{fig:Matter Tree}
\end{figure}

Others include:
\begin{itemize}[noitemsep, nolistsep]
  \item Atomic Weight
  \item Chemical Properties
  \item Physical Properties
  \item \nameref{sec:Periodic Table}
  \item And many others
\end{itemize}

\subsection{Significant Figures} \label{subsec:Sig Figs}
\begin{definition}[Significant Figures] \label{def:Sig Figs}
  \emph{Significant Figures} or \emph{Sig Figs} are ways to handle uncertainty in our measurements.
  In general, we treat the data that we receive as inexact numbers, thus we must confirm our suspicions several times.
  Additionally, \nameref{def:Precision} and \nameref{def:Accuracy} are used interchangably when they shouldn't.
\end{definition}

\begin{definition}[Precision] \label{def:Precision}
  \emph{Precision} is defined as the closeness of data points to each other.
  If you think about a dartboard, this would be all the darts landing right next to each other.
\end{definition}

\begin{definition}[Accuracy] \label{def:Accuracy}
  \emph{Accuracy} is defined as how close your data is to the predicted true real value.
  \begin{remark}
    Generally, this must be done with a minimum of 3 trials, but more will yield more accurate data.
  \end{remark}
\end{definition}

\subsubsection{Rules for Significant Figures} \label{subsec:Rules for Sig Figs}
\begin{enumerate}[noitemsep, nolistsep]
\item 0s between any non-zero digit is significant ($100$, both 0s are significant)
\item 0s at the beginning of an integer are not significant ($010 = 10$)
\item 0s at the end are significant is the number is a decimal (0.003050 has 4 sig figs)
\item 0s at the end, if there is no decimal/fractional portion, are not significant (16000 has 2 sig figs)
\end{enumerate}

\begin{example}[]{Addition and Subtraction of Significant Figures}
  Add $20.3056$, $1.34$, and $54.2$ and keeping in mind significant figures.

  \tcblower

  You want to find the least precise number first, in this case it is $54.2$ because it only has one decimal place.
  This also determines how many decimal places to go past on the solution.
  Adding these 3 together gives $75.8456$, but because of $54.2$, it becomes $75.8$.
\end{example}

\begin{example}[]{Multiplication and Division of Significant Figures}
  Multiply $3.4456$ and $2.15$ keeping in mind significant figures.

  \tcblower

  You find the number with the least number of significant figures and use that.
  So, $2.15$ has 3 sig figs, that's the same amount your answer must have.
  \[ 3.4456 \times 2.15 = 7.40804 \]
  But because we can only have 3 sig figs in our answer, $7.41$ is our solution.
\end{example}
 % Section 1

\section{History of Chemistry} \label{sec:History of Chemistry}
\subsection{Dalton} \label{subsec:Dalton}
Dalton created the first meaningful definition of an atom.
He made several claims:
\begin{enumerate}[noitemsep, nolistsep]
\item Atoms are very small
\item The same element's atoms are identical, but different elements have different atoms.
\item Atoms are neither created, nor destroyed (\nameref{def:Law of Conservation of Matter})
\item Compounds are 2 or more elements together.
\end{enumerate}

\begin{definition}[Law of Conservation of Matter] \label{def:Law of Conservation of Matter}
  Matter is neither created, nor destroyed.
  It can \emph{only} change forms.
\end{definition}

\subsection{Thomson} \label{subsec:Thomson}
Thomson made several discoveries about atoms and their constituent particles.
For his experimentation, he used a cathode ray (a beam of positively charged ions) and magnets.
He discovered the \nameref{def:Electron}.

\begin{definition}[Electron] \label{def:Electron}
  The \emph{electron} is one of 3 particles that make up an atom.
  Electrons are negatively charged particles that are contained \emph{outside} of the nucleus.
  An electron's position and velocity can not be known simultaneously.
  This is known as the \nameref{def:Heisenbergs Uncertainty Principle}
\end{definition}
The cathode ray deflected from the ``negative'' magnetic plate to the ``positive''.
From this he calculated the \nameref{def:Magnetic Deflection}.

\begin{definition}[Magnetic Deflection] \label{def:Magnetic Deflection}
  When Thomson deflected his cathode ray with magnets, he measured how far it deviated from the starting line.
  \begin{equation} \label{eq:Magnetic Deflection}
    1.76 \times 10^{8} \si{\coulomb / \gram}
  \end{equation}
\end{definition}

\subsection{Millikan} \label{subsec:Millikan}
Millikan made 2 significant contributions to the model of the atom.
He discovered the charge of a single \nameref{def:Electron} and the mass of a single \nameref{def:Electron}.
\begin{equation} \label{eq:Electron Charge}
  1.602 \times 10^{-19} \si{\coulomb}
\end{equation}
\begin{equation} \label{eq:Electron Mass}
  9.10938 \times 10^{-28} \si{\gram}
\end{equation}

Both the \nameref{eq:Electron Charge} and \nameref{eq:Electron Mass} were drawn from \nameref{subsec:Thomson}'s work with \nameref{def:Magnetic Deflection}.

\subsection{Becquerel} \label{subsec:Becquerel}
Becquerel did his work with high energy radiation caused by radioactivity.
He found that there were 3 types of particles released by radioactive decay.
\begin{enumerate}[noitemsep, nolistsep]
  \item Alpha Particles ($\alpha$) - Positively charged particles that are charged helium atoms.
  \item Beta Particles ($\beta$) - Negatively charged particles that are essentially high speed electrons.
  \item Gamma particles ($\gamma$) - Uncharged particles that have next to no mass and are quite energetic.
\end{enumerate}

\subsection{Johnson} \label{subsec:Johnson}
Johnson developed one of the first models for single atoms.
This was called the \nameref{def:Plum Pudding Model}.

\begin{definition}[Plum Pudding Model] \label{def:Plum Pudding Model}
  The \emph{Plum Pudding Model} is a visualization of an atom.
  It is based off the plum pudding desert, which was one of Johnson's favorites.
  The positive charges were held together in a ``soft'' shell.
  The electrons were evenly distributed on the outer surface of the positively charged ``plum.''
  One of the hallmarks of this model was that the entire atom was \emph{not} empty space.
\end{definition}

\subsection{Rutherford} \label{subsec:Rutherford}
Rutherford performed experiments with $\alpha$-particles.
He ``shot'' these particles at a piece of gold foil and observed what happened with after the particles passed through.

\begin{definition}[Rutherford Model] \label{def:Rutherford Model}
  This is, more or less, the next model of the atom.
  Rutherford challenged \nameref{subsec:Johnson}'s \nameref{def:Plum Pudding Model} of the atom.
  When Rutherford sent the beam of alpha particles through the gold foil, he found most of them didn't deflect, i.e. hit any thing.
  However some did, and were scattered in all directions.
  Rutherford proved that atoms are mostly empty space, with the positive charges being held in a small dense area he called the \emph{nucleus}.
  \begin{remark}
    One thing to note about the \emph{nucleus} in the \nameref{def:Rutherford Model} is that there was no concept of the neutron yet.
    Since neutrons are uncharged particles, they were not discovered until much later.
  \end{remark}
\end{definition}

\subsection{Atomic Mass Units, \si{\atomicmassunit}} \label{subsec:AMU}
Eventually the neutron was discovered and the current understanding of the fundamental particles in atoms was completed.
These include the:
\begin{itemize}[noitemsep, nolistsep]
  \item Proton ($p^{+}$)
  \item Neutron ($N^{0}$)
  \item Electron ($e^{-}$)
\end{itemize}

The mass of each of these particles was found and the Atomic Mass Unit was developed to make calculations easier.
\begin{equation} \label{eq:AMU Equivalency}
  1 \si{\atomicmassunit} = 1.66054 \times 10^{-24} \si{\gram}
\end{equation}
Thus, the atomic mass for each particles is as follows:
\begin{itemize}[noitemsep, nolistsep]
  \item Proton ($p^{+}$) - $1.672623 \times 10^{-24} \si{\gram} = 1.0074 \si{\atomicmassunit}$
  \item Neutron ($N^{0}$) - $1.674927 \times 10^{-24} \si{\gram} = 1.0087 \si{\atomicmassunit}$
  \item Electron ($e^{-}$) - $9.109383 \times 10^{-28} \si{\gram} = 5.486 \times 10^{-4} \si{\atomicmassunit}$
\end{itemize} % Section 2

\section{The Periodic Table} \label{sec:Periodic Table}
The Periodic Table was developed as a way to categorize the many chemical elements in the world.
Most of the elements on the table are naturally occurring, but some of the heaviest elements have been synthesized in laboratories.

\begin{definition}[The Periodic Table] \label{def:Periodic Table}
  \emph{The Periodic Table} is an arrangement of elements by their atomic number, $Z$, or the number of protons in the nucleus.
  \begin{remark}
    The number of protons is the \emph{ONLY} thing that determines what an element is and where it is located on \nameref{def:Periodic Table}.
    If an element has a different number of neutrons, that is an \nameref{def:Isotope}.
    If an element has a different number of electrons, that is an \nameref{def:Ion}.
  \end{remark}
  \begin{remark}
    On \nameref{def:Periodic Table}, the elements are always in their electroneutral form.
    This means they have the same number of protons and electrons.
  \end{remark}
\end{definition}

A single element drawn out of \nameref{def:Periodic Table} will look like \figurename.
\begin{figure}[h!]
  %\input{./Drawings/Chem_122/Single_Element_Periodic_Table.tikz}
  \caption{Example Element from Periodic Table}
  \label{fig:Single Element Periodic Table}
\end{figure}

There are 4 parts of each element in \nameref{def:Periodic Table}.
\begin{enumerate}[noitemsep, nolistsep]
  \item The element's name
  \item The element's atomic number, $Z$, which is also the number of protons in the element
  \item The element's symbol
  \item The element's \emph{AVERAGE} atomic mass (\si{\gram / \mole}), $A$
\end{enumerate}
An element is usually written as such: $_{Z}^{A}\mathrm{X}$.
\begin{example}[]{Element Notation}
  What is Copper's (Cu) notation in text?
  
  \tcblower
  
  \ch{ {}_{29}^{63.546} Cu}
\end{example}

\subsection{Variations of Atoms} \label{subsec:Variations of Atoms}
\begin{definition}[Isotope] \label{def:Isotope}
  An \emph{isotope} is the same element, but with a different number of neutrons.
  While this does \emph{NOT} change the element, it may change some of the properties of this element.
\end{definition}

\begin{example}[]{Number of Neutrons}
  What is the number of neutrons in the average isotope of these elements: \ch{{}_{14}^{28}Si}, \ch{{}_{14}^{29}Si}, \si{{}_{14}^{30}Si}?
  
  \tcblower
  
  Since Silicon, \ch{Si} has 14 protons, so subtract 14 from each of the total nuclei weight. \\
  
  \begin{enumerate}[noitemsep, nolistsep]
    \item \ch{ {}_{14}^{28}Si} has 14 neutrons
    \item \ch{ {}_{14}^{29}Si} has 15 neutrons
    \item \ch{ {}_{14}^{30}Si} has 16 neutrons
  \end{enumerate}
\end{example}

\begin{definition}[Ion] \label{def:Ion}
  An \emph{ion} is the same element as the non-ionic form, however, the number of electrons and protons is different.
  This means that an ion can be positively or negatively charged.
  \begin{remark}
    The number of electrons in relation to protons determines the type of ion it is.
    \begin{itemize}[noitemsep, nolistsep]
      \item If there are more electrons than protons, it is a negatively charged ion.
      \item If there are more protons than electrons, it is a positively charged ion.
    \end{itemize}
  \end{remark}
  \begin{remark}
    In general:
    \begin{itemize}[noitemsep, nolistsep]
      \item \nameref{def:Metal}s are usually cations
      \item \nameref{def:Non-Metal}s are usually anions
    \end{itemize}
  \end{remark}
\end{definition}

\begin{example}[]{What is that Element?}
  Given that an element has 38 protons, 50 neutrons, and 36 electrons, what is the element?
  
  \tcblower
  
  Since there are 38 protons, that makes it Strontium (\ch{Sr}).
  The atomic weight will be $38 + 50 = 88$.
  The ion is $38 + -36 = +2$.
  Therefore, the written atom would be \ch{ {}_{38}^{88}Sr^{+2}}.
\end{example}

\begin{example}[]{An Ion's Protons Neutrons and Electrons}
  Given the element \ch{ {}_{33}^{75}As^{-3}}, how many protons, neutrons and electrons are there?
  
  \tcblower
  
  Astatine has 33 protons.
  The number of neutrons is $75 - 33 = 42$.
  Since this is Astatine, the number of electrons is $\lvert -33 - -3 \rvert = 36$.
\end{example}

\subsection{Types of Materials} \label{subsec:Types of Materials}
\begin{definition}[Metal] \label{def:Metal}
  A \emph{metal} is a type of material that:
  \begin{itemize}[noitemsep, nolistsep]
    \item Conducts heat well
    \item Conducts electricity well by freely losing electrons
    \item Their nuclei are congregated and their electrons form an outer electron cloud
  \end{itemize}
  \begin{remark}
    For \nameref{def:Metal}s, there are variations of some metals.
    For example, Iron (\RomanNumeral{2}) and Iron (\RomanNumeral{3}).
    These are \emph{NOT} different ions.
    Rather, they are Iron in different \nameref{def:Oxidation State}s.
  \end{remark}
\end{definition}

\begin{definition}[Non-Metal] \label{def:Non-Metal}
  A \emph{non-metal} is a type of material that:
  \begin{itemize}[noitemsep, nolistsep]
    \item Do not conduct heat well
    \item Do not conduct electricity well
    \item Gain electrons easily due to high electronegativity
  \end{itemize}
\end{definition}

\begin{definition}[Metalloid] \label{def:Metalloid}
  A \emph{metalloid} is a type of material that has properties of both \nameref{def:Metal} and \nameref{def:Non-Metal}.
  These elements on right on the edge between \nameref{def:Metal}s and \nameref{def:Non-Metal}s on the periodic table.
  These include:
  \begin{itemize}[noitemsep, nolistsep]
    \item Boron, \ch{B}
    \item Silicon, \ch{Si}
    \item Germanium, \ch{Ge}
    \item Astatine, \ch{As}
    \item Antimony, \ch{Sb}
    \item Tellurium, \ch{Te}
  \end{itemize}
\end{definition}

\begin{example}[]{Metal Nonmetal or Metalloid?}
  Are these elements \nameref{def:Metal}s, \nameref{def:Non-Metal}s, or \nameref{def:Metalloid}s?
  \begin{enumerate}[noitemsep, nolistsep]
    \item Phosphorus, \ch{P}
    \item Osmium, \ch{Os}
    \item Selenium, \ch{Se}
    \item Thallium, \ch{Tl}
    \item Nickel, \ch{Ni}
    \item Argon, \ch{Ar}
    \item Tellurium, \ch{Te}
  \end{enumerate}
  
  \tcblower
  
  \begin{enumerate}[noitemsep, nolistsep]
    \item Phosphorus, \ch{P} is a Non-Metal
    \item Osmium, \ch{Os} is a Metal
    \item Selenium, \ch{Se} is a Non-Metal
    \item Thallium, \ch{Tl} is a Metal
    \item Nickel, \ch{Ni} is a Metal
    \item Argon, \ch{Ar} is a Non-Metal
    \item Tellurium, \ch{Te} is a Metalloid
  \end{enumerate}
\end{example}
%%% Local Variables:
%%% mode: latex
%%% TeX-master: "../Chem_122-Reference_Sheet"
%%% End:
 % Section 3

\section{Quantum Chemistry} \label{sec:Quantum Chemistry}
This section is a brief introduction to how we have discovered certain properties of atoms due to quantum mechanics and physics.
One of the biggest ideas in quantum mechanics is \nameref{def:Heisenbergs Uncertainty Principle}.
\begin{definition}[Heisenberg's Uncertainty Principle] \label{def:Heisenbergs Uncertainty Principle}
  \emph{Heisenberg's Uncertainty Principle} states that it is impossible to accurately know both the position and velocity of a particle in a system.
  \begin{equation} \label{eq:Heisenbergs Uncertainty Principle}
    \Delta x \cdot \Delta p \geq \frac{h}{4 \pi}
  \end{equation}
  \begin{remark}
    The parameters for \Cref{eq:Heisenbergs Uncertainty Principle} are below.
    \begin{itemize}[noitemsep, nolistsep]
    \item $\Delta x$ is the change in position of the ``thing''
    \item $\Delta p$ is the change in the momentum of the ``thing''
    \item $h$ is Planck's Constant.
    \end{itemize}
  \end{remark}
\end{definition}
 % Section ??

%====================================APPENDIX====================================
\appendix
\counterwithin{equation}{section}
\counterwithin{definition}{subsection}

\subsection{Physical Constants} \label{app:Physical Constants}
	\begin{table}[h!]
		\centering
		\begin{tabular}{|c|c|c|}
			\hline
			\textbf{Constant Name} & \textbf{Variable Letter} & \textbf{Value} \\ \hline
			Boltzmann Constant & $R$ & $8.314 \si{\joule / \mole~\kelvin}$ \\ \hline
			Universal Gravitational & $G$ & $6.67408 \times 10^{-11} \si{\meter^{3}~\kilogram^{-1}~\second^{-2}}$ \\ \hline
			Planck's Constant & $h$ & $6.62607004 \times 10^{-34} \si{\meter \kilogram / \second}$ \\ \hline
			Speed of Light & $c$ & $299792458 \si{\meter / \second}$ \\ \hline
			Mass of Earth & $m_{Earth}$ & $5.972 \times 10^{24} \si{\kilogram}$ \\ \hline
			Diameter of Earth & $d_{Earth}$ & $12742 \si{\kilo\meter}$ \\ \hline
		\end{tabular}
	\end{table}
\subsection{Trigonometry} \label{app:Trig}
	\subsubsection{Trigonometric Formulas} \label{subsubsec:Trig Formulas}
		\begin{equation} \label{eq:Sin plus Sin with diff Angles}
			\sin \left( \alpha \right) + \sin \left( \beta \right) = 2 \sin \left( \frac{\alpha + \beta}{2} \right) \cos\left( \frac{\alpha - \beta}{2} \right)  
		\end{equation}
		\begin{equation} \label{eq:Cosine-Sine Product}
			\cos \left( \theta \right) \sin \left( \theta \right) = \frac{1}{2} \sin \left( 2 \theta \right)
		\end{equation}
\subsection{Calculus} \label{app:Calculus}
	\subsubsection{Fundamental Theorems of Calculus} \label{subsubsec:Fundamental Theorem of Calculus}
		\begin{definition}[First Fundamental Theorem of Calculus] \label{def:1st Fundamental Theorem of Calculus}
			The \emph{first fundamental theorem of calculus} states that, if $f$ is continuous on the closed interval $\left[ a,b \right]$ and $F$ is the indefinite integral of $f$ on $\left[ a,b \right]$, then 
			\begin{equation} \label{eq:1st Fundamental Theorem of Calculus}
				\int_{a}^{b}f \left( x \right) dx = F \left( b \right) - F \left( a \right)
			\end{equation}
		\end{definition}
		\begin{definition}[Second Fundamental Theorem of Calculus] \label{def:2nd Fundamental Theorem of Calculus}
			The \emph{second fundamental theorem of calculus} holds for $f$ a continuous function on an open interval $I$ and $a$ any point in $I$, and states that if $F$ is defined by
			\begin{equation*}
				F \left( x \right) = \int_{a}^{x} f \left( t \right) dt,
			\end{equation*}
			then
			\begin{equation} \label{eq:2nd Fundamental Theorem of Calculus}
				\begin{aligned}
					\frac{d}{dx} \int_{a}^{x} f \left( t \right) dt &= f \left( x \right) \\
					F' \left( x \right) &= f \left( x \right) \\
				\end{aligned}
			\end{equation}
		\end{definition}
\section{Complex Numbers}
	\begin{equation} \label{eq:Exponential to Rectangular}
		A e^{-ix} = A \left[ \cos \left( x \right) + i\sin \left( x \right) \right]
	\end{equation}

\end{document}