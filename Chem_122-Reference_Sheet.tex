\documentclass[10pt,letterpaper,final,twoside,notitlepage]{article}
\usepackage[margin=.5in]{geometry}
\usepackage[utf8]{inputenc}
\usepackage[english]{babel}
\usepackage{amsmath}
\usepackage{amsfonts}
\usepackage{amssymb}
\usepackage{amsthm} % Gives us plain, definition, and remark to use in \theoremstyle{style}
\usepackage{graphicx}

\usepackage{hyperref} % Generate hyperlinks to referenced items
\usepackage[noabbrev,nameinlink]{cleveref} % Fancy cross-references in the document everywhere
\usepackage{nameref} % Can make references by name to places
\usepackage{subcaption} % Allows for multiple figures in one Figure environment
\usepackage{siunitx} % Gives us ways to typeset units for stuff
\usepackage{enumitem} % Provides [noitemsep, nolistsep] for more compact lists
\usepackage{chngcntr} % Allows us to tamper with the counter a little more
\usepackage{empheq} % Allow boxing of equations in special math environments
\usepackage{tcolorbox} % Allows us to create boxes of various types for examples
\usepackage{tikz} % Allows us to create TikZ and PGF Pictures
\usetikzlibrary{trees}
%\usepackage{ctable} % Greater control over tables and how they look

\graphicspath{{./Drawings/Chem_122}} % Uncomment this to use pictures in this document
\counterwithin{equation}{section} % Uncomment to number eqns with sec nums too

\theoremstyle{plain}
\newtheorem{theorem}{Theorem}
\counterwithin{theorem}{section}

\theoremstyle{definition}
\newtheorem{definition}{Defn}
\newtheorem{corollary}{Corollary}[section]

\theoremstyle{remark}
\newtheorem{remark}{Remark}[definition]
\newtheorem*{remark*}{Remark}
%\counterwithin{definition}{subsection} % Uncomment to have definitions use section.subsection numbering

% Create a special list that handles properties. It can be broken and restarted
\newlist{propertylist}{enumerate}{1} % {Name}{Template}{Max Depth}
\setlist[propertylist, 1]{label=\textbf{(\roman*)}, noitemsep, nolistsep} % Set options

% Create a special list that handles enumerate starting with lower letters. Breakable/Restartable.
\newlist{boldalphlist}{enumerate}{1} % {Name}{Template}{Max Depth}
\setlist[boldalphlist, 1]{label=\textbf{(\alph*)}, noitemsep, nolistsep} % Set options

% Create a tcolorbox for examples
% Argument #1 is optional, given by [], that is the textbook's problem number
% Argument #2 is mandatory, given by {}, that is the title for the example
\newtcolorbox[auto counter,
number within=section,
number format=\arabic,
crefname={example}{examples}, % Define reference format for cref (No Capitals)
Crefname={Example}{Examples}, % Reference format for cleveref (With Capitals)
]{example}[2][]{ % The [2][] Means the first argument is optional
	width=\textwidth,
	title={Example \thetcbcounter: #2. #1},
	fonttitle=\bfseries,
	label={ex:#2},
	nameref=#2,
	colbacktitle=white!100!black,
	coltitle=black!100!white,
	colback=white!100!black,
	upperbox=visible,
	lowerbox=visible,
	sharp corners=all
}

% Redefine the 'end of proof' symbol to be a black square, not blank
\renewcommand\qedsymbol{$\blacksquare$} % Change proofs to have black square at end

\DeclareMathOperator{\RealNums}{\mathbb{R}}
\DeclareMathOperator*{\argmax}{argmax} % Thin Space and subscripts are UNDER in display

\author{Karl Hallsby}
\title{Math 252 Differential Equations}

\begin{document}
\pagenumbering{roman} % i, ii, iii on beginning pages, that don't have content
\tableofcontents
\clearpage
\pagenumbering{arabic} % 1, 2, 3 on content pages

\section{Introduction} \label{sec:Introduction}
\subsection{Basic Chemistry Things} \label{subsec:Basic Chemistry Things}
\begin{definition}[Chemistry] \label{def:Chemistry}
  \emph{Chemistry} is the study of \nameref{def:Matter} and its changes.
  \begin{remark}
    We tend to use the macroscopic world to visualize the microscopic world.
  \end{remark}
\end{definition}

\begin{definition}[Matter] \label{def:Matter}
  \emph{Matter} is ``stuff'' that has both mass and volume.
\end{definition}

\begin{definition}[Scientific Method] \label{def:Scientific Method}
  The \emph{scientific method} is a systematic approach to research that utilizes qualitative or quantitative measurements.
\end{definition}

\begin{definition}[Hypothesis] \label{def:Hypothesis}
  A \emph{hypothesis} is a tentative explanation that will be tested using the \nameref{def:Scientific Method}.
\end{definition}

\begin{definition}[Law] \label{def:Law}
  A \emph{law} is a statement of a relationship between phenomena that is alwasy the same, under the same conditions.
  \begin{remark}
    These tend to be drawn from large amounts of data.
  \end{remark}
\end{definition}

\begin{example}[]{Law 1}
  Chlorine (Cl) is a highly reactive gas.
\end{example}

\begin{example}[]{Law 2}
  Matter is neither created nor destroyed.
\end{example}

\begin{definition}[Theory] \label{def:Theory}
  A \emph{theory} is a unifying principle that explains a body of facts based on facts and laws.
  These are constantly tested for validity.
  \begin{remark}
    A \nameref{def:Hypothesis} can turn into a \nameref{def:Theory} with enough experimentation and acceptance.
  \end{remark}
\end{definition}

\begin{example}[]{Theory 1}
  Reactivity of elements depends on the element's electron $\left( e^{-} \right)$ configuration.
\end{example}

\begin{example}[]{Theory 2}
  All matter is made up of tiny, indestructible particles, called atoms.
\end{example}

\subsection{Matter} \label{subsec:Matter}
As \cref{def:Matter} said, matter must have both volume and mass.
Matter can have several \nameref{def:Matter State}s.

\begin{definition}[Matter State] \label{def:Matter State}
  A \emph{matter state} or \emph{state of matter} is just the configuration of atoms in a particular material.
  There are 3 common states:
  \begin{enumerate}[noitemsep, nolistsep]
  \item Solid
  \item Liquid
  \item Gas
  \end{enumerate}
\end{definition}

But, \nameref{def:Matter} can be categorized in a different way as well.
% \begin{tikzpicture}
\end{tikzpicture}

\subsection{Significant Figures} \label{subsec:Sig Figs}
\begin{definition}[Significant Figures] \label{def:Sig Figs}
  \emph{Significant Figures} or \emph{Sig Figs} are ways to handle uncertainty in our measurements.
  In general, we treat the data that we receive as inexact numbers, thus we must confirm our suspicions several times.
  Additionally, \nameref{def:Precision} and \nameref{def:Accuracy} are used interchangably when they shouldn't.
\end{definition}

\begin{definition}[Precision] \label{def:Precision}
  \emph{Precision} is defined as the closeness of data points to each other.
  If you think about a dartboard, this would be all the darts landing right next to each other.
\end{definition}

\begin{definition}[Accuracy] \label{def:Accuracy}
  \emph{Accuracy} is defined as how close your data is to the predicted true real value.
  \begin{remark}
    Generally, this must be done with a minimum of 3 trials, but more will yield more accurate data.
  \end{remark}
\end{definition}

\subsubsection{Rules for Significant Figures} \label{subsec:Rules for Sig Figs}
\begin{enumerate}[noitemsep, nolistsep]
\item 0s between any non-zero digit is significant ($100$, both 0s are significant)
\item 0s at the beginning of an integer are not significant ($010 = 10$)
\item 0s at the end are significant is the number is a decimal (0.003050 has 4 sig figs)
\item 0s at the end, if there is no decimal/fractional portion, are not significant (16000 has 2 sig figs)
\end{enumerate}

   % Section 1

%====================================APPENDIX====================================
\appendix
\counterwithin{equation}{section}
\counterwithin{definition}{subsection}

\subsection{Trigonometry} \label{app:Trig}
	\subsubsection{Trigonometric Formulas} \label{subsubsec:Trig Formulas}
		\begin{equation} \label{eq:Sin plus Sin with diff Angles}
			\sin \left( \alpha \right) + \sin \left( \beta \right) = 2 \sin \left( \frac{\alpha + \beta}{2} \right) \cos\left( \frac{\alpha - \beta}{2} \right)  
		\end{equation}
		\begin{equation} \label{eq:Cosine-Sine Product}
			\cos \left( \theta \right) \sin \left( \theta \right) = \frac{1}{2} \sin \left( 2 \theta \right)
		\end{equation}
	
	\subsubsection{Euler Equivalents of Trigonometric Functions} \label{subsubsec:Euler Equivalents}
		\begin{equation} \label{eq:Euler Sin}
			\sin \left( x \right) = \frac{e^{\imath x} + e^{-\imath x}}{2}
		\end{equation}
		\begin{equation} \label{eq:Euler Cos}
			\cos \left( x \right) = \frac{e^{\imath x} - e^{-\imath x}}{2 \imath}
		\end{equation}
		\begin{equation} \label{eq:Euler Sinh}
			\sinh \left( x \right) = \frac{e^{x} - e^{-x}}{2}
		\end{equation}
		\begin{equation} \label{eq:Euler Cosh}
			\cosh \left( x \right) = \frac{e^{x} + e^{-x}}{2}
		\end{equation}
\section{Calculus}\label{app:Calculus}
\subsection{L'Hopital's Rule}\label{subsec:LHopitals_Rule}
L'Hopital's Rule can be used to simplify and solve expressions regarding limits that yield irreconcialable results.
\begin{lemma}[L'Hopital's Rule]\label{lemma:LHopitals_Rule}
  If the equation
  \begin{equation*}
    \lim\limits_{x \rightarrow a} \frac{f(x)}{g(x)} =
    \begin{cases}
      \frac{0}{0} \\
      \frac{\infty}{\infty} \\
    \end{cases}
  \end{equation*}
  then \Cref{eq:LHopitals_Rule} holds.
  \begin{equation}\label{eq:LHopitals_Rule}
    \lim\limits_{x \rightarrow a} \frac{f(x)}{g(x)} = \lim\limits_{x \rightarrow a} \frac{f'(x)}{g'(x)}
  \end{equation}
\end{lemma}

\subsection{Fundamental Theorems of Calculus}\label{subsec:Fundamental Theorem of Calculus}
\begin{definition}[First Fundamental Theorem of Calculus]\label{def:1st Fundamental Theorem of Calculus}
  The \emph{first fundamental theorem of calculus} states that, if $f$ is continuous on the closed interval $\left[ a,b \right]$ and $F$ is the indefinite integral of $f$ on $\left[ a,b \right]$, then

  \begin{equation}\label{eq:1st Fundamental Theorem of Calculus}
    \int_{a}^{b}f \left( x \right) dx = F \left( b \right) - F \left( a \right)
  \end{equation}
\end{definition}

\begin{definition}[Second Fundamental Theorem of Calculus]\label{def:2nd Fundamental Theorem of Calculus}
  The \emph{second fundamental theorem of calculus} holds for $f$ a continuous function on an open interval $I$ and $a$ any point in $I$, and states that if $F$ is defined by

  \begin{equation*}
    F \left( x \right) = \int_{a}^{x} f \left( t \right) dt,
  \end{equation*}
  then
  \begin{equation}\label{eq:2nd Fundamental Theorem of Calculus}
    \begin{aligned}
      \frac{d}{dx} \int_{a}^{x} f \left( t \right) dt &= f \left( x \right) \\
      F' \left( x \right) &= f \left( x \right) \\
    \end{aligned}
  \end{equation}
\end{definition}

\begin{definition}[argmax]\label{def:argmax}
  The arguments to the \emph{argmax} function are to be maximized by using their derivatives.
  You must take the derivative of the function, find critical points, then determine if that critical point is a global maxima.
  This is denoted as
  \begin{equation*}\label{eq:argmax}
    \argmax_{x}
  \end{equation*}
\end{definition}

\subsection{Rules of Calculus}\label{subsec:Rules of Calculus}
\subsubsection{Chain Rule}\label{subsubsec:Chain Rule}
\begin{definition}[Chain Rule]\label{def:Chain Rule}
  The \emph{chain rule} is a way to differentiate a function that has 2 functions multiplied together.

  If
  \begin{equation*}
    f(x) = g(x) \cdot h(x)
  \end{equation*}
  then,
  \begin{equation}\label{eq:Chain Rule}
    \begin{aligned}
      f'(x) &= g'(x) \cdot h(x) + g(x) \cdot h'(x) \\
      \frac{df(x)}{dx} &= \frac{dg(x)}{dx} \cdot g(x) + g(x) \cdot \frac{dh(x)}{dx} \\
    \end{aligned}
  \end{equation}
\end{definition}

\subsection{Useful Integrals}\label{subsec:Useful_Integrals}
\begin{equation}\label{eq:Cosine_Indefinite_Integral}
  \int \cos(x) \; dx = \sin(x)
\end{equation}

\begin{equation}\label{eq:Sine_Indefinite_Integral}
  \int \sin(x) \; dx = -\cos(x)
\end{equation}

\begin{equation}\label{eq:x_Cosine_Indefinite_Integral}
  \int x \cos(x) \; dx = \cos(x) + x \sin(x)
\end{equation}
\Cref{eq:x_Cosine_Indefinite_Integral} simplified with Integration by Parts.

\begin{equation}\label{eq:x_Sine_Indefinite_Integral}
  \int x \sin(x) \; dx = \sin(x) - x \cos(x)
\end{equation}
\Cref{eq:x_Sine_Indefinite_Integral} simplified with Integration by Parts.

\begin{equation}\label{eq:x_Squared_Cosine_Indefinite_Integral}
  \int x^{2} \cos(x) \; dx = 2x \cos(x) + (x^{2} - 2) \sin(x)
\end{equation}
\Cref{eq:x_Squared_Cosine_Indefinite_Integral} simplified by using Integration by Parts twice.

\begin{equation}\label{eq:x_Squared_Sine_Indefinite_Integral}
  \int x^{2} \sin(x) \; dx = 2x \sin(x) - (x^{2} - 2) \cos(x)
\end{equation}
\Cref{eq:x_Squared_Sine_Indefinite_Integral} simplified by using Integration by Parts twice.

\begin{equation}\label{eq:Exponential_Cosine_Indefinite_Integral}
  \int e^{\alpha x} \cos(\beta x) \; dx = \frac{e^{\alpha x} \bigl( \alpha \cos(\beta x) + \beta \sin(\beta x) \bigr)}{\alpha^{2} + \beta^{2}} + C
\end{equation}

\begin{equation}\label{eq:Exponential_Sine_Indefinite_Integral}
  \int e^{\alpha x} \sin(\beta x) \; dx = \frac{e^{\alpha x} \bigl( \alpha \sin(\beta x) - \beta \cos(\beta x) \bigr)}{\alpha^{2}+\beta^{2}} + C
\end{equation}

\begin{equation}\label{eq:Exponential_Indefinite_Integral}
  \int e^{\alpha x} \; dx = \frac{e^{\alpha x}}{\alpha}
\end{equation}

\begin{equation}\label{eq:x_Exponential_Indefinite_Integral}
  \int x e^{\alpha x} \; dx = e^{\alpha x} \left( \frac{x}{\alpha} - \frac{1}{\alpha^{2}} \right)
\end{equation}
\Cref{eq:x_Exponential_Indefinite_Integral} simplified with Integration by Parts.

\begin{equation}\label{eq:Inverse_x_Indefinite_Integral}
  \int \frac{dx}{\alpha + \beta x} = \int \frac{1}{\alpha + \beta x} \; dx = \frac{1}{\beta} \ln (\alpha + \beta x)
\end{equation}

\begin{equation}\label{eq:Inverse_x_Squared_Indefinite_Integral}
  \int \frac{dx}{\alpha^{2} + \beta^{2} x^{2}} = \int \frac{1}{\alpha^{2} + \beta^{2} x^{2}} \; dx = \frac{1}{\alpha \beta} \arctan \left( \frac{\beta x}{\alpha} \right)
\end{equation}

\begin{equation}\label{eq:a_Exponential_Indefinite_Integral}
  \int \alpha^{x} \; dx = \frac{\alpha^{x}}{\ln(\alpha)}
\end{equation}

\begin{equation}\label{eq:a_Exponential_Derivative}
  \frac{d}{dx} \alpha^{x} = \frac{d\alpha^{x}}{dx} = \alpha^{x} \ln(x)
\end{equation}

\subsection{Leibnitz's Rule}\label{subsec:Leibnitzs_Rule}
\begin{lemma}[Leibnitz's Rule]\label{lemma:Leibnitzs_Rule}
  Given
  \begin{equation*}
    g(t) = \int_{a(t)}^{b(t)} f(x, t) \, dx
  \end{equation*}
  with $a(t)$ and $b(t)$ differentiable in $t$ and $\frac{\partial f(x, t)}{\partial t}$ continuous in both $t$ and $x$, then
  \begin{equation}\label{eq:Leibnitzs_Rule}
    \frac{d}{dt} g(t) = \frac{d g(t)}{dt} = \int_{a(t)}^{b(t)} \frac{\partial f(x, t)}{\partial t} \, dx + f \bigl[ b(t), t \bigr] \, \frac{d b(t)}{dt} - f \bigl[ a(t), t \bigr] \, \frac{d a(t)}{dt}
  \end{equation}
\end{lemma}



\end{document}