\documentclass[10pt,letterpaper,final,twoside,notitlepage]{article}
\usepackage[margin=.5in]{geometry}
\usepackage[utf8]{inputenc}
\usepackage[english]{babel}
\usepackage{amsmath}
\usepackage{amsfonts}
\usepackage{amssymb}
\usepackage{amsthm} % Gives us plain, definition, and remark to use in \theoremstyle{style}
\usepackage{graphicx}

\usepackage{hyperref} % Generate hyperlinks to referenced items
\usepackage[noabbrev,nameinlink]{cleveref} % Fancy cross-references in the document everywhere
\usepackage{nameref} % Can make references by name to places
\usepackage{subcaption} % Allows for multiple figures in one Figure environment
\usepackage{siunitx} % Gives us ways to typeset units for stuff
\usepackage{enumitem} % Provides [noitemsep, nolistsep] for more compact lists
\usepackage{chngcntr} % Allows us to tamper with the counter a little more
\usepackage{empheq} % Allow boxing of equations in special math environments
\usepackage{tcolorbox} % Allows us to create boxes of various types for examples
\usepackage{tikz} % Allows us to create TikZ and PGF Pictures
\usetikzlibrary{trees}
%\usepackage{ctable} % Greater control over tables and how they look

% Topic Specific Packages
\usepackage{chemfig} % TikZ dependency. Used to draw chemical stuff
\usepackage{chemformula} % Provides the \ch{formula} command to typeset chemical equations

\graphicspath{{./Drawings/Chem_122}} % Uncomment this to use pictures in this document
\counterwithin{equation}{section} % Uncomment to number eqns with sec nums too

\theoremstyle{plain}
\newtheorem{theorem}{Theorem}
\counterwithin{theorem}{section}

\theoremstyle{definition}
\newtheorem{definition}{Defn}
\newtheorem{corollary}{Corollary}[section]

\theoremstyle{remark}
\newtheorem{remark}{Remark}[definition]
\newtheorem*{remark*}{Remark}
%\counterwithin{definition}{subsection} % Uncomment to have definitions use section.subsection numbering

% Create a special list that handles properties. It can be broken and restarted
\newlist{propertylist}{enumerate}{1} % {Name}{Template}{Max Depth}
\setlist[propertylist, 1]{label=\textbf{(\roman*)}, noitemsep, nolistsep} % Set options

% Create a special list that handles enumerate starting with lower letters. Breakable/Restartable.
\newlist{boldalphlist}{enumerate}{1} % {Name}{Template}{Max Depth}
\setlist[boldalphlist, 1]{label=\textbf{(\alph*)}, noitemsep, nolistsep} % Set options

% Create a tcolorbox for examples
% Argument #1 is optional, given by [], that is the textbook's problem number
% Argument #2 is mandatory, given by {}, that is the title for the example
\newtcolorbox[auto counter,
number within=section,
number format=\arabic,
crefname={example}{examples}, % Define reference format for cref (No Capitals)
Crefname={Example}{Examples}, % Reference format for cleveref (With Capitals)
]{example}[2][]{ % The [2][] Means the first argument is optional
	width=\textwidth,
	title={Example \thetcbcounter: #2. #1},
	fonttitle=\bfseries,
	label={ex:#2},
	nameref=#2,
	colbacktitle=white!100!black,
	coltitle=black!100!white,
	colback=white!100!black,
	upperbox=visible,
	lowerbox=visible,
	sharp corners=all
}

% Redefine the 'end of proof' symbol to be a black square, not blank
\renewcommand\qedsymbol{$\blacksquare$} % Change proofs to have black square at end

\DeclareMathOperator{\RealNums}{\mathbb{R}}
\DeclareMathOperator*{\argmax}{argmax} % Thin Space and subscripts are UNDER in display

\author{Karl Hallsby}
\title{Math 252 Differential Equations}

\begin{document}
\pagenumbering{roman} % i, ii, iii on beginning pages, that don't have content
\tableofcontents
\clearpage
\pagenumbering{arabic} % 1, 2, 3 on content pages

\section{Introduction} \label{sec:Introduction}
\subsection{Basic Chemistry Things} \label{subsec:Basic Chemistry Things}
\begin{definition}[Chemistry] \label{def:Chemistry}
  \emph{Chemistry} is the study of \nameref{def:Matter} and its changes.
  \begin{remark}
    We tend to use the macroscopic world to visualize the microscopic world.
  \end{remark}
\end{definition}

\begin{definition}[Matter] \label{def:Matter}
  \emph{Matter} is ``stuff'' that has both mass and volume.
\end{definition}

\begin{definition}[Scientific Method] \label{def:Scientific Method}
  The \emph{scientific method} is a systematic approach to research that utilizes qualitative or quantitative measurements.
\end{definition}

\begin{definition}[Hypothesis] \label{def:Hypothesis}
  A \emph{hypothesis} is a tentative explanation that will be tested using the \nameref{def:Scientific Method}.
\end{definition}

\begin{definition}[Law] \label{def:Law}
  A \emph{law} is a statement of a relationship between phenomena that is alwasy the same, under the same conditions.
  \begin{remark}
    These tend to be drawn from large amounts of data.
  \end{remark}
\end{definition}

\begin{example}[]{Law 1}
  Chlorine (Cl) is a highly reactive gas.
\end{example}

\begin{example}[]{Law 2}
  Matter is neither created nor destroyed.
\end{example}

\begin{definition}[Theory] \label{def:Theory}
  A \emph{theory} is a unifying principle that explains a body of facts based on facts and laws.
  These are constantly tested for validity.
  \begin{remark}
    A \nameref{def:Hypothesis} can turn into a \nameref{def:Theory} with enough experimentation and acceptance.
  \end{remark}
\end{definition}

\begin{example}[]{Theory 1}
  Reactivity of elements depends on the element's electron $\left( e^{-} \right)$ configuration.
\end{example}

\begin{example}[]{Theory 2}
  All matter is made up of tiny, indestructible particles, called atoms.
\end{example}

\subsection{Matter} \label{subsec:Matter}
As \cref{def:Matter} said, matter must have both volume and mass.
Matter can have several \nameref{def:Matter State}s.

\begin{definition}[Matter State] \label{def:Matter State}
  A \emph{matter state} or \emph{state of matter} is just the configuration of atoms in a particular material.
  There are 3 common states:
  \begin{enumerate}[noitemsep, nolistsep]
  \item Solid
  \item Liquid
  \item Gas
  \end{enumerate}
\end{definition}

But, \nameref{def:Matter} can be categorized in a different way as well.
% \begin{tikzpicture}
\end{tikzpicture}

\subsection{Significant Figures} \label{subsec:Sig Figs}
\begin{definition}[Significant Figures] \label{def:Sig Figs}
  \emph{Significant Figures} or \emph{Sig Figs} are ways to handle uncertainty in our measurements.
  In general, we treat the data that we receive as inexact numbers, thus we must confirm our suspicions several times.
  Additionally, \nameref{def:Precision} and \nameref{def:Accuracy} are used interchangably when they shouldn't.
\end{definition}

\begin{definition}[Precision] \label{def:Precision}
  \emph{Precision} is defined as the closeness of data points to each other.
  If you think about a dartboard, this would be all the darts landing right next to each other.
\end{definition}

\begin{definition}[Accuracy] \label{def:Accuracy}
  \emph{Accuracy} is defined as how close your data is to the predicted true real value.
  \begin{remark}
    Generally, this must be done with a minimum of 3 trials, but more will yield more accurate data.
  \end{remark}
\end{definition}

\subsubsection{Rules for Significant Figures} \label{subsec:Rules for Sig Figs}
\begin{enumerate}[noitemsep, nolistsep]
\item 0s between any non-zero digit is significant ($100$, both 0s are significant)
\item 0s at the beginning of an integer are not significant ($010 = 10$)
\item 0s at the end are significant is the number is a decimal (0.003050 has 4 sig figs)
\item 0s at the end, if there is no decimal/fractional portion, are not significant (16000 has 2 sig figs)
\end{enumerate}

   % Section 1

\section{History of Chemistry} \label{sec:History of Chemistry}
\subsection{Dalton} \label{subsec:Dalton}
Dalton created the first meaningful definition of an atom.
He made several claims:
\begin{enumerate}[noitemsep, nolistsep]
\item Atoms are very small
\item The same element's atoms are identical, but different elements have different atoms.
\item Atoms are neither created, nor destroyed (\nameref{def:Law of Conservation of Matter})
\item Compounds are 2 or more elements together.
\end{enumerate}

\begin{definition}[Law of Conservation of Matter] \label{def:Law of Conservation of Matter}
  Matter is neither created, nor destroyed.
  It can \emph{only} change forms.
\end{definition}

\subsection{Thomson} \label{subsec:Thomson}
Thomson made several discoveries about atoms and their constiuent particles.
For his experimentation, he used a cathode ray (a beam of postively charged ions) and magnets.
He discovered the \nameref{def:Electron}.

\begin{definition}[Electron] \label{def:Electron}
  The \emph{electron} is one of 3 particles that make up an atom.
  Electrons are negatively charged particles that are contained \emph{outside} of the nucleus.
  An electron's position and velocity can not be known simultaneously.
  This is known as the \nameref{def:Heisenbergs Uncertainty Principle}
\end{definition}
The cathode ray deflected from the ``negative'' magnetic plate to the ``postive''.
From this he calculated the \nameref{def:Magnetic Deflection}.

\begin{definition}[Magnetic Deflection] \label{def:Magnetic Deflection}
  When Thomson deflected his cathode ray with magnets, he measured how far it deviated from the starting line.
  \begin{equation} \label{eq:Magnetic Deflection}
    1.76 \times 10^{8} \si{\coulomb / \gram}
  \end{equation}
\end{definition}
 % Section 2

\section{The Periodic Table} \label{sec:Periodic Table}
The Periodic Table was developed as a way to categorize the many chemical elements in the world.
Most of the elements on the table are naturally occurring, but some of the heaviest elements have been synthesized in laboratories.

\begin{definition}[The Periodic Table] \label{def:Periodic Table}
  \emph{The Periodic Table} is an arrangement of elements by their atomic number, $Z$, or the number of protons in the nucleus.
  \begin{remark}
    The number of protons is the \emph{ONLY} thing that determines what an element is and where it is located on \nameref{def:Periodic Table}.
    If an element has a different number of neutrons, that is an \nameref{def:Isotope}.
    If an element has a different number of electrons, that is an \nameref{def:Ion}.
  \end{remark}
  \begin{remark}
    On \nameref{def:Periodic Table}, the elements are always in their electroneutral form.
    This means they have the same number of protons and electrons.
  \end{remark}
\end{definition}

A single element drawn out of \nameref{def:Periodic Table} will look like \figurename.
\begin{figure}[h!]
  %\begin{tikzpicture}
\end{tikzpicture}
  \caption{Example Element from Periodic Table}
  \label{fig:Single Element Periodic Table}
\end{figure}

There are 4 parts of each element in \nameref{def:Periodic Table}.
\begin{enumerate}[noitemsep, nolistsep]
  \item The element's name
  \item The element's atomic number, $Z$, which is also the number of protons in the element
  \item The element's symbol
  \item The element's \emph{AVERAGE} atomic mass (\si{\gram / \mole}), $A$
\end{enumerate}
An element is usually written as such: $_{Z}^{A}\mathrm{X}$.
\begin{example}[]{Element Notation}
  What is Copper's (Cu) notation in text?
  
  \tcblower
  
  \ch{ {}_{29}^{63.546} Cu}
\end{example}

\subsection{Variations of Atoms} \label{subsec:Variations of Atoms}
\begin{definition}[Isotope] \label{def:Isotope}
  An \emph{isotope} is the same element, but with a different number of neutrons.
  While this does \emph{NOT} change the element, it may change some of the properties of this element.
\end{definition}

\begin{example}[]{Number of Neutrons}
  What is the number of neutrons in the average isotope of these elements: \ch{{}_{14}^{28}Si}, \ch{{}_{14}^{29}Si}, \si{{}_{14}^{30}Si}?
  
  \tcblower
  
  Since Silicon, \ch{Si} has 14 protons, so subtract 14 from each of the total nuclei weight. \\
  
  \begin{enumerate}[noitemsep, nolistsep]
    \item \ch{ {}_{14}^{28}Si} has 14 neutrons
    \item \ch{ {}_{14}^{29}Si} has 15 neutrons
    \item \ch{ {}_{14}^{30}Si} has 16 neutrons
  \end{enumerate}
\end{example}

\begin{definition}[Ion] \label{def:Ion}
  An \emph{ion} is the same element as the non-ionic form, however, the number of electrons and protons is different.
  This means that an ion can be positively or negatively charged.
  \begin{remark}
    The number of electrons in relation to protons determines the type of ion it is.
    \begin{itemize}[noitemsep, nolistsep]
      \item If there are more electrons than protons, it is a negatively charged ion.
      \item If there are more protons than electrons, it is a positively charged ion.
    \end{itemize}
  \end{remark}
  \begin{remark}
    In general:
    \begin{itemize}[noitemsep, nolistsep]
      \item \nameref{def:Metal}s are usually cations
      \item \nameref{def:Non-Metal}s are usually anions
    \end{itemize}
  \end{remark}
\end{definition}

\begin{example}[]{What is that Element?}
  Given that an element has 38 protons, 50 neutrons, and 36 electrons, what is the element?
  
  \tcblower
  
  Since there are 38 protons, that makes it Strontium (\ch{Sr}).
  The atomic weight will be $38 + 50 = 88$.
  The ion is $38 + -36 = +2$.
  Therefore, the written atom would be \ch{ {}_{38}^{88}Sr^{+2}}.
\end{example}

\begin{example}[]{An Ion's Protons Neutrons and Electrons}
  Given the element \ch{ {}_{33}^{75}As^{-3}}, how many protons, neutrons and electrons are there?
  
  \tcblower
  
  Astatine has 33 protons.
  The number of neutrons is $75 - 33 = 42$.
  Since this is Astatine, the number of electrons is $\lvert -33 - -3 \rvert = 36$.
\end{example}

\subsection{Types of Materials} \label{subsec:Types of Materials}
\begin{definition}[Metal] \label{def:Metal}
  A \emph{metal} is a type of material that:
  \begin{itemize}[noitemsep, nolistsep]
    \item Conducts heat well
    \item Conducts electricity well by freely losing electrons
    \item Their nuclei are congregated and their electrons form an outer electron cloud
  \end{itemize}
  \begin{remark}
    For \nameref{def:Metal}s, there are variations of some metals.
    For example, Iron (\RomanNumeral{2}) and Iron (RomanNumeral{3}).
    These are \emph{NOT} different ions.
    Rather, they are Iron in different \nameref{def:Oxidation States}.
  \end{remark}
\end{definition}

\begin{definition}[Non-Metal] \label{def:Non-Metal}
  A \emph{non-metal} is a type of material that:
  \begin{itemize}[noitemsep, nolistsep]
    \item Do not conduct heat well
    \item Do not conduct electricity well
    \item Gain electrons easily due to high electronegativity
  \end{itemize}
\end{definition}

\begin{definition}[Metalloid] \label{def:Metalloid}
  A \emph{metalloid} is a type of material that has properties of both \nameref{def:Metal} and \nameref{def:Non-Metal}.
  These elements on right on the edge between \nameref{def:Metal}s and \nameref{def:Non-Metal}s on the periodic table.
  These include:
  \begin{itemize}[noitemsep, nolistsep]
    \item Boron, \ch{B}
    \item Silicon, \ch{Si}
    \item Germanium, \ch{Ge}
    \item Astatine, \ch{As}
    \item Antimony, \ch{Sb}
    \item Tellurium, \ch{Te}
  \end{itemize}
\end{definition}

\begin{example}[]{Metal Nonmetal or Metalloid?}
  Are these elements \nameref{def:Metal}s, \nameref{def:Non-Metal}s, or \nameref{def:Metalloid}s?
  \begin{enumerate}[noitemsep, nolistsep]
    \item Phosphorus, \ch{P}
    \item Osmium, \ch{Os}
    \item Selenium, \ch{Se}
    \item Thallium, \ch{Tl}
    \item Nickel, \ch{Ni}
    \item Argon, \ch{Ar}
    \item Tellurium, \ch{Te}
  \end{enumerate}
  
  \tcblower
  
  \begin{enumerate}[noitemsep, nolistsep]
    \item Phosphorus, \ch{P} is a Non-Metal
    \item Osmium, \ch{Os} is a Metal
    \item Selenium, \ch{Se} is a Non-Metal
    \item Thallium, \ch{Tl} is a Metal
    \item Nickel, \ch{Ni} is a Metal
    \item Argon, \ch{Ar} is a Non-Metal
    \item Tellurium, \ch{Te} is a Metalloid
  \end{enumerate}
\end{example} % Section 3

\section{Quantum Chemistry} \label{sec:Quantum Chemistry}
This section is a brief introduction to how we have discovered certain properties of atoms due to quantum mechanics and physics.
One of the biggest ideas in quantum mechanics is \nameref{def:Heisenbergs Uncertainty Principle}.
\begin{definition}[Heisenberg's Uncertainty Principle] \label{def:Heisenbergs Uncertainty Principle}
  \emph{Heisenberg's Uncertainty Principle} states that it is impossible to accurately know both the position and velocity of a particle in a system.
  \begin{equation} \label{eq:Heisenbergs Uncertainty Principle}
    \Delta x \cdot \Delta p \geq \frac{h}{4 \pi}
  \end{equation}
  \begin{remark}
    The parameters for \Cref{eq:Heisenbergs Uncertainty Principle} are below.
    \begin{itemize}[noitemsep, nolistsep]
    \item $\Delta x$ is the change in position of the ``thing''
    \item $\Delta p$ is the change in the momentum of the ``thing''
    \item $h$ is Planck's Constant.
    \end{itemize}
  \end{remark}
\end{definition}
 % Section ??

%====================================APPENDIX====================================
\appendix
\counterwithin{equation}{section}
\counterwithin{definition}{subsection}

\subsection{Physical Constants} \label{app:Physical Constants}
	\begin{table}[h!]
		\centering
		\begin{tabular}{|c|c|c|}
			\hline
			\textbf{Constant Name} & \textbf{Variable Letter} & \textbf{Value} \\ \hline
			Boltzmann Constant & $R$ & $8.314 \si{\joule / \mole~\kelvin}$ \\ \hline
			Universal Gravitational & $G$ & $6.67408 \times 10^{-11} \si{\meter^{3}~\kilogram^{-1}~\second^{-2}}$ \\ \hline
			Planck's Constant & $h$ & $6.62607004 \times 10^{-34} \si{\meter \kilogram / \second}$ \\ \hline
			Speed of Light & $c$ & $299792458 \si{\meter / \second}$ \\ \hline
			Mass of Earth & $m_{Earth}$ & $5.972 \times 10^{24} \si{\kilogram}$ \\ \hline
			Diameter of Earth & $d_{Earth}$ & $12742 \si{\kilo\meter}$ \\ \hline
		\end{tabular}
	\end{table}
\subsection{Trigonometry} \label{app:Trig}
	\subsubsection{Trigonometric Formulas} \label{subsubsec:Trig Formulas}
		\begin{equation} \label{eq:Sin plus Sin with diff Angles}
			\sin \left( \alpha \right) + \sin \left( \beta \right) = 2 \sin \left( \frac{\alpha + \beta}{2} \right) \cos\left( \frac{\alpha - \beta}{2} \right)  
		\end{equation}
		\begin{equation} \label{eq:Cosine-Sine Product}
			\cos \left( \theta \right) \sin \left( \theta \right) = \frac{1}{2} \sin \left( 2 \theta \right)
		\end{equation}
	
	\subsubsection{Euler Equivalents of Trigonometric Functions} \label{subsubsec:Euler Equivalents}
		\begin{equation} \label{eq:Euler Sin}
			\sin \left( x \right) = \frac{e^{\imath x} + e^{-\imath x}}{2}
		\end{equation}
		\begin{equation} \label{eq:Euler Cos}
			\cos \left( x \right) = \frac{e^{\imath x} - e^{-\imath x}}{2 \imath}
		\end{equation}
		\begin{equation} \label{eq:Euler Sinh}
			\sinh \left( x \right) = \frac{e^{x} - e^{-x}}{2}
		\end{equation}
		\begin{equation} \label{eq:Euler Cosh}
			\cosh \left( x \right) = \frac{e^{x} + e^{-x}}{2}
		\end{equation}
\section{Calculus}\label{app:Calculus}
\subsection{L'Hopital's Rule}\label{subsec:LHopitals_Rule}
L'Hopital's Rule can be used to simplify and solve expressions regarding limits that yield irreconcialable results.
\begin{lemma}[L'Hopital's Rule]\label{lemma:LHopitals_Rule}
  If the equation
  \begin{equation*}
    \lim\limits_{x \rightarrow a} \frac{f(x)}{g(x)} =
    \begin{cases}
      \frac{0}{0} \\
      \frac{\infty}{\infty} \\
    \end{cases}
  \end{equation*}
  then \Cref{eq:LHopitals_Rule} holds.
  \begin{equation}\label{eq:LHopitals_Rule}
    \lim\limits_{x \rightarrow a} \frac{f(x)}{g(x)} = \lim\limits_{x \rightarrow a} \frac{f'(x)}{g'(x)}
  \end{equation}
\end{lemma}

\subsection{Fundamental Theorems of Calculus}\label{subsec:Fundamental Theorem of Calculus}
\begin{definition}[First Fundamental Theorem of Calculus]\label{def:1st Fundamental Theorem of Calculus}
  The \emph{first fundamental theorem of calculus} states that, if $f$ is continuous on the closed interval $\left[ a,b \right]$ and $F$ is the indefinite integral of $f$ on $\left[ a,b \right]$, then

  \begin{equation}\label{eq:1st Fundamental Theorem of Calculus}
    \int_{a}^{b}f \left( x \right) dx = F \left( b \right) - F \left( a \right)
  \end{equation}
\end{definition}

\begin{definition}[Second Fundamental Theorem of Calculus]\label{def:2nd Fundamental Theorem of Calculus}
  The \emph{second fundamental theorem of calculus} holds for $f$ a continuous function on an open interval $I$ and $a$ any point in $I$, and states that if $F$ is defined by

  \begin{equation*}
    F \left( x \right) = \int_{a}^{x} f \left( t \right) dt,
  \end{equation*}
  then
  \begin{equation}\label{eq:2nd Fundamental Theorem of Calculus}
    \begin{aligned}
      \frac{d}{dx} \int_{a}^{x} f \left( t \right) dt &= f \left( x \right) \\
      F' \left( x \right) &= f \left( x \right) \\
    \end{aligned}
  \end{equation}
\end{definition}

\begin{definition}[argmax]\label{def:argmax}
  The arguments to the \emph{argmax} function are to be maximized by using their derivatives.
  You must take the derivative of the function, find critical points, then determine if that critical point is a global maxima.
  This is denoted as
  \begin{equation*}\label{eq:argmax}
    \argmax_{x}
  \end{equation*}
\end{definition}

\subsection{Rules of Calculus}\label{subsec:Rules of Calculus}
\subsubsection{Chain Rule}\label{subsubsec:Chain Rule}
\begin{definition}[Chain Rule]\label{def:Chain Rule}
  The \emph{chain rule} is a way to differentiate a function that has 2 functions multiplied together.

  If
  \begin{equation*}
    f(x) = g(x) \cdot h(x)
  \end{equation*}
  then,
  \begin{equation}\label{eq:Chain Rule}
    \begin{aligned}
      f'(x) &= g'(x) \cdot h(x) + g(x) \cdot h'(x) \\
      \frac{df(x)}{dx} &= \frac{dg(x)}{dx} \cdot g(x) + g(x) \cdot \frac{dh(x)}{dx} \\
    \end{aligned}
  \end{equation}
\end{definition}

\subsection{Useful Integrals}\label{subsec:Useful_Integrals}
\begin{equation}\label{eq:Cosine_Indefinite_Integral}
  \int \cos(x) \; dx = \sin(x)
\end{equation}

\begin{equation}\label{eq:Sine_Indefinite_Integral}
  \int \sin(x) \; dx = -\cos(x)
\end{equation}

\begin{equation}\label{eq:x_Cosine_Indefinite_Integral}
  \int x \cos(x) \; dx = \cos(x) + x \sin(x)
\end{equation}
\Cref{eq:x_Cosine_Indefinite_Integral} simplified with Integration by Parts.

\begin{equation}\label{eq:x_Sine_Indefinite_Integral}
  \int x \sin(x) \; dx = \sin(x) - x \cos(x)
\end{equation}
\Cref{eq:x_Sine_Indefinite_Integral} simplified with Integration by Parts.

\begin{equation}\label{eq:x_Squared_Cosine_Indefinite_Integral}
  \int x^{2} \cos(x) \; dx = 2x \cos(x) + (x^{2} - 2) \sin(x)
\end{equation}
\Cref{eq:x_Squared_Cosine_Indefinite_Integral} simplified by using Integration by Parts twice.

\begin{equation}\label{eq:x_Squared_Sine_Indefinite_Integral}
  \int x^{2} \sin(x) \; dx = 2x \sin(x) - (x^{2} - 2) \cos(x)
\end{equation}
\Cref{eq:x_Squared_Sine_Indefinite_Integral} simplified by using Integration by Parts twice.

\begin{equation}\label{eq:Exponential_Cosine_Indefinite_Integral}
  \int e^{\alpha x} \cos(\beta x) \; dx = \frac{e^{\alpha x} \bigl( \alpha \cos(\beta x) + \beta \sin(\beta x) \bigr)}{\alpha^{2} + \beta^{2}} + C
\end{equation}

\begin{equation}\label{eq:Exponential_Sine_Indefinite_Integral}
  \int e^{\alpha x} \sin(\beta x) \; dx = \frac{e^{\alpha x} \bigl( \alpha \sin(\beta x) - \beta \cos(\beta x) \bigr)}{\alpha^{2}+\beta^{2}} + C
\end{equation}

\begin{equation}\label{eq:Exponential_Indefinite_Integral}
  \int e^{\alpha x} \; dx = \frac{e^{\alpha x}}{\alpha}
\end{equation}

\begin{equation}\label{eq:x_Exponential_Indefinite_Integral}
  \int x e^{\alpha x} \; dx = e^{\alpha x} \left( \frac{x}{\alpha} - \frac{1}{\alpha^{2}} \right)
\end{equation}
\Cref{eq:x_Exponential_Indefinite_Integral} simplified with Integration by Parts.

\begin{equation}\label{eq:Inverse_x_Indefinite_Integral}
  \int \frac{dx}{\alpha + \beta x} = \int \frac{1}{\alpha + \beta x} \; dx = \frac{1}{\beta} \ln (\alpha + \beta x)
\end{equation}

\begin{equation}\label{eq:Inverse_x_Squared_Indefinite_Integral}
  \int \frac{dx}{\alpha^{2} + \beta^{2} x^{2}} = \int \frac{1}{\alpha^{2} + \beta^{2} x^{2}} \; dx = \frac{1}{\alpha \beta} \arctan \left( \frac{\beta x}{\alpha} \right)
\end{equation}

\begin{equation}\label{eq:a_Exponential_Indefinite_Integral}
  \int \alpha^{x} \; dx = \frac{\alpha^{x}}{\ln(\alpha)}
\end{equation}

\begin{equation}\label{eq:a_Exponential_Derivative}
  \frac{d}{dx} \alpha^{x} = \frac{d\alpha^{x}}{dx} = \alpha^{x} \ln(x)
\end{equation}

\subsection{Leibnitz's Rule}\label{subsec:Leibnitzs_Rule}
\begin{lemma}[Leibnitz's Rule]\label{lemma:Leibnitzs_Rule}
  Given
  \begin{equation*}
    g(t) = \int_{a(t)}^{b(t)} f(x, t) \, dx
  \end{equation*}
  with $a(t)$ and $b(t)$ differentiable in $t$ and $\frac{\partial f(x, t)}{\partial t}$ continuous in both $t$ and $x$, then
  \begin{equation}\label{eq:Leibnitzs_Rule}
    \frac{d}{dt} g(t) = \frac{d g(t)}{dt} = \int_{a(t)}^{b(t)} \frac{\partial f(x, t)}{\partial t} \, dx + f \bigl[ b(t), t \bigr] \, \frac{d b(t)}{dt} - f \bigl[ a(t), t \bigr] \, \frac{d a(t)}{dt}
  \end{equation}
\end{lemma}


\section{Complex Numbers}\label{sec:Complex_Numbers}
\begin{definition}[Complex Number]\label{def:Complex_Number}
  A \emph{complex number} is a hyper real number system.
  This means that two real numbers, $a, b \in \RealNumbers$, are used to construct the set of complex numbers, denoted $\ComplexNumbers$.

  A complex number is written, in Cartesian form, as shown in \Cref{eq:Complex_Number} below.
  \begin{equation}\label{eq:Complex_Number}
    z = a \pm ib
  \end{equation}
  where
  \begin{equation}\label{eq:Imaginary_Value}
    i = \sqrt{-1}
  \end{equation}

  \begin{remark*}[$i$ vs. $j$ for Imaginary Numbers]
    Complex numbers are generally denoted with either $i$ or $j$.
    Electrical engineering regularly makes use of $j$ as the imaginary value.
    This is because alternating current $i$ is already taken, so $j$ is used as the imaginary value instad.
  \end{remark*}
\end{definition}

\subsection{Parts of a Complex Number}\label{subsec:Complex_Number_Parts}
A \nameref{def:Complex_Number} is made of up 2 parts:
\begin{enumerate}[noitemsep]
\item \nameref{def:Real_Part}
\item \nameref{def:Imaginary_Part}
\end{enumerate}

\begin{definition}[Real Part]\label{def:Real_Part}
  The \emph{real part} of an imaginary number, denoted with the $\Re$ operator, is the portion of the \nameref{def:Complex_Number} with no part of the imaginary value $i$ present.

  If $z = x + iy$, then
  \begin{equation}\label{eq:Real_Part}
    \Real{z} = x
  \end{equation}

  \begin{remark}[Alternative Notation]\label{rmk:Real_Part_Alternative_Notation}
    The \nameref{def:Real_Part} of a number sometimes uses a slightly different symbol for denoting the operation.
    It is:
    \begin{equation*}
      \mathfrak{Re}
    \end{equation*}
  \end{remark}
\end{definition}

\begin{definition}[Imaginary Part]\label{def:Imaginary_Part}
  The \emph{imaginary part} of an imaginary number, denoted with the $\Im$ operator, is the portion of the \nameref{def:Complex_Number} where the imaginary value $i$ is present.

  If $z = x + iy$, then
  \begin{equation}\label{eq:Imaginary_Part}
    \Imag{z} = y
  \end{equation}

  \begin{remark}[Alternative Notation]\label{rmk:Imaginary_Part_Alternative_Notation}
    The \nameref{def:Imaginary_Part} of a number sometimes uses a slightly different symbol for denoting the operation.
    It is:
    \begin{equation*}
      \mathfrak{Im}
    \end{equation*}
  \end{remark}
\end{definition}

\subsection{Binary Operations}\label{subsec:Binary_Operations}

%%% Local Variables:
%%% mode: latex
%%% TeX-master: shared
%%% End:


\subsection{Complex Conjugates}\label{app:Complex_Conjugates}
\begin{definition}[Complex Conjugate]\label{def:Complex_Conjugate}
  The conjugate of a complex number is called its \emph{complex conjugate}.
  The complex conjugate of a complex number is the number with an equal real part and an imaginary part equal in magnitude but opposite in sign.
  If we have a complex number as shown below,
  \begin{equation*}
    z = a \pm bi
  \end{equation*}

  then, the conjugate is denoted and calculated as shown below.
  \begin{equation}\label{eq:Complex_Conjugates}
    \Conjugate{z} = a \mp bi
  \end{equation}
\end{definition}

The \nameref{def:Complex_Conjugate} can also be denoted with an asterisk ($*$).
This is generally done for complex functions, rather than single variables.
\begin{equation}\label{eq:Complex_Conjugates_Asterisk}
  z^{*} = \Conjugate{z}
\end{equation}

%%% Local Variables:
%%% mode: latex
%%% TeX-master: shared
%%% End:


\subsection{Geometry of Complex Numbers}\label{subsec:Geometry_Complex_Numbers}
So far, we have viewed \nameref{def:Complex_Number}s only algebraically.
However, we can also view them geometrically as points on a 2 dimensional \nameref{def:Argand_Plane}.

\begin{definition}[Argand Plane]\label{def:Argand_Plane}
  An \emph{Argane Plane} is a standard two dimensional plane whose points are all elements of the complex numbers, $z \in \ComplexNumbers$.
  This is taken from Descarte's definition of a completely real plane.

  The Argand plane contains 2 lines that form the axes, that indicate the real component and the imaginary component of the complex number specified.
\end{definition}

A \nameref{def:Complex_Number} can be viewed as a point in the \nameref{def:Argand_Plane}, where the \nameref{def:Real_Part} is the ``$x$''-component and the \nameref{def:Imaginary_Part} is the ``$y$''-component.

By plotting this, you see that we form a right triangle, so we can find the hypotenuse of that triangle.
This hypotenuse is the distance the point $p$ is from the origin, refered to as the \nameref{def:Complex_Number_Modulus}.
\begin{remark*}
  When working with \nameref{def:Complex_Number}s geometrically, we refer to the points, where they are defined like so:
  \begin{equation*}
    z = x + iy = p(x, y)
  \end{equation*}

  Note that $p$ is \textbf{not} a function of $x$ and $y$.
  Those are the values that inform us \textbf{where} $p$ is located on the \nameref{def:Argand_Plane}.
\end{remark*}

\subsubsection{Modulus of a Complex Number}\label{subsubsec:Complex_Number_Modulus}
\begin{definition}[Modulus]\label{def:Complex_Number_Modulus}
  The \emph{modulus} of a \nameref{def:Complex_Number} is the distance from the origin to the complex point $p$.
  This is based off the Pythagorean Theorem.
  \begin{equation}\label{eq:Complex_Number_Modulus}
    \begin{aligned}
      {\lvert z \rvert}^{2} = x^{2} + y^{2} &= z \Conjugate{z} \\
      \lvert z \rvert &= \sqrt{x^{2} + y^{2}}
    \end{aligned}
  \end{equation}
\end{definition}

\begin{propertylist}
\item The \emph{Law of Moduli} states that $\lvert z w \rvert = \lvert z \rvert \lvert w \rvert$.\label{prop:Law_of_Moduli}.
\end{propertylist}

We can prove \Cref{prop:Law_of_Moduli} using an algebraic identity.
\begin{proof}[Prove \Cref*{prop:Law_of_Moduli}]
  Let $z$ and $w$ be complex numbers ($z, w \in \ComplexNumbers$).
  We are asked to prove
  \begin{equation*}
    \lvert z w \rvert = \lvert z \rvert \lvert w \rvert
  \end{equation*}

  But, it is actually easier to prove
  \begin{equation*}
    {\lvert z w \rvert}^{2} = {\lvert z \rvert}^{2} {\lvert w \rvert}^{2}
  \end{equation*}

  We start by simplifying the ${\lvert z w \rvert}^{2}$ equation above.
  \begin{align*}
    {\lvert z w \rvert}^{2} &= {\lvert z \rvert}^{2} {\lvert w \rvert}^{2} \\
    \intertext{Using the definition of the \nameref{def:Complex_Number_Modulus} of a \nameref{def:Complex_Number} in \Cref{eq:Complex_Number_Modulus}, we can expand the modulus.}
                            &= (z w) (\Conjugate{z w}) \\
    \intertext{Using \Cref{prop:Complex_Conjugate_Split} for multiplication allows us to do the next step.}
                            &= (z w) (\Conjugate{z} \Conjugate{w}) \\
    \intertext{Using Multiplicative Associativity and Multiplicative Commutativity, we can simplify this further.}
                            &= (z \Conjugate{z}) (w \Conjugate{w}) \\
                            &= {\lvert z \rvert}^{2} {\lvert w \rvert}^{2}
  \end{align*}

  Note how we never needed to define $z$ or $w$, so this is as general a result as possible.
\end{proof}

\paragraph{Algebraic Effects of the Modulus' \Cref*{prop:Law_of_Moduli}}\label{par:Law_of_Moduli-Algebraic_Effects}
For this section, let $z = x_{1} + iy_{1}$ and $w = x_{2} + iy_{2}$.
Now,
\begin{align*}
  z w &= (x_{1}x_{2} - y_{1}y_{2}) + i(x_{1}y_{2} + x_{2}y_{1}) \\
  {\lvert z w \rvert}^{2} &= {(x_{1}x_{2} - y_{1}y_{2})}^{2} + {(x_{1}y_{2} + x_{2}y_{1})}^{2} \\
      &= \left( x_{1}^{2} + x_{2}^{2} \right) \left( x_{2}^{2} + y_{2}^{2} \right) \\
      &= {\lvert z \rvert}^{2} {\lvert w \rvert}^{2}
\end{align*}

However, the Law of Moduli (\Cref{prop:Law_of_Moduli}) does \textbf{not} hold for a hyper complex number system one that uses 2 or more imaginaries, i.e.\ $z = a + iy + jz$.
But, the Law of Moduli (\Cref{prop:Law_of_Moduli}) \textbf{does} hold for hyper complex number system that uses 3 imaginaries, $a = z + iy + jz + k \ell$.

\paragraph{Conceptual Effects of the Modulus' \Cref*{prop:Law_of_Moduli}}\label{par:Law_of_Moduli-Conceptual_Effects}
We are interested in seeing if $\lvert z w \rvert = (x_{1}^{2} + y_{1}^{2})(x_{2}^{2}+y_{2}^{2})$ can be extended to more complex terms (3 terms in the complex number).

However, Langrange proved that the equation below \textbf{always} holds.
Note that the $z$ below has no relation to the $z$ above.
\begin{equation*}
  (x_{1} + y_{1} + z_{1}) \neq X^{2} + Y^{2} + Z^{2}
\end{equation*}

%%% Local Variables:
%%% mode: latex
%%% TeX-master: shared
%%% End:


\subsection{Circles and Complex Numbers}\label{subsec:Circles_Complex_Numbers}
We need to define both a center and a radius, just like with regular purely real values.
\Cref{eq:Circles_Complex_Numbers} defines the relation required for a circle using \nameref{def:Complex_Number}s.
\begin{equation}\label{eq:Circles_Complex_Numbers}
  \lvert z - a \rvert = r
\end{equation}

\begin{example}[Lecture 2, Example 1]{Convert to Circle}
  Given the expression below, find the location of the center of the circle and the radius of the circle?
  \begin{equation*}
    \lvert 5 iz + 10 \rvert = 7
  \end{equation*}
  \tcblower{}
  This is just a matter of simplification and moving terms around.
  \begin{align*}
    \lvert 5 iz + 10 \rvert &= 7 \\
    \lvert 5i (z + \frac{10}{5i}) \rvert &= 7 \\
    \lvert 5i (z + \frac{2}{i}) \rvert &= 7 \\
    \lvert 5i (z + \frac{2}{i} \frac{-i}{-i}) \rvert &= 7 \\
    \lvert 5i (z - 2i) \rvert &= 7 \\
    \intertext{Now using the Law of Moduli (\Cref{prop:Law_of_Moduli}) $\lvert a b \rvert = \lvert a \rvert \lvert b \rvert$, we can simplify out the extra imaginary term.}
    \lvert 5i \rvert \lvert z-2i \rvert &= 7 \\
    5 \lvert z - 2i \rvert &= 7 \\
    \lvert z - 2i \rvert = \frac{7}{5}
  \end{align*}

  Thus, the circle formed by the equation $\lvert 5 iz + 10 \rvert = 7$ is actually $\lvert z - 2i \rvert = \frac{7}{5}$, with a center at $a = 2i$ and a radius of $\frac{7}{5}$.
\end{example}

\subsubsection{Annulus}\label{subsubsec:Annulus}
\begin{definition}[Annulus]\label{def:Annulus}
  An \emph{annulus} is a region that is bounded by 2 concentric circles.
  This takes the form of \Cref{eq:Annulus}.
  \begin{equation}\label{eq:Annulus}
    r_{1} \leq \lvert z - a \rvert \leq r_{2}
  \end{equation}

  In \Cref{eq:Annulus}, each of the $\leq$ symbols could also be replaced with $<$.
  This leads to 3 different possibilities for the annulus:
  \begin{enumerate}[noitemsep]
  \item If both inequality symbols are $\leq$, then it is a \textbf{Closed Annulus}.
  \item If both inequality symbols are $<$, then it is an \textbf{Open Annulus}.
  \item If \textbf{only one} inequality symbol $<$ and the other $\leq$, then it is not an \textbf{Open Annulus}.
  \end{enumerate}
\end{definition}


%%% Local Variables:
%%% mode: latex
%%% TeX-master: shared
%%% End:



%%% Local Variables:
%%% mode: latex
%%% TeX-master: shared
%%% End:

\end{document}