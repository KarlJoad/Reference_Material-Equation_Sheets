\documentclass[10pt,letterpaper,final,twoside,notitlepage]{article}
\usepackage[margin=.5in]{geometry} % 1/2 inch margins on all pages
\usepackage[utf8]{inputenc} % Define the input encoding
\usepackage[USenglish]{babel} % Define language used
\usepackage{amsmath}
\usepackage{mathtools} % Allow for text and math in align* environment.
\usepackage{amsfonts}
\usepackage{amssymb}
\usepackage{amsthm} % Gives us plain, definition, and remark to use in \theoremstyle{style}
\usepackage{thmtools}
\usepackage{thm-restate}
\usepackage{graphicx}

\usepackage[
backend=biber,
style=alphabetic,
citestyle=authoryear]{biblatex} % Must include citation somewhere in document to print bibliography
\usepackage{hyperref} % Generate hyperlinks to referenced items
\usepackage[nottoc]{tocbibind} % Prints the Reference/Bibliography in TOC as well
\usepackage[noabbrev,nameinlink]{cleveref} % Fancy cross-references in the document everywhere
\usepackage{nameref} % Can make references by name to places
\usepackage{caption} % Allows for greater control over captions in figure, algorithm, table, etc. environments
\usepackage{subcaption} % Allows for multiple figures in one Figure environment
\usepackage[binary-units=true]{siunitx} % Gives us ways to typeset units for stuff
\usepackage{csquotes} % Context-sensitive quotation facilities
\usepackage{enumitem} % Provides [noitemsep, nolistsep] for more compact lists
\usepackage{chngcntr} % Allows us to tamper with the counter a little more
\usepackage{empheq} % Allow boxing of equations in special math environments
\usepackage[x11names]{xcolor} % Gives access to coloring text in environments or just text, MUST be before tikz
\usepackage{tcolorbox} % Allows us to create boxes of various types for examples
\usepackage{tikz} % Allows us to create TikZ and PGF Pictures
\usepackage{ctable} % Greater control over tables and how they look
\usepackage{multirow} % Allow us to have a single cell in a table span multiple rows
\usepackage{titling} % Put document information throughout the document programmatically
\usepackage[linesnumbered,ruled,vlined]{algorithm2e} % Allows us to write algorithms in a nice style.

\counterwithin{figure}{section}
\counterwithin{table}{section}
\counterwithin{equation}{section}
\counterwithin{algocf}{section}
\crefname{algocf}{algorithm}{algorithms}
\Crefname{algocf}{Algorithm}{Algorithms}
\setcounter{secnumdepth}{4}
\setcounter{tocdepth}{4} % Include \paragraph in toc
\crefname{paragraph}{paragraph}{paragraphs}
\Crefname{paragraph}{Paragraph}{Paragraphs}

% Create a theorem environment
\theoremstyle{plain}
\newtheorem{theorem}{Theorem}[section]
% Create a numbered theorem-like environment for lemmas
\newtheorem{lemma}{Lemma}[theorem]

% Create a definition environment
\theoremstyle{definition}
\newtheorem{definition}{Defn}
\newtheorem{corollary}{Corollary}[section]
% \begin{definition}[Term] \label{def:}
% 		Make sure the term is emphasized with \emph{term}.
%		This ensures that if \emph is changed, it shows up everywhere
% \end{definition}

% Create a numbered remark environment numbered based on definition
% NOTE: This version of remark MUST go inside a definition environment
\theoremstyle{remark}
\newtheorem{remark}{Remark}[definition]
%\counterwithin{definition}{subsection} % Uncomment to have definitions use section.subsection numbering

% Create an unnumbered remark environment for general use
% NOTE: This version of remark has NO restrictions on placement
\newtheorem*{remark*}{Remark}

% Create a special list that handles properties. It can be broken and restarted
\newlist{propertylist}{enumerate}{1} % {Name}{Template}{Max Depth}
% [newlistname, LevelsToApplyTo]{formatting options}
\setlist[propertylist, 1]{label=\textbf{(\roman*)}, ref=\textbf{(\roman*)}, noitemsep, nolistsep}
\crefname{propertylisti}{property}{properties}
\Crefname{propertylisti}{Property}{Properties}

% Create a special list that handles enumerate starting with lower letters. Breakable/Restartable.
\newlist{boldalphlist}{enumerate}{1} % {Name}{Template}{Max Depth}
% [newlistname, LevelsToApplyTo]{formatting options}
\setlist[boldalphlist, 1]{label=\textbf{(\alph*)}, ref=\alph*, noitemsep, nolistsep} % Set options

\newlist{nocrefenumerate}{enumerate}{1} % {Name}{Template}{Max Depth}
% [newlistname, LevelsToApplyTo]{formatting options}
\setlist[nocrefenumerate, 1]{label=(\arabic*), ref=(\arabic*), noitemsep, nolistsep}

% Create a list that allows for deeper nesting of numbers. By default enumerate only allows depth=4.
\newlist{nestednums}{enumerate}{6}
% [newlistname, LevelsToApplyTo]{formatting options}
\setlist[nestednums]{noitemsep, label*=\arabic*.}

\tcbuselibrary{breakable} % Allow tcolorboxes to be broken across pages
% Create a tcolorbox for examples
% /begin{example}[extra name]{NAME}
% Create a tcolorbox for examples
% Argument #1 is optional, given by [], that is the textbook's problem number
% Argument #2 is mandatory, given by {}, that is the title for the example
% Avoid putting special characters, (), [], {}, ",", etc. in the title.
\newtcolorbox[auto counter,
number within=section,
number format=\arabic,
crefname={example}{examples}, % Define reference format for cref (No Capitals)
Crefname={Example}{Examples}, % Reference format for cleveref (With Capitals)
]{example}[2][]{ % The [2][] Means the first argument is optional
  width=\textwidth,
  title={Example \thetcbcounter: #2. #1}, % Parentheses and commas are not well supported
  fonttitle=\bfseries,
  label={ex:#2},
  nameref=#2,
  colbacktitle=white!100!black,
  coltitle=black!100!white,
  colback=white!100!black,
  upperbox=visible,
  lowerbox=visible,
  sharp corners=all,
  breakable
}

% Create a tcolorbox for general use
\newtcolorbox[% auto counter,
% number within=section,
% number format=\arabic,
% crefname={example}{examples}, % Define reference format for cref (No Capitals)
% Crefname={Example}{Examples}, % Reference format for cleveref (With Capitals)
]{blackbox}{
  width=\textwidth,
  % title={},
  fonttitle=\bfseries,
  % label={},
  % nameref=,
  colbacktitle=white!100!black,
  coltitle=black!100!white,
  colback=white!100!black,
  upperbox=visible,
  lowerbox=visible,
  sharp corners=all
}

% Redefine the 'end of proof' symbol to be a black square, not blank
\renewcommand\qedsymbol{$\blacksquare$} % Change proofs to have black square at end

\renewcommand{\Re}{\operatorname{Re}} % Redefine to use the command, but not the fraktur version
\renewcommand{\Im}{\operatorname{Im}} % Redefine to use the command, but not the fraktur version
% Math Operators that are useful to abstract the written math away to one spot
\DeclareMathOperator{\RealNumbers}{\mathbb{R}}
\newcommand{\TextRealNumbers}{$\RealNumbers$}
\DeclareMathOperator{\AllIntegers}{\mathbb{Z}}
\newcommand{\TextAllIntegers}{$\AllIntegers$}
\DeclareMathOperator{\PositiveInts}{\mathbb{Z}^{+}}
\newcommand{\TextPositiveInts}{$\PositiveInts$}
\DeclareMathOperator{\NegativeInts}{\mathbb{Z}^{-}}
\newcommand{\TextNegativeInts}{$\NegativeInts$}
\DeclareMathOperator{\NaturalNumbers}{\mathbb{N}}
\newcommand{\TextNaturalNumbers}{$\NaturalNumbers$}
\DeclareMathOperator{\ComplexNumbers}{\mathbb{C}}
\newcommand{\TextComplexNumbers}{$\ComplexNumbers$}
\DeclareMathOperator{\RationalNumbers}{\mathbb{Q}}
\newcommand{\TextRationalNumbers}{$\RationalNumbers$}
\DeclareMathOperator*{\argmax}{argmax} % Thin Space and subscripts are UNDER in display
\DeclareMathOperator{\Lapl}{\mathcal{L}} % Declare a Laplace symbol to be used
\DeclareMathOperator{\UnitStep}{\mathcal{U}}
\DeclareMathOperator{\sinc}{sinc} % sinc(x) = (sin(pi x)/(pi x))
\DeclareMathOperator{\XOR}{\oplus}

\newcommand{\rbpRegister}{\texttt{\%rbp}}
\newcommand{\rspRegister}{\texttt{\%rsp}}
\newcommand{\ripRegister}{\texttt{\%rip}}
\newcommand{\raxRegister}{\texttt{\%rax}}
\newcommand{\rbxRegister}{\texttt{\%rbx}}

%%% Local Variables:
%%% mode: latex
%%% TeX-master: shared
%%% End:


% These packages are more specific to certain documents, but will be availabe in the template
% \usepackage{esint} % Provides us with more types of integral symbols to use
% \usepackage[outputdir=./TeX_Output]{minted} % Allow us to nicely typeset 300+ programming languages
% This document must be compiled with the -shell-escape flag if the packages above are uncommented

% \graphicspath{{./Drawings/Course}} % Uncomment this to use pictures in this document
% \addbibresource{./Bibliographies/CourseNum-Name.bib}

% Math Operators that are useful to abstract the written math away to one spot
% These are supposed to be document-specific mathematical operators that will make life easier
% Many fundamental operators are defined in Reference_Sheet_Preamble.tex
\DeclareMathOperator{\Prob}{\operatorname{P}}
\DeclareMathOperator{\Given}{\vert}
\DeclareMathOperator{\ExpectedValue}{\operatorname{\mathbb{E}}}
\DeclareMathOperator{\Variance}{\operatorname{VAR}}
\DeclareMathOperator{\StdDev}{\operatorname{STD}}
\DeclareMathOperator{\Covariance}{\operatorname{Cov}}
\DeclareMathOperator{\CorrCoeff}{\rho}

\newcommand{\SetOrder}[1]{\lvert #1 \rvert}

\begin{titlepage}
  \title{ETSN10: Network Architecture and Performance - Reference Sheet}
  \author{Karl Hallsby}
  \date{Last Edited: \today} % We want to inform people when this document was last edited
\end{titlepage}

\begin{document}
\pagenumbering{gobble}
\maketitle
\pagenumbering{roman} % i, ii, iii on beginning pages, that don't have content
\tableofcontents
\clearpage
\listoftheorems[ignoreall, show={definition, Definition}]
\clearpage
\pagenumbering{arabic} % 1,2,3 on content pages

\section{Networking Review}\label{sec:Networking_Review}
This course assumes that you already know about basic networking terms and models.
However, this section is meant as a refresher for people who have already taken that particular set of courses, or as a quick introduction for those who have not taken those courses yet.

\subsection{Networking Stack}\label{subsec:Networking_Stack}
\begin{definition}[Networking Stack]\label{def:Networking_Stack}
  The \emph{networking stack} is a set of layers that are built on top of another that are used to implement inter-computer communication.
  There are 5 layers:
  \begin{enumerate}[noitemsep]
  \item \nameref{def:Physical_Layer}
  \item \nameref{def:Data_Link_Layer}
  \item \nameref{def:Network_Layer}
  \item \nameref{def:Transport_Layer}
  \item \nameref{def:Application_Layer}
  \end{enumerate}

  Each of these fulfill different roles, allowing for different implementations to be used interchangeably.
  Additionally, not all points in the network graph require the entire networking stack.
  For example, a router or switch does not require the \nameref{def:Transport_Layer} or the \nameref{def:Application_Layer}, since the job of this hardware is to just move packets around.
\end{definition}

Each layer of the networking stack add a header and possibly a footer.
Information in headers and footers includes source and destination addresses, checksums, packet size, and protocol identifiers.

\begin{definition}[Encapsulation]\label{def:Encapsulation}
  The process of adding headers and footers is called \emph{encapsulation}.
\end{definition}

\begin{definition}[Decapsulation]\label{def:Decapsulation}
  The process of removing the headers and footers of a \nameref{def:Packet} at the other end is called \emph{decapsulation}.
\end{definition}

%%% Local Variables:
%%% mode: latex
%%% TeX-master: "../../ETSN10-Network_Architecture_Performance-Reference_Sheet"
%%% End:


\subsection{Physical Layer}\label{subsec:Physical_Layer}
\begin{definition}[Physical Layer]\label{def:Physical_Layer}
  The \emph{physical layer} in the \nameref{def:Networking_Stack} is the lowest layer in the stack.
  It consists of the physical connection that is made between computers and the modulation and coding to represent data as a signal on that connection.
  For obvious reasons, all devices \textbf{must} implement this layer.

  Some examples of this layer are:
  \begin{itemize}[noitemsep]
  \item Copper twisted-pair cables
  \item Fiber optic wires
  \item Radio transmission
  \end{itemize}

  \begin{remark}
    The design of different \nameref{def:Physical_Layer}s is more a hardware design question than that of a software implementation issue.
    So, it is not discussed in this course, beyond the use of cellular networks.
  \end{remark}
\end{definition}

%%% Local Variables:
%%% mode: latex
%%% TeX-master: "../../ETSN10-Network_Architecture_Performance-Reference_Sheet"
%%% End:


\subsection{Data-Link Layer}\label{subsec:Data_Link_Layer}
\begin{definition}[Data-Link Layer]\label{def:Data_Link_Layer}
  The \emph{data-link layer} is the next layer in the \nameref{def:Networking_Stack}.
  All devices must implement this layer.
  The data-link layer is responsible for getting data between two devices that have a \nameref{def:Physical_Layer} connection between them.
  It is also known as the \emph{Medium Access Control (MAC) layer}.
  It controls when each device should use the link, and may also include error correction and retransmission.

  Some examples of these are:
  \begin{itemize}[noitemsep]
  \item Ethernet
  \item Point-to-Point Protocol (PPP)
  \item 802.11 (WiFi)
  \item Bluetooth
  \end{itemize}
\end{definition}

%%% Local Variables:
%%% mode: latex
%%% TeX-master: "../../ETSN10-Network_Architecture_Performance-Reference_Sheet"
%%% End:


\subsection{Network Layer}\label{subsec:Network_Layer}
\begin{definition}[Network Layer]\label{def:Network_Layer}
  The \emph{network layer} is the third layer in the \nameref{def:Networking_Stack}.
  This is the last layer that every device must have.
  It is responsible for end-to-end data delivery between two devices (hosts) anywhere on the network.
  It is responsible for routing and forwarding.

  Some examples of this layer are:
  \begin{itemize}[noitemsep]
  \item Internet Protocol (IP)
  \item Internet Control Message Protocol (ICMP)
  \item \nameref{def:Routing_Protocol}s
  \end{itemize}
\end{definition}

\begin{definition}[Routing]\label{def:Routing}
  \emph{Routing} is the process of determining and selecting the best end-to-end path through a network.

  There are algorithms designed to discover the other participants in the network by analyzing the grpah that is created when each participant is considered a node in this graph.
\end{definition}

\begin{definition}[Forwarding]\label{def:Forwarding}
  \emph{Forwarding} is the process of selecting the next hop for the given packet.
\end{definition}

\begin{definition}[Circuit Switching]\label{def:Circuit_Switching}
  In \emph{circuit switching}, dedicated resources are allocated end-to-end for a particular flow of data.
  No other flows may use those resources while the connection is maintained.
  Meaning, that during a connection, there is a single dedicated circuit between the host and the receiver.
\end{definition}

\begin{definition}[Packet Switching]\label{def:Packet_Switching}
  In \emph{packet switching}, no resources are allocated to a flow, and each flow’s data is broken up into discrete packets.
  Each \nameref{def:Packet} is routed and forwarded independently.
\end{definition}

\begin{definition}[Packet]\label{def:Packet}
  A connection from one device to another is a stream.
  However, to effectively implement many things in the \nameref{def:Networking_Stack}, the stream must be broken into several discrete \emph{packets}.
\end{definition}

\subsubsection{Discovering Routing Information}\label{subsubsec:Discovering_Route_Info}
\paragraph{Dijkstra's Algorithm}\label{par:Dijkstras_Algorithm}
\nameref{algo:Dijkstras_Algorithm} starts at the source and builds up the shortest path step-by-step.

There are 2 main problems with \nameref{algo:Dijkstras_Algorithm}:
\begin{enumerate}[noitemsep]
\item The weights along the edges of the graph cannot be negative.
  \begin{itemize}[noitemsep]
  \item If this were to occur, then we would have an infinite cycle going through the negative edge.
  \end{itemize}
\item We \textbf{must} know the entire network's topology before beginning the algorithm.
  \begin{itemize}[noitemsep]
  \item Typically, this is not possible on vast networks, like the Internet.
  \end{itemize}
\end{enumerate}

\begin{algorithm}[H]
  \DontPrintSemicolon{}
  \SetKwInOut{Input}{Input}\SetKwInOut{Output}{Output}
  \SetKw{OR}{\textbf{OR}}
  \Input{A graph with weights on each edge representing ``distance'' between the nodes.}
  \Output{The path through the graph with the least weight from some starting node.}
  \BlankLine{}

   Assign tentative distance estimate to each node: 0 for the initial node, $\infty$ for all other nodes. \;
   Mark all nodes except the initial node as unvisited. \;
   Set the initial node as current node. \;
   \For{The current node's unvisited neighbours}{
     Calculate their distance. \;
     Compare this new distance against the current distance. \;
     Take whichever distance is smaller and use that as the new value. \;
     Mark the current node as visited. \;
     \If{Reached target node \OR{} No more unvisited nodes with distance $< \infty$}{
       Stop and terminate the algorithm. \;
     }{
       Set the unvisited node with the smallest distance as the new current node. \;
     }
   }
  \caption{Dijkstra's Algorithm}
  \label{algo:Dijkstras_Algorithm}
\end{algorithm}

\paragraph{Bellman-Ford Algorithm}\label{par:Bellman_Ford_Algorithm}
The Bellman-Ford algorithm works with graphs that have negative weight values on edges.
It also \textbf{does not} require us to know the entire network's topology before beginning.
However, it is much slower than \nameref{algo:Dijkstras_Algorithm}.

\begin{algorithm}[H]
  \DontPrintSemicolon{}
  \SetKwInOut{Input}{Input}\SetKwInOut{Output}{Output}
  \SetKw{AND}{\textbf{AND}}
  \Input{A graph with weights on each edge representing ``distance'' between the nodes.}
  \Output{The path through the graph with the least weight from some starting node.}
  \BlankLine{}

  Set the distance for the source node to 0 and for all other nodes to $\infty$. \;
  Set the predecessor of all nodes to null. \;
  \For{$n \in N$ where $N$ is the number of nodes \AND{} $n \leq N-1$}{
    We repeat $N-1$ times, because $N-1$ is the maximum length of a non-cyclic path. \;
    \For{Each edge $(u, v)$}{
      \If{distance to $u$, plus the edge weight $<$ current distance to $v$}{
        We have discovered a shorter path to $v$. \;
        Set the distance to $v$ to this new value and the predecessor of $v$ to $u$. \;
      }
    }
  }
  
  \For{each edge $(u, v)$}{
    \If{distance to $u$ plus the edge weight $<$ the distance to $v$}{
      The graph contains a negative-weight cycle. \;
      Terminate. \;
    }{
      We have found the shortest paths to each node from the source. \;
      Terminate. \;
    }
  }
  \caption{Bellman-Ford Algorithm}
  \label{algo:Bellman_Ford_Algorithm}
\end{algorithm}

\subsubsection{Routing Information}\label{subsubsec:Routing_Info}
\begin{definition}[Routing Protocol]\label{def:Routing_Protocol}
  A \emph{routing protocol} is used to construct a \nameref{def:Routing_Table} in each node.
\end{definition}

\begin{definition}[Routing Table]\label{def:Routing_Table}
  The \emph{routing table} is a table that contains entries on the shortest distances through the current network graph.
  This is used to help determine where to next forward packets onto.

  Each node has a routing table that contains the cost to the destination and the next hop for each destination.
  For a valid routing table, we need to have:
  \begin{enumerate}[noitemsep]
  \item Which destination the packets are going to?
  \item What is the next hop after the destination?
  \item The ``metric'' or weight of that particular path through the network graph?
    \begin{itemize}[noitemsep]
    \item We need this term so we can update the routing table as we discover new routes.
    \item This way if a new route with a lower cost presents itself, we can replace the higher-cost one with the lower one.
    \end{itemize}
  \end{enumerate}
\end{definition}

There are 2 ways to collect the routing information necessary to properly route packets through a network.
These 2 routing tables are:
\begin{enumerate}[noitemsep]
\item \nameref{par:Static_Routing_Tables}
\item \nameref{par:Dynamic_Routing_Tables}
\end{enumerate}

\paragraph{Static Routing Tables}\label{par:Static_Routing_Tables}
\begin{definition}[Static Routing Table]\label{def:Static_Routing_Table}
  \emph{Static routing table}s are manually configured by a user or a program before the system begins routing the information.
\end{definition}

This means that there is no overhead to attempt to figure out the network's topology.
The information needed has already been given to the device's \nameref{def:Routing_Table}, so that step is completed.

\paragraph{Dynamic Routing Tables}\label{par:Dynamic_Routing_Tables}
\begin{definition}[Dynamic Routing Table]\label{def:Dynamic_Routing_Table}
  A \emph{dynamic routing table} is a \nameref{def:Routing_Table} that is built automatically by the device.
  To do this, a \nameref{def:Routing_Protocol} is used to build the table while the device is running.
  This is done using either:
  \begin{enumerate}[noitemsep]
  \item \nameref{def:Distance_Vector_Routing_Protocol}
  \item \nameref{def:Link_State_Routing_Protocol}
  \end{enumerate}
\end{definition}

\begin{definition}[Distance Vector Routing Protocol]\label{def:Distance_Vector_Routing_Protocol}
  The \emph{distance vector routing algorithm} is a \nameref{def:Routing_Protocol} that uses \nameref{algo:Bellman_Ford_Algorithm} to construct a \nameref{def:Routing_Table}.
  
  Each node only begins with knowledge of their immediate neighbours and the costs to reach them.
  \begin{itemize}[noitemsep]
  \item Nodes then send this information (the routing table) to their neighbours.
  \item If a neighbour sends us a route that is shorter than one we already have, update our table to reflect this.
  \item After updating, send the new table to our neighbours.
  \item If a node goes down, discard any lines in the routing table that have it as the next hop and follow the above to find a new route.
  \end{itemize}
\end{definition}

\begin{definition}[Link State Routing Protocol]\label{def:Link_State_Routing_Protocol}
  The \emph{link state routing protocol} is a \nameref{def:Routing_Protocol} that uses \nameref{algo:Dijkstras_Algorithm} to construct a \nameref{def:Routing_Table}.
  Thus, we need to know what the whole network looks like.
  
  \begin{itemize}[noitemsep]
  \item Each node floods the network with the list of nodes it can connect to and the costs to them.
  \item Every node builds up a picture of the entire network, then can use \nameref{algo:Dijkstras_Algorithm} to determine the shortest path to each destination.
  \item The \nameref{def:Routing_Table} is then constructed based on the computed shortest paths.
  \end{itemize}
\end{definition}

However, the problem with both \nameref{def:Distance_Vector_Routing_Protocol} and \nameref{def:Link_State_Routing_Protocol} is that they do not scale well.
However, to help with this, both protocols are contained within a single \nameref{def:Autonomous_System}.

\begin{definition}[Autonomous System]\label{def:Autonomous_System}
  An \emph{autonomous system} is a smaller network, like a business or home, that performs a \nameref{def:Routing_Protocol} upon itself.
  It is done this way because both the \nameref{def:Distance_Vector_Routing_Protocol} and \nameref{def:Link_State_Routing_Protocol} do not scale to Internet-sized networks well.

  To combat this, each of the autonomous systems has a \nameref{def:Speaker_Node} that speaks to the outside world.
  This speaker node is used in the \nameref{def:Path_Vector_Routing_Protocol}.
\end{definition}

\begin{definition}[Speaker Node]\label{def:Speaker_Node}
  A \emph{speaker node} is a specially designated node within a single \nameref{def:Autonomous_System} that communicates \textbf{only with other speaker nodes}.
  Then, the \nameref{def:Path_Vector_Routing_Protocol} is performed upon all the speaker nodes to construct a wider network graph consisting only of the \nameref{def:Autonomous_System}s that the speaker nodes belong to.
\end{definition}

\begin{definition}[Path Vector Routing Protocol]\label{def:Path_Vector_Routing_Protocol}
  Between \nameref{def:Autonomous_System}s, we use a variant of \nameref{def:Distance_Vector_Routing_Protocol} called \emph{path vector routing protocol}.
  Each \nameref{def:Autonomous_System} has its own \nameref{def:Speaker_Node}.
  Only the \nameref{def:Speaker_Node}s can communicate across the \nameref{def:Autonomous_System} boundary, and exchange information about which destinations they can reach and the paths to them.
\end{definition}

\subsubsection{IP Addresses}\label{subsubsec:IP_Addresses}
\begin{definition}[IP Address]\label{def:IP_Address}
  \emph{IP Address}es are the unique end-point routing identifiers used on \nameref{def:Packet}s to deliver their data.
  A natural analogy is that of apartment numbers on apartment buildings.

  IPv4 uses 32 bits, split up into 4 8-bit chunks.
  Each chunk is read as its decimal equivalent, for humans.
  IPv6 uses 128 bits, where every 16 bits are interpreted as a hexadecimal number.
\end{definition}

\begin{definition}[Subnet]\label{def:Subnet}
  All hosts that are in the subnet will have \nameref{def:IP_Address}es that match the number of bits that are present in the subnet.
  Keeping with the apartment analogy, the subnet is like the address of the apartment building.

  For example, in \texttt{192.168.2.0/24}, the \texttt{24} means the first 24 bits of the subnet are 1's (\texttt{255.255.255.0}).
  So, the first 24 bits, or 3 \nameref{def:IP_Address} blocks (\texttt{192.168.2}), will remain the same for every client in that subnet.
\end{definition}

\begin{definition}[Classless Inter-Domain Routing]\label{def:Classless_Interdomain_Routing}
  \emph{Classless Inter-Domain Routing} or \emph{CIDR} (pronounced ``cider'') is a hierarchical way to organize \nameref{def:IP_Address}es.
  
  This allowed:
  \begin{itemize}[noitemsep]
  \item \nameref{def:Subnet}s to be of any length.
  \item Destinations in a router’s routing and forwarding tables may be full \nameref{def:IP_Address}es or \nameref{def:Subnet}s.
    \begin{itemize}[noitemsep]
    \item A destination \nameref{def:IP_Address} will then be matched to the most specific destination in the table when making forwarding decisions.
    \end{itemize}
  \end{itemize}
\end{definition}

%%% Local Variables:
%%% mode: latex
%%% TeX-master: "../../ETSN10-Network_Architecture_Performance-Reference_Sheet"
%%% End:


\subsection{Transport Layer}\label{subsec:Transport_Layer}
\begin{definition}[Transport Layer]\label{def:Transport_Layer}
  The \emph{transport layer} is built where the \nameref{def:Network_Layer} has delivered all the data to the end host.
  This means that only the source and destination hosts have this layer.
  So, routers and switches do not have this, but your phone, laptop, and the server you're connecting to do.
  
  This layer may be responsible for:
  \begin{itemize}[noitemsep]
  \item \nameref{def:Connection_Oriented_Communication}
  \item Multiplexing different data flows
  \item Reliable data delivery
  \item Data flow control
  \item Data congestion control
  \end{itemize}
  Some implementation of this layer do not handle all of these functions, but all \textbf{must} handle the multiplexing of different data flows.

  There are 2 main transport layers in-use today:
  \begin{enumerate}[noitemsep]
  \item \nameref{def:Transmission_Control_Protocol}
  \item \nameref{def:User_Datagram_Protocol}
  \end{enumerate}
\end{definition}

\begin{definition}[Connection-Oriented Communication]\label{def:Connection_Oriented_Communication}
  A \emph{connection-oriented communication} system means that a connection must be established before any data is transferred.
  In addition, this connection must be maintained throughout the transmission of data.

  \begin{remark}[Connectionless Communication]\label{rmk:Connectionless_Communication}
    In a \emph{connectionless communication} system, hosts can send data at any time without prior connection being made.
  \end{remark}
\end{definition}

\begin{definition}[Transmission Control Protocol]\label{def:Transmission_Control_Protocol}
  \textbf{TODO!}
\end{definition}

\begin{definition}[User Datagram Protocol]\label{def:User_Datagram_Protocol}
  \textbf{TODO!}
\end{definition}

%%% Local Variables:
%%% mode: latex
%%% TeX-master: "../../ETSN10-Network_Architecture_Performance-Reference_Sheet"
%%% End:


\subsection{Application Layer}\label{subsec:Application_Layer}

%%% Local Variables:
%%% mode: latex
%%% TeX-master: "../../ETSN10-Network_Architecture_Performance-Reference_Sheet"
%%% End:


%%% Local Variables:
%%% mode: latex
%%% TeX-master: "../ETSN10-Network_Architecture_Performance-Reference_Sheet"
%%% End:


\section{Probability Review}\label{sec:Probability_Review}
This section is meant to quick review and introduce the equations that will be used throughout this course.
It is not meant to be comprehensive and/or in-depth.
For more information about the topic of probability and statistics, refer to the \href{file:./Math_374-Reference_Sheet.pdf}{Math 374 - Probability and Statistics} document.

%%% Local Variables:
%%% mode: latex
%%% TeX-master: "../ETSN10-Network_Architecture_Performance-Reference_Sheet"
%%% End:


\section{Performance Evaluation}\label{sec:Performance_Evaluation}
We do this to:
\begin{itemize}[noitemsep]
\item Evaluate existing systems
\item Design new network systems
\item Predict system behaviours under different conditions
\end{itemize}

\subsection{Performance Measures}\label{subsec:Performance_Measures}
How do we measure the performance of a large complex network?

\begin{itemize}[noitemsep]
\item Data transfer speed
\item Reliability:
  \begin{itemize}[noitemsep]
  \item Guaranteed throughput
  \item Guarantee of any other performance measurement
  \item Integrity of data
  \item Predictability of errors
  \item Uptime/Downtime/Availability
  \end{itemize}

\item Security
\item User satisfaction
\item Sustainability
\item Maintainability
\item Throughput/Goodput
\item Delay/Latency
\item Energy Efficiency
\item Jitter (Delay variance)
\item Packet Loss
\end{itemize}

\subsection{Performance Evaluation}\label{subsec:Performance_Evaluation}
How can we evaluate the performance of a large complex network?

\begin{itemize}[noitemsep]
\item Analysis: Mathematical modelling, calculations.
\item Simulation: Software implementation of system model.
\item Real-World Experimentation: Testing the actual system.
\end{itemize}

\begin{table}[h!]
  \centering
  \begin{tabular}{p{6cm}p{6cm}p{6cm}}
    \toprule
    \multicolumn{1}{c}{\textbf{Analysis}} & \multicolumn{1}{c}{\textbf{Simulation}} & \multicolumn{1}{c}{\textbf{Experimentation}} \\
    \midrule
    --- Requires detailed understanding of system properties & + Only requires modelling the environment with a straightforward implementation & ++ No modelling or understanding of how the system required \\
    \midrule
    --- Usually requires approximations and simplifying assumptions. & + Possible to implement complex details of system without approximation & ++ Captures complete behaviour of system and environment without approximation. \\
    \midrule
    ++ Allows for deep insight for a broad range of scenarios. & + Allows insight to broad range of scenarios. & --- Requires deployment of every scenarios tested and may be difficult to reproduce. \\
    \midrule
    + Rare events and boundary cases are included. & + Study of rare events is tricky, but possible. & --- Rare events may be impossible to study. \\
    \bottomrule
  \end{tabular}
  \caption{Performance Evaluation Pros and Cons}
  \label{tab:Performance_Evaluation_Pros_Cons}
\end{table}

\subsection{Statistical Data Analysis}\label{subsec:Statistical_Data_Analysis}
Only analysis produces exact results.
Simulation and experimentation produce samples from some underlying random distribution.
This means we need to perform statistical analysis of these resuilts.

\subsubsection{Sampling}\label{subsubsec:Sampling}
We assume a random variable $Z$ with an unknown probability distribution, but we can assume a distribution to start with.
We estimate the key distribution metrics:
\begin{itemize}[noitemsep]
\item Mean (1st moment)
\item Variance (2nd moment)
\item Variance of the variance (3rd moment)
\end{itemize}

We obtain $n$ \textbf{independent} samples, $z_{1}, z_{2}, \ldots, z_{n}$.
To estimate the mean/expected value, we use the equation below.
\begin{equation}\label{eq:Sample_Mean}
  \bar{z} = \frac{1}{n} \sum\limits_{i=1}^{n}z_{i}
\end{equation}

Where $\bar{z}$ is also a \nameref{def:Random_Variable}.
So, we can perform an expected value calculation on $\bar{z}$.
\begin{equation}\label{eq:Sample_Expected_Value}
  \ExpectedValue[\bar{z}] = \ExpectedValue \left[ \frac{1}{n} \sum\limits_{i=1}^{n} z_{i} \right] = \frac{1}{n} \sum\limits_{i=1}^{n} \ExpectedValue[z_{i}]
\end{equation}

So, as $n \rightarrow \infty$, $\ExpectedValue[\bar{z}] \rightarrow \mu$.

%%% Local Variables:
%%% mode: latex
%%% TeX-master: "../ETSN10-Network_Architecture_Performance-Reference_Sheet"
%%% End:


\section{Queuing Theory}\label{sec:Queuing_Theory}
In queuing theory, we view a network as a collection as first-in-first-out (FIFO) data structures.
Queuing theory provides a probabilistic analysis of these queues.

\begin{definition}[Queuing System]\label{def:Queuing_System}
  In a \emph{queuing system} model, there is a FIFO queue with a task arrival rate of $\lambda$.
  The server processes these with a service time of $\mu$.
\end{definition}

\subsection{Little's Law}\label{subsec:Littles_Law}
\begin{definition}[Little's Law]\label{def:Littles_Law}\label{def:Littles_Formula}
  \emph{Little's Law} or \emph{Little's Formula} models the mean number of tasks in a queue-based system.
  This applies to \textbf{any system in equilibrium}, as long as the system does not create or destroy tasks itself.

  \begin{equation}\label{eq:Littles_Law}%\label{eq:Littles_Formula}
    r = \lambda T_{r}
  \end{equation}
  \begin{itemize}[noitemsep]
  \item $r$ --- The mean number of tasks in a queuing system.
  \item $\lambda$ --- The average arrival rate of tasks to the system.
  \item $T_{r}$ --- The mean time for which the task sits in the queue waiting.
  \end{itemize}
\end{definition}

%%% Local Variables:
%%% mode: latex
%%% TeX-master: "../ETSN10-Network_Architecture_Performance-Reference_Sheet"
%%% End:


%====================================APPENDIX====================================
\appendix
\counterwithin{definition}{subsection}

\clearpage
\section{Complex Numbers}
	\begin{equation} \label{eq:Exponential to Rectangular}
		A e^{-ix} = A \left[ \cos \left( x \right) + i\sin \left( x \right) \right]
	\end{equation}

\clearpage
\subsection{Trigonometry} \label{app:Trig}
	\subsubsection{Trigonometric Formulas} \label{subsubsec:Trig Formulas}
		\begin{equation} \label{eq:Sin plus Sin with diff Angles}
			\sin \left( \alpha \right) + \sin \left( \beta \right) = 2 \sin \left( \frac{\alpha + \beta}{2} \right) \cos\left( \frac{\alpha - \beta}{2} \right)  
		\end{equation}
		\begin{equation} \label{eq:Cosine-Sine Product}
			\cos \left( \theta \right) \sin \left( \theta \right) = \frac{1}{2} \sin \left( 2 \theta \right)
		\end{equation}

\clearpage
\subsection{Calculus} \label{app:Calculus}
	\subsubsection{Fundamental Theorems of Calculus} \label{subsubsec:Fundamental Theorem of Calculus}
		\begin{definition}[First Fundamental Theorem of Calculus] \label{def:1st Fundamental Theorem of Calculus}
			The \emph{first fundamental theorem of calculus} states that, if $f$ is continuous on the closed interval $\left[ a,b \right]$ and $F$ is the indefinite integral of $f$ on $\left[ a,b \right]$, then 
			\begin{equation} \label{eq:1st Fundamental Theorem of Calculus}
				\int_{a}^{b}f \left( x \right) dx = F \left( b \right) - F \left( a \right)
			\end{equation}
		\end{definition}
		\begin{definition}[Second Fundamental Theorem of Calculus] \label{def:2nd Fundamental Theorem of Calculus}
			The \emph{second fundamental theorem of calculus} holds for $f$ a continuous function on an open interval $I$ and $a$ any point in $I$, and states that if $F$ is defined by
			\begin{equation*}
				F \left( x \right) = \int_{a}^{x} f \left( t \right) dt,
			\end{equation*}
			then
			\begin{equation} \label{eq:2nd Fundamental Theorem of Calculus}
				\begin{aligned}
					\frac{d}{dx} \int_{a}^{x} f \left( t \right) dt &= f \left( x \right) \\
					F' \left( x \right) &= f \left( x \right) \\
				\end{aligned}
			\end{equation}
		\end{definition}

\clearpage
\section{Laplace Transform}\label{app:Laplace_Transform}
\subsection{Laplace Transform}\label{subsec:Laplace_Transform}
\begin{definition}[Laplace Transform]\label{def:Laplace_Transform}
  The \emph{Laplace transformation} operation is denoted as $\Lapl \lbrace x(t) \rbrace$ and is defined as
  \begin{equation}\label{eq:Laplace_Transform}
    X(s) = \int\limits_{-\infty}^{\infty} x(t) e^{-st} dt
  \end{equation}
\end{definition}

\subsection{Inverse Laplace Transform}\label{subsec:Inverse_Laplace_Transform}
\begin{definition}[Inverse Laplace Transform]\label{def:Inverse_Laplace_Transform}
  The \emph{inverse Laplace transformation} operation is denoted as $\Lapl^{-1} \lbrace X(s) \rbrace$ and is defined as
  \begin{equation}\label{eq:Inverse_Laplace_Transform}
    x(t) = \frac{1}{2j \pi} \int_{\sigma-\infty}^{\sigma+\infty} X(s) e^{st} \, ds
  \end{equation}
\end{definition}



% To make this print, you must include a citation somewhere in the document
\clearpage
\printbibliography{}
\end{document}

%%% Local Variables:
%%% mode: latex
%%% TeX-master: t
%%% End:
