\documentclass[10pt,letterpaper,final,twoside,notitlepage]{article}
\usepackage[margin=.5in]{geometry} % 1/2 inch margins on all pages
\usepackage[utf8]{inputenc} % Define the input encoding
\usepackage[USenglish]{babel} % Define language used
\usepackage{amsmath,amsfonts,amssymb}
\usepackage{amsthm} % Gives us plain, definition, and remark to use in \theoremstyle{style}
\usepackage{mathtools} % Allow for text and math in align* environment.
\usepackage{thmtools}
\usepackage{thm-restate}
\usepackage{graphicx}

\usepackage[
backend=biber,
style=alphabetic,
citestyle=authoryear]{biblatex} % Must include citation somewhere in document to print bibliography
\usepackage{hyperref} % Generate hyperlinks to referenced items
\usepackage[nottoc]{tocbibind} % Prints the Reference/Bibliography in TOC as well
\usepackage[noabbrev,nameinlink]{cleveref} % Fancy cross-references in the document everywhere
\usepackage{nameref} % Can make references by name to places
\usepackage{caption} % Allows for greater control over captions in figure, algorithm, table, etc. environments
\usepackage{subcaption} % Allows for multiple figures in one Figure environment
\usepackage[binary-units=true]{siunitx} % Gives us ways to typeset units for stuff
\usepackage{csquotes} % Context-sensitive quotation facilities
\usepackage{enumitem} % Provides [noitemsep, nolistsep] for more compact lists
\usepackage{chngcntr} % Allows us to tamper with the counter a little more
\usepackage{empheq} % Allow boxing of equations in special math environments
\usepackage[x11names]{xcolor} % Gives access to coloring text in environments or just text, MUST be before tikz
\usepackage{tcolorbox} % Allows us to create boxes of various types for examples
\usepackage{tikz} % Allows us to create TikZ and PGF Pictures
\usepackage{ctable} % Greater control over tables and how they look
\usepackage{diagbox} % Allow us to have shared diagonal cells in tables
\usepackage{multirow} % Allow us to have a single cell in a table span multiple rows
\usepackage{titling} % Put document information throughout the document programmatically
\usepackage[linesnumbered,ruled,vlined]{algorithm2e} % Allows us to write algorithms in a nice style.

\counterwithin{figure}{section}
\counterwithin{table}{section}
\counterwithin{equation}{section}
\counterwithin{algocf}{section}
\crefname{algocf}{algorithm}{algorithms}
\Crefname{algocf}{Algorithm}{Algorithms}
\setcounter{secnumdepth}{4}
\setcounter{tocdepth}{4} % Include \paragraph in toc
\crefname{paragraph}{paragraph}{paragraphs}
\Crefname{paragraph}{Paragraph}{Paragraphs}

% Create a theorem environment
\theoremstyle{plain}
\newtheorem{theorem}{Theorem}[section]
% Create a numbered theorem-like environment for lemmas
\newtheorem{lemma}{Lemma}[theorem]

% Create a definition environment
\theoremstyle{definition}
\newtheorem{definition}{Defn}
\newtheorem{corollary}{Corollary}[section]
% \begin{definition}[Term] \label{def:}
%   Make sure the term is emphasized with \emph{term}.
%   This ensures that if \emph is changed, it shows up everywhere
% \end{definition}

% Create a numbered remark environment numbered based on definition
% NOTE: This version of remark MUST go inside a definition environment
\theoremstyle{remark}
\newtheorem{remark}{Remark}[definition]
%\counterwithin{definition}{subsection} % Uncomment to have definitions use section.subsection numbering

% Create an unnumbered remark environment for general use
% NOTE: This version of remark has NO restrictions on placement
\newtheorem*{remark*}{Remark}

% Create a special list that handles properties. It can be broken and restarted
\newlist{propertylist}{enumerate}{1} % {Name}{Template}{Max Depth}
% [newlistname, LevelsToApplyTo]{formatting options}
\setlist[propertylist, 1]{label=\textbf{(\roman*)}, ref=\textbf{(\roman*)}, noitemsep, nolistsep}
\crefname{propertylisti}{property}{properties}
\Crefname{propertylisti}{Property}{Properties}

% Create a special list that handles enumerate starting with lower letters. Breakable/Restartable.
\newlist{boldalphlist}{enumerate}{1} % {Name}{Template}{Max Depth}
% [newlistname, LevelsToApplyTo]{formatting options}
\setlist[boldalphlist, 1]{label=\textbf{(\alph*)}, ref=\alph*, noitemsep, nolistsep} % Set options

\newlist{nocrefenumerate}{enumerate}{1} % {Name}{Template}{Max Depth}
% [newlistname, LevelsToApplyTo]{formatting options}
\setlist[nocrefenumerate, 1]{label=(\arabic*), ref=(\arabic*), noitemsep, nolistsep}

% Create a list that allows for deeper nesting of numbers. By default enumerate only allows depth=4.
\newlist{nestednums}{enumerate}{6}
% [newlistname, LevelsToApplyTo]{formatting options}
\setlist[nestednums]{noitemsep, label*=\arabic*.}

\tcbuselibrary{breakable} % Allow tcolorboxes to be broken across pages
% Create a tcolorbox for examples
% /begin{example}[extra name]{NAME}
% Create a tcolorbox for examples
% Argument #1 is optional, given by [], that is the textbook's problem number
% Argument #2 is mandatory, given by {}, that is the title for the example
% Avoid putting special characters, (), [], {}, ",", etc. in the title.
\newtcolorbox[auto counter,
number within=section,
number format=\arabic,
crefname={example}{examples}, % Define reference format for cref (No Capitals)
Crefname={Example}{Examples}, % Reference format for cleveref (With Capitals)
]{example}[2][]{ % The [2][] Means the first argument is optional
  width=\textwidth,
  title={Example \thetcbcounter: #2. #1}, % Parentheses and commas are not well supported
  fonttitle=\bfseries,
  label={ex:#2},
  nameref=#2,
  colbacktitle=white!100!black,
  coltitle=black!100!white,
  colback=white!100!black,
  upperbox=visible,
  lowerbox=visible,
  sharp corners=all,
  breakable
}

% Create a tcolorbox for general use
\newtcolorbox[% auto counter,
% number within=section,
% number format=\arabic,
% crefname={example}{examples}, % Define reference format for cref (No Capitals)
% Crefname={Example}{Examples}, % Reference format for cleveref (With Capitals)
]{blackbox}{
  width=\textwidth,
  % title={},
  fonttitle=\bfseries,
  % label={},
  % nameref=,
  colbacktitle=white!100!black,
  coltitle=black!100!white,
  colback=white!100!black,
  upperbox=visible,
  lowerbox=visible,
  sharp corners=all
}

% Redefine the 'end of proof' symbol to be a black square, not blank
\renewcommand{\qedsymbol}{$\blacksquare$} % Change proofs to have black square at end

% Common Mathematical Stuff
\newcommand{\Abs}[1]{\ensuremath{\lvert #1 \rvert}}
\newcommand{\DNE}{\ensuremath{\mathrm{DNE}}} % Used when limit of function Does Not Exist

% Complex Numbers functions
\renewcommand{\Re}{\operatorname{Re}} % Redefine to use the command, but not the fraktur version
\renewcommand{\Im}{\operatorname{Im}} % Redefine to use the command, but not the fraktur version
\newcommand{\Real}[1]{\ensuremath{\Re \lbrace #1 \rbrace}}
\newcommand{\Imag}[1]{\ensuremath{\Im \lbrace #1 \rbrace}}
\newcommand{\Conjugate}[1]{\ensuremath{\overline{#1}}}
\newcommand{\Modulus}[1]{\ensuremath{\lvert #1 \rvert}}
\DeclareMathOperator{\PrincipalArg}{\ensuremath{Arg}}

% Math Operators that are useful to abstract the written math away to one spot
% Number Sets
\DeclareMathOperator{\RealNumbers}{\ensuremath{\mathbb{R}}}
\DeclareMathOperator{\AllIntegers}{\ensuremath{\mathbb{Z}}}
\DeclareMathOperator{\PositiveInts}{\ensuremath{\mathbb{Z}^{+}}}
\DeclareMathOperator{\NegativeInts}{\ensuremath{\mathbb{Z}^{-}}}
\DeclareMathOperator{\NaturalNumbers}{\ensuremath{\mathbb{N}}}
\DeclareMathOperator{\ComplexNumbers}{\ensuremath{\mathbb{C}}}
\DeclareMathOperator{\RationalNumbers}{\ensuremath{\mathbb{Q}}}

% Calculus operators
\DeclareMathOperator*{\argmax}{argmax} % Thin Space and subscripts are UNDER in display

% Signal and System Functions
\DeclareMathOperator{\UnitStep}{\mathcal{U}}
\DeclareMathOperator{\sinc}{sinc} % sinc(x) = (sin(pi x)/(pi x))

% Transformations
\DeclareMathOperator{\Lapl}{\mathcal{L}} % Declare a Laplace symbol to be used

% Logical Operators
\DeclareMathOperator{\XOR}{\oplus}

% x86 CPU Registers
\newcommand{\rbpRegister}{\texttt{\%rbp}}
\newcommand{\rspRegister}{\texttt{\%rsp}}
\newcommand{\ripRegister}{\texttt{\%rip}}
\newcommand{\raxRegister}{\texttt{\%rax}}
\newcommand{\rbxRegister}{\texttt{\%rbx}}

%%% Local Variables:
%%% mode: latex
%%% TeX-master: shared
%%% End:


% These packages are more specific to certain documents, but will be availabe in the template
% \usepackage{esint} % Provides us with more types of integral symbols to use
% \usepackage[outputdir=./TeX_Output]{minted} % Allow us to nicely typeset 300+ programming languages
% This document must be compiled with the -shell-escape flag if the packages above are uncommented

% \graphicspath{{./Drawings/Course}} % Uncomment this to use pictures in this document
% \addbibresource{./Bibliographies/CourseNum-Name.bib}

% Math Operators that are useful to abstract the written math away to one spot
% These are supposed to be document-specific mathematical operators that will make life easier
% Many fundamental operators are defined in Reference_Sheet_Preamble.tex
\DeclareMathOperator{\Prob}{\operatorname{P}}
\DeclareMathOperator{\Given}{\vert}
\DeclareMathOperator{\ExpectedValue}{\operatorname{\mathbb{E}}}
\DeclareMathOperator{\Variance}{\operatorname{VAR}}
\DeclareMathOperator{\StdDev}{\operatorname{STD}}
\DeclareMathOperator{\Covariance}{\operatorname{Cov}}
\DeclareMathOperator{\CorrCoeff}{\rho}

\newcommand{\SetOrder}[1]{\lvert #1 \rvert}

\begin{titlepage}
  \title{ETSN10: Network Architecture and Performance - Reference Sheet}
  \author{Karl Hallsby}
  \date{Last Edited: \today} % We want to inform people when this document was last edited
\end{titlepage}

\begin{document}
\pagenumbering{gobble}
\maketitle
\pagenumbering{roman} % i, ii, iii on beginning pages, that don't have content
\tableofcontents
\clearpage
\listoftheorems[ignoreall, show={definition, Definition}]
\clearpage
\pagenumbering{arabic} % 1,2,3 on content pages

\section{Networking Review}\label{sec:Networking_Review}
This course assumes that you already know about basic networking terms and models.
However, this section is meant as a refresher for people who have already taken that particular set of courses, or as a quick introduction for those who have not taken those courses yet.

\subsection{Networking Stack}\label{subsec:Networking_Stack}

%%% Local Variables:
%%% mode: latex
%%% TeX-master: "../../ETSN10-Network_Architecture_Performance-Reference_Sheet"
%%% End:


\subsection{Physical Layer}\label{subsec:Physical_Layer}
\begin{definition}[Physical Layer]\label{def:Physical_Layer}
  The \emph{physical layer} in the \nameref{def:Networking_Stack} is the lowest layer in the stack.
  It consists of the physical connection that is made between computers and the modulation and coding to represent data as a signal on that connection.
  For obvious reasons, all devices \textbf{must} implement this layer.

  Some examples of this layer are:
  \begin{itemize}[noitemsep]
  \item Copper twisted-pair cables
  \item Fiber optic wires
  \item Radio transmission
  \end{itemize}

  \begin{remark}
    The design of different \nameref{def:Physical_Layer}s is more a hardware design question than that of a software implementation issue.
    So, it is not discussed in this course, beyond the use of cellular networks.
  \end{remark}
\end{definition}

%%% Local Variables:
%%% mode: latex
%%% TeX-master: "../../ETSN10-Network_Architecture_Performance-Reference_Sheet"
%%% End:


\subsection{Data-Link Layer}\label{subsec:Data_Link_Layer}

%%% Local Variables:
%%% mode: latex
%%% TeX-master: "../../ETSN10-Network_Architecture_Performance-Reference_Sheet"
%%% End:


\subsection{Networking Layer}\label{subsec:Networking_Layer}
\begin{definition}[Networking Layer]\label{def:Networking_Layer}
  \textbf{TODO!}
\end{definition}

\begin{definition}[Routing]\label{def:Routing}
  \emph{Routing} is the process of determining where to forward packets onto next.

  There are algorithms designed to discover the other participants in the network by analyzing the grpah that is created when each participant is considered a node in this graph.
\end{definition}

\begin{definition}[Routing Protocol]\label{def:Routing_Protocol}
  A \emph{routing protocol} is used to construct a \nameref{def:Routing_Table} in each node.
\end{definition}

\begin{definition}[Routing Table]\label{def:Routing_Table}
  The \emph{routing table} is a table that contains entries on the shortest distances through the current network graph.
  This is used to help determine where to next forward packets onto.

  Each node has a routing table that contains the cost to the destination and the next hop for each destination.
  For a valid routing table, we need to have:
  \begin{enumerate}[noitemsep]
  \item Which destination the packets are going to?
  \item What is the next hop after the destination?
  \item The ``metric'' or weight of that particular path through the network graph?
    \begin{itemize}[noitemsep]
    \item We need this term so we can update the routing table as we discover new routes.
    \item This way if a new route with a lower cost presents itself, we can replace the higher-cost one with the lower one.
    \end{itemize}
  \end{enumerate}
\end{definition}

\begin{definition}[Forwarding]\label{def:Forwarding}
  \emph{Forwarding} is the process of moving a packet from one place to the next based on the \nameref{def:Routing_Table} present in this node and where the packet would like to go next.
\end{definition}

\subsubsection{Discovering Routing Information}\label{subsubsec:Discovering_Route_Info}
\paragraph{Dijkstra's Algorithm}\label{par:Dijkstras_Algorithm}
There are 2 main problems with Dijkstra's Algorithm:
\begin{enumerate}[noitemsep]
\item The weights along the edges of the graph cannot be negative.
\item We \textbf{must} know the entire network's topology before beginning the algorithm.
  \begin{itemize}[noitemsep]
  \item Typically, this is not possible on vast networks, like the Internet.
  \end{itemize}
\end{enumerate}

\begin{algorithm}[H]
  \DontPrintSemicolon{}
  \SetKwInOut{Input}{Input}\SetKwInOut{Output}{Output}
  \Input{A graph with weights on each edge representing ``distance'' between the nodes.}
  \Output{The path through the graph with the least weight from some starting node.}
  \BlankLine{}

  \textbf{TODO!}\;
  \textbf{From Networking Review Lecture Video 2.} \;
  
  \caption{Dijkstra's Algorithm}
  \label{algo:Dijkstras_Algorithm}
\end{algorithm}

\paragraph{Bellman-Ford Algorithm}\label{par:Bellman_Ford_Algorithm}
The Bellman-Ford algorithm works with graphs that have negative weight values on edges.
It also \textbf{does not} require us to know the entire network's topology before beginning.
However, it is much slower than \nameref{algo:Dijkstras_Algorithm}.

\begin{algorithm}[H]
  \DontPrintSemicolon{}
  \SetKwInOut{Input}{Input}\SetKwInOut{Output}{Output}
  \Input{A graph with weights on each edge representing ``distance'' between the nodes.}
  \Output{The path through the graph with the least weight from some starting node.}
  \BlankLine{}

  \textbf{TODO!} \;
  \textbf{From Networking Review Lecture Video 2.} \;

  \caption{Bellman-Ford Algorithm}
  \label{algo:Bellman_Ford_Algorithm}
\end{algorithm}

\subsubsection{Routing Information}\label{subsubsec:Routing_Info}
There are 2 ways to collect the routing information necessary to properly route packets through a network.
These 2 routing tables are:
\begin{enumerate}[noitemsep]
\item \nameref{par:Static_Routing_Tables}
\item \nameref{par:Dynamic_Routing_Tables}
\end{enumerate}

\paragraph{Static Routing Tables}\label{par:Static_Routing_Tables}
The static routing tables are manually configured by a user or a program before the system begins routing the information.
This means that there is not overhead when trying to determine where the packet should next be forwarded to.

\paragraph{Dynamic Routing Tables}\label{par:Dynamic_Routing_Tables}
To use a dynamic \nameref{def:Routing_Table}, we must use a \nameref{def:Routing_Protocol} to build the table as we go, rather than have a completely finished table when starting.
These are built using either:
\begin{enumerate}[noitemsep]
\item \nameref{def:Distance_Vector_Routing}
\item \nameref{def:Link_State_Routing}
\end{enumerate}

\begin{definition}[Distance Vector Routing Protocol]\label{def:Distance_Vector_Routing_Protocol}
  The \emph{distance vector routing algorithm} is a \nameref{def:Routing_Protocol} that uses \nameref{algo:Bellman_Ford_Algorithm} to construct a \nameref{def:Routing_Table}.
  \textbf{TODO!}
\end{definition}

\begin{definition}[Link State Routing Protocol]\label{def:Link_State_Routing_Protocol}
  The \emph{link state routing protocol} is a \nameref{def:Routing_Protocol} that uses \nameref{algo:Dijkstras_Algorithm} to construct a \nameref{def:Routing_Table}.
  Thus, we need to know what the whole network looks like.
  \textbf{TODO!}
\end{definition}

However, the problem with both \nameref{def:Distance_Vector_Routing_Protocol} and \nameref{def:Link_State_Routing_Protocol} is that they do not scale well.
However, to help with this, both protocols are contained within a single \nameref{def:Autonomous_System}.

\begin{definition}[Autonomous System]\label{def:Autonomous_System}
  An \emph{autonomous system} is a smaller network, like a business or home, that performs a \nameref{def:Routing_Protcol} upon itself.
  It is done this way because both the \nameref{def:Distance_Vector_Routing_Protocol} and \nameref{def:Link_State_Routing_Protocol} do not scale to Internet-sized networks well.

  To combat this, each of the autonomous systems has a speaker node that speaks to the outside world.
  This speaker node is used in the \nameref{def:Path_Vector_Routing_Protocol}.
\end{definition}

\begin{definition}[Path Vector Routing Protocol]\label{def:Path_Vector_Routing_Protocol}
  
\end{definition}

\subsubsection{IP Addresses}\label{subsubsec:IP_Addresses}
\begin{definition}[IP Address]\label{def:IP_Address}
  \textbf{TODO!}
\end{definition}

\begin{definition}[Subnet]\label{def:Subnet}
  All hosts that are in the subnet will have \nameref{def:IP_Address}es that match the number of bits that are present in the subnet.

  For example, in \texttt{192.168.2.0/24}, the \texttt{24} means the first 24 bits of the subnet are 1's.
  So, the first 3 \nameref{def:IP_Address} blocks \texttt{192.168.2} will remain the same for every client in that subnet.
  \textbf{TODO!}
\end{definition}

\begin{definition}[Classless Inter-Domain Routing]
  \textbf{TODO!}
\end{definition}

%%% Local Variables:
%%% mode: latex
%%% TeX-master: "../../ETSN10-Network_Architecture_Performance-Reference_Sheet"
%%% End:


\subsection{Transport Layer}\label{subsec:Transport_Layer}
\begin{definition}[Transport Layer]\label{def:Transport_Layer}
  The \emph{transport layer} is built where the \nameref{def:Network_Layer} has delivered all the data to the end host.

  This means that only the source and destination hosts have this layer.
  So, routers and switches do not have this, but your phone, laptop, and the server you're connecting to do.

  This layer may be responsible for:
  \begin{itemize}[noitemsep]
  \item \nameref{def:Connection_Oriented_Communication}
  \item \nameref{subsubsec:Multiplexing_Data_Flows}
  \item Reliable data delivery
  \item Data flow control
  \item Data congestion control
  \end{itemize}
  Some implementation of this layer do not handle all of these functions, but all \textbf{must} handle the multiplexing of different data flows.

  There are 2 main transport layers in-use today:
  \begin{enumerate}[noitemsep]
  \item \nameref{def:Transmission_Control_Protocol}
  \item \nameref{def:User_Datagram_Protocol}
  \end{enumerate}
\end{definition}

\begin{definition}[Connection-Oriented Communication]\label{def:Connection_Oriented_Communication}
  A \emph{connection-oriented communication} system means that a connection must be established before any data is transferred.
  In addition, this connection must be maintained throughout the transmission of data.

  \begin{remark}[Connectionless Communication]\label{rmk:Connectionless_Communication}
    In a \emph{connectionless communication} system, hosts can send data at any time without prior connection being made.
  \end{remark}
\end{definition}

\subsubsection{Transport Protocols}\label{subsubsec:Transport_Protocols}
\begin{definition}[User Datagram Protocol]\label{def:User_Datagram_Protocol}
  \emph{User Datagraph Protocol (UDP)} is the simplest transport protocol available.
  It performs multiplexing and error checking, \textbf{but nothing else}.

  When a host wants to send data to another host, it just sends it, making UDP a \nameref{rmk:Connectionless_Communication} protocol.
  If the data gets lost, too bad.
\end{definition}

\begin{definition}[Transmission Control Protocol]\label{def:Transmission_Control_Protocol}
  \emph{Transmission Control Protocol (TCP)} is a \nameref{def:Connection_Oriented_Communication} protocol.
  It performs:
  \begin{itemize}[noitemsep]
  \item Multiplexing
  \item Reliable data delivery
  \item Error detection
  \item Ordered data delivery
  \item \nameref{def:Flow_Control}
  \item \nameref{def:Congestion_Control}
  \end{itemize}
\end{definition}

\nameref{def:Transmission_Control_Protocol} uses a 3-way hand shake to establish a connection.
\begin{definition}[Three-Way Handshake]\label{def:TCP_3_Way_Handshake}
  \nameref{def:Transmission_Control_Protocol} uses a \emph{Three-Way Handshake} to establish a connection.

  The client sends a \texttt{SYN} message to the recipient.
  The receiver sends back a \texttt{SYN\textunderscore{}ACK} message to acknowledge the receipt of the original \texttt{SYN} message.
  The client then sends another \texttt{ACK} to the receiver to acknowledge the receipt of the \texttt{SYN\textunderscore{}ACK} message.

  Because this handshake is asymmetrical, there is a difference between servers and clients.
  A server must have a \nameref{def:Port} open, and be listening for client connections.
\end{definition}

\begin{definition}[Four-Way Handshake]\label{def:TCP_4_Way_Handshake}
  \nameref{def:Transmission_Control_Protocol} uses a \emph{four-way handshake} to close a connection.
  Each host can close its side of the connection independently.
\end{definition}

\subsubsection{Multiplexing Data Flows}\label{subsubsec:Multiplexing_Data_Flows}
\begin{definition}[Port]\label{def:Port}
  A \emph{port} is a single instance of a data flow.
  There are many different flows of data.
  These may be from different applications or different instances of the same application.
  In both \nameref{def:Transmission_Control_Protocol} and \nameref{def:User_Datagram_Protocol}, flows are given unique \emph{port numbers}.

  Some of these are standard for particular applications, e.g.\ port 80 for HTTP (web), port 25 for SMTP (email).
  The transport protocol uses the port number to deliver data to the correct application.
\end{definition}

\subsubsection{Data Delivery}\label{subsubsec:Data_Delivery}
\paragraph{\nameref*{subsubsec:Data_Delivery} in \nameref*{def:Transmission_Control_Protocol}}\label{par:TCP_Data_Delivery}
\begin{definition}[Sequence Number]\label{def:Sequence_Number}
  \nameref{def:Transmission_Control_Protocol} uses \emph{sequence number}s to make sure data is complete and in order when it is delivered.
  It also includes error detection, and segments with errors are retransmitted.
\end{definition}

\subsubsection{Flow Control}\label{subsubsec:TCP_Flow_Control}
\begin{restatable}[Flow Control]{definition}{defFlowControl}\label{def:Flow_Control}
  \emph{Flow control} refers to signalling between the sender and receiver to ensure the sender does not send data faster than the receiver can process.

  A receiving host may have limited buffer space for incoming messages and takes time to process each message.
\end{restatable}

\paragraph{\nameref*{subsubsec:TCP_Flow_Control} in \nameref*{def:Transmission_Control_Protocol}}\label{par:TCP_Flow_Control}
\begin{restatable}[Sliding Window]{definition}{defSlidingWindow}\label{def:Sliding_Window}
  \nameref{def:Flow_Control} in \nameref{def:Transmission_Control_Protocol} uses a \emph{sliding window} mechanism.
\end{restatable}

\subsubsection{Congestion Control}\label{subsubsec:TCP_Congestion_Control}
\begin{restatable}[Congestion Control]{definition}{defCongestionControl}\label{def:Congestion_Control}
  \emph{Congestion control} refers to mechanisms for detecting and reducing \nameref{def:Congestion}.
\end{restatable}

\begin{restatable}[Congestion]{definition}{defCongestion}\label{def:Congestion}
    \emph{Congestion} in a network occurs when there is too much data being sent and the network is unable to deliver it all.
  This can result in data loss or long delays.
\end{restatable}

\paragraph{\nameref*{subsubsec:TCP_Congestion_Control} in \nameref*{def:Transmission_Control_Protocol}}\label{par:TCP_Congestion_Control}
\begin{definition}[Congestion Window]\label{def:Congestion_Window}
  In \nameref{def:Transmission_Control_Protocol}, lost data is considered a sign of \nameref{def:Congestion} and the sender should reduce its rate.
  The sender has a \emph{congestion window}, which refers to the maximum number of unacknowledged segments that may be in transit at a time.
\end{definition}

A \nameref{def:Transmission_Control_Protocol} connection begins in \textbf{slow start}, where the \nameref{def:Congestion_Window} is doubled every round trip.
When a threshold is reached, it changes to \emph{congestion avoidance}, where the congestion window increases by 1 maximum segment size each time.
When packet loss is detected, the congestion window is halved.

\begin{equation}\label{eq:TCP_Maximum_Bandwidth}
  \mathrm{BW}_{Max} = \frac{\mathrm{MSS} \times \sqrt{\frac{3}{2}}}{\mathrm{RTT} \times \sqrt{p}}
\end{equation}
\begin{itemize}[noitemsep]
\item $\mathrm{BW}_{Max}$: Maximum bandwidth.
\item $\mathrm{MSS}$: Maximum \nameref{def:Transmission_Control_Protocol} \nameref{def:Packet} size.
\item $\mathrm{RTT}$: Round trip time.
\item $p$: \nameref{def:Packet} loss probability.
\end{itemize}

%%% Local Variables:
%%% mode: latex
%%% TeX-master: "../../ETSN10-Network_Architecture_Performance-Reference_Sheet"
%%% End:


\subsection{Application Layer}\label{subsec:Application_Layer}
\begin{definition}[Application Layer]\label{def:Application_Layer}
  The \emph{application layer} is the last and highest layer in the \nameref{def:Networking_Stack}.
  This layer is created by the actual applications that interface with the user, their email client, web browser, games, etc.
  This is the layer where data is actually generated and consumed by any application that communicates over the Internet.

  Some examples of item in this layer are:
  \begin{itemize}[noitemsep]
  \item Hypertext Transfer Protocol (HTTP)
  \item Simple Mail Transfer Protocol (SMTP)
  \item Extensible Messaging and Presence Protocol (XMPP)
  \item Skype
  \end{itemize}
\end{definition}

%%% Local Variables:
%%% mode: latex
%%% TeX-master: "../../ETSN10-Network_Architecture_Performance-Reference_Sheet"
%%% End:


%%% Local Variables:
%%% mode: latex
%%% TeX-master: "../ETSN10-Network_Architecture_Performance-Reference_Sheet"
%%% End:


\section{Probability Review}\label{sec:Probability_Review}
This section is meant to quick review and introduce the equations that will be used throughout this course.
It is not meant to be comprehensive and/or in-depth.
For more information about the topic of probability and statistics, refer to the \href{file:./Math_374-Reference_Sheet.pdf}{Math 374 - Probability and Statistics} document.

\subsection{Axioms of Probability}\label{subsec:Axioms_of_Probability}
\begin{definition}[Sample Space]\label{def:Sample_Space}
  The \emph{sample space} is the set of all possible outcomes in a random experiment.
  It is denoted with the capital Greek omega.

  \begin{equation}\label{eq:Sample_Space}
    \Omega
  \end{equation}
\end{definition}

\begin{definition}[Event]\label{def:Event}
  An \emph{event} is a subset of the \nameref{def:Sample_Space} that we are interested in.
  These are generally denoted with capital letters.

  \begin{equation}\label{eq:Event}
    A \subseteq \Omega
  \end{equation}
\end{definition}

\begin{definition}[Mutually Exclusive]\label{def:Mutually_Exclusive}
  Any two \nameref{def:Event}s are \emph{mutually exclusive} if the equation below holds.

  \begin{equation}\label{eq:Mutually_Exclusive}
    \Prob(A \cup B) = \Prob(A) + \Prob(B)
  \end{equation}
\end{definition}

Laws that follow from the above definitions (\Crefrange{def:Sample_Space}{def:Mutually_Exclusive}).
\begin{enumerate}[noitemsep]
\item The conjugate of the \nameref{def:Event} occurring, i.e.\ the \nameref{def:Event} \textbf{not} occurring is:
  \begin{equation}\label{eq:Conjugate_Event}
    \Prob(\bar{A}) = 1 - \Prob(A)
  \end{equation}

\item The probability of the union of 2 \nameref{def:Event}s is:
  \begin{equation}\label{eq:Union_Probability}
    \Prob(A \cap B) = \Prob(A) + \Prob(B) - \Prob(A \cup B)
  \end{equation}
  \begin{itemize}[noitemsep]
  \item If $A$ and $B$ are \nameref{def:Mutually_Exclusive}, then $\Prob(A \cup B) = 0$.
  \end{itemize}
\end{enumerate}

\subsection{Conditional Probability}\label{subsec:Conditional_Probability}
\begin{definition}[Conditional Probability]\label{def:Conditional_Probability}
  \emph{Conditional probability} is the probability of an \nameref{def:Event} occurring when it is known that another \nameref{def:Event} occurred.

  \begin{equation}\label{eq:Conditional_Probability}
    \Prob(A \Given B) = \frac{\Prob(A \cap B)}{\Prob(B)}
  \end{equation}
\end{definition}

\begin{definition}[Independent]\label{def:Events_Independent}
  \nameref{def:Event}s are \emph{independent} if the probability of the events' intersection is the same as their probabilities multipled together.

  \begin{equation}\label{eq:Events_Independent}
    \Prob(A \cap B) = \Prob(A) \Prob(B)
  \end{equation}

  \begin{remark}[\nameref*{def:Conditional_Probability} and \nameref*{def:Events_Independent} Events]
    If $A$ and $B$ are \nameref{def:Event}s and are \nameref{def:Events_Independent}, then

    \begin{equation}\label{eq:Conditional_Probability_Events_Independent}
      \begin{aligned}
        \Prob(A \Given B) &= \Prob(A) \\
        \Prob(B \Given A) &= \Prob(B) \\
      \end{aligned}
    \end{equation}
  \end{remark}
\end{definition}

\subsection{Random Variables}\label{subsec:Random_Variables}
\begin{definition}[Random Variable]\label{def:Random_Variable}
  A \emph{random variable} is a mapping from an \nameref{def:Event}'s outcome to a real number.

  There are 2 types of random variables, based on what the mapping ends up with:
  \begin{enumerate}[noitemsep]
  \item \nameref{def:Discrete_Random_Variable}s are mapped to integers, $\AllIntegers$.
  \item \nameref{def:Continuous_Random_Variable}s are mapped to the real numbers, $\RealNumbers$.
  \end{enumerate}
\end{definition}

\subsubsection{Discrete Random Variables}\label{subsubsec:Discrete_Random_Variables}
\begin{definition}[Discrete Random Variable]\label{def:Discrete_Random_Variable}
  A \emph{Discrete Random Variable} is one whose values are mapped from an \nameref{def:Event}'s outcome to the integer numbers ($\AllIntegers$).
  These \nameref{def:Random_Variable}s are drawn from outcomes that are finite (sides on a die) or countably infinite.

  The probability of a single value of the discrete random variable is denoted differently here than in the course material.
  The subscript refers to which discrete random variable we are working with (in this case $X$) and the variable in parentheses is the value we are calculating for (in this case $x \in X$).
  \begin{equation}\label{eq:Discrete_Random_Variable-Single_Value}
    p_{X}(x)
  \end{equation}

  The sum of all probabilities for values that the discrete random variable can take \textbf{must} sum to 1.
  \begin{equation}\label{eq:Discrete_Random_Variable-Sum_to_One}
    \sum\limits_{x \in X} p_{X}(x) = 1
  \end{equation}

  The mean or expected value of a discrete random variables is shown below:
  \begin{equation}\label{eq:Discrete_Random_Variable-Expected_Value}
    \begin{aligned}
      \mu = \sum\limits_{x \in X} x p_{X}(x) \\
      \ExpectedValue[x] = \sum\limits_{x \in X} x p_{X}(x) \\
    \end{aligned}
  \end{equation}

  The variance of a discrete random variable is how ``off'' a value from the random variable is from the mean/expected value.
  \begin{equation}\label{eq:Discrete_Random_Variable-Variance}
    \begin{aligned}
      \sigma^{2} &= \sum\limits_{x \in X} {\left( x - \mu \right)}^{2} p_{X}(x) \\
      \Variance[x] &= \sum\limits_{x \in X} {\bigl( x - \ExpectedValue[x] \bigr)}^{2} p_{X}(x) \\
    \end{aligned}
  \end{equation}

  The standard deviation is the square root of the variance.
  \begin{equation}\label{eq:Discrete_Random_Variable-Standard_Deviation}
    \begin{aligned}
      \sigma = \sqrt{\sigma^{2}} &= \sqrt{\sum\limits_{x \in X} {\left( x - \mu \right)}^{2} p_{X}(x)} \\
      \StdDev[x] = \sqrt{\Variance[x]} &= \sqrt{\sum\limits_{x \in X} {\bigl( x - \ExpectedValue[x] \bigr)}^{2} p_{X}(x)} \\
    \end{aligned}
  \end{equation}
\end{definition}

There are 5 different \nameref{def:Discrete_Random_Variable} distributions that we will be heavily utilizing in this course.

\paragraph{Uniform Random Variable}\label{par:Uniform_Random_Variable}
\begin{definition}[Uniform Random Variable]\label{def:Uniform_Random_Variable}
  The \emph{uniform random variable} is a \nameref{def:Discrete_Random_Variable} whose probabilities for each outcome is equal.

  For a \nameref{def:Discrete_Random_Variable} $X$, which has $\SetOrder{X}$ possible values,
  \begin{equation}\label{eq:Uniform_Random_Variable-Probability_Distribution}
    p_{X}(x) = \frac{1}{\SetOrder{X}}
  \end{equation}
\end{definition}

\paragraph{Bernoulli Random Variable}\label{par:Bernoulli_Random_Variable}
\paragraph{Binomial Random Variable}\label{par:Binomial_Random_Variable}
\paragraph{Geometric Random Variable}\label{par:Geometric_Random_Variable}
\paragraph{Poisson Random Variable}\label{par:Poisson_Random_Variable}
%%% Local Variables:
%%% mode: latex
%%% TeX-master: "../ETSN10-Network_Architecture_Performance-Reference_Sheet"
%%% End:


\section{Performance Evaluation}\label{sec:Performance_Evaluation}
We do this to:
\begin{itemize}[noitemsep]
\item Evaluate existing systems
\item Design new network systems
\item Predict system behaviors under different conditions
\end{itemize}

\subsection{Performance Measures}\label{subsec:Performance_Measures}
How do we measure the performance of a large complex network?

\begin{itemize}[noitemsep]
\item Data transfer speed
\item Reliability:
  \begin{itemize}[noitemsep]
  \item Guaranteed throughput
  \item Guarantee of any other performance measurement
  \item Integrity of data
  \item Predictability of errors
  \item Uptime/Downtime/Availability
  \end{itemize}

\item Security
\item User satisfaction
\item Sustainability
\item Maintainability
\item Throughput/Goodput
\item Delay/Latency
\item Energy Efficiency
\item Jitter (Delay variance)
\item Packet Loss
\end{itemize}

\subsection{Performance Evaluation}\label{subsec:Performance_Evaluation}
How can we evaluate the performance of a large complex network?

\begin{itemize}[noitemsep]
\item Analysis: Mathematical modeling, calculations.
\item Simulation: Software implementation of system model.
\item Real-World Experimentation: Testing the actual system.
\end{itemize}

\begin{table}[h!]
  \centering
  \begin{tabular}{p{6cm}p{6cm}p{6cm}}
    \toprule
    \multicolumn{1}{c}{\textbf{Analysis}} & \multicolumn{1}{c}{\textbf{Simulation}} & \multicolumn{1}{c}{\textbf{Experimentation}} \\
    \midrule
    --- Requires detailed understanding of system properties & + Only requires modeling the environment with a straightforward implementation & ++ No modeling or understanding of how the system required \\
    \midrule
    --- Usually requires approximations and simplifying assumptions. & + Possible to implement complex details of system without approximation & ++ Captures complete behavior of system and environment without approximation. \\
    \midrule
    ++ Allows for deep insight for a broad range of scenarios. & + Allows insight to broad range of scenarios. & --- Requires deployment of every scenarios tested and may be difficult to reproduce. \\
    \midrule
    + Rare events and boundary cases are included. & + Study of rare events is tricky, but possible. & --- Rare events may be impossible to study. \\
    \bottomrule
  \end{tabular}
  \caption{Performance Evaluation Pros and Cons}
  \label{tab:Performance_Evaluation_Pros_Cons}
\end{table}

\subsection{Statistical Data Analysis}\label{subsec:Statistical_Data_Analysis}
Only analysis produces exact results.
Simulation and experimentation produce samples from some underlying random distribution.
This means we need to perform statistical analysis of these results.

\subsubsection{Sample Mean}\label{subsubsec:Sample_Mean}
We assume a random variable $Z$ with an unknown probability distribution, but we can assume a distribution to start with.
We estimate the key distribution metrics:
\begin{itemize}[noitemsep]
\item Mean (1st moment)
\item Variance (2nd moment)
\item Variance of the variance (3rd moment)
\end{itemize}

We obtain $n$ \textbf{independent} samples, $z_{1}, z_{2}, \ldots, z_{n}$.
To estimate the mean/expected value, we use the equation below.
\begin{equation}\label{eq:Sample_Mean}
  \bar{z} = \frac{1}{n} \sum\limits_{i=1}^{n}z_{i}
\end{equation}

Where $\bar{z}$ is also a \nameref{def:Random_Variable}.
So, we can perform an expected value calculation on $\bar{z}$.
\begin{equation}\label{eq:Sample_Expected_Value}
  \ExpectedValue[\bar{z}] = \ExpectedValue \left[ \frac{1}{n} \sum\limits_{i=1}^{n} z_{i} \right] = \frac{1}{n} \sum\limits_{i=1}^{n} \ExpectedValue[z_{i}]
\end{equation}

So, as $n \rightarrow \infty$, $\ExpectedValue[\bar{z}] \rightarrow \mu$.

\subsubsection{Sample Variance}\label{subsubsec:sample_Variance}
Now that we have found the sample mean (\Cref{eq:Sample_Mean}), we can attempt to find the variance.

We start by estimating the variance:
\begin{align*}
  \Variance \bigl[ \ExpectedValue[\bar{z}] \bigr] &= \ExpectedValue \left[ {(\bar{z} - \mu)}^{2} \right] \\
  {(\bar{z} - \mu)}^{2} &= \frac{1}{n^{2}} {\left( \sum\limits_{i=1}^{n} \left(z_{i} - \mu \right) \right)}^{2} \\
                                                  &= \frac{1}{n^{2}} \sum\limits_{i=1}^{n} {(z_{i} - \mu)}^{2} + \sum\limits_{i=1}^{n}\sum\limits_{j \neq i} (z_{i} - \mu) (z_{j} - \mu) \\
  \ExpectedValue \left[ {(\bar{z} - \mu)}^{2} \right] &= \frac{1}{n^{2}} \sum\limits_{i=1}^{n} \ExpectedValue \left[ {(z_{i} - \mu)}^{2} \right] + \sum\limits_{i=1}^{n}\sum\limits_{j \neq i} \ExpectedValue[(z_{i}-\mu)] \ExpectedValue[(z_{j} - \mu)] \\
                                                  &= \frac{1}{n^{2}} \left( n \sigma^{2} + 0 \right) \\
                                                  &= \frac{\sigma^{2}}{n}
\end{align*}

The estimated variance is:
\begin{equation*}
  \Variance[\bar{z}] = \frac{\sigma^{2}}{n}  
\end{equation*}
meaning that the variance of the sample mean $\bar{z}$ around the population mean $\mu$ decreases as the number of trials $n$ increases.

The sample variance is the variance of the sample data around its mean.
\begin{equation}\label{eq:Biased_Sample_Variance}
  \widehat{V} = \frac{1}{n} \sum\limits_{i=1}^{n} {(z_{i} - \bar{z})}^{2}
\end{equation}

The expected value of the sample variance, $\ExpectedValue[\widehat{V}]$ is
\begin{equation*}
  \ExpectedValue[\widehat{V}] = \frac{n-1}{n} \sigma^{2}
\end{equation*}

This means that the expected value of the variance is actually slightly biased by the $n-1$ numerator.
To correct for this, we use \nameref{def:Bessels_Correction}.
\begin{definition}[Bessel's Correction]\label{def:Bessels_Correction}
  The division of $n-1$ instead of $n$ in \Cref{eq:Unbiased_Sample_Variance} is called \emph{Bessel's Correction}.
\end{definition}

\begin{definition}[Unbiased Sample Variance]\label{def:Unbiased_Sample_Variance}
  By using \Cref{def:Bessels_Correction}, instead of the normal $n$ value, we find ourselves with the \emph{unbiased sample variance}.
  \begin{equation}\label{eq:Unbiased_Sample_Variance}
    s^{2} = \frac{1}{n-1} \sum\limits_{i=1}^{n} {(z_{i} - \bar{z})}^{2}
  \end{equation}
\end{definition}


\subsubsection{Confidence Intervals}\label{subsubsec:Confidence_Intervals}
\begin{definition}[Confidence Interval]\label{def:Confidence_Interval}
  \emph{Confidence interval}s are a tool to describe how much we can trust the set of results that we gather.
  Essentially, how confident we are in the results that we gathered.
  More mathematically, they describe how sure we are (95\%-confident, 99\%-confident, etc.) we are that the data point we gathered is the same as the underlying distribution's value.

  The confidence interval is derived from the \nameref{def:Gaussian_Random_Variable}.
  Thus, it is only useful if the sample size is large, because of the Central Limit Theorem.
  It is derived from the below equation:
  \begin{equation}\label{eq:Confidence_Interval_Entire}
    \Prob \left( \lvert \bar{z} - \mu \rvert \leq \alpha \frac{\sigma}{\sqrt{n}} \right)
  \end{equation}

  Some common values of this are:
  \begin{itemize}[noitemsep]
  \item $= 0.9$ for $\alpha = 1.645$
  \item $= 0.95$ for $\alpha = 1.96$
  \item $= 0.99$ for $\alpha = 2.576$
  \end{itemize}

  The actual interval is calculated with this equation:
  \begin{equation}\label{eq:Confidence_Interval_Sigma}
    \left[ \bar{z} - \alpha \frac{\sigma}{\sqrt{n}}, \bar{z} + \alpha \frac{\sigma}{\sqrt{n}} \right]
  \end{equation}

  However, because the standard deviation $\sigma$ is not known, we estimate using the square root of the \nameref{def:Unbiased_Sample_Variance} $\sqrt{s^{2}}$ instead.
  So, the actual confidence interval is:
  \begin{equation}\label{eq:Confidence_Interval_Sqrt_S}
    \left[ \bar{z} - \alpha \frac{\sqrt{s^{2}}}{\sqrt{n}}, \bar{z} + \alpha \frac{\sqrt{s^{2}}}{\sqrt{n}} \right]
  \end{equation}
\end{definition}
%%% Local Variables:
%%% mode: latex
%%% TeX-master: "../ETSN10-Network_Architecture_Performance-Reference_Sheet"
%%% End:


\section{Queueing Theory}\label{sec:Queueing_Theory}

%%% Local Variables:
%%% mode: latex
%%% TeX-master: "../ETSN10-Network_Architecture_Performance-Reference_Sheet"
%%% End:


%====================================APPENDIX====================================
\appendix
\counterwithin{definition}{subsection}

\clearpage
\section{Complex Numbers}\label{sec:Complex_Numbers}
\begin{definition}[Complex Number]\label{def:Complex_Number}
  A \emph{complex number} is a hyper real number system.
  This means that two real numbers, $a, b \in \RealNumbers$, are used to construct the set of complex numbers, denoted $\ComplexNumbers$.

  A complex number is written, in Cartesian form, as shown in \Cref{eq:Complex_Number} below.
  \begin{equation}\label{eq:Complex_Number}
    z = a \pm ib
  \end{equation}
  where
  \begin{equation}\label{eq:Imaginary_Value}
    i = \sqrt{-1}
  \end{equation}

  \begin{remark*}[$i$ vs. $j$ for Imaginary Numbers]
    Complex numbers are generally denoted with either $i$ or $j$.
    Electrical engineering regularly makes use of $j$ as the imaginary value.
    This is because alternating current $i$ is already taken, so $j$ is used as the imaginary value instad.
  \end{remark*}
\end{definition}

\subsection{Parts of a Complex Number}\label{subsec:Complex_Number_Parts}
A \nameref{def:Complex_Number} is made of up 2 parts:
\begin{enumerate}[noitemsep]
\item \nameref{def:Real_Part}
\item \nameref{def:Imaginary_Part}
\end{enumerate}

\begin{definition}[Real Part]\label{def:Real_Part}
  The \emph{real part} of an imaginary number, denoted with the $\Re$ operator, is the portion of the \nameref{def:Complex_Number} with no part of the imaginary value $i$ present.

  If $z = x + iy$, then
  \begin{equation}\label{eq:Real_Part}
    \Real{z} = x
  \end{equation}

  \begin{remark}[Alternative Notation]\label{rmk:Real_Part_Alternative_Notation}
    The \nameref{def:Real_Part} of a number sometimes uses a slightly different symbol for denoting the operation.
    It is:
    \begin{equation*}
      \mathfrak{Re}
    \end{equation*}
  \end{remark}
\end{definition}

\begin{definition}[Imaginary Part]\label{def:Imaginary_Part}
  The \emph{imaginary part} of an imaginary number, denoted with the $\Im$ operator, is the portion of the \nameref{def:Complex_Number} where the imaginary value $i$ is present.

  If $z = x + iy$, then
  \begin{equation}\label{eq:Imaginary_Part}
    \Imag{z} = y
  \end{equation}

  \begin{remark}[Alternative Notation]\label{rmk:Imaginary_Part_Alternative_Notation}
    The \nameref{def:Imaginary_Part} of a number sometimes uses a slightly different symbol for denoting the operation.
    It is:
    \begin{equation*}
      \mathfrak{Im}
    \end{equation*}
  \end{remark}
\end{definition}

\subsection{Binary Operations}\label{subsec:Binary_Operations}

%%% Local Variables:
%%% mode: latex
%%% TeX-master: shared
%%% End:


\subsection{Complex Conjugates}\label{app:Complex_Conjugates}
\begin{definition}[Complex Conjugate]\label{def:Complex_Conjugate}
  The conjugate of a complex number is called its \emph{complex conjugate}.
  The complex conjugate of a complex number is the number with an equal real part and an imaginary part equal in magnitude but opposite in sign.
  If we have a complex number as shown below,
  \begin{equation*}
    z = a \pm bi
  \end{equation*}

  then, the conjugate is denoted and calculated as shown below.
  \begin{equation}\label{eq:Complex_Conjugates}
    \Conjugate{z} = a \mp bi
  \end{equation}
\end{definition}

The \nameref{def:Complex_Conjugate} can also be denoted with an asterisk ($*$).
This is generally done for complex functions, rather than single variables.
\begin{equation}\label{eq:Complex_Conjugates_Asterisk}
  z^{*} = \Conjugate{z}
\end{equation}

%%% Local Variables:
%%% mode: latex
%%% TeX-master: shared
%%% End:


\subsection{Geometry of Complex Numbers}\label{subsec:Geometry_Complex_Numbers}
So far, we have viewed \nameref{def:Complex_Number}s only algebraically.
However, we can also view them geometrically as points on a 2 dimensional \nameref{def:Argand_Plane}.

\begin{definition}[Argand Plane]\label{def:Argand_Plane}
  An \emph{Argane Plane} is a standard two dimensional plane whose points are all elements of the complex numbers, $z \in \ComplexNumbers$.
  This is taken from Descarte's definition of a completely real plane.

  The Argand plane contains 2 lines that form the axes, that indicate the real component and the imaginary component of the complex number specified.
\end{definition}

A \nameref{def:Complex_Number} can be viewed as a point in the \nameref{def:Argand_Plane}, where the \nameref{def:Real_Part} is the ``$x$''-component and the \nameref{def:Imaginary_Part} is the ``$y$''-component.

By plotting this, you see that we form a right triangle, so we can find the hypotenuse of that triangle.
This hypotenuse is the distance the point $p$ is from the origin, refered to as the \nameref{def:Complex_Number_Modulus}.
\begin{remark*}
  When working with \nameref{def:Complex_Number}s geometrically, we refer to the points, where they are defined like so:
  \begin{equation*}
    z = x + iy = p(x, y)
  \end{equation*}

  Note that $p$ is \textbf{not} a function of $x$ and $y$.
  Those are the values that inform us \textbf{where} $p$ is located on the \nameref{def:Argand_Plane}.
\end{remark*}

\subsubsection{Modulus of a Complex Number}\label{subsubsec:Complex_Number_Modulus}
\begin{definition}[Modulus]\label{def:Complex_Number_Modulus}
  The \emph{modulus} of a \nameref{def:Complex_Number} is the distance from the origin to the complex point $p$.
  This is based off the Pythagorean Theorem.
  \begin{equation}\label{eq:Complex_Number_Modulus}
    \begin{aligned}
      {\lvert z \rvert}^{2} = x^{2} + y^{2} &= z \Conjugate{z} \\
      \lvert z \rvert &= \sqrt{x^{2} + y^{2}}
    \end{aligned}
  \end{equation}
\end{definition}

\begin{propertylist}
\item The \emph{Law of Moduli} states that $\lvert z w \rvert = \lvert z \rvert \lvert w \rvert$.\label{prop:Law_of_Moduli}.
\end{propertylist}

We can prove \Cref{prop:Law_of_Moduli} using an algebraic identity.
\begin{proof}[Prove \Cref*{prop:Law_of_Moduli}]
  Let $z$ and $w$ be complex numbers ($z, w \in \ComplexNumbers$).
  We are asked to prove
  \begin{equation*}
    \lvert z w \rvert = \lvert z \rvert \lvert w \rvert
  \end{equation*}

  But, it is actually easier to prove
  \begin{equation*}
    {\lvert z w \rvert}^{2} = {\lvert z \rvert}^{2} {\lvert w \rvert}^{2}
  \end{equation*}

  We start by simplifying the ${\lvert z w \rvert}^{2}$ equation above.
  \begin{align*}
    {\lvert z w \rvert}^{2} &= {\lvert z \rvert}^{2} {\lvert w \rvert}^{2} \\
    \intertext{Using the definition of the \nameref{def:Complex_Number_Modulus} of a \nameref{def:Complex_Number} in \Cref{eq:Complex_Number_Modulus}, we can expand the modulus.}
                            &= (z w) (\Conjugate{z w}) \\
    \intertext{Using \Cref{prop:Complex_Conjugate_Split} for multiplication allows us to do the next step.}
                            &= (z w) (\Conjugate{z} \Conjugate{w}) \\
    \intertext{Using Multiplicative Associativity and Multiplicative Commutativity, we can simplify this further.}
                            &= (z \Conjugate{z}) (w \Conjugate{w}) \\
                            &= {\lvert z \rvert}^{2} {\lvert w \rvert}^{2}
  \end{align*}

  Note how we never needed to define $z$ or $w$, so this is as general a result as possible.
\end{proof}

\paragraph{Algebraic Effects of the Modulus' \Cref*{prop:Law_of_Moduli}}\label{par:Law_of_Moduli-Algebraic_Effects}
For this section, let $z = x_{1} + iy_{1}$ and $w = x_{2} + iy_{2}$.
Now,
\begin{align*}
  z w &= (x_{1}x_{2} - y_{1}y_{2}) + i(x_{1}y_{2} + x_{2}y_{1}) \\
  {\lvert z w \rvert}^{2} &= {(x_{1}x_{2} - y_{1}y_{2})}^{2} + {(x_{1}y_{2} + x_{2}y_{1})}^{2} \\
      &= \left( x_{1}^{2} + x_{2}^{2} \right) \left( x_{2}^{2} + y_{2}^{2} \right) \\
      &= {\lvert z \rvert}^{2} {\lvert w \rvert}^{2}
\end{align*}

However, the Law of Moduli (\Cref{prop:Law_of_Moduli}) does \textbf{not} hold for a hyper complex number system one that uses 2 or more imaginaries, i.e.\ $z = a + iy + jz$.
But, the Law of Moduli (\Cref{prop:Law_of_Moduli}) \textbf{does} hold for hyper complex number system that uses 3 imaginaries, $a = z + iy + jz + k \ell$.

\paragraph{Conceptual Effects of the Modulus' \Cref*{prop:Law_of_Moduli}}\label{par:Law_of_Moduli-Conceptual_Effects}
We are interested in seeing if $\lvert z w \rvert = (x_{1}^{2} + y_{1}^{2})(x_{2}^{2}+y_{2}^{2})$ can be extended to more complex terms (3 terms in the complex number).

However, Langrange proved that the equation below \textbf{always} holds.
Note that the $z$ below has no relation to the $z$ above.
\begin{equation*}
  (x_{1} + y_{1} + z_{1}) \neq X^{2} + Y^{2} + Z^{2}
\end{equation*}

%%% Local Variables:
%%% mode: latex
%%% TeX-master: shared
%%% End:


\subsection{Circles and Complex Numbers}\label{subsec:Circles_Complex_Numbers}
We need to define both a center and a radius, just like with regular purely real values.
\Cref{eq:Circles_Complex_Numbers} defines the relation required for a circle using \nameref{def:Complex_Number}s.
\begin{equation}\label{eq:Circles_Complex_Numbers}
  \lvert z - a \rvert = r
\end{equation}

\begin{example}[Lecture 2, Example 1]{Convert to Circle}
  Given the expression below, find the location of the center of the circle and the radius of the circle?
  \begin{equation*}
    \lvert 5 iz + 10 \rvert = 7
  \end{equation*}
  \tcblower{}
  This is just a matter of simplification and moving terms around.
  \begin{align*}
    \lvert 5 iz + 10 \rvert &= 7 \\
    \lvert 5i (z + \frac{10}{5i}) \rvert &= 7 \\
    \lvert 5i (z + \frac{2}{i}) \rvert &= 7 \\
    \lvert 5i (z + \frac{2}{i} \frac{-i}{-i}) \rvert &= 7 \\
    \lvert 5i (z - 2i) \rvert &= 7 \\
    \intertext{Now using the Law of Moduli (\Cref{prop:Law_of_Moduli}) $\lvert a b \rvert = \lvert a \rvert \lvert b \rvert$, we can simplify out the extra imaginary term.}
    \lvert 5i \rvert \lvert z-2i \rvert &= 7 \\
    5 \lvert z - 2i \rvert &= 7 \\
    \lvert z - 2i \rvert = \frac{7}{5}
  \end{align*}

  Thus, the circle formed by the equation $\lvert 5 iz + 10 \rvert = 7$ is actually $\lvert z - 2i \rvert = \frac{7}{5}$, with a center at $a = 2i$ and a radius of $\frac{7}{5}$.
\end{example}

\subsubsection{Annulus}\label{subsubsec:Annulus}
\begin{definition}[Annulus]\label{def:Annulus}
  An \emph{annulus} is a region that is bounded by 2 concentric circles.
  This takes the form of \Cref{eq:Annulus}.
  \begin{equation}\label{eq:Annulus}
    r_{1} \leq \lvert z - a \rvert \leq r_{2}
  \end{equation}

  In \Cref{eq:Annulus}, each of the $\leq$ symbols could also be replaced with $<$.
  This leads to 3 different possibilities for the annulus:
  \begin{enumerate}[noitemsep]
  \item If both inequality symbols are $\leq$, then it is a \textbf{Closed Annulus}.
  \item If both inequality symbols are $<$, then it is an \textbf{Open Annulus}.
  \item If \textbf{only one} inequality symbol $<$ and the other $\leq$, then it is not an \textbf{Open Annulus}.
  \end{enumerate}
\end{definition}


%%% Local Variables:
%%% mode: latex
%%% TeX-master: shared
%%% End:



%%% Local Variables:
%%% mode: latex
%%% TeX-master: shared
%%% End:

\clearpage
\subsection{Trigonometry} \label{app:Trig}
	\subsubsection{Trigonometric Formulas} \label{subsubsec:Trig Formulas}
		\begin{equation} \label{eq:Sin plus Sin with diff Angles}
			\sin \left( \alpha \right) + \sin \left( \beta \right) = 2 \sin \left( \frac{\alpha + \beta}{2} \right) \cos\left( \frac{\alpha - \beta}{2} \right)  
		\end{equation}
		\begin{equation} \label{eq:Cosine-Sine Product}
			\cos \left( \theta \right) \sin \left( \theta \right) = \frac{1}{2} \sin \left( 2 \theta \right)
		\end{equation}
	
	\subsubsection{Euler Equivalents of Trigonometric Functions} \label{subsubsec:Euler Equivalents}
		\begin{equation} \label{eq:Euler Sin}
			\sin \left( x \right) = \frac{e^{\imath x} + e^{-\imath x}}{2}
		\end{equation}
		\begin{equation} \label{eq:Euler Cos}
			\cos \left( x \right) = \frac{e^{\imath x} - e^{-\imath x}}{2 \imath}
		\end{equation}
		\begin{equation} \label{eq:Euler Sinh}
			\sinh \left( x \right) = \frac{e^{x} - e^{-x}}{2}
		\end{equation}
		\begin{equation} \label{eq:Euler Cosh}
			\cosh \left( x \right) = \frac{e^{x} + e^{-x}}{2}
		\end{equation}

\clearpage
\section{Calculus}\label{app:Calculus}
\subsection{L'Hopital's Rule}\label{subsec:LHopitals_Rule}
L'Hopital's Rule can be used to simplify and solve expressions regarding limits that yield irreconcialable results.
\begin{lemma}[L'Hopital's Rule]\label{lemma:LHopitals_Rule}
  If the equation
  \begin{equation*}
    \lim\limits_{x \rightarrow a} \frac{f(x)}{g(x)} =
    \begin{cases}
      \frac{0}{0} \\
      \frac{\infty}{\infty} \\
    \end{cases}
  \end{equation*}
  then \Cref{eq:LHopitals_Rule} holds.
  \begin{equation}\label{eq:LHopitals_Rule}
    \lim\limits_{x \rightarrow a} \frac{f(x)}{g(x)} = \lim\limits_{x \rightarrow a} \frac{f'(x)}{g'(x)}
  \end{equation}
\end{lemma}

\subsection{Fundamental Theorems of Calculus}\label{subsec:Fundamental Theorem of Calculus}
\begin{definition}[First Fundamental Theorem of Calculus]\label{def:1st Fundamental Theorem of Calculus}
  The \emph{first fundamental theorem of calculus} states that, if $f$ is continuous on the closed interval $\left[ a,b \right]$ and $F$ is the indefinite integral of $f$ on $\left[ a,b \right]$, then

  \begin{equation}\label{eq:1st Fundamental Theorem of Calculus}
    \int_{a}^{b}f \left( x \right) dx = F \left( b \right) - F \left( a \right)
  \end{equation}
\end{definition}

\begin{definition}[Second Fundamental Theorem of Calculus]\label{def:2nd Fundamental Theorem of Calculus}
  The \emph{second fundamental theorem of calculus} holds for $f$ a continuous function on an open interval $I$ and $a$ any point in $I$, and states that if $F$ is defined by

  \begin{equation*}
    F \left( x \right) = \int_{a}^{x} f \left( t \right) dt,
  \end{equation*}
  then
  \begin{equation}\label{eq:2nd Fundamental Theorem of Calculus}
    \begin{aligned}
      \frac{d}{dx} \int_{a}^{x} f \left( t \right) dt &= f \left( x \right) \\
      F' \left( x \right) &= f \left( x \right) \\
    \end{aligned}
  \end{equation}
\end{definition}

\begin{definition}[argmax]\label{def:argmax}
  The arguments to the \emph{argmax} function are to be maximized by using their derivatives.
  You must take the derivative of the function, find critical points, then determine if that critical point is a global maxima.
  This is denoted as
  \begin{equation*}\label{eq:argmax}
    \argmax_{x}
  \end{equation*}
\end{definition}

\subsection{Rules of Calculus}\label{subsec:Rules of Calculus}
\subsubsection{Chain Rule}\label{subsubsec:Chain Rule}
\begin{definition}[Chain Rule]\label{def:Chain Rule}
  The \emph{chain rule} is a way to differentiate a function that has 2 functions multiplied together.

  If
  \begin{equation*}
    f(x) = g(x) \cdot h(x)
  \end{equation*}
  then,
  \begin{equation}\label{eq:Chain Rule}
    \begin{aligned}
      f'(x) &= g'(x) \cdot h(x) + g(x) \cdot h'(x) \\
      \frac{df(x)}{dx} &= \frac{dg(x)}{dx} \cdot g(x) + g(x) \cdot \frac{dh(x)}{dx} \\
    \end{aligned}
  \end{equation}
\end{definition}

\subsection{Useful Integrals}\label{subsec:Useful_Integrals}
\begin{equation}\label{eq:Cosine_Indefinite_Integral}
  \int \cos(x) \; dx = \sin(x)
\end{equation}

\begin{equation}\label{eq:Sine_Indefinite_Integral}
  \int \sin(x) \; dx = -\cos(x)
\end{equation}

\begin{equation}\label{eq:x_Cosine_Indefinite_Integral}
  \int x \cos(x) \; dx = \cos(x) + x \sin(x)
\end{equation}
\Cref{eq:x_Cosine_Indefinite_Integral} simplified with Integration by Parts.

\begin{equation}\label{eq:x_Sine_Indefinite_Integral}
  \int x \sin(x) \; dx = \sin(x) - x \cos(x)
\end{equation}
\Cref{eq:x_Sine_Indefinite_Integral} simplified with Integration by Parts.

\begin{equation}\label{eq:x_Squared_Cosine_Indefinite_Integral}
  \int x^{2} \cos(x) \; dx = 2x \cos(x) + (x^{2} - 2) \sin(x)
\end{equation}
\Cref{eq:x_Squared_Cosine_Indefinite_Integral} simplified by using Integration by Parts twice.

\begin{equation}\label{eq:x_Squared_Sine_Indefinite_Integral}
  \int x^{2} \sin(x) \; dx = 2x \sin(x) - (x^{2} - 2) \cos(x)
\end{equation}
\Cref{eq:x_Squared_Sine_Indefinite_Integral} simplified by using Integration by Parts twice.

\begin{equation}\label{eq:Exponential_Cosine_Indefinite_Integral}
  \int e^{\alpha x} \cos(\beta x) \; dx = \frac{e^{\alpha x} \bigl( \alpha \cos(\beta x) + \beta \sin(\beta x) \bigr)}{\alpha^{2} + \beta^{2}} + C
\end{equation}

\begin{equation}\label{eq:Exponential_Sine_Indefinite_Integral}
  \int e^{\alpha x} \sin(\beta x) \; dx = \frac{e^{\alpha x} \bigl( \alpha \sin(\beta x) - \beta \cos(\beta x) \bigr)}{\alpha^{2}+\beta^{2}} + C
\end{equation}

\begin{equation}\label{eq:Exponential_Indefinite_Integral}
  \int e^{\alpha x} \; dx = \frac{e^{\alpha x}}{\alpha}
\end{equation}

\begin{equation}\label{eq:x_Exponential_Indefinite_Integral}
  \int x e^{\alpha x} \; dx = e^{\alpha x} \left( \frac{x}{\alpha} - \frac{1}{\alpha^{2}} \right)
\end{equation}
\Cref{eq:x_Exponential_Indefinite_Integral} simplified with Integration by Parts.

\begin{equation}\label{eq:Inverse_x_Indefinite_Integral}
  \int \frac{dx}{\alpha + \beta x} = \int \frac{1}{\alpha + \beta x} \; dx = \frac{1}{\beta} \ln (\alpha + \beta x)
\end{equation}

\begin{equation}\label{eq:Inverse_x_Squared_Indefinite_Integral}
  \int \frac{dx}{\alpha^{2} + \beta^{2} x^{2}} = \int \frac{1}{\alpha^{2} + \beta^{2} x^{2}} \; dx = \frac{1}{\alpha \beta} \arctan \left( \frac{\beta x}{\alpha} \right)
\end{equation}

\begin{equation}\label{eq:a_Exponential_Indefinite_Integral}
  \int \alpha^{x} \; dx = \frac{\alpha^{x}}{\ln(\alpha)}
\end{equation}

\begin{equation}\label{eq:a_Exponential_Derivative}
  \frac{d}{dx} \alpha^{x} = \frac{d\alpha^{x}}{dx} = \alpha^{x} \ln(x)
\end{equation}

\subsection{Leibnitz's Rule}\label{subsec:Leibnitzs_Rule}
\begin{lemma}[Leibnitz's Rule]\label{lemma:Leibnitzs_Rule}
  Given
  \begin{equation*}
    g(t) = \int_{a(t)}^{b(t)} f(x, t) \, dx
  \end{equation*}
  with $a(t)$ and $b(t)$ differentiable in $t$ and $\frac{\partial f(x, t)}{\partial t}$ continuous in both $t$ and $x$, then
  \begin{equation}\label{eq:Leibnitzs_Rule}
    \frac{d}{dt} g(t) = \frac{d g(t)}{dt} = \int_{a(t)}^{b(t)} \frac{\partial f(x, t)}{\partial t} \, dx + f \bigl[ b(t), t \bigr] \, \frac{d b(t)}{dt} - f \bigl[ a(t), t \bigr] \, \frac{d a(t)}{dt}
  \end{equation}
\end{lemma}



\clearpage
\section{Laplace Transform}\label{app:Laplace_Transform}
\subsection{Laplace Transform}\label{subsec:Laplace_Transform}
\begin{definition}[Laplace Transform]\label{def:Laplace_Transform}
  The \emph{Laplace transformation} operation is denoted as $\Lapl \lbrace x(t) \rbrace$ and is defined as
  \begin{equation}\label{eq:Laplace_Transform}
    X(s) = \int\limits_{-\infty}^{\infty} x(t) e^{-st} dt
  \end{equation}
\end{definition}

\subsection{Inverse Laplace Transform}\label{subsec:Inverse_Laplace_Transform}
\begin{definition}[Inverse Laplace Transform]\label{def:Inverse_Laplace_Transform}
  The \emph{inverse Laplace transformation} operation is denoted as $\Lapl^{-1} \lbrace X(s) \rbrace$ and is defined as
  \begin{equation}\label{eq:Inverse_Laplace_Transform}
    x(t) = \frac{1}{2j \pi} \int_{\sigma-\infty}^{\sigma+\infty} X(s) e^{st} \, ds
  \end{equation}
\end{definition}

\subsection{Properties of the Laplace Transform}\label{subsec:Laplace_Transform_Properties}
\subsubsection{Linearity}\label{subsubsec:Laplace_Linearity}
The \nameref{def:Laplace_Transform} is a linear operation, meaning it obeys the laws of linearity.
This means \Cref{eq:Laplace_Linearity} must hold.
\begin{subequations}\label{eq:Laplace_Linearity}
  \begin{equation}\label{eq:Laplace_Linearity_Time}
    x(t) = \alpha_{1} x_{1}(t) + \alpha_{2} x_{2}(t)
  \end{equation}
  \begin{equation}\label{eq:Laplace_Linearity_Frequency}
    X(s) = \alpha_{1} X_{1}(s) + \alpha_{2} X_{2}(s)
  \end{equation}
\end{subequations}

\subsubsection{Time Scaling}\label{subsubsec:Laplace_Time_Scaling}
Scaling in the time domain (expanding or contracting) yields a slightly different transform.
However, this only makes sense for $\alpha > 0$ in this case.
This is seen in \Cref{eq:Laplace_Time_Scaling}.
\begin{equation}\label{eq:Laplace_Time_Scaling}
  \Lapl \bigl\lbrace x(\alpha t) \bigr\rbrace = \frac{1}{\alpha} X \left( \frac{s}{\alpha} \right)
\end{equation}

\subsubsection{Time Shift}\label{subsubsec:Laplace_Time_Shift}
Shifting in the time domain means to change the point at which we consider $t=0$.
\Cref{eq:Laplace_Time_Shifting} below holds for shifting both forward in time and backward.
\begin{equation}\label{eq:Laplace_Time_Shifting}
  \Lapl \bigl\lbrace x(t-a) \bigr\rbrace = X(s) e^{-a s}
\end{equation}

\subsubsection{Frequency Shift}\label{subsubsec:Laplace_Frequency_Shift}
Shifting in the frequency domain means to change the complex exponential in the time domain.
\begin{equation}\label{eq:Laplace_Frequency_Shift}
  \Lapl^{-1} \bigl\lbrace X(s-a) \bigr\rbrace = x(t)e^{at}
\end{equation}

\subsubsection{Integration in Time}\label{subsubsec:Laplace_Time_Integration}
Integrating in time is equivalent to scaling in the frequency domain.
\begin{equation}\label{eq:Laplace_Time_Integration}
  \Lapl \left\lbrace \int_{0}^{t} x(\lambda) \, d\lambda \right\rbrace = \frac{1}{s} X(s)
\end{equation}

\subsubsection{Frequency Multiplication}\label{subsubsec:Laplace_Frequency_Multiplication}
Multiplication of two signals in the frequency domain is equivalent to a convolution of the signals in the time domain.
\begin{equation}\label{eq:Laplace_Frequency_Multiplication}
  \Lapl \bigl\lbrace x(t) * v(t) \bigr\rbrace = X(s) V(s)
\end{equation}

\subsubsection{Relation to Fourier Transform}\label{subsubsec:Fourier_Transform_Relation}
The Fourier transform looks and behaves very similarly to the Laplace transform.
In fact, if $X(\omega)$ exists, then \Cref{eq:Fourier_Laplace_Transform_Relation} holds.
\begin{equation}\label{eq:Fourier_Laplace_Transform_Relation}
  X(s) = X(\omega) \vert_{\omega = \frac{s}{j}}
\end{equation}

\subsection{Theorems}\label{subsec:Laplace_Theorems}
There are 2 theorems that are most useful here:
\begin{enumerate}[noitemsep]
\item \nameref{thm:Laplace_Initial_Value_Theorem}
\item \nameref{thm:Laplace_Final_Value_Theorem}
\end{enumerate}


%%% Local Variables:
%%% mode: latex
%%% TeX-master: shared
%%% End:


% To make this print, you must include a citation somewhere in the document
\clearpage
\printbibliography{}
\end{document}

%%% Local Variables:
%%% mode: latex
%%% TeX-master: t
%%% End:
