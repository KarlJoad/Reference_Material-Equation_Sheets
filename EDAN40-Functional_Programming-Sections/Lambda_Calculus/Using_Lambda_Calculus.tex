\subsection{Using Lambda Calculus}\label{subsec:Using_Lambda_Calculus}
When using \nameref{def:Lambda_Calculus}, there are some things to keep in mind.
\begin{enumerate}[noitemsep]
\item Reductions involve reducing or manipulating an expression somehow.
  There are 3 kinds of reductions, the first 2 being the most common.
  \begin{enumerate}[noitemsep]
  \item \nameref{def:Alpha_Conversion}
  \item \nameref{def:Beta_Reduction}
  \item \nameref{def:Eta_Reduction}
  \end{enumerate}
\item Computations in $\lambda$ calculus is performed by performing successive $\beta$-reductions whenever possible and for as long as possible.
\end{enumerate}

\begin{definition}[Alpha Conversion]\label{def:Alpha_Conversion}
  \emph{Alpha conversion} (\emph{$\alpha$-conversion}) is the process of replacing one variable name with another.
  For example, in the expression, $\lambda x. zx$, an $\alpha$-conversion of $x$ to $y$ results in $\lambda y. zy$.
\end{definition}


%%% Local Variables:
%%% mode: latex
%%% TeX-master: "../../EDAN40-Functional_Programming-Reference_Sheet"
%%% End:
