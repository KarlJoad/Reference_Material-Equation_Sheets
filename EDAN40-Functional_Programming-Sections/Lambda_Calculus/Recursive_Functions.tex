\subsection{Recursive Functions}\label{subsec:Recursive_Functions}
The type of a function that takes in a finite number of arguments, $k$ (called the function's arity) is denoted $\NaturalNumbers^{k}$, where $\NaturalNumbers$ is the set of all natural numbers.

Functions $\NaturalNumbers^{k} \rightarrow \NaturalNumbers$ represents all computable functions.
\begin{enumerate}[noitemsep]
\item \emph{Succcessor:} The function $s : \NaturalNumbers \rightarrow \NaturalNumbers$ given by the definition $s(x) = x + 1$ is computable.
\item \emph{Zero:} The function $z : \NaturalNumbers^{0} \rightarrow \NaturalNumbers$ given by the definition $z() = 0$ is computable.
\item \emph{Projections:} The functions $\pi_{n}^{k} : \NaturalNumbers^{k} \rightarrow \NaturalNumbers$ given by the definition $\pi_{n}^{k}(x_{1, \ldots, x_{k}}) = x_{n}$ for $1 \leq n \leq k$.
\item \emph{Composition:} If the function $f : \NaturalNumbers^{k} \rightarrow \NaturalNumbers$ and functions $g_{1}, \ldots, g_{k} : \NaturalNumbers^{n} \rightarrow \NaturalNumbers$ are computable, then so is the function $f \circ (g_{1}, \ldots, g_{k}) : \NaturalNumbers^{n} \rightarrow \NaturalNumbers$ such that when given the input $\hat{x} = x_{1}, \ldots, x_{n}$ it returns $f(g_{1}(\hat{x}), \ldots g_{k}(\hat{x}))$.
\item \emph{Primitive Recursion:} If $h_{i} : \NaturalNumbers^{n-1} \rightarrow \NaturalNumbers$ and $g_{i} : \NaturalNumbers^{n+k} \rightarrow \NaturalNumbers$ are computable, for $1 \leq i \leq k$, then so are the functions $f_{i} : \NaturalNumbers^{n} \rightarrow \NaturalNumbers$ for $1 \leq i \leq k$ defined by the mutual induction as follows:
  \begin{align*}
    f_{i}(0, \hat{x}) &\overset{df}{=} h_{i}(\hat{x}) \\
    f_{i}(x+1, \hat{x}) &\overset{df}{=} g_{i} \bigl( x, \hat{x}, f_{1}(x, \hat{x}), \ldots, f_{k}(x, \hat{x}) \bigr)
  \end{align*}
  where $\hat{x} = x_{2}, \ldots, x_{n}$.
\item \emph{Unbounded Minimization:} If $g : \NaturalNumbers^{n+1} \rightarrow \NaturalNumbers$ is computable, then so is the function $f : \NaturalNumbers^{n} \rightarrow \NaturalNumbers$ that when given input $\hat{x} = x_{1}, \ldots, x_{n}$ gives the least $y$ such that $g(y, \hat{x})$ is defined for all $z \leq y$ and $g(y, \hat{x}) = 0$, if such a $y$ exists otherwise, it is undefined.
  We denote this by
  \begin{equation*}
    f(\hat{x}) = \mu y. (g (y, \hat{x}) = 0)
  \end{equation*}
\end{enumerate}

\begin{itemize}[noitemsep]
\item \emph{Primitive Recusive functions} obey rules 1 through 5.
  \begin{itemize}[noitemsep]
  \item These are totally recursive functions.
  \end{itemize}
\item \emph{$mu$-recursive functions} obey rules 1 through 6.
  \begin{itemize}[noitemsep]
  \item These are partially recursive functions.
  \end{itemize}
\end{itemize}

There exists a complete non-primitve totally recursive function called Ackermann's function, shown in \Cref{eq:Ackermanns_Function}.
\begin{equation}\label{eq:Ackermanns_Function}
  \begin{aligned}
    A(0, y) &= y+1 \\
    A(x+1, 0) &= A(x, 1) \\
    A(x+1, y+1) = A \bigl( x, A(x+1, y) \bigr) \\
  \end{aligned}
\end{equation}

%%% Local Variables:
%%% mode: latex
%%% TeX-master: "../../EDAN40-Functional_Programming-Reference_Sheet"
%%% End:
