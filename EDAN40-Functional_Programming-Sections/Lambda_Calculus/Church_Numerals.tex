\subsection{Church Numerals}\label{subsec:Church_Numerals}
Church numerals are a way of defining the integer numbers, including 0 that is completely expressible in \nameref{def:Lambda_Calculus}.
In terms of $\lambda$-functions, it is expressed like so:
\begin{equation}\label{eq:Church_Numerals}
  \begin{aligned}
    0 &= \lambda f.\lambda x.x \\
    1 &= \lambda f.\lambda x. fx \\
    2 &= \lambda f.\lambda x. f(f x) \\
    3 &= \lambda f.\lambda x. f(f (f x)) \\
    &\vdots \\
    n &= \lambda f. \lambda x. f^{n} x \\
  \end{aligned}
\end{equation}

Then, we can define the successor function (\texttt{succ}) as seen in \Cref{eq:Church_Numerals_Succ}
\begin{equation}\label{eq:Church_Numerals_Succ}
  \lambda m. \lambda f. \lambda x. f (m f x)
\end{equation}


%%% Local Variables:
%%% mode: latex
%%% TeX-master: "../../EDAN40-Functional_Programming-Reference_Sheet"
%%% End:
