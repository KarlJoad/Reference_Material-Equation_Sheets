\subsection{Pure Lambda Calculus}\label{subsec:Pure_Lambda_Calculus}.
In pure lambda calculus, there are only 3 things, called \nameref{tab:Lambda_Terms}.

\begin{table}[h!tbp]
  \centering
  \begin{tabular}{ccp{15cm}}
    \toprule
    Syntax & Name & \multicolumn{1}{c}{Description} \\
    \midrule
    $x$ & Variable & A character or string representing a parameter or mathematical/logical value. \\
$ (\lambda x. M)$ & $\lambda$-Abstraction & Function definition ($M$ is a lambda term). The variable $x$ becomes bound in the expression. \\
    $(M N)$ & $\lambda$-Application & Applying a function to an argument. M and N are lambda terms. \\
    \bottomrule
  \end{tabular}
\caption{$\lambda$-Terms}
\label{tab:Lambda_Terms}
\end{table}

Note that in the case of $\lambda$-Application, the operation \textbf{IS NOT} associative.
This means that, in general (there are exceptions), the following relation does \textbf{not} hold.
\begin{equation*}
  (MN) P = M(NP)
\end{equation*}

Additionally, $\lambda$-Application is assumed to be left-associative, meaning
\begin{equation*}
  MNP = (MN) P
\end{equation*}


%%% Local Variables:
%%% mode: latex
%%% TeX-master: "../../EDAN40-Functional_Programming-Reference_Sheet"
%%% End:
