\section{Declaring Types}\label{sec:Declaring_Types}
New \nameref{def:Type}s that more closely model your problem can be very helpful when writing a program.
This also helps ensure that the type behaves according to the type checking rules you write for it.

\subsection{Type Declarations}\label{subsec:Type_Declarations}

%%% Local Variables:
%%% mode: latex
%%% TeX-master: "../../EDAN40-Functional_Programming-Reference_Sheet"
%%% End:


\begin{definition}[Type Alias]\label{def:Type_Alias}
  A \emph{type alias} is the process of creating another name for an already existing \nameref{def:Type}.
  The langauge system can just ``replace'' each occurrence of a type alias with its definition until it has fully expanded the \nameref{def:Type}s until it has reached its base types.
  Type aliases can also be nested, allowing for one to define itself \textbf{IN TERMS OF ANOTHER}.

  \begin{remark}[Recursive Types]\label{rmk:Recursive_Type_Alias}
    \textbf{IT IS NOT POSSIBLE TO CONSTRUCT A RECURSIVE DATA TYPE USING A \nameref{def:Type_Alias}!!}
    That can only be done with the \haskellinline{data} keyword.
  \end{remark}
\end{definition}

%%% Local Variables:
%%% mode: latex
%%% TeX-master: "../EDAN40-Functional_Programming-Reference_Sheet"
%%% End:
