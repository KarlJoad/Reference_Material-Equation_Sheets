\section{Introduction}\label{sec:Introduction}
This section is dedicated to giving a small introduction to functional programming.
Functional Programming is a style of programming, nothing else.
In this style, the basic method of computation is the evaluation of expressions as arguments to functions, which themselves return expressions.

\begin{quote}
  ``Functional programming is so called because a program consists entirely of functions. [$\ldots$]
  These functions are much like ordinary mathematical functions [$\ldots$] defined by ordinary equations'' (John Hughes)
\end{quote}

If you want to view all possible language categories, visit \href{https://en.wikipedia.org/wiki/Programming_paradigm}{Wikipedia's Programming Paradigms}.

\begin{definition}[Imperative Programming Language]\label{def:Imperative_Programming_Language}
  \emph{Imperative programming languages} have a programming paradigm that uses statements that change a program's state.
  An imperative program consists of commands for the computer to perform.
  Imperative programming focuses on describing how a program operates.
\end{definition}

\begin{restatable}[Functional Programming Language]{definition}{defFunctionalProgrammingLanguage}\label{def:Functional_Programming_Language}
  \emph{Functional programming languages} treat computation as the evaluation of mathematical functions and avoids changing-state and mutable data.
  It is a declarative programming paradigm in that programming is done with expressions or declarations instead of statements.
  In functional code, the output value of a function depends only on its arguments, so calling a function with the same value for an argument always produces the same result.

  This is in contrast to \nameref{def:Imperative_Programming_Language}s where, in addition to a function's arguments, global program state can affect a function's resulting value.
  Eliminating side effects, that is, changes in state that do not depend on the function inputs, can make understanding a program easier, which is one of the key motivations for the development of functional programming.

  Because of its close relationship to mathematics, it is much easier to develop mathematical techniques for reasoning about and proving the behavior of programs developed in functional languages.
  These techniques are important tools for helping us to ensure that programs work properly without having to resort to tedious testing and debugging which can only show the presence of errors, never their absence.
  Moreover, they provide important tools for documenting the reasoning that went into the formulation of a program, making the code easier to understand and maintain.

  \begin{remark}[Course Language]
    The languages of use in this course is Haskell.
    It is a purely functional language that supports impure actions with \nameref{def:Monad}s.
  \end{remark}
\end{restatable}

Functional programming is very nice because it allows us to perform certain actions that are quite natural quite easily.
For example,
\begin{itemize}[noitemsep]
\item \nameref{subsubsec:Higher_Order_Functions}
  \begin{itemize}[noitemsep]
  \item Functions that take functions as arguments and return functions as expressions
  \item Used frequently
  \item Currying
  \item How to use effectively?
  \end{itemize}

\item \nameref{subsubsec:Infinite_Data_Structures}
  \begin{itemize}[noitemsep]
  \item Nice idea that is easily proven in functional languages
  \end{itemize}

\item Lazy evaluation (This is a function unique to Haskell)
  \begin{itemize}[noitemsep]
  \item Only evaluate expressions \textbf{ONLY WHEN NEEDED}
  \item This also allow us to deal with idea of infinite data structures
  \end{itemize}
\end{itemize}

\subsection{Rewrite Semantics}\label{subsec:Rewrite_Semantics}
One of the key strengths of \nameref{def:Functional_Programming_Language}s is the fact we can easily perform \nameref{def:Rewrite_Semantics} on any given \nameref{def:Expression}.

\begin{definition}[Rewrite Semantics]\label{def:Rewrite_Semantics}
  \emph{Rewrite semantics} is the process of rewriting and deconstructing an \nameref{def:Expression} into its constiuent parts.
  Rewrite semantics answers the question ``How do we extract values from functions?''
\end{definition}

\begin{listing}[h!tbp]
\haskellsourcefile{./EDAN40-Functional_Programming-Sections/Introduction/Code/Rewrite_Factorial.hs}
\caption{\nameref{def:Rewrite_Semantics} of a Factorial Function}
\label{lst:Rewrite_Semantics}
\end{listing}

\begin{definition}[Expression]\label{def:Expression}
  An \emph{expression} is a combination of one or more \nameref{def:Operand}s and operators that the programming language interprets (according to its particular rules of precedence and of association) and computes to produce another value.

  \begin{remark}[Overloading]\label{rmk:Overloading_Expressions}
    An expression can be overloaded if there is more than one definition for an operator.
  \end{remark}
\end{definition}

\begin{definition}[Operand]\label{def:Operand}
  An \emph{operand} is a:
  \begin{itemize}[noitemsep]
  \item Constant
  \item Variable
  \item Another \nameref{def:Expression}
  \item Result from function calls
  \end{itemize}
\end{definition}

%%% Local Variables:
%%% mode: latex
%%% TeX-master: "../../EDAN40-Functional_Programming-Reference_Sheet"
%%% End:


\subsection{Paradigm Differences}\label{subsec:Paradigm_Differences}
Functional programming is a completely different paradigm of programming than traditional imperative programming.
One of the biggest differences is that \textbf{side effects are NOT allowed}.

\subsubsection{Side Effects}\label{subsubsec:Side_Effects}
Side effects are typically defined as being function-local.
So, we can assign variables, make lists, etc. \textbf{so long as the effects are destroyed upon leaving the function}.
Additionally, nothing globally usable can/should be changed.

\begin{listing}[h!tbp]
\begin{minted}[frame=lines,linenos,style=emacs,autogobble=true,breaklines=true]{c}
public int f(int x) {
	int t1 = g(x) + g(x);
	int t2 = 2 * g(x);
	return t1-t2;
}
// We should probably get 0 back.
// f(x) = t1-t2 = g(x) + g(x) - 2*g(x) = 0

// But, if g(x) is defined like so,
public int g(int x) {
	int y = input.nextInt();
	return y;
}
// The two instances of g(x) (g(x) + g(x)) can be different values,
// This invalidates the result we reached made earlier.
\end{minted}
\caption{C-Like Code with Side Effects}
\label{lst:Side_Effects}
\end{listing}

\subsubsection{Syntactic Differences}\label{subsubsec:Syntactic_Differences}
The \texttt{=} symbol has different meanings in \nameref{def:Functional_Programming_Language}s.
In functional languages, \texttt{=}, is the mathematical definition of equivalence.
Whereas in \nameref{def:Imperative_Programming_Language}s, \texttt{=} is the assignment of values to memory locations.

Typically, \nameref{def:Functional_Programming_Language}s do not have a way to directly access memory, since that is an inherently stateful change, breaking the rules of ``side-effect free''.
However, ``variables'' \textbf{do} exist, but they are different.
\begin{itemize}[noitemsep]
\item Variables are \textbf{NAMED} expressions, not locations in memory
\item When ``reassigning'' a variable, the old value that name pointed to is discarded, and a new one created.
\end{itemize}

\subsubsection{Tendency Towards Recursion}\label{subsubsec:Tendency_Recursion}
Most \nameref{def:Functional_Programming_Language}s use recursion more than they use iteration.
This is possible because recursion can express all solutions that iteration can, but that does not hold true the other way around.
Recursion is also intimately tied to the computability of an \nameref{def:Expression}.

Take the code snippet below as an example.
It sums all values from a list of arbitrary size by taking the front element of the provided list (\texttt{x}) and adding that to the results of adding the rest of the list (\texttt{xs}) together.

\begin{listing}[h!tbp]
\haskellsourcefile{./EDAN40-Functional_Programming-Sections/Introduction/Code/sum1.hs}
\caption{Basic List Summation}
\label{lst:Recursion_List_Summation}
\end{listing}

\subsubsection{Higher-Order Functions}\label{subsubsec:Higher_Order_Functions}
Similarly to what we defined in \Cref{lst:Recursion_List_Summation}, say we want to define the operations:
\begin{itemize}[noitemsep]
\item Multiplying all elements together
\item Finding if any elements are \texttt{True}.
\item Finding if all the elements are \texttt{True}.
\end{itemize}

It would look like the code shown below.
The code from \Cref{lst:Recursion_List_Summation} will be included.
\begin{listing}[h!tbp]
\haskellsourcefile{./EDAN40-Functional_Programming-Sections/Introduction/Code/Many_Funcs_No_Higher_Order.hs}
\caption{List Comprehension Functions, No Higher-Order Functions Used}
\label{lst:Many_Funcs_No_Higher_Order}
\end{listing}

If you look at each of the functions, you will notice something in common between all of them.
\begin{itemize}[noitemsep]
\item There is a default value, depending on the operation, for when the list is empty.
\item There is an operation applied between the current element and,
\item The rest of the list is recursively operated upon.
\end{itemize}

If we instead used a higher-order function, we can define all of those functions with just one higher-order function.
\begin{listing}[h!tbp]
\haskellsourcefile{./EDAN40-Functional_Programming-Sections/Introduction/Code/Many_Funcs_Higher_Order.hs}
\caption{List Comprehension Functions, Higher-Order Functions Used}
\label{lst:Many_Funcs_Higher_Order}
\end{listing}

\subsubsection{Infinite Data Structures}\label{subsubsec:Infinite_Data_Structures}
One of the benefits of lazy evaluation, and allowing higher-order functions, is that infinite data structures can be created.
So, we could have a list of \textbf{all} integers, but we will not run out of memory (probably).
Because of lazy evaluation, the values from these infinite data structures are computed \textbf{on when needed}.

For example, we find all prime numbers, starting with 2, using the Eratosthenes Sieve method (\Cref{lst:Infinite_Data_Structure}).
This method states we take \textbf{ALL} integers, starting from 2
\begin{enumerate}[noitemsep]
\item Make a list out of them.
\item Take the first element out.
\item Remove all multiples of that number.
\item Put that number into a list of primes.
\item Repeat from step 2, until you find all the prime numbers you want.
\end{enumerate}

In Haskell, this looks like:
\begin{listing}[h!tbp]
\haskellsourcefile{./EDAN40-Functional_Programming-Sections/Introduction/Code/Eratosthenes_Primes.hs}
\caption{Infinite Data Structure, All Primes by Eratosthenes Sieve}
\label{lst:Infinite_Data_Structure}
\end{listing}

%%% Local Variables:
%%% mode: latex
%%% TeX-master: "../../EDAN40-Functional_Programming-Reference_Sheet"
%%% End:


%%% Local Variables:
%%% mode: latex
%%% TeX-master: "../EDAN40-Functional_Programming-Reference_Sheet"
%%% End:
