\subsection{Infinite Structures}\label{subsec:Infinite_Structures}
The use of \nameref{def:Lazy_Evaluation} allows Haskell to program with infinite structures.

\begin{definition}[Lazy Evaluation]\label{def:Lazy_Evaluation}
  \emph{Lazy evaluation} is the combination of call-by-name evaluation and the use of pointers to share references to common \nameref{def:Expression}s.
  Lazy evaluation has the property that it ensures that evaluation terminates as often as possible.
  Sharing ensures that lazy evaluation never requires more steps than call-by-value evaluation.
  In addition, using lazy evaluation expressions are only evaluated as much as required by the context in which they are used.
\end{definition}

In this section, we will use the definition of \texttt{ones} below.
\begin{listing}[h!tbp]
\haskellsourcefile{./EDAN40-Functional_Programming-Sections/Lazy_Evaluation/Code/ones.hs}
\caption{Definition of an Infinite List of ones for \Cref{subsec:Infinite_Structures}}
\label{lst:Ones_Definition}
\end{listing}


%%% Local Variables:
%%% mode: latex
%%% TeX-master: "../../EDAN40-Functional_Programming-Reference_Sheet"
%%% End:
