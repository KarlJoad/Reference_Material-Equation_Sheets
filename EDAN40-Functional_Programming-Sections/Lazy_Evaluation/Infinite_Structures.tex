\subsection{Infinite Structures}\label{subsec:Infinite_Structures}
The use of \nameref{def:Lazy_Evaluation} allows Haskell to program with infinite structures.

\begin{definition}[Lazy Evaluation]\label{def:Lazy_Evaluation}
  \emph{Lazy evaluation} is the combination of call-by-name evaluation and the use of pointers to share references to common \nameref{def:Expression}s.
  Lazy evaluation has the property that it ensures that evaluation terminates as often as possible.
  Sharing ensures that lazy evaluation never requires more steps than call-by-value evaluation.
  In addition, using lazy evaluation expressions are only evaluated as much as required by the context in which they are used.
\end{definition}

In this section, we will use the definition of \texttt{ones} below.
\begin{listing}[h!tbp]
\haskellsourcefile{./EDAN40-Functional_Programming-Sections/Lazy_Evaluation/Code/ones.hs}
\caption{Definition of an Infinite List of ones for \Cref{subsec:Infinite_Structures}}
\label{lst:Ones_Definition}
\end{listing}

No matter what evaluation strategy you use, name or value, this definition of \texttt{ones} will never terminate.
However, when we attempt to use it (for example with \haskellinline{head ones}), depending on the evaluation method, we might get an answer.

Using call-by-value evaluation results in non-termination.
\begin{align*}
  &= head ones \\
  &= head (1 : ones) \\
  &= head (1 : (1 : ones)) \\
  &= head (1 : (1 : (1 : ones))) \\
  \vdots
\end{align*}

Using call-by-name evaluation, sharing is not required in this example results in termination in two steps.
\begin{align*}
  &= head ones \\
  &= head (1 : ones) \\
  &= 1
\end{align*}

This works because \nameref{def:Lazy_Evaluation} proceeds in a lazy manner as its name suggests, only evaluating arguments as and when strictly necessary to produce results.

%%% Local Variables:
%%% mode: latex
%%% TeX-master: "../../EDAN40-Functional_Programming-Reference_Sheet"
%%% End:
