\subsection{Strict Application}\label{subsec:Strict_Application}
Haskell uses \nameref{def:Lazy_Evaluation} by default, but also provides a special \nameref{def:Strict} version of \nameref{def:Function} application, written as \texttt{\$!}, which can be useful.

\begin{definition}[Strict]\label{def:Strict}
  \emph{Strict} function application, in Haskell, means that the specified argument provided to the \nameref{def:Function} is evaluated with \nameref{subsec:Pass_by_Value}, rather than \nameref{subsec:Pass_by_Name}.

  Formally, an expression of the form \haskellinline{f $! x} is only a \nameref{def:Reducible_Expression} once evaluation of the argument \texttt{x}, which itself uses \nameref{def:Lazy_Evaluation}, has reached the point where it is known that the result is not an undefined value.
  At that point the expression can be reduced to the normal application \haskellinline{f x}.
\end{definition}

\begin{itemize}[noitemsep]
\item If the argument has a basic type, such as \texttt{Int} or \texttt{Bool}, then top-level evaluation is simply complete evaluation.
\item For a \nameref{def:Tuple} type such as \texttt{(Int, Bool)}, evaluation is performed until a pair of expressions is obtained, but no further.
\item For a \nameref{def:List} type, evaluation is performed until the empty list or the cons of two expressions is obtained.
\end{itemize}

When used with \nameref{subsec:Curried_Function_Types} with multiple arguments, \nameref{def:Strict} application can be used to force top-level evaluation of any combination of arguments.
A curried function application of \haskellinline{f x y} has 3 possible different behaviors with \texttt{\$!}:
\begin{enumerate}[noitemsep]
\item \haskellinline{(f $! x) y}: Force top-level evaluation of \texttt{x}.
\item \haskellinline{(f x) $! y}: Force top-level evaluation of \texttt{y}.
\item \haskellinline{(f $! x) $! y}: Force top-level evaluation of \texttt{x} and \texttt{y}.
\end{enumerate}

\nameref{def:Strict} application is mainly used to improve the space performance of programs.
Take the code in \Cref{lst:Lazy_vs_Strict_Function_Application} as an example.

\begin{listing}[h!tbp]
\haskellsourcefile{./EDAN40-Functional_Programming-Sections/Lazy_Evaluation/Code/Strict_Function_Application.hs}
\caption{Lazy vs. Strict Function Application}
\label{lst:Lazy_vs_Strict_Function_Application}
\end{listing}


%%% Local Variables:
%%% mode: latex
%%% TeX-master: "../../EDAN40-Functional_Programming-Reference_Sheet"
%%% End:
