\subsection{Modular Programming}\label{subsec:Modular_Programming}
\nameref{def:Lazy_Evaluation} also allows us to separate control from data in our computations.
For example, a list of three ones can be produced by selecting the first three elements (control) of the infinite list of ones (data):
\begin{haskellsource}
> take 3 ones
[1, 1, 1]
\end{haskellsource}

The data is only evaluated as much as required by the control, and these two take it in turn to perform reductions.
Without \nameref{def:Lazy_Evaluation}, the control and data parts would need to be combined in the form of a single function.
Allowing us to modularize by separating them into logically distinct parts is an important goal in programming.
Being able to separate control from data is one of the most important benefits of lazy evaluation.


%%% Local Variables:
%%% mode: latex
%%% TeX-master: "../../EDAN40-Functional_Programming-Reference_Sheet"
%%% End:
