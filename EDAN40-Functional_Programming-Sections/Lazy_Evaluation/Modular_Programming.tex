\subsection{Modular Programming}\label{subsec:Modular_Programming}
\nameref{def:Lazy_Evaluation} also allows us to separate control from data in our computations.
For example, a list of three ones can be produced by selecting the first three elements (control) of the infinite list of ones (data):
\begin{haskellsource}
> take 3 ones
[1, 1, 1]
\end{haskellsource}

The data is only evaluated as much as required by the control, and these two take it in turn to perform reductions.
Without \nameref{def:Lazy_Evaluation}, the control and data parts would need to be combined in the form of a single function.
Allowing us to modularize by separating them into logically distinct parts is an important goal in programming.
Being able to separate control from data is one of the most important benefits of lazy evaluation.

A good example of this is generating \textbf{ALL} the prime numbers, an inherently infinite task.
Using the Eratosthenes Sieve method:
\begin{enumerate}[noitemsep]
\item Write down the infinite sequence $2, 3, 4, 5, 6, \ldots$
\item Mark the first number, $p$, in the sequence as prime
\item Delete all multiples of p from the sequence
\item Return to the second step
\end{enumerate}

\begin{listing}[h!tbp]
\haskellsourcefile{./EDAN40-Functional_Programming-Sections/Lazy_Evaluation/Code/Eratosthenes_Primes.hs}
\caption{Eratosthenes Primes Algorithm}
\label{lst:Eratosthenes_Primes}
\end{listing}

Lazy evaluation ensures that this program does indeed produce the infinite list of all prime numbers.
\begin{enumerate}[noitemsep]
\item Start with the infinite list \texttt{[2..]}.
\item Apply the \texttt{sieve} function that retains the first number \texttt{p} as being prime.
\item Filter all multiples of \texttt{p} from this list.
\item \texttt{sieve} calls itself recursively with the list it just obtained (with all multiples of \texttt{p} filtered out).
\end{enumerate}

If we simply ask for the value of the \texttt{primes} expression, we get an infinite list of primes back.
By freeing the generation of prime numbers from the constraint of finiteness, we have obtained a modular program on which different control parts can be used in different situations.

\subsubsection{Caveats with Modular Programming}\label{subsubsec:Modular_Programming_Caveats}
Even though we can modularize our code between control and data, we need to be careful when the data is an infinite structure (list or recursion).
In \Cref{lst:Infinite_List_Gotchas}, this is described more.

\begin{listing}[h!tbp]
\haskellsourcefile{./EDAN40-Functional_Programming-Sections/Lazy_Evaluation/Code/Infinite_List_Gotchas.hs}
\caption{Infinite List Gotchas}
\label{lst:Infinite_List_Gotchas}
\end{listing}

%%% Local Variables:
%%% mode: latex
%%% TeX-master: "../../EDAN40-Functional_Programming-Reference_Sheet"
%%% End:
