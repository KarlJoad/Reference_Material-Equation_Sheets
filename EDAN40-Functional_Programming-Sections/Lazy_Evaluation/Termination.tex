\subsection{Termination}\label{subsec:Termination}
Throughout this section, we will use a recursive definition that yields a value similar to infinity.
\haskellsourcefile{./EDAN40-Functional_Programming-Sections/Lazy_Evaluation/Code/infty.hs}

In practice, trying to evaluate \texttt{infty} will quickly exhaust the available memory and produce an error message.
Consider the expression \haskellinline{fst (0, inf)}, where \texttt{fst} is the library function that selects the first component of a pair, defined by \haskellinline{fst (x, y) = x}.

Call-by-value evaluation with this expression also results in non-termination.
In contrast, call-by-name evaluation terminates in just one step, by immediately applying the definition of \texttt{fst} avoiding the evaluation of the non-terminating expression.

This shows that call-by-name evaluation may produce a result when call-by-value evaluation fails to terminate.
In general, we have the following important property: if there exists any evaluation sequence that terminates for a given expression, then call-by-name evaluation will also terminate for this expression, and produce the same final result.

%%% Local Variables:
%%% mode: latex
%%% TeX-master: "../../EDAN40-Functional_Programming-Reference_Sheet"
%%% End:
