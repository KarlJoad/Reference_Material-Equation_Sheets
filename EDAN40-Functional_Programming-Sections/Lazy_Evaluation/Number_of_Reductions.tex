\subsection{Number of Reductions}\label{subsec:Number_of_Reductions}
Call-by-name evaluation may require more steps than call-by-value evaluation, in particular when an argument is used more than once in the body of a function.
Generally, we have the following property: arguments are evaluated precisely once using call-by-value evaluation, but may be evaluated many times using call-by-name.

In this section, we will use the definition of \texttt{square} below.
\begin{listing}[h!tbp]
\haskellsourcefile{./EDAN40-Functional_Programming-Sections/Lazy_Evaluation/Code/square.hs}
\caption{Definition of Squaring an Integer for \Cref{subsec:Termination}}
\label{lst:Square_Definition}
\end{listing}

Below, we evaluate the expression \haskellinline{square (1+2)} according to the definition of \texttt{square} in \Cref{lst:Square_Definition}.
Each line of derivation performs one reduction.

Using call-by-value evaluation, we have:
\begin{align*}
  &= square (1+2) \\
  &= square 3 \\
  &= 3 * 3 \\
  &= 9
\end{align*}

Using call-by-name evaluation, we have:
\begin{align*}
  &= square (1+2) \\
  &= (1+2) * (1+2) \\
  &= 3 * (1+2) \\
  &= 3 * 3 \\
  &= 9
\end{align*}

Thus, call-by-name requires one extra reduction step, because \texttt{1+2} must be evaluated twice.
Fortunately, this problem can be solved easily by using pointers to indicate sharing of expressions during evaluation.
Rather than physically copying an argument if it is used many times in the body of a function, we simply keep one copy of the argument and make many pointers to it.
In this manner, any reductions that are performed on the argument are automatically shared between each of the pointers to that argument.
This works because the evaluation of an \nameref{def:Expression} depends only on the expression itself, not the state of the program.

The use of call-by-name evaluation in conjunction with sharing is called \nameref{def:Lazy_Evaluation}, making Haskell a lazy programming language.

%%% Local Variables:
%%% mode: latex
%%% TeX-master: "../../EDAN40-Functional_Programming-Reference_Sheet"
%%% End:
