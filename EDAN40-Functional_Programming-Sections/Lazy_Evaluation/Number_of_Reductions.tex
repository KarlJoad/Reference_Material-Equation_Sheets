\subsection{Number of Reductions}\label{subsec:Number_of_Reductions}
Call-by-name evaluation may require more steps than call-by-value evaluation, in particular when an argument is used more than once in the body of a function.
Generally, we have the following property: arguments are evaluated precisely once using call-by-value evaluation, but may be evaluated many times using call-by-name.

In this section, we will use the definition of \texttt{square} below.
\begin{listing}[h!tbp]
\haskellsourcefile{./EDAN40-Functional_Programming-Sections/Lazy_Evaluation/Code/square.hs}
\caption{Definition of Squaring an Integer for \Cref{subsec:Termination}}
\label{lst:Square_Definition}
\end{listing}

Below, we evaluate the expression \haskellinline{square (1+2)} according to the definition of \texttt{square} in \Cref{lst:Square_Definition}.
Each line of derivation performs one reduction.

Using call-by-value evaluation, we have:
\begin{align*}
  &= square (1+2) \\
  &= square 3 \\
  &= 3 * 3 \\
  &= 9
\end{align*}

Using call-by-name evaluation, we have:
\begin{align*}
  &= square (1+2) \\
  &= (1+2) * (1+2) \\
  &= 3 * (1+2) \\
  &= 3 * 3 \\
  &= 9
\end{align*}


%%% Local Variables:
%%% mode: latex
%%% TeX-master: "../../EDAN40-Functional_Programming-Reference_Sheet"
%%% End:
