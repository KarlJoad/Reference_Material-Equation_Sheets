\subsection{Recursive Types}\label{subsec:Recursive_Types}
Some \nameref{def:Type}s lend themselves to being self-recursive.
As we saw earlier, we cannot use the \haskellinline{type} keyword to construct these.
Instead, we must use the \haskellinline{data} keyword to build new \nameref{def:Type}s that \textbf{are} self-recursive.

2 examples are shown below.
\begin{enumerate}[noitemsep]
\item \nameref{lst:Tree_Recursive_Type}, \Cref{lst:Tree_Recursive_Type}
\item \nameref{lst:List_Recursive_Type}, \Cref{lst:List_Recursive_Type}
\end{enumerate}

\begin{listing}[h!tbp]
\haskellsourcefile{./EDAN40-Functional_Programming-Sections/Declaring_Types/Code/Recursive_Type-Tree.hs}
\caption{Tree Recursive Type}
\label{lst:Tree_Recursive_Type}
\end{listing}

\begin{listing}[h!tbp]
\haskellsourcefile{./EDAN40-Functional_Programming-Sections/Declaring_Types/Code/Recursive_Type-List.hs}
\caption{List Recursive Type}
\label{lst:List_Recursive_Type}
\end{listing}

%%% Local Variables:
%%% mode: latex
%%% TeX-master: "../../EDAN40-Functional_Programming-Reference_Sheet"
%%% End:
