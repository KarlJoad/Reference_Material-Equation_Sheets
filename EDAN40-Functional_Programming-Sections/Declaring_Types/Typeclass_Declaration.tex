\subsection{Typeclass Declaration}\label{subsec:Typeclass_Declaration}
In Haskell, a new \nameref{def:Typeclass} can be defined with the \haskellinline{class} keyword.
For example, the standard library defines the \haskellinline{Eq} typeclass.

\begin{listing}[h!tbp]
\haskellsourcefile{./EDAN40-Functional_Programming-Sections/Declaring_Types/Code/Eq_Typeclass-Definition.hs}
\caption{Haskell's Standard Library Implementation of \haskellinline{Eq} Typeclass}
\label{lst:Eq_Typeclass_Definition}
\end{listing}

This states that for any \nameref{def:Type} \texttt{a} to be part of the \texttt{Eq} \nameref{def:Typeclass}, it must support the equality and inequality operators/functions.
Note that because of the way that inequality is defined, with its reliance on equality, only equality needs to be defined.

%%% Local Variables:
%%% mode: latex
%%% TeX-master: "../../EDAN40-Functional_Programming-Reference_Sheet"
%%% End:
