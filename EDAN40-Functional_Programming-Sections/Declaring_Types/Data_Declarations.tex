\subsection{Data Declarations}\label{subsec:Data_Declarations}
To declare a completely new \nameref{def:Type} with its own typechecking rules is done with the \haskellinline{data} keyword.
For example, the \haskellinline{Bool} \nameref{def:Type} is defined in the standard library as
\begin{haskellsource}
data Bool = False | True
\end{haskellsource}

The \texttt{|} symbol is read as ``or''.
Each of the new values is a \nameref{def:Type_Constructor}, \texttt{False} and \texttt{True} in this case.
The first letter of these constructors must be a capital letter, and constructor names must be unique to modules.
It is important to note that the constructor values, names, and the data type's name are all arbitrary.
The only way to attach any meaning to one of these constructor types is by defining functions on them.
For example, the below code is functionally identical to the definition of \haskellinline{Bool}.
\begin{haskellsource}
data A = B | C
\end{haskellsource}

\begin{definition}[Type Constructor]\label{def:Type_Constructor}
  A \emph{type constructor} is a function that constructs a new type in the compiler/runtime system.
\end{definition}


%%% Local Variables:
%%% mode: latex
%%% TeX-master: "../../EDAN40-Functional_Programming-Reference_Sheet"
%%% End:
