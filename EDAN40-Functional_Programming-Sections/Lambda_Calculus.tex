\section{Lambda Calculus}\label{sec:Lambda_Calculus}
\nameref{def:Lambda_Calculus} is the basis of functional programming and all the languages that express it.
The first programming language to use \nameref{def:Lambda_Calculus}, and thus be a \nameref{def:Functional_Programming_Language}, was \textsc{lisp} in the 1960s.

\begin{definition}[Lambda Calculus]\label{def:Lambda_Calculus}
  \emph{Lambda Calculus} (\emph{$\lambda$-calculus}) is a formal system in mathematical logic for expressing computation based on function abstraction and application using variable binding and substitution.
  It is a universal model of computation that can be used to simulate any Turing machine.
  It was introduced by the mathematician Alonzo Church in the 1930s as part of his research into the foundations of mathematics.
\end{definition}

An expression in \nameref{def:Lambda_Calculus} (not to be confused with \nameref{def:Lambda_Expression}s) is defined as follows:
\begin{equation}\label{eq:Basic_Lambda_Function}
  \lambda x. E(x)
\end{equation}

\Cref{eq:Basic_Lambda_Function} denotes a function that, given an input $x$, computes $E(x)$.
To apply this function, one substitutes the given input for the variable and evaluates the function body.

For example, the below function is equivalent to \haskellinline{succ} in Haskell.

\begin{equation}\label{eq:Succ_Lambda_Function}
  \lambda x.(x+1)
\end{equation}

To apply \Cref{eq:Succ_Lambda_Function}, for example to 7, one performs substitution of the lambda's parameter.

\begin{align*}
(\lambda x.(x+1))7 &= (7+1) && \text{Replace all occurrences of $x$ in $\lambda x$ with 7.} \\
                   &= 8 && \text{Apply the summation operation, which we assume to be primitive.} \\
\end{align*}

\begin{remark*}
  The definition of lambda functions in $\lambda$-calculus makes \textbf{ALL} functions curried.
\end{remark*}

This is well illustrated in \Cref{eq:Higher_Order_Lambda_Function}.
\begin{equation}\label{eq:Higher_Order_Lambda_Function}
  \lambda f. \lambda g. \lambda x. f \bigl( g(x) \bigr)
\end{equation}

If we want to use this function in \Cref{eq:Higher_Order_Lambda_Function}, by applying \Cref{eq:Succ_Lambda_Function}, we can.
This process is shown below.
\begin{align*}
  \intertext{$y$ and $z$ are identical to $x$ from their parent equation. They have been changed for the reasons of legibility.}
  \Bigl( \lambda f. \lambda g. \lambda x. f \bigl( g(x) \bigr) \Bigr) \bigl( \lambda y. (y+1) \bigr) \bigl( \lambda z. (z+1) \bigr) &= \Biggl( \lambda g. \lambda x. \biggl( \Bigl( \lambda y. (y+1) \Bigr) \Bigr(g (x) \Bigr) \biggr) \Biggr) \bigl( \lambda z. (z+1) \bigr) \\
  \shortintertext{Replace all instances of $f$ with the $\lambda y. (y+1)$ function.}
                                                                                                                                    &= \lambda x. \biggl( \bigl( \lambda y. (y+1) \bigr) \Bigl( \bigl( \lambda z. (z+1) \bigr) x \Bigr) \biggr) \\
  \intertext{Note that we \textbf{ARE NOT} told the order of application of the functions.}
  \shortintertext{Apply $x$ to the $\lambda z. (z+1)$ function.}
                                                                                                                                    &= \lambda x. \Bigl( \bigl( \lambda y. (y+1) \bigr) (x+1) \Bigr) \\
  \shortintertext{Apply $x+1$ to $\lambda y. (y+1)$.}
                                                                                                                                    &= \lambda x. \bigl( (x+1) + 1 \bigr) \\
\end{align*}

There are several variations of \nameref{def:Lambda_Calculus}, one of which is \emph{Typed Lambda Calculus}.
However, we are only concerned with \nameref{subsec:Pure_Lambda_Calculus}.

\subsection{Pure Lambda Calculus}\label{subsec:Pure_Lambda_Calculus}.
In pure lambda calculus, there are only 3 things, called \nameref{tab:Lambda_Terms}.

\begin{table}[h!tbp]
  \centering
  \begin{tabular}{ccp{15cm}}
    \toprule
    Syntax & Name & \multicolumn{1}{c}{Description} \\
    \midrule
    $x$ & Variable & A character or string representing a parameter or mathematical/logical value. \\
$ (\lambda x. M)$ & $\lambda$-Abstraction & Function definition ($M$ is a lambda term). The variable $x$ becomes bound in the expression. \\
    $(M N)$ & $\lambda$-Application & Applying a function to an argument. M and N are lambda terms. \\
    \bottomrule
  \end{tabular}
\caption{$\lambda$-Terms}
\label{tab:Lambda_Terms}
\end{table}


%%% Local Variables:
%%% mode: latex
%%% TeX-master: "../../EDAN40-Functional_Programming-Reference_Sheet"
%%% End:


\subsection{Using Lambda Calculus}\label{subsec:Using_Lambda_Calculus}
When using \nameref{def:Lambda_Calculus}, there are some things to keep in mind.
\begin{enumerate}[noitemsep]
\item Reductions involve reducing or manipulating an expression somehow.
  There are 3 kinds of reductions, the first 2 being the most common.
  \begin{enumerate}[noitemsep]
  \item \nameref{def:Alpha_Conversion}
  \item \nameref{def:Beta_Reduction}
  \item \nameref{def:Eta_Reduction}
  \end{enumerate}
\item Computations in $\lambda$ calculus is performed by performing successive $\beta$-reductions whenever possible and for as long as possible.
\end{enumerate}

\begin{definition}[Alpha Conversion]\label{def:Alpha_Conversion}
  \emph{Alpha conversion} (\emph{$\alpha$-conversion}) is the process of replacing one variable name with another.
  For example, in the expression, $\lambda x. zx$, an $\alpha$-conversion of $x$ to $y$ results in $\lambda y. zy$.
\end{definition}

\begin{definition}[Beta Reduction]\label{def:Beta_Reduction}
  \emph{Beta reduction} (\emph{$\beta$-reduction}) is the process of following substitution rules defined by the $\lambda$ functions.
  For example, performing a $\beta$-reduction on $(\lambda n. n \times 2) 7$ results in the expression $7 \times 2$.
\end{definition}

\begin{definition}[Eta Reduction]\label{def:Eta_Reduction}
  \emph{Eta reduction} (\emph{$\eta$-reduction}) is a reduction of a $\lambda$ function to a single expression when the variable does not appear free.
  In practice, an $\eta$-reduction is the process of converting a regular function to Point-free form.
  For example, an $\eta$-reduction on $\lambda x. f x$ would yield $f$.
\end{definition}

There is a key theorem, the \nameref{thm:Church_Rosser_Theorem} that discusses the order of computation in \nameref{def:Lambda_Calculus}, namely that the order of computation does not matter.

\begin{theorem}[Church-Rosser Theorem]\label{thm:Church_Rosser_Theorem}
  The order of reductions does not matter, as there will always be some final reduction that is common to any path taken.
\end{theorem}

Discussed earlier, and now here, is the concept of \nameref{def:Normal_Form}.
Haskell tends to use \nameref{def:Weak_Head_Normal_Form}, but \nameref{def:Lambda_Calculus} only uses \nameref{def:Normal_Form}.
\begin{definition}[Normal Form]\label{def:Normal_Form}
  \emph{Normal form} is an expression that can have no more \nameref{def:Beta_Reduction} apply.
  This is equivalent to the halting state of a Turing Machine.
\end{definition}

However, not all terms have a \nameref{def:Normal_Form}.
These expressions correspond to non-halting computations in a Turing Machine.
An example is shown in \Cref{eq:No_Normal_Form}.
\begin{equation}\label{eq:No_Normal_Form}
  (\lambda x. x x) (\lambda x. x x)
\end{equation}

\Cref{eq:No_Normal_Form} cannot be reduced to a \nameref{def:Normal_Form} as the following derivation shows.
I have performed an \nameref{def:Alpha_Conversion} on the second lambda function, to make things clearer.
I went from $x$ to $y$.
\begin{align*}
  \intertext{Apply the $\lambda y$ expression as the variable $x$ in the $\lambda x$ expression.}
  (\lambda x. x x) (\lambda y. y y) &= (\lambda y. y y) (\lambda y. y y)
\end{align*}

%%% Local Variables:
%%% mode: latex
%%% TeX-master: "../../EDAN40-Functional_Programming-Reference_Sheet"
%%% End:


%%% Local Variables:
%%% mode: latex
%%% TeX-master: "../EDAN40-Functional_Programming-Reference_Sheet"
%%% End:
