\section{Lambda Calculus}\label{sec:Lambda_Calculus}
\nameref{def:Lambda_Calculus} is the basis of functional programming and all the languages that express it.
The first programming language to use \nameref{def:Lambda_Calculus}, and thus be a \nameref{def:Functional_Programming_Language}, was \textsc{lisp} in the 1960s.

\begin{definition}[Lambda Calculus]\label{def:Lambda_Calculus}
  \emph{Lambda Calculus} (\emph{$\lambda$-calculus}) is a formal system in mathematical logic for expressing computation based on function abstraction and application using variable binding and substitution.
  It is a universal model of computation that can be used to simulate any Turing machine.
  It was introduced by the mathematician Alonzo Church in the 1930s as part of his research into the foundations of mathematics.
\end{definition}

An expression in \nameref{def:Lambda_Calculus} (not to be confused with \nameref{def:Lambda_Expression}s) is defined as follows:
\begin{equation}\label{eq:Basic_Lambda_Function}
  \lambda x. E(x)
\end{equation}

\Cref{eq:Basic_Lambda_Function} denotes a function that, given an input $x$, computes $E(x)$.
To apply this function, one substitutes the given input for the variable and evaluates the function body.

For example, the below function is equivalent to \haskellinline{succ} in Haskell.

\begin{equation}\label{eq:Succ_Lambda_Function}
  \lambda x.(x+1)
\end{equation}

To apply \Cref{eq:Succ_Lambda_Function}, for example to 7, one performs substitution of the lambda's parameter.

\begin{align*}
(\lambda x.(x+1))7 &= (7+1) && \text{Replace all occurrences of $x$ in $\lambda x$ with 7.} \\
                   &= 8 && \text{Apply the summation operation, which we assume to be primitive.} \\
\end{align*}

%%% Local Variables:
%%% mode: latex
%%% TeX-master: "../EDAN40-Functional_Programming-Reference_Sheet"
%%% End:
