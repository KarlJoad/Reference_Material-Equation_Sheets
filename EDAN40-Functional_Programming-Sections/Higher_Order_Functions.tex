\section{Higher-Order Functions}\label{sec:Higher_Order_Functions}
Higher-order functions allow common programming patters to be encapsulated as functions.
This is why these are called functional programming languages, because their common abstraction/encapsulation pattern is done with functions.

\begin{definition}[Higher-Order Function]\label{def:Higher_Order_Function}
  A function that takes a function as an argument \textbf{or} returns a function as a result is a \emph{higher-order function}.
\end{definition}

In Haskell, functions with multiple arguments are usually defined using the notion of currying.
That is, the arguments are taken one at a time by exploiting the fact that functions can return functions as results.
In addition, it is also permissible to define functions that take functions as arguments.

\begin{remark*}
  It was an important breakthrough to figure out that tradition functions that use tuples to pass in their arguments are equivalent to their curried counterparts.
\end{remark*}

Because of this unique property in Haskell (Haskell is not the only language that allows this), functions can be partially applied.

\begin{definition}[Partial Application]\label{def:Partial_Application}
  \emph{Partial application} is when a curried function receives fewer parameters than it needs to fully compute a value.
  By only providing $m=n-x$ arguments (where $n > x$), the called function returns a function.
  This returned function can then have its last $m$ parameters applied to compute something.

  \begin{remark}[Partially Applied]\label{rmk:Partially_Applied}
    Functions that have had \nameref{def:Partial_Application} used on them are said to be \emph{partially applied}.
  \end{remark}
\end{definition}

The benefit of \nameref{def:Higher_Order_Function}s are:
\begin{enumerate}[noitemsep]
\item Common programming idioms from other languages can be encoded using functions in the language itself.
\item Domain-specific languages can be defined as collections of \nameref{def:Higher_Order_Function}s.
\item Algebraic properties of \nameref{def:Higher_Order_Function}s can be used to reason about them and programs as a whole.
\end{enumerate}

\subsection{List Processing}\label{subsec:List_Processing}

%%% Local Variables:
%%% mode: latex
%%% TeX-master: "../../EDAN40-Functional_Programming-Reference_Sheet"
%%% End:


\subsection{The \texorpdfstring{\haskellinline{foldr}}{\texttt{foldr}} Function}\label{subsec:Foldr_Function}

%%% Local Variables:
%%% mode: latex
%%% TeX-master: "../../EDAN40-Functional_Programming-Reference_Sheet"
%%% End:


\subsection{The \texorpdfstring{\haskellinline{foldl}}{\texttt{foldl}} Function}\label{subsec:Foldl_Function}
In addition to being able to define operations that are assumed to be right-associative, we can write functions that are assumed to be left-associative with the \haskellinline{foldl} function.
The general basis for which the \haskellinline{foldl} function was written is shown in \Cref{lst:Foldl_Basis}.

\begin{listing}[h!tbp]
\begin{haskellsource}
f v [] = v
f v (x:xs) = f (v ? x) xs
-- Where ? is some operator
\end{haskellsource}
\caption{The Basis of Defining the \haskellinline{foldl} Function}
\label{lst:Foldl_Basis}
\end{listing}

The function maps the empty list to the accumulator value v.
Any non-empty list is mapped to the result of recursively processing the tail using a new accumulator value obtained by applying an operator \texttt{?} to the current value and the head of the list.

Many of the operations that \nameref{subsec:Foldr_Function} defines can also be defined using \haskellinline{foldl}.
When this is the case, the only way to choose which to use is based on the efficiency of the function itself.
This is dependent on Haskell's underlying mechanisms, which is discussed later.
For now, both of them work equally well.

In practice, just as with \haskellinline{foldr} it is best to think of the behaviour of \haskellinline{foldl} in a non-recursive manner.
It is better to think of \haskellinline{foldl}, in terms of an operator $\oplus$ (Was \texttt{?} earlier. The actual symbol doesn't matter.) that is assumed to associate to the left, as summarised by \Cref{eq:General_foldl_Operation}.

\begin{equation}\label{eq:General_foldl_Operation}
  foldl (\oplus) v [x_{0}, x_{1}, x_{2}, \ldots, x_{n}] = (\cdots ((v \oplus x_{0}) \oplus x_{1}) \cdots) \oplus x_{n}
\end{equation}

%%% Local Variables:
%%% mode: latex
%%% TeX-master: "../../EDAN40-Functional_Programming-Reference_Sheet"
%%% End:


\subsection{Composition Operator}\label{subsec:Composition_Operator}

%%% Local Variables:
%%% mode: latex
%%% TeX-master: "../../EDAN40-Functional_Programming-Reference_Sheet"
%%% End:


%%% Local Variables:
%%% mode: latex
%%% TeX-master: "../EDAN40-Functional_Programming-Reference_Sheet"
%%% End:
