\subsection{Rewrite Semantics}\label{subsec:Rewrite_Semantics}
One of the key strengths of \nameref{def:Functional_Programming_Language}s is the fact we can easily perform \nameref{def:Rewrite_Semantics} on any given \nameref{def:Expression}.

\begin{definition}[Rewrite Semantics]\label{def:Rewrite_Semantics}
  \emph{Rewrite semantics} is the process of rewriting and deconstructing an \nameref{def:Expression} into its constiuent parts.
  Rewrite semantics answers the question ``How do we extract values from functions?''
\end{definition}

\begin{listing}[h!tbp]
\haskellsourcefile{./EDAN40-Functional_Programming-Sections/Introduction/Code/Rewrite_Factorial.hs}
\caption{\nameref{def:Rewrite_Semantics} of a Factorial Function}
\label{lst:Rewrite_Semantics}
\end{listing}

\begin{definition}[Expression]\label{def:Expression}
  An \emph{expression} is a combination of one or more \nameref{def:Operand}s and operators that the programming language interprets (according to its particular rules of precedence and of association) and computes to produce another value.

  \begin{remark}[Overloading]\label{rmk:Overloading_Expressions}
    An expression can be overloaded if there is more than one definition for an operator.
  \end{remark}
\end{definition}

\begin{definition}[Operand]\label{def:Operand}
  An \emph{operand} is a:
  \begin{itemize}[noitemsep]
  \item Constant
  \item Variable
  \item Another \nameref{def:Expression}
  \item Result from function calls
  \end{itemize}
\end{definition}

%%% Local Variables:
%%% mode: latex
%%% TeX-master: "../../EDAN40-Functional_Programming-Reference_Sheet"
%%% End:
