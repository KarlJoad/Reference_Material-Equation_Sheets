\subsection{Language Basics}\label{subsec:Lang_Basics}
All of the functions and operations presented below come from \textbf{The Standard Prelude}.
The library file \emph{Prelude.hs} is loaded first by the REPL (Read, Evaluate, Print, Loop) environment that we will use.
It defines:
\begin{itemize}[noitemsep]
\item \nameref{subsubsec:Math_Ops}
\item \nameref{subsubsec:List_Ops}
\item And other conveniences for writing Haskell.
\end{itemize}

\subsubsection{Mathematical Operations}\label{subsubsec:Math_Ops}
\emph{Prelude.hs} defines the basic mathematical \textbf{integer} functions of:
\begin{itemize}[noitemsep]
\item Addition
\item Subtraction
\item Multiplication
\item Division
\item Exponentiation
\end{itemize}
\begin{listing}[h!tbp]
\begin{minted}{haskellsource}
> 2+3
5
>2-3
-1
> 2*3
6
>7 `div` 2
3
> 2^3
8
\end{minted}
\caption{Integer Mathematical Operations}
\label{lst:Int_Math_Ops}
\end{listing}

\paragraph{Precedences}\label{par:Math_Precedences}
Just like in normal mathematics, there exists a precedence to disambiguate mathematical expressions containing multiple, different operations.
In order of highest-to-lowest precedence:
\begin{enumerate}[noitemsep]
\item Negation
\item Exponentiation
\item Multiplication and Division
\item Addition and Subtraction
\end{enumerate}

\paragraph{Associativity}\label{par:Math_Associativity}
Just like in normal mathematics, there are rules associativity rules to disambiguate mathematical expressions containing multiple of the same operations.
There are only 2 types of associativity, left and right.
\begin{enumerate}[noitemsep]
\item Left Associative:
  \begin{itemize}[noitemsep]
  \item Everything else.
  \item Addition. $2+3+4 = (2+3)+4$
  \item Subtraction. $2-3-4 = (2-3)-4$
  \item Multiplication. $2*3*4 = (2*3)*4$
  \item Division. $2 \div 3 \div 4 = (2 \div 3) \div 4$
  \end{itemize}
\item Right Associative:
  \begin{itemize}[noitemsep]
  \item Exponentiation. $2^{3^{4}} = 2^{(3^{4})}$
  \item Negation. $--2 = -(-2)$
  \end{itemize}
\end{enumerate}

\begin{remark*}[Types of Associativity]
  Technically, there are 3 types of associativity.
  \begin{enumerate}[noitemsep]
  \item Left-Associative
  \item Right-Associative
  \item Non-Associative
  \end{enumerate}

  Non-associativity means that it does not have an implicit associativity rule associated with it.
  It could also mean it is neither left-, nor right-associative.
\end{remark*}

\subsubsection{List Operations}\label{subsubsec:List_Ops}

\subsubsection{Function Application}\label{subsubsec:Function_Application}

\subsubsection{Haskell Files/Scripts}\label{subsubsec:Haskell_Scripts}


%%% Local Variables:
%%% mode: latex
%%% TeX-master: "../../EDAN40-Functional_Programming-Reference_Sheet"
%%% End:
