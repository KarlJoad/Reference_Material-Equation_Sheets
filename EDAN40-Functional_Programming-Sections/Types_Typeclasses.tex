\section{Types and Typeclasses}\label{sec:Types_Typeclasses}
\begin{definition}[Type]\label{def:Type}
  A \emph{type} is a collection of related values.
  For example, the \haskellinline{Bool} type contains 2 values, \haskellinline{True} and \haskellinline{False}.
  The definition of a type limits the types and amount of values that an expression of that type can take.
\end{definition}

In Haskell, to denote the \nameref{def:Type} of an expression, you use the \texttt{::} symbol.
Namely, \haskellinline{e :: T} meaning that the \nameref{def:Expression} \texttt{e} ``has type'' \texttt{T}, or \texttt{e} ``is of type'' \texttt{T}.

\begin{blackbox}
  {\large{\textbf{EVERY \nameref{def:Expression} MUST HAVE A \nameref{def:Type}!}}}
\end{blackbox}

To ensure that every \nameref{def:Expression} has a \nameref{def:Type} before execution, Haskell (like most other functional languages) uses \nameref{def:Type_Inferencing} to attempt to determine the type of every expression.

\begin{definition}[Type Inferencing]\label{def:Type_Inferencing}
  \emph{Type inferencing} is the process of solving the system of type equations introduced by expressions and operators/functions used on them to determine the \nameref{def:Type} of every \nameref{def:Expression} in a program.

  When an operator or function is used, it has a type signature.
  This indicates what \nameref{def:Type}s it takes in and what type(s) it returns.
  If there are operators/functions with pre-determined type signatures, like addition, then the types of the expressions that are fed into the addition operator need to be something that can be added.
  After the language system has gone through the whole file/program, a system of equations for the type of every expression can be created.
  By solving this system of equations, the \nameref{def:Type} of every expression can be determined.

  \begin{remark}[Done Statically]\label{rmk:Type_Inferencing_Static}
    It is important to note that \nameref{def:Type_Inferencing} is done \textbf{statically}, meaning it is done before the program is even executed.
    When compiled, the \nameref{def:Type}s are determined during compilation.
    When loaded into the REPL, the \nameref{def:Type}s are determined when the file is parsed and interpreted.
  \end{remark}

  \begin{remark}[Doesn't Catch Everything]
    \nameref{def:Type_Inferencing} does \textbf{NOT} catch all possible errors.
    Namely, it does not catch runtime errors, such as division-by-zero.
    However, it does remove a large category of errors that present themselves due to inappropriate typing.
  \end{remark}
\end{definition}

Because \nameref{def:Type_Inferencing} is done statically, Haskell programs are both \emph{strongly typed} and \emph{type safe}.
Meaning all \nameref{def:Type} errors that could occur in a program are found before program execution.
The use of \nameref{def:Type_Inferencing} is the reason that \textbf{all execution paths must have the same type}.

\begin{remark*}
  The determination of \nameref{def:Type}s in a program using \nameref{def:Type_Inferencing} where selection statements have different \nameref{def:Type}s is an undecidable problem.
  Thus, the language designers enforced programmers to write all \nameref{def:Conditional_Expression}, \nameref{def:Guarded_Equation}s and pattern matching paths to have the same \nameref{def:Type}.
\end{remark*}

\subsection{Basic Types}\label{subsec:Basic_Types}
There are a variety of basic types built into the Haskell language.

\subsubsection{\texorpdfstring{\haskellinline{Bool}}{\texttt{Bool}}}\label{subsubsec:Bool_Type}
The \haskellinline{Bool} \nameref{def:Type} contains the two logical values, \haskellinline{True} and \haskellinline{False}.

\subsubsection{\texorpdfstring{\haskellinline{Char}}{\texttt{Char}}}\label{subsubsec:Char_Type}

\subsubsection{\texorpdfstring{\haskellinline{String}}{\texttt{String}}}\label{subsubsec:String_Type}

\subsubsection{\texorpdfstring{\haskellinline{Int}}{\texttt{Int}}}\label{subsubsec:Int_Type}

\subsubsection{\texorpdfstring{\haskellinline{Integer}}{\texttt{Integer}}}\label{subsubsec:Integer_Type}

\subsubsection{\texorpdfstring{\haskellinline{Float}}{\texttt{Float}}}\label{subsubsec:Float_Type}

%%% Local Variables:
%%% mode: latex
%%% TeX-master: "../../EDAN40-Functional_Programming-Reference_Sheet"
%%% End:


\subsection{List Types}\label{subsec:List_Types}
\begin{definition}[List]\label{def:List}
  In Haskell, a \emph{list} is a sequence of elements that are \textbf{all the same \nameref{def:Type}}.
  The elements are enclosed within brackets, \texttt{[} and \texttt{]}, with each element separated by commas.

  \Cref{lst:List_Examples} shows examples of lists and their synatax.
\end{definition}


%%% Local Variables:
%%% mode: latex
%%% TeX-master: "../../EDAN40-Functional_Programming-Reference_Sheet"
%%% End:


\subsection{Tuple Types}\label{subsec:Tuple_Types}
\begin{definition}[Tuple]\label{def:Tuple}
  A \emph{tuple} is a \textbf{finite} sequence of components/elements that may have different types.
  The elements are enclosed in parentheses and separated by commas.

  Some examples of tuples are shown in \Cref{lst:Tuple_Examples}.
\end{definition}

\begin{listing}[h!tbp]
\haskellsourcefile{./EDAN40-Functional_Programming-Sections/Types_Typeclasses/Code/Tuple_Examples.hs}
\caption{Example of Tuples in Haskell}
\label{lst:Tuple_Examples}
\end{listing}

The number of elements in a tuple is the \emph{arity} of the tuple.
The empty tuple has an arity of 0.
A tuple of arity 2 is a pair, arity 3 is a triple, and so on.

\begin{blackbox}
  {\large{\textbf{\nameref{def:Tuple}s of arity 1 are NOT allowed!!}}}
\end{blackbox}

There are 3 major things to remember about \nameref{def:Tuple} types:
\begin{enumerate}[noitemsep]
\item The type of a tuple conveys its arity.
\item There are not restirctions on the type of the elements in a tuple, they do not even have to have the same type.
\item Tuples must \textbf{ALWAYS} have a finite arity, to ensure \nameref{def:Type_Inferencing} works.
\end{enumerate}

%%% Local Variables:
%%% mode: latex
%%% TeX-master: "../../EDAN40-Functional_Programming-Reference_Sheet"
%%% End:


\subsection{Function Types}\label{subsec:Function_Types}
\begin{definition}[Function]\label{def:Function}
  A \emph{function} a mapping from arguments of one \nameref{def:Type} to another.
  The type signature of a function is written $\mathtt{T_{1}} -> \mathtt{T_{2}}$.

  Some examples of functions are shown in \Cref{lst:Function_Examples}.

  \begin{remark}[Pureness of Definition]\label{rmk:Function_Definition_Pureness}
    This is the purest definition of a function, drawn right from mathematics.
    This interpretation forms the basis of \nameref{def:Functional_Programming_Language}s.
    This definition also prevents the introduction of side-effects, another basis for the developement of \nameref{def:Functional_Programming_Language}s.
  \end{remark}
\end{definition}

\begin{listing}[h!tbp]
\haskellsourcefile{./EDAN40-Functional_Programming-Sections/Types_Typeclasses/Code/Function_Examples.hs}
\caption{Example of Functions in Haskell}
\label{lst:Function_Examples}
\end{listing}

Type signatures can be provided for a function, although the language system will typically figure out the most generic \nameref{def:Type}s possible for a function because of the \nameref{def:Type_Inferencing} system.
However, including them is good practice for documentation and for helping to write recursive functions.

\begin{blackbox}
  The only way to handle \textbf{MULTIPLE} parameters passed into and returned from a single function is to put them into either a \nameref{def:List} or \nameref{def:Tuple}.
\end{blackbox}

\begin{blackbox}
  There is \textbf{NO RESTRICTION} for a function to be \emph{total} on the arguments provided to a function, namely there may be some arguments for which the function is undefined.
\end{blackbox}

%%% Local Variables:
%%% mode: latex
%%% TeX-master: "../../EDAN40-Functional_Programming-Reference_Sheet"
%%% End:


%%% Local Variables:
%%% mode: latex
%%% TeX-master: "../EDAN40-Functional_Programming-Reference_Sheet"
%%% End:
