\section{Types and Typeclasses}\label{sec:Types_Typeclasses}
\begin{definition}[Type]\label{def:Type}
  A \emph{type} is a collection of related values.
  For example, the \haskellinline{Bool} type contains 2 values, \haskellinline{True} and \haskellinline{False}.
  The definition of a type limits the types and amount of values that an expression of that type can take.
\end{definition}

In Haskell, to denote the \nameref{def:Type} of an expression, you use the \texttt{::} symbol.
Namely, \haskellinline{e :: T} meaning that the \nameref{def:Expression} \texttt{e} ``has type'' \texttt{T}, or \texttt{e} ``is of type'' \texttt{T}.

\begin{blackbox}
  {\large{\textbf{EVERY \nameref{def:Expression} MUST HAVE A \nameref{def:Type}!}}}
\end{blackbox}


%%% Local Variables:
%%% mode: latex
%%% TeX-master: "../EDAN40-Functional_Programming-Reference_Sheet"
%%% End:
