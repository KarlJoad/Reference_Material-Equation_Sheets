\section{Types and Typeclasses}\label{sec:Types_Typeclasses}
\begin{definition}[Type]\label{def:Type}
  A \emph{type} is a collection of related values.
  For example, the \haskellinline{Bool} type contains 2 values, \haskellinline{True} and \haskellinline{False}.
  The definition of a type limits the types and amount of values that an expression of that type can take.
\end{definition}

In Haskell, to denote the \nameref{def:Type} of an expression, you use the \texttt{::} symbol.
Namely, \haskellinline{e :: T} meaning that the \nameref{def:Expression} \texttt{e} ``has type'' \texttt{T}, or \texttt{e} ``is of type'' \texttt{T}.

\begin{blackbox}
  {\large{\textbf{EVERY \nameref{def:Expression} MUST HAVE A \nameref{def:Type}!}}}
\end{blackbox}

To ensure that every \nameref{def:Expression} has a \nameref{def:Type} before execution, Haskell (like most other functional languages) uses \nameref{def:Type_Inferencing} to attempt to determine the type of every expression.

\begin{definition}[Type Inferencing]\label{def:Type_Inferencing}
  \emph{Type inferencing} is the process of solving the system of type equations introduced by expressions and operators/functions used on them to determine the \nameref{def:Type} of every \nameref{def:Expression} in a program.

  When an operator or function is used, it has a type signature.
  This indicates what \nameref{def:Type}s it takes in and what type(s) it returns.
  If there are operators/functions with pre-determined type signatures, like addition, then the types of the expressions that are fed into the addition operator need to be something that can be added.
  After the language system has gone through the whole file/program, a system of equations for the type of every expression can be created.
  By solving this system of equations, the \nameref{def:Type} of every expression can be determined.

  \begin{remark}[Done Statically]\label{rmk:Type_Inferencing_Static}
    It is important to note that \nameref{def:Type_Inferencing} is done \textbf{statically}, meaning it is done before the program is even executed.
    When compiled, the \nameref{def:Type}s are determined during compilation.
    When loaded into the REPL, the \nameref{def:Type}s are determined when the file is parsed and interpreted.
  \end{remark}

  \begin{remark}[Doesn't Catch Everything]
    \nameref{def:Type_Inferencing} does \textbf{NOT} catch all possible errors.
    Namely, it does not catch runtime errors, such as division-by-zero.
    However, it does remove a large category of errors that present themselves due to inappropriate typing.
  \end{remark}
\end{definition}

Because \nameref{def:Type_Inferencing} is done statically, Haskell programs are both \emph{strongly typed} and \emph{type safe}.
Meaning all \nameref{def:Type} errors that could occur in a program are found before program execution.
The use of \nameref{def:Type_Inferencing} is the reason that \textbf{all execution paths must have the same type}.

\begin{remark*}
  The determination of \nameref{def:Type}s in a program using \nameref{def:Type_Inferencing} where selection statements have different \nameref{def:Type}s is an undecidable problem.
  Thus, the language designers enforced programmers to write all \nameref{def:Conditional_Expression}, \nameref{def:Guarded_Equation}s and pattern matching paths to have the same \nameref{def:Type}.
\end{remark*}

\subsection{Basic Types}\label{subsec:Basic_Types}
There are a variety of basic types built into the Haskell language.

\subsubsection{\texorpdfstring{\haskellinline{Bool}}{\texttt{Bool}}}\label{subsubsec:Bool_Type}
The \haskellinline{Bool} \nameref{def:Type} contains the two logical values, \haskellinline{True} and \haskellinline{False}.

\subsubsection{\texorpdfstring{\haskellinline{Char}}{\texttt{Char}}}\label{subsubsec:Char_Type}
The \haskellinline{Char} \nameref{def:Type} contains all the single characters that are available from a typical keyboard, including many control characters.
Characters in Haskell must be enclosed beween forward quotes; for example, \haskellinline{'c'}.

\subsubsection{\texorpdfstring{\haskellinline{String}}{\texttt{String}}}\label{subsubsec:String_Type}
The \haskellinline{String} \nameref{def:Type} is a \nameref{def:Type_Alias} for \haskellinline{[Char]}.
This means it also includes all the characters available from a typical keyboard, and the control characters, however more than one character can be present.
Strings in Haskell are enclosed between double quotes; for example, \haskellinline{"string"}.

\begin{remark*}
  The statement about allowing more than one character be present is a little misleading.
  Since the \haskellinline{String} \nameref{def:Type} is actually \haskellinline{[Char]}, a list of single characters.
  Haskell just \textbf{shows} them to us nicely formatted.
\end{remark*}

\subsubsection{\texorpdfstring{\haskellinline{Int}}{\texttt{Int}}}\label{subsubsec:Int_Type}
The \haskellinline{Int} \nameref{def:Type} represents fixed-precision integers, analogous to the signed integer system of traditional imperative languages.
The number of bits (32 or 64) depends on your computer's architecture.
On a 64-bit computer, 64-bit integers will be used, making the limits of the representable integers $[-2^{63}, 2^{63}-1]$.
Likewise, on a 32-bit computer, 32-bit integers will be used, setting the limits of the system in the interval $[-2^{31}, 2^{31} - 1]$.

\subsubsection{\texorpdfstring{\haskellinline{Integer}}{\texttt{Integer}}}\label{subsubsec:Integer_Type}
The \haskellinline{Integer} \nameref{def:Type} represented arbitrary-precision integers.
These are integers that can be \textbf{any possible value}, limited only by the memory capabilities of your computer.
Thus, expressions of type \haskellinline{Integer} can be as big as your memory allows.

\haskellinline{Int} and \haskellinline{Integer} are also different in terms of their performance.
Typically, \haskellinline{Int} will run faster because there is dedicated hardware to perform the computation, whereas \haskellinline{Integer}s will need to be handled in software as a sequence of digits.

\subsubsection{\texorpdfstring{\haskellinline{Float}}{\texttt{Float}}}\label{subsubsec:Float_Type}

%%% Local Variables:
%%% mode: latex
%%% TeX-master: "../../EDAN40-Functional_Programming-Reference_Sheet"
%%% End:


\subsection{List Types}\label{subsec:List_Types}
\begin{definition}[List]\label{def:List}
  In Haskell, a \emph{list} is a sequence of elements that are \textbf{all the same \nameref{def:Type}}.
  The elements are enclosed within brackets, \texttt{[} and \texttt{]}, with each element separated by commas.

  \Cref{lst:List_Examples} shows examples of lists and their synatax.
\end{definition}

Below are some examples of lists.
\begin{listing}[h!tbp]
\haskellsourcefile{./EDAN40-Functional_Programming-Sections/Types_Typeclasses/Code/List_Examples.hs}
\caption{Example of Lists in Haskell}
\label{lst:List_Examples}
\end{listing}


%%% Local Variables:
%%% mode: latex
%%% TeX-master: "../../EDAN40-Functional_Programming-Reference_Sheet"
%%% End:


\subsection{Tuple Types}\label{subsec:Tuple_Types}
\begin{definition}[Tuple]\label{def:Tuple}
  A \emph{tuple} is a \textbf{finite} sequence of components/elements that may have different types.
  The elements are enclosed in parentheses and separated by commas.

  Some examples of tuples are shown in \Cref{lst:Tuple_Examples}.
\end{definition}

\begin{listing}[h!tbp]
\haskellsourcefile{./EDAN40-Functional_Programming-Sections/Types_Typeclasses/Code/Tuple_Examples.hs}
\caption{Example of Tuples in Haskell}
\label{lst:Tuple_Examples}
\end{listing}

The number of elements in a tuple is the \emph{arity} of the tuple.
The empty tuple has an arity of 0.
A tuple of arity 2 is a pair, arity 3 is a triple, and so on.

\begin{blackbox}
  {\large{\textbf{\nameref{def:Tuple}s of arity 1 are NOT allowed!!}}}
\end{blackbox}

There are 3 major things to remember about \nameref{def:Tuple} types:
\begin{enumerate}[noitemsep]
\item The type of a tuple conveys its arity.
\item There are not restirctions on the type of the elements in a tuple, they do not even have to have the same type.
\item Tuples must \textbf{ALWAYS} have a finite arity, to ensure \nameref{def:Type_Inferencing} works.
\end{enumerate}

%%% Local Variables:
%%% mode: latex
%%% TeX-master: "../../EDAN40-Functional_Programming-Reference_Sheet"
%%% End:


\subsection{Function Types}\label{subsec:Function_Types}

%%% Local Variables:
%%% mode: latex
%%% TeX-master: "../../EDAN40-Functional_Programming-Reference_Sheet"
%%% End:


\subsection{Curried Function Types}\label{subsec:Curried_Function_Types}

%%% Local Variables:
%%% mode: latex
%%% TeX-master: "../../EDAN40-Functional_Programming-Reference_Sheet"
%%% End:


\subsection{Polymorphic Types}\label{subsec:Polymorphic_Types}
In some cases, \nameref{def:Function}s can accept any \nameref{def:Type} as an argument.
For example, the \haskellinline{length} function will find the length of any provided lists, regardless of the type of the elements in the list.

To represent this variability in the \nameref{def:Type} a function may have, we use a \nameref{def:Type_Variable}.

\begin{definition}[Type Variable]\label{def:Type_Variable}
  A \emph{type variable} is like a traditional variable, except instead of representing a variety of values, it represents a variety of \nameref{def:Type}s.
  In Haskell, these are denoted with lowercase letters, typically just a single letter, in the \textbf{TYPE SIGNATURE}.
  \begin{remark}[Lowercase Letter Distinction]
    The distinction between \nameref{def:Type_Variable}s and \nameref{def:Expression}s named with lowercase letters is important to make, because a lowercase letter is a type variable only when used in a type signature.
    If a lowercase letter is used anywhere else, it is considered an expression.
  \end{remark}

  Some examples of functions that use type variables are shown in \Cref{lst:Polymorphic_Function_Examples}.
\end{definition}

These \nameref{def:Type_Variable}s are what make a function \nameref{def:Polymorphic}.

\begin{definition}[Polymorphic]\label{def:Polymorphic}
  A \emph{polymorphic} ``thing'' is one that can be multiple \nameref{def:Type}s.
  In Haskell's case, functions are the only item that can be polymorphic, thus creating polymorphic functions.
  These polymorphic functions are denoted by the \nameref{def:Type_Variable} in their type signature.

  Some examples of polymorphic functions are shown in \Cref{lst:Polymorphic_Function_Examples}.
\end{definition}

\begin{listing}[h!tbp]
\haskellsourcefile{./EDAN40-Functional_Programming-Sections/Types_Typeclasses/Code/Polymorphic_Function_Examples.hs}
\caption{Polymorphic Functions Using Type Variables}
\label{lst:Polymorphic_Function_Examples}
\end{listing}

%%% Local Variables:
%%% mode: latex
%%% TeX-master: "../../EDAN40-Functional_Programming-Reference_Sheet"
%%% End:


\subsection{Overloaded Types}\label{subsec:Overloaded_Types}
Some \nameref{def:Function}s and operators are overloaded, allowing for multiple definitions of a function using the same symbol.
The correct version of the function is chosen based on the \nameref{def:Type}s of the arguments provided to the function.
For example,
\begin{haskellsource}
> 1 + 2
3

> 1.1 + 2.2
3.3
\end{haskellsource}

To continue ensuring that only the correct types are still fed into the function, we use \nameref{def:Typeclass}es.
These are also sometimes called \nameref{rmk:Class_Constraint}s.
Some examples of \nameref{def:Typeclass}es/\nameref{rmk:Class_Constraint}s are shown in \Cref{lst:Overloaded_Types_Examples}.

\begin{listing}[h!tbp]
\haskellsourcefile{./EDAN40-Functional_Programming-Sections/Types_Typeclasses/Code/Overloaded_Types_Examples.hs}
\caption{Overloaded Function Types Examples}
\label{lst:Overloaded_Types_Examples}
\end{listing}


%%% Local Variables:
%%% mode: latex
%%% TeX-master: "../../EDAN40-Functional_Programming-Reference_Sheet"
%%% End:


\subsection{Typeclasses}\label{subsec:Typeclasses}
\begin{definition}[Typeclass]\label{def:Typeclass}
  A \emph{typeclass} is a way to group \nameref{def:Type}s together and ensure that any \nameref{def:Type} in the same typeclass has the same operations available to it.
  This is a way to group together \nameref{def:Type}s that support certain overloaded functions, called methods.

  A single \nameref{def:Type} can belong to multiple typeclasses.
  Making typeclasses similar to interfaces from the OOP world.

  \begin{remark}[Class Constraint]\label{rmk:Class_Constraint}
    A \emph{class constraint} is a way to constrain the \nameref{def:Type}s that an expression can take.
    In this section, we discuss some of the basic \nameref{def:Typeclass}es that form these class constraints.
  \end{remark}

  \begin{remark}[Class vs. Typeclass]\label{rmk:Class_vs_Typeclass}
    In Haskell, \nameref{def:Typeclass}es are called \emph{class}es.
    I deliberately call them a different name to make it clear that we are not using the term class like used in Object-Oriented Programming.
  \end{remark}
\end{definition}

\subsubsection{\texorpdfstring{\haskellinline{Eq}}{\texttt{Eq}}}\label{subsubsec:Eq_Typeclass}
The \haskellinline{Eq} \nameref{def:Typeclass} contains \nameref{def:Type}s that can be compared for equality and inequality.
For a \nameref{def:Type} to be part of this class, they must provide the methods seen in \Cref{lst:Eq_Typeclass_Methods}.

\begin{listing}[h!tbp]
\haskellsourcefile{./EDAN40-Functional_Programming-Sections/Types_Typeclasses/Code/Eq_Typeclass_Methods.hs}
\caption{\haskellinline{Eq} Typeclass Required Methods}
\label{lst:Eq_Typeclass_Methods}
\end{listing}

\textbf{ALL} the basic \nameref{def:Type}s are part of this \nameref{def:Typeclass}, including lists and tuples (if their elements are instances of types in this typeclass).
The only basic type that is \textbf{NOT} in this \nameref{def:Typeclass} are \nameref{def:Function}s, because there is no general way to compare two functions for equality.

\subsubsection{\texorpdfstring{\haskellinline{Ord}}{\texttt{Ord}}}\label{subsubsec:Ord_Typeclass}
The \haskellinline{Ord} \nameref{def:Typeclass} contains \nameref{def:Type}s that can be compared for relative value.
For a \nameref{def:Type} to be part of this class, they must provide the methods seen in \Cref{lst:Ord_Typeclass_Methods}.

\begin{listing}[h!tbp]
\haskellsourcefile{./EDAN40-Functional_Programming-Sections/Types_Typeclasses/Code/Ord_Typeclass_Methods.hs}
\caption{\haskellinline{Ord} Typeclass Required Methods}
\label{lst:Ord_Typeclass_Methods}
\end{listing}

\textbf{ALL} the basic \nameref{def:Type}s are part of this \nameref{def:Typeclass}, including lists and tuples (if their elements are instances of types in this typeclass).
\haskellinline{String}s, lists, and tuples are ordered lexicographically, meaning they are sorted according to their first distinguishing element.
The only basic type that is \textbf{NOT} in this \nameref{def:Typeclass} are \nameref{def:Function}s, because there is no general way to compare two functions for their values.

\subsubsection{\texorpdfstring{\haskellinline{Show}}{\texttt{Show}}}\label{subsubsec:Show_Typeclass}
The \haskellinline{Show} \nameref{def:Typeclass} contains \nameref{def:Type}s that can be converted to a string of characters using the \haskellinline{show} function.
For a \nameref{def:Type} to be part of this class, they must provide the methods seen in \Cref{lst:Show_Typeclass_Methods}.

\begin{listing}[h!tbp]
\haskellsourcefile{./EDAN40-Functional_Programming-Sections/Types_Typeclasses/Code/Show_Typeclass_Methods.hs}
\caption{\haskellinline{Show} Typeclass Required Methods}
\label{lst:Show_Typeclass_Methods}
\end{listing}

\textbf{ALL} the basic \nameref{def:Type}s are part of this \nameref{def:Typeclass}, including lists and tuples (if their elements are instances of types in this typeclass).
The only basic type that is \textbf{NOT} in this \nameref{def:Typeclass} are \nameref{def:Function}s, because there is no general way to show the value of a function.

\subsubsection{\texorpdfstring{\haskellinline{Read}}{\texttt{Read}}}\label{subsubsec:Read_Typeclass}
The \haskellinline{Read} \nameref{def:Typeclass} contains \nameref{def:Type}s whose values can be converted from strings of characters to values.
For a \nameref{def:Type} to be part of this class, they must provide the methods seen in \Cref{lst:Read_Typeclass_Methods}.

\begin{listing}[h!tbp]
\haskellsourcefile{./EDAN40-Functional_Programming-Sections/Types_Typeclasses/Code/Read_Typeclass_Methods.hs}
\caption{\haskellinline{Read} Typeclass Required Methods}
\label{lst:Read_Typeclass_Methods}
\end{listing}

\textbf{ALL} the basic \nameref{def:Type}s are part of this \nameref{def:Typeclass}, including lists and tuples (if their elements are instances of types in this typeclass).
The only basic type that is \textbf{NOT} in this \nameref{def:Typeclass} are \nameref{def:Function}s, because there is no general way to show the value of a function.

\subsubsection{\texorpdfstring{\haskellinline{Num}}{\texttt{Num}}}\label{subsubsec:Num_Typeclass}
The \haskellinline{Num} \nameref{def:Typeclass} contains \nameref{def:Type}s whose values are numeric, and are part of the \nameref{subsubsec:Eq_Typeclass} and \nameref{subsubsec:Show_Typeclass} typeclasses.
For a \nameref{def:Type} to be part of this class, they must provide the methods seen in \Cref{lst:Num_Typeclass_Methods}.

\subsubsection{\texorpdfstring{\haskellinline{Integral}}{\texttt{Integral}}}\label{subsubsec:Integral_Typeclass}
\subsubsection{\texorpdfstring{\haskellinline{Fractional}}{\texttt{Fractional}}}\label{subsubsec:Fractional_Typeclass}
%%% Local Variables:
%%% mode: latex
%%% TeX-master: "../../EDAN40-Functional_Programming-Reference_Sheet"
%%% End:


%%% Local Variables:
%%% mode: latex
%%% TeX-master: "../EDAN40-Functional_Programming-Reference_Sheet"
%%% End:
