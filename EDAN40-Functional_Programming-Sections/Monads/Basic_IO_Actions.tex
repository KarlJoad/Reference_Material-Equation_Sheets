\subsection{Basic \texorpdfstring{\haskellinline{IO}}{\texttt{IO}} Actions}\label{subsec:Basic_IO_Actions}
There are 3 main actions in the \texttt{IO} monadic typeclass that are useful for our use.
\begin{nocrefenumerate}
\item \haskellinline{getChar}\label{act:IO_getChar}
\item \haskellinline{putChar}\label{act:IO_putChar}
\item \haskellinline{return}\label{act:IO_return}
\end{nocrefenumerate}

The \texttt{getChar} action, (Action~\ref{act:IO_getChar}) reads a character from the keyboard, echoes it to the screen, and returns the character as its result value.
If there are no characters waiting to be read from the keyboard, \texttt{getChar} waits until one is typed.
\begin{listing}[h!tbp]
\begin{haskellsource}
getChar :: IO Char
getChar = ...
\end{haskellsource}
\caption{\texttt{getChar} Type Signature}
\label{lst:IO_getChar_Type}
\end{listing}
The actual definition for \texttt{getChar} is built-in to the particular Haskell system you are using, and cannot be defined within Haskell itself.

The \texttt{putChar c} action, (Action~\ref{act:IO_putChar}) writes the character \texttt{c} to the screen, and returns no result value, represented by the empty tuple.
\begin{listing}[h!tbp]
\begin{haskellsource}
putChar :: Char -> IO ()
putChar c = ...
\end{haskellsource}
\caption{\texttt{putChar} Type Signature}
\label{lst:IO_putChar_Type}
\end{listing}
The actual definition for \texttt{putChar} is built-in to the particular Haskell system you are using, and cannot be defined within Haskell itself.

The \texttt{return v} action, (Action~\ref{act:IO_return}) simply returns the result value \texttt{v} without performing any interaction.
\begin{listing}[h!tbp]
\begin{haskellsource}
return :: a -> IO a
return v = \world -> (v, world)
\end{haskellsource}
\caption{\texttt{return} Type Signature}
\label{lst:IO_return_Type}
\end{listing}
\texttt{return} provides a bridge from the setting of pure expressions without side effects to that of impure actions with side effects.

\begin{blackbox}
  \textbf{Once we are impure we are impure for ever, with no possibility for redemption!}
  As a result, we may suspect that impurity quickly permeates entire programs, but in practice this is usually not the case.
  For most Haskell programs, the vast majority of functions do not involve interaction, and the relatively small number of those that do are at the outermost level.
\end{blackbox}

%%% Local Variables:
%%% mode: latex
%%% TeX-master: "../../EDAN40-Functional_Programming-Reference_Sheet"
%%% End:
