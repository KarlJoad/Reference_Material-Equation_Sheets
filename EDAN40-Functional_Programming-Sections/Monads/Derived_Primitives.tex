\subsection{Derived Primitives}\label{subsec:Derived_Primitives}
We can perform actions on \nameref{def:Monad}s quite similarly to the way we operate on other \nameref{def:Type}s in Haskell.
We can extend the basic ones and get more complex ones as we go.
Here, we introduce 3 more monadic functions that use the \texttt{IO} \nameref{def:Monad} to perform many operations that are useful for interaction.
\begin{enumerate}[noitemsep]
\item \haskellinline{getLine}\label{act:IO_getLine}
  The definition of the \texttt{getLine} function is shown in \Cref{lst:IO_getLine_Definition}.
\end{enumerate}

\begin{listing}[h!tbp]
\haskellsourcefile{./EDAN40-Functional_Programming-Sections/Monads/Code/IO_getLine_Definition.hs}
\caption{\texttt{getLine} Definition}
\label{lst:IO_getLine_Definition}
\end{listing}


%%% Local Variables:
%%% mode: latex
%%% TeX-master: "../../EDAN40-Functional_Programming-Reference_Sheet"
%%% End:
