\section{Lazy Evaluation}\label{sec:Lazy_Evaluation}
So far, the basic method of computation in Haskell has been the application of \nameref{def:Function}s to arguments.
Because these functions are pure, changing the order in which functions are applied does not affect the final result.
This is not specific to simple examples, but an important general property of function application in Haskell.
Formally, in Haskell any two different ways of evaluating the same expression will always produce the same final value, provided that they both terminate.

An \nameref{def:Expression} that has the form of a \nameref{def:Function} applied to one or more arguments that can be ``reduced'' by performing the application is called a \nameref{def:Reducible_Expression}

\begin{definition}[Reducible Expression]\label{def:Reducible_Expression}
  A \emph{reducible expression} or \emph{redex} has the form of a \nameref{def:Function} applied to one or more arguments that can be ``reduced'' by performing the appropriate function application.
  These reductions do not necessarily decrease the size of an \nameref{def:Expression}, but often do.
\end{definition}

Consider the expression \haskellinline{mult (1+2, 2+3)}.
This expression contains three redexes:
\begin{enumerate}[noitemsep]
\item The sub-expression $1+2$, the \texttt{+} function applied with two arguments.
\item The sub-expression $2+3$, the \texttt{+} function applied with two arguments.
\item The entire expression itself, which has the form of the function \texttt{mult} applied to a pair of arguments.
\end{enumerate}

Performing the corresponding reductions gives the expressions:
\begin{enumerate}[noitemsep]
\item \haskellinline{mult (3, 2 + 3)}
\item \haskellinline{mult (1 + 2, 5)}
\item \haskellinline{(1 + 2) * (2 + 3)}
\end{enumerate}

The order in which we evaluate these redexes determines what ``Pass-by'' type we use.

\subsection{Pass-by-Value}\label{subsec:Pass_by_Value}
Innermost evaluation, always chooses the innermost redex, meaning it contains no other redex.
If there is more than one innermost redex, by convention we choose that which begins at the leftmost position in the expression.

Innermost evaluation can also be characterised in terms of how arguments are passed to \nameref{def:Function}s.
In particular, using this strategy ensures that the argument of a function is always fully evaluated before the function itself is applied.
That is, arguments are \emph{Passed-By-Value}.

\subsection{Pass-by-Name}\label{subsec:Pass_by_Name}
Outermost evaluation, dual to innermost evaluation, is to always choose a redex that is outermost, meaning it is contained in no other redex.
If there is more than one such redex, then we choose the one that begins at the leftmost position.

For example, the expression \haskellinline{mult (1+2) (2+3)} is contained in no other redex and is hence outermost within itself.
\begin{align*}
  &= mult(1+2, 2+3) \\
  &= (1+2) * (2+3) \\
  &= (3) * (2+3) \\
  &= 3 * (5) \\
  &= 15
\end{align*}

In terms of how arguments are passed to functions, using outermost evaluation allows functions to be applied before their arguments are evaluated.
For this reason, we say that arguments are \emph{Passed-By-Name}.

\begin{remark*}
  Many built-in functions require their arguments to be evaluated before being applied, even when using outermost evaluation.
  For example, built-in arithmetic operators ($*$ and $+$) cannot be applied until their two arguments have been evaluated to numbers.
  Functions with this property are called \nameref{def:Strict}.
\end{remark*}

\subsection{Reduction of Lambda Expressions}\label{subsec:Lambda_Expression_Reduction}
If we write a curried version of \texttt{mult}, \haskellinline{mult x = \y = x * y}.
The two arguments are now substituted into the body of the function mult one at a time.
This is the expected behavior of currying, rather than at the same time as in the previous section.
This behaviour arises because \haskellinline{mult 3} is the leftmost innermost redex in the expression \haskellinline{mult 3 (2+3)}, as opposed to $2+3$ in the expression \haskellinline{mult (3, 2 + 3)}.

In Haskell, the selection of redexes within lambda expressions is prohibited.
The rationale for not ``reducing under lambdas'' is that functions are viewed as black boxes that we are not permitted to look inside.

\begin{blackbox}
  \begin{center}
    {\large{Formally, the only operation that can be performed on a function is that of applying it to an argument.
        As such, reduction within the body of a function is only permitted once the function has been applied.}}
  \end{center}
\end{blackbox}

\subsection{Termination}\label{subsec:Termination}
Throughout this section, we will use a recursive definition that yields a value similar to infinity.
\begin{listing}[h!tbp]
\haskellsourcefile{./EDAN40-Functional_Programming-Sections/Lazy_Evaluation/Code/infty.hs}
\caption{Definition of Infinity for \Cref{subsec:Termination}}
\label{lst:Infty_Definition}
\end{listing}

In practice, trying to evaluate \texttt{infty} will quickly exhaust the available memory and produce an error message.
Consider the expression \haskellinline{fst (0, inf)}, where \texttt{fst} is the library function that selects the first component of a pair, defined by \haskellinline{fst (x, y) = x}.

Call-by-value evaluation with this expression also results in non-termination.
In contrast, call-by-name evaluation terminates in just one step, by immediately applying the definition of \texttt{fst} avoiding the evaluation of the non-terminating expression.

This shows that call-by-name evaluation may produce a result when call-by-value evaluation fails to terminate.
In general, we have the following important property: if there exists any evaluation sequence that terminates for a given expression, then call-by-name evaluation will also terminate for this expression, and produce the same final result.

In summary, call-by-name evaluation is preferable to call-by-value for the purpose of ensuring that evaluation terminates as often as possible.

%%% Local Variables:
%%% mode: latex
%%% TeX-master: "../../EDAN40-Functional_Programming-Reference_Sheet"
%%% End:



%%% Local Variables:
%%% mode: latex
%%% TeX-master: "../EDAN40-Functional_Programming-Reference_Sheet"
%%% End:
