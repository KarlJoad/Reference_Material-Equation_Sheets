\subsection{Typeclasses}\label{subsec:Typeclasses}
\begin{definition}[Typeclass]\label{def:Typeclass}
  A \emph{typeclass} is a way to group \nameref{def:Type}s together and ensure that any \nameref{def:Type} in the same typeclass has the same operations available to it.
  This is a way to group together \nameref{def:Type}s that support certain overloaded functions, called methods.

  A single \nameref{def:Type} can belong to multiple typeclasses.
  Making typeclasses similar to interfaces from the OOP world.

  \begin{remark}[Class Constraint]\label{rmk:Class_Constraint}
    A \emph{class constraint} is a way to constrain the \nameref{def:Type}s that an expression can take.
    In this section, we discuss some of the basic \nameref{def:Typeclass}es that form these class constraints.
  \end{remark}

  \begin{remark}[Class vs. Typeclass]\label{rmk:Class_vs_Typeclass}
    In Haskell, \nameref{def:Typeclass}es are called \emph{class}es.
    I deliberately call them a different name to make it clear that we are not using the term class like used in Object-Oriented Programming.
  \end{remark}
\end{definition}


%%% Local Variables:
%%% mode: latex
%%% TeX-master: "../../EDAN40-Functional_Programming-Reference_Sheet"
%%% End:
