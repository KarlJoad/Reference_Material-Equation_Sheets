\subsection{List Types}\label{subsec:List_Types}
\begin{definition}[List]\label{def:List}
  In Haskell, a \emph{list} is a sequence of elements that are \textbf{all the same \nameref{def:Type}}.
  The elements are enclosed within brackets, \texttt{[} and \texttt{]}, with each element separated by commas.

  \Cref{lst:List_Examples} shows examples of lists and their synatax.
\end{definition}

Below are some examples of lists.
\begin{listing}[h!tbp]
\haskellsourcefile{./EDAN40-Functional_Programming-Sections/Types_Typeclasses/Code/List_Examples.hs}
\caption{Example of Lists in Haskell}
\label{lst:List_Examples}
\end{listing}

The number of elements in a list is its length, where the empty list has length zero and does not contain any elements.
The empty list is unique in that it can also be considered an element for other lists, allowing for the construction of lists using the \texttt{cons} operator \haskellinline{:}.

There are 3 major things to remember about \nameref{def:List} types:
\begin{enumerate}[noitemsep]
\item The type of a list conveys no information about its length.
\item There are no restrictions on the type of the elements of a list, so long as they are all the same type.
\item There is no restriction that a list must have a finite length.
\end{enumerate}

%%% Local Variables:
%%% mode: latex
%%% TeX-master: "../../EDAN40-Functional_Programming-Reference_Sheet"
%%% End:
