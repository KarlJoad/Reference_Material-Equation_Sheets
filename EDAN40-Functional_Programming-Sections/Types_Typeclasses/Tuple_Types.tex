\subsection{Tuple Types}\label{subsec:Tuple_Types}
\begin{definition}[Tuple]\label{def:Tuple}
  A \emph{tuple} is a \textbf{finite} sequence of components/elements that may have different types.
  The elements are enclosed in parentheses and separated by commas.

  Some examples of tuples are shown in \Cref{lst:Tuple_Examples}.
\end{definition}

\begin{listing}[h!tbp]
\haskellsourcefile{./EDAN40-Functional_Programming-Sections/Types_Typeclasses/Code/Tuple_Examples.hs}
\caption{Example of Tuples in Haskell}
\label{lst:Tuple_Examples}
\end{listing}

The number of elements in a tuple is the \emph{arity} of the tuple.
The empty tuple has an arity of 0.
A tuple of arity 2 is a pair, arity 3 is a triple, and so on.

\begin{blackbox}
  {\large{\textbf{\nameref{def:Tuple}s of arity 1 are NOT allowed!!}}}
\end{blackbox}

There are 3 major things to remember about \nameref{def:Tuple} types:
\begin{enumerate}[noitemsep]
\item The type of a tuple conveys its arity.
\item There are not restirctions on the type of the elements in a tuple, they do not even have to have the same type.
\item Tuples must \textbf{ALWAYS} have a finite arity, to ensure \nameref{def:Type_Inferencing} works.
\end{enumerate}

%%% Local Variables:
%%% mode: latex
%%% TeX-master: "../../EDAN40-Functional_Programming-Reference_Sheet"
%%% End:
