\subsection{Tuple Types}\label{subsec:Tuple_Types}
\begin{definition}[Tuple]\label{def:Tuple}
  A \emph{tuple} is a \textbf{finite} sequence of components/elements that may have different types.
  The elements are enclosed in parentheses and separated by commas.

  Some examples of tuples are shown in \Cref{lst:Tuple_Examples}.
\end{definition}

\begin{listing}[h!tbp]
\haskellsourcefile{./EDAN40-Functional_Programming-Sections/Types_Typeclasses/Code/Tuple_Examples.hs}
\caption{Example of Tuples in Haskell}
\label{lst:Tuple_Examples}
\end{listing}

The number of elements in a tuple is the \emph{arity} of the tuple.
The empty tuple has an arity of 0.
A tuple of arity 2 is a pair, arity 3 is a triple, and so on.


%%% Local Variables:
%%% mode: latex
%%% TeX-master: "../../EDAN40-Functional_Programming-Reference_Sheet"
%%% End:
