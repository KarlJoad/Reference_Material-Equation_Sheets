\subsection{Overloaded Types}\label{subsec:Overloaded_Types}
Some \nameref{def:Function}s and operators are overloaded, allowing for multiple definitions of a function using the same symbol.
The correct version of the function is chosen based on the \nameref{def:Type}s of the arguments provided to the function.
For example,
\begin{haskellsource}
> 1 + 2
3

> 1.1 + 2.2
3.3
\end{haskellsource}

To continue ensuring that only the correct types are still fed into the function, we use \nameref{def:Typeclass}es.
These are also sometimes called \nameref{rmk:Class_Constraint}s.
Some examples of \nameref{def:Typeclass}es/\nameref{rmk:Class_Constraint}s are shown in \Cref{lst:Overloaded_Types_Examples}.

\begin{listing}[h!tbp]
\haskellsourcefile{./EDAN40-Functional_Programming-Sections/Types_Typeclasses/Code/Overloaded_Types_Examples.hs}
\caption{Overloaded Function Types Examples}
\label{lst:Overloaded_Types_Examples}
\end{listing}


%%% Local Variables:
%%% mode: latex
%%% TeX-master: "../../EDAN40-Functional_Programming-Reference_Sheet"
%%% End:
