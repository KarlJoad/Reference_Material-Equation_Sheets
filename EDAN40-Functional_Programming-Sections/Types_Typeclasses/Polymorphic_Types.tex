\subsection{Polymorphic Types}\label{subsec:Polymorphic_Types}
In some cases, \nameref{def:Function}s can accept any \nameref{def:Type} as an argument.
For example, the \haskellinline{length} function will find the length of any provided lists, regardless of the type of the elements in the list.

To represent this variability in the \nameref{def:Type} a function may have, we use a \nameref{def:Type_Variable}.

\begin{definition}[Type Variable]\label{def:Type_Variable}
  A \emph{type variable} is like a traditional variable, except instead of representing a variety of values, it represents a variety of \nameref{def:Type}s.
  In Haskell, these are denoted with lowercase letters, typically just a single letter, in the \textbf{TYPE SIGNATURE}.
  \begin{remark}[Lowercase Letter Distinction]
    The distinction between \nameref{def:Type_Variable}s and \nameref{def:Expression}s named with lowercase letters is important to make, because a lowercase letter is a type variable only when used in a type signature.
    If a lowercase letter is used anywhere else, it is considered an expression.
  \end{remark}

  Some examples of functions that use type variables are shown in \Cref{lst:Polymorphic_Function_Examples}.
\end{definition}

These \nameref{def:Type_Variable}s are what make a function \nameref{def:Polymorphic}.

\begin{definition}[Polymorphic]\label{def:Polymorphic}
  A \emph{polymorphic} ``thing'' is one that can be multiple \nameref{def:Type}s.
  In Haskell's case, functions are the only item that can be polymorphic, thus creating polymorphic functions.
  These polymorphic functions are denoted by the \nameref{def:Type_Variable} in their type signature.

  Some examples of polymorphic functions are shown in \Cref{lst:Polymorphic_Function_Examples}.
\end{definition}

%%% Local Variables:
%%% mode: latex
%%% TeX-master: "../../EDAN40-Functional_Programming-Reference_Sheet"
%%% End:
