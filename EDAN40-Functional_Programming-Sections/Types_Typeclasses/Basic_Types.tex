\subsection{Basic Types}\label{subsec:Basic_Types}
There are a variety of basic types built into the Haskell language.

\subsubsection{\texorpdfstring{\haskellinline{Bool}}{\texttt{Bool}}}\label{subsubsec:Bool_Type}
The \haskellinline{Bool} \nameref{def:Type} contains the two logical values, \haskellinline{True} and \haskellinline{False}.

\subsubsection{\texorpdfstring{\haskellinline{Char}}{\texttt{Char}}}\label{subsubsec:Char_Type}
The \haskellinline{Char} \nameref{def:Type} contains all the single characters that are available from a typical keyboard, including many control characters.
Characters in Haskell must be enclosed beween forward quotes; for example, \haskellinline{'c'}.

\subsubsection{\texorpdfstring{\haskellinline{String}}{\texttt{String}}}\label{subsubsec:String_Type}
The \haskellinline{String} \nameref{def:Type} is a \nameref{def:Type_Alias} for \haskellinline{[Char]}.
This means it also includes all the characters available from a typical keyboard, and the control characters, however more than one character can be present.
Strings in Haskell are enclosed between double quotes; for example, \haskellinline{"string"}.

\begin{remark*}
  The statement about allowing more than one character be present is a little misleading.
  Since the \haskellinline{String} \nameref{def:Type} is actually \haskellinline{[Char]}, a list of single characters.
  Haskell just \textbf{shows} them to us nicely formatted.
\end{remark*}

\subsubsection{\texorpdfstring{\haskellinline{Int}}{\texttt{Int}}}\label{subsubsec:Int_Type}
The \haskellinline{Int} \nameref{def:Type} represents fixed-precision integers, analogous to the signed integer system of traditional imperative languages.
The number of bits (32 or 64) depends on your computer's architecture.
On a 64-bit computer, 64-bit integers will be used, making the limits of the representable integers $[-2^{63}, 2^{63}-1]$.
Likewise, on a 32-bit computer, 32-bit integers will be used, setting the limits of the system in the interval $[-2^{31}, 2^{31} - 1]$.

\subsubsection{\texorpdfstring{\haskellinline{Integer}}{\texttt{Integer}}}\label{subsubsec:Integer_Type}
The \haskellinline{Integer} \nameref{def:Type} represented arbitrary-precision integers.
These are integers that can be \textbf{any possible value}, limited only by the memory capabilities of your computer.
Thus, expressions of type \haskellinline{Integer} can be as big as your memory allows.

\haskellinline{Int} and \haskellinline{Integer} are also different in terms of their performance.
Typically, \haskellinline{Int} will run faster because there is dedicated hardware to perform the computation, whereas \haskellinline{Integer}s will need to be handled in software as a sequence of digits.

\subsubsection{\texorpdfstring{\haskellinline{Float}}{\texttt{Float}}}\label{subsubsec:Float_Type}
The \haskellinline{Float} \nameref{def:Type} represents single-precision floating-point numbers.
Single-precision refers to the number of bits used to represent the number, in this case, 32.
Floating-point refers to the ability to represent values like $\frac{1}{3}$, $\pi$, and any other value with a fractional portion.

%%% Local Variables:
%%% mode: latex
%%% TeX-master: "../../EDAN40-Functional_Programming-Reference_Sheet"
%%% End:
