\section{Recursion}\label{sec:Recursion}
In Haskell, there is \textbf{no} concept of iteration.
By extension, this also means that there is no \texttt{while}, \texttt{for}, and/or \texttt{do-while} looping structures.
Instead of using iteration, \nameref{def:Recursion} is used instead.

\begin{definition}[Recursion]\label{def:Recursion}
  \emph{Recursion} is the process of defining something in terms of itself, typically in a smaller amount/value.
  By ``slightly'' reducing the size of the problem through each recursive ``iteration'', the problem domain is smaller.
  Eventually, the problem domain becomes small enough that the solution is simple.
  This is called the \emph{base case}.
  Once the simple solution has been found, then all recursions can start building from that solution.

  \begin{remark}[Recursive]\label{rmk:Recursive}
    A function that uses recursion is said to be \emph{recursive}.
  \end{remark}
\end{definition}

The reason that \nameref{def:Recursion} is used instead of iteration is for multiple reasons.
\begin{itemize}[noitemsep]
\item \nameref{def:Recursion} can compute everything iteration can.
\item \nameref{def:Recursion} is inductively provable.
\item \nameref{def:Recursion} is side-effect free.
\item \nameref{def:Recursion} is a natural way to work with algebraic datatypes, lists, and the infinite structures of Functional Programming.
\end{itemize}

One of the first \nameref{rmk:Recursive} programs most people write is the factorial function.
\begin{listing}[h!tbp]
\haskellsourcefile{./EDAN40-Functional_Programming-Sections/Recursion/Code/Factorial.hs}
\caption{Factorial, Recursively Defined}
\label{lst:Recursive_Factorial}
\end{listing}

Sometimes library functions simplify the way a function is written, but many function also have simple and natural definitions using \nameref{def:Recursion}.

\begin{remark*}
  Note that throughout this section, we are regularly writing functions with the same name as functions included in the \texttt{Prelude.hs} library file.
  To prevent possible namespace collisions, functions I am writing that have the same name as \texttt{Prelude.hs} library functions will be named slightly differently, typically preceeded with \texttt{my}.
\end{remark*}

\begin{remark*}
  Throughout this section, many \nameref{rmk:Recursive} functions will be created, many of them quite inefficient.
  However, they are written this way to emphasize clarity and convey what recursion is and how to write recursive structures.
\end{remark*}

\subsection{List Recursion}\label{subsec:List_Recursion}

%%% Local Variables:
%%% mode: latex
%%% TeX-master: "../../EDAN40-Functional_Programming-Reference_Sheet"
%%% End:


\subsection{Multiple Arguments}\label{subsec:Multiple_Arguments}

%%% Local Variables:
%%% mode: latex
%%% TeX-master: "../../EDAN40-Functional_Programming-Reference_Sheet"
%%% End:


\subsection{Multiple Recursion}\label{subsec:Multiple_Recursion}

%%% Local Variables:
%%% mode: latex
%%% TeX-master: "../../EDAN40-Functional_Programming-Reference_Sheet"
%%% End:


\subsection{Mutual Recursion}\label{subsec:Mutual_Recursion}

%%% Local Variables:
%%% mode: latex
%%% TeX-master: "../../EDAN40-Functional_Programming-Reference_Sheet"
%%% End:


\subsection{Help With Recursion}\label{subsec:Help_Recursion}
Defining \nameref{rmk:Recursive} functions may seem easy, but it can be quite difficult.
Here, I will include a list of steps that may make it easier to think about and write new recursive functions.

\begin{enumerate}[noitemsep]
\item Define the type of the function. \Cref{subsubsec:Help_Recursion-Function_Type}
\item Enumerate the cases. \Cref{subsubsec:Help_Recursion-Enumerate_Cases}
\item Define the Simple cases, The Base Cases. \Cref{subsubsec:Help_Recursion-Base_Case}
\item Define the other cases. \Cref{subsubsec:Help_Recursion-Other_Cases}
\item Generalize and Simplify. \Cref{subsubsec:Help_Recursion-Generalize}
\end{enumerate}

\subsubsection{Define Function Type}\label{subsubsec:Help_Recursion-Function_Type}
When first starting your function definition, think about what you want this function to do.
\begin{itemize}[noitemsep]
\item What type do you want it to take in?
\item What type should it output?
\item What should be returned by the function?
\end{itemize}

Do not worry about generalizing the types of the function for now, that can come once you've written the thing.

\subsubsection{Enumerate Cases}\label{subsubsec:Help_Recursion-Enumerate_Cases}
What are the cases you can have?
You will need to have one or more base cases, to eventually terminate the recursion.
You also need cases that slowly break your information down into smaller pieces that you can perform some action on.

For example, with lists, there are 2 general cases:
\begin{enumerate}[noitemsep]
\item Empty Lists --- Base case
\item Non-empty Lists --- Other Case(s)
\end{enumerate}

\subsubsection{Define Simple Cases}\label{subsubsec:Help_Recursion-Base_Case}
In the simple cases, there is usually an obvious solution to the problem at hand.
For example, if you are writing the \haskellinline{product} function for lists, the base case would be the empty list.
In that case, you would simply return 1, because that is the multiplicative identity.

\subsubsection{Define Other Cases}\label{subsubsec:Help_Recursion-Other_Cases}
Here is where some of the most difficult work can happen.
When you are \textbf{NOT} in the base case, what do you want to do?
Additionally, how do you eventually get down to the base case?

\subsubsection{Generalize/Simplify}\label{subsubsec:Help_Recursion-Generalize}
Once you have finished writing the recursive function, and tested it to make sure it works, you can clean up your work.
You can start by generalizing the type, if possible.
For example, if a recursive function is acting on a list, but does not use the information within the list, then the type can be generalized.
If the function started with a type signature
\begin{haskellsource}
[Int] -> Int
\end{haskellsource}
but you never use the contents of the list, then you should be able to generalize the function to
\begin{haskellsource}
[a] -> Int
\end{haskellsource}

Additionally, if you should also look at your function to see if a library function can achieve many of the same things as your custom recursion does.
For example, if you want to apply a function to every element in a list, and you've written your own implementation, switch to the \haskellinline{map} function instead.

%%% Local Variables:
%%% mode: latex
%%% TeX-master: "../../EDAN40-Functional_Programming-Reference_Sheet"
%%% End:


%%% Local Variables:
%%% mode: latex
%%% TeX-master: "../EDAN40-Functional_Programming-Reference_Sheet"
%%% End:
