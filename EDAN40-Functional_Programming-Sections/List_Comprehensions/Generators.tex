\subsection{Generators}\label{subsec:Generators}
In mathematics, the comprehension notation can be used to construct new sets from existing sets.
For example, to get the set squares from one to five is written like this in mathematics.
\begin{equation}\label{eq:Generator}
  \lbrace 1, 4, 9, 16, 25 \rbrace = \bigl\lbrace x^{2} \vert x \in \lbrace 1 \ldots 5 \lbrace \bigr\rbrace
\end{equation}

\Cref{eq:Generator} would be said to contain all numbers $x^{2}$ where each $x$ is an element of the set $\lbrace 1 \ldots 5 \rbrace$.
In Haskell, this same thing would be written as
\begin{listing}[h!tbp]
\begin{haskellsource}
> [x^2 | x <- [1..5]]
[1, 4, 9, 16, 25]
\end{haskellsource}
\caption{Haskell List Comprehensions}
\label{lst:List_Comprehension}
\end{listing}

\begin{definition}[Generator]\label{def:Generator}
  A \emph{generator} is an expression that generates all possible values in a set.
  In the cases above, \haskellinline{x <- [1..5]} is a generator.
\end{definition}

In addition to the usual usage of \nameref{def:Generator}s to get values out of a list/set, the wildcard symbol \haskellinline{_} can be used too.
\begin{listing}[h!tbp]
\haskellsourcefile{./EDAN40-Functional_Programming-Sections/List_Comprehensions/Code/Wildcard_Generator.hs}
\caption{Wildcard Generator}
\label{lst:Wildcard_Generator}
\end{listing}

\subsubsection{Multiple Generators}\label{subsubsec:Multiple_Generators}
Multiple generators can be defined together using a comma between them.
See \Cref{lst:Multiple_Generators} below.
\begin{listing}[h!tbp]
\haskellsourcefile{./EDAN40-Functional_Programming-Sections/List_Comprehensions/Code/Multiple_Generators.hs}
\caption{Multiple Generators}
\label{lst:Multiple_Generators}
\end{listing}

\subsubsection{Dependent Generators}\label{subsubsec:Dependent_Generators}
Lastly, more deeply nested \nameref{def:Generator}s can depend on earlier ones.
In \Cref{lst:Dependent_Generators}, we generate the same lists as in \Cref{lst:Multiple_Generators}, but having the second \nameref{def:Generator} being dependent on the first.
\begin{listing}[h!tbp]
\haskellsourcefile{./EDAN40-Functional_Programming-Sections/List_Comprehensions/Code/Dependent_Generators.hs}
\caption{Dependent Generators}
\label{lst:Dependent_Generators}
\end{listing}

%%% Local Variables:
%%% mode: latex
%%% TeX-master: "../../EDAN40-Functional_Programming-Reference_Sheet"
%%% End:
