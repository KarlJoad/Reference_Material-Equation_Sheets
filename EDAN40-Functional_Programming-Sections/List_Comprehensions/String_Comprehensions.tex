\subsection{String Comprehensions}\label{subsec:String_Comprehensions}
\begin{blackbox}
  \textbf{Strings in Haskell are nothing but lists of characters (there is no null terminator in the list).}
\end{blackbox}

For example, \haskellinline{``abc''::String} is just an abbreviation for \haskellinline{[`a', `b', `c']::[Char]}.
Because strings are just special kinds of lists, any polymorphic function on lists can also be used with strings.
\begin{listing}[h!tbp]
\haskellsourcefile{./EDAN40-Functional_Programming-Sections/List_Comprehensions/Code/Polymorphic_List_Comprehensions.hs}
\caption{Polymorphic List Comprehensions Used on Strings}
\label{lst:Polymorphic_List_Comprehensions_Strings}
\end{listing}

For the same reason, list comprehensions can also be used to define functions on strings.
\begin{listing}[h!tbp]
\haskellsourcefile{./EDAN40-Functional_Programming-Sections/List_Comprehensions/Code/String_List_Comprehensions.hs}
\caption{List Comprehensions Used on \haskellinline{String}s}
\label{lst:String_List_Comprehensions}
\end{listing}

%%% Local Variables:
%%% mode: latex
%%% TeX-master: "../../EDAN40-Functional_Programming-Reference_Sheet"
%%% End:
