\subsection{The \texorpdfstring{\haskellinline{zip}}{texttt{zip}} Function}\label{subsec:Zip_Function}
The library function \haskellinline{zip} produces a new list by pairing successive elements from two existing lists until either or both are exhausted.
This means that the list returned by \haskellinline{zip} will always be limited by the shortest list provided.
The \haskellinline{zip} function is often useful when programming with list comprehensions.
\begin{align*}
  listOfXs &= [x_{1}, x_{2}, x_{3}, \ldots] \\
  listOfYs &= [y_{1}, y_{2}, y_{3}, \ldots] \\
  \mathtt{zip} listOfXs listOfYs &= [(x_{1}, y_{1}), (x_{2}, y_{2}), \ldots]
\end{align*}


An example here is determining if the elements in a list are sorted, in ascending value.
\begin{listing}[h!tbp]
\haskellsourcefile{./EDAN40-Functional_Programming-Sections/List_Comprehensions/Code/Zip_Function.hs}
\caption{Using the \haskellinline{zip} Function}
\label{lst:Zip_Function}
\end{listing}

%%% Local Variables:
%%% mode: latex
%%% TeX-master: "../../EDAN40-Functional_Programming-Reference_Sheet"
%%% End:
