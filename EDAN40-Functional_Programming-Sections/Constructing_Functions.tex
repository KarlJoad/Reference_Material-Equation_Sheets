\section{Constructing Functions}\label{sec:Constructing_Functions}
The most straight-forward way of constructing functions is to use functions that are already provided.
For example, to define reciprocation of an integer or rational number, we would write:
\begin{listing}[h!tbp]
\begin{haskellsource}
  recip n = 1 / n
  -- The / symbol is a function that is allowed to use infix notation.
\end{haskellsource}
\caption{Define Function From Others}
\label{lst:Define_Func_From_Others}
\end{listing}

\subsection{Conditional Expressions}\label{subsec:Conditional_Expressions}

%%% Local Variables:
%%% mode: latex
%%% TeX-master: "../../EDAN40-Functional_Programming-Reference_Sheet"
%%% End:


\subsection{Guarded Equations}\label{subsec:Guarded_Equations}
\begin{definition}[Guarded Equation]\label{def:Guarded_Equation}
  A \emph{guarded equation} is one in which a series of \nameref{def:Guard}s are used to choose between a sequence of results that all have the same type.
\end{definition}

\begin{definition}[Guard]\label{def:Guard}
  A \emph{guard} is a conditional predicate that is used to construct \nameref{def:Guarded_Equation}s.
  The guarding predicate is denoted with a \texttt{|} and is read as ``such that''.
\end{definition}

Using \nameref{def:Guarded_Equation}s to define functions is an alternative to the use of \nameref{def:Conditional_Expression}s.
The benefit of using \nameref{def:Guarded_Equation}s is that functions with multiple \nameref{def:Guard}s are easier to read.

\begin{blackbox}
  In \nameref{def:Guarded_Equation}s, just like in \nameref{def:Conditional_Expression}s, all possible options \textbf{MUST} have the same type.
\end{blackbox}

An example of the same absolute value function from \Cref{lst:Conditional_Expression} is shown in \Cref{lst:Guarded_Equation}.
\begin{listing}[h!tbp]
\haskellsourcefile{./EDAN40-Functional_Programming-Sections/Constructing_Functions/Code/Guarded_Equations.hs}
\caption{A Guarded Equation in Haskell}
\label{lst:Guarded_Equation}
\end{listing}

The use of \haskellinline{otherwise} in \Cref{lst:Guarded_Equation} denotes a ``default'' case.
If none of the previous \nameref{def:Guard}s apply to the argument given to the function, then the \haskellinline{otherwise} option is chosen.
One thing to note is that the \nameref{def:Guard}s are checked in the order they are written.
So, if \haskellinline{otherwise} appears before the end of the function, it will be matched early.
\begin{listing}[h!tbp]
\haskellsourcefile{./EDAN40-Functional_Programming-Sections/Constructing_Functions/Code/Guarded_Equations-Early_Match.hs}
\caption{A Guarded Equation with Early Matching}
\label{lst:Guarded_Equation_Early_Match}
\end{listing}

%%% Local Variables:
%%% mode: latex
%%% TeX-master: "../../EDAN40-Functional_Programming-Reference_Sheet"
%%% End:


\subsection{Pattern Matching}\label{subsec:Pattern_Matching}

%%% Local Variables:
%%% mode: latex
%%% TeX-master: "../../EDAN40-Functional_Programming-Reference_Sheet"
%%% End:


\subsection{Lambda Expressions}\label{subsec:Lambda_Expressions}

%%% Local Variables:
%%% mode: latex
%%% TeX-master: "../../EDAN40-Functional_Programming-Reference_Sheet"
%%% End:


\subsection{Sections}\label{subsec:Sections}

%%% Local Variables:
%%% mode: latex
%%% TeX-master: "../../EDAN40-Functional_Programming-Reference_Sheet"
%%% End:


%%% Local Variables:
%%% mode: latex
%%% TeX-master: "../EDAN40-Functional_Programming-Reference_Sheet"
%%% End:
