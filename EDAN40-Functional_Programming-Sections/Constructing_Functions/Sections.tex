\subsection{Sections}\label{subsec:Sections}
\begin{definition}[Section]\label{def:Section}
  A \emph{section} is a way of writing expressions as infix or prefix with some number of pre-provided arguments.
  In general, if $\oplus$ is an operator, then expressions of the form ($\oplus$), ($x \oplus$), and ($\oplus y$) for arguments $x$ and $y$ are called sections, whose meaning as functions can be formalised using lambda expressions as follows:
  \begin{subequations}\label{eq:Section}
    \begin{equation}\label{subeq:Section_No_Params}
      (\oplus) = \lambda x \rightarrow (\lambda y \rightarrow x \oplus y)
    \end{equation}
    \begin{equation}\label{subeq:Section_First_Param}
      (x \oplus) = \lambda y \rightarrow x \oplus y
    \end{equation}
    \begin{equation}\label{subeq:Section_Second_Param}
      (\oplus y) = \lambda x \rightarrow x \oplus y
    \end{equation}
  \end{subequations}
\end{definition}

\nameref{def:Section}s have 3 main applications:
\begin{enumerate}[noitemsep]
\item They can be used to construct a number of simple but useful functions in a particularly compact way, as shown in the following examples:
  \begin{itemize}[noitemsep]
  \item $(+)$ is the addition function $\lambda x \rightarrow (\lambda y \rightarrow x + y )$
  \item $(1+)$ is the successor function $\lambda y \rightarrow 1 + y$
  \item $(1/)$ is the reciprocation function $\lambda y \rightarrow 1 / y$
  \item $(*2)$ is the doubling function $\lambda x \rightarrow x * 2$
  \item $(/2)$ is the halving function $\lambda x \rightarrow x / 2$
\end{itemize}
\item \nameref{def:Section}s are necessary when stating the type of operators, because an operator itself is not a valid expression in Haskell
\item \nameref{def:Section}s are also necessary when using operators as arguments to other functions.
\end{enumerate}

%%% Local Variables:
%%% mode: latex
%%% TeX-master: "../../EDAN40-Functional_Programming-Reference_Sheet"
%%% End:
