\subsection{Lambda Expressions}\label{subsec:Lambda_Expressions}
\begin{definition}[Lambda Expression]\label{def:Lambda_Expression}
  \emph{Lambda expression}s are an alternative to defining functions using equations.
  Lambda expressions are made using:
  \begin{itemize}[noitemsep]
  \item A pattern for each of the arguments.
  \item A body that specifies how the result can be calculated in terms of the arguments.
  \item But do not give a name for the function itself.
  \end{itemize}

  In other words, lambda expressions are nameless functions.
\end{definition}

The use of \nameref{def:Lambda_Expression}s comes from the invention of \nameref{def:Lambda_Calculus}.
These are typically represented with the lower-case Greek letter $\lambda$ on paper.

In Haskell, \nameref{def:Lambda_Expression}s are written as seen in \Cref{lst:Lambda_Expression}.
\begin{listing}[h!tbp]
\haskellsourcefile{./EDAN40-Functional_Programming-Sections/Constructing_Functions/Code/Lambda_Expressions.hs}
\caption{Lambda Expressions in Haskell}
\label{lst:Lambda_Expression}
\end{listing}

\nameref{def:Lambda_Expression}s are useful for several reasons.
\begin{enumerate}[noitemsep]
\item They can be used to formalise the meaning of curried function definitions.
\item They are useful when defining functions that return functions as results by their very nature, rather than as a consequence of currying.
\item Can be used to avoid having to name a function that is only referenced once.
\end{enumerate}

%%% Local Variables:
%%% mode: latex
%%% TeX-master: "../../EDAN40-Functional_Programming-Reference_Sheet"
%%% End:
