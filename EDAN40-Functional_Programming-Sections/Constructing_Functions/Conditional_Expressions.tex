\subsection{Conditional Expressions}\label{subsec:Conditional_Expressions}
\begin{definition}[Conditional Expression]\label{def:Conditional_Expression}
  A \emph{conditional expression} is one that chooses a path of execution based on some predicate/condition.
  In most languages, this is shown with the \texttt{if-then-else} structures.
\end{definition}

Because in Haskell, we have \nameref{def:Conditional_Expression}s, rather than a conditional statement, there must be a type for the expression.
To ensure that these conditional expressions can be typechecked:
\begin{blackbox}
  \large{\textbf{All possible options MUST have the same type.}}
\end{blackbox}

So, the first function in \Cref{lst:Example_Conditional_Expression} would \textbf{NOT} be compilable, because of a type error.
The second would compile.
\begin{listing}[h!tbp]
\haskellsourcefile{./EDAN40-Functional_Programming-Sections/Constructing_Functions/Code/Conditional_Expression.hs}
\end{listing}

\begin{blackbox}
  In addition, \textbf{every} \texttt{if} \textbf{MUST} have a paired \texttt{else}.
\end{blackbox}

%%% Local Variables:
%%% mode: latex
%%% TeX-master: "../../EDAN40-Functional_Programming-Reference_Sheet"
%%% End:
