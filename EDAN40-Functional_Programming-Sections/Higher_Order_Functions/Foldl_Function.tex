\subsection{The \texorpdfstring{\haskellinline{foldl}}{\texttt{foldl}} Function}\label{subsec:Foldl_Function}
In addition to being able to define operations that are assumed to be right-associative, we can write functions that are assumed to be left-associative with the \haskellinline{foldl} function.
The general basis for which the \haskellinline{foldl} function was written is shown in \Cref{lst:Foldl_Basis}.

\begin{listing}[h!tbp]
\begin{haskellsource}
f v [] = v
f v (x:xs) = f (v ? x) xs
-- Where ? is some operator
\end{haskellsource}
\caption{The Basis of Defining the \haskellinline{foldl} Function}
\label{lst:Foldl_Basis}
\end{listing}

The function maps the empty list to the accumulator value v.
Any non-empty list is mapped to the result of recursively processing the tail using a new accumulator value obtained by applying an operator \texttt{?} to the current value and the head of the list.

Many of the operations that \nameref{subsec:Foldr_Function} defines can also be defined using \haskellinline{foldl}.
When this is the case, the only way to choose which to use is based on the efficiency of the function itself.
This is dependent on Haskell's underlying mechanisms, which is discussed later.
For now, both of them work equally well.

In practice, just as with \haskellinline{foldr} it is best to think of the behaviour of \haskellinline{foldl} in a non-recursive manner.
It is better to think of \haskellinline{foldl}, in terms of an operator $\oplus$ (Was \texttt{?} earlier. The actual symbol doesn't matter.) that is assumed to associate to the left, as summarised by \Cref{eq:General_foldl_Operation}.

\begin{equation}\label{eq:General_foldl_Operation}
  foldl (\oplus) v [x_{0}, x_{1}, x_{2}, \ldots, x_{n}] = (\cdots ((v \oplus x_{0}) \oplus x_{1}) \cdots) \oplus x_{n}
\end{equation}

%%% Local Variables:
%%% mode: latex
%%% TeX-master: "../../EDAN40-Functional_Programming-Reference_Sheet"
%%% End:
