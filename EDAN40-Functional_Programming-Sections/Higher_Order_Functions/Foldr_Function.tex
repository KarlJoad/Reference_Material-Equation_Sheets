\subsection{The \texorpdfstring{\haskellinline{foldr}}{\texttt{foldr}} Function}\label{subsec:Foldr_Function}
Many functions that have a list as an argument can be defined by the recursion pattern in \Cref{lst:Foldr_Basis}.

\begin{listing}[h!tbp]
\begin{haskellsource}
f [] = v
f (x:xs) = x ? f x
-- ? is SOME operator that is applied to the ehad of the list and the result of
-- Recursively processing the tail of the list.
\end{haskellsource}
\caption{The Basis for Defining the \haskellinline{foldr} Function}
\label{lst:Foldr_Basis}
\end{listing}

\haskellinline{foldr} is an abbreviation for ``fold right''.
What this really means is that the operator being applied is assumed to be right-associating.
In practice, however, it is best to think of the behaviour of \haskellinline{foldr f v} in a non-recursive manner.
Instead, \haskellinline{foldr} replaces every \haskellinline{cons} in a list by the function \texttt{f}, and the empty list at the end by the value \texttt{v}.
So, for example \haskellinline{foldr (+) 0 [1, 2, 3]} becomes $(1+(2+(3+0)))$ (The 0 is for the empty list).

The definition of \haskellinline{foldr}, like many other \nameref{def:Higher_Order_Function}s, is recursive.

\begin{listing}[h!tbp]
\haskellsourcefile{./EDAN40-Functional_Programming-Sections/Higher_Order_Functions/Code/foldr_Definition.hs}
\caption{The Basis for Defining the \haskellinline{foldr} Function}
\label{lst:Foldr_Basis}
\end{listing}

The general operation of the \haskellinline{foldr} function is shown in \Cref{eq:General_foldr_Operation}.
\begin{equation}\label{eq:General_foldr_Operation}
  foldr (\oplus) v [x_{0}, x_{1}, x_{2}, \ldots, x_{n}] = x_{0} \oplus (x_{1} \oplus (x_{2} \oplus (\cdots (x_{n} \oplus v) \cdots )))
\end{equation}

%%% Local Variables:
%%% mode: latex
%%% TeX-master: "../../EDAN40-Functional_Programming-Reference_Sheet"
%%% End:
