\section{Monads}\label{sec:Monads}
So far, we have only written Haskell programs as \nameref{def:Batch_Program}s.
However, most people today like \nameref{def:Interactive_Program}s.

\begin{definition}[Batch Program]\label{def:Batch_Program}
  A \emph{batch program} is a program that takes some input from the user at the beginning of program execution and returns some result afterwards.
  During the computations, there is no further interaction between user and program.
\end{definition}

\begin{definition}[Interactive Program]\label{def:Interactive_Program}
  An \emph{interactive program} is a program that may take some input from the user at the beginning of execution, and may return some result afterwards, \textbf{AND} will ask for user interaction during its life.
  This means the user can provide new data to program during its execution.
\end{definition}

By the very nature of \nameref{def:Interactive_Program}s, side effects are needed.
These side effects and their functionality are handled by \nameref{def:Monad}s in Haskell.

\begin{definition}[Monad]\label{def:Monad}
  \emph{Monads} are fairly unique to Haskell.
  They represent a way to support side-effect creating operations in the purely functional language.

  One example of a monad is the \texttt{IO} typeclass.
  In most type signatures, monadic elements are indicated with an \texttt{m}.
\end{definition}

We generalise the type for interactive programs to also return a result value, with the type of such values being a parameter of the \texttt{IO} (in this case) type:
\begin{haskellsource}
type IO a = World -> (a, World)
\end{haskellsource}

Expressions of type \haskellinline{IO a} are called actions.
For example, \haskellinline{IO Char} is the type of actions that return a character, while \haskellinline{IO ()} is the type of actions that return the empty tuple \texttt{()} as a dummy result value.

\subsection{Basic \texorpdfstring{\haskellinline{IO}}{\texttt{IO}} Actions}\label{subsec:Basic_IO_Actions}
There are 3 main actions in the \texttt{IO} monadic typeclass that are useful for our use.
\begin{nocrefenumerate}
\item \haskellinline{getChar}\label{act:IO_getChar}
\item \haskellinline{putChar}\label{act:IO_putChar}
\item \haskellinline{return}\label{act:IO_return}
\end{nocrefenumerate}

The \texttt{getChar} action, (Action~\ref{act:IO_getChar}) reads a character from the keyboard, echoes it to the screen, and returns the character as its result value.
If there are no characters waiting to be read from the keyboard, \texttt{getChar} waits until one is typed.
\begin{listing}[h!tbp]
\begin{haskellsource}
getChar :: IO Char
getChar = ...
\end{haskellsource}
\caption{\texttt{getChar} Type Signature}
\label{lst:IO_getChar_Type}
\end{listing}
The actual definition for \texttt{getChar} is built-in to the particular Haskell system you are using, and cannot be defined within Haskell itself.

The \texttt{putChar c} action, (Action~\ref{act:IO_putChar}) writes the character \texttt{c} to the screen, and returns no result value, represented by the empty tuple.
\begin{listing}[h!tbp]
\begin{haskellsource}
putChar :: Char -> IO ()
putChar c = ...
\end{haskellsource}
\caption{\texttt{putChar} Type Signature}
\label{lst:IO_putChar_Type}
\end{listing}
The actual definition for \texttt{putChar} is built-in to the particular Haskell system you are using, and cannot be defined within Haskell itself.

The \texttt{return v} action, (Action~\ref{act:IO_return}) simply returns the result value \texttt{v} without performing any interaction.
\begin{listing}[h!tbp]
\begin{haskellsource}
return :: a -> IO a
return v = \world -> (v, world)
\end{haskellsource}
\caption{\texttt{return} Type Signature}
\label{lst:IO_return_Type}
\end{listing}
\texttt{return} provides a bridge from the setting of pure expressions without side effects to that of impure actions with side effects.


%%% Local Variables:
%%% mode: latex
%%% TeX-master: "../../EDAN40-Functional_Programming-Reference_Sheet"
%%% End:


\subsection{Sequencing}\label{subsec:Sequencing}
The natural way of combining two actions is to perform one after the other in sequence, with the modified world returned by the first action becoming the current world for the second, by means of a sequencing operator, read as ``then''.
It is defined in \Cref{lst:Then_Operator_Definition}.

\begin{listing}[h!tbp]
\begin{haskellsource}
(>>=) :: m a -> (a -> m b) -> m b
f >>= g = \world -> case f world of
                         (v, world') -> g v world'
\end{haskellsource}
\caption{\haskellinline{>>=} Operator Definition}
\label{lst:Then_Operator_Definition}
\end{listing}

In words, we apply the action \texttt{f} to the current world, then apply the function \texttt{g} to the result value and the modified world as a second action, yielding the final result.


%%% Local Variables:
%%% mode: latex
%%% TeX-master: "../../EDAN40-Functional_Programming-Reference_Sheet"
%%% End:


%%% Local Variables:
%%% mode: latex
%%% TeX-master: "../EDAN40-Functional_Programming-Reference_Sheet"
%%% End:
