\section{Monads}\label{sec:Monads}
So far, we have only written Haskell programs as \nameref{def:Batch_Program}s.
However, most people today like \nameref{def:Interactive_Program}s.

\begin{definition}[Batch Program]\label{def:Batch_Program}
  A \emph{batch program} is a program that takes some input from the user at the beginning of program execution and returns some result afterwards.
  During the computations, there is no further interaction between user and program.
\end{definition}

\begin{definition}[Interactive Program]\label{def:Interactive_Program}
  An \emph{interactive program} is a program that may take some input from the user at the beginning of execution, and may return some result afterwards, \textbf{AND} will ask for user interaction during its life.
  This means the user can provide new data to program during its execution.
\end{definition}

By the very nature of \nameref{def:Interactive_Program}s, side effects are needed.
These side effects and their functionality are handled by \nameref{def:Monad}s in Haskell.

\begin{definition}[Monad]\label{def:Monad}
  \emph{Monads} are fairly unique to Haskell.
  They represent a way to support side-effect creating operations in the purely functional language.

  One example of a monad is the \texttt{IO} typeclass.
  In most type signatures, monadic elements are indicated with an \texttt{m}.
\end{definition}

We generalise the type for interactive programs to also return a result value, with the type of such values being a parameter of the \texttt{IO} (in this case) type:
\begin{haskellsource}
type IO a = World -> (a, World)
\end{haskellsource}

Expressions of type \haskellinline{IO a} are called actions.
For example, \haskellinline{IO Char} is the type of actions that return a character, while \haskellinline{IO ()} is the type of actions that return the empty tuple \texttt{()} as a dummy result value.

%%% Local Variables:
%%% mode: latex
%%% TeX-master: "../EDAN40-Functional_Programming-Reference_Sheet"
%%% End:
