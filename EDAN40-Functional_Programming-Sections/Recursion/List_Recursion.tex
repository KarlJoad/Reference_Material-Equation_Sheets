\subsection{List Recursion}\label{subsec:List_Recursion}
Lists, being potentially infinite, and composed of a single element and the rest of the list (see \nameref{par:List_Cons_Function}, \Cref{par:List_Cons_Function}), make for natural targets of \nameref{def:Recursion}.

In every case of list recursion, you need to identify the base case.
Typically, this is the empty list.
If you wanted to find the product of a list of numbers, you could use the recursive definition shown in \Cref{lst:List_Recursion-Product}.
\begin{listing}[h!tbp]
\haskellsourcefile{./EDAN40-Functional_Programming-Sections/Recursion/Code/List_Recursion-Product.hs}
\caption{Product of a List}
\label{lst:List_Recursion-Product}
\end{listing}

In addition, you can use the \texttt{\_} wildcard in your list recursion as well, if you don't actually care what that particular element's value is.
\begin{listing}[h!tbp]
\haskellsourcefile{./EDAN40-Functional_Programming-Sections/Recursion/Code/List_Recursion-Length.hs}
\caption{Length of a List}
\label{lst:List_Recursion-Length}
\end{listing}


%%% Local Variables:
%%% mode: latex
%%% TeX-master: "../../EDAN40-Functional_Programming-Reference_Sheet"
%%% End:
