\subsection{Mutual Recursion}\label{subsec:Mutual_Recursion}
\begin{definition}[Mutual Recursion]\label{def:Mutual_Recursion}
  \emph{Mutual recursion} is when a set of two or more functions are defined in terms of each other.

  \begin{remark}[Mutually Recursive]\label{rmk:Mutually_Recursive}
    A set of functions that use \nameref{def:Mutual_Recursion} are said to be \emph{mutually recursive}.
  \end{remark}
\end{definition}

The easiest mutually recursive functions to visualize are the \haskellinline{odd} and \haskellinline{even} functions.
As a side note, these functions are not actually implemented this way, for efficiency's sake.
\begin{listing}[h!tbp]
\haskellsourcefile{./EDAN40-Functional_Programming-Sections/Recursion/Code/Even_Odd.hs}
\caption{Mutually Recursive \haskellinline{odd} and \haskellinline{even} Functions}
\label{lst:Mutually_Recursive-Even_Odd}
\end{listing}

%%% Local Variables:
%%% mode: latex
%%% TeX-master: "../../EDAN40-Functional_Programming-Reference_Sheet"
%%% End:
