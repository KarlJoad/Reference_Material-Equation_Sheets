\subsection{Multiple Recursion}\label{subsec:Multiple_Recursion}
\begin{definition}[Multiple Recursion]\label{def:Multiple_Recursion}
  \emph{Multiple recursion} is when a functions is applied more than once in its own definition.

  \begin{remark}[Multiply Recursive]\label{rmk:Multiply_Recursive}
    A function that uses \nameref{def:Multiple_Recursion} is said to be \emph{multiply recursive}.
  \end{remark}
\end{definition}

The first multiply recursive function most people write is a function to calculate the Fibonacci sequence.
\begin{listing}[h!tbp]
\haskellsourcefile{./EDAN40-Functional_Programming-Sections/Recursion/Code/Fibonacci.hs}
\caption{Multiple Recursion in Fibonacci's Sequence}
\label{lst:Multiple_Recursion-Fibonacci}
\end{listing}

Multiply recursive functions may require more than one base case to terminate the recursion, depending on how the function works.

%%% Local Variables:
%%% mode: latex
%%% TeX-master: "../../EDAN40-Functional_Programming-Reference_Sheet"
%%% End:
