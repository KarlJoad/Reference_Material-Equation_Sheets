\documentclass[10pt,letterpaper,final,twoside,notitlepage]{article}
\usepackage[margin=.5in]{geometry} % 1/2 inch margins on all pages
\usepackage[utf8]{inputenc} % Define the input encoding
\usepackage[USenglish]{babel} % Define language used
\usepackage{amsmath,amsfonts,amssymb}
\usepackage{amsthm} % Gives us plain, definition, and remark to use in \theoremstyle{style}
\usepackage{mathtools} % Allow for text and math in align* environment.
\usepackage{thmtools}
\usepackage{thm-restate}
\usepackage{graphicx}

\usepackage[
backend=biber,
style=alphabetic,
citestyle=authoryear]{biblatex} % Must include citation somewhere in document to print bibliography
\usepackage{hyperref} % Generate hyperlinks to referenced items
\usepackage[nottoc]{tocbibind} % Prints the Reference/Bibliography in TOC as well
\usepackage[noabbrev,nameinlink]{cleveref} % Fancy cross-references in the document everywhere
\usepackage{nameref} % Can make references by name to places
\usepackage{caption} % Allows for greater control over captions in figure, algorithm, table, etc. environments
\usepackage{subcaption} % Allows for multiple figures in one Figure environment
\usepackage[binary-units=true]{siunitx} % Gives us ways to typeset units for stuff
\usepackage{csquotes} % Context-sensitive quotation facilities
\usepackage{enumitem} % Provides [noitemsep, nolistsep] for more compact lists
\usepackage{chngcntr} % Allows us to tamper with the counter a little more
\usepackage{empheq} % Allow boxing of equations in special math environments
\usepackage[x11names]{xcolor} % Gives access to coloring text in environments or just text, MUST be before tikz
\usepackage{tcolorbox} % Allows us to create boxes of various types for examples
\usepackage{tikz} % Allows us to create TikZ and PGF Pictures
\usepackage{ctable} % Greater control over tables and how they look
\usepackage{diagbox} % Allow us to have shared diagonal cells in tables
\usepackage{multirow} % Allow us to have a single cell in a table span multiple rows
\usepackage{titling} % Put document information throughout the document programmatically
\usepackage[linesnumbered,ruled,vlined]{algorithm2e} % Allows us to write algorithms in a nice style.

\counterwithin{figure}{section}
\counterwithin{table}{section}
\counterwithin{equation}{section}
\counterwithin{algocf}{section}
\crefname{algocf}{algorithm}{algorithms}
\Crefname{algocf}{Algorithm}{Algorithms}
\setcounter{secnumdepth}{4}
\setcounter{tocdepth}{4} % Include \paragraph in toc
\crefname{paragraph}{paragraph}{paragraphs}
\Crefname{paragraph}{Paragraph}{Paragraphs}

% Create a theorem environment
\theoremstyle{plain}
\newtheorem{theorem}{Theorem}[section]
% Create a numbered theorem-like environment for lemmas
\newtheorem{lemma}{Lemma}[theorem]

% Create a definition environment
\theoremstyle{definition}
\newtheorem{definition}{Defn}
\newtheorem{corollary}{Corollary}[section]
% \begin{definition}[Term] \label{def:}
%   Make sure the term is emphasized with \emph{term}.
%   This ensures that if \emph is changed, it shows up everywhere
% \end{definition}

% Create a numbered remark environment numbered based on definition
% NOTE: This version of remark MUST go inside a definition environment
\theoremstyle{remark}
\newtheorem{remark}{Remark}[definition]
%\counterwithin{definition}{subsection} % Uncomment to have definitions use section.subsection numbering

% Create an unnumbered remark environment for general use
% NOTE: This version of remark has NO restrictions on placement
\newtheorem*{remark*}{Remark}

% Create a special list that handles properties. It can be broken and restarted
\newlist{propertylist}{enumerate}{1} % {Name}{Template}{Max Depth}
% [newlistname, LevelsToApplyTo]{formatting options}
\setlist[propertylist, 1]{label=\textbf{(\roman*)}, ref=\textbf{(\roman*)}, noitemsep, nolistsep}
\crefname{propertylisti}{property}{properties}
\Crefname{propertylisti}{Property}{Properties}

% Create a special list that handles enumerate starting with lower letters. Breakable/Restartable.
\newlist{boldalphlist}{enumerate}{1} % {Name}{Template}{Max Depth}
% [newlistname, LevelsToApplyTo]{formatting options}
\setlist[boldalphlist, 1]{label=\textbf{(\alph*)}, ref=\alph*, noitemsep, nolistsep} % Set options

\newlist{nocrefenumerate}{enumerate}{1} % {Name}{Template}{Max Depth}
% [newlistname, LevelsToApplyTo]{formatting options}
\setlist[nocrefenumerate, 1]{label=(\arabic*), ref=(\arabic*), noitemsep, nolistsep}

% Create a list that allows for deeper nesting of numbers. By default enumerate only allows depth=4.
\newlist{nestednums}{enumerate}{6}
% [newlistname, LevelsToApplyTo]{formatting options}
\setlist[nestednums]{noitemsep, label*=\arabic*.}

\tcbuselibrary{breakable} % Allow tcolorboxes to be broken across pages
% Create a tcolorbox for examples
% /begin{example}[extra name]{NAME}
% Create a tcolorbox for examples
% Argument #1 is optional, given by [], that is the textbook's problem number
% Argument #2 is mandatory, given by {}, that is the title for the example
% Avoid putting special characters, (), [], {}, ",", etc. in the title.
\newtcolorbox[auto counter,
number within=section,
number format=\arabic,
crefname={example}{examples}, % Define reference format for cref (No Capitals)
Crefname={Example}{Examples}, % Reference format for cleveref (With Capitals)
]{example}[2][]{ % The [2][] Means the first argument is optional
  width=\textwidth,
  title={Example \thetcbcounter: #2. #1}, % Parentheses and commas are not well supported
  fonttitle=\bfseries,
  label={ex:#2},
  nameref=#2,
  colbacktitle=white!100!black,
  coltitle=black!100!white,
  colback=white!100!black,
  upperbox=visible,
  lowerbox=visible,
  sharp corners=all,
  breakable
}

% Create a tcolorbox for general use
\newtcolorbox[% auto counter,
% number within=section,
% number format=\arabic,
% crefname={example}{examples}, % Define reference format for cref (No Capitals)
% Crefname={Example}{Examples}, % Reference format for cleveref (With Capitals)
]{blackbox}{
  width=\textwidth,
  % title={},
  fonttitle=\bfseries,
  % label={},
  % nameref=,
  colbacktitle=white!100!black,
  coltitle=black!100!white,
  colback=white!100!black,
  upperbox=visible,
  lowerbox=visible,
  sharp corners=all
}

% Redefine the 'end of proof' symbol to be a black square, not blank
\renewcommand{\qedsymbol}{$\blacksquare$} % Change proofs to have black square at end

% Common Mathematical Stuff
\newcommand{\Abs}[1]{\ensuremath{\lvert #1 \rvert}}
\newcommand{\DNE}{\ensuremath{\mathrm{DNE}}} % Used when limit of function Does Not Exist

% Complex Numbers functions
\renewcommand{\Re}{\operatorname{Re}} % Redefine to use the command, but not the fraktur version
\renewcommand{\Im}{\operatorname{Im}} % Redefine to use the command, but not the fraktur version
\newcommand{\Real}[1]{\ensuremath{\Re \lbrace #1 \rbrace}}
\newcommand{\Imag}[1]{\ensuremath{\Im \lbrace #1 \rbrace}}
\newcommand{\Conjugate}[1]{\ensuremath{\overline{#1}}}
\newcommand{\Modulus}[1]{\ensuremath{\lvert #1 \rvert}}
\DeclareMathOperator{\PrincipalArg}{\ensuremath{Arg}}

% Math Operators that are useful to abstract the written math away to one spot
% Number Sets
\DeclareMathOperator{\RealNumbers}{\ensuremath{\mathbb{R}}}
\DeclareMathOperator{\AllIntegers}{\ensuremath{\mathbb{Z}}}
\DeclareMathOperator{\PositiveInts}{\ensuremath{\mathbb{Z}^{+}}}
\DeclareMathOperator{\NegativeInts}{\ensuremath{\mathbb{Z}^{-}}}
\DeclareMathOperator{\NaturalNumbers}{\ensuremath{\mathbb{N}}}
\DeclareMathOperator{\ComplexNumbers}{\ensuremath{\mathbb{C}}}
\DeclareMathOperator{\RationalNumbers}{\ensuremath{\mathbb{Q}}}

% Calculus operators
\DeclareMathOperator*{\argmax}{argmax} % Thin Space and subscripts are UNDER in display

% Signal and System Functions
\DeclareMathOperator{\UnitStep}{\mathcal{U}}
\DeclareMathOperator{\sinc}{sinc} % sinc(x) = (sin(pi x)/(pi x))

% Transformations
\DeclareMathOperator{\Lapl}{\mathcal{L}} % Declare a Laplace symbol to be used

% Logical Operators
\DeclareMathOperator{\XOR}{\oplus}

% x86 CPU Registers
\newcommand{\rbpRegister}{\texttt{\%rbp}}
\newcommand{\rspRegister}{\texttt{\%rsp}}
\newcommand{\ripRegister}{\texttt{\%rip}}
\newcommand{\raxRegister}{\texttt{\%rax}}
\newcommand{\rbxRegister}{\texttt{\%rbx}}

%%% Local Variables:
%%% mode: latex
%%% TeX-master: shared
%%% End:


% These packages are more specific to certain documents, but will be availabe in the template
% \usepackage{esint} % Provides us with more types of integral symbols to use
\usepackage[outputdir=./TeX_Output]{minted} % Allow us to nicely typeset 300+ programming languages
% This document must be compiled with the -shell-escape flag if the packages above are uncommented

% \graphicspath{{./Drawings/EDAP05-Concepts_Programming_Languages}} % Uncomment this to use pictures in this document
\addbibresource{./Bibliographies/EDAP05-Concepts_Programming_Languages.bib}

% Math Operators that are useful to abstract the written math away to one spot
% These are supposed to be document-specific mathematical operators that will make life easier
% Many fundamental operators are defined in Reference_Sheet_Preamble.tex
% These commands can ONLY be used in math mode
\newcommand{\Nonterminal}[1]{\langle \text{#1} \rangle} % This command should be used for text
\newcommand{\MathNonterminal}[1]{\langle #1 \rangle} % This command should be used if there are math things present, subscripts, superscripts, etc.
\newcommand{\SemanticInput}[1]{\textcolor{Green4}{#1}}
\DeclareMathOperator{\EvaluatesTo}{\Downarrow}
\DeclareMathOperator{\DataType}{\textcolor{Blue4}{\tau}}

\begin{titlepage}
  \title{EDAP05: Concepts of Programming Languages - Reference Sheet}
  \author{Karl Hallsby}
  \date{Last Edited: \today} % We want to inform people when this document was last edited
\end{titlepage}

\begin{document}
\pagenumbering{gobble}
\maketitle
\pagenumbering{roman} % i, ii, iii on beginning pages, that don't have content
\tableofcontents
\clearpage
\pagenumbering{arabic} % 1,2,3 on content pages

\nocite{Sebesta2012}

\section{Language vs. Language Implementation}\label{sec:Lang_vs_Lang_Implementation}
It is important that we make the distinction between a programming language and the programming language's implementation.

\begin{itemize}[noitemsep]
\item A programming language and its implementation are completely separate things
  \begin{itemize}[noitemsep]
  \item Technically, they are related, in that a programming language implementation is one way to fulfill the specifications that the programming language introduces
  \item You can implement a programming language in different ways. For example, C has these well-known implementations:
    \begin{itemize}[noitemsep]
    \item gcc
    \item LLVM/clang
    \item MSVC
    \end{itemize}
  \end{itemize}
\end{itemize}

\subsection{Influences on Language Design}\label{subsec:Influences_Language_Design}
The 2 other major influences on the design of programming languages have been:
\begin{enumerate}[noitemsep]
\item \nameref{subsubsec:Computer_Architecture}
\item \nameref{subsubsec:Programming_Design_Methodologies}
\end{enumerate}

\subsubsection{Computer Architecture}\label{subsubsec:Computer_Architecture}
The prevalent computer architecture used is the \nameref{def:von_Neumann_Architecture}.
This is in contrast to the \nameref{def:Harvard_Architecture}, and its descendant \nameref{def:Modified_Harvard_Architecture}.

\begin{definition}[von Neumann Architecture]\label{def:von_Neumann_Architecture}
  In the \emph{von Neumann architecture}, named after John von Neumann, instructions and data are stored in a shared memory location.
  The central processing unit, CPU, is separate from the memory, meaning it must fetch the instructions and data from memory before doing something.
  When the CPU computes something, it needs to store the result \emph{back} in memory.
  The constant fetching of instructions/data and storage of results in memory means there is a bottleneck, the \emph{\nameref{rmk:von_Neumann_Bottleneck}}.

  \begin{remark}[von Neumann Bottleneck]\label{rmk:von_Neumann_Bottleneck}
    The shared bus between the program memory and data memory leads to the \emph{von Neumann bottleneck}, the limited throughput (data transfer rate) between the CPU and memory compared to the amount of memory.
    Because the single bus can only access one of the two classes of memory at a time, throughput is lower than the rate at which the CPU can work.
    This seriously limits the effective processing speed when the CPU is required to perform minimal processing on large amounts of data. The CPU is continually forced to wait for needed data to move to or from memory.

    Since CPU speed and memory size have increased much faster than the throughput between them, the bottleneck has become more of a problem.
  \end{remark}

  The execution of a machine code program on a \nameref{def:von_Neumann_Architecture} computer occurs in a process called the \emph{fetch-execute cycle}.
  To find where each instruction is in memory, the CPU needs to have a \emph{program counter}.

  Functional or applicative programming languages, where applying functions to parameters does not lend itself to the \nameref{def:von_Neumann_Architecture}.

  \begin{remark}[Cache in the \nameref*{def:von_Neumann_Architecture}]
    In the original \nameref{def:von_Neumann_Architecture}, there was no such thing as \emph{cache} on the CPU.\@
    In modern computers, cache is located on the CPU directly, and acts similarly to memory.
    However, it copies a block of memory into the cache and feeds the CPU from that, refreshing the cache less periodically, and allowing for fater instruction/data access rates.
    This is an example of the \nameref{def:Harvard_Architecture} in the traditional \nameref{def:von_Neumann_Architecture} making a \nameref{def:Modified_Harvard_Architecture}.
  \end{remark}

  \begin{remark}[Alternative Names]\label{rmk:von_Neumann_Architecture_Alternative_Names}
    The \nameref{def:von_Neumann_Architecture} can also be called:
    \begin{itemize}[noitemsep]
    \item von Neumann Model
    \item Princeton Architecture
    \item Dataflow Model
    \end{itemize}
  \end{remark}

  \begin{remark}[Alternative Architectures]\label{rmk:von_Neumann_Architecture_Alternatives}
    The \nameref{def:von_Neumann_Architecture} is one way to implement a computational model.
    There are alternatives, namely the \nameref{def:Harvard_Architecture} and its descendant \nameref{def:Modified_Harvard_Architecture}.
  \end{remark}
\end{definition}

\begin{definition}[Harvard Architecture]\label{def:Harvard_Architecture}
  The \emph{Harvard architecture} is a computer architecture with separate storage and signal pathways for instructions and data.
  It contrasts with the \nameref{def:von_Neumann_Architecture}, where program instructions and data share the same memory and pathways.

  This partition of instructions and data means the CPU can simultaneously read an instruction and perform data memory access.
  Additionally, the address space for the instructions and data are separate, meaning instruction address zero is not the same as data address zero.
\end{definition}

\begin{definition}[Modified Harvard Architecture]\label{def:Modified_Harvard_Architecture}
  Most modern computers act as \emph{both} \nameref{def:von_Neumann_Architecture} machines and \nameref{def:Harvard_Architecture} machines.
  These have been called \emph{modified Harvard architecture}s.
  The \emph{modified Harvard architecture} is also a variation of the \nameref{def:Harvard_Architecture} that allows the contents of the instruction memory to be accessed as data.
  The different types of modified Harvard architectures are discussed in \Cref{rmk:Types_Modified_Harvard_Architecture}.

  \begin{remark}[Modern CPU Architecture]\label{rmk:Modern_CPU_Architecture}
    In modern CPUs, with both their system memory and on-chip cache, they act as both \nameref{def:von_Neumann_Architecture} machines and \nameref{def:Harvard_Architecture} machines.
    The CPU acts as:
    \begin{itemize}[noitemsep]
    \item A \nameref{def:Harvard_Architecture} machine when the CPU is accessing its on-chip cache.
    \item A \nameref{def:von_Neumann_Architecture} machine when the CPU is accessing the system memory.
    \end{itemize}
  \end{remark}

  \begin{remark}[Types of Modified Harvard Architectures]\label{rmk:Types_Modified_Harvard_Architecture}
    There are many different types of \nameref{def:Modified_Harvard_Architecture}s.
    Some of the major ones are discussed here:
    \begin{itemize}[noitemsep]
    \item Split-cache (or almost-\nameref{def:von_Neumann_Architecture} architecture)
      \begin{itemize}[noitemsep]
      \item The most common modification builds a memory hierarchy with a CPU cache separating instructions and data.
      \item This unifies all except small portions of the data and instruction address spaces, providing the von Neumann model.
      \end{itemize}
    \item Instruction-Memory-as-Data Architecture
      \begin{itemize}[noitemsep]
      \item Another change preserves the ``separate address space'' nature of a \nameref{def:Harvard_Architecture} machine, but provides special machine operations to access the contents of the instruction memory as data.
      \item Because data is not directly executable as instructions, there are 2 different operations possible:
        \begin{enumerate}[noitemsep]
        \item Read access: initial data values can be copied from the instruction memory into the data memory when the program starts.
          Or, if the data is not to be modified (it might be a constant value, such as pi, or a text string), it can be accessed by the running program directly from instruction memory without taking up space in data memory (which is often at a premium).
        \item Write access: a capability for reprogramming is generally required; few computers are purely ROM-based.
          For example, a microcontroller usually has operations to write to the flash memory used to hold its instructions.
          This capability may be used for purposes including software updates. EEPROM/PROM replacement is an alternative method.
        \end{enumerate}
      \end{itemize}
    \item Data-Memory-as-Instruction Architecture
      \begin{itemize}[noitemsep]
      \item A few \nameref{def:Harvard_Architecture} processors can execute instructions fetched from any memory segment
      \item Unlike the original Harvard processor, which can only execute instructions fetched from the program memory segment.
      \item Such processors, like other \nameref{def:Harvard_Architecture} processors, and unlike pure \nameref{def:von_Neumann_Architecture}, can read an instruction and read a data value simultaneously, \textbf{if they're in separate memory segments}, since the processor has (at least) two separate memory segments with independent data buses.
      \item The most obvious programmer-visible difference between this kind of modified Harvard architecture and a pure von Neumann architecture is that – when executing an instruction from one memory segment – the same memory segment cannot be simultaneously accessed as data.
      \end{itemize}
    \end{itemize}
  \end{remark}
\end{definition}

The \nameref{def:von_Neumann_Architecture} models variables incredibly well, as memory cells, assignment statements as the writing of data back to memory, and iteration.
In fact, the \nameref{def:von_Neumann_Architecture} models iteration so well, that it encourages iteration over recursion (when possible), sometimes at the detriment of the overall program.

\subsubsection{Programming Design Methodologies}\label{subsubsec:Programming_Design_Methodologies}
Starting in the 1960s, bigger and more complicated programs were being written for more complicated things (controlling whole facilities, worldwide airline reservation systems, etc.).
New software development methodologies appeared, and a shift from procedure-oriented to data-oriented design methodologies emerged.

Data-oriented models emphasize:
\begin{itemize}[noitemsep]
\item Data design
\item Abstract data types to solve problems
\end{itemize}

This data-oriented design led to the to development of object-oriented design.
%%% Local Variables:
%%% mode: latex
%%% TeX-master: "../EDAP05-Concepts_Programming_Languages-Reference_Sheet"
%%% End:


\section{Programming Language Implementations}\label{sec:Lang_Implementations}
There are 3 main ways for a programming language to be implemented:
\begin{enumerate}[noitemsep]
\item \nameref{subsec:Interpretation}
\item \nameref{subsec:Compilation}
\item \nameref{subsec:Hybrid_Implementation}
\end{enumerate}

There are benefits and drawbacks for each of these implementations:
\begin{table}[h!]
  \centering
  \begin{tabular}{cccc}
    \toprule
    Property & \nameref{subsec:Interpretation} & \nameref{subsec:Compilation} & \nameref{subsec:Hybrid_Implementation} \\
    \midrule
    Execution Performance & Slow & Fast & Fast \\
    Turnaround & Fast & Slow (Compile and Link) & Fast (Compile when needed) \\
    Language Flexibility & High & Limited & High \\
    \bottomrule
  \end{tabular}
  \caption{Pros and Cons for Programming Language Implementations}
  \label{tab:Lang_Implementations_Pros_Cons}
\end{table}

There is a trade-off to be made between:
\begin{itemize}[noitemsep]
\item Language flexibility
\item CPU time / RAM time
\end{itemize}

\subsection{Interpretation}\label{subsec:Interpretation}
\begin{definition}[Interpretation]\label{def:Interpretation}
  If a programming language is implemented with \emph{interpretation}, is \emph{interpreted}, then there is an intermediate program that runs between the source code and what the CPU can run on.
  When a programming language is interpreted, there is \textbf{no} translation of the high-level source language to anything else.
  The \emph{interpreter} uses the high-level source code directly.
  This \emph{interpreter} reads the high-level source code, then alternates between:
  \begin{itemize}[noitemsep]
  \item Figure out next command
    \begin{itemize}[noitemsep]
    \item This means that the current instruction is parsed in
    \item Equivalent commands are generated in the CPU-specific or VM-specific instruction sets from the high-level source code
    \end{itemize}
  \item Execute Command
  \end{itemize}

  Interpretation allows for easy implementation of source-level debugging.
  Meaning when semantic analysis occurs on the program while running, the errors are returned in a fashion that makes sense with relation to the high-level language.
  For instance, if there is an array index error, the error could refer to the index itself, the name of the array, and its line.

  \begin{remark}[Interpretation Drawbacks]\label{rmk:Interpretation_Drawbacks}
    There is roughly a 10--100 times performance slowdown.
    The main bottleneck in an interpreted language is the instruction decoding from the high-level source to something the interprett can use.
    This is because \textit{every} instruction must be decoded \textit{every} time.

    Additionally, interpreted programs take up more space on disk in a form not designed to be space efficient.
    In memory, interpreted programs take up more space because the symbol table and interpreter must be in memory at the same time to make the program run.
  \end{remark}
\end{definition}

Some examples of languages with an \nameref{def:Interpretation} implementation are:
\begin{itemize}[noitemsep]
\item Python
\item Perl
\item Ruby
\item Bash
\item AWK
\item $\cdots$
\end{itemize}

\subsection{Compilation}\label{subsec:Compilation}
\begin{definition}[Compilation]\label{def:Compilation}
  If a programming language is implemented with \emph{compilation}, is \emph{compiled}, then there are several programs that must be run before the high-level source code can be run.
  \begin{enumerate}[noitemsep]
  \item The \nameref{def:Compiler}
  \item The \nameref{def:Assembler}
  \item The \nameref{def:Linker}
  \item The \nameref{def:Loader}
  \end{enumerate}
\end{definition}

\begin{definition}[Compiler]\label{def:Compiler}
  The \emph{compiler} is the main program needed in a compiled language implementation.
  It is responsible for taking the high-level source code written in some language, and converting it to assembly code, which can then be run through an \nameref{def:Assembler}.

  The steps involved in a compiler are:
  \begin{enumerate}[noitemsep]
  \item Lexical Analysis/Tokenizing: Convert the text in the input file into a set of tokens
  \item Syntactic Analysis/Parsing: Convert the tokens into a parse tree representing all the tokens in the program in a hierarchical and prioritative manner
  \item Semantic Analysis: ``Interpret'' the program and ensure that everything expressed in the program is correct.
    \begin{itemize}[noitemsep]
    \item This is where compile-time errors are \textbf{usually} caught. Though, this is just a generalization.
    \item Type analysis is typically handled here for instance
    \end{itemize}
    
  \item Optimize the Code: The output assembly code could be optimized before actually making the output. Take care of that here.
  \item Output Assembly: With the potentially optimized machine-equivalent code from our program, write out the equivalent assembly, and finish the compilation process.
  \end{enumerate}

  \begin{remark}
    The specifics of a \nameref{def:Compiler}'s implementation are \textbf{not} discussed in this course, but it is useful to know the basics of the compilation process.
    For both the implementation details, please refer to \href{run:./EDAN65-Compilers-Reference_Sheet.pdf}{EDAN65:Compilers-Reference Material}.
  \end{remark}
\end{definition}

\begin{definition}[Assembler]\label{def:Assembler}
  The \emph{assembler} is an intermediate program used after the \nameref{def:Compiler} has been run.
  The assembler takes the assembly code that the \nameref{def:Compiler} outputs and applies a one-to-one mapping.
  Since all assembly code is just an abstraction and humanization of machine code in a one-to-one mapping fashion, the assembler takes the assembly code and converts it to its equivalent machine code.

  \begin{remark}
    This particular program is not discussed heavily in this course.
  \end{remark}
\end{definition}

\begin{definition}[Linker]\label{def:Linker}
  The \emph{linker} is an intermediate program, that may be provided by the operating system or may be provided by that language implementation's tooling.
  It is run after the \nameref{def:Compiler} and/or the \nameref{def:Assembler} have been run.
  \begin{itemize}[noitemsep]
  \item Provided by operating system
    \begin{itemize}[noitemsep]
    \item If the programming language implementation relies on the operating system and critical portions of the system.
    \end{itemize}
  \item Provided by the language implementation's tooling
    \begin{itemize}[noitemsep]
    \item If the implementation provides certain libraries, it will likely have their own linker too.
    \end{itemize}
  \end{itemize}

  \begin{remark}
    This particular program is not discussed in this course.
  \end{remark}
\end{definition}

\begin{definition}[Loader]\label{def:Loader}
  The \emph{loader} is the program provided by the operating system that loads the specified program into main memory and begins execution.
  
  \begin{remark}
    This particular program is not discussed in this course.
  \end{remark}
\end{definition}

Some examples of languages with a \nameref{def:Compilation} implementation are:
\begin{itemize}[noitemsep]
\item C
\item C++
\item SML
\item Haskell
\item FORTRAN
\item $\cdots$
\end{itemize}

\subsection{Hybrid Implementation}\label{subsec:Hybrid_Implementation}
\begin{definition}[Hybrid Implementation]\label{def:Hybrid_Implementation}
  A programming language can be implemented with a \emph{hybrid implementation}.
  This means that it takes some aspects of a language implemented by \nameref{def:Interpretation} and some aspects of the language implemented with \nameref{def:Compilation}.

  Typically what happens is the high-level source language translates the source language to an intermediate language that allows for easy interpretation.
  This is faster because instructions in the source language are only decoded once.

  For example, Java does this with their Just-In-Time (JIT) compilation scheme, which translates all instructions to an intermediate language, then translates those to machine code on-the-fly when needed.

  Some examples of \nameref{def:Hybrid_Implementation} are:
  \begin{itemize}[noitemsep]
  \item Java
  \item Scala
  \item C\#
  \item JavaScript
  \item $\cdots$
  \end{itemize}
\end{definition}

One way to implement a language with \nameref{def:Hybrid_Implementation} is with \nameref{subsubsec:Dynamic_Compilation}.
\subsubsection{Dynamic Compilation}\label{subsubsec:Dynamic_Compilation}
\begin{itemize}[noitemsep]
\item Idea: behind dynamic compilation is that code is compiled \emph{while executing}.
\item Theory: The best of \nameref{def:Interpretation} and \nameref{def:Compilation} worlds.
\item Practice:
  \begin{itemize}[noitemsep]
  \item Difficult to build
  \item Memory usage can increase (sometimes dramatically)
  \item Performance can be higher than pre-compiled code, because only the code needed is compiled.
  \end{itemize}
\end{itemize}

%%% Local Variables:
%%% mode: latex
%%% TeX-master: "../EDAP05-Concepts_Programming_Languages-Reference_Sheet"
%%% End:


\section{Language Critique}\label{sec:Language_Critique}
There are several very open-ended questions that need to be asked when categorizing and critiqueing languages:
\begin{enumerate}[noitemsep]
\item What programming language is best for \emph{what task}?
\item \emph{What criteria} do we measure?
  \begin{itemize}[noitemsep]
  \item Most criteria do not have good measurement tools.
  \end{itemize}

\item \emph{How} do we obtain measurements for these criteria?
\end{enumerate}

These are all qualities of:
\begin{itemize}[noitemsep]
\item The language
\item The language implementation(s)
\item The available tooling for the language and that particular implementation
\item The available libraries for the language and that particular implementation
\item Other infrastructure
  \begin{itemize}[noitemsep]
  \item User groups
  \item Books
  \item etc.
  \end{itemize}
\end{itemize}

\begin{table}[h!]
  \centering
  \begin{tabular}{cccc}
    \toprule
    & \multicolumn{3}{c}{Criteria} \\
    \midrule
    Characteristic & \nameref{subsec:Readability} & \nameref{subsec:Writability} & \nameref{subsec:Reliability} \\
    \midrule
    \nameref{subsubsec:Simplicity} & \checkmark{} & \checkmark{} &  \checkmark{} \\
    \nameref{subsubsec:Orthogonality} & \checkmark{} & \checkmark{} & \checkmark{} \\
    \nameref{subsubsec:Data_Types} & \checkmark{} & \checkmark{} & \checkmark{} \\
    \nameref{subsubsec:Syntax_Design} & \checkmark{} & \checkmark{} & \checkmark{} \\
    \midrule
    \nameref{subsubsec:Abstraction_Support} & & \checkmark{} & \checkmark{} \\
    \nameref{subsubsec:Expressivity} & & \checkmark{} & \checkmark{} \\
    \midrule
    \nameref{subsubsec:Type_Checking} & & & \checkmark{} \\
    \nameref{subsubsec:Exception_Handling} & & & \checkmark{} \\
    Restricted \nameref{subsubsec:Aliasing} & & & \checkmark{} \\
    \bottomrule
  \end{tabular}
  \caption[Language Evaluation Criteria]{Language Evaluation Criteran and the Characteristics that Affect Them}
  \label{tab:Language_Evaluation_Criteria}
\end{table}

Some additional criteria that could be used to evaluate programming languages are:
\begin{itemize}[noitemsep]
\item Portability: Ease with with programs can be moved from one implementation to another
\item Generality: The applicability to a wide range of applications
\item Well-Definedness: The completeness and precision of the language's official defining document
\end{itemize}

Some of criteria are given different weightings/importance by different people, thus making each slightly subjective.
Additionally, many of these criteria are not precisely defined, nor are they exactly measurable.

\subsection{Readability}\label{subsec:Readability}
\begin{definition}[Readability]\label{def:Readability}
  \emph{Readability} is how easily a program can be read and understood \emph{by a human}.
  Some languages do not support certain functions, but programmers try to make the language do what it is not designed to do.
  This will lead to complicated and difficult-to-read programs.
  
  The idea of program readability was first presented as the software life-cycle concept~\parencite{Booch1987}.
  The initial coding was downplayed compared to earlier, and the maintenance and improvement of the code was brought to the forefront.
\end{definition}

\subsubsection{Simplicity}\label{subsubsec:Simplicity}
There are 2 main factors and 1 minor factor for a language's simplicity:
\begin{enumerate}[noitemsep]
\item The number of features present in the language.
\item The \nameref{def:Feature_Multiplicity} of a language.
\item The ability to \nameref{def:Overload_Operator}s.
\end{enumerate}

Assembly language is on the most-simple end of the simplicity spectrum.
In assembly, the form and meaning of most statements are incredibly simple, but without more complex control statements, the program's structure is less obvious.

\begin{definition}[Feature Multiplicity]\label{def:Feature_Multiplicity}
  \emph{Feature multiplicity} is when there is more than one way to accomplish a particular operation with language built-in features.
  For example, in Java these are all equivalent when evaluated as standalone expressions:
  \begin{enumerate}[noitemsep]
  \item \texttt{count = count + 1}
  \item \texttt{count += 1}
  \item \texttt{count++}
  \item \texttt{++count}
  \end{enumerate}
\end{definition}

\begin{minted}[frame=lines,linenos]{python3}
def d(x):
    r = x[::-1]
    return x == r
\end{minted}

\begin{definition}[Overload Operator]\label{def:Overload_Operator}
  An \emph{overloaded operator} is one where a single symbol has more than one meaning.
  For example the \texttt{+} operator can be overloaded to add 2 integers, 2 floating-point numbers, or an integer and a floating-point number.
  This overloading helps \emph{improve} the \nameref{subsubsec:Simplicity} of a language.
\end{definition}

\begin{minted}[frame=lines,linenos]{python3}
x = 3 + 4  # Evaluates to 7
y = 3.0 + 4.0  # Evaluates to 7.0
z = 3 + 4.0  # Evaluates to 7.0
\end{minted}

\subsubsection{Orthogonality}\label{subsubsec:Orthogonality}
\begin{definition}[Orthogonality]\label{def:Orthogonality}
  \emph{Orthogonality} in a programming language means that a relatively small set of primitive constructs can be combined in a relatively small number of ways to build the control and data structures of a language.
  Additionally, every possible combination of primitives is legal and meaningful.

  The more orthogonal a language, the fewer exceptions to the language rules there can be.
  These fewer exceptions means there is a higher degree of regularity in the language design, making it easier to learn, read, and understand.

  \begin{remark}[Over-Orthogonality]\label{rmk:Over_Orthogonality}
    Too much orthogonality can cause problems.
    Having too much combinational freedom with primitive constructs and their combinations can make for an extremely complex compound construct.
    This leads to unnecessary complexity.
  \end{remark}
\end{definition}

\begin{minted}[frame=lines,linenos]{c}
// global variable section

float f1 = 2.0f * 2.0f;
float f2 = sqrt(2.0f); // error
\end{minted}

\subsubsection{Data Types}\label{subsubsec:Data_Types}
The use of data types conveys intent when reading and writing the program.
For example, a boolean data type conveys a true/false value better than an integer that is 1/0 for true/false respectively.
\begin{itemize}[noitemsep]
\item \texttt{timeOut = 1}
\item \texttt{timeOut = true}
\end{itemize}

\begin{minted}[frame=lines,linenos]{java}
enum Color {
  Red, Green, Blue
};

Color c = readColorFromUser();
\end{minted}

\subsubsection{Syntax Design}\label{subsubsec:Syntax_Design}
There are 2 main syntactic design choices that affect \nameref{subsec:Readability}:
\begin{enumerate}[noitemsep]
\item \nameref{par:Reserved_Words}
\item \nameref{par:Syntax_Form_Meaning}
\end{enumerate}

\paragraph{Reserved/Special Words}\label{par:Reserved_Words}
\begin{definition}[Reserved Word]\label{def:Reserved_Word}
  \emph{Reserved word}s are words that are reserved by the language constructors because those particular words have a meaning in the language.
  For example:
  \begin{itemize}[noitemsep]
  \item \texttt{while}
  \item \texttt{class}
  \item \texttt{for}
  \end{itemize}
\end{definition}

There are also special words and matching characters that can denote groups of instructions.
\begin{itemize}[noitemsep]
\item C and its decendants
  \begin{itemize}[noitemsep]
  \item Matching brances
  \item \texttt{\{} and \texttt{\}}
  \end{itemize}
\item Ada/Fortran 95 and their decendants:
  \begin{itemize}[noitemsep]
  \item Distinct closing syntax for each statement group
  \item \texttt{end if} to end an if statement
  \end{itemize}
\end{itemize}

Also, can these \nameref{def:Reserved_Word}s be used as names for program variables?
If so, this will increase overall complexity of a program.
The code block below, from Fortran 95, illustrates this point.
\begin{minted}[frame=lines,linenos]{fortran}
program hello
  implicit none
  integer end, do
  do = 0
  end = 10
  do do=do, end
    print *,do
  end do
end program hello
\end{minted}

\paragraph{Form and Meaning}\label{par:Syntax_Form_Meaning}
Statements should be designed such that their appearance partially indicates what their purpose is.
For example, the UNIX command \texttt{grep} gives no hint at what it is supposed to do, unless you know the text editor \texttt{ed}.

Semantics, or meaning, should follow directly from syntax or form.
In some cases, this principle is violated by 2 language constructs that are identical or similar in appearance, but different in meaning, depending on the context.
For example, C's \texttt{static} \nameref{def:Reserved_Word}s.

\subsection{Writability}\label{subsec:Writability}
Writability is a measure of how easy it is to write a program in a language for a given problem domain.
This is closely related to the language characteristics presented in \Cref{subsec:Readability}, \nameref{subsec:Readability}.
The definition of the problem domain is incredibly important, because C would not be used to make a GUI, and Visual BASIC would not be used to make an operating system.

\subsubsection{\nameref{subsubsec:Simplicity} and \nameref{subsubsec:Orthogonality}}\label{subsubsec:Written_Simplicity_Orthogonality}
Programmers might not know all the features for a language.
Or, the might know about them, but use them incorrectly.
This means there should be a smaller number of primitive constructs and a consistent set of rules for combining them.
This reduces the number of primitive constructs in a language, and allows a programmer to design a complex solution by only using a simple set of primitive constructs.
By reducing the orthogonality of a program, the total possible combinations of constructs is reduced, simplifying the process of reading and writing the program.

\subsubsection{Support for Abstraction}\label{subsubsec:Abstraction_Support}
\begin{definition}[Abstraction]\label{def:Abstraction}
  \emph{Abstraction} is the ability to define and then use complicated structures or operations in ways that allow many of the details to be ignored.
  This is a key concept in modern programming language design.

  There are 2 categories of abstraction:
  \begin{enumerate}[noitemsep]
  \item \nameref{par:Process_Abstraction}
  \item \nameref{par:Data_Abstraction}
  \end{enumerate}
\end{definition}

\paragraph{Process Abstraction}\label{par:Process_Abstraction}
Process abstraction is the use of a subprogram to implement some block of code used multiple times.
For example, a sorting algorithm.
If the code for the algorithm could not be factored out into a separate piece of code, the algorithm would need to be copied everywhere it was used.
This would lead to additional complexity and reduce the \nameref{subsec:Readability} of the code.

\paragraph{Data Abstraction}\label{par:Data_Abstraction}
For example, representing a binary tree in C++/Java is done by making a tree node class that has 2 pointers and an integer.
This abstraction is more natural to think about than what would need to be done in Fortran 77.
In Fortran 77, there would need to be 3 parallel integer arrays, where 2 of the integers in each array would be used as subscripts to find their children.

\subsubsection{Expressivity}\label{subsubsec:Expressivity}
Expressivity has several characteristics.
\begin{enumerate}[noitemsep]
\item Verbosity of the language
  \begin{itemize}[noitemsep]
  \item The amount of code needed to describe some computation to the computer.
  \end{itemize}
\item Powerful/Convenient way to specify computations.
  \begin{itemize}[noitemsep]
  \item \texttt{count++} vs. \texttt{count = count + 1} to increment a value in Java
  \end{itemize}
\end{enumerate}

\subsection{Reliability}\label{subsec:Reliability}
Reliability is a measure of the program performing to its specifications reliably under all conditions.

\subsubsection{Type Checking}\label{subsubsec:Type_Checking}
\begin{definition}[Type Checking]\label{def:Type_Checking}
  \emph{Type checking} is a process for testing for type errors in a given program, by the compiler or the interpreter, depending on its implementation (\nameref{subsec:Interpretation} vs. \nameref{def:Compilation}).
  Runtime type checking is expensive, so compile-time checking is preferred.
\end{definition}

The earlier that type checking can occur reduces the potential errors, and corrective actions can be taken.

\subsubsection{Exception Handling}\label{subsubsec:Exception_Handling}
The programming language should have the ability to intercept runtime errors, along with other unusual conditions, take corrective actions, the continue normally is traditionally called \emph{exception handling}.

\subsubsection{Aliasing}\label{subsubsec:Aliasing}
\begin{definition}[Aliasing]\label{def:Aliasing}
  \emph{Aliasing} is having 2 or more distinct names that can be used to access the same memory location.
  Most languages allow for 2 pointers to point to the same thing in memory, but others prevent this completely.
\end{definition}

Sometimes, \nameref{def:Aliasing} is used to overcome deficiencies in the language's \nameref{subsubsec:Abstraction_Support}.
Others greatly restrict possible \nameref{def:Aliasing} to increase their \nameref{subsec:Reliability}.

\subsubsection{\nameref{subsec:Readability} and \nameref{subsec:Writability}}\label{subsubsec:Reliable_Readability_and_Writability}
The \nameref{subsec:Readability} and \nameref{subsec:Writability} greatly influence a program's \nameref{subsec:Reliability}.
A program written in a language that exceeds the languages original problem domain will use unnatural approaches to solve the problem.
These unnatural approaches are less likely to be correct for all possible situations.
Thus, the easier a program is to write, the more likely it is to be correct for all possible situations.

Programs that are difficult to read will affect the writing and maintenance phases of the software's life cycle.

\subsection{Cost}\label{subsec:Cost}
There are several parts that increase the cost of a programming language.
\begin{enumerate}[noitemsep]
\item Training programmers in a new language
  \begin{itemize}[noitemsep]
  \item Function of \nameref{subsubsec:Simplicity} and \nameref{subsubsec:Orthogonality}
  \item Function of programmer experience
  \end{itemize}
  
\item Writing software
  \begin{itemize}[noitemsep]
  \item Function of \nameref{subsec:Writability} of the language
  \end{itemize}
  
\item Compilation time
  \begin{itemize}[noitemsep]
  \item Time to compile a program
  \item Resources required to compile a program in a language
  \end{itemize}

\item Run time
  \begin{itemize}[noitemsep]
  \item Performance during runtime
  \item Dependent on the effort made to optimize the input source code
  \end{itemize}

\item Financial cost of special software
  \begin{itemize}[noitemsep]
  \item The cost of using the \nameref{def:Compiler} for a language for instance.
  \item Languages with free \nameref{def:Compiler}s or interpreters tend to be accepted more quickly than languages with a financial cost
  \end{itemize}

\item Cost of limited reliability
  \begin{itemize}[noitemsep]
  \item Maintenance time
    \begin{itemize}[noitemsep]
    \item Corrections made to errors in the program
    \item Modifications to add new functionality
    \end{itemize}

  \item Insurance cost, in special cases
    \begin{itemize}[noitemsep]
    \item Airplanes
    \item Nuclear power plants
    \item X-Ray machines
    \end{itemize}
  \end{itemize}
\end{enumerate}

%%% Local Variables:
%%% mode: latex
%%% TeX-master: "../EDAP05-Concepts_Programming_Languages-Reference_Sheet"
%%% End:


\section{Backus-Naur Form and Context-Free Grammars}\label{sec:BNF_and_CFGs}
In the 1950s, there were 2 men, \href{https://en.wikipedia.org/wiki/Noam_Chomsky}{Noam Chomsky} and \href{https://en.wikipedia.org/wiki/John_Backus}{John Backus}, that were working completely separately who were trying to formally describe language.
They actually ended up developing very similar answers to that problem.

\begin{remark*}
  \nameref{def:Context_Free_Grammar}s are referred to only as grammars throughout this document.
  Also, the terms BNF (\nameref{def:CFG_BNF_Form}) and grammar are used interchangeably.
\end{remark*}

\subsection{Context-Free Grammars}\label{subsec:CFGs}
Chomsky, a linguist, described 4 classes of grammars that define 4 classes of languages, which are given in \Cref{tab:Formal_Grammar_Hierarchy}

There exists a hierarchy for the definition of Grammars that define \nameref{def:Language}s.
It is called the \emph{\nameref{tab:Formal_Grammar_Hierarchy}}.
\begin{table}[h!]
  \centering
  \begin{tabular}{ccc}
    \toprule
    Grammar & Rule Patterns & Type \\
    \midrule
    Regular & $\Nonterminal{X} \rightarrow a \Nonterminal{Y}$ or $\Nonterminal{X} \rightarrow a$ or $\Nonterminal{X} \rightarrow \epsilon$ & 3 \\
    Context-Free & $\Nonterminal{X} \rightarrow \gamma$ & 2 \\
    Context-Sensitive & $\alpha \Nonterminal{X} \beta \rightarrow \alpha \gamma \beta$ & 1 \\
    Arbitrary & $\gamma \rightarrow \delta$ & 0 \\
    \bottomrule
  \end{tabular}
  \caption{Chomsky Hierarchy of Formal Grammars}
  \label{tab:Formal_Grammar_Hierarchy}
\end{table}

Regular grammars have the same power as regular expressions, meaning they can be used to find tokens in a program.

\begin{remark*}
  Where $a$ is a \nameref{def:Terminal_Symbol}, $\alpha$, $\beta$, $\gamma$, and $\delta$ are \emph{sequences} of symbols (\nameref{def:Terminal_Symbol}s or \nameref{def:Nonterminal_Symbol}s).

  Type(3) $\subset$ Type(2) $\subset$ Type(1) $\subset$ Type(0)
\end{remark*}

\subsection{Backus-Naur Form}\label{subsec:BNF}
It is important to discuss where Backus-Naur form came from, and how ti has been modified since.
Originally, there was the \nameref{def:CFG_Canonical_Form}.
This only allowed for one production per line, and did not support options, repetition, etc.
\begin{definition}[Canonical Form]\label{def:CFG_Canonical_Form}
  The \emph{canonical form} of a \nameref{def:Context_Free_Grammar} is the most formal use of a \nameref{def:Context_Free_Grammar}.

  \begin{equation}\label{eq:CFG_Canonical_Form}
    \begin{aligned}
      \Nonterminal{A} &\rightarrow \Nonterminal{B} \; d \; e \; C \; f \\
      \Nonterminal{A} &\rightarrow g \; \Nonterminal{A} \\
    \end{aligned}
  \end{equation}

  The \nameref{def:CFG_Canonical_Form} is:
  \begin{itemize}[noitemsep]
  \item The core formalism for \nameref{def:Context_Free_Grammar}s
  \item Useful for proving properties and explaining algorithms
  \end{itemize}
\end{definition}

When John Backus was working on ALGOL 58, his published paper used a new formal notation for specifying programming language syntax.
\href{https://en.wikipedia.org/wiki/Peter_Naur}{Peter Naur} slightly modified Backus's original syntax which developed \nameref{def:CFG_BNF_Form}.

\begin{definition}[Backus-Naur Form]\label{def:CFG_BNF_Form}
  The \emph{Backus-Naur form} of a \nameref{def:Context_Free_Grammar} is an extension of the \nameref{def:CFG_Canonical_Form}.
  This form is less formal than the \nameref{def:CFG_Canonical_Form}, but is allows for condensation of multiple productions that have the same nonterminal on the left-hand side to the same production.
  This is done with the $\vert$ symbol.

  For example, \Cref{eq:CFG_BNF_Form} is equivalent to \Cref{eq:CFG_Canonical_Form}.
  \begin{equation}\label{eq:CFG_BNF_Form}
    \Nonterminal{A} \rightarrow \Nonterminal{B} \; d \; e \; \Nonterminal{C} \; f \: \vert \: g\; \Nonterminal{A}
  \end{equation}
\end{definition}

\nameref{def:CFG_BNF_Form} has some inconveniences, and has been extended to avoid thse issues.
These extensions have been formalized and called \nameref{def:CFG_EBNF_Form}.
\nameref{def:CFG_EBNF_Form} will not be used much in this course, but it is a good way to quickly and succinctly express a \nameref{def:Context_Free_Grammar}.

\begin{definition}[Extended Backus-Naur Form]\label{def:CFG_EBNF_Form}
  The \emph{Extended Backus-Naur form} of a \nameref{def:Context_Free_Grammar} is an extension of the \emph{Backus-Naur Form}.
  This is a more informal implementation of a \nameref{def:Context_Free_Grammar}.
  This informality allows for some additional constructs in the \nameref{def:Production} rules.

  These include:
  \begin{enumerate}[noitemsep]

  \item Repetition with the Kleene Star (*), or with $\lbrace \text{ repItem } \rbrace$
    \begin{itemize}[noitemsep]
    \item Means that portion of the \nameref{def:Production} can be repeated 0 or more times.
    \end{itemize}
  \item Single Optionals, denoted as $(\text{ op1 } \vert \text{ op2 } \vert \ldots)$
    \begin{itemize}[noitemsep]
    \item Means select one of the options present between the parentheses.
    \end{itemize}
  \item Optional portions of the \nameref{def:Production}, denoted with $[\text{ op }]$
    \begin{itemize}[noitemsep]
    \item Means that portion of the \nameref{def:Production} is an optional part of the entire \nameref{def:Production}.
    \end{itemize}
  \end{enumerate}

  The \nameref{def:CFG_EBNF_Form} is:
  \begin{itemize}[noitemsep]
  \item Compact, easy to read and write
  \item Common notation for practical use
  \end{itemize}
\end{definition}

\subsection{Use Today}\label{subsec:Use_Today}
\begin{definition}[Metalanguage]\label{def:Metalanguage}
  \emph{Metalanguage}s are languages that are used to describe other languages.
  \nameref{def:Context_Free_Grammar}s are one example used as a metalanguage for programming languages.
\end{definition}

\begin{definition}[Context-Free Grammar]\label{def:Context_Free_Grammar}
  A \emph{context-free grammar} or \emph{CFG} is a way to define a set of \textit{strings} that form a \nameref{def:Language}.
  Each string is a finite sequence of \nameref{def:Terminal_Symbol} taken from a finite \nameref{def:Alphabet}.
  This is done with one or more \nameref{def:Production}s, where each production can have both \nameref{def:Nonterminal_Symbol} and \nameref{def:Terminal_Symbol}.

  More formally, a \nameref{def:Context_Free_Grammar} is defined as $G = (N, T, P, S)$, where
  \begin{itemize}[noitemsep]
  \item $N$, the set of \nameref{def:Nonterminal_Symbol}s
  \item $T$, the set of \nameref{def:Terminal_Symbol}s
  \item $P$, the set of production rules, each with the form
    \begin{equation*}
      \MathNonterminal{X} \rightarrow \MathNonterminal{Y_{1}} \MathNonterminal{Y_{2}} \ldots \MathNonterminal{Y_{n}} \: \text{where } \MathNonterminal{X} \in N, x \geq 0, \text{and } \MathNonterminal{Y_{k}} \in N \cup T
    \end{equation*}
  \item $S$, the start symbol (one of the \nameref{def:Nonterminal_Symbol}s, $N$). $S \in N$.
  \end{itemize}
\end{definition}

\begin{definition}[Language]\label{def:Language}
  A \emph{language} is the set of \textbf{\textup{all}} strings that can be formed by the \nameref{def:Production}s in the \nameref{def:Context_Free_Grammar}.
\end{definition}

\begin{definition}[Production]\label{def:Production}
  A \emph{production} is a rule that defines the relation between a single \nameref{def:Nonterminal_Symbol} and a string comprised of \nameref{def:Nonterminal_Symbol}s, \nameref{def:Terminal_Symbol}s, and the \nameref{def:Empty_String}.
  These can be though of as abstractions for syntactic structures.

  The are denoted as shown below:
  \begin{equation}\label{eq:Production}
    p_{0}: \Nonterminal{A} \rightarrow \alpha
  \end{equation}

  \begin{remark}
    There \emph{can} be multiple productions for the same \nameref{def:Nonterminal_Symbol}
  \end{remark}
\end{definition}

\begin{definition}[Nonterminal Symbol]\label{def:Nonterminal_Symbol}
  A \emph{nonterminal symbol}, or just \emph{nonterminal}, is a symbol that is used in the \nameref{def:Context_Free_Grammar} as a symbol for a \nameref{def:Production}.
\end{definition}

\begin{definition}[Terminal Symbol]\label{def:Terminal_Symbol}
  A \emph{terminal symbol}, or just \emph{terminal}, is a symbol that cannot be derived any further.
  This is a symbol that is part of the \nameref{def:Alphabet} that is used to form the \nameref{def:Language}.

  \begin{remark}
    These terminals could be tokens defined by a regular grammar or a regular expression.
    They might just be abstractions of sequences or sets of symbols from the \nameref{def:Alphabet}.
  \end{remark}
\end{definition}

\begin{definition}[Start Symbol]\label{def:Start_Symbol}
  The \emph{start symbol} is a \nameref{def:Nonterminal_Symbol} which is specially designated as the start point of a \nameref{def:Derivation} for a grammar.

  Other than the fact a \nameref{def:Derivation} starts with this \nameref{def:Nonterminal_Symbol} and its associated \nameref{def:Production}, it is not special.
\end{definition}

\begin{definition}[Empty String]\label{def:Empty_String}
  The \emph{empty string} is a special symbol that is neither a \nameref{def:Nonterminal_Symbol} nor a \nameref{def:Terminal_Symbol}.
  The empty string is a \emph{metasymbol}.
  It is a unique symbol meant to represent the lack of a string.
  It is denoted with the lowercase Greek epsilon, $\epsilon$ or $\varepsilon$.
\end{definition}

\begin{definition}[Alphabet]\label{def:Alphabet}
  The finite set of \nameref{def:Nonterminal_Symbol}s that can be used to form a \nameref{def:Language}.
\end{definition}

\subsubsection{Multiple Productions on Single Line}\label{subsubsec:Multiple_Productions_One_Line}
This is briefly discussed in \Cref{def:CFG_BNF_Form}.
What this allows us to do is combine multiple \nameref{def:Production}s that have the same \nameref{def:Nonterminal_Symbol} on the left-hand side to a single line, or single \nameref{def:Production}.

For example,
\begin{equation}\label{eq:Multiple_Productions_Multiple_Lines}
  \begin{aligned}
    \Nonterminal{if stmt} &\rightarrow \mathtt{if} \Bigl( \Nonterminal{logic expr} \Bigr) \Nonterminal{stmt} \\
    \Nonterminal{if stmt} &\rightarrow \mathtt{if} \Bigl( \Nonterminal{logic expr} \Bigr) \Nonterminal{stmt} \; \mathtt{else} \; \Nonterminal{stmt} \\
  \end{aligned}
\end{equation}
can be combined to
\begin{equation}\label{eq:Multiple_Productions_One_Line}
  \begin{aligned}
    \Nonterminal{if stmt} \rightarrow &\mathtt{if} \Bigl( \Nonterminal{logic expr} \Bigr) \Nonterminal{stmt} \\
    \vert &\mathtt{if} \Bigl( \Nonterminal{logic expr} \Bigr) \Nonterminal{stmt} \; \mathtt{else} \; \Nonterminal{stmt}
  \end{aligned}
\end{equation}

\subsubsection{Describing Lists}\label{subsubsec:Describing_Lists}
A \nameref{def:Production} is recursive if its left-hand side \nameref{def:Nonterminal_Symbol} appears somewhere ont he right-hand side.
This recursive property is useful for constructing variable-length lists.

This is a small extension of using \nameref{subsubsec:Multiple_Productions_One_Line}.
\begin{equation}\label{eq:Describing_Lists}
  \begin{aligned}
    \Nonterminal{ident list} \rightarrow &\text{identifier} \\
    \vert &\text{identifier}, \Nonterminal{ident list} \\
  \end{aligned}
\end{equation}

\subsubsection{Grammars and Derivations}\label{subsubsec:Grammars_and_Derivations}
\begin{definition}[Derivation]\label{def:Derivation}
  A \emph{derivation} is the use of \nameref{def:Production} applications to parse a given input string.
  \Cref{ex:Left-Most Derivation} demonstrates this.
\end{definition}

\begin{example}[]{Left-Most Derivation}
  Perform a \nameref{def:Leftmost_Derivation} of the sentence
  \begin{equation*}
    \mathtt{begin} \; A = B + C ; \: B = C \; \mathtt{end}
  \end{equation*}

  With the grammar
  \begin{align*}
    \Nonterminal{program} \rightarrow &\mathtt{begin} \; \Nonterminal{stmt list} \; \mathtt{end} \\
    \Nonterminal{stmt list} \rightarrow &\Nonterminal{stmt} \\
                          \vert &\Nonterminal{stmt} ; \Nonterminal{stmt list} \\
    \Nonterminal{stmt} \rightarrow &\Nonterminal{var} = \Nonterminal{expression} \\
    \Nonterminal{var} \rightarrow &A \vert B \vert C \\
    \Nonterminal{expression} \rightarrow &\Nonterminal{var} + \Nonterminal{var} \\
                          \vert &\Nonterminal{var} - \Nonterminal{var} \\
                          \vert &\Nonterminal{var}
  \end{align*}
  \tcblower{}
  \begin{align*}
    \Nonterminal{program} &\Rightarrow \mathtt{begin} \:\: \Nonterminal{stmt list} \:\: \mathtt{end} \\
                          &\Rightarrow \mathtt{begin} \:\: \Nonterminal{stmt}\: ; \: \Nonterminal{stmt list} \:\: \mathtt{end} \\
                          &\Rightarrow \mathtt{begin} \:\: \Nonterminal{var} \: = \: \Nonterminal{expression}\: ; \: \Nonterminal{stmt list} \:\: \mathtt{end} \\
                          &\Rightarrow \mathtt{begin} \:\: A \: = \: \Nonterminal{expression}\: ; \: \Nonterminal{stmt list} \:\: \mathtt{end} \\
                          &\Rightarrow \mathtt{begin} \:\: A \: = \: \Nonterminal{var} + \Nonterminal{var}\: ; \: \Nonterminal{stmt list} \:\: \mathtt{end} \\
                          &\Rightarrow \mathtt{begin} \:\: A = B + \Nonterminal{var}\: ; \: \Nonterminal{stmt list} \:\: \mathtt{end} \\
                          &\Rightarrow \mathtt{begin} \:\: A = B + C\: ; \: \Nonterminal{stmt list} \:\: \mathtt{end} \\
                          &\Rightarrow \mathtt{begin} \:\: A = B + C\: ; \: \Nonterminal{stmt} \:\: \mathtt{end} \\
                          &\Rightarrow \mathtt{begin} \:\: A = B + C\: ; \: \Nonterminal{var} = \Nonterminal{expression} \:\: \mathtt{end} \\
                          &\Rightarrow \mathtt{begin} \:\: A = B + C\: ; \: B = \Nonterminal{expression} \:\: \mathtt{end} \\
                          &\Rightarrow \mathtt{begin} \:\: A = B + C\: ; \: B = \Nonterminal{var} \:\: \mathtt{end} \\
                          &\Rightarrow \mathtt{begin} \:\: A = B + C\: ; \: B = C \:\: \mathtt{end} 
  \end{align*}
\end{example}

In general, \nameref{def:Derivation}s occur from left-to-right, which is one L in the 2 different types of \nameref{def:Derivation}s.
Both types of \nameref{def:Derivation}, \nameref{def:Leftmost_Derivation} and \nameref{def:Rightmost_Derivation}, will yield the same result when a \nameref{def:Derivation} is successfully completed.

\begin{definition}[Leftmost Derivation]\label{def:Leftmost_Derivation}
  \emph{Leftmost derivation}, or \emph{LL} derivation, stands for \emph{left-to-right leftmost derivation}.
  Starting from the left of the sentence, you always derive the left-most \nameref{def:Nonterminal_Symbol}, until you reach a \nameref{def:Terminal_Symbol}.
  Once all symbols present in the sentence are \nameref{def:Terminal_Symbol}, you are done.
\end{definition}

\begin{definition}[Rightmost Derivation]\label{def:Rightmost_Derivation}
  \emph{Rightmost derivation}, or \emph{LR} derivation, stands for \emph{left-to-right rightmost derivation}.
  Starting from the left of the sentence, you always derive the right-most \nameref{def:Nonterminal_Symbol}, until you reach a \nameref{def:Terminal_Symbol}.
  Once all symbols present in the sentence are \nameref{def:Terminal_Symbol}, you are done.
\end{definition}

\begin{example}[]{Right-Most Derivation}
  Perform a \nameref{def:Rightmost_Derivation} of the sentence
  \begin{equation*}
    \mathtt{begin} \; A = B + C ; \: B = C \; mathtt{end}
  \end{equation*}

  With the grammar
  \begin{align*}
    \Nonterminal{program} \rightarrow &\mathtt{begin} \; \Nonterminal{stmt list} \; \mathtt{end} \\
    \Nonterminal{stmt list} \rightarrow &\Nonterminal{stmt} \\
                          \vert &\Nonterminal{stmt} ; \Nonterminal{stmt list} \\
    \Nonterminal{stmt} \rightarrow &\Nonterminal{var} = \Nonterminal{expression} \\
    \Nonterminal{var} \rightarrow &A \vert B \vert C \\
    \Nonterminal{expression} \rightarrow &\Nonterminal{var} + \Nonterminal{var} \\
                          \vert &\Nonterminal{var} - \Nonterminal{var} \\
                          \vert &\Nonterminal{var}
  \end{align*}
  \tcblower{}
  \begin{align*}
    \Nonterminal{program} &\Rightarrow \mathtt{begin} \:\: \Nonterminal{stmt list} \:\: \mathtt{end} \\
                          &\Rightarrow \mathtt{begin} \:\: \Nonterminal{stmt}\: ; \: \Nonterminal{stmt list} \:\: \mathtt{end} \\
                          &\Rightarrow \mathtt{begin} \:\: \Nonterminal{stmt} \: ; \: \Nonterminal{stmt} \:\: \mathtt{end} \\
                          &\Rightarrow \mathtt{begin} \:\: \Nonterminal{stmt} \: ; \: \Nonterminal{var} = \Nonterminal{expression} \:\: \mathtt{end} \\
                          &\Rightarrow \mathtt{begin} \:\: \Nonterminal{stmt} \: ; \: \Nonterminal{var} = \Nonterminal{var} \:\: \mathtt{end} \\
                          &\Rightarrow \mathtt{begin} \:\: \Nonterminal{stmt} \: ; \: \Nonterminal{var} = C \:\: \mathtt{end} \\
                          &\Rightarrow \mathtt{begin} \:\: \Nonterminal{stmt} \: ; \: B = C \:\: \mathtt{end} \\
                          &\Rightarrow \mathtt{begin} \:\: \Nonterminal{var} \: = \: \Nonterminal{expression} \: ; \: B = C \:\: \mathtt{end} \\
                          &\Rightarrow \mathtt{begin} \:\: \Nonterminal{var} \: = \: \Nonterminal{var} \: + \: \Nonterminal{var} \: ; \: B = C \:\: \mathtt{end} \\
                          &\Rightarrow \mathtt{begin} \:\: \Nonterminal{var} \: = \: \Nonterminal{var} \: + C \: ; \: B = C \:\: \mathtt{end} \\
                          &\Rightarrow \mathtt{begin} \:\: \Nonterminal{var} \: = \: B + C \: ; \: B = C \:\: \mathtt{end} \\
                          &\Rightarrow \mathtt{begin} \:\: A = B + C\: ; \: B = C \:\: \mathtt{end} 
  \end{align*}
\end{example}

\subsubsection{Parse Trees}\label{subsubsec:Parse_Trees}
Parse trees are hierarchical structures that describe the same information as a \nameref{def:Derivation} in a visual format.
Every internal node is labeled with a \nameref{def:Nonterminal_Symbol} and every leaf is labeled with a \nameref{def:Terminal_Symbol}

\subsubsection{Ambiguities}\label{subsubsec:Ambiguities}
\begin{definition}[Ambiguous]\label{def:Ambiguous}
  A \nameref{def:Context_Free_Grammar} is said to be \emph{ambiguous} or has \emph{ambiguities} if there is more than one way to derive the same string in a grammar.

  The grammar below is ambiguous because there are multiple ways to parse the string: ``\texttt{statement;statement;statement}''.
  \begin{equation}\label{eq:Ambiguous}
    \begin{aligned}
      p_{0} &: \Nonterminal{start} \rightarrow \Nonterminal{program} \$ \\
      p_{1} &: \Nonterminal{program} \rightarrow \Nonterminal{statement} \\
      p_{2} &: \Nonterminal{statement} \rightarrow \Nonterminal{statement ``;'' statement} \\
      p_{3} &: \Nonterminal{statement} \rightarrow \text{ID ``='' INT} \\
      p_{4} &: \Nonterminal{statement} \rightarrow \epsilon \\
    \end{aligned}
  \end{equation}
\end{definition}

\paragraph{Dangling \texttt{if-then-else}}\label{par:Dangling_if-else}
\texttt{if-then-else} statements are usually defined to have an \texttt{else} clause, that when present, matches with the nearest previous unmatched \texttt{then}.
This can be represented with the \nameref{def:Production}s shown in \Cref{eq:Dangling_if-else}.
\begin{equation}\label{eq:Dangling_if-else}
  \begin{aligned}
    \Nonterminal{stmt} \rightarrow &\Nonterminal{matched} \: \vert \: \Nonterminal{unmatched} \\
    \Nonterminal{matched} \rightarrow &\mathtt{if} \; \Nonterminal{logic expr} \; \mathtt{then} \; \Nonterminal{matched} \; \mathtt{else} \; \Nonterminal{matched} \\
    \vert \: &\text{any non-if statement} \\\
    \Nonterminal{unmatched} \rightarrow &\mathtt{if} \; \Nonterminal{logic expr} \; \mathtt{then} \; \Nonterminal{stmt} \\
    \vert \: &\mathtt{if} \; \Nonterminal{logic expr} \; \mathtt{then} \; \Nonterminal{matched} \; \mathtt{else} \; \Nonterminal{unmatched}
  \end{aligned}
\end{equation}

\subsubsection{Operator Precedence}\label{subsubsec:Operator_Precedence}
To handle the precedence of operators, we need to define a ``priority level'' to our \nameref{def:Production}s.
It is good to note that the further \emph{down} an expression is in the parse tree, the higher its priority in mathematics.
\begin{equation}\label{eq:Operator_Precedence}
  \begin{aligned}
    \Nonterminal{assign} \rightarrow &\Nonterminal{id} = \Nonterminal{expression} \\
    \Nonterminal{id} \rightarrow &A \: \vert \: B \: \vert C \\
    \Nonterminal{expression} \rightarrow &\Nonterminal{expression} + \Nonterminal{multiplicative expression} \\
    \vert &\Nonterminal{multiplicative expression} \\
    \Nonterminal{multiplicative expression} \rightarrow &\Nonterminal{multiplicative expression} * \Nonterminal{factor} \\
    &\Nonterminal{factor} \\
    \Nonterminal{factor} \rightarrow &( \: \Nonterminal{expression} \: ) \\
    \vert &\Nonterminal{id} \\
  \end{aligned}
\end{equation}

\subsubsection{Operator Associativity}\label{subsubsec:Operator_Associativity}
We need to make sure that operators are associated with each other correctly.
If we need to make an operator right associative, we just need to flip the terms in \Cref{eq:Operator_Precedence} around.
The operators in \Cref{eq:Operator_Precedence} are left associative as they are right now.
\begin{equation}\label{eq:Operator_Associativity}
  \begin{aligned}
    \Nonterminal{factor} \rightarrow &\Nonterminal{expression} ** \Nonterminal{factor} \\
    \vert &\Nonterminal{term} \\
    \Nonterminal{term} \rightarrow &( \: \Nonterminal{expression} \: ) \\
    \vert &\Nonterminal{id} \\
  \end{aligned}
\end{equation}
%%% Local Variables:
%%% mode: latex
%%% TeX-master: "../EDAP05-Concepts_Programming_Languages-Reference_Sheet"
%%% End:


\section{Natural Semantics}\label{sec:Natural_Semantics}
Natural semantics assumes that we know the syntax of the language.
We will assume that we have a \nameref{def:CFG_BNF_Form} \nameref{def:Context_Free_Grammar}.
We will also assume that ambiguities have been resolved, somehow.

\begin{definition}[Semantics]\label{def:Semantics}
  \emph{Semantics} is the act of attaching a meaning to a syntactic construct.
  For instance, we know the value of 1 to be 1, but does $\text{one} = 1$?
  We have defined the semantics of $\text{one}$ to be equivalent to the numeral 1.

  We can defined these semantic relationships with \nameref{def:Evaluation_Relation}s.
\end{definition}

\begin{definition}[Evaluation Relation]\label{def:Evaluation_Relation}
  The \emph{evaluation relation} states the relation between a program $p$ and the program's result $v$:
  \begin{equation}\label{eq:Evaluation_Relation}
    p \EvaluatesTo v
  \end{equation}

  This is read as ``$p$ evaluates to $v$''.

  \begin{remark}
    The value of the left-hand side is an arbitrary set of operator(s).
    For instance, if we wanted to specify addition, but using the \texttt{add} operator, then $p$ would look $\mathtt{e_{1} \: add \: e_{2}}$.
    However, on the right-hand side, we \emph{MUST} specify the operation that will take place.
    The right-hand side is a mathematical construction that the reader will understand.
  \end{remark}
\end{definition}

\begin{definition}[Metavariable]\label{def:Metavariable}
  A \emph{metavariable} is a variable in the \nameref{def:Metalanguage}.
  In \Cref{eq:Evaluation_Relation}, $p$ and $v$ are metavariables.
  
  $p$ is a metavariable that can contain any input program, and $v$ is a metavariable that can contain any result that the language might compute.
\end{definition}

\begin{blackbox}
  Throughout this section, we will use this \nameref{def:CFG_BNF_Form} \nameref{def:Context_Free_Grammar}.
  \begin{equation}\label{eq:DivAdd_Grammar}
    \begin{aligned}
      \Nonterminal{expr} &\rightarrow nat \\
      &\vert \Nonterminal{expr} \; \SemanticInput{\mathtt{+}} \; \Nonterminal{expr} \\
      &\vert \Nonterminal{expr} \; \SemanticInput{\mathtt{div}} \; \Nonterminal{expr} \\
    \end{aligned}
  \end{equation}
\end{blackbox}

Let's start with the programming language specified by the \nameref{def:Context_Free_Grammar} below.
\begin{equation*}
  \Nonterminal{W} \rightarrow \SemanticInput{\mathtt{ett}} \vert \SemanticInput{\text{\texttt{tv\r{a}}}} \vert \SemanticInput{\mathtt{tre}}
\end{equation*}
This allows only 3 programs, each of which is a single word in Swedish.
We can define the semantics (attach meaning) with the following rules:
\begin{align*}
  \SemanticInput{\mathtt{ett}} &\EvaluatesTo 1 & \SemanticInput{\text{\texttt{tv\r{a}}}} &\EvaluatesTo 2 & \SemanticInput{\mathtt{tre}} &\EvaluatesTo 3 \\
\end{align*}

\subsection{Ambiguous Semantics}\label{subsec:Ambiguous_Semantics}
Just like in \nameref{def:Context_Free_Grammar}s, there can be ambiguities in \nameref{def:Semantics} too.
They occur when there is more than 1 specification for a symbol.
For example,
\begin{equation*}
  \Nonterminal{Q} \rightarrow \SemanticInput{\mathtt{eins}} \; \vert \; \SemanticInput{\mathtt{zwei}} \; \vert \; \SemanticInput{\mathtt{?}}
\end{equation*}
with the \nameref{def:Semantics} below:
\begin{align*}
  \SemanticInput{\mathtt{eins}} &\EvaluatesTo 1 & \SemanticInput{\mathtt{zwei}} &\EvaluatesTo 2 & \SemanticInput{\mathtt{?}} &\EvaluatesTo 0 & \SemanticInput{\mathtt{?}} &\EvaluatesTo 3 \\
\end{align*}

In this course, we are interested in unambiguous \nameref{def:Semantics}.
When language designers design a programming language, they try to avoid these ambiguities.

We can normally check if a set of semantic rules are unambiguous by checking that the left-hand side of each of the \nameref{def:Evaluation_Relation}s does not overlap with another \nameref{def:Evaluation_Relation}.
If there is an overlap, then it \textbf{must} be shown that the same result will be yielded.

\subsection{Conditional Rules}\label{subsec:Conditional_Rules}
When we want a \nameref{def:Metavariable} to only get a value under certain conditions, we have a special notation for that.
For example, say we want $n$ to be a natural number that is used in the grammar of \Cref{eq:DivAdd_Grammar}.
We can write this condition as:
\begin{equation}\label{eq:Conditional_Rules}
  \frac{n \in \NaturalNumbers}{n \EvaluatesTo \mathtt{asNat}(n)}
\end{equation}

The writers of these semantic grammars favor simplicity, so they might omit the \texttt{asNat} function by arguing ``separating between natural numbers and their textual representation is overly pedantic, as long as there is no ambiguity''.
They might write the rule in \Cref{eq:Conditional_Rules} as:
\begin{equation}\label{eq:Conditional_Rules_Simplified}
  \frac{n \in \NaturalNumbers}{n \EvaluatesTo n}
\end{equation}

\subsection{Recursion}\label{subsec:Semantics_Recursion}
If we want to construct the addition in \Cref{eq:DivAdd_Grammar}, then we can write:
\begin{equation}\label{eq:Semantics_Recursion}
  \frac{e_{1} \EvaluatesTo n_{1} \:\: e_{2} \EvaluatesTo n_{2}}{e_{1}+e_{2} \EvaluatesTo n_{1}+n_{2}}
\end{equation}

\subsection{Completeness}\label{subsec:Semantic_Completeness}
A semantic specification is \emph{complete} when all possible cases have been defined, in some way.
For example:
\begin{equation}\label{eq:Incomplete_Semantic_Rule}
  \frac{e_{1} \EvaluatesTo n_{1} \:\: e_{2} \EvaluatesTo n_{2}}{e_{1} \:\SemanticInput{\mathtt{div}}\: e_{2} \EvaluatesTo \Bigl\lfloor \frac{n_{1}}{n_{2}} \Bigr\rfloor}
\end{equation}
This rule ignores the case when $n_{2} = 0$, which means we divide by 0.
In mathematics, this is an undefined operation.

There are 2 ways to solve this problem:
\begin{enumerate}[noitemsep]
\item \textbf{Adding a Meta-Rule:} Add an informal rule that states if the \nameref{def:Semantics} of an operation is not defined, then execution aborts with an error.
  This is a solution, but it allows the language designer to solve corner cases with strange behavior.
  This also prevents any forms of error recovery, harming \nameref{subsec:Reliability}.
\item \textbf{Error Values:} Extend the $\EvaluatesTo$ \nameref{def:Evaluation_Relation} operator so that we can return \emph{error values} along with intended results.
  This may require adding \textbf{MANY} new rules to the language.
\end{enumerate}

\begin{example}[Lecture 3]{Natural Semantics}
  For the given \nameref{def:Context_Free_Grammar}, construct rules for the Natural~\nameref{def:Semantics} of this language?
  Then, perform a proof/derivation using those rules for the statement $\max (1+2) (1+1)$?
  Ignore the parentheses; those are only used to show that the 2 expressions are separate.
  Assume there is a process before applying the \nameref{def:Context_Free_Grammar} which handles the parentheses.
  \begin{align*}
    \Nonterminal{expr} &\longrightarrow \SemanticInput{\underline{\mathtt{num}}} \\
                       &\vert \Nonterminal{expr} \SemanticInput{+} \Nonterminal{expr} \\
                       &\vert \, \SemanticInput{\mathtt{\max}} \Nonterminal{expr} \Nonterminal{expr}
  \end{align*}
  \tcblower{}
  \textbf{NOTE}: Green will represent input to the rule and blue will represent rule output.
  When deriving statements with these rules, there may be additional colors to represent various values.

  The easiest rule to define is the rule for \SemanticInput{\underline{\texttt{num}}}.
  \begin{equation*}
    \frac{\SemanticInput{n} \in \SemanticInput{\underline{\mathtt{num}}}}{\SemanticInput{n} \EvaluatesTo \textcolor{Blue4}{n}} \textcolor{red}{(num)}
  \end{equation*}

  Next, we define the rule for the addition.
  \begin{equation*}
    \frac{\textcolor{Orange3}{e_{1}} \in \Nonterminal{expr} \: \textcolor{Orange3}{e_{2}} \in \Nonterminal{expr} \: \textcolor{Orange3}{e_{1}} \EvaluatesTo \textcolor{Blue4}{n_{1}} \: \textcolor{Orange3}{e_{2}} \EvaluatesTo \textcolor{Blue4}{n_{2}}}{\textcolor{Orange3}{e_{1}} \SemanticInput{\mathtt{+}} \textcolor{Orange3}{e_{2}} \EvaluatesTo \textcolor{Blue4}{n_{1} + n_{2}}} \textcolor{red}{(add)}
  \end{equation*}
  \begin{remark*}
    In this case, the expression's rule actually returns the computed number from the expression $n_{1} + n_{2}$, rather than the 2 expressions added together (which would be had if we had written $e_{1} + e_{2}$ instead).
  \end{remark*}

  \begin{remark*}
    From here on out, if there is $e_{x} \EvaluatesTo n_{x}$, then we also implicitly say $e_{x} \in \Nonterminal{expr}$.
  \end{remark*}

  Now we have to define the \texttt{max} operation.
  In this case, we need 2 rules to specify the 2 cases, when the left expression is the relative maximum, and when the right expression is the relative maximum.
  \begin{equation*}
    \frac{e_{1} \EvaluatesTo \textcolor{Blue4}{n_{1}} \: e_{2} \EvaluatesTo \textcolor{Blue4}{n_{2}} \: \textcolor{Blue4}{n_{1}} \geq \textcolor{Blue4}{n_{2}}}{\SemanticInput{\mathtt{\max}} \: e_{1} \; e_{2} \EvaluatesTo \textcolor{Blue4}{n_{1}}} \textcolor{red}{(\text{max left})}
  \end{equation*}
  \begin{equation*}
    \frac{e_{1} \EvaluatesTo \textcolor{Blue4}{n_{1}} \: e_{2} \EvaluatesTo \textcolor{Blue4}{n_{2}} \: \textcolor{Blue4}{n_{1}} \leq \textcolor{Blue4}{n_{2}}}{\SemanticInput{\mathtt{\max}} \: e_{1} \; e_{2} \EvaluatesTo \textcolor{Blue4}{n_{2}}} \textcolor{red}{(\text{max right})}
  \end{equation*}
  \begin{remark*}
    Although the 2 equations for determining the relative maxima both have an equality aspect, in pure mathematics, that's fine.
    In mathematical terms, if both rules apply, because $e_{1} == e_{2}$, then, both are technically maxima and should be returned.

    However, this will change during implementation, because one of the rules will be checked first, which necessitates that the rules be made mutually exclusive.
  \end{remark*}

  Now that all the necessary rules have been developed, we can apply them to the statement $\max (1+2) (1+1)$.
  \begin{equation*}
    % Derivation of max started
    \dfrac{
      % Expression 1 derivation
      \dfrac{
        % Expression 1 Left number Derivation
        \dfrac{\SemanticInput{1} \in \NaturalNumbers}{\SemanticInput{1} \EvaluatesTo \textcolor{Purple2}{n_{1} = 1}} \textcolor{red}{(num)} \:\:
        % Expression 1 Right number derivation
        \dfrac{\SemanticInput{2} \in \NaturalNumbers}{\SemanticInput{2} \EvaluatesTo \textcolor{Purple2}{n_{2} = 2}} \textcolor{red}{(num)}}
      % Expression 1 bottom of derivation
      {\textcolor{Orange3}{e_{1}} = \SemanticInput{1+2} \EvaluatesTo \textcolor{Burlywood4}{n_{1}} = \textcolor{Purple2}{n_{1}} + \textcolor{Purple2}{n_{2}} = \textcolor{Purple2}{1}+\textcolor{Purple2}{2}=\textcolor{Burlywood4}{3}} \textcolor{red}{(add)} \:\:
      % Expression 2 derivation
      \dfrac{
        % Expression 2 Left number Derivation
        \dfrac{\SemanticInput{1} \in \NaturalNumbers}{\SemanticInput{1} \EvaluatesTo \textcolor{Purple2}{n_{1} = 1}} \textcolor{red}{(num)} \:\:
        % Expression 2 Right number derivation
        \dfrac{\SemanticInput{1} \in \NaturalNumbers}{\SemanticInput{1} \EvaluatesTo \textcolor{Purple2}{n_{2} = 1}} \textcolor{red}{(num)}}
      % Expression 2 bottom of derivation
      {\textcolor{Orange3}{e_{2}} = \SemanticInput{1+1} \EvaluatesTo \textcolor{Burlywood4}{n_{2}} = \textcolor{Purple2}{n_{1}} + \textcolor{Purple2}{n_{2}} = \textcolor{Purple2}{1}+\textcolor{Purple2}{1} = \textcolor{Burlywood4}{2}} \textcolor{red}{(add)} \:\:
      % Checking n1 geq n2
      \textcolor{Burlywood4}{n_{1}} \geq \textcolor{Burlywood4}{n_{2}} = \textcolor{Burlywood4}{3} \geq \textcolor{Burlywood4}{2}}
    % Bottom of the max rule derivation
    {\SemanticInput{\mathtt{\max}} \textcolor{Orange3}{\underbrace{(\SemanticInput{1+2}\textcolor{Orange3}{)}}_{\textcolor{Orange3}{e_{1}}}} \textcolor{Orange3}{\underbrace{\textcolor{Orange3}{(\SemanticInput{1+1}\textcolor{Orange3}{)}}}_{\textcolor{Orange3}{e_{2}}}} \EvaluatesTo \textcolor{Burlywood4}{n_{1}} = \textcolor{Blue4}{3}} \textcolor{red}{(\text{max left})}
  \end{equation*}

  We chose the \textcolor{red}{max left} rule only \textbf{after} the values of $\textcolor{Orange3}{e_{1}}$ and $\textcolor{Orange3}{e_{2}}$ were calculated.
\end{example}

\subsection{Language with Variables}\label{subsec:Language_with_Variables}
For this section, we are going to use \Cref{eq:LetAdd_Grammar} as our \nameref{def:Context_Free_Grammar}.
\begin{blackbox}
  \begin{equation}\label{eq:LetAdd_Grammar}
    \begin{aligned}
      \Nonterminal{expr} &\rightarrow nat \\
      &\vert id \\
      &\vert \Nonterminal{expr} \SemanticInput{\mathtt{+}} \Nonterminal{expr} \\
      &\vert \SemanticInput{\mathtt{let}}\: id = \Nonterminal{expr} \SemanticInput{\mathtt{in}} \Nonterminal{expr} \\
    \end{aligned}
  \end{equation}
\end{blackbox}

\subsubsection{Environments}\label{subsubsec:Semantic_Environments}
\begin{definition}[Environment]\label{def:Semantic_Environment}
  An \emph{environment} is a mathematical object that tells us the bindings of variables.
  We will use the following notation:
  \begin{equation}\label{eq:Environment}
    E_{n} = \lbrace \SemanticInput{\mathtt{v_{1}}} \mapsto m_{1}, \SemanticInput{\mathtt{v_{2}}} \mapsto m_{2} \rbrace
  \end{equation}
  where $v$ is a variable, and $n$ is the number of variables present.
  So, for example:
  \begin{equation*}
    E_{2} = \lbrace \SemanticInput{\mathtt{a}} \mapsto 1, \SemanticInput{\mathtt{b}} \mapsto 7 \rbrace
  \end{equation*}

  \begin{remark}[Empty \nameref*{def:Semantic_Environment}]\label{rmk:Empty_Environment}
    The \emph{empty environment} is written
    \begin{equation}\label{eq:Empty_Environment}
      E_{\emptyset} = \lbrace \rbrace
    \end{equation}
  \end{remark}
\end{definition}

To retrieve an value from an \nameref{def:Semantic_Environment}, where the \nameref{def:Semantic_Environment} being used is \Cref{eq:Environment}, we write:
\begin{equation}\label{eq:Semantic_Environment_Get_Variable}
  E(\SemanticInput{\mathtt{a}}) \Rightarrow 1
\end{equation}

If we want to update an \nameref{def:Semantic_Environment} with a new binding,
\begin{subequations}\label{eq:Update_Environment}
  \begin{equation}\label{subeq:Update_Environment_New_Variable}
    E_{2} [\SemanticInput{\mathtt{c}} \mapsto 42] = \lbrace \SemanticInput{\mathtt{a}} \mapsto 1, \SemanticInput{\mathtt{b}} \mapsto 7, \SemanticInput{\mathtt{c}} \mapsto 42 \rbrace
  \end{equation}
  \begin{equation}\label{subeq:Update_Environment_Update_Variable}
    E_{2} [\SemanticInput{\mathtt{a}} \mapsto 0] = \lbrace \SemanticInput{\mathtt{a}} \mapsto 0, \SemanticInput{\mathtt{b}} \mapsto 7\rbrace
  \end{equation}
\end{subequations}

We define this as:
\begin{equation}\label{eq:Update_Environment_Definition}
  E[x \mapsto v](y) =
  \begin{cases}
    v & \Longleftrightarrow x=y \\
    E(y) & \text{otherwise}
  \end{cases}
\end{equation}
This means that if $x=y$, then $x$ is remapped (updated) to the value $v$.

\subsubsection{Defining \nameref*{def:Semantics} with Environments}\label{subsubsec:Define_Semantics_with_Environments}
\nameref{def:Semantic_Environment}s, $E$, can now become parameters to our \nameref{def:Evaluation_Relation}s, $\EvaluatesTo$.
This is written
\begin{equation}\label{eq:Define_Semantics_with_Environments}
  E \vdash p \EvaluatesTo n
\end{equation}
Which means $p$ can be drawn from the environment $E$, and should evaluate to a number, $n$ ($n$ is defined elsewhere).
If $p$ is \textbf{not} defined in the \nameref{def:Semantic_Environment}, but is possible to evaluate, for example $p=4$, then nothing happens with the \nameref{def:Semantic_Environment}.

Essentially, using $E \vdash p$ means that $p$ has access to the \nameref{def:Metavariable}s present in the \nameref{def:Semantic_Environment}, but $p$ is not required to be drawn from $E$.

\begin{example}[Lecture 3]{Natural Semantics with Environments}
  Given the \nameref{def:Context_Free_Grammar},
  \begin{align*}
    \Nonterminal{expr} &\longrightarrow \SemanticInput{\underline{\mathtt{num}}} \\
                       &\vert \Nonterminal{expr} \SemanticInput{+} \Nonterminal{expr} \\
                       &\vert \, \SemanticInput{\mathtt{\max}} \Nonterminal{expr} \Nonterminal{expr} \\
                       &\vert \, \SemanticInput{\text{\texttt{let \underline{id}}} = } \Nonterminal{expr} \SemanticInput{\mathtt{in}} \Nonterminal{expr} \\
                       &\vert \, \SemanticInput{\mathtt{\underline{id}}}
  \end{align*}
  Compute the result of $\text{\texttt{let }} \mathtt{x = 2} \text{\texttt{ in }} \mathtt{x+3}$?
  \tcblower{}
  We will start by using the rules that we defined in \Cref{ex:Natural Semantics}.
  We will also need to add new semantic rules for \SemanticInput{\texttt{id}} and the \texttt{let} statement.

  \begin{equation*}
    \frac{\SemanticInput{i} \in \SemanticInput{\mathtt{\underline{id}}}}{E \vdash \SemanticInput{i} \EvaluatesTo E(\SemanticInput{i})} \textcolor{red}{(id)}
  \end{equation*}
  \begin{equation*}
    \frac{E \vdash e_{1} \EvaluatesTo \textcolor{Blue4}{n_{1}} \:\: E[\SemanticInput{i} \mapsto \textcolor{Blue4}{n_{1}}] \vdash e_{1} \EvaluatesTo e_{2}}{E \vdash \SemanticInput{\text{\texttt{let }}\mathtt{i =}} \; e_{1} \; \SemanticInput{\mathtt{\text{in}}} \; e_{2} \EvaluatesTo \textcolor{Blue4}{n_{2}}} \textcolor{red}{(let)}
  \end{equation*}

  We start by applying the most applicable rule, the \textcolor{red}{(let)} rule.
  \begin{equation*}
    \dfrac{
      % The number used in the let-assignment
      \dfrac{\SemanticInput{2} \in \NaturalNumbers}{E_{\emptyset} \vdash \SemanticInput{2} \EvaluatesTo \textcolor{Blue4}{2}} \textcolor{red}{(num)} \:\:
      % The addition operation inside the in-block
      \dfrac{
        % The dereferencing of the ID and grabbing the value mapped to the variable
        \dfrac{\SemanticInput{\mathtt{x}} \in \SemanticInput{\mathtt{\underline{id}}}}
        % The bottom of the dereferencing of the ID
        {E_{\emptyset} [\SemanticInput{\mathtt{x}} \mapsto \textcolor{Blue4}{2}] \vdash \SemanticInput{\mathtt{x}} \EvaluatesTo E(\SemanticInput{\mathtt{x}}) \Rightarrow \textcolor{Orange3}{2}} \textcolor{red}{(id)} \:\:
        % The number, 3, added with x in the in-block
        \dfrac{\SemanticInput{3} \in \NaturalNumbers}
        % The bottom of the num derivation
        {E_{\emptyset} [\SemanticInput{\mathtt{x}} \mapsto \textcolor{Blue4}{2}] \vdash \SemanticInput{3} \EvaluatesTo \textcolor{Orange3}{3}} \textcolor{red}{(num)}}
      % Bottom of the addition derivation
      {E_{\emptyset}[\SemanticInput{\mathtt{x}} \mapsto \textcolor{Blue4}{2}] \vdash \SemanticInput{\mathtt{x+3}} \EvaluatesTo \textcolor{Orange3}{2+3} = \textcolor{Burlywood4}{5}} \textcolor{red}{(add)}}
    % Bottom portion of let expression
    {E_{\emptyset} \vdash \SemanticInput{\text{\texttt{let }}} \mathtt{\SemanticInput{x=}\,2} \SemanticInput{\text{\texttt{ in }}} \mathtt{x+3} \EvaluatesTo \textcolor{Burlywood4}{5}} \textcolor{red}{(let)}
  \end{equation*}

  Once we've applied the \textcolor{red}{(let)} rule, we apply the \textcolor{red}{(num)} rule, because the expression being assigned is just a number.
  The result of that expression evaluation is piped into the next \textcolor{red}{(add)} derivation.
  From there, it is mapped to the input \SemanticInput{\texttt{x}}, and the \texttt{x+3} is calculated.
  The first thing is to get the value of \SemanticInput{\texttt{x}} from the mapping with the \textcolor{red}{(id)} rule.
  Once the $E(\SemanticInput{\mathtt{x}})$ returns the value of the variable \SemanticInput{\texttt{x}}, we can use it.
  The other operand in the addition is found using the \textcolor{red}{(num)} rule.
  Then the addition occurs, and the derivation completes.
\end{example}
  
%%% Local Variables:
%%% mode: latex
%%% TeX-master: "../EDAP05-Concepts_Programming_Languages-Reference_Sheet"
%%% End:


\section{Names}\label{sec:Names}
\emph{Names} or \emph{identifiers} are, obviously, names given to things.
They can identify:
\begin{itemize}[noitemsep]
\item Variables
\item Subprograms
\item Formal Parameters
\item Other program constructs
\end{itemize}

\subsection{Issues}\label{subsec:Names_Issues}
There are 2 questions that need to be asked when designing the names or identifiers possible in a language.
\begin{enumerate}[noitemsep]
\item Are names case-sensitive? For example, are these identifiers different?
  \begin{itemize}[noitemsep]
  \item \texttt{myvariable}
  \item \texttt{MYVARIABLE}
  \item \texttt{MyVariable}
  \item \texttt{myVariable}
  \end{itemize}
\item Are the special words of the language \nameref{def:Reserved_Word}s or are they \nameref{def:Keyword}s?
\end{enumerate}

\subsection{Name Forms}\label{subsec:Name_Forms}
How is a name/identifier defined?
\begin{itemize}[noitemsep]
\item Is there a character limit on the identifier/name?
\item Are all characters in the identifier/name significant?
\item What characters are allowed in the identifier/name?
\item Are there special characters required by a language?
  \begin{itemize}[noitemsep]
  \item \texttt{\$} being required in front of identifiers in PHP
  \item \texttt{\$}, \texttt{@}, \texttt{\%} specifying a ``type''  in Perl
  \item \texttt{@} and \texttt{@@} to denote an instance or class variable in Ruby, respectively
  \end{itemize}
\end{itemize}

Some languages are \emph{case-sensitive}.
C, Java, C++, etc.\ would all treat \texttt{rose}, \texttt{ROSE}, and \texttt{Rose} differently.
This could be a detriment to readability, because names that \textit{look} similar are actually not.
In terms of writability, the programmer must remember the exact typecasing of the identifier/name.

\subsection{Special Names}\label{subsec:Special_Names}
There are \nameref{def:Reserved_Word}s and \nameref{def:Keyword}s.
They are similar in that the programming language specification defines that these words have special meanings when constructing programs.
However, the 2 differ in how these words can potentially be reused.

\begin{definition}[Keyword]\label{def:Keyword}
  \emph{Keyword}s are words that are defined by the language constructors to have some special meaning.
  However, it only has these special meanings in \textbf{\emph{certain contexts}}.
  This means you can define a keyword as a variable and use it together with the keyword.
  For example, this is a perfectly valid piece of Fortran code:
\begin{minted}[frame=lines,linenos]{fortran}
Integer Apple
Integer = 4
\end{minted}
\end{definition}

\begin{definition}[Reserved Word]\label{def:Reserved_Word}
  \emph{Reserved word}s are words that are reserved by the language constructors because those particular words have a meaning in the language.
  These words cannot be used as identifiers for \textbf{ANYTHING} else.
  For example:
  \begin{itemize}[noitemsep]
  \item \texttt{while}
  \item \texttt{class}
  \item \texttt{for}
  \end{itemize}

  \begin{remark}[Too Many \nameref{def:Reserved_Word}s]\label{rmk:Too_Many_Reserved_Words}
    The potential problem with \nameref{def:Reserved_Word}s is that if a language has a large number of reserved words, the user might have a hard time creating names for themselves.
    Unfortunately, the most commonly chosen words by programs are usually \nameref{def:Reserved_Word}s.
    For example,
    \begin{itemize}[noitemsep]
    \item \texttt{LENGTH}
    \item \texttt{BOTTOM}
    \item \texttt{DESTINATION}
    \item \texttt{COUNT}
    \end{itemize}
  \end{remark}
\end{definition}

\subsection{Variables}\label{subsec:Variables}
\begin{definition}[Variable]\label{def:Variable}
  A program \emph{variable} is an abstraction of a computer \nameref{def:Memory} cell or a collection of \nameref{def:Memory} cells.
  A variable can be characterized by a sextuple of attributes:
  \begin{enumerate}[noitemsep]
  \item \nameref{subsubsec:Variable_Name}
  \item \nameref{subsubsec:Variable_Address}
  \item \nameref{subsubsec:Variable_Value}
  \item \nameref{subsubsec:Variable_Type}
  \item \nameref{subsubsec:Storage_Bindings_and_Lifetime}
  \item \nameref{subsec:Variable_Scope}
  \end{enumerate}
\end{definition}

\subsubsection{Name}\label{subsubsec:Variable_Name}
Most \nameref{def:Variable}s have names.
These are symbolic references to the value that is actually stored.
There are various issues that may arise with the name of a variable, which were discussed earlier.

\subsubsection{Address}\label{subsubsec:Variable_Address}
\begin{definition}[Address]\label{def:Variable_Address}
  The \emph{address} of a \nameref{def:Variable} is the machine's memory address with which the \nameref{def:Variable} is associated.

  The address of a variable is sometimes called its \emph{L-Value}.
  This is because the address is required when the name of a variable appears on the left-hand side of an assignment statement.

  \begin{remark}[Alias]
    An \emph{alias} is having another \nameref{def:Variable} have the same \nameref{def:Variable_Address}, so the 2 \nameref{def:Variable}s point to the same value in \nameref{def:Memory}.
  \end{remark}
\end{definition}

For some languages, it is possible for the same \nameref{def:Variable} to be associated with different addresses at different times during the \nameref{def:Variable}'s lifetime.

\subsubsection{Type}\label{subsubsec:Variable_Type}
\begin{definition}[Type]\label{def:Variable_Type}
  The \emph{type} of a \nameref{def:Variable} determines the range of values that \nameref{def:Variable} can store.
  For example, the \texttt{int} type in Java specifies a value range of $-2147483648$ to $2147483647$.
  It is a 32-bit signed integer.
\end{definition}

\subsubsection{Value}\label{subsubsec:Variable_Value}
\begin{definition}[Value]\label{def:Variable_Value}
  The \emph{value} of a \nameref{def:Variable} is the contents of the \nameref{def:Memory} cell or cells associated with the \nameref{def:Variable}.
  The value of a variable is sometimes called it \emph{R-Value}.
  This is because the value of the \nameref{def:Variable} is required on the right-hand side of an assignment statement.
  To access the \textit{r}-value, the \textit{l}-value must be determined first.
  
  \begin{remark}[Abstract Memory Cells]\label{rmk:Abstract_Memory_Cells}
    While in hardware, the individual sizes of \nameref{def:Memory} are fixed, we can think of \nameref{def:Memory} as having \emph{abstract memory cells}, that can accomodate anything we attempt to put into \nameref{def:Memory}.
    This means that a single-precision floating point number technically takes up 4 bytes, 32 bits, of \nameref{def:Memory} cells, that number only takes one abstract memory cell.
  \end{remark}
\end{definition}

\subsection{Binding}\label{subsec:Binding}
\begin{definition}[Binding]\label{def:Binding}
  A \emph{binding} is an association between an attribute and an entity.
  This association can be between:
  \begin{itemize}[noitemsep]
  \item A variable
    \begin{itemize}[noitemsep]
    \item Its type
    \item Its value
    \end{itemize}
  \item An operation
    \begin{itemize}[noitemsep]
    \item Its symbol
    \end{itemize}
  \end{itemize}

  The time at which a binding occurs is called the \nameref{def:Binding_Time}.
\end{definition}

\begin{definition}[Binding Time]\label{def:Binding_Time}
  The time at which \nameref{def:Binding} occurs is called the \emph{binding time}.
  These include:
  \begin{itemize}[noitemsep]
  \item Language Design Time
    \begin{itemize}[noitemsep]
    \item Defining \texttt{*} to represent multiplication
    \end{itemize}
  \item Language Implementation Time
    \begin{itemize}[noitemsep]
    \item Having an \texttt{int} in C be a 32-bit signed integer
    \end{itemize}
  \item Compiler Time
    \begin{itemize}[noitemsep]
    \item The type of a variable in a Java program
    \end{itemize}
  \item Link Time
    \begin{itemize}[noitemsep]
    \item A call to a library subprogram is bound to the subprogram code
    \end{itemize}
  \item Load Time
    \begin{itemize}[noitemsep]
    \item \nameref{def:Variable} bound when loaded into \nameref{def:Memory}
    \item Could happen at run time too
    \end{itemize}
  \item Run Time
    \begin{itemize}[noitemsep]
    \item \nameref{def:Variable} bound when loaded into \nameref{def:Memory}
    \item Could happen at compile time too
    \end{itemize}
  \end{itemize}
\end{definition}

We need to know the \nameref{def:Binding_Time}s for the attributes of a program to understand the semantics of the programming language.

\subsubsection{Binding of Attributes to \nameref*{def:Variable}s}\label{subsubsec:Binding_Attributes_Variables}
\begin{definition}[Static]\label{def:Static_Variable_Binding}
  A \nameref{def:Binding} is \emph{static} if the \nameref{def:Binding} first occurs before run time begins and remains unchanged throughout program execution.
  An example of this is declaring a \nameref{def:Variable} as an \texttt{int} in C.
  Throughout the whole C program, that \nameref{def:Variable} can only hold signed 32-big integers.
\begin{minted}[frame=lines,linenos]{c}
int x = 4;
float x = 4.0; // Error here, x already declared
x = 4.0 // Error here, x is of int type
\end{minted}
\end{definition}

\begin{definition}[Dynamic]\label{def:Dynamic_Variable_Binding}
  A \nameref{def:Binding} is \emph{dynamic} if the \nameref{def:Binding} occurs during run time, or can change in the course of program execution.
  An example of this is declaring a \nameref{def:Variable} in Python.
\begin{minted}[frame=lines,linenos]{python3}
x = 4
x = [1, 2, 3]
x = 'dynamically bound string'
\end{minted}
  All three lines have a variable declaration, where the \nameref{def:Binding} occur during the program's execution and changed during it.
\end{definition}

We are only concerned with the distinction between \nameref{def:Static_Variable_Binding} and \nameref{def:Dynamic_Variable_Binding} \nameref{def:Variable} \nameref{def:Binding}.
Meaning, we will ignore how hardware may bind and unbind things repeatedly when it is switching and moving things around.

\subsubsection{Type \nameref*{subsec:Binding}s}\label{subsubsec:Type_Bindings}
Before a \nameref{def:Variable} can be used or referenced in a program, its \nameref{def:Variable_Type} must be declared.
A \nameref{def:Variable}'s \emph{type} determines the range of values that can be stored in the \nameref{def:Variable}.
In a more abstract sense, it also determines what kind of operations make sense and are possible to use on these \nameref{def:Variable}s.
There are 2 important aspects of this \nameref{def:Binding}:
\begin{enumerate}[noitemsep]
\item How the \nameref{def:Variable} \nameref{def:Variable_Type} is specified
\item When the \nameref{def:Binding} takes place
\end{enumerate}

There are 2 ways to bind \nameref{def:Variable_Type}s to \nameref{def:Variable}s:
\begin{enumerate}[noitemsep]
\item \nameref{par:Static_Variable_Type_Binding}
  \begin{itemize}[noitemsep]
  \item \nameref{def:Explicit_Static_Variable_Type_Binding}
  \item \nameref{def:Implicit_Static_Variable_Type_Binding}
  \end{itemize}
\item \nameref{par:Dynamic_Variable_Type_Binding}
\end{enumerate}

\paragraph{Static \nameref*{def:Variable_Type} \nameref*{subsec:Binding}}\label{par:Static_Variable_Type_Binding}
\begin{definition}[Static]\label{def:Static_Variable_Type_Binding}
  \emph{Static} \nameref{def:Binding} of \nameref{def:Variable}s means that the \nameref{def:Variable_Type} of a \nameref{def:Variable} is given to the program, either \nameref{def:Explicit_Static_Variable_Type_Binding}ly or \nameref{def:Implicit_Static_Variable_Type_Binding}ly before run time begins.
  Once the \nameref{def:Variable_Type} is declared, it cannot be changed throughout the entire program's execution.

  There are 2 ways to \nameref{def:Static_Variable_Type_Binding}ly bind a \nameref{def:Variable_Type} to a \nameref{def:Variable}:
  \begin{enumerate}[noitemsep]
  \item \nameref{def:Explicit_Static_Variable_Type_Binding}ly
  \item \nameref{def:Implicit_Static_Variable_Type_Binding}ly
  \end{enumerate}
\end{definition}

\begin{definition}[Explicit]\label{def:Explicit_Static_Variable_Type_Binding}
  An \emph{explicit} \nameref{def:Static_Variable_Type_Binding} \nameref*{def:Variable_Type} \nameref*{def:Binding} is a statement that explicitly sets each \nameref{def:Variable} to its respective \nameref{def:Variable_Type}.
  These are statements in a program that lists variable names and specifies that they are of a particular \nameref{def:Variable_Type}.
  For example,
\begin{minted}[frame=lines,linenos]{c}
int x = 0;
float x = 0.0;
char x = 'x';
\end{minted}
\end{definition}

\begin{definition}[Implicit]\label{def:Implicit_Static_Variable_Type_Binding}
  An \emph{implicit} \nameref{def:Static_Variable_Type_Binding} \nameref*{def:Variable_Type} \nameref*{def:Binding} declaration associates \nameref{def:Variable}s with \nameref{def:Variable_Type}s through default conventions, rather than \nameref{def:Explicit_Static_Variable_Type_Binding} declaration statements.
  The first appearance of a \nameref{def:Variable} name is its implicit declaration.

  \begin{remark}[Effects on \nameref*{subsec:Reliability}]
    While \nameref{def:Implicit_Static_Variable_Type_Binding} can be helpful for programmers, they can be quite detrimental to \nameref{subsec:Reliability} because the compilation process cannot determine some type errors and some programmer errors.
  \end{remark}
\end{definition}

\nameref{def:Implicit_Static_Variable_Type_Binding} declarations are handled by the language processor (\nameref{def:Compiler} or Interpreter).
There are several ways to have \nameref{def:Implicit_Static_Variable_Type_Binding} declarations work, some of which are:
\begin{itemize}[noitemsep]
\item Naming conventions
  \begin{itemize}[noitemsep]
  \item In \textbf{Fortran}, if an identifier starts with
    \begin{itemize}[noitemsep]
    \item \texttt{I}, \texttt{J}, \texttt{K}, \texttt{L}, \texttt{M}, or \texttt{N}, or their lowercase versions, it is \nameref{def:Implicit_Static_Variable_Type_Binding}ly declared to be an \texttt{Integer} type.
    \item Otherwise, it is \nameref{def:Implicit_Static_Variable_Type_Binding}ly declared to be a \texttt{Real} type.
    \end{itemize}
  \item In \textbf{Perl}, an identifier must be preceded by a special character denoting the \nameref{def:Variable_Type}. This method forms separate namespaces for each \nameref{def:Variable} \nameref{def:Variable_Type}.
    \begin{itemize}[noitemsep]
    \item \texttt{\$}, is a scalar. This holds numbers and strings
    \item \texttt{@}, is an array.
    \item \texttt{\%}, is a hash structure.
    \item The separate namespaces means that all 3 of these variables are considered unique, and potentially unrelated.
      \begin{itemize}[noitemsep]
      \item \texttt{\$apple}
      \item \texttt{@apple}
      \item \texttt{\%apple}
      \end{itemize}
    \end{itemize}
  \end{itemize}
  
\item Context or type inference
  \begin{itemize}[noitemsep]
  \item In \textbf{C\#}, a \texttt{var} declaration for a \nameref{def:Variable} must include an initial value, which determines the \nameref{def:Variable_Type} of the \nameref{def:Variable}.
\begin{minted}[frame=lines,linenos]{csharp}
var sum = 0;
var total = 0.0;
var name = "Fred";
\end{minted}
  \item \texttt{sum}, \texttt{total}, and \texttt{name} are an \texttt{int}, \texttt{float}, and \texttt{string}, respectively.
  \end{itemize}
\end{itemize}

\begin{remark*}
  Both \nameref{def:Explicit_Static_Variable_Type_Binding} and \nameref{def:Implicit_Static_Variable_Type_Binding} declarations create \nameref{def:Static_Variable_Binding} \nameref{def:Binding}s to \nameref{def:Variable_Type}s.
\end{remark*}

\paragraph{Dynamic \nameref*{def:Variable_Type} \nameref*{subsec:Binding}}\label{par:Dynamic_Variable_Type_Binding}
With \nameref*{par:Dynamic_Variable_Type_Binding}, the \nameref{def:Variable_Type} of a \nameref{def:Variable}:
\begin{itemize}[noitemsep]
\item Is not specified by a declaration statement
\item Cannot be determined by the spelling of the \nameref{def:Variable}'s name
\end{itemize}

\begin{definition}[Dynamic]\label{def:Dynamic_Variable_Type_Binding}
  A \emph{dynamic} \nameref{def:Binding} happens when a \nameref{def:Variable} is bound to a \nameref{def:Variable_Type} \textbf{when it is assigned a \nameref{def:Variable_Value}.}
  Such an assignment might also bind the \nameref{def:Variable} to an \nameref{def:Variable_Address}.

  Any \nameref{def:Variable} can be assigned any \nameref{def:Variable_Type}.
  A \nameref{def:Variable}'s \nameref{def:Variable_Type} can be changed any number of times during program execution.

  The name of the \nameref{def:Variable} is bound to the \nameref{def:Variable}, then the \nameref{def:Variable} is bound to a \nameref{def:Variable_Type} and given its \nameref{def:Variable_Value}.

  2 programming languages that use this are:
  \begin{enumerate}[noitemsep]
  \item Python
  \item Ruby
  \end{enumerate}

  \begin{remark}[Benefits of \nameref*{def:Dynamic_Variable_Type_Binding} \nameref*{def:Binding}]\label{rmk:Dynamic_Variable_Type_Bindign-Benefits}
    The primary benefit of having \nameref{def:Dynamic_Variable_Type_Binding} \nameref{def:Binding} is the programming flexibility it provides.
  \end{remark}

  \begin{remark}[Drawbacks of \nameref*{def:Dynamic_Variable_Type_Binding} \nameref*{def:Binding}]\label{rmk:Dynamic_Variable_Type_Binding-Drawbacks}
    The 2 major disadvantages are:
    \begin{enumerate}[noitemsep]
    \item Programs are less reliable, because error-detection of the compiler/interpreter is diminished relative to a compiler/interpreter for a language with \nameref{def:Static_Variable_Type_Binding} \nameref{def:Variable_Type} \nameref{def:Binding}s.
    \item The \nameref{subsec:Cost} is quite high because of the \nameref{def:Type_Checking} that must occur at run time. Also, every variable must have a run-time descriptor to describe the \nameref{def:Variable}'s current \nameref{def:Variable_Type}.
    \end{enumerate}
  \end{remark}
\end{definition}

\begin{remark*}
  \nameref{def:Dynamic_Variable_Type_Binding} \nameref{def:Variable_Type} \nameref{def:Binding} is usually implemented with \nameref{subsec:Interpretation}.
  This is because:
  \begin{itemize}[noitemsep]
  \item The overall \nameref{subsec:Cost} of \nameref{def:Variable_Type} handing is hidden by the \nameref{subsec:Cost} of the interpreter.
  \item The \nameref{def:Variable_Type} of an operation's operands must be known to translate the instruction to the correct machine code instruction, which isn't possible with \nameref{def:Dynamic_Variable_Type_Binding} \nameref{def:Variable_Type} \nameref{def:Binding}.
  \end{itemize}
\end{remark*}

\subsubsection{Storage \nameref*{subsec:Binding}s and Lifetime}\label{subsubsec:Storage_Bindings_and_Lifetime}
The \nameref{def:Memory} cell to which a \nameref{def:Variable} is bound must be pulled from the pool of available \nameref{def:Memory}.
The act of binding the \nameref{def:Variable_Value} to a \nameref{def:Variable} is called \nameref{def:Variable_Memory_Allocation}.
The act of unbinding is called \nameref{def:Variable_Memory_Deallocation}.

\begin{definition}[Allocation]\label{def:Variable_Memory_Allocation}
  \emph{Allocation} is the act of binding a \nameref{def:Variable_Value} to a \nameref{def:Memory} cell for a \nameref{def:Variable}.
\end{definition}

\begin{definition}[Deallocation]\label{def:Variable_Memory_Deallocation}
  \emph{Deallocation} is the process of placing a \nameref{def:Memory} cell that has been unbound from a variable back into the pool of available \nameref{def:Memory}.
\end{definition}

\begin{definition}[Lifetime]\label{def:Variable_Lifetime}
  The \emph{lifetime} of a \nameref{def:Variable} is the time in which the \nameref{def:Variable} is bound to a \nameref{def:Memory} cell.
  The lifetime of a \nameref{def:Variable} starts when it is bound to a cell and ends when it has been unbound from that cell.

  We will split the discussion of \nameref*{subsubsec:Storage_Bindings_and_Lifetime} of scalar \nameref{def:Variable}s into 4 categories, according to their \nameref{def:Variable_Lifetime}s.
  \begin{itemize}[noitemsep]
  \item \nameref{par:Static_Variable_Binding_Lifetime}
  \item \nameref{par:Stack-Dynamic_Variable_Binding_Lifetime}
  \item \nameref{par:Explicit_Heap-Dynamic_Variable_Binding_Lifetime}
  \item \nameref{par:Implicit_Heap-Dynamic_Variable_Binding_Lifetime}
  \end{itemize}
\end{definition}

\paragraph{Static Variables}\label{par:Static_Variable_Binding_Lifetime}
\begin{definition}[Static Variable]\label{def:Static_Variable_Binding_Lifetime}
  \emph{Static variable}s are those that are bound to \nameref{def:Memory} cells before program execution begins and remain bound until the program terminates.
  They are placed in the ``static'' section of \nameref{def:Memory}.
  \nameref{def:Static_Variable_Binding_Lifetime}s can be used as globally accessible \nameref{def:Variable}s, or ensure that subprograms are history-sensitive.

  The pros and cons of \nameref{def:Static_Variable_Binding_Lifetime}s are:
  \begin{itemize}[noitemsep]
  \item Pros
    \begin{itemize}[noitemsep]
    \item Efficiency. All \nameref{def:Memory} addressing is done with absolute addresses, making things very fast.
    \item No cost to allocate and deallocate the \nameref{def:Memory} during run-time.
    \item Programs can be history sensitive.
    \end{itemize}
  \item Cons
    \begin{itemize}[noitemsep]
    \item Reduced flexibility. If there is a language that only has \nameref{def:Static_Variable_Binding_Lifetime}s, then recursive subprograms are impossible.
    \item \nameref{def:Memory} cannot be shared between inactive and active subprograms.
    \end{itemize}
  \end{itemize}
\end{definition}

\begin{remark*}
  In C and C++, \texttt{static} can be set on functions, making the \nameref{def:Variable}s declared in the function \nameref{def:Static_Variable_Binding_Lifetime}.
\end{remark*}

\begin{remark*}
  In Java, C++, and C\#, \texttt{static} can appear on classes, meaning class \nameref{def:Variable}s are created statically some time before the class is first instantiated.
\end{remark*}

\paragraph{Stack-Dynamic Variables}\label{par:Stack-Dynamic_Variable_Binding_Lifetime}
\begin{definition}[Stack-Dynamic Variable]\label{def:Stack-Dynamic_Variable_Binding_Lifetime}
  \emph{Stack-dynamic variable}s are those whose storage \nameref{def:Binding}s are created when their declaration statements are elaborated.
  These are allocated from the run-time \nameref{def:Call_Stack}.
  Thus, when a function on the \nameref{def:Call_Stack} is \texttt{return}ed from, all the variables here lose their value.

  \begin{remark}
    These are the variables that are most commonly used, and are usually function-local variables
  \end{remark}

  \begin{remark}
    In languages that allow for variable declaration anywhere in the function, like Java and C++, the \nameref{def:Stack-Dynamic_Variable_Binding_Lifetime}s may be bound to storage at the beginning of the block, thus starting the variable's \nameref{def:Variable_Lifetime}.
    In these cases, the \nameref{def:Variable} becomes visible at the declaration, but the storage \nameref{def:Binding} occurs when the block begins execution.
    So, it is both in \nameref{def:Variable_Scope} and has begun its \nameref{def:Variable_Lifetime}, but has no useful value.
  \end{remark}

  The advantages and disadvantages of \nameref{def:Stack-Dynamic_Variable_Binding_Lifetime}s, compared to \nameref{def:Static_Variable_Binding_Lifetime}s, are:
  \begin{itemize}[noitemsep]
  \item Advantages
    \begin{itemize}[noitemsep]
    \item Allows for recursive subprograms that have local variables
    \item All subprograms can share the same memory space for their locals, allowing for a smaller memory footprint, by only having some variables bound to storage at once.
    \end{itemize}
  \item Disadvantages
    \begin{itemize}[noitemsep]
    \item Runtime overhead of \nameref{def:Variable_Memory_Allocation} and \nameref{def:Variable_Memory_Deallocation}.
    \item Slower accessing of \nameref{def:Stack-Dynamic_Variable_Binding_Lifetime}s because of indirect addressing.
    \item Subprograms cannot be history-sensitive with just \nameref{def:Stack-Dynamic_Variable_Binding_Lifetime}s.
    \end{itemize}
  \end{itemize}
\end{definition}

\begin{definition}[Elaboration]\label{def:Variable_Storage_Binding_Elaboration}
  \emph{Elaboration} of a \nameref{def:Variable} declaration refers to the storage \nameref{def:Variable_Memory_Allocation} and \nameref{def:Binding} process indicated by the declaration, which takes place when execution reaches that code.

  This occurs at run time.
\end{definition}

\paragraph{Explicit Heap-Dynamic Variables}\label{par:Explicit_Heap-Dynamic_Variable_Binding_Lifetime}
\begin{definition}[Explicit Heap-Dynamic Variable]\label{def:Explicit_Heap-Dynamic_Variable_Binding_Lifetime}
  \emph{Explicit heap-dynamic variables} are nameless (abstract) memory cells that are allocated and deallocated by explicit run-time instructions written by the programmer.
  These \nameref{def:Variable}s are allocated to and deallocated from the \nameref{def:Heap}.
  They can only be referenced through pointers or reference variables.

  The pointer/reference can only be created and returned by:
  \begin{itemize}[noitemsep]
  \item An operator (in C++), \texttt{new}
  \item A subprogram (in C), \texttt{malloc}
  \end{itemize}

  Some languages include ways to destroy these pointers/references:
  \begin{itemize}[noitemsep]
  \item An operator (in C++), \texttt{delete}
  \item A subprogram (in C), \texttt{free}
  \end{itemize}

  The advantages and disadvantages of these types of \nameref{def:Variable}s are:
  \begin{itemize}[noitemsep]
  \item Advantages
    \begin{itemize}[noitemsep]
    \item \nameref{def:Explicit_Heap-Dynamic_Variable_Binding_Lifetime} are often used to construct dynamic data structures, like linked lists and trees. These are built conveniently using pointers and data.
    \end{itemize}
  \item Disadvantages
    \begin{itemize}[noitemsep]
    \item The difficulty of using pointer/reference variables correctly.
    \item The \nameref{subsec:Cost} of using these pointers/reference variables.
    \item Complexity of the required storage management implementation (Although, this is a question of \nameref{def:Heap} management, which is costly and complicated and completely separate discussion).
    \end{itemize}
  \end{itemize}
\end{definition}

An example of an \nameref{def:Explicit_Heap-Dynamic_Variable_Binding_Lifetime} is shown below.
\begin{minted}[frame=lines,linenos]{c++}
int *intnode; // Create a pointer
intnode = new int; // Create the heap-dynamic variable
...
delete intnode; // Deallocate the heap-dynamic variable to which intnode points
\end{minted}

The \nameref{def:Heap} is highly disorganized because of the unpredictability of its use.
There are 2 ways to manage the \nameref{def:Heap}:
\begin{enumerate}[noitemsep]
\item Explicit \nameref{def:Variable_Memory_Deallocation}
  \begin{itemize}[noitemsep]
  \item The programmer must explicitly free the \nameref{def:Memory} themselves.
  \item C and C++ require this with their \texttt{free} and \texttt{delete} subprograms/operators, respectively.
  \end{itemize}
\item Implicit \nameref{def:Variable_Memory_Deallocation}
  \begin{itemize}[noitemsep]
  \item The programming language has facilities, called \emph{garbage collection} that automatically manages the \nameref{def:Heap}.
  \item There are many algorithms that handle garbage collection, some of which are faster, others slower.
  \end{itemize}
\end{enumerate}

\paragraph{Implicit Heap-Dynamic Variables}\label{par:Implicit_Heap-Dynamic_Variable_Binding_Lifetime}
\begin{definition}[Implicit Heap-Dynamic Variable]\label{def:Implicit_Heap-Dynamic_Variable_Binding_Lifetime}
  \emph{Implicit heap-dynamic variable}s are bound to \nameref{def:Heap} storage \textbf{only when they are assigned \nameref{def:Variable_Value}s}.
  In many regards, \nameref{def:Implicit_Heap-Dynamic_Variable_Binding_Lifetime}s and \nameref{def:Explicit_Heap-Dynamic_Variable_Binding_Lifetime}s are quite similar.

  However, \nameref{def:Implicit_Heap-Dynamic_Variable_Binding_Lifetime}s have \textbf{ALL} their attributes bound \textbf{EVERY} time they are assigned.
  The advantages and disadvantages of these types of \nameref{def:Variable}s are:
  \begin{itemize}[noitemsep]
  \item Advantages
    \begin{itemize}[noitemsep]
    \item Highest degree of flexibility, allowing for highly generic code
    \end{itemize}
  \item Disadvantages
    \begin{itemize}[noitemsep]
    \item Run time overhead of maintaining all the dynamic attributes, which could include subscript types and ranges
    \item Loss of some error dectection by the compiler/interpreter
    \end{itemize}
  \end{itemize}
\end{definition}

\subsection{Scope}\label{subsec:Variable_Scope}
\begin{definition}[Scope]\label{def:Variable_Scope}
  The \emph{scope} of a \nameref{def:Variable} is the range of statements in which the \nameref{def:Variable} is \nameref{def:Visible_Variable}.
\end{definition}

\nameref{def:Variable_Scope} might seem similar to \nameref{def:Variable_Lifetime}, but they are different.
Here are 2 examples that illustrate this point:
\begin{enumerate}[noitemsep]
\item At the second \texttt{print(x)}, \texttt{x} is \emph{in scope} (visible), but is \emph{dead} (deallocated).
\begin{minted}[frame=lines,linenos]{c}
int f(void) {
  int *x;
  x = (int *) calloc(sizeof(int), 1);
  print(x);
  free(x);
  print(x);
}
\end{minted}
  
\item When executing inside of \texttt{g(y)}, the \texttt{x} in function \texttt{f} is \emph{alive} (allocated), but is \emph{out of scope} (not visible).
\begin{minted}[frame=lines,linenos]{python3}
def f(x):
  return g(7)
def g(y):
  print (y)
  return
\end{minted}
\end{enumerate}

\begin{definition}[Visible]\label{def:Visible_Variable}
  A \nameref{def:Variable} is \emph{visible} in a statements if it can be referenced in that statement.

  \begin{remark}[In-Scope]\label{rmk:In_Scope_Variable}
    Sometimes, a \nameref{def:Variable} that is \nameref{def:Visible_Variable} is called \emph{in-\nameref{def:Variable_Scope}}.
  \end{remark}
\end{definition}

\begin{definition}[Local Variable]\label{def:Local_Variable}
  A \nameref{def:Variable} is a \emph{local variable} in a program unit or block if it is declared there.
  \nameref{def:Variable}s defined within subprograms are also local variables.
  The \nameref{def:Variable_Scope} is usually the body of the subprogram in which they are defined.

  \begin{remark}[Storage Binding]
    If the \nameref{def:Local_Variable} is a \nameref{def:Stack-Dynamic_Variable_Binding_Lifetime}, it is bound to storage when the subprogram begins and unbound when that execution terminates.
  \end{remark}
\end{definition}

\begin{definition}[Nonlocal Variable]\label{def:Nonlocal_Variable}
  The nonlocal variables of a program are visible with that particular program unit or block, but are not declared there.

  \begin{remark}[Global Variables]
    Global variables are a special case of \nameref{def:Nonlocal_Variable}s.
    These are discussed in \Cref{subsubsec:Variable_Global_Scope}.
  \end{remark}
\end{definition}

\subsubsection{Static Scope}\label{subsubsec:Static_Scope}
\begin{definition}[Static Scoping]\label{def:Static_Scoping}
  \emph{Static scoping} is a way to statically determine the scope of a \nameref{def:Variable}.
  When there is a reference to a \nameref{def:Variable}, the attributes of the \nameref{def:Variable} can be determined by finding the statement in which it is declared (either explicitly or implicitly).
  This makes it easy for a human reader and compiler/interpreter to figure out the \nameref{def:Variable_Type} of every \nameref{def:Variable} in the program.

  There are 2 types of statically-scoped languages:
  \begin{enumerate}[noitemsep]
  \item Subprograms can be nested inside programs (Python)
  \item Subprograms cannot be nested inside programs (Java)
  \end{enumerate}

  \begin{remark}[Lexical Scoping]\label{rmk:Variable_Lexical_Scoping}
    \nameref{def:Static_Scoping} is sometimes called \emph{lexical scoping}.
  \end{remark}
\end{definition}

\nameref{def:Static_Scoping} creates a tree-like structure, where each \nameref{def:Variable} declared in a program unit/block has a \nameref{def:Variable_Static_Parent}.
Then, each \nameref{def:Variable_Static_Parent} has a list of \nameref{def:Variable_Static_Ancestor}s.

\begin{definition}[Static Parent]\label{def:Variable_Static_Parent}
  If the \nameref{def:Variable} referenced is not present as a \nameref{def:Local_Variable}, then we have to go to the next outer program unit or block, the \emph{static parent}.
\end{definition}

\begin{definition}[Static Ancestor]\label{def:Variable_Static_Ancestor}
  The \emph{static ancestor}s are all the \nameref{def:Variable_Static_Parent}s to that particular program unit/block.
\end{definition}

This is illustrated by finding \texttt{x} in \texttt{sub2()} in this JavaScript function.
\begin{minted}[frame=lines,linenos]{javascript}
function big() {
  function sub1() {
    var x = 7;
    sub2();
  }
  function sub2() {
    var y = x;
  }
  var x = 3;
  sub1();
}
\end{minted}
The \texttt{x} in \texttt{sub2()} refers to the \texttt{x=3} in \texttt{big()}, because \texttt{sub1()} is not a \nameref{def:Variable_Static_Ancestor} of \texttt{sub2()}.
However, inside of \texttt{sub1()}, the use of the variable \texttt{x} would refer to the \texttt{x=7} value, and never the \texttt{x=3} value.

In some languages, if a \nameref{def:Variable} is declared in a sub-program unit/block, like in \texttt{sub1()}, then preceding the \nameref{def:Variable} name with the outer program unit/block will give the outer \nameref{def:Variable} \nameref{def:Variable_Value}.

\subsubsection{Dynamic Scope}\label{subsubsec:Dynamic_Scope}
\begin{definition}[Dynamic Scoping]\label{def:Dynamic_Scoping}
  \emph{Dynamic scoping} is based on the calling sequence of subprograms, and not their spatial relationship to each other.
  Thus, the scope can only be determined at run time.

  Some languages that implement this are:
  \begin{itemize}[noitemsep]
  \item APL
  \item SNOBOL4
  \item Early LISP
  \item Perl (Allowed, but must be said explicitly)
  \item Common LISP (Allowed, but must be said explicitly)
  \end{itemize}
\end{definition}

To illustrate \nameref{def:Dynamic_Scoping}, look at the code block below, and assume it is in a language that uses \nameref{def:Dynamic_Scoping}.
\begin{minted}[frame=lines,linenos]{javascript}
function big() {
  function sub1() {
    var x = 7;
    sub2();
  }
  function sub2() {
    var y = x;
  }
  var x = 3;
  sub1();
  sub2();
}
\end{minted}

The call to \texttt{sub1()} by \texttt{big()}, which then calls \texttt{sub2()}.
In that running of \texttt{sub2()}, the use of \texttt{x} cannot be determined locally (within that function).
Thus, it goes to its dynamic parent, \texttt{sub1()} and finds a declaration for \texttt{x}.
Thus, the \texttt{y} in \texttt{sub2()} evaluates to \texttt{y=7}.

The next instruction, the call to \texttt{sub2()} by \texttt{big()}, produces a different result.
In that running of \texttt{sub2()}, the use of \texttt{x} cannot be determined locally (within that function).
Thus, it goes to its dynamic parent, \texttt{big()} and finds a declaration for \texttt{x}.
Thus, the \texttt{y} in \texttt{sub2()} evaluates to \texttt{y=3}.

\subsubsection{Blocks}\label{subsubsec:Variable_Blocks}
New \nameref{subsubsec:Static_Scope}s can be defined in the middle of executing code.
This allows a small section to have its own \nameref{def:Local_Variable}s.

\begin{definition}[Block]\label{def:Block_Scope}
  A \emph{block} is a section of code that has its own \nameref{def:Local_Variable}s.
  These \nameref{def:Local_Variable}s are \textbf{not} shared with any \nameref{def:Variable_Static_Ancestor}s.
\end{definition}

The use of \nameref{def:Block_Scope}s create a \nameref{def:Block_Structured_Language}.

\begin{definition}[Block-Structured Language]\label{def:Block_Structured_Language}
  The use of \nameref{def:Block_Scope}s to create the \nameref{subsubsec:Static_Scope}s creates a \emph{block-structured language}.
\end{definition}

Consider the following C function:
\begin{minted}[frame=lines,linenos]{c}
void sub() {
  int count;
  ...
  while (...) {
    int count;
    count++;
  }
  ...
}
\end{minted}
The \texttt{count} inside the \texttt{while} loop is that loop's local count, and does not reference the \texttt{count} in \texttt{sub}.

\paragraph{Blocks in Functional Languages}\label{par:Variable_Block_in_Functional_Languages}
Since \nameref{def:Variable}s in \nameref{def:Functional_Programming_Language}s actually evaluate and store expressions, they behave differently.
Each functional language handles this differently, so you will have to look at the language specification to find out exactly how \nameref{def:Variable}s are scoped.

\subsubsection{Declaration Order}\label{subsubsec:Variable_Declaration_Order}
Some languages require that all \nameref{def:Variable} declarations occur at the beginning of a function (C89).

In some languages, \nameref{def:Variable}s cannot used before they have been declared.
\begin{itemize}[noitemsep]
\item Some of these languages allow for \nameref{def:Variable}s to be declared anywhere in the function, but can only be referenced \textbf{after} their declaration until the end of their scope.
\item Some of these languages allow for \nameref{def:Variable}s to be declared anywhere in the function, but if used before their declaration, they use a value like \texttt{undefined} (JavaScript).
\end{itemize}

This is a highly language-dependent thing, and one must consult with the language specification to figure out exactly how it works.

\subsubsection{Global Scope}\label{subsubsec:Variable_Global_Scope}
\begin{definition}[Global Variable]\label{def:Global_Variable}
  These are usually \nameref{def:Variable}s that sit outside of all functions.
  They can be accessed from anywhere in the program.
  They can also be defined and/or declared in other files in the program's project.
\end{definition}

It important to note the difference between a \nameref{def:Global_Variable}'s declaration and definition.
\begin{itemize}[noitemsep]
\item Declaration: The \nameref{def:Variable_Type}s and attributes are bound, but the \nameref{def:Memory} space required is \textbf{not} allocated.
\item Definition: The \nameref{def:Variable_Type}s and attributes are bound, but the \nameref{def:Memory} space required \textbf{is} allocated.
\end{itemize}

\begin{remark*}
  This is a highly language-dependent thing, and one must consult with the language specification to figure out exactly how it works.
\end{remark*}

\subsection{Referencing Environments}\label{subsec:Referencing_Environments}
\begin{definition}[Referencing Environment]\label{def:Referencing_Environment}
  The \emph{referencing environment} of a statement is the collection of all \nameref{def:Variable}s that are visible in the statement.
\end{definition}

\subsubsection{\nameref*{subsec:Referencing_Environments} in Languages with \nameref*{subsubsec:Static_Scope}}\label{subsubsec:Static_Scope_Referencing_Environment}
In a language that uses \nameref{subsubsec:Static_Scope}ing, the referencing environment includes all \nameref{def:Local_Variable}s in its \nameref{def:Variable_Scope} and all \nameref{def:Variable}s in the \nameref{def:Variable_Static_Ancestor} scopes that are \nameref{def:Visible_Variable}.
This also includes all function definitions and \nameref{def:Global_Variable}s up to that point.

The code block below and \Cref{tab:Static_Scope_Referencing_Environment-Execution} illustrate how a \nameref{def:Referencing_Environment} behave in a language that uses \nameref{def:Static_Scoping}.

\begin{minted}[frame=lines,linenos]{python3}
g = 3  # A global variable
def sub1():
    a = 5  # Create a local variable
    b = 7  # Create another local variable
    ...    # EXECUTION POINT 1
    def sub2():
        global g  # The global variable g is assignable here now
        c = 9 # Create a new local variable
        ...  # EXECUTION POINT 2
        def sub3():
            nonlocal c:  # Makes the nonlocal variable "c" visible here
            g = 11;
            ...  # EXECUTION POINT 3
\end{minted}

\begin{table}[h!]
  \centering
  \begin{tabular}{cl}
    \toprule
    Execution Point & \multicolumn{1}{c}{Referencing Environment} \\
    \midrule
    1 & Locals \texttt{a} and \texttt{b} of \texttt{sub1}, the global \texttt{g}, for reference and not assignment \\
    2 & Local \texttt{c} of \texttt{sub2}, the global \texttt{g} for reference and assignment (\texttt{global} \nameref{def:Reserved_Word}) \\
    3 & Local \texttt{g} of \texttt{sub3}, the nonloca \texttt{c} of \texttt{sub2} (\texttt{nonlocal} \nameref{def:Reserved_Word}) \\
    \bottomrule
  \end{tabular}
  \caption{\nameref{def:Referencing_Environment} of a Statically Scoped Program}
  \label{tab:Static_Scope_Referencing_Environment-Execution}
\end{table}

\subsubsection{\nameref*{subsec:Referencing_Environments} in Languages with \nameref*{subsubsec:Dynamic_Scope}}\label{subsubsec:Dynamic_Scope_Referencing_Environment}
In a language that uses \nameref{subsubsec:Dynamic_Scope}ing, the referencing environment includes all the \nameref{def:Local_Variable}s, the \nameref{def:Variable}s of all other \nameref{def:Subprogram_Active} subprograms, and the subprogram names.
This means that some \nameref{def:Variable}s in \nameref{def:Subprogram_Active} subprograms can be hidden from the referencing environment.

The code block below and \Cref{tab:Dynamic_Scope_Referencing_Environment-Execution} illustrate how a \nameref{def:Referencing_Environment} behave in a language that uses \nameref{def:Dynamic_Scoping}.

\begin{minted}[frame=lines,linenos]{c++}
void sub1() {
  int a, b;
  ...       // EXECUTION POINT 1
}
void sub2() {
  int b, c;
  ...       // EXECUTION POINT 2
  sub1();
}
void main() {
  int c, d;
  ...       // EXECUTION POINT 3
  sub2();
}
\end{minted}

\begin{table}[h!]
  \centering
  \begin{tabular}{cl}
    \toprule
    Execution Point & \multicolumn{1}{c}{Referencing Environment} \\
    \midrule
    1 & \texttt{a} and \texttt{b} of \texttt{sub1}, \texttt{c} of \texttt{sub2}, \texttt{d} of \texttt{main} \\
                    & \texttt{c} of \texttt{main} is hidden by \texttt{sub2} and \texttt{b} of \texttt{sub2} hidden by \texttt{sub1} \\
    2 & \texttt{b} and \texttt{c} of \texttt{sub2}, \texttt{d} of \texttt{main} \\
                    & \texttt{c} of \texttt{main} is hidden by \texttt{sub2} \\
    3 & \texttt{c} and \texttt{d} of \texttt{main} \\
    \bottomrule
  \end{tabular}
  \caption{\nameref{def:Referencing_Environment} of a Dynamically Scoped Program}
  \label{tab:Dynamic_Scope_Referencing_Environment-Execution}
\end{table}

\begin{definition}[Active]\label{def:Active}
  A subprogram is \emph{active} if its execution has begin, but not yet terminated.
\end{definition}

%%% Local Variables:
%%% mode: latex
%%% TeX-master: "../EDAP05-Concepts_Programming_Languages-Reference_Sheet"
%%% End:


\section{Data Types}\label{sec:Data_Types}
\begin{definition}[Data Type]\label{def:Data_Type}
  A \emph{data type} defines a collection of data values and a set of predefined operations on those values.
  The data types present in a language used for a particular problem should closely mirror the objects in the real-world the program is solving.

  They can be mathematically defined as
  \begin{equation}\label{eq:Data_Type}
    v : \DataType
  \end{equation}
  where $v$ is a value and $\tau$ is a type.

  \begin{remark}[Has the Type]\label{rmk:Has_the_Type}
    To say ``$v$ has the type $\tau$''
    \begin{equation}\label{eq:Has_the_Type}
      v \in \DataType
    \end{equation}
  \end{remark}
\end{definition}

User-defined data types allow for:
\begin{itemize}[noitemsep]
\item Improved readability with better named \nameref{def:Data_Type}s.
\item Improved modifiability with programmers having to change just one common data type somewhere for a large change throughout a program.
\end{itemize}

If we take user-defined \nameref{def:Data_Type}s further, we end up with \emph{abstract data types}.
These force an interface for a particular data type, which is then visible to the user, and the data and background operations are hidden away.

Because of the wide variety of \nameref{def:Data_Type}s present today, it is more useful to think about \nameref{def:Variable}s in terms of \nameref{def:Descriptor}.

\begin{definition}[Descriptor]\label{def:Descriptor}
  A \emph{descriptor} is the collection of attributes of a \nameref{def:Variable}.
  In an implementation, a descriptor is an area of \nameref{def:Memory} that stores the attributes of a \nameref{def:Variable}.
  
  There are a 2 cases for these:
  \begin{enumerate}[noitemsep]
  \item If all attributes are static, then they are known at compile-time, and the \nameref{def:Compiler} can use the symbol table to construct everything.
  \item If all attributes are dynamic, then the symbol table and all attributes must be stored in \nameref{def:Memory} during program execution.
  \end{enumerate}

  Descriptors are used for \nameref{def:Type_Checking} and building the code for \nameref{def:Variable_Memory_Allocation} and \nameref{def:Variable_Memory_Deallocation} operations.
\end{definition}

\begin{definition}[Type Error]\label{def:Type_Error}
  A \emph{type error} is an attempt to perform an operation that requires an input value of type $\tau$ with a value $v$ even though $v : \tau$ does not hold.
\end{definition}

\begin{definition}[Type Preservation]\label{def:Type_Preservation}
  A type system has the \emph{type preservation} (or \emph{subject reduction}) property if for any $e \EvaluatesTo v$, $e : \DataType$ implies $v : \DataType$.

  There are 3 properties we want to have a type preserving type system to have:
  \begin{enumerate}[noitemsep]
  \item \emph{Type Preservation}: The predictions of the type system agree with the evaluation rules.
  \item \emph{Progress}: The type system only assigns a type if the evaluation rules will not get ``stuck'' due to a missing semantic rule. This is not the same as guaranteeing that the program itself terminates, meaning but it does guarantee that the language implementation will never run into a situation in which it doesn't know what to do next.
  \item \emph{Termination}: We want the type system to be decidable, that is, we want an automatic mechanism that performs type checking.
  \end{enumerate}
\end{definition}

\subsection{Primitive Data Types}\label{subsec:Primitive_Data_Types}
\begin{definition}[Primitive Data Type]\label{def:Primitive_Data_Type}
  \emph{Primitive data type}s are \nameref{def:Data_Type}s that are \textbf{not} defined in terms of other data types.
  Nearly all programming languages provide these.
  Some are reflections of hardware, like integers, and others require only little software support for their impmlementation, floating-point numbers for instance.
\end{definition}

\subsubsection{Numeric Types}\label{subsubsec:Numeric_Primitive_Data_Types}
This section will discuss the 4 main types of numeric \nameref{def:Data_Type}s present in most programming languages.

\begin{definition}[Numeric Data Type]\label{def:Numeric_Data_Type}
  \emph{Numeric data type}s are \nameref{def:Data_Type}s that handle numbers.
\end{definition}

\paragraph{Integer}\label{par:Integer_Numeric_Primitive_Data_Type}
Integers are the most common \nameref{def:Numeric_Data_Type}.
Many languages support several sizes.
Java supports 4: \texttt{byte}, \texttt{short}, \texttt{int}, and \texttt{long}.
There can be unsigned integers as well.

If there are signed integers, the negative integers are stored in \nameref{def:Memory} in \nameref{def:Integer_Twos_Complement}.
\begin{definition}[Twos Complement]\label{def:Integer_Twos_Complement}
  \emph{Twos complement} is a way to store negative integers.
  To find the twos complement:
  \begin{enumerate}[noitemsep]
  \item The magnitude of the integer is found in binary
  \item The logical complement of that is computed
  \item One (1) is added to the logical complement
  \end{enumerate}
  
  \begin{remark}
    Using \nameref{def:Integer_Twos_Complement} is similar to adding by a negative number to an integer instead of performing subtraction.
  \end{remark}
\end{definition}

\paragraph{Floating-Point}\label{par:Floating_Point_Numeric_Primitive_Data_Type}
\begin{definition}[Floating-Point]\label{def:Floating_Point}
  \emph{Floating-point} \nameref{def:Data_Type}s model real (fractional/rational) numbers.
  However, these representations are only approximations for many real values.
  For example, $\pi$ cannot be represented in floating-point notation.

  Most programming languages implement 2 types of floating-point \nameref{def:Data_Type}s.
  \begin{enumerate}[noitemsep]
  \item \texttt{float}: The standard size, 32 bits (4 bytes). Represent the real number as a decimal and exponent, like scientific notation.
    \begin{itemize}[noitemsep]
    \item The first bit is a \emph{sign bit} (1 for negative)
    \item The next 8 bits are for the \emph{exponent}, normalized so that a real number raised to $-127$, i.e. $x^{-127}$, has an exponent bit value of 0.
      \begin{itemize}[noitemsep]
      \item This does mean that $x^{128}$ would have a bit-valued exponent of $255$.
      \end{itemize}
    \item The last 23 bits are for the fraction, called the \emph{mantissa}. This is the fractional portion of the scientific notation, represented in binary, with the first 1 of the bit sequence left off.
    \end{itemize}
  \item \texttt{double}: Used in cases where larger/smaller fractions or larger/smaller exponents are needed, 64 bits (8 bytes).
    \begin{itemize}[noitemsep]
    \item The first bit is a \emph{sign bit} (1 for negative)
    \item The next 11 bits are for the \emph{exponent}, normalized so that a real number raised to $-1023$, i.e. $x^{-1023}$, has an exponent bit value of 0.
      \begin{itemize}[noitemsep]
      \item This does mean that $x^{1024}$ would have a bit-valued exponent of $2048$.
      \end{itemize}
    \item The last 52 bits are for the fraction, called the \emph{mantissa}. This is the fractional portion of the scientific notation, represented in binary, with the first 1 of the bit sequence left off.
    \end{itemize}
  \end{enumerate}
\end{definition}

\nameref{def:Floating_Point} numbers are are specified in IEEE Floating-Point Standard 754.

\begin{definition}[Floating-Point Precision]\label{def:Floating_Point_Precision}
  \emph{Precision} is the accuracy of the fraction part of the \nameref{def:Floating_Point} number, and how well it represents the real number's value.
\end{definition}

\begin{definition}[Floating-Point Range]\label{def:Floating_Point_Range}
  \emph{Range} is a combination of the range of fractions and the range of the exponents.
\end{definition}

\paragraph{Complex}\label{par:Complex_Numeric_Primitive_Data_Type}
Some programming languages support complex numbers natively, and they also support complex-number mathematical operations natively.
The imaginary portion of the number is typically denoted with \texttt{j} or \texttt{J}.

\paragraph{Decimal}\label{par:Decimal_Numeric_Primitive_Data_Type}
In a computer, decimal numbers are stored in \nameref{def:Binary_Coded_Decimal}.
There is also special hardware to support hardware-level mathematical operations on these types of numbers.
If this hardware is not present, the calculations can be simulated in software.

\begin{definition}[Binary Coded Decimal]\label{def:Binary_Coded_Decimal}
  \emph{Binary Coded Decimal}, or \emph{BCD}, is a way to represent decimal numbers with perfect accuracy, albeit at the expense of some space.
  There is a one-to-one mapping of the binary representations of these numbers to decimal, and anything greater than 9 is discarded.
  \begin{table}[h!]
    \centering
    \begin{tabular}{cc}
      \toprule
      Decimal & \nameref{def:Binary_Coded_Decimal} \\
      \midrule
      0 & 0000 \\
      1 & 0001 \\
      2 & 0010 \\
      3 & 0011 \\
      4 & 0100 \\
      5 & 0101 \\
      6 & 0110 \\
      7 & 0111 \\
      8 & 1000 \\
      9 & 1001 \\
      \midrule
      X & 1010 \\
      X & 1011 \\
      X & 1100 \\
      X & 1101 \\
      X & 1110 \\
      X & 1111 \\
      \bottomrule
    \end{tabular}
    \caption{Binary Coded Decimal}
    \label{tab:Binary_Coded_Decimal}
  \end{table}

  \begin{remark}
    These numbers are usually stored 2 per byte, because each only takes 4 bits.
  \end{remark}
\end{definition}

\subsubsection{Boolean Types}\label{subsubsec:Boolean_Primitive_Data_Types}
\begin{definition}[Boolean Data Type]\label{def:Boolean_Data_Type}
  \emph{Boolean data type}s only store 2 values: \texttt{true} and \texttt{false}.
  Some older language implementations did not support these, but most do today.
  If a language does not support a boolean data type, then 0 is considered false, and 1 is considered true.

  \begin{remark}[Storage in Memory]\label{rmk:Boolean_Storage_in_Memory}
    Although a single bit can represent a \nameref{def:Boolean_Data_Type}, single bits of \nameref{def:Memory} cannot be efficiently access on many machines.
    Thus, \nameref{def:Boolean_Data_Type}s are usually stored in a single byte.
  \end{remark}
\end{definition}

\subsubsection{Character Types}\label{subsubsec:Character_Primitive_Data_Types}
Characters are stored in \nameref{def:Memory} as numeric encodings.
These are usually single characters, \textbf{not multiple characters together (strings)}.

Characters were originally handled by ASCII, but now there are several encodings, with Unicode being more commonly used now.
Unicode supports all human languages, glyphs, and other characters, like emojis.
The first 128 characters of Unicode match up with ASCII for intercompatibility.

ASCII required 8 bits, Unicode (UTF-16) uses 16.

\subsection{Character String Types}\label{subsec:Character_String_Types}
\begin{definition}[Character String Type]\label{def:Character_String_Type}
  A \emph{character string type} is one in which the values consist of sequences of characters.
\end{definition}

\subsubsection{Design Issues}\label{subsubsec:Character_String_Types_Design_Issues}
There are 2 questions that need to be answered when designing a language implementation when it comes to strings.
\begin{itemize}[noitemsep]
\item Should strings be a special kind of character array or a primitive type?
\item Should strings have static or dynamic lengths?
\end{itemize}

\subsubsection{Strings and Their Operations}\label{subsubsec:String_Types_and_Ops}
The most common string operations are:
\begin{itemize}[noitemsep]
\item Assignment
  \begin{itemize}[noitemsep]
  \item What happens when a string is longer than expected? C/C++'s \texttt{strcpy} function
  \end{itemize}
\item Concatenation
\item Substring Reference
  \begin{itemize}[noitemsep]
  \item Discussed more in the context of arrays, where substring references are called slices.
  \end{itemize}
\item Comparison
  \begin{itemize}[noitemsep]
  \item How do we compare 2 strings, where one is longer than the other?
  \end{itemize}
\item Pattern Matching
\end{itemize}

In C and C++, strings are terminated with the null character, \texttt{00}.
This way we do not need to track the length of a string.

Object-Oriented Languages (Java, Ruby, C\#) use classes to represent strings.
The only field in these objects is a constant string.

Python supports strings as a primitive type, and supports array-like operations on them.

Some languages have \nameref{def:Regular_Expression}s built in, like Perl, JavaScript, Ruby, and PHP.\@
Others have libraries that handle \nameref{def:Regular_Expression}s.

\begin{definition}[Regular Expression]\label{def:Regular_Expression}
    A \emph{regular expression}, sometimes called a \emph{regex} is a way to define a sequence of characters to form strings.
\end{definition}

\subsubsection{String Length Options}\label{subsubsec:String_Type_Length_Options}
\begin{definition}[Static Length String]\label{def:Static_Length_String}
  A \emph{static length string} has its length set at the time of string creation.
  It is static, in that the length cannot be changed later in the program's execution.
\end{definition}

\begin{definition}[Dynamic Length String]\label{def:Dynamic_Length_String}
  A \emph{dynamic length string} has its length set at the time of string creation.
  However, strings can change their length, and there is no set maximum size they can have.
\end{definition}

\begin{definition}[Limited Dynamic Length String]\label{def:Limited_Dynamic_Length_String}
  A \emph{dynamic length string} has its length set at the time of string creation.
  However, the string can be redefined later in the program, so long as the new string is the same length or shorter than when the string \nameref{def:Variable} was defined.
\end{definition}

\subsubsection{Evaluation}\label{subsubsec:String_Type_Evaluation}
Primitive string type implementations would require there to be predefined functions for many string operations.
If there aren't, then programming in that language becomes more cumbersome.

\nameref{def:Dynamic_Length_String}s are the most flexible, but the overhead of their implementation should be weighed against that flexibility.

\subsubsection{Implementation of Character String Types}\label{subsubsec:Implementation_of_Character_String_Types}
Software is used to implement string storage, retrieval, and manipulation.
When a language uses character arrays to store character string types, the language usually supplies few operations.

A \nameref{def:Descriptor} for a \nameref{def:Static_Length_String} has 3 fields:
\begin{enumerate}[noitemsep]
\item Name of the type
\item The type's length in characters
\item Address of the first character
\end{enumerate}

A \nameref{def:Descriptor} for a \nameref{def:Limited_Dynamic_Length_String} has 4 fields:
\begin{enumerate}[noitemsep]
\item Name of the type
\item The type's maximum length in characters
\item The length of the currently stored string
\item The address of the first character
\end{enumerate}

A \nameref{def:Descriptor} for a \nameref{def:Dynamic_Length_String} is more difficult to handle because of its dynamic nature.
There are 3 approaches to storing these:
\begin{enumerate}[noitemsep]
\item Strings stored in a linked list. If the string gets longer, individual nodes can be allocated from anywhere in the \nameref{def:Heap}.
  \begin{itemize}[noitemsep]
  \item A drawback of this is that extra storage of the links
  \item The necessary complexity of string operations
  \end{itemize}
\item Store strings as arrays of pointers to individual characters on the \nameref{def:Heap}
  \begin{itemize}[noitemsep]
  \item This uses more memory, but processing is faster than the linked list approach.
  \end{itemize}
\item Store complete strings in adjacent cells, and when a new longer string comes along, store the whole thing in a new area in the \nameref{def:Heap} and deallocate the old location.
  \begin{itemize}[noitemsep]
  \item Less storage required compared to the linked list approach
  \item Allocation and deallocation of the string is more difficult
  \end{itemize}
\end{enumerate}

\subsection{User-Defined Ordinal Types}\label{subsec:User_Defined_Ordinal_Types}
\begin{definition}[Ordinal Type]\label{def:Ordinal_Type}
  An \emph{ordinal type} is a \nameref{def:Variable_Type} in which the range of possible values can be associated with the set of positive integers.

  For example, in Java, the primitive ordinal types are: \texttt{int}, \texttt{char}, and \texttt{boolean}.

  \begin{remark}
    There are 2 \nameref*{subsec:User_Defined_Ordinal_Types} that are supported by most programming languages:
    \begin{itemize}[noitemsep]
    \item \nameref{subsubsec:Enumeration_Types}
    \item \nameref{subsubsec:Subrange_Types}
    \end{itemize}
  \end{remark}
\end{definition}

\subsubsection{Enumeration Types}\label{subsubsec:Enumeration_Types}
\begin{definition}[Enumeration Type]\label{def:Enumeration_Type}
  An \emph{enumeration type} is one in which all of the possible values, which are named constants, are provided (enumerated) in the definition.
  Enumeration types provide a way of defining and grouping collections of named constants, called \emph{enumeration constants}.

  This is an example of an enumeration type in C\#:
\begin{minted}[frame=lines,linenos]{csharp}
enum days {Mon, Tue, Wed, Thu, Fri, Sat, Sun};
\end{minted}

  \begin{remark}
    Typically, each of the numeration constants is implicitly assigned an integer literal, though they can be given integer literals explicitly too.
  \end{remark}
\end{definition}

The design issues for \nameref{def:Enumeration_Type}s are:
\begin{itemize}[noitemsep]
\item Is an enumeration constant allowed to appear in more than one \nameref{def:Enumeration_Type} definition, and if so, how is the type of an occurance of that enumeration constant in the program checked?
\item Are enumeration constants coerced to integers?
\item Are any other types coerced to an \nameref{def:Enumeration_Type}?
\end{itemize}

\paragraph{Designs}\label{par:Enumeration_Types_Designs}
In languages without native support of \nameref{def:Enumeration_Type}s, they are simulated with integer values.
For example,
\begin{minted}[frame=lines,linenos]{c}
int red = 0, blue = 1;
\end{minted}

However, this can lead to unexpected behavior.
For example, the variables \texttt{red} and \texttt{blue} can be added together.
In essence, there would be no \nameref{def:Type_Checking}.
The value for those variables could be overwritten somewhere.
Though, that issue would be solved by making the variable a constant instead.

C and Pascal introduced the use of \nameref{def:Enumeration_Type}s.
These implicitly use default values, integers, as the enumeration constants.
However, the values can be set explicitly, by the programmer.
With these \nameref{def:Enumeration_Type}s, we have and avoid these issues:
\begin{minted}[frame=lines,linenos]{c}
enum colors {red, blue, green, yellow, black};
colors myColor = blue, yourColor = red;
myColor++; // Valid code, sets myColor from blue to green
myColor = 4; // Illegal
myColor = (colors) 4; // Legal because 4 is being typecast
\end{minted}

These help prevent some issues, but not all.

The next iteration was in Ada.
They allowed for \emph{overloaded literals} in their \nameref{def:Enumeration_Type}s.
This means there were enumeration constants shared between 2 \nameref{def:Enumeration_Type}s in the same referencing environment.
In this case, the value must be determinable from the context of the \nameref{def:Enumeration_Type}.
Sometimes, a mroe explicit specification must be used.
Additionally, because the enumerations constants were \textbf{not} coerced to integers, nor were the \textbf{enumeration variables}, the range of operations and range of values for the enumeration constants was limited.
This allowed the compiler to pick up many more errors.

\begin{remark*}
  None of the relatively recent scripting kinds of languages include \nameref{def:Enumeration_Type}s.
  These include Perl, JavaScript, PHP, Python, Ruby, and Lua.
\end{remark*}

\paragraph{Evaluation}\label{par:Enumeration_Types_Evaluation}
Enhancements to both \nameref{subsec:Readability} and \nameref{subsec:Reliability}.
\begin{itemize}[noitemsep]
\item \nameref{subsec:Readability} is enhanced by better named values
\item \nameref{subsec:Reliability} is enhanced by being able to perform \nameref{def:Type_Checking} on the \nameref{def:Enumeration_Type}s.
  \begin{itemize}[noitemsep]
  \item No arithmetic operations allowed on \nameref{def:Enumeration_Type}s.
  \item No enumeration variable can be assigned a value outside the \nameref{def:Enumeration_Type}'s assigned range.
  \end{itemize}
\end{itemize}

\subsubsection{Subrange Types}\label{subsubsec:Subrange_Types}
\begin{definition}[Subrange Type]\label{def:Subrange_Type}
  A \emph{subrange type} is a contiguous sequence of an \nameref{def:Ordinal_Type}.
  For example, this is a subrange: \texttt{12..14}.
\end{definition}

\paragraph{Ada's Design}\label{par:Adas_Subrange_Types_Design}
Ada included \nameref{def:Subrange_Type}s in \nameref{def:Ada_Subtype}s.

\begin{definition}[Subtype]\label{def:Ada_Subtype}
  A \emph{subtype} in Ada is an extension, usually constrained, version of existing types.
  For example,
\begin{minted}[frame=lines,linenos]{ada}
type Days is (Mon, Tue, Wed, Thu, Fri, Sat, Sun);
subtype Weekdays is Days range Mon..Fri;
Day1 : Days;
Day2 : Weekdays;
...
Day2 := Day1; -- Will only work if Day1 has Mon-Fri, fails if Day1 = Sat or Sun
\end{minted}
\end{definition}

The compiler generates range-checking code for every assignment to the subrange type.
Subranges require run-time range checking.

\paragraph{Evaluation}\label{par:Subrange_Types_Evaluation}
\nameref{def:Subrange_Type}s improve \nameref{subsec:Readability} by making it clear that \nameref{def:Variable}s of \nameref{def:Ada_Subtype}s can only store a certain range of values.
\nameref{subsec:Reliability} is increased with \nameref{def:Subrange_Type}s because assigning a value to a subrange variable outside its range is detected as an error.

\subsubsection{Implementation of \nameref*{subsec:User_Defined_Ordinal_Types}}\label{subsubsec:Implementation_User_Defined_Ordinal_Types}
\nameref{def:Enumeration_Type}s are usually implemented on integers.
However, without restrictions on ranges of values and possible operations, this does \textbf{not} improve \nameref{subsec:Reliability}.

\nameref{def:Subrange_Type}s are implemented the same way as their parent types, except range checks are implicity included by the compiler in every assignment of a variable or expression to a subrange variable.

\subsection{List Types}\label{subsec:List_Types}
Lists were first supported in LISP.\@
\begin{definition}[List]\label{def:List}
  A \emph{list} is a data structure heavily used in \nameref{def:Functional_Programming_Language}s.
  They are similar to arrays in other languages, but they may lazily evaluated and may be infinite.

  \begin{remark}
    Lists, because of their (potentially, depends on the language) inherently infinite nature, have always been part of \nameref{def:Functional_Programming_Language}s, but are making their way to \nameref{def:Imperative_Programming_Language}s too.
  \end{remark}
\end{definition}

\begin{remark*}
  \theauthor{} went a little ham on this section because he uses Emacs and writes ELisp to customize it.
  He is also \emph{really} interested in \nameref{def:Functional_Programming_Language}s.
\end{remark*}

Lists in Scheme, LISP, and Common LISP are written as such:
\begin{minted}[frame=lines,linenos]{lisp}
(A B C D) ; List of 4 elements
(A (B C) D) ; List of 3 elements, with the middle being a 2 element nested list
\end{minted}

In LISP and its descendants, data and code have the same syntactic form, meaning this could be interpreted as a function call to \texttt{A} with \texttt{B} and \texttt{C} being parameters; or as a list of 3 elements.
\begin{minted}[frame=lines,linenos]{lisp}
(A B C)
\end{minted}

Lists in Scheme, LISP, and Common LISP can be considered linked-lists with immutable nodes.
This means there are operations to get the current node's data and to get the next nodes in the rest of the list.
\begin{minted}[frame=lines,linenos]{scheme}
; The ' in front of a list means to interpret the list as data and not a function call
(CAR '(A B C)) ; Returns the element A
(CDR '(A B C)) ; Returns the list (B C)
\end{minted}

There are 2 ways these lists can be constructed:
\begin{enumerate}[noitemsep]
\item \texttt{CONS} takes 2 parameters and returns a list with the first parameter as the first element and the second parameter as the remainder of the list.
\begin{minted}[frame=lines,linenos]{common-lisp}
(CONS 'A '(B C)) ; Returns (A B C)
\end{minted}
\item \texttt{LIST} takes any number of parameters and returns a new list with the parameters as the new list's elements
\begin{minted}[frame=lines,linenos]{common-lisp}
(LIST 'A 'B '(C D)) ; Returns (A B (C D))
\end{minted}
\end{enumerate}

The empty list \texttt{()} is also denoted as \texttt{nil}.
\texttt{nil} also serves as the \texttt{false} value of the language, and everything else is \texttt{true}.

\begin{definition}[List Comprehension]\label{def:List_Comprehension}
  A \emph{list comprehension} is an idea from set notation and set theory.
  Essentially, a list comprehension applies a function to every element in a given \nameref{def:Array}/\nameref{def:List}, and a new \nameref{def:Array}/\nameref{def:List} is constructed from the results.

  2 examples of this are shown below, the first in Haskell, the second in Python 3.
\begin{minted}[frame=lines,linenos]{haskell}
[n * n | n <- [1..10]]
\end{minted}
\begin{minted}[frame=lines,linenos]{python3}
[x * x for x in range(1, 11, 1)]
\end{minted}
\end{definition}
\subsection{Arrays}\label{subsec:Arrays}
\begin{definition}[Array]\label{def:Array}
  An \emph{array} is a homogeneous aggregate of data elements in which an individual element is identified by its position in the aggregate, relative to the first element.
  The individual elements of an array are of the same \nameref{def:Data_Type}.
  References to individual array elements are specified using subscript expressions
\end{definition}

\subsubsection{Design Issues}\label{subsubsec:Arrays-Design_Issues}
\begin{itemize}[noitemsep]
\item What types are legal for subscripts?
\item Are subscripting expressions in element references range checked?
\item When are subscript ranges bound?
\item When does array allocation take place?
\item Are jagged and/or rectangular multidimensional arrays allowed, or both?
\item Can arrays be initialized when they have their storage allocated?
\item What king of slices are allowed, if any?
\end{itemize}

\subsubsection{Arrays and Indices}\label{subsubsec:Arrays-Arrays_and_Indices}
Specific elements in an array are referenced by means of the name of the aggregate and a dynamic selector, known as \emph{subscript}s or \emph{indices}.
If all of the subscripts used are constants, the selector is static; otherwise, it is dynamic.
Arrays are sometimes called \emph{finite mapping}s, because they map \nameref{def:Memory} cells to values, with a finite length.

Most languages use brackets, \texttt{[} and \texttt{]}, to denote the array indices.
However, some languages use parentheses.

The type of the subscripts are usually integers, but Ada also allows any \nameref{def:Ordinal_Type} to be used as a subscript.
Some languages check the bounds of the array accesses through subscripts, though some don't.
Some languages allow the index to be a negative integer, in which case, it is the index of that element starting from the end as 0.

\subsubsection{Subscript Bindings}\label{subsubsec:Arrays-Subscript_Bindings}
The binding of the subscript type to an array variable is usually static, but the value ranges are sometimes dynamic.
Some languages have an implicit lower bound on the subscript range, usually 0.

\subsubsection{Array Categories}\label{subsubsec:Arrays-Categories}
There are 5 categories of arrays, based on the binding to subscript ranges, the binding to storage, and from where the storage is allocated.
\begin{enumerate}[noitemsep]
\item \nameref{def:Static_Array}
\item \nameref{def:Fixed_Stack_Dynamic_Array}
\item \nameref{def:Stack_Dynamic_Array}
\item \nameref{def:Fixed_Heap_Dynamic_Array}
\item \nameref{def:Heap_Dynamic_Array}
\end{enumerate}

\begin{definition}[Static Array]\label{def:Static_Array}
  A \emph{static array} is one in which the subscript ranges are statically bound and the storage allocation is static.
  Meaning the entire array, other than the values it contains are created before runtime.

  The advantages and disadvantages of this are:
  \begin{itemize}[nosep,noitemsep]
  \item Advantages
    \begin{itemize}[nosep,noitemsep]
    \item Efficiency, there is no dynamic allocation or deallocation overhead required.
    \end{itemize}
  \item Disadvantages
    \begin{itemize}[nosep,noitemsep]
    \item Flexibility, the storage for the array is fixed for the entire execution of the program.
    \end{itemize}
  \end{itemize}
\end{definition}

\begin{definition}[Fixed Stack-Dynamic Array]\label{def:Fixed_Stack_Dynamic_Array}
  In a \emph{fixed stack-dynamic array}, the subscript ranges are statically bound, but the storage allocation is done at declaration elaboration time during runtime.

  The advantages and disadvantages of this are:
  \begin{itemize}[nosep,noitemsep]
  \item Advantages
    \begin{itemize}[nosep,noitemsep]
    \item If 2 subprograms both have large arrays, they can use the same space, so long as only a single one is running at a time.
    \end{itemize}
  \item Disadvantages
    \begin{itemize}[nosep,noitemsep]
    \item The time required to allocate and deallocate the array.
    \end{itemize}
  \end{itemize}
\end{definition}

\begin{definition}[Stack-Dynamic Array]\label{def:Stack_Dynamic_Array}
  In a \emph{stack-dynamic array}, the subscript ranges and storage allocation are done during declaration elaboration time during program execution.
  However, once the subscript ranges are bound and the storage allocatec, they are both fixed throughout the lifetime of the array.

  The advantages and disadvantages of this are:
  \begin{itemize}[nosep,noitemsep]
  \item Advantages
    \begin{itemize}[nosep,noitemsep]
    \item Flexibility, the size of the array doesn't need to be known until just before the array is used.
    \end{itemize}
  \item Disadvantages
    \begin{itemize}[nosep,noitemsep]
    \item The time required to allocate and deallocate the array.
    \item Determine the subscript ranges and bind them.
    \end{itemize}
  \end{itemize}
\end{definition}

\begin{definition}[Fixed Heap-Dynamic Array]\label{def:Fixed_Heap_Dynamic_Array}
  A \emph{fixed heap-dynamic array} is similar to a \nameref{def:Fixed_Stack_Dynamic_Array}, in that subscript ranges and storage binding are done on demand at runtime, and are fixed throughout the array's lifetime.
  However, the storage is allocated from the heap instead of the stack.

  The advantages and disadvantages of this are:
  \begin{itemize}[nosep,noitemsep]
  \item Advantages
    \begin{itemize}[nosep,noitemsep]
    \item Flexibility, the array can be any size to fit any problem.
    \end{itemize}
  \item Disadvantages
    \begin{itemize}[nosep,noitemsep]
    \item The allocation and deallocation time is much longer on the heap compared to the stack.
    \end{itemize}
  \end{itemize}
\end{definition}

\begin{definition}[Heap-Dynamic Array]\label{def:Heap_Dynamic_Array}
  A \emph{heap-dynamic array} is one in which the binding of subscript ranges and storage allocation is done during program execution, \textbf{and can change any number of times during the array's lifetime}.

  The advantages and disadvantages of this are:
  \begin{itemize}[nosep,noitemsep]
  \item Advantages
    \begin{itemize}[nosep,noitemsep]
    \item Flexibility, arrays can grow and shrink during program execution to fit the needs of the problem.
    \end{itemize}
  \item Disadvantages
    \begin{itemize}[nosep,noitemsep]
    \item Allocation and deallocation take longer on the heap.
    \item Allocation and deallocation may happen several times during a program's execution.
    \end{itemize}
  \end{itemize}
\end{definition}

\begin{definition}[Length]\label{def:Array-Length}
  The \emph{length} of an \nameref{def:Array} is defined to be the number of storage elements present in it.
  This means that a list that is declared to have 10 elements, but none assigned, has a length of 10.

  \begin{remark}[The Empty Array]\label{rmk:Array-Length-Empty_Array}
    \emph{The empty array} is defined to have a length of 0.
  \end{remark}
\end{definition}

\begin{definition}[Sparse]\label{def:Array-Sparse}
  If an array is \emph{sparse}, it means that not all elements present in an array are filled with user-input data.
  For example, if you have an array with \nameref{def:Array-Length} 10 in JavaScript, you can add an element to position 50, to create an array that now has \nameref{def:Array-Length} 51, with only 11 elements.
\end{definition}

\subsubsection{Array Initialization}\label{subsubsec:Arrays-Initialization}
\subsubsection{Array Operations}\label{subsubsec:Arrays-Operations}
\subsubsection{Rectangular and Jagged Arrays}\label{subsubsec:Arrays-Rectangular_Jagged}
\subsubsection{Slices}\label{subsubsec:Arrays-Slices}
\subsubsection{Evaluations}\label{subsubsec:Arrays-Evaluations}
\subsubsection{Implementation of Array Types}\label{subsubsec:Arrays-Implementation}

\subsection{Associative Arrays}\label{subsec:Associative_Arrays}
\begin{definition}[Associative Array]\label{def:Associative_Array}
  An \emph{associative array} is an unordered collection of data elements that are indexed by an equal number of values, called \emph{keys}.
  The user-defined keys must be stored in the structure, along with the values to be stored.

  To store the keys, they must be \emph{hashed}.
  This is done with a hash function.

  \begin{remark}[Alternative Names]
    There are many alternative names for \nameref{def:Associative_Array}s.
    \begin{itemize}[noitemsep]
    \item Hashtable
    \item Hashmap
    \item \nameref{def:Associative_Array}
    \item Dictionary
    \end{itemize}
  \end{remark}

  \begin{remark}[Improving Capabilities]
    What the associative array/dictionary/hashtable are allowed to store depends on the language.
    In Python and Ruby, objects and \nameref{def:Primitive_Data_Type}s can be stored.
    In PHP, integers or strings can be keys.
    In Ruby, any object can be a key.
  \end{remark}
\end{definition}

\subsubsection{Structure and Operations}\label{subsubsec:Structure_and_Ops_Associative_Arrays}
The code examples here are done in Perl, but it is similar in most other languages.
For more specific code, you will have to visit the language's documentation.
\begin{minted}[frame=lines,linenos]{perl}
# Defining an associative array
%salaries = ("Gary" => 75000, "Perry" => 57000, "Mary" => 55750);

# Switch from % for hash to $ for single values in the hash
# That should be a dollar sign, not a Stirling Pound symbol

# Changing a value in the associative array
$salaries{"Perry"} = 58850;

# Removing a key-value pair from the associative array
delete $salaries{"Gary"};
\end{minted}

The size of a Perl \nameref{def:Associative_Array} is dynamic and will grow and shrink as needed.

\nameref{def:Associative_Array}s are very useful if there is a lot of searching that needs to be done, because the hashing of the key, then the search term allows for $O(1)$ lookup speeds for an element located anywhere in the \nameref{def:Associative_Array}.

\subsubsection{Implementing \nameref*{subsec:Associative_Arrays}}\label{subsubsec:Implementing_Associative_Arrays}
The implementation for an \nameref{def:Associative_Array} differs between languages.
However, in all languages, eventually the \nameref{def:Associative_Array} can get ``full''.
This is when the collision chance of 2 hashes modulo the length of the structure gets too high.
This means that if an element is already stored, and something new were hashed and modulo-d the length of the structure, there is a very high probability that the new element would ``collide'' with the old one.

To handle this, \nameref{def:Associative_Array}s are grown.
Every key-value pair will need to have their key recalculated.
This is a slightly time-costly operation, $O(n)$, but the spatial-cost is much higher, because there are 2 potentially very large arrays in memory at the same time.

\subsection{Record Types}\label{subsec:Record_Data_Types}
\begin{definition}[Record]\label{def:Record_Data_Type}
  A \emph{record} is an aggregate of data elements in which the individual elements are identified by names and accessed through offsets from the beginning of the structure, similar to arrays.
  The offset from the head of the record to reach any \nameref{def:Record_Data_Type_Field} is known statically, because the sizes and \nameref{def:Variable_Type}s are known at compile time.

  Records are used to model a collection of data in which the individual elements, the \nameref{def:Record_Data_Type_Field}s are not of the same \nameref{def:Variable_Type} or size.

  \begin{remark}[Clarification]
    An important clarification here is that the \nameref{def:Record_Data_Type}, defined in \Cref{def:Record_Data_Type} is \textbf{NOT} a record in a database \textbf{in any way}.
  \end{remark}

  \begin{remark}
    A \nameref{def:Record_Data_Type} is similar to a heterogeneous array, but they differ in one key way.
    A heterogeneous array is an array of pointers to areas of \nameref{def:Memory} that may be discontinuous.
    However, all the \nameref{def:Record_Data_Type_Field}s in a \nameref{def:Record_Data_Type} all reside in adjacent \nameref{def:Memory} locations.
  \end{remark}

  \begin{remark}[\nameref*{def:Record_Data_Type} vs. Object]
    A \nameref{def:Record_Data_Type} and an object are quite similar.
    However, the differences between them depend on the language.
  \end{remark}
\end{definition}

\begin{definition}[Field]\label{def:Record_Data_Type_Field}
  A \emph{field} is an element a \nameref{def:Record_Data_Type}.
  These are fixed length, with fixed \nameref{def:Variable_Type}, meaning the \nameref{def:Memory} address can be statically calculated to reach any \nameref{def:Record_Data_Type_Field} in the \nameref{def:Record_Data_Type}.

  Fields are referenced by their identifier, rather than an index.
\end{definition}

\subsubsection{Definitions of \nameref*{def:Record_Data_Type}s}\label{subsubsec:Definitions_of_Records}
There are 2 design questions that need to be asked when defining \nameref{def:Record_Data_Type}s.
\begin{enumerate}[noitemsep]
\item What is the syntactic form of references to \nameref{def:Record_Data_Type_Field}s?
\item Are elliptical references allowed?
\end{enumerate}

Below are 2 blocks of code, the first from COBOL, the second from Ada.
Both describe an employee.

\begin{minted}[frame=lines,linenos]{cobol}
01  EMPLOYEE-RECORD.
    02 EMPLOYEE-NAME.
       05 FIRST      PICTURE IS X(20).
       05 MIDDLE     PICTURE IS X(20).
       05 LAST       PICTURE IS X(20).
    02 HOURLY-RATE   PICTURE IS 99V99.
\end{minted}
The numbers \texttt{01}, \texttt{02}, and \texttt{05} are \emph{level numbers}, which indicate relative hierarchical values.
Any line that is followed by a line with a higher-level number is itself a record.
\texttt{PICTURE} clauses show the formats of the storage locations.
\texttt{X(20)} is a 20 character alphanumeric string and \texttt{99V99} is a 4 decimal digit number with the dcimal in the middle.

However, Ada does not have the level numbers like COBOL, so they allow for nesting \nameref{def:Record_Data_Type} structures inside \nameref{def:Record_Data_Type} declarations.
\begin{minted}[frame=lines,linenos]{ada}
type Employee_Name_Type is record
   First : String (1..20);
   Middle : String (1..20);
   Last : String (1..20);
end record;
type Employee_Record_Type is record   Employee_Name: Employee_Name_Type;
   Hourly_Rate: Float;
end record;
Employee_Record: Employee_Record_Type;
\end{minted}

In Java and C\#, \nameref{def:Record_Data_Type}s can be defined as data classes, whith nested \nameref{def:Record_Data_Type}s defined as nested classes.
Lua's tables serve this purpose.

\subsubsection{References to \nameref*{def:Record_Data_Type} \nameref*{def:Record_Data_Type_Field}s}\label{subsubsec:References_to_Record_Fields}
There are many ways to refer to individual \nameref{def:Record_Data_Type_Field}s.
We will look at the way COBOL referenced \nameref{def:Record_Data_Type_Field}s, and the \nameref{def:Record_Data_Type_Field_Access-Dot_Notation}.

\begin{definition}[Dot Notation]\label{def:Record_Data_Type_Field_Access-Dot_Notation}
  Most programming languages use \emph{dot notation} for \nameref{def:Record_Data_Type_Field} references, where components necessary to reach the \nameref{def:Record_Data_Type_Field} are connected with periods.
  The outermost \nameref{def:Record_Data_Type} goes on the left, and gets more specific as it grows to the right.
\end{definition}

There are 2 examples below of accessing the middle name of the Employee \nameref{def:Record_Data_Type}s we made in the previous section.

\begin{minted}[frame=lines,linenos]{cobol}
MIDDLE OF EMPLOYEE-NAME OF EMLOYEE-RECORD
\end{minted}

\begin{minted}[frame=lines,linenos]{ada}
Employee_Record.Employee_Name.Middle;
\end{minted}

There are 2 ways to make a reference to a \nameref{def:Record_Data_Type} \nameref{def:Record_Data_Type_Field}.
\begin{enumerate}[noitemsep]
\item \nameref{def:Fully_Qualified_Reference}s
\item \nameref{def:Elliptical_Reference}s
\end{enumerate}

\begin{definition}[Fully Qualified Reference]\label{def:Fully_Qualified_Reference}
  A \emph{fully qualified reference} to a record field is one in which to access a \nameref{def:Record_Data_Type_Field}, the programmer \textbf{MUST} specify all intermediate \nameref{def:Record_Data_Type}s to go through.

  An alternative to fully qualified reference is the \nameref{def:Elliptical_Reference}.
\end{definition}

\begin{definition}[Elliptical Reference]\label{def:Elliptical_Reference}
  An \emph{elliptical reference} allows a programmer to specify the \nameref{def:Record_Data_Type_Field} and omit any to all parent \nameref{def:Record_Data_Type}s; so long as the resulting reference in unambiguous in the referencing environment (the \nameref{def:Variable_Scope}).

  These are a programmer convenience, but are \textbf{incredibly} difficult to compile, because of the alaborate data structures and procedures required to correctly identify the referenced field.
  There is also a slight loss of \nameref{subsec:Readability}.
\end{definition}

\subsubsection{Evaluation of \nameref*{subsec:Record_Data_Types}}\label{subsubsec:Evaluation_of_Record_Types}
\nameref{def:Record_Data_Type}s are valuable to programming and programming languages.
Their design is straightforward and use is safe.

\nameref{def:Record_Data_Type}s and \nameref{def:Array}s are quite similar, but differ in some key ways
\begin{itemize}[noitemsep]
\item \nameref{def:Array}s:
  \begin{itemize}[noitemsep]
  \item All data values have the same \nameref{def:Variable_Type}. This allows for easy subscripting of \nameref{def:Memory} addresses
  \end{itemize}
\item \nameref{def:Record_Data_Type}
  \begin{itemize}[noitemsep]
  \item When the collection of data values is heterogeneous
  \item The different data \nameref{def:Record_Data_Type_Field}s are not processes the same way
  \item The \nameref{def:Record_Data_Type_Field}s are not processed in any particular order
  \item \nameref{def:Record_Data_Type_Field}s are like named subscripts
  \item Since \nameref{def:Record_Data_Type_Field} names are static, they provide efficient access to fields.
  \end{itemize}
\end{itemize}

\subsubsection{Implementation of \nameref*{subsec:Record_Data_Types}}\label{subsubsec:Implementation_of_Record_Types}
The \nameref{def:Record_Data_Type_Field}s of \nameref{def:Record_Data_Type}s are stored in adjacent \nameref{def:Memory} locations.
But because each \nameref{def:Record_Data_Type_Field} may be of a different size, the offset address, relative to the beginning of the \nameref{def:Record_Data_Type} is associated with each field.
These are calculated at compile-time.
This way, there are no runtime calculations that need to be done.

\subsection{Tuple Types}\label{subsec:Tuple_Types}
\begin{definition}[Tuple]\label{def:Tuple}
  A \emph{tuple} is a \nameref{def:Variable_Type} that is similar to a \nameref{def:Record_Data_Type}, except the elements are \textbf{not} named.

  \begin{remark}
    \nameref{def:Tuple}s can be used to return multiple values from a function in languages that do not natively support that.
  \end{remark}

  \begin{remark}
    Python natively supports these.
    These behave similarly to \nameref{def:List}s, but are immutable.
  \end{remark}
\end{definition}

\begin{minted}[frame=lines,linenos]{python3}
myTuple = (3, 5.8, 'apple')
\end{minted}

\subsection{Union Types}\label{subsec:Union_Types}
\begin{definition}[Union]\label{def:Union_Type}
  A \emph{union} is a \nameref{def:Data_Type} whose \nameref{def:Variable}s may store different \nameref{def:Variable_Type} \nameref{def:Variable_Value}s at different times during program execution.
  \textbf{However, ONLY ONE value is stored at a time.}

  This means that in a C-program, which uses \nameref{def:Free_Union}s will only hold one thing at a time, and \textbf{based on the field that you call in the union, the program will interpret the bit pattern differently}.
  This site gives a good exposition of this point: \href{https://www.tutorialspoint.com/cprogramming/c_unions.htm}{Greater Exposition}.

  The union will allocate the maximum space required by all the \nameref{def:Variable_Type}s of the union, and only use the parts it needs, based on what \nameref{def:Variable_Type}s are actually in use.
\end{definition}

\subsubsection{Design Issues}\label{subsubsec:Union_Types-Design_Issues}
\begin{itemize}[noitemsep]
\item Should \nameref{def:Type_Checking} be required?
  \begin{itemize}[noitemsep]
  \item This \nameref{def:Type_Checking} will have to be dynamic, running during program execution.
  \end{itemize}
\item Should \nameref{def:Union_Type}s be embedded in \nameref{def:Record_Data_Type}?
\end{itemize}

\subsubsection{Discriminated vs. Free Unions}\label{subsubsec:Union_Types-Discriminated_vs_Free}
The \texttt{union} construct in C/C++ is used to specify \nameref{def:Union_Type} structures.
These are called \nameref{def:Free_Union}s.
These stand in stark contrast to \nameref{def:Discriminated_Union}s.
\begin{definition}[Free Union]\label{def:Free_Union}
  A \emph{free union} is a \nameref{def:Union_Type} that have \textbf{\textit{NO}} \nameref{def:Type_Checking} enforced on their use.
  For example, this C snippet:
\begin{minted}[frame=lines,linenos]{c}
union flexType {
    int intEl;
    float floatEl;.
};
union flexType el1;
float x;
...
el1.intEl = 27;
x = el1.floatEl; // NOT type checked, because current type of el1 cannot be determined
\end{minted}
  The last assignment is not type checked, because the the current \nameref{def:Variable_Type} of \texttt{el1} cannot be checked.
  So, \texttt{27} is assigned to the \texttt{float} variable \texttt{x}, which is nonsense.
\end{definition}

\begin{definition}[Discriminated Union]\label{def:Discriminated_Union}
  A \emph{discriminated union} makes use of a \emph{tag} or \emph{discriminant} as a \nameref{def:Data_Type} indicator.
  These allow for the \nameref{def:Type_Checking} of \nameref{def:Union_Type}s during runtime.
\end{definition}

\subsubsection{Ada \nameref*{subsec:Union_Types}}\label{subsubsec:Ada_Union_Types}
Ada allows the use to specify variables for a variant \nameref{def:Record_Data_Type} that will only store one of the possible \nameref{def:Variable_Type} values in the variant.

\begin{definition}[Constrained Variant Variable]\label{def:Union_Type-Constrained_Variant_Variable}
  A \emph{constrained variant variable} is when a language allows the programmer to specify the types present in a variant \nameref{def:Data_Type}, allowing for static \nameref{def:Type_Checking}.
  These enforce that variants can only be changed by assigning the entire record at a time.

  This is shown in the code snippet below.
\begin{minted}[frame=lines,linenos]{ada}
type Shape is (Circle, Triangle, Rectangle); -- Construct enumeration for types of shapes possible
type Colors is (Red, Green, Blue); -- Construct enumeration for colors
type Figure (Form : Shape) is record -- Figure has variant Form records of type Shape (from enumeration)
    Filled : Boolean;
    Color : Colors;
    case Form is
        when Circle =>
            Diameter : Float;
        when Triangle =>
            Left_Side : Integer;
            Right_Side : Integer;
            Angle : Float;
        when Rectangle =>
            Side_1 : Integer;
            Side_2 : Integer;
    end case
end record;
...
Figure_1 : Figure; -- Unconstrained variant record of the record type Figure, and has no initial values
Figure_2 : Figure(Form => Triangle); -- Constrain the variant record to a Triangle
-- Figure_1's type can be changed by the assignment of a whole record
Figure_1 := (Filled => True,
             Color => Blue,
             Form => Rectangle,
             Side_1 => 12,
             Side_2 => 3);
\end{minted}
\end{definition}

\subsubsection{Evaluation}\label{subsubsec:Union_Types-Evaluation}
\nameref{def:Union_Type}s are potentially unsafe constructs in some languages, because they cannot be type checked.
C and C++ are not strongly typed for this reason.
However, Ada, ML, Haskell, and F\# are strongly typed, because they can perform \nameref{def:Type_Checking} on \nameref{def:Union_Type}s.
Some languages, like Java and C\# do not even include the ability to construct a \nameref{def:Union_Type}.

\subsubsection{Implementation of \nameref*{subsec:Union_Types}}\label{subsubsec:Union_Types-Implementation}
\nameref{def:Union_Type}s are implemented by using the same address for a single \nameref{def:Union_Type}, no matter its variant.
Enough storage is allocated for the largest possible variant.
Then, depending on the variant, the space will be used according to how the \nameref{def:Union_Type} was defined.
The tag of the \nameref{def:Union_Type} variant is stored in its \nameref{def:Descriptor}.

\subsection{Type Equivalence}\label{subsec:Type_Equivalence}
This section does not deal with \nameref{def:Type_Compatibility}, which works for scalar \nameref{def:Data_Type}s, but rather \nameref{def:Type_Equivalence}.
\begin{definition}[Type Compatibility]\label{def:Type_Compatibility}
  \emph{Type compatibility} dictates the \nameref{def:Data_Type} of operands that are acceptable for eacch of the operations of the language.
  This is called compatibility because there are cases when the \nameref{def:Data_Type} of the operand can be implicitly converted by the compiler or run-time system to make the operand acceptable to the operator.

  An example of this is the addition of an integer number and a real number.
  The integer number is typecast to a real number, then the addition is performed.
\end{definition}

\nameref{def:Type_Compatibility} rules are strict for predefined scalar types.
However, structured \nameref{def:Data_Type}s such as \nameref{def:Array}s, \nameref{def:Record_Data_Type}, and others require more complex rules.
Since \nameref{def:Data_Type} coercion is unlikely, the question is if the 2 \nameref{def:Data_Type}s are equivalent.

\begin{definition}[Type Equivalence]\label{def:Type_Equivalence}
  \emph{Type equivalence} is a strict form of \nameref{def:Type_Compatibility}.
  It is when an operand of one \nameref{def:Data_Type} can be substituted for another operand of the same type, without \nameref{def:Data_Type} coercion.

  The design of type equivalence rules influence the design of \nameref{def:Data_Type}s and operations provided for values of those types.

  There are 2 approaches to determining \nameref{def:Type_Equivalence}:
  \begin{enumerate}[noitemsep]
  \item \nameref{def:Type_Equivalence-Name}
  \item \nameref{def:Type_Equivalence-Structure}
  \end{enumerate}

  There is a general algorithm for determining if two \nameref{def:Data_Type}s are equivalent.
  ``$\ldots$ when all constant expressions are replaced by their values and all type names are replaced by their definitions.
  In the case of recursive types, the expansion is the infinite limit of the partial expansions $\ldots$''.
  Meaning you replace everything you can, and if there is an infinite recursion somewhere, you cut it off eventually, usually when you pass the same point again.

  There are 4 types of equivalence that can be established.
  \begin{enumerate}[noitemsep]
  \item \nameref{def:Type_Equivalence-Primitive}
  \item \nameref{def:Type_Equivalence-Structure}
  \item \nameref{def:Type_Equivalence-Reference}
  \item \nameref{def:Type_Equivalence-User_Defined}
  \end{enumerate}
\end{definition}

Different languages use different approaches and combinations of these types of \nameref{def:Type_Equivalence}.
You would have to look at the language's specification to find out exactly what is used where.
Additionally, object-oriented languages present their own, unique, type of \nameref{def:Type_Compatibility} with object compatibility and its relationship to the inheritance hierarchy.

\begin{definition}[Name Type Equivalence]\label{def:Type_Equivalence-Name}
  \emph{Name type equivalence} means that 2 \nameref{def:Variable}s have \nameref{def:Type_Equivalence} if they are defined in:
  \begin{itemize}[noitemsep]
  \item The same declaration
  \item In Declarations that use the same \nameref{def:Data_Type} name
  \end{itemize}

  This is easier to implement, but more restrictive.
  In a strict interpretation, a \nameref{def:Variable} whose \nameref{def:Variable_Type} is a subrange of the integers would \textbf{not} be equivalent to an integer \nameref{def:Variable_Type} \nameref{def:Variable}.
  For example,
\begin{minted}[frame=lines,linenos]{ada}
type Indextype is 1..100;
count : Integer;
index : Indextype;
\end{minted}
  \texttt{count} and \texttt{index} could \textbf{not} be substituted for each other.

  \begin{remark}
    To use \nameref{def:Type_Equivalence-Name}, all \nameref{def:Variable_Type}s must have names.
    If the language supports anonymous \nameref{def:Data_Type}s, then they must be given internal names by the compiler/interpreter.
  \end{remark}
\end{definition}

\begin{definition}[Primitive Equivalence]\label{def:Type_Equivalence-Primitive}
  \emph{Primitive equivalence} directly compares two values by comparing their bit patterns in memory.
\end{definition}

\begin{definition}[Structure Type Equivalence]\label{def:Type_Equivalence-Structure}
  \emph{Structure type equivalence} means 2 \nameref{def:Variable}s have \nameref{def:Type_Equivalence} if their types have identical structures.
  The entire structure's \nameref{def:Variable_Type}s must be compared to determine equivalence.
  This may potentially invole recursing through the structure or iterating over the structure.
  However, this is more flexible than \nameref{def:Type_Equivalence-Name}, but more difficult to implement.

  Ada, with its hyper-strict \nameref{def:Type_Equivalence} has defined 2 ways to make new types.
  \begin{enumerate}[noitemsep]
  \item \nameref{def:Derived_Type}
  \item \nameref{def:Subtype}
  \end{enumerate}
\end{definition}

\begin{definition}[Reference Equivalence]\label{def:Type_Equivalence-Reference}
  \emph{Reference Equivalence} checks whether two pointers/references point to the same address in \nameref{def:Memory}.
\end{definition}

\begin{definition}[User-Defined Equivalence]\label{def:Type_Equivalence-User_Defined}
  \emph{User-defined equivalence} performs arbitrary equality checking but isn't provided by the language itself, and must be specified by the programmer.
\end{definition}

\begin{definition}[Derived Type]\label{def:Derived_Type}
  A \emph{derived type} is a new \nameref{def:Data_Type} that is based on some previously defined \nameref{def:Data_Type}, that \textbf{is not equivalent, but may have an identical structure}.
  Derived types inherit all the properties of their parent types.

  Take the following Ada code snippet as an example.
\begin{minted}[frame=lines,linenos]{ada}
type Celsius is new Float;
type Fahrenheit is new Float;
\end{minted}
  These are not equivalent, though they have identical structures.
  They are also not type equivalent to any other \texttt{Float} type.
\end{definition}

\begin{definition}[Subtype]\label{def:Subtype}
  A \emph{subtype} is a possibly range-constrained version of an existing type.
  A subtype \textbf{is equivalent with its parent type}.

  Take the following Ada code snippet as an example.
\begin{minted}[frame=lines,linenos]{ada}
subtype Small_type is Integer range 0..99;
\end{minted}
  The \texttt{Small\textunderscore{}type} is equivalent to the \texttt{Integer} type.
\end{definition}
%%% Local Variables:
%%% mode: latex
%%% TeX-master: "../EDAP05-Concepts_Programming_Languages-Reference_Sheet"
%%% End:


\section{Advanced Data Types}\label{sec:Advanced_Data_Types}
\begin{definition}[Polymorphism]\label{def:Polymorphism}
  
\end{definition}

\subsection{Parametric Polymorphism}\label{subsec:Parametric_Polymorphism}
\begin{definition}[Parametric Polymorphism]\label{def:Parametric_Polymorphism}

  \begin{remark}[Generics]\label{rmk:Generics}
    Outside of the \nameref{def:Functional_Programming_Language} world, \nameref{def:Parametric_Polymorphism} is usually refered to as \emph{generics}.
    Subprograms that make use of these generics are called \emph{generic subprograms}.
  \end{remark}
\end{definition}

\subsection{Ad-Hoc Polymorphism}\label{subsec:Ad_Hoc_Polymorphism}
\begin{definition}[Ad-Hoc Polymorphism]\label{def:Ad_Hoc_Polymorphism}
  
\end{definition}

\subsection{Subtype Polymorphism}\label{subsec:Subtype_Polymorphism}
\begin{definition}[Subtype Polymorphism]\label{def:Subtype_Polymorphism}
  
\end{definition}

%%% Local Variables:
%%% mode: latex
%%% TeX-master: "../EDAP05-Concepts_Programming_Languages-Reference_Sheet"
%%% End:


\section{Expressions}\label{sec:Expressions}
\begin{definition}[Expression]\label{def:Expression}
  An \emph{expression} is a combination of one or more \nameref{def:Operand}s and operators that the programming language interprets (according to its particular rules of precedence and of association) and computes to produce another value.
\end{definition}

\begin{definition}[Operand]\label{def:Operand}
  An \emph{operand} is a:
  \begin{itemize}[noitemsep]
  \item Constant
  \item Variable
  \item Another \nameref{def:Expression}
  \item Result from function calls
  \end{itemize}
\end{definition}

\subsection{Arithmetic Expressions}\label{subsec:Arithmetic_Expressions}
One of the main goals of high-level languages was to have automatic evaluation of \nameref{def:Expression}s similar to those in math, science, and engineering.
Most of these characteristics came from mathematics directly.

\subsubsection{Arity}\label{subsubsec:Operator_Arity} % Not in textbook, from video
\begin{definition}[Arity]\label{def:Arity}
  \emph{Arity} is the number of \nameref{def:Operand}s that must be present to evaluate that \nameref{def:Expression}.
  In most programming languages, there are 3:
  \begin{enumerate}[noitemsep]
  \item \nameref{def:Arity-Unary}
  \item \nameref{def:Arity-Binary}
  \item \nameref{def:Arity-Ternary}
  \end{enumerate}
\end{definition}

\begin{definition}[Unary]\label{def:Arity-Unary}
  A \emph{unary} operation requires a single \nameref{def:Operand}.
  For example, negation, \texttt{-x}
\end{definition}

\begin{definition}[Binary]\label{def:Arity-Binary}
  A \emph{binary} operation requires 2 \nameref{def:Operand}s.
  For example, addition, \texttt{y + z}

  \begin{remark}
    These are usually use \nameref{def:Fixity-Infix} notation.
  \end{remark}
\end{definition}

\begin{definition}[Ternary]\label{def:Arity-Ternary}
  A \emph{ternary} operation requires 3 \nameref{def:Operand}s.
  For example, \verb|w = if x ? y : z|.

  \begin{remark}
    As far as I know, the only ternary operator is a single-line if statement.
  \end{remark}
\end{definition}

\subsubsection{Fixity}\label{subsubsec:Operator_Fixity} % Not in textbook, from video
\subsubsection{Operator Evaluation Order}\label{subsubsec:Operator_Evaluation_Order}
\paragraph{Precedence}\label{par:Operator_Evaluation_Order-Precedence}
\paragraph{Associativity}\label{par:Operator_Evaluation_Order-Associativity}
\paragraph{Parentheses}\label{par:Operator_Evaluation_Order-Parentheses}
\paragraph{Ruby Expressions}\label{par:Operator_Evaluation_Order-Ruby_Expressions}
\paragraph{LISP Expressions}\label{par:Operator_Evaluation_Order-LISP_Expressions}
\paragraph{Conditional Expressions}\label{par:Operator_Evaluation_Order-Conditional_Expressions}

\subsubsection{Operand Evaluation Order}\label{subsubsec:Operand_Evaluation_Order}
\paragraph{Side Effects}\label{par:Operand_Evaluation-Side_Effects}
\paragraph{Referential Transparency and Side Effects}\label{par:Operand_Evaluation-Referential_Transparency_Side_Effects}

\subsection{Relational and Boolean Expressions}\label{subsec:Relational_Boolean_Expressions}
\subsubsection{Relational Expressions}\label{subsubsec:Relational_Expressions}
\subsubsection{Boolean Expressions}\label{subsubsec:Boolean_Expressions}

\subsection{Short-Circuit Evaluation}\label{subsec:Short_Circuit_Evaluation}

%%% Local Variables:
%%% mode: latex
%%% TeX-master: "../EDAP05-Concepts_Programming_Languages-Reference_Sheet"
%%% End:


\section{Assignment Statements}\label{sec:Assignment_Statement}
Assignments are one of the central constructs in \nameref{def:Imperative_Programming_Language}s.
Assignments allow progarmmers to dynamically change the bindings of \nameref{def:Binding}s to \nameref{def:Variable}s.

\subsection{Simple Assignments}\label{subsec:Simple_Assignments}
Nearly all languages use the \texttt{=} symbol as the assignment operator.
In these languages, the use of \texttt{==} as the equivalence \nameref{def:Relational_Operator} is common.

2 notable exceptions to this assignment symbol are:
\begin{enumerate}[noitemsep]
\item ALGOL 60, with \texttt{:=}
\item Ada, with \texttt{:=}
\end{enumerate}

They chose this set of symbols so as to be able to use the \texttt{=} as the equivalence \nameref{def:Relational_Operator}.
The destination of the \nameref{def:Variable_Value} has varied widely in many languages.
Additionally, the appearance of an assignment as a stand-alone statement is also dependent on language.

\subsection{Conditional Targets}\label{subsec:Conditional_Targets}
Some languages allow for assignments to take place in conditional \nameref{def:Block_Scope}s.
These assignments bind the \nameref{def:Variable_Value} to a \emph{conditional target}.

In Perl for example,
\begin{minted}[frame=lines,linenos]{perl}
($flag ? $count1 : $count2) = 0;
\end{minted}
which is equivalent to
\begin{minted}[frame=lines,linenos]{perl}
if ($flag) {
    $count1 = 0;
} else {
    $count2 = 0;
}
\end{minted}

\subsection{Compound Assignment Operators}\label{subsec:Compound_Assignment_Operators}
\begin{definition}[Compound Assignment Operator]\label{def:Compound_Assignment_Operator}
  A \emph{compound assignment operator} is a shorthand method of specifying a commonly needed form of assignment.
  The overwritting of a \nameref{def:Variable}'s \nameref{def:Variable_Value} with some new value that is dependent on the \nameref{def:Variable}'s previous \nameref{def:Variable_Value}.

  For example:
\begin{minted}[frame=lines,linenos]{python3}
sum += value;
# This is equivalent to
sum = sum + value;
\end{minted}

  \begin{remark}[Supported Operators]\label{rmk:Supported_Compount_Assignment_Operators}
    If a language supports an addition \nameref{def:Compound_Assignment_Operator}, then it is likely that is supports it for all the other binary arithmetic operations.
    They are likely to be denoted \texttt{-=}, \texttt{*=}, \texttt{/=}, and \texttt{\%=} for subtraction, multiplication, division, and modulo, respectively.
  \end{remark}
\end{definition}

\subsection{Unary Assignment Operators}\label{subsec:Unary_Assignment_Operators}
Many C-based langagues support 2 special unary arithmetic operators, that are abbreviated assignment statements.
These are:
\begin{enumerate}[noitemsep]
\item Increment by 1, usually written \texttt{++}
  \begin{itemize}[noitemsep]
  \item This is equivalent to \texttt{i = i + 1}, where \texttt{i} is a defined, in-scope, aliveinteger \nameref{def:Variable}
  \end{itemize}
\item Decrement by 1, usually written \texttt{--}
  \begin{itemize}[noitemsep]
  \item This is equivalent to \texttt{i = i - 1}, where \texttt{i} is a defined, in-scope, aliveinteger \nameref{def:Variable}
  \end{itemize}
\end{enumerate}

Both of these can be \nameref{def:Fixity-Prefix} and \nameref{def:Fixity-Suffix} operators.
The different fixities mean different things.
\begin{itemize}[noitemsep]
\item \nameref{def:Fixity-Prefix}, \texttt{sum = ++count}
  \begin{itemize}[noitemsep]
  \item First increment \texttt{count}
  \item Then assign the result to \texttt{sum}
  \end{itemize}
\item \nameref{def:Fixity-Suffix}, \texttt{sum = count++}
  \begin{itemize}[noitemsep]
  \item First assign the value of \texttt{count} to \texttt{sum}
  \item Increment the value stored in \texttt{count} by 1
  \item Note, the value in \texttt{sum} will be the value that was in \texttt{count} \textbf{BEFORE} the increment occurred.
  \end{itemize}
\end{itemize}

\begin{remark*}[Multiple Unary Operators Associativity]
  When there are multiple unary operators on a single \nameref{def:Variable} at a single time, they are \textbf{RIHT-TO-LEFT ASSOCIATIVE}.
  Thus,
\begin{minted}[frame=lines,linenos]{python3}
- count ++
# is handled like
- (count++)
# rather than
(- count) ++
\end{minted}
\end{remark*}

\subsection{Assignment as an Expression}\label{subsec:Assignment_as_Expression}
In C-based langauges, the assignment statement also produces a result that is the same as the value assigned.
This means that the assignment statement can be treated as an \nameref{def:Expression}, and an \nameref{def:Operand} as well, and used as such.
For example, this is a valid while-loop in C.
\begin{minted}[frame=lines,linenos]{c}
while ((ch = getchar()) != EOF) { ... }
\end{minted}

This statement gets a character from the standard input, which is usually the keyboard.
If the letter input to the program is \textbf{not} the End-Of-File character, then the statement(s) in the while-loop's body execute.
The character was \textbf{ALSO} assigned to the character \nameref{def:Variable}, \texttt{ch}.

\subsubsection{Side Effects}\label{subsubsec:Assignment_as_Expression-Side_Effects}
Allowing this action in programs means that \nameref{def:Expression}s can be more difficult to read and udnerstand.
Instead of thinking of these kinds of \nameref{def:Expression}s as \nameref{def:Expression}s, they must be though of as a list of instructions with a strange order of execution.
For example,
\begin{minted}[frame=lines,linenos]{c}
a = b + (c = d / b) - 1;
// This gets translated to
c = d / b;
a = b + c - 1;
\end{minted}

In C programs, this is a big cause of problems.
Because both \texttt{if (x = y)} and \texttt{if (x == y)} are valid, but the second is performing a \nameref{def:Relational_Operator}, while the first is performing an assignment.
Java and C\# have avoided this problem by \textbf{only} allowing boolean expressions in their \texttt{if} statements.

%%% Local Variables:
%%% mode: latex
%%% TeX-master: "../EDAP05-Concepts_Programming_Languages-Reference_Sheet"
%%% End:


\section{Statement-Level Control Structures}\label{sec:Statement_Level_Control_Structures}
\begin{definition}[Control Statement]\label{def:Control_Statement}
  \emph{Control statement}s provide the ability to:
  \begin{itemize}[noitemsep]
  \item Select alternative control flow paths of statement execution
  \item Cause repeated execution of statements or sequences of statements
  \end{itemize}
\end{definition}

\begin{definition}[Control Structure]\label{def:Control_Structure}
  A \emph{control structure} is a \nameref{def:Control_Statement} and the collection of statements whose execution it controls.
\end{definition}

\subsection{Selection Statements}\label{subsec:Selection_Statements}
\begin{definition}[Selection Statement]\label{def:Selection_Statement}
  A \emph{selection statement} provides a means of choosing between 2 or more execution paths in a program.

  There are 2 general categories of these:
  \begin{enumerate}[noitemsep]
  \item \nameref{subsubsec:2_Way_Selection}
  \item \nameref{subsubsec:N_Way_Selection}
  \end{enumerate}
\end{definition}

\subsubsection{2 Way Selection}\label{subsubsec:2_Way_Selection}
\paragraph{Design Issues}\label{par:2_Way_Selection-Design_Issues}
\paragraph{The Control Expression}\label{par:2_Way_Selection-Control_Expression}
\paragraph{Clause Form}\label{par:2_Way_Selection-Clause_Form}
\paragraph{Nesting Selectors}\label{par:2_Way_Selection-Nesting}
\paragraph{Selector Expressions}\label{par:2_Way_Selection-Selector_Expressions}

\subsubsection{N Way Selection}\label{subsubsec:N_Way_Selection}
\paragraph{Design Issues}\label{par:N_Way_Selection-Design_Issues}
\paragraph{Examples of Multiple Selectors}\label{par:N_Way_Selection-Examples}

\subsection{Iterative Statements}\label{subsec:Iterative_Statements}
\begin{definition}[Iterative Statement]\label{def:Iterative_Statement}
  An \emph{iterative statement} is one that causes a statement or collection of statements to be executed zero ore more times.
  These are frequently called \emph{loops}.

  There are 2 questions that need to be answered to design iterative statements.
  \begin{enumerate}[noitemsep]
  \item How is the iteration controlled?
  \item Where should the control mechanism appear in the loop statement, i.e.\ when is it executed in relation to the code in the statement body?
  \end{enumerate}

  \begin{remark}[Iteration Statement]\label{rmk:Iteration_Statement}
    An \emph{iteration statement} is the combination of the \nameref{def:Iterative_Statement-Body} and the control mechanism.
  \end{remark}
\end{definition}

\nameref{subsubsec:Counter_Controlled_Loops} and \nameref{subsubsec:Counter_Controlled_Loops}, and a combination of the 2, are the main ways to control an \nameref{def:Iterative_Statement}.
The 2 main choices of when to execute the control mechanism of the loop are:
\begin{enumerate}[noitemsep]
\item At the beginning of the loop, before the body. \nameref{def:Iterative_Statement-Pretest_Control}
\item At the end of the loop, after the body. \nameref{def:Iterative_Statement-Posttest_Control}
\end{enumerate}

\begin{definition}[Body]\label{def:Iterative_Statement-Body}
  The \emph{body} of an \nameref{def:Iterative_Statement} is the collection of statements whose execution is controlled by the iteration station statement.
\end{definition}

\begin{definition}[Pretest]\label{def:Iterative_Statement-Pretest_Control}
  A \emph{pretest} \nameref{def:Iterative_Statement} means the test for loop completion, the control mechanism, is evaluated before the loop \nameref{def:Iterative_Statement-Body} executes.
\end{definition}

\begin{definition}[Posttest]\label{def:Iterative_Statement-Posttest_Control}
  A \emph{posttest} \nameref{def:Iterative_Statement} means the test for loop completion, the control mechanism, is evaluated after the loop \nameref{def:Iterative_Statement-Body} executes.
\end{definition}

\subsubsection{Counter-Controlled Loops}\label{subsubsec:Counter_Controlled_Loops}
\begin{definition}[Counter Iteration Statement]\label{def:Counter_Iteration_Statement}
  A \emph{counter iteration statement} has:
  \begin{itemize}[noitemsep]
  \item A \emph{loop variable}, where the count value is maintained
  \item The \emph{loop parameters}. These include
    \begin{itemize}[noitemsep]
    \item A means of specifying the \emph{initial} and \emph{terminal} values of the loop variable
    \item The \emph{step size}, the difference between sequential loop variable values
    \end{itemize}
  \end{itemize}
\end{definition}

\nameref{subsubsec:Logically_Controlled_Loops} are more \textbf{general} than \nameref{subsubsec:Counter_Controlled_Loops}, they are not necessarily more commonly used.
The greater complexity of \nameref{subsubsec:Counter_Controlled_Loops} means that their design is more demanding.

\begin{remark*}[Machine Instructions]
  Sometimes there are machine-level instructions for performing \nameref{subsubsec:Counter_Controlled_Loops}.
  However, these are rare, because a language can outlive a processor's ISA.\@
\end{remark*}

\paragraph{Design Issues}\label{par:Counter_Controlled_Loops-Design_Issues}
\begin{itemize}[noitemsep]
\item What are the \nameref{def:Data_Type}, and \nameref{def:Variable_Scope} of the loop variable?
\item Should it be legal for the loop variable or loop parameters to be changed in the loop, and if so, does it affect the loop control?
\item Should the loop parameters be evaluated only once, or once for each iteration?
\end{itemize}

\paragraph{The \texttt{for} Statement of Ada}\label{par:Counter_Controlled_Loops-Ada}
The Ada \texttt{for} statement can be generalized to
\begin{minted}[frame=lines,linenos]{ada}
for variable in [reverse] discrete_range loop
    -- Body statements
end loop;
\end{minted}
\begin{itemize}[noitemsep]
\item The \texttt{discrete\textunderscore{}range} is a \nameref{def:Subrange_Type} of an integer or \nameref{def:Enumeration_Type}, such as \texttt{1..10} or \texttt{Monday..Friday}.
\item The \texttt{reverse} \nameref{def:Reserved_Word} indicates the values of the \texttt{discrete\textunderscore{}range} are assigned in reverse order.
\item The \texttt{variable} is only in scope of the loop
  \begin{itemize}[noitemsep]
  \item It is implicitly declared at the beginning of the \texttt{for} statement
  \item and implicitly undeclared after loop termination
  \item It also shadows/masks variables with the same name as it
  \end{itemize}
\item The \texttt{body} cannot assign the loop variable a new value
\end{itemize}

The operational semantics are given below.
\begin{verbatim}
  [define for_var (its type is that of the discrete range)]
  [evaluate discrete range]
loop:
  if [there are no elements left in the discrete range] goto out
  for_var = [next element of discrete range]
  [loop body]
  goto loop
out:
  [undefine for_var]
\end{verbatim}

\paragraph{The \texttt{for} Statement of the C-Based Languages}\label{par:Counter_Controlled_Loops-C_Langs}
The C \texttt{for} statement can be generalized to
\begin{minted}[frame=lines,linenos]{c}
for (expression_1; expression_2; expression_3) {
    // Body statements
}
\end{minted}
\begin{itemize}[noitemsep]
\item \texttt{expression\textunderscore{}1} is usually an assignment statement to create the loop variable, which can produce results
  \begin{itemize}[noitemsep]
  \item This is only evaluated once, before the loop begins
  \item In early versions of C, these could \textbf{not} be definitions (\texttt{int count = 0;})
  \end{itemize}
\item \texttt{expression\textunderscore{}2} is the loop control, it makes comparisons against the loop variable
  \begin{itemize}[noitemsep]
  \item This is evaluated before every iteration of the loop body
  \end{itemize}
\item \texttt{expression\textunderscore{}3} manipulates the loop variable
  \begin{itemize}[noitemsep]
  \item This is evaluated after every iteration of the loop body
  \end{itemize}
\item All 3 of these expressions are optional.
  \begin{itemize}[noitemsep]
  \item Having no \texttt{expression\textunderscore{}2} means the comparison is always true, which may result in an infinite loop
  \item If \texttt{expression\textunderscore{}1} is absent, no loop variable is initialized, but that doesn't stop other \nameref{def:Variable}s declared elsewhere from being part of the control.
  \end{itemize}
\end{itemize}

Because C expressions can be used as statements, expression evaluations are shown as statements in this operational semantics.
\begin{verbatim}
  expression_1
loop:
  if expression_2 = 0 goto out
  [loop body]
  expression_3
  goto loop
out:
  ...
\end{verbatim}

C's design choices for its \texttt{for} loop are as follows:
\begin{itemize}[noitemsep]
\item No explicit loop variables or loop parameters
\item All involved variables can be changed in the loop body
\item Expressions are evaluated in the order shown before.
  \begin{itemize}[noitemsep]
  \item There can even be multiple expressions present in each of the \texttt{expression}s.
  \end{itemize}
\item You can branch into a C \texttt{for} loop body
\end{itemize}

\paragraph{The \texttt{for} Statement of Python}\label{par:Counter_Controlled_Loops-Python}
The Python \texttt{for} statement can be generalized to
\begin{minted}[frame=lines,linenos]{python3}
for loop_variable in object:
    # Body statements
[else:
    # Else clause]
\end{minted}

\begin{itemize}[noitemsep]
\item The \texttt{loop\textunderscore{}variable} is assigned the one of the values in the object, which is often a range, for each iteration.
\item The \texttt{else} clause is executed if the loop terminates \textbf{normally}
\end{itemize}

\subsubsection{Logically Controlled Loops}\label{subsubsec:Logically_Controlled_Loops}
Collections of statements must be executed repeatedly, but should be controlled by a Boolean expression rather than a counter.
Logically controlled loops are convenient for this.
They are also more general than \nameref{subsubsec:Counter_Controlled_Loops}.

\paragraph{Design Issues}\label{par:Logically_Controlled_Loops-Design_Issues}
\begin{itemize}[noitemsep]
\item Should the control be \nameref{def:Iterative_Statement-Pretest_Control} or \nameref{def:Iterative_Statement-Posttest_Control}?
\item Should the logically controlled loop be a special form of a counting loop or a separate statement?
\end{itemize}

\paragraph{Examples}\label{par:Logically_Controlled_Loops-Examples}
C-based programming languages have both \nameref{def:Iterative_Statement-Pretest_Control} and \nameref{def:Iterative_Statement-Posttest_Control} \nameref{subsubsec:Logically_Controlled_Loops}.
These loops have these respective forms:
\begin{minted}[frame=lines,linenos]{c}
while (control_expression) {
    // Loop body
}
\end{minted}
\begin{minted}[frame=lines,linenos]{c}
do
    // Loop body
while (control_expression);
\end{minted}

The \nameref{def:Iterative_Statement-Pretest_Control} (\texttt{while}) executes as long as the \texttt{control\textunderscore{}expression} evaluates to true.
The \nameref{def:Iterative_Statement-Posttest_Control} (\texttt{do-while}) executes as long as the \texttt{control\textunderscore{}expression} evaluates to true.
However, the \texttt{do-while} loop will cause the loop body to be executed at least once.

The operational semantics are given below, with the \texttt{while} coming first, then the \texttt{do-while}.
\begin{verbatim}
loop:
  if control-expression is falst go to outlive
  [loop body]
  goto loop
out:
  ...
\end{verbatim}

\begin{verbatim}
loop:
  [loop body]
  if control_expression is true goto loop
\end{verbatim}

In C, it is possible to branch into these statement's bodies.
Java's \texttt{while} and \texttt{do-while} statements are similar, but they require the \texttt{control\textunderscore{}expression} be \texttt{boolean} type.

\subsubsection{Iteration Based on Data Structures}\label{subsubsec:Iteration_Based_on_Data_Structures}
A general data-based iteration statement uses a user-defined data structure and user-defined iteration to go through the elements in the structure.
An example of this is in Python, shown below
\begin{minted}[frame=lines,linenos]{python3}
for count in range(0, 9, 2)
    # Body
\end{minted}
This would set the value of \texttt{count} to \texttt{0, 2, 4, 6, 8} before each iteration.

The user-defined iterator function must be history sensitive.
It should also terminate when it fails to find more elements.

User-defined \nameref{rmk:Iteration_Statement}s are more commonly used and more important in object-oriented programming, because of the heavy use of abstract data types for data structures.
In Java, this is done by having a class implement the \texttt{Iterable} interface.
It is called the \texttt{foreach} loop, and is written as
\begin{minted}[frame=lines,linenos]{java}
for (String myElement : myList) {
    ...
}
\end{minted}

%%% Local Variables:
%%% mode: latex
%%% TeX-master: "../EDAP05-Concepts_Programming_Languages-Reference_Sheet"
%%% End:


\section{Subprograms}\label{sec:Subprograms}
This is a way to perform \nameref{par:Process_Abstraction}.
This generally improves \nameref{subsec:Readability} and \nameref{subsec:Reliability}.

\subsection{General Characteristics}\label{subsec:Suprogram_Characteristics}
\begin{itemize}[noitemsep]
\item Each subprogram has a single entry point
\item The calling program unit is suspended during the execution of the called subprogram, which implies there is only one subprogram in execution at any given time
\item Control always returns to the caller whenthe subprogram terminates
\end{itemize}

Alternatives to these generalizations result in coroutines and concurrent units.

\subsection{Basic Definitions}\label{subsec:Subprogram_Definitions}
\begin{definition}[Subprogram Definition]\label{def:Subprogram_Definition}
  A \emph{subprogram definition} describes the interface to and the actions of the subprogram abstraction
\end{definition}

\begin{definition}[Subprogram Call]\label{def:Subprogram_Call}
  A \emph{subprogram call} is the explicit request that a specific subprogram be executed.

  \begin{remark}[Call]\label{rmk:Subprogram_Call}
    This is generally shortened to just a \emph{call}.
  \end{remark}
\end{definition}

\begin{definition}[Active]\label{def:Subprogram_Active}
  A subprogram is said to be \emph{active} if it has been called, but not yet completed its execution.
\end{definition}

\begin{definition}[Subprogram Header]\label{def:Subprogram_Header}
  A \emph{subprogram header} is the first part of a \nameref{def:Subprogram_Definition}.
  This serves several purposes:
  \begin{enumerate}[noitemsep]
  \item Specifies that the following syntactic unit is a \nameref{def:Subprogram_Definition} of some kind.
  \item Provides a name for the subprogram, if it's not an anonymous subprogram.
  \item Optionally specify a list of parameters.
  \end{enumerate}
\end{definition}

\begin{definition}[Subprogram Body]\label{def:Subprogram_Body}
  The \emph{subprogram body} defines the actions that the subprogram takes when there is a \nameref{def:Subprogram_Call}.
  The body may be delimited with curly-braces, \texttt{\{} and \texttt{\}}.
  It may be whitespace delimited, Python.
  It may also have an \texttt{end} statement that ends the execution of that block.
\end{definition}

Python is unique in that its \nameref{def:Subprogram_Definition}s can be executed in control-statement blocks.
For example,
\begin{minted}[frame=lines,linenos]{python3}
if conditional_expression :
    def func(...):
        ...
else:
    def func(...):
        ...
\end{minted}
This means that there are 2 possible \nameref{def:Subprogram_Definition}s possible during runtime, and which one is currently valid depends on the result of the \texttt{conditional\textunderscore{}expression}.

\begin{definition}[Parameter Profile]\label{def:Subprogram_Parameter_Profile}
  The \emph{parameter profile} of a subprogram contains the number, order, and types of its \nameref{def:Formal_Parameter}s.
\end{definition}

\begin{definition}[Protocol]\label{def:Subprogram_Protocol}
  The \emph{protocol} of a subprogram is its \nameref{def:Subprogram_Parameter_Profile}, and if its a function, its return type.
\end{definition}

\begin{definition}[Subprogram Declaration]\label{def:Subprogram_Declaration}
  A \emph{subprogram declaration} is the act of providing type and name information, but not giving any \nameref{def:Subprogram_Body}.
  This is needed in languages that do not allow forward references to subprograms.

  \begin{remark}[Prototype]\label{rmk:Subprogram_Prototype}
    In C/C++, if a subprogram needs to be declared, it is called a \emph{prototype}.
    These are generally specified in \emph{header} files, with a file extension of \texttt{.h}.
  \end{remark}
\end{definition}

\subsection{Parameters}\label{subsec:Subprogram_Parameters}
Subprograms typically want access to nonlocal data to perform their computations.
There are 2 ways to gain access to this nonlocal data:
\begin{enumerate}[noitemsep]
\item Direct access to nonlocal \nameref{def:Variable}s (\nameref{def:Global_Variable}s)
\item Parameter passing
\end{enumerate}

Data that is passed to the subprogram as a parameter is acccessed through names local to the subprogram.
Parameter passing is more flexible, because if direct access is used, new storage needs to be allocated for computation results.
Direct access also leads to issues with \nameref{def:Variable}s being visible to places they shouldn't be.
Pure \nameref{def:Functional_Programming_Language}s avoid this by having all their data being immutable.

\begin{definition}[Formal Parameter]\label{def:Formal_Parameter}
  A \emph{formal parameter} are the parameters present in the \nameref{def:Subprogram_Header}.
  These are sometimes thought of as ``dummy variables'' because they aren't normal \nameref{def:Variable}s.
  They are only bound to storage when the subprogram is called, and that storage is often through some other program \nameref{def:Variable}s.

  \begin{remark}[Parameter]\label{rmk:Parameter}
    Sometimes \nameref{def:Formal_Parameter}s are just called \emph{parameter}s, usually when \nameref{def:Actual_Parameter}s are called \nameref{rmk:Argument}s.
  \end{remark}
\end{definition}

\begin{definition}[Actual Parameter]\label{def:Actual_Parameter}
  An \emph{actual parameter} is the parameter that is bound to the \nameref{def:Formal_Parameter} of a subprogram.
  
  \begin{remark}[Argument]\label{rmk:Argument}
    Sometimes \nameref{def:Actual_Parameter}s are called \emph{argument}s, usually when \nameref{def:Formal_Parameter}s are called \nameref{rmk:Parameter}s.
  \end{remark}
\end{definition}

The binding of \nameref{def:Actual_Parameter}s to \nameref{def:Formal_Parameter}s is usually done by position.
So, the first \nameref{def:Actual_Parameter} is bound to the first \nameref{def:Formal_Parameter}.
However, this is only a good method when the number of parameters is small.

When the \nameref{def:Formal_Parameter} list gets long, it is hard to get all of the \nameref{def:Actual_Parameter}s in the right order.
One solution is to use \nameref{def:Keyword_Parameter}s.

\begin{definition}[Keyword Parameter]\label{def:Keyword_Parameter}
  \emph{Keyword parameter}s have the name of the \nameref{def:Formal_Parameter} usable by the \nameref{def:Actual_Parameter} to bind the \nameref{def:Variable_Value}.

  For example, if \texttt{sumer} has the \nameref{def:Formal_Parameter}s \texttt{length}, \texttt{list}, and \texttt{sum}:
\begin{minted}[frame=lines,linenos]{python3}
sumer(length = my_length, list = my_array, sum = my_sum)
\end{minted}

  Advantages and Disadvantages of keyword parameters:
  \begin{itemize}[noitemsep]
  \item Advantages
    \begin{itemize}[noitemsep]
    \item No need to remember \nameref{def:Formal_Parameter} order.
    \end{itemize}
  \item Disadvantages
    \begin{itemize}[noitemsep]
    \item Need to remember the name of the \nameref{def:Formal_Parameter}s.
    \end{itemize}
  \end{itemize}

  \begin{remark}[End of \nameref{def:Actual_Parameter}]\label{rmk:Keyword_Parameters_at_End}
    In an \nameref{def:Actual_Parameter} list, all parameters after a \nameref{def:Keyword_Parameter} \textbf{must} be keyworded, because the list may not be well-formed enough for parameters to line up by position.
  \end{remark}
\end{definition}

Languages that support default values on \nameref{def:Formal_Parameter}s handle them differently.
In Python, regular \nameref{def:Formal_Parameter}s and ones with default values can be in any order.
However, in C++, which does not support \nameref{def:Keyword_Parameter}s, \nameref{def:Formal_Parameter}s with default values must be at the end of the \nameref{def:Subprogram_Header}.
This is illustrated in the next 2 code blocks, which are in Python and C++, respectively.
\begin{minted}[frame=lines,linenos]{python3}
def compute_pay(income, exemptions = 1, tax_rate)
\end{minted}

\begin{minted}[frame=lines,linenos]{c++}
float compute_pay(float income, float tax_rate, int exemptions = 1)
\end{minted}

Other languages have more varied and interesting ways to pass \nameref{def:Actual_Parameter}s to subprograms.
Look at the language specification for more details.

\subsection{Local Referencing Environments}\label{subsec:Local_Referencing_Environments}
Issues related to \nameref{def:Variable}s defined within subprograms.

\subsubsection{Local Variables}\label{subsubsec:Local_Variables}
The definition of \nameref{def:Local_Variable}s is given in \Cref{def:Local_Variable}.
These can be either \nameref{def:Static_Variable_Binding_Lifetime} or \nameref{def:Stack-Dynamic_Variable_Binding_Lifetime}.

If \nameref{def:Local_Variable}s are \nameref{def:Stack-Dynamic_Variable_Binding_Lifetime}, they are bound to storage when the subprogram begins and unbound when that execution terminates.
The advantages and disadvantages of \nameref{def:Stack-Dynamic_Variable_Binding_Lifetime} are:
\begin{itemize}[noitemsep]
\item Advantages
  \begin{itemize}[noitemsep]
  \item Allows for recursive subprograms
  \item Inactive subprograms can share \nameref{def:Memory} with the active subprogram
  \end{itemize}
\item Disadvantages
  \begin{itemize}[noitemsep]
  \item The cost of the time requried to allocate, initialize, and deallocate these variables
  \item The indirect \nameref{def:Memory} accesses to the data
  \item When all \nameref{def:Variable}s are \nameref{def:Stack-Dynamic_Variable_Binding_Lifetime}, subprograms cannot be history sensitive.
  \end{itemize}
\end{itemize}

However, the advantages and disadvantages of \nameref{def:Static_Variable_Binding_Lifetime} are:
\begin{itemize}[noitemsep]
\item Advantages
  \begin{itemize}[noitemsep]
  \item No runtime overhead to allocate/deallocate the storage
  \item Direct \nameref{def:Memory} access (Absolute addressing)
  \end{itemize}
\item Disadvantages
  \begin{itemize}[noitemsep]
  \item Inability to support recursion
  \item Cannot share \nameref{def:Memory} with inactive subprograms.
  \end{itemize}
\end{itemize}

Most contemporary programming languages make their \nameref{def:Local_Variable}s \nameref{def:Stack-Dynamic_Variable_Binding_Lifetime} by defautlt.
However, this can usually be overridden with a \texttt{static} keyword.

\subsubsection{Nested Subprograms}\label{subsubsec:Nested_Subprograms}
The idea was to create a hierarchy of logic and \nameref{def:Variable_Scope}s.
The motivation was that if a subprogram is only used within one other subprogram, why not place it there, and hide it from the rest of the program?

\nameref{def:Static_Scoping} is usually used in languages that allow nested subprograms.
Language support of this featuer depends \textbf{heavily} on the language.
You will have to chek the language specification to find out if the programming language supports them.

\subsection{Parameter-Passing Methods}\label{subsec:Parameter_Passing_Methods}
How are \nameref{def:Actual_Parameter}s passed to the subprograms?

\subsubsection{Semantic Models of Parameter Passing}\label{subsubsec:Semantic_Models_Parameter_Passing}
\nameref{def:Formal_Parameter}s are characterized by 1 of 3 distinct \nameref{def:Semantics} models:
\begin{enumerate}[noitemsep]
\item The \nameref{def:Formal_Parameter}s receive data from the corresponding \nameref{def:Actual_Parameter}. This is called \textbf{\nameref{def:Parameter_Passing-In_Mode}}.
\item The \nameref{def:Formal_Parameter}s  can transmit the computed data back to the \nameref{def:Actual_Parameter}. This is called \textbf{\nameref{def:Parameter_Passing-Out_Mode}}.
\item They can do both 1 and 2. This is called \textbf{\nameref{def:Parameter_Passing-Inout_Mode}}.
\end{enumerate}

\begin{definition}[In Mode]\label{def:Parameter_Passing-In_Mode}
  \nameref{def:Formal_Parameter}s can receive data from the corresponding \nameref{def:Actual_Parameter}.
  This is called \emph{in mode}.

  This is generally used when passing \nameref{def:Actual_Parameter}s to a subprogram.
\end{definition}

\begin{definition}[Out Mode]\label{def:Parameter_Passing-Out_Mode}
  \nameref{def:Formal_Parameter}s transmit the computed data back to the corresponding \nameref{def:Actual_Parameter}s.
  This is called \emph{out mode}.

  This is similar to a return value, but acts \emph{only} on the actual parameters.
\end{definition}

\begin{definition}[In/Out Mode]\label{def:Parameter_Passing-Inout_Mode}
  The \nameref{def:Formal_Parameter}s can both transmit and receive data from the corresponding \nameref{def:Actual_Parameter}s.
  This is called \emph{in/out mode}.

  This is generally used when a \nameref{def:Subprogram_Call} takes in \nameref{def:Actual_Parameter}s and returns values in those same parameters.
\end{definition}

There are 2 conceptual models of how data transfers take place in parameter transmission:
\begin{enumerate}[noitemsep]
\item An actual value is copied (to the caller, to the callee, or both).
\item Or an access path is transmitted. This is usually a pointer/reference.
\end{enumerate}

\subsubsection{Implementation Models of Parameter Passing}\label{subsubsec:Implementation_Models_Parameter_Passing}
A great variety of implementations for parameter passing have been put together.
Here, we list some of them, and discuss their respective advantages and disadvantages.

\begin{remark*}[Call-by-...]
  All of these models can have the ``Pass-by'' replaced with ``Call-by'', which means the same thing.
\end{remark*}

\paragraph{Pass-by-Value}\label{par:Parameter_Passing-Pass_By_Value}
\begin{definition}[Pass-by-Value]\label{def:Pass_By_Value}
  When a parameter is \emph{passed-by-value}, the value of the \nameref{def:Actual_Parameter} is used to initialize the corresponding \nameref{def:Formal_Parameter}, which then acts as a \nameref{def:Local_Variable} in the subprogram.
  This implements \nameref{def:Parameter_Passing-In_Mode} \nameref{def:Semantics}.

  Passing a parameter by value is typically done by copying the \nameref{def:Actual_Parameter}'s \nameref{def:Variable_Value} for the \nameref{def:Formal_Parameter}.
  This means we don't have to make the \nameref{def:Memory} cell read-only, because making cells read-only can be difficult.

  The advantages and disadvantages are:
  \begin{itemize}[noitemsep]
  \item Advantages
    \begin{itemize}[noitemsep]
    \item Fast to copy scalars in both linkage cost and access time
    \end{itemize}
  \item Disadvantages
    \begin{itemize}[noitemsep]
    \item Additional \nameref{def:Memory} is required for the \nameref{def:Formal_Parameter}'s new value.
    \item The \nameref{def:Actual_Parameter} must be copied to the storage area for the corresponding \nameref{def:Formal_Parameter}
    \item Difficult to implement by transmitting access paths
    \item This copying can be expensive if the \nameref{def:Actual_Parameter} is large, like an array with many elements.
    \end{itemize}
  \end{itemize}
\end{definition}

\paragraph{Pass-by-Result}\label{par:Parameter_Passing-Pass_By_Result}
\begin{definition}[Pass-by-Result]\label{def:Pass_By_Result}
  \emph{Pass-by-result} is an implementation for \nameref{def:Parameter_Passing-Out_Mode} parameters.
  When a parameter is passed-by-result, no \nameref{def:Variable_Value} is transmitted to the subprogram.
  The corresponding \nameref{def:Formal_Parameter} acts as a \nameref{def:Local_Variable}, but before control is transferred back to the caller, the \nameref{def:Formal_Parameter}'s result is transmitted back to the caller's \nameref{def:Actual_Parameter}.

  The advantages and disadvantages of pass-by-result are fairly similar to \nameref{def:Pass_By_Value}, but with some extra disadvantages.
  \begin{itemize}[noitemsep]
  \item Advantages
    \begin{itemize}[noitemsep]
    \item Fast to copy scalars
    \end{itemize}
  \item Disadvantages
    \begin{itemize}[noitemsep]
    \item If the values are returned by copying the value into the \nameref{def:Actual_Parameter}, extra storage and copy operations are required.
    \item Difficult to transmit an access path.
      \begin{itemize}[noitemsep]
      \item Need to ensure the initial value of the \nameref{def:Actual_Parameter} is \textbf{not} used in the subprogram.
      \end{itemize}
    \item There can be an \nameref{def:Actual_Parameter} collision.
\begin{minted}[frame=lines,linenos]{csharp}
void Fixer(out int x, out int y) {
  x = 17;
  y = 35;
}
...
f.Fixer(out a, out a); // What gets assigned first? Is a=17 or a=35? Who knows?
\end{minted}
    \item A similar issue arises when the implementor can choose between 2 different times to evaluate the addresses of the \nameref{def:Actual_Parameter}s. This is illustrated below.
\begin{minted}[frame=lines,linenos]{csharp}
void DoIt(out int x, int index) {
  x = 17;
  index = 42;
}
...
sub = 21;
f.doIt(out list[sub], out sub); // What gets assigned to list[sub], because sub is unknown: sub=21? sub=42?
\end{minted}
    \end{itemize}
  \end{itemize}
\end{definition}

\paragraph{Pass-by-Value-Result}\label{par:Parameter_Passing-Pass_By_Value_Result}
\begin{definition}[Pass-by-Value-Result]\label{def:Pass_By_Value_Result}
  \emph{Pass-by-value-result} is an implementation of the \nameref{def:Parameter_Passing-Inout_Mode} model in which actual values are copied.
  It is a combination of \nameref{def:Pass_By_Value} and \nameref{def:Pass_By_Result}.
  The \nameref{def:Variable_Value} of the \nameref{def:Actual_Parameter} is used to initialize the \nameref{def:Formal_Parameter}, which then acts as a \nameref{def:Local_Variable}.
  At subprogram termination, the value of the \nameref{def:Formal_Parameter}'s \nameref{def:Local_Variable} is copied back to the \nameref{def:Formal_Parameter}, then copied back to the \nameref{def:Actual_Parameter}.

  \begin{remark}[Pass-by-Copy]\label{rmk:Pass_By_Copy}
    \nameref{def:Pass_By_Value_Result} is sometimes called \emph{pass-by-copy}, because the \nameref{def:Actual_Parameter} is copied to the \nameref{def:Formal_Parameter} at the subprogram's start, and then copied back at the subprogram's termination.
  \end{remark}

  The advantages and disadvantages of pass-by-value-result are shared with both \nameref{def:Pass_By_Value} and \nameref{def:Pass_By_Result}.
\end{definition}

\paragraph{Pass-by-Reference}\label{par:Parameter_Passing-Pass_By_Reference}
\begin{definition}[Pass-by-Reference]\label{def:Pass_By_Reference}
  \emph{Pass-by-reference} is a second implementation model for \nameref{def:Parameter_Passing-Inout_Mode}.
  In this case, rather than copying the actual data \nameref{def:Variable_Value}s back and forth, pass-by-reference transmits an access path, usually an address/pointer/reference, to the called subprogram.
  This allows the subprogram to access the \textbf{SAME} value as the calling program.

  The advantages and disadvantages of this implementation are:
  \begin{itemize}[noitemsep]
  \item Advantages
    \begin{itemize}[noitemsep]
    \item The passing process is efficient, in terms of time and space.
      \begin{itemize}[noitemsep]
      \item No duplicate space is required
      \item There is also no copying required
      \end{itemize}
    \end{itemize}
  \item Disadvantages
    \begin{itemize}[noitemsep]
    \item Access to the \nameref{def:Formal_Parameter}s will be slower than \nameref{def:Pass_By_Value}, because of the indirect addressing required.
    \item There may be inadvertent and erroneous changes to the \nameref{def:Actual_Parameter}'s \nameref{def:Variable_Value}.
    \item \nameref{def:Aliasing} can occur.
      \begin{itemize}[noitemsep]
      \item This harms \nameref{subsec:Readability} and \nameref{subsec:Reliability}.
      \end{itemize}
\begin{minted}[frame=lines,linenos]{c++}
void func(int &first, int &second);
func(total, total); // When using first or second, they both point to the same ``total'' variable
fun(list[i], list[j]); // Assuming i==j is true, list[i] and list[j] both point to the same value
\end{minted}
    \item Program verification is more difficult.
    \end{itemize}
  \end{itemize}
\end{definition}

\paragraph{Pass-by-Name}\label{par:Parameter_Passing-Pass_By_Name}
\begin{definition}[Pass-by-Name]\label{def:Pass_By_Name}
  \emph{Pass-by-name} is an \nameref{def:Parameter_Passing-Inout_Mode} parameter transmission method that does not correspond to a single implementation model.
  When \nameref{def:Actual_Parameter} are passed by name, every occurrence of the actual parameter is textually substituted for the corresponding \nameref{def:Formal_Parameter}.
  A pass-by-name \nameref{def:Formal_Parameter} is bound to an access method at the time of the program call, but the actual binding to a \nameref{def:Variable_Value} or an \nameref{def:Variable_Address} is delayed until the \nameref{def:Formal_Parameter} is assigned or referenced.

  Implementing this requires a subprogram be passed to the called subprogram to evaluate the \nameref{def:Variable_Address} or \nameref{def:Variable_Value} of the \nameref{def:Formal_Parameter}.

  The advantages and disadvantages of this are:
  \begin{itemize}[noitemsep]
  \item Advantages
    \begin{itemize}[noitemsep]
    \item \nameref{def:Formal_Parameter}s are lazily evaluated, so they will only be evaluated once they are needed.
    \item However, the \nameref{def:Formal_Parameter} must be evaluated \textbf{EACH TIME}, making this a very inefficient method.
    \item Can add \nameref{subsubsec:Expressivity}.
    \end{itemize}
  \item Disadvantages
    \begin{itemize}[noitemsep]
    \item Pass-by-name parameters are difficult to implement
    \item Pass-by-name parameters are inefficient
    \item Add significant complexity to the program
      \begin{itemize}[noitemsep]
      \item Reduced \nameref{subsec:Readability}
      \item Reduced \nameref{subsec:Reliability}
      \end{itemize}
    \end{itemize}
  \end{itemize}

  \begin{remark}[Languages using \nameref*{def:Pass_By_Name}]\label{rmk:Langauges_Using_Pass_By_Name}
    There are no widely-used high-level languages that use \nameref{def:Pass_By_Name}.
    However, it is used at compile time by macros is assembly languages and for generic parameters of generic subprograms.
  \end{remark}
\end{definition}

\paragraph{Pass-by-Need}\label{par:Parameter_Passing-Pass_By_Need}
\begin{definition}[Pass-by-Need]\label{def:Pass_By_Need}
  This is a fairly niche way to pass \nameref{def:Actual_Parameter}s around to a subprogram's \nameref{def:Formal_Parameter}s.
  It is only used by the Haskell language.

  This is \textbf{VERY} similar to the \nameref{def:Pass_By_Name} implementation, but instead of evaluating the \nameref{def:Actual_Parameter} each time the \nameref{def:Formal_Parameter} appears, the \nameref{def:Actual_Parameter} is evaluated \emph{AT MOST ONCE}.
  Then, every subsequent occurrence of the \nameref{def:Actual_Parameter} has it value substituted by \nameref{def:Referential_Transparency}, or if its a function call, the result of the \nameref{def:Function_Pure} function.

  The advantages and disadvantages of pass-by-need are:
  \begin{itemize}[noitemsep]
  \item Advantages
    \begin{itemize}[noitemsep]
    \item More time-efficient than \nameref{def:Pass_By_Name}
    \end{itemize}
  \item Disadvantages
    \begin{itemize}[noitemsep]
    \item Less flexity in the presence of updating \nameref{def:Variable}s.
      \begin{itemize}[noitemsep]
      \item Haskell avoids this problem because its ``variables'' are immutable, which makes this a moot point.
      \end{itemize}
    \end{itemize}
  \end{itemize}
\end{definition}

\subsection{Parameters That Are Subprograms}\label{subsec:Parameters_are_Subprograms}
Being able to use \nameref{sec:Subprograms} as \nameref{def:Actual_Parameter}s in a program can help simplify the programming of certain problems.
There are 2 major complications of this desire:
\begin{enumerate}[noitemsep]
\item The \nameref{def:Type_Checking} of the parameters of the activations of the subprogram that was passed as a parameter.
\item In languages that allow \nameref{subsubsec:Nested_Subprograms}, there is a question of what \nameref{def:Referencing_Environment} should be used. There are 3 possible solutions:
  \begin{enumerate}[noitemsep]
  \item \emph{\nameref{def:Shallow_Binding}}
  \item \emph{\nameref{def:Deep_Binding}}
  \item \emph{\nameref{def:Ad_Hoc_Binding}}
  \end{enumerate}
\end{enumerate}

\begin{definition}[Shallow Binding]\label{def:Shallow_Binding}
  In \emph{shallow binding}, the \nameref{def:Referencing_Environment} of the call statement that enacts the subprogram is used.

  The \nameref{def:Referencing_Environment} where the call statement to the passed subprogram occurs is \nameref{def:Referencing_Environment} when the subprogram is executed.

  \begin{remark}[Shallow Binding Program Output]\label{rmk:Shallow_Binding-Program_Output}
    In the program below, the output of \texttt{sub2} is \texttt{4}, because shallow binding binds the \nameref{def:Referencing_Environment} of \texttt{sub2} to \texttt{sub4}'s \nameref{def:Referencing_Environment}, where \texttt{sub2} was executed.
  \end{remark}
\end{definition}

\begin{definition}[Deep Binding]\label{def:Deep_Binding}
  In \emph{deep binding}, the \nameref{def:Referencing_Environment} of the definition of the passed subprogram is used.

  The \nameref{def:Referencing_Environment} of the definition of the subprogram is used as the \nameref{def:Referencing_Environment}.

  \begin{remark}[Deep Binding Program Output]\label{rmk:Deep_Binding-Program_Output}
    In the program below, the output of \texttt{sub2} is \texttt{1}, because deep binding binds the \nameref{def:Referencing_Environment} of \texttt{sub2} to \texttt{sub1}'s \nameref{def:Referencing_Environment}, where \texttt{sub2} was defined.
  \end{remark}
\end{definition}

\begin{definition}[Ad-Hoc Binding]\label{def:Ad_Hoc_Binding}
  In \emph{ad-hod binding}, the \nameref{def:Referencing_Environment} of the call statement that \emph{passed} the program as an \nameref{def:Actual_Parameter} is used.

  The \nameref{def:Referencing_Environment} where the call statement passes the subprogram is used as the \nameref{def:Referencing_Environment} for the subprogram.

  \begin{remark}[Ad-Hoc Binding Program Output]\label{rmk:Ad_Hoc_Binding-Program_Output}
    In the program below, the output of \texttt{sub2} is \texttt{3}, because ad-hoc binding binds the \nameref{def:Referencing_Environment} of \texttt{sub2} to \texttt{sub3}'s \nameref{def:Referencing_Environment}, where \texttt{sub2} was originally passed as a subprogram.
  \end{remark}
\end{definition}

This code block is written with JavaScript's syntax, but depending on the binding used, the output of \texttt{sub2} will be different.
\begin{minted}[frame=lines,linenos]{javascript}
function sub1() {
  var x;
  function sub2() {
    alert(x); // Creates a box with the value of x
  };
  function sub3() {
    var x;
    x = 3;
    sub4(sub2);
  };
  function sub4(subx) {
    var x;
    x = 4;
    subx();
  };
  x = 1;
  sub3();
};
\end{minted}

\subsection{Closures}\label{subsec:Closures}
\begin{definition}[Closure]\label{def:Closure}
  A \emph{closure} is a subprogram and the \nameref{def:Referencing_Environment} where it was defined.
  The \nameref{def:Referencing_Environment} is needed if the subprogram can be called from any arbitrary place in the program.

  \begin{remark}[\nameref*{subsubsec:Static_Scope}d, No Nested Subprograms]
    If a statically-scoped programming language does not support nested subprograms, then it usually doesn't support \nameref{def:Closure}s either.
    All of the \nameref{def:Variable}s in the \nameref{def:Referencing_Environment} of a subprogram are accessible, regardless of the place in the program where the subprogram is called.
  \end{remark}
\end{definition}

When subprograms can be nested, the subprogram can use its \nameref{def:Local_Variable}s, the \nameref{def:Global_Variable}s, and any \nameref{def:Variable}s declared in parent subprograms.
This isn't an issue when the nested subprogram can only be called where the enclosing scopes are active and \nameref{def:Visible_Variable}.

However, if the nested subprogram can be called from elsewhere.
This can happen if the subprogram is passed as a parameter or assigned to a variable.
This also means that the subprogram could be called after its parent subprograms have terminated, meaning some of the \nameref{def:Variable}s defined there are no longer available.
To prevent this, we have to have special \nameref{def:Variable}s which are said to have \emph{\nameref{def:Unlimited_Extent}}.

\begin{definition}[Unlimited Extent]\label{def:Unlimited_Extent}
  A \nameref{def:Variable} whose lifetime is that of the whole program, and are required by nested subprograms said to have \emph{unlimited extent}.

  These variables are usually heap-dynamic, rather than \nameref{def:Stack-Dynamic_Variable_Binding_Lifetime}s.
\end{definition}

%%% Local Variables:
%%% mode: latex
%%% TeX-master: "../EDAP05-Concepts_Programming_Languages-Reference_Sheet"
%%% End:


\section{Implementing \nameref*{sec:Subprograms}}\label{sec:Implementing_Subprograms}
\subsection{General Semantics of Calls and Returns}\label{subsec:Semantics_of_Calls_and_Returns}
\begin{definition}[Subprogram Linkage]\label{def:Subprogram_Linkage}
  The \texttt{call} and \texttt{return} operations are together called \emph{subprogram linkage}.
\end{definition}

There are several steps that must occur for a subprogram's \texttt{call} to work.
\begin{enumerate}[noitemsep]
\item Include the implementation of that language's parameter-passing method.
\item If local variables are not static, there must be space allocation for the \nameref{def:Local_Variable}s declared in the subprogram, and bind them to storage.
\item Must save the execution status of the CPU, everything required to jump back to the point where the \texttt{all} occurs. This includes:
  \begin{itemize}[noitemsep]
  \item CPU status bits
  \item The Environment Pointer (\nameref{def:Dynamic_Link})
  \item Register values
  \end{itemize}
\item Arrange the transfer of control to the code of the subprogram, and ensure control can be returned to the proper place when execution is done.
\item If the language supports nested subprograms, there needs to be a mechanism to provide access to the parent subprogram's variables. (\nameref{def:Static_Link})
\end{enumerate}

The required actions for a subprogram to return its execution to the parent subprogram are:
\begin{enumerate}[noitemsep]
\item If the parameters are \nameref{def:Parameter_Passing-Out_Mode} or \nameref{def:Parameter_Passing-Inout_Mode}, the local values of the associated \nameref{def:Formal_Parameter}s must be copied to the \nameref{def:Actual_Parameter}s.
\item All storage used for \nameref{def:Local_Variable}s must be deallocated.
\item Restore the execution status of the calling parent subprogram.
\item Return control to the calling parent subprogram.
\end{enumerate}

\subsection{Implementing ``Simple'' Subprograms}\label{subsec:Implementing_Simple_Subprograms}
In this case, ``simple'' means subprograms that cannot be nested and all local variables are static.

There must be storage for:
\begin{itemize}[noitemsep]
\item Status information for the caller program.
\item Parameters
\item The return \nameref{def:Memory} address
\item Return values for functions
\item Temporaries used by the code of the subprogram(s)
\end{itemize}

Some of the semantic actions in the next 2 sections (\Crefrange{subsubsec:Implementing_Simple_Subprogram-Call}{subsubsec:Implementing_Simple_Subprogram-Return}) can be occur at 2 different times during \nameref{def:Subprogram_Linkage}.
These are called the \emph{prologue} and \emph{epilogue} of the \nameref{def:Subprogram_Linkage}.

\subsubsection{Semantics of the Subprogram \texttt{Call}}\label{subsubsec:Implementing_Simple_Subprogram-Call}
The following steps must be followed:
\begin{enumerate}[noitemsep]
\item Save the execution status of the current program unit.
\item Compute and pass the parameters.
\item Pass the return address to be called at the end of subprogram execution.
\item Transfer control to the called subprogram.
\end{enumerate}

The last 3 actions must be done by the caller program.
The first action could be done by the caller program or the called subprogram.

\subsubsection{Semantics of the Subprogram \texttt{Return}}\label{subsubsec:Implementing_Simple_Subprogram-Return}
These are the steps that must be followed to \texttt{return} from a subprogram:
\begin{enumerate}[noitemsep]
\item If parameters are \nameref{def:Pass_By_Value_Result} or \nameref{def:Parameter_Passing-Inout_Mode}, the current values of those \nameref{def:Formal_Parameter}s are moved/made available to the corresponding \nameref{def:Actual_Parameter}s.
\item If the subprogram is a function, the functional value is moved to a place accessible to the caller.
\item The execution status of the caller program is restored.
\item Control is transferred back to the caller program.
\end{enumerate}

The first, third, and fourth steps could be handled by the called subprogram.

\subsubsection{Activation Records for ``Simple'' Subprograms}\label{subsubsec:Implementing_Simple_Subprogram-Activation_Record}
\begin{definition}[Activation Record]\label{def:Activation_Record-Simple_Subprograms}
  An \emph{activation record} for a ``simple'' subprogram is the noncode portion, because the data it describes is only relevant during the activation/execution of the subprogram.
  A concrete example of an activation record is called an \emph{activation record instance}.

  In this ``simple'' subprogram setup, activation records are of fixed size.
  There can also only be one active version of a given subprogram at a time (since recursion isn't supported with just \nameref{def:Static_Variable_Binding_Lifetime}s).
\end{definition}

%%% Local Variables:
%%% mode: latex
%%% TeX-master: "../EDAP05-Concepts_Programming_Languages-Reference_Sheet"
%%% End:


\section{Abstract Data Types and Encapsulation Constructs}\label{sec:Abstract_Data_Types_Encapsulation_Constructs}
\subsection{The Concept of Abstraction}\label{subsec:Concept_Abstraction}
\begin{definition}[Abstraction]\label{def:Abstraction}
  An \emph{abstraction} is a view or representation of an entity that includes only the most significant attributes.
  In a general sense, abstraction allows one to collect instances of entities into groups in which their common attributes need not be considered, and their unique attributes separate entities which may be from the same group.

  There is:
  \begin{itemize}[noitemsep]
  \item \nameref{par:Process_Abstraction}, which is discussed elsewhere, with subprograms.
  \item \nameref{par:Data_Abstraction}.
  \end{itemize}
\end{definition}

\subsection{Introduction to Data Abstraction}\label{subsec:Intro_Data_Abstraction}
An \emph{\nameref{def:Abstract_Data_Type}} is a data structure, in the form of a record, but includes subprograms that manipulate its data.
It is an enclosure that only includes the data representations of one specific \nameref{def:Data_Type}, and the subprograms provide operations for that type.
This allows unnecessary details of the type to be hidden from units outside the enclosure.

\begin{definition}[Object]\label{def:Object}
  An instance of an \nameref{def:Abstract_Data_Type} is called an \emph{object}.
\end{definition}

\nameref{def:Abstract_Data_Type}s are used to combat program complexity by grouping things together similarly to how we would group them in the real-world as humans.

\subsubsection{Floating-Point as an Abstract Data Type}\label{subsubsec:Floating_Point_Abstract_Data_Type}
Technically, all \nameref{def:Data_Type}s are \nameref{def:Abstract_Data_Type}s.
These are implementations of information hiding, as the programmer does not usually (C/C++ are semi-counter-examples) have direct access to manipulate the bits that make up the number.
They only have the operations presented to them by the language designer and implementer.

Overall, this improved program \nameref{subsec:Reliability} and portability.

\subsubsection{User-Defined Abstract Data Types}\label{subsubsec:User_Defined_Abstract_Data_Types}
\begin{definition}[Abstract Data Type]\label{def:Abstract_Data_Type}
  An \emph{abstract data type} has 2 components:
  \begin{enumerate}[noitemsep]
  \item The enclosure that ``displays'' what operations are possible on this \nameref{def:Data_Type} is called the \nameref{def:ADT_Interface}.
  \item The code that implements the functionality specified by the interface is called the \nameref{def:ADT_Implementation}.
  \end{enumerate}

  An abstract data type is a \nameref{def:Data_Type} that satisfies the following conditions:
  \begin{itemize}[noitemsep]
  \item The representation of objects of the type is hidden from the program units that use that type, so they only direct operations possible on those objects are those provided in the abstract data type's definition. This improves:
    \begin{itemize}[noitemsep]
    \item Increases \nameref{subsec:Reliability}
    \item Clients (Units using an abstract data type) cannot manipulate the underlying representation of objects directly
    \item Objects can be changed only through the provided operations
    \item Reduces the range of code and the number of \nameref{def:Variable}s the programmer must be aware of when reading/writing a program
    \item Reduces the likelihood of naming conflicts
    \end{itemize}
  \item The declarations of the type and the protocols of the operations on objects of the abstract data type, which provide the type's interface, are contained in a single syntactic unit. This benefits the language by:
    \begin{itemize}[noitemsep]
    \item Organizing the program into logical units that can be compiled separately.
    \end{itemize}
  \item The type's interface does not depend on the underlying representation of the objects, or the implementation of the operations.
    \begin{itemize}[noitemsep]
    \item For example, if a stack is implemented with a linked list, then needs to be changed to an array-like structure, the underlying representation can be changed without affecting any clients that are using the subprograms and \nameref{def:Variable}s.
    \item Accessing and modifying data in an abstract data type is done with \emph{getters} and \emph{setters} that allow clients indirect access to the hidden data. There are 3 reasons why this is an improvement:
      \begin{enumerate}[noitemsep]
      \item Read-only access can be provided, by having a getter method, but no corresponding setter method.
      \item Constraints can be included in setters. The setter can enforce the range that a data value can take.
      \item The actual implementation of the data memvers can be changed without affecting the clients, if getters and setters are the only access.
      \end{enumerate}
    \end{itemize}
  \item Also, other program units are allowed to create \nameref{def:Variable}s of the defined abstract data type.
  \end{itemize}
\end{definition}

\begin{definition}[Interface]\label{def:ADT_Interface}
  An \emph{interface} is the programmer-usable ``contract'' that can be used for an \nameref{def:Abstract_Data_Type}.
  It ensures that all programmers who use this \nameref{def:Abstract_Data_Type} have a common set of operations that behave in a defined manner.

  An example of an interface is C and C++'s header files (\texttt{*.h} for C, and \texttt{*.hpp} for C++).

  \begin{remark}
    The \nameref{def:ADT_Interface}, usually, does not contain any code.
    The code that implements the \nameref{def:ADT_Interface} is in the \nameref{def:ADT_Implementation} file.
    However, in Java, an \nameref{def:Abstract_Data_Type} requires that the method have a \nameref{def:Subprogram_Definition} at the same time as its \nameref{def:Subprogram_Declaration}.
  \end{remark}

  \begin{remark}[Reliance on Specification]\label{rmk:ADT_Interface_Specification}
    An \nameref{def:Abstract_Data_Type}'s \nameref{def:ADT_Interface} must have a specification to ensure that the \nameref{def:Abstract_Data_Type} has the expected operations.
    This is further discussed in \Cref{subsubsec:Specification_Abstract_Datatype}.
  \end{remark}
\end{definition}

\begin{definition}[Implementation]\label{def:ADT_Implementation}
  An \emph{implementation} is the language/project designer's implementation of the ``contract'' specified by the \nameref{def:ADT_Interface}.
  For any given problem, there may be a single \nameref{def:ADT_Interface}, but there may be many different possible implementations.

  For example, an integer vector, \texttt{IntVector} that is created with default values in every element would have a single \nameref{def:ADT_Interface}, but we could \emph{implement} the \texttt{IntVector} with several different data structures.
  We could use:
  \begin{itemize}[noitemsep]
  \item An array, for quick random accesses
  \item A linked list, for efficient memory usage
  \item A binary tree for relatively efficient lookups and appending of values, ith efficient memory usage.
  \end{itemize}

  \begin{remark}[Reliance on Specification]\label{rmk:ADT_Implementation_Specification}
    An \nameref{def:Abstract_Data_Type}'s \nameref{def:ADT_Implementation} must have a specification to ensure that the \nameref{def:Abstract_Data_Type} operates as expected.
    This is further discussed in \Cref{subsubsec:Specification_Abstract_Datatype}.
  \end{remark}
\end{definition}

\subsection{Design Issues for Abstract Data Types}\label{subsec:Abstract_Data_Type_Design_Issues}
\begin{itemize}[noitemsep]
\item The \nameref{def:Abstract_Data_Type} name must be externally visible, to allow for object creation
\item The \nameref{def:Abstract_Data_Type} representation must be hidden.
\item There are few built-in operations by a language for \nameref{def:Abstract_Data_Type} operations
  \begin{itemize}[noitemsep]
  \item If there are some defined, they are the most basic ones: assignment, comparison.
  \item Overloading of subprograms should be allowed
  \end{itemize}
\item The form of the container for the interface to the \nameref{def:Abstract_Data_Type}.
\item Whether \nameref{def:Abstract_Data_Type} can be parameterized.
\item What access controls are provided, and how are such controls specified?
\item Is the specification of the \nameref{def:Abstract_Data_Type} physically separate from its implementation?
\end{itemize}

\subsubsection{Specification of Abstract Datatypes}\label{subsubsec:Specification_Abstract_Datatype}
\begin{definition}[Specification]\label{def:ADT_Specification}
  We need a \emph{specification} of the \nameref{def:Abstract_Data_Type} to ensure that the it is implemented with the correct functionality and that it operates as expected.
  We also need to specify the actions that are required for the \nameref{def:Abstract_Data_Type} in the \nameref{def:ADT_Interface}.
\end{definition}

For example, if we have an \nameref{def:ADT_Interface} of an \texttt{IntVector}, that supports the following operations:
\begin{itemize}[noitemsep]
\item create : \texttt{\SemanticType{int}} $\rightarrow$ \texttt{\SemanticType{IntVector}}
\item length : \texttt{\SemanticType{IntVector}} $\rightarrow$ \texttt{\SemanticType{int}}
\item append : \texttt{(\SemanticType{IntVector}, \SemanticType{int})} $\rightarrow$ \texttt{()}
\item get : \texttt{(\SemanticType{IntVector}, \SemanticType{int})} $\rightarrow$ \texttt{\SemanticType{int}}
\end{itemize}

In a code example, we can write our \nameref{def:ADT_Interface} as a Rust Trait.
\inputminted[frame=lines,linenos]{rust}{./EDAP05-Concepts_Programming_Languages-Sections/Abstract_Data_Types/Code/IntVector_Interface.rs}









\subsection{Variance of Types}\label{subsec:Type_Variance}
\begin{definition}[Covariance]\label{def:Type_Covariance}
  Let $\DataType$ be a type constructor with formal type parameters $\DataType_{1}, \ldots , \DataType_{k}$, such that $\SemanticType{T} = \DataType [\DataType_{1}, \ldots, \DataType_{k}]$ is a type.
  Let $i \in \lbrace 1,\ldots, k \rbrace$.

  If for all $\DataType_{i}' <: \DataType_{i}$ we can always substitute a value of type $\DataType [\DataType_{1}',\ldots, \DataType_{i}', \ldots, \DataType_{k}']$ in a context that expects a value of type $\DataType [\DataType_{1}, \ldots, \DataType_{i}, \ldots, \DataType_{k}]$ without violating type preservation then $\DataType_{i}$ is \emph{covariant} in $\SemanticType{T}$.
\end{definition}

\begin{definition}[Contravariance]\label{Type_Contravariance}
  Let $\DataType$ be a type constructor with formal type parameters $\DataType_{1}, \ldots , \DataType_{k}$, such that $\SemanticType{T} = \DataType [\DataType_{1}, \ldots, \DataType_{k}]$ is a type.
  Let $i \in \lbrace 1,\ldots, k \rbrace$.

  If for all $\DataType_{i}' :> \DataType_{i}$ we can always substitute a value of type $\DataType [\DataType_{1}',\ldots, \DataType_{i}', \ldots, \DataType_{k}']$ in a context that expects a value of type $\DataType [\DataType_{1}, \ldots, \DataType_{i}, \ldots, \DataType_{k}]$ without violating type preservation then $\DataType_{i}$ is \emph{contravariant} in $\SemanticType{T}$.
\end{definition}

\begin{definition}[Invariance]\label{def:Type_Invariance}
  Let $\DataType$ be a type constructor with formal type parameters $\DataType_{1}, \ldots , \DataType_{k}$, such that $\SemanticType{T} = \DataType [\DataType_{1}, \ldots, \DataType_{k}]$ is a type.
  Let $i \in \lbrace 1,\ldots, k \rbrace$.


  If $\DataType_{i}$ is neither covariant nor contravariant in $\SemanticType{T}$, then $\DataType_{i}$ is \emph{invariant} in $\SemanticType{T}$.
\end{definition}

\begin{definition}[Bivariance]\label{def:Type_Bivariance}
  Let $\DataType$ be a type constructor with formal type parameters $\DataType_{1}, \ldots , \DataType_{k}$, such that $\SemanticType{T} = \DataType [\DataType_{1}, \ldots, \DataType_{k}]$ is a type.
  Let $i \in \lbrace 1,\ldots, k \rbrace$.

  If for all $\DataType_{i}' <: \DataType_{i}$ we can always substitute a value of type $\DataType [\DataType_{1}',\ldots, \DataType_{i}', \ldots, \DataType_{k}']$ in a context that expects a value of type $\DataType [\DataType_{1}, \ldots, \DataType_{i}, \ldots, \DataType_{k}]$ without violating type preservation \textbf{AND}, if for all $\DataType_{i}' :> \DataType_{i}$ we can always substitute a value of type $\DataType' [\DataType_{1}', \ldots, \DataType_{i}', \ldots, \DataType_{k}']$ in a context that expects a value of type $\DataType [\DataType_{1}, \ldots, \DataType_{i}, \ldots, \DataType_{k}]$ without violating type preservation then $\DataType_{i}$ is \emph{bivariant} in $\SemanticType{T}$.

  \begin{remark}
    If both the input and output allow for the type to be both broadened and narrowed, it is not terribly interesting to study.
    Thus, we will not be studying them in much detail in this class.
  \end{remark}
\end{definition}

%%% Local Variables:
%%% mode: latex
%%% TeX-master: "../../EDAP05-Concepts_Programming_Languages-Reference_Sheet"
%%% End:


%%% Local Variables:
%%% mode: latex
%%% TeX-master: "../EDAP05-Concepts_Programming_Languages-Reference_Sheet"
%%% End:

%====================================APPENDIX====================================
\appendix
\counterwithin{definition}{subsection}

\clearpage
\section{Computer Components}\label{app:Computer_Components}
\subsection{Central Processing Unit}\label{subsec:CPU}
\begin{definition}[Central Processing Unit]\label{def:CPU}
  The \emph{Central Processing Unit}, \emph{CPU}, is a chip that performs all actions in the computer.
  It calculates mathematical and logical values and acts based on them.
  It has several components built onto it, and can be thought of as the ``brain'' of the computer.

  The design of a CPU determines some of the functionality it has.
  Therefore, more specialized processors can be made for special tasks, and more general processors can be built to handle a wide variety of calculations.
\end{definition}

\subsubsection{Registers}\label{subsubsec:Registers}
\begin{definition}[Register]\label{def:Register}
  A \emph{register} is a data storage mechanism built directly onto the \nameref{def:CPU}.
  It is several hundred times faster than the system \nameref{def:Memory}.
  Registers are generally used when the currently running program is performing calculations.
  Since they are so fast, they are used as both source and destination operands in instructions.

  \begin{remark}
    Depending on the \nameref{def:CPU} architecture, there may be cases when \nameref{def:Register}s behave slightly differently between processors.
    This is something that can only be found by checking the \nameref{def:CPU} manufacturer's documentation.
  \end{remark}
\end{definition}

\subsubsection{Program Counter}\label{subsubsec:Program_Counter}
\subsubsection{Arithmetic Logic Unit}\label{subsubsec:ALU}
\subsubsection{Cache}\label{subsubsec:CPU_Cache}

\subsection{Memory}\label{subsec:Memory}
\begin{definition}[Memory]\label{def:Memory}
  \emph{Memory}, or \emph{RAM} (\emph{Random Access Memory}), is a \nameref{def:Volatile} data storage mechanism.
  It is directly connected to the \nameref{def:CPU}.
  This is the location that the \nameref{def:CPU} writes to when it cannot or should not store something in the \nameref{def:CPU}'s \nameref{def:Register}s.

  \begin{remark}[Volatility]
    \nameref{def:Memory} is volatile because each of the cells is a small capacitor.
    In between the clock cycles on the \nameref{def:CPU} and \nameref{def:Memory}, the capacitors discharge.
    On the clock cycle, the capacitors are refreshed with electrical power, which does one of 2 things:
    \begin{enumerate}[noitemsep]
    \item Keep the data bits the same, 1 to 1.
    \item Update the data bits from 0 to 1.
    \end{enumerate}
  \end{remark}
\end{definition}

\begin{definition}[Volatile]\label{def:Volatile}
  If a data storage mechanism is called \emph{volatile}, it means that once the storage mechanism loses power, the data is lost.
  This is in contrast to \nameref{def:Non-Volatile} data storage mechanisms.
\end{definition}

\subsection{Disk}\label{subsec:Disk}
\begin{definition}[Non-Volatile]\label{def:Non-Volatile}
  If a data storage mechanism is called \emph{non-volatile}, it means that once the storage device loses power, the data is still safely stored.
  This is in contrast to \nameref{def:Volatile} data storage mechanisms.
\end{definition}

\subsection{Fetch-Execute Cycle}\label{subsec:Fetch_Execute_Cycle}
%%% Local Variables:
%%% mode: latex
%%% TeX-master: shared
%%% End:


\clearpage
\section{History of Programming Languages}\label{sec:Programming_Language_History}
\subsection{Zuse's Plankalk\"{u}l}\label{subsec:Zuses_Plankalkul}
\subsection{Pseudocodes}\label{subsec:Pseudocodes}
\subsection{Fortran}\label{subsec:Fortran}
\subsection{Functional Programming: LISP}\label{subsec:Functional_Programming-LISP}
\subsection{ALGOL 60}\label{subsec:ALGOL_60}
\subsection{COBOL}\label{subsec:COBOL}
\subsection{Timesharing: BASIC}\label{subsec:BASIC}
\subsection{PL/I}\label{subsec:PLI}
\subsection{Early Dynamic Languages: APL and SNOBOL}\label{subsec:Dynamic_Languages-APL_and_SNOBOL}
\subsection{Data Abstraction: SIMULA 67}\label{subsec:SIMULA_67}
\subsection{Orthogonality: ALGOL 68}\label{subsec:ALGOL_68}
\subsection{ALGOL Descendants}\label{subsec:ALGOL_Descendants}
\subsection{Logical Programming: Prolog}\label{subsec:Logical_Programming-Prolog}
\subsection{Ada}\label{subsec:Ada}
\subsection{Object-Oriented Programming: Smalltalk}\label{subsec:OOP-Smalltalk}
\subsection{Combine Imperative and OOP Features: C++}\label{subsec:Imperative_OOP-CPP}
\subsection{Java}\label{subsec:Java}
\subsection{Scripting Languages}\label{subsec:Scripting_Languages}
\subsection{Flagship .NET Language: C\#}\label{subsec:C_Sharp}
\subsection{Markup/Programming Hybrid Languages}\label{subsec:Markup_Programming_Hybrid_Languages}

%%% Local Variables:
%%% mode: latex
%%% TeX-master: "../EDAP05-Concepts_Programming_Languages-Reference_Sheet"
%%% End:


\clearpage
\subsection{Trigonometry} \label{app:Trig}
	\subsubsection{Trigonometric Formulas} \label{subsubsec:Trig Formulas}
		\begin{equation} \label{eq:Sin plus Sin with diff Angles}
			\sin \left( \alpha \right) + \sin \left( \beta \right) = 2 \sin \left( \frac{\alpha + \beta}{2} \right) \cos\left( \frac{\alpha - \beta}{2} \right)  
		\end{equation}
		\begin{equation} \label{eq:Cosine-Sine Product}
			\cos \left( \theta \right) \sin \left( \theta \right) = \frac{1}{2} \sin \left( 2 \theta \right)
		\end{equation}
	
	\subsubsection{Euler Equivalents of Trigonometric Functions} \label{subsubsec:Euler Equivalents}
		\begin{equation} \label{eq:Euler Sin}
			\sin \left( x \right) = \frac{e^{\imath x} + e^{-\imath x}}{2}
		\end{equation}
		\begin{equation} \label{eq:Euler Cos}
			\cos \left( x \right) = \frac{e^{\imath x} - e^{-\imath x}}{2 \imath}
		\end{equation}
		\begin{equation} \label{eq:Euler Sinh}
			\sinh \left( x \right) = \frac{e^{x} - e^{-x}}{2}
		\end{equation}
		\begin{equation} \label{eq:Euler Cosh}
			\cosh \left( x \right) = \frac{e^{x} + e^{-x}}{2}
		\end{equation}

\clearpage
\section{Calculus}\label{app:Calculus}
\subsection{L'Hopital's Rule}\label{subsec:LHopitals_Rule}
L'Hopital's Rule can be used to simplify and solve expressions regarding limits that yield irreconcialable results.
\begin{lemma}[L'Hopital's Rule]\label{lemma:LHopitals_Rule}
  If the equation
  \begin{equation*}
    \lim\limits_{x \rightarrow a} \frac{f(x)}{g(x)} =
    \begin{cases}
      \frac{0}{0} \\
      \frac{\infty}{\infty} \\
    \end{cases}
  \end{equation*}
  then \Cref{eq:LHopitals_Rule} holds.
  \begin{equation}\label{eq:LHopitals_Rule}
    \lim\limits_{x \rightarrow a} \frac{f(x)}{g(x)} = \lim\limits_{x \rightarrow a} \frac{f'(x)}{g'(x)}
  \end{equation}
\end{lemma}

\subsection{Fundamental Theorems of Calculus}\label{subsec:Fundamental Theorem of Calculus}
\begin{definition}[First Fundamental Theorem of Calculus]\label{def:1st Fundamental Theorem of Calculus}
  The \emph{first fundamental theorem of calculus} states that, if $f$ is continuous on the closed interval $\left[ a,b \right]$ and $F$ is the indefinite integral of $f$ on $\left[ a,b \right]$, then

  \begin{equation}\label{eq:1st Fundamental Theorem of Calculus}
    \int_{a}^{b}f \left( x \right) dx = F \left( b \right) - F \left( a \right)
  \end{equation}
\end{definition}

\begin{definition}[Second Fundamental Theorem of Calculus]\label{def:2nd Fundamental Theorem of Calculus}
  The \emph{second fundamental theorem of calculus} holds for $f$ a continuous function on an open interval $I$ and $a$ any point in $I$, and states that if $F$ is defined by

  \begin{equation*}
    F \left( x \right) = \int_{a}^{x} f \left( t \right) dt,
  \end{equation*}
  then
  \begin{equation}\label{eq:2nd Fundamental Theorem of Calculus}
    \begin{aligned}
      \frac{d}{dx} \int_{a}^{x} f \left( t \right) dt &= f \left( x \right) \\
      F' \left( x \right) &= f \left( x \right) \\
    \end{aligned}
  \end{equation}
\end{definition}

\begin{definition}[argmax]\label{def:argmax}
  The arguments to the \emph{argmax} function are to be maximized by using their derivatives.
  You must take the derivative of the function, find critical points, then determine if that critical point is a global maxima.
  This is denoted as
  \begin{equation*}\label{eq:argmax}
    \argmax_{x}
  \end{equation*}
\end{definition}

\subsection{Rules of Calculus}\label{subsec:Rules of Calculus}
\subsubsection{Chain Rule}\label{subsubsec:Chain Rule}
\begin{definition}[Chain Rule]\label{def:Chain Rule}
  The \emph{chain rule} is a way to differentiate a function that has 2 functions multiplied together.

  If
  \begin{equation*}
    f(x) = g(x) \cdot h(x)
  \end{equation*}
  then,
  \begin{equation}\label{eq:Chain Rule}
    \begin{aligned}
      f'(x) &= g'(x) \cdot h(x) + g(x) \cdot h'(x) \\
      \frac{df(x)}{dx} &= \frac{dg(x)}{dx} \cdot g(x) + g(x) \cdot \frac{dh(x)}{dx} \\
    \end{aligned}
  \end{equation}
\end{definition}

\subsection{Useful Integrals}\label{subsec:Useful_Integrals}
\begin{equation}\label{eq:Cosine_Indefinite_Integral}
  \int \cos(x) \; dx = \sin(x)
\end{equation}

\begin{equation}\label{eq:Sine_Indefinite_Integral}
  \int \sin(x) \; dx = -\cos(x)
\end{equation}

\begin{equation}\label{eq:x_Cosine_Indefinite_Integral}
  \int x \cos(x) \; dx = \cos(x) + x \sin(x)
\end{equation}
\Cref{eq:x_Cosine_Indefinite_Integral} simplified with Integration by Parts.

\begin{equation}\label{eq:x_Sine_Indefinite_Integral}
  \int x \sin(x) \; dx = \sin(x) - x \cos(x)
\end{equation}
\Cref{eq:x_Sine_Indefinite_Integral} simplified with Integration by Parts.

\begin{equation}\label{eq:x_Squared_Cosine_Indefinite_Integral}
  \int x^{2} \cos(x) \; dx = 2x \cos(x) + (x^{2} - 2) \sin(x)
\end{equation}
\Cref{eq:x_Squared_Cosine_Indefinite_Integral} simplified by using Integration by Parts twice.

\begin{equation}\label{eq:x_Squared_Sine_Indefinite_Integral}
  \int x^{2} \sin(x) \; dx = 2x \sin(x) - (x^{2} - 2) \cos(x)
\end{equation}
\Cref{eq:x_Squared_Sine_Indefinite_Integral} simplified by using Integration by Parts twice.

\begin{equation}\label{eq:Exponential_Cosine_Indefinite_Integral}
  \int e^{\alpha x} \cos(\beta x) \; dx = \frac{e^{\alpha x} \bigl( \alpha \cos(\beta x) + \beta \sin(\beta x) \bigr)}{\alpha^{2} + \beta^{2}} + C
\end{equation}

\begin{equation}\label{eq:Exponential_Sine_Indefinite_Integral}
  \int e^{\alpha x} \sin(\beta x) \; dx = \frac{e^{\alpha x} \bigl( \alpha \sin(\beta x) - \beta \cos(\beta x) \bigr)}{\alpha^{2}+\beta^{2}} + C
\end{equation}

\begin{equation}\label{eq:Exponential_Indefinite_Integral}
  \int e^{\alpha x} \; dx = \frac{e^{\alpha x}}{\alpha}
\end{equation}

\begin{equation}\label{eq:x_Exponential_Indefinite_Integral}
  \int x e^{\alpha x} \; dx = e^{\alpha x} \left( \frac{x}{\alpha} - \frac{1}{\alpha^{2}} \right)
\end{equation}
\Cref{eq:x_Exponential_Indefinite_Integral} simplified with Integration by Parts.

\begin{equation}\label{eq:Inverse_x_Indefinite_Integral}
  \int \frac{dx}{\alpha + \beta x} = \int \frac{1}{\alpha + \beta x} \; dx = \frac{1}{\beta} \ln (\alpha + \beta x)
\end{equation}

\begin{equation}\label{eq:Inverse_x_Squared_Indefinite_Integral}
  \int \frac{dx}{\alpha^{2} + \beta^{2} x^{2}} = \int \frac{1}{\alpha^{2} + \beta^{2} x^{2}} \; dx = \frac{1}{\alpha \beta} \arctan \left( \frac{\beta x}{\alpha} \right)
\end{equation}

\begin{equation}\label{eq:a_Exponential_Indefinite_Integral}
  \int \alpha^{x} \; dx = \frac{\alpha^{x}}{\ln(\alpha)}
\end{equation}

\begin{equation}\label{eq:a_Exponential_Derivative}
  \frac{d}{dx} \alpha^{x} = \frac{d\alpha^{x}}{dx} = \alpha^{x} \ln(x)
\end{equation}

\subsection{Leibnitz's Rule}\label{subsec:Leibnitzs_Rule}
\begin{lemma}[Leibnitz's Rule]\label{lemma:Leibnitzs_Rule}
  Given
  \begin{equation*}
    g(t) = \int_{a(t)}^{b(t)} f(x, t) \, dx
  \end{equation*}
  with $a(t)$ and $b(t)$ differentiable in $t$ and $\frac{\partial f(x, t)}{\partial t}$ continuous in both $t$ and $x$, then
  \begin{equation}\label{eq:Leibnitzs_Rule}
    \frac{d}{dt} g(t) = \frac{d g(t)}{dt} = \int_{a(t)}^{b(t)} \frac{\partial f(x, t)}{\partial t} \, dx + f \bigl[ b(t), t \bigr] \, \frac{d b(t)}{dt} - f \bigl[ a(t), t \bigr] \, \frac{d a(t)}{dt}
  \end{equation}
\end{lemma}



\clearpage
\section{Complex Numbers}\label{sec:Complex_Numbers}
\begin{definition}[Complex Number]\label{def:Complex_Number}
  A \emph{complex number} is a hyper real number system.
  This means that two real numbers, $a, b \in \RealNumbers$, are used to construct the set of complex numbers, denoted $\ComplexNumbers$.

  A complex number is written, in Cartesian form, as shown in \Cref{eq:Complex_Number} below.
  \begin{equation}\label{eq:Complex_Number}
    z = a \pm ib
  \end{equation}
  where
  \begin{equation}\label{eq:Imaginary_Value}
    i = \sqrt{-1}
  \end{equation}

  \begin{remark*}[$i$ vs. $j$ for Imaginary Numbers]
    Complex numbers are generally denoted with either $i$ or $j$.
    Electrical engineering regularly makes use of $j$ as the imaginary value.
    This is because alternating current $i$ is already taken, so $j$ is used as the imaginary value instad.
  \end{remark*}
\end{definition}

\subsection{Parts of a Complex Number}\label{subsec:Complex_Number_Parts}
A \nameref{def:Complex_Number} is made of up 2 parts:
\begin{enumerate}[noitemsep]
\item \nameref{def:Real_Part}
\item \nameref{def:Imaginary_Part}
\end{enumerate}

\begin{definition}[Real Part]\label{def:Real_Part}
  The \emph{real part} of an imaginary number, denoted with the $\Re$ operator, is the portion of the \nameref{def:Complex_Number} with no part of the imaginary value $i$ present.

  If $z = x + iy$, then
  \begin{equation}\label{eq:Real_Part}
    \Real{z} = x
  \end{equation}

  \begin{remark}[Alternative Notation]\label{rmk:Real_Part_Alternative_Notation}
    The \nameref{def:Real_Part} of a number sometimes uses a slightly different symbol for denoting the operation.
    It is:
    \begin{equation*}
      \mathfrak{Re}
    \end{equation*}
  \end{remark}
\end{definition}

\begin{definition}[Imaginary Part]\label{def:Imaginary_Part}
  The \emph{imaginary part} of an imaginary number, denoted with the $\Im$ operator, is the portion of the \nameref{def:Complex_Number} where the imaginary value $i$ is present.

  If $z = x + iy$, then
  \begin{equation}\label{eq:Imaginary_Part}
    \Imag{z} = y
  \end{equation}

  \begin{remark}[Alternative Notation]\label{rmk:Imaginary_Part_Alternative_Notation}
    The \nameref{def:Imaginary_Part} of a number sometimes uses a slightly different symbol for denoting the operation.
    It is:
    \begin{equation*}
      \mathfrak{Im}
    \end{equation*}
  \end{remark}
\end{definition}

\subsection{Binary Operations}\label{subsec:Binary_Operations}

%%% Local Variables:
%%% mode: latex
%%% TeX-master: shared
%%% End:


\subsection{Complex Conjugates}\label{app:Complex_Conjugates}
\begin{definition}[Complex Conjugate]\label{def:Complex_Conjugate}
  The conjugate of a complex number is called its \emph{complex conjugate}.
  The complex conjugate of a complex number is the number with an equal real part and an imaginary part equal in magnitude but opposite in sign.
  If we have a complex number as shown below,
  \begin{equation*}
    z = a \pm bi
  \end{equation*}

  then, the conjugate is denoted and calculated as shown below.
  \begin{equation}\label{eq:Complex_Conjugates}
    \Conjugate{z} = a \mp bi
  \end{equation}
\end{definition}

The \nameref{def:Complex_Conjugate} can also be denoted with an asterisk ($*$).
This is generally done for complex functions, rather than single variables.
\begin{equation}\label{eq:Complex_Conjugates_Asterisk}
  z^{*} = \Conjugate{z}
\end{equation}

%%% Local Variables:
%%% mode: latex
%%% TeX-master: shared
%%% End:


\subsection{Geometry of Complex Numbers}\label{subsec:Geometry_Complex_Numbers}
So far, we have viewed \nameref{def:Complex_Number}s only algebraically.
However, we can also view them geometrically as points on a 2 dimensional \nameref{def:Argand_Plane}.

\begin{definition}[Argand Plane]\label{def:Argand_Plane}
  An \emph{Argane Plane} is a standard two dimensional plane whose points are all elements of the complex numbers, $z \in \ComplexNumbers$.
  This is taken from Descarte's definition of a completely real plane.

  The Argand plane contains 2 lines that form the axes, that indicate the real component and the imaginary component of the complex number specified.
\end{definition}

A \nameref{def:Complex_Number} can be viewed as a point in the \nameref{def:Argand_Plane}, where the \nameref{def:Real_Part} is the ``$x$''-component and the \nameref{def:Imaginary_Part} is the ``$y$''-component.

By plotting this, you see that we form a right triangle, so we can find the hypotenuse of that triangle.
This hypotenuse is the distance the point $p$ is from the origin, refered to as the \nameref{def:Complex_Number_Modulus}.
\begin{remark*}
  When working with \nameref{def:Complex_Number}s geometrically, we refer to the points, where they are defined like so:
  \begin{equation*}
    z = x + iy = p(x, y)
  \end{equation*}

  Note that $p$ is \textbf{not} a function of $x$ and $y$.
  Those are the values that inform us \textbf{where} $p$ is located on the \nameref{def:Argand_Plane}.
\end{remark*}

\subsubsection{Modulus of a Complex Number}\label{subsubsec:Complex_Number_Modulus}
\begin{definition}[Modulus]\label{def:Complex_Number_Modulus}
  The \emph{modulus} of a \nameref{def:Complex_Number} is the distance from the origin to the complex point $p$.
  This is based off the Pythagorean Theorem.
  \begin{equation}\label{eq:Complex_Number_Modulus}
    \begin{aligned}
      {\lvert z \rvert}^{2} = x^{2} + y^{2} &= z \Conjugate{z} \\
      \lvert z \rvert &= \sqrt{x^{2} + y^{2}}
    \end{aligned}
  \end{equation}
\end{definition}

\begin{propertylist}
\item The \emph{Law of Moduli} states that $\lvert z w \rvert = \lvert z \rvert \lvert w \rvert$.\label{prop:Law_of_Moduli}.
\end{propertylist}

We can prove \Cref{prop:Law_of_Moduli} using an algebraic identity.
\begin{proof}[Prove \Cref*{prop:Law_of_Moduli}]
  Let $z$ and $w$ be complex numbers ($z, w \in \ComplexNumbers$).
  We are asked to prove
  \begin{equation*}
    \lvert z w \rvert = \lvert z \rvert \lvert w \rvert
  \end{equation*}

  But, it is actually easier to prove
  \begin{equation*}
    {\lvert z w \rvert}^{2} = {\lvert z \rvert}^{2} {\lvert w \rvert}^{2}
  \end{equation*}

  We start by simplifying the ${\lvert z w \rvert}^{2}$ equation above.
  \begin{align*}
    {\lvert z w \rvert}^{2} &= {\lvert z \rvert}^{2} {\lvert w \rvert}^{2} \\
    \intertext{Using the definition of the \nameref{def:Complex_Number_Modulus} of a \nameref{def:Complex_Number} in \Cref{eq:Complex_Number_Modulus}, we can expand the modulus.}
                            &= (z w) (\Conjugate{z w}) \\
    \intertext{Using \Cref{prop:Complex_Conjugate_Split} for multiplication allows us to do the next step.}
                            &= (z w) (\Conjugate{z} \Conjugate{w}) \\
    \intertext{Using Multiplicative Associativity and Multiplicative Commutativity, we can simplify this further.}
                            &= (z \Conjugate{z}) (w \Conjugate{w}) \\
                            &= {\lvert z \rvert}^{2} {\lvert w \rvert}^{2}
  \end{align*}

  Note how we never needed to define $z$ or $w$, so this is as general a result as possible.
\end{proof}

\paragraph{Algebraic Effects of the Modulus' \Cref*{prop:Law_of_Moduli}}\label{par:Law_of_Moduli-Algebraic_Effects}
For this section, let $z = x_{1} + iy_{1}$ and $w = x_{2} + iy_{2}$.
Now,
\begin{align*}
  z w &= (x_{1}x_{2} - y_{1}y_{2}) + i(x_{1}y_{2} + x_{2}y_{1}) \\
  {\lvert z w \rvert}^{2} &= {(x_{1}x_{2} - y_{1}y_{2})}^{2} + {(x_{1}y_{2} + x_{2}y_{1})}^{2} \\
      &= \left( x_{1}^{2} + x_{2}^{2} \right) \left( x_{2}^{2} + y_{2}^{2} \right) \\
      &= {\lvert z \rvert}^{2} {\lvert w \rvert}^{2}
\end{align*}

However, the Law of Moduli (\Cref{prop:Law_of_Moduli}) does \textbf{not} hold for a hyper complex number system one that uses 2 or more imaginaries, i.e.\ $z = a + iy + jz$.
But, the Law of Moduli (\Cref{prop:Law_of_Moduli}) \textbf{does} hold for hyper complex number system that uses 3 imaginaries, $a = z + iy + jz + k \ell$.

\paragraph{Conceptual Effects of the Modulus' \Cref*{prop:Law_of_Moduli}}\label{par:Law_of_Moduli-Conceptual_Effects}
We are interested in seeing if $\lvert z w \rvert = (x_{1}^{2} + y_{1}^{2})(x_{2}^{2}+y_{2}^{2})$ can be extended to more complex terms (3 terms in the complex number).

However, Langrange proved that the equation below \textbf{always} holds.
Note that the $z$ below has no relation to the $z$ above.
\begin{equation*}
  (x_{1} + y_{1} + z_{1}) \neq X^{2} + Y^{2} + Z^{2}
\end{equation*}

%%% Local Variables:
%%% mode: latex
%%% TeX-master: shared
%%% End:


\subsection{Circles and Complex Numbers}\label{subsec:Circles_Complex_Numbers}
We need to define both a center and a radius, just like with regular purely real values.
\Cref{eq:Circles_Complex_Numbers} defines the relation required for a circle using \nameref{def:Complex_Number}s.
\begin{equation}\label{eq:Circles_Complex_Numbers}
  \lvert z - a \rvert = r
\end{equation}

\begin{example}[Lecture 2, Example 1]{Convert to Circle}
  Given the expression below, find the location of the center of the circle and the radius of the circle?
  \begin{equation*}
    \lvert 5 iz + 10 \rvert = 7
  \end{equation*}
  \tcblower{}
  This is just a matter of simplification and moving terms around.
  \begin{align*}
    \lvert 5 iz + 10 \rvert &= 7 \\
    \lvert 5i (z + \frac{10}{5i}) \rvert &= 7 \\
    \lvert 5i (z + \frac{2}{i}) \rvert &= 7 \\
    \lvert 5i (z + \frac{2}{i} \frac{-i}{-i}) \rvert &= 7 \\
    \lvert 5i (z - 2i) \rvert &= 7 \\
    \intertext{Now using the Law of Moduli (\Cref{prop:Law_of_Moduli}) $\lvert a b \rvert = \lvert a \rvert \lvert b \rvert$, we can simplify out the extra imaginary term.}
    \lvert 5i \rvert \lvert z-2i \rvert &= 7 \\
    5 \lvert z - 2i \rvert &= 7 \\
    \lvert z - 2i \rvert = \frac{7}{5}
  \end{align*}

  Thus, the circle formed by the equation $\lvert 5 iz + 10 \rvert = 7$ is actually $\lvert z - 2i \rvert = \frac{7}{5}$, with a center at $a = 2i$ and a radius of $\frac{7}{5}$.
\end{example}

\subsubsection{Annulus}\label{subsubsec:Annulus}
\begin{definition}[Annulus]\label{def:Annulus}
  An \emph{annulus} is a region that is bounded by 2 concentric circles.
  This takes the form of \Cref{eq:Annulus}.
  \begin{equation}\label{eq:Annulus}
    r_{1} \leq \lvert z - a \rvert \leq r_{2}
  \end{equation}

  In \Cref{eq:Annulus}, each of the $\leq$ symbols could also be replaced with $<$.
  This leads to 3 different possibilities for the annulus:
  \begin{enumerate}[noitemsep]
  \item If both inequality symbols are $\leq$, then it is a \textbf{Closed Annulus}.
  \item If both inequality symbols are $<$, then it is an \textbf{Open Annulus}.
  \item If \textbf{only one} inequality symbol $<$ and the other $\leq$, then it is not an \textbf{Open Annulus}.
  \end{enumerate}
\end{definition}


%%% Local Variables:
%%% mode: latex
%%% TeX-master: shared
%%% End:



%%% Local Variables:
%%% mode: latex
%%% TeX-master: shared
%%% End:

% To make this print, you must include a citation somewhere in the document
\clearpage
\printbibliography{}
\end{document}
%%% Local Variables:
%%% mode: latex
%%% TeX-master: t
%%% End:
