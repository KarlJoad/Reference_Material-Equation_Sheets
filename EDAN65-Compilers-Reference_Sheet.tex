\documentclass[10pt,letterpaper,final,twoside,notitlepage]{article}
\usepackage[margin=.5in]{geometry} % 1/2 inch margins on all pages
\usepackage[utf8]{inputenc} % Define the input encoding
\usepackage[USenglish]{babel} % Define language used
\usepackage{amsmath}
\usepackage{amsfonts}
\usepackage{amssymb}
\usepackage{amsthm} % Gives us plain, definition, and remark to use in \theoremstyle{style}
\usepackage{graphicx}

\usepackage{hyperref} % Generate hyperlinks to referenced items
\usepackage[noabbrev,nameinlink]{cleveref} % Fancy cross-references in the document everywhere
\usepackage{nameref} % Can make references by name to places
\usepackage{subcaption} % Allows for multiple figures in one Figure environment
\usepackage{siunitx} % Gives us ways to typeset units for stuff
\usepackage{enumitem} % Provides [noitemsep, nolistsep] for more compact lists
\usepackage{chngcntr} % Allows us to tamper with the counter a little more
\usepackage{empheq} % Allow boxing of equations in special math environments
\usepackage{tcolorbox} % Allows us to create boxes of various types for examples
\usepackage{tikz} % Allows us to create TikZ and PGF Pictures
% \usepackage{ctable} % Greater control over tables and how they look
\usepackage[outputdir=./TeX_Output]{minted} % Allow us to nicely typeset 300+ programming languages

% Create a special minted environment just for java source code.
\newminted[javasource]{java}{
%  frame=lines,
  linenos
}

% Create a theorem environment
\theoremstyle{plain}
\newtheorem{theorem}{Theorem}

% Create a definition environment
\theoremstyle{definition}
\newtheorem{definition}{Defn}
\newtheorem{corollary}{Corollary}[section]
% \begin{definition}[Term] \label{def:}
% 		Make sure the term is emphasized with \emph{term}.
%		This ensures that if \emph is changed, it shows up everywhere
% \end{definition}

% Create a numbered remark environment numbered based on definition
% NOTE: This version of remark MUST go inside a definition environment
\theoremstyle{remark}
\newtheorem{remark}{Remark}[definition]
%\counterwithin{definition}{subsection} % Uncomment to have definitions use section.subsection numbering

% Create an unnumbered remark environment for general use
% NOTE: This version of remark has NO restrictions on placement
\newtheorem*{remark*}{Remark}

% Create a tcolorbox for examples
\newtcolorbox[auto counter,
number within=section,
number format=\arabic,
crefname={example}{examples}, % Define reference format for cref (No Capitals)
Crefname={Example}{Examples}, % Reference format for cleveref (With Capitals)
]{example}[2][]{
	width=\textwidth,
	title={Example \thetcbcounter: #2. #1},
	fonttitle=\bfseries,
	label={ex:#2},
	nameref=#2,
	colbacktitle=white!100!black,
	coltitle=black!100!white,
	colback=white!100!black,
	upperbox=visible,
	lowerbox=visible,
	sharp corners=all
}

% Redefine the 'end of proof' symbol to be a black square, not blank
\renewcommand\qedsymbol{$\blacksquare$} % Change proofs to have black square at end

% Math Operators that are useful to abstract the written math away to one spot
\DeclareMathOperator{\RealNums}{\mathbb{R}}
\DeclareMathOperator*{\argmax}{argmax} % Thin Space and subscripts are UNDER in display

\begin{titlepage}
  \title{EDAN65: Compilers - Reference Sheet}
  \author{Karl Hallsby}
  \date{Last Edited: \today} % We want to inform people when this document was last edited
\end{titlepage}

\begin{document}
\pagenumbering{gobble}
\maketitle
\pagenumbering{roman} % i, ii, iii on beginning pages, that don't have content
\tableofcontents
\clearpage
\pagenumbering{arabic} % 1,2,3 on content pages

\section{Introduction}\label{sec:Introduction}
There are numerous steps in the compilation process of a standard program.
Each phase converts the program from one representation to another.

\begin{enumerate}
\item \nameref{sec:Lexical Analysis}
\item \nameref{def:Syntactic Analysis}
\item \nameref{def:Semantic Analysis}
\item Intermediate Code Generation
\item Optimization
\item Target Code Generation
\end{enumerate}

\begin{definition}[Syntactic Analysis (Parsing)]\label{def:Syntactic Analysis}
  \emph{Syntactic Analysis} or \emph{Parsing} is the process where tokens are input and an AST (Abstract Syntax Tree) is created.
  This AST is generated based on the input source code and the \nameref{def:Lexical Analysis} that occurs.

  This code would generate an error during the \nameref{def:Syntactic Analysis}.
  \begin{javasource}
    int r( {
      return 3;
    }
  \end{javasource}
  This wouldn't fail during \nameref{def:Lexical Analysis} because the scanner doesn't care that the parentheses don't match.
  All that it cares about is that there are parentheses that it needs to mark.
  During the \nameref{def:Syntactic Analysis} we find out that the syntax would be wrong.
  This would happen because we can't line our tokens up correctly in our AST.
  
  \begin{remark}
    \nameref{def:Syntactic Analysis} \emph{ONLY} handles the reading in of tokens and creating an Abstract Syntax Tree.
    It \emph{DOES NOT} attach any meaning to anything.
    Therefore, this does not return an error during \nameref{def:Syntactic Analysis}.
    \begin{javasource}
      integer q() {
        return 3;
      }
    \end{javasource}
    However, it does return an error during \nameref{def:Semantic Analysis}.
  \end{remark}
\end{definition}

\begin{definition}[Semantic Analysis]\label{def:Semantic Analysis}
  \emph{Semantic Analysis} is the phase of the compilation process that takes the AST (Abstract Syntax Tree) and attaches some semblance of meaning to the tokens in the tree.
  We determine what each ``phrase'' means, relate the uses of variables to their definitions, check types of expressions, and request translations of each ``phrase''.
  This is the point in the compilation process where the strings that were read in by the scanner and organized by the parser have any meaning. Before this, the only things that can be caught are token errors, and the like.
  So, this will generate an error that is caught during \nameref{def:Semantic Analysis}.
  \begin{javasource}
    integer q() {
      return 3;
    }
  \end{javasource}
  Because \verb|integer| isn't a valid keyword in the Java language, at least not by default, and not capitalized like that, it gets caught during \nameref{def:Semantic Analysis}.

  This would also generate an error during \nameref{def:Semantic Analysis}.
  \begin{javasource}
    int p(int x) {
      int y;
      y = x * 2;
    }
  \end{javasource}

  Both of these wouldn't be caught before the \nameref{def:Semantic Analysis} because the tokens read in during \nameref{def:Lexical Analysis} and organized during \nameref{def:Syntactic Analysis} do not have any meaning any earlier.
\end{definition}


\section{Lexical Analysis/Scanning}\label{sec:Lexical Analysis}
\begin{definition}[Lexical Analysis (Scanning)]\label{def:Lexical Analysis}
  \emph{Lexical Analysis} or \emph{Scanning} is the phase of the compilation process that reads in the source code text.
  It breaks the things it reads into \emph{tokens}.
  \begin{remark}
    \nameref{def:Lexical Analysis} \emph{ONLY} handles the reading IN of source code and the outputting of tokens.
    It \emph{DOES NOT} attach any meaning or put anything together.
  \end{remark}

  This means that these are the \textbf{\emph{ONLY}} types of errors that will be caught.
  \begin{javasource}
    int #s() {
      return 3;
    }
  \end{javasource}

  Because the \# token isn't understood by the scanner, the whole thing fails.
  The Scanner is just a simple look up device. It can only find things that it knows about.
  If it sees something that it has no clue about, it fails.
\end{definition}

There are several ways to implement a scanner.
One of the most common ways is the use of a \nameref{def:Finite State Automaton} or Finite State Machine through \nameref{def:Regular_Expression}s.

\subsection{Regular Expressions}\label{subsec:Regular_Expressions}
\begin{definition}[Regular Expression]\label{def:Regular_Expression}
  A \emph{regular expression}, sometimes called a \emph{regex} is a way to define a sequence of characters to form strings.
\end{definition}

There are 2 types of \nameref{def:Regular_Expression}s, based on the features available to make the regular expression.
\begin{enumerate}[noitemsep]
\item \nameref{def:Regex_Core_Notation}
\item \nameref{def:Regex_Extended_Notation}
\end{enumerate}

\begin{definition}[Core Notation]\label{def:Regex_Core_Notation}
  The \emph{core notation} of a \nameref{def:Regular_Expression} has a small number of features available.
  These are shown in \Cref{tab:Regex_Core_Notation}.

  \begin{table}[h!]
    \centering
    \begin{tabular}{ccc}
      \toprule
      \nameref{def:Regular_Expression} & Read As & Called \\
      \midrule
      $a$ & a & Symbol \\
      $M \vert N$ & $M$ or $N$ & Alternative \\
      $MN$ & $M$ followed by $N$ & Concatenation \\
      $\epsilon$ & The \nameref{def:Empty_String} & Epsilon \\
      $M*$ & Zero or more $M$ & Repetition (Kleene Star) \\
      $(M)$ & & Scope \\
      \bottomrule
    \end{tabular}
    \caption{\nameref{def:Regular_Expression} \nameref{def:Regex_Core_Notation}}
    \label{tab:Regex_Core_Notation}
  \end{table}
  Where $a$ is a symbol in the alphabet.
  $M$ and $N$ are \nameref{def:Regular_Expression}s.
\end{definition}

\begin{definition}[Extended Notation]\label{def:Regex_Extended_Notation}
  The \emph{extended notation} of a \nameref{def:Regular_Expression} contains all the features of the \nameref{def:Regex_Core_Notation}, and some additional features.
  These additional features \emph{can} be represented in the \nameref{def:Regex_Core_Notation}, but are confusing to read and write.
  
  The \nameref{def:Regex_Core_Notation} features are shown in \Cref{tab:Regex_Core_Notation}, the additional features added by the \nameref{def:Regex_Extended_Notation} are shown in \Cref{tab:Regex_Extended_Notation}.

  \begin{table}[h!]
    \centering
    \begin{tabular}{ccc}
      \toprule
      Extended \nameref{def:Regular_Expression} & Read As & Means \\
      \midrule
      $M+$ & One or more & $M M*$ \\
      $M$? & Optional & $\epsilon \lvert M$ \\
      $[aou]$ & One of $\ldots$ (a character class) & $a \vert o \vert u$ \\
      $[a-zA-Z]$ & & $a \vert b \vert \ldots \vert z \vert A \vert B \ldots \vert Z$ \\
      $[\wedge 0-9]$ & \multirow{2}{*}{Not} & \multirow{2}{*}{One character, but any one of those listed} \\
      Appel Notation: $\thicksim{}[0-9]$ & & \\
      ``a+b'' & The string & a \+ b \\
      \bottomrule
    \end{tabular}
    \caption{\nameref{def:Regular_Expression} \nameref{def:Regex_Extended_Notation}}
    \label{tab:Regex_Extended_Notation}
  \end{table}
\end{definition}

\subsection{Finite State Automata}\label{subsec:Finite State Automata}
Finite state automata are used for regular expressions (regex's) to determine a matching word.

\begin{definition}[Finite State Automaton]\label{def:Finite State Automaton}
  A \emph{finite state automaton} or \emph{finite state machine} is a mathematical model of computation.
  It is an abstract machine that can be in exactly one of a finite number of states at any given time.
  The FSM can change from one state to another in response to some external inputs; the change from one state to another is called a transition.
  An FSM is defined by a list of its states, its initial state, and the conditions for each transition.

  There are 2 types of finite statue automata:
  \begin{enumerate}[noitemsep, nolistsep]
  \item \nameref{def:DeterministicFiniteAutomataDFA}
  \item \nameref{def:Non-deterministicFiniteAutomataNFA}
  \end{enumerate}

  A deterministic finite state automata can be constructed to be equivalent to any non-deterministic one.

  \begin{remark}
    The plural of Finite State Automaton is Finite State Automata.
  \end{remark}
\end{definition}

\begin{definition}[Deterministic Finite Automata (DFA)]\label{def:DeterministicFiniteAutomataDFA}
  In a \emph{deterministic finite automaton} or \emph{DFA}, no two edges leaving from the same state are labeled with the same symbol.
  Additionally, there cannot be an edge that matches the empty string, $\epsilon$.
  A deterministic finite automaton will eventually terminate when it steps through all of its states necessary to reach the accepting state.
  
  The key difference between a \nameref{def:DeterministicFiniteAutomataDFA} and a \nameref{def:Non-deterministicFiniteAutomataNFA} is that you can always figure out the path that a deterministic finite automaton will take.
\end{definition}

\begin{definition}[Non-deterministic Finite Automata (NFA)]\label{def:Non-deterministicFiniteAutomataNFA}
  A \emph{non-deterministic finite automaton}, or \emph{NFA}, is one that has multiple edges leaving a single state that have the same symbol.
  It may also have special edges labeled with the empty string $\epsilon$, which is when a state is followed without ``eating'' any of the input string.
  A non-deterministic finite automaton may eventually terminate when it steps through all its states necessary to reach its accepting state.
  
  The key difference between a \nameref{def:Non-deterministicFiniteAutomataNFA} and a \nameref{def:DeterministicFiniteAutomataDFA} is that you cannot always determine the exact path that the \nameref{def:Finite State Automaton} will will take.
\end{definition}

\subsubsection{Converting a NFA to a DFA}\label{subsubsec:Convert_NFA_to_DFA}
There are a few steps for converting a non-deterministic finite automaton to a deterministic one.
\begin{enumerate}[noitemsep]
\item Start at the start state and enter it
\item Follow all the states that accept the empty string $\epsilon$ and combine them with the start state.
\item After that, read in the first character/word from the input and follow all the states that you combined.
  \begin{itemize}[noitemsep]
  \item This means that you will be following multiple states or edges at the same time.
  \end{itemize}
\item Continue doing this until you combine all the states down to a deterministic finite automaton.
  \begin{itemize}[noitemsep]
  \item You can have multiple instances of the same state, i.e., you can have state 5 in two different state bubbles, so long as the list of states inside is unique.
  \end{itemize}
\item The end states are found by taking the end states from the non-deterministic finite automaton and placing them in the deterministic finite automaton.
  \begin{itemize}[noitemsep]
  \item This means that if state 3 is and end state in the non-deterministic finite automaton, then every occurrence of state 3 in the deterministic finite automaton will be and end state.
  \end{itemize}
\end{enumerate}

%%% Local Variables:
%%% mode: latex
%%% TeX-master: "../EDAN65-Compilers-Reference_Sheet"
%%% End:

%====================================APPENDIX====================================
\appendix
\counterwithin{equation}{section}
\counterwithin{definition}{subsection}

% \clearpage
% \subsection{Trigonometry} \label{app:Trig}
	\subsubsection{Trigonometric Formulas} \label{subsubsec:Trig Formulas}
		\begin{equation} \label{eq:Sin plus Sin with diff Angles}
			\sin \left( \alpha \right) + \sin \left( \beta \right) = 2 \sin \left( \frac{\alpha + \beta}{2} \right) \cos\left( \frac{\alpha - \beta}{2} \right)  
		\end{equation}
		\begin{equation} \label{eq:Cosine-Sine Product}
			\cos \left( \theta \right) \sin \left( \theta \right) = \frac{1}{2} \sin \left( 2 \theta \right)
		\end{equation}
	
	\subsubsection{Euler Equivalents of Trigonometric Functions} \label{subsubsec:Euler Equivalents}
		\begin{equation} \label{eq:Euler Sin}
			\sin \left( x \right) = \frac{e^{\imath x} + e^{-\imath x}}{2}
		\end{equation}
		\begin{equation} \label{eq:Euler Cos}
			\cos \left( x \right) = \frac{e^{\imath x} - e^{-\imath x}}{2 \imath}
		\end{equation}
		\begin{equation} \label{eq:Euler Sinh}
			\sinh \left( x \right) = \frac{e^{x} - e^{-x}}{2}
		\end{equation}
		\begin{equation} \label{eq:Euler Cosh}
			\cosh \left( x \right) = \frac{e^{x} + e^{-x}}{2}
		\end{equation}

% \clearpage
% \section{Calculus}\label{app:Calculus}
\subsection{L'Hopital's Rule}\label{subsec:LHopitals_Rule}
L'Hopital's Rule can be used to simplify and solve expressions regarding limits that yield irreconcialable results.
\begin{lemma}[L'Hopital's Rule]\label{lemma:LHopitals_Rule}
  If the equation
  \begin{equation*}
    \lim\limits_{x \rightarrow a} \frac{f(x)}{g(x)} =
    \begin{cases}
      \frac{0}{0} \\
      \frac{\infty}{\infty} \\
    \end{cases}
  \end{equation*}
  then \Cref{eq:LHopitals_Rule} holds.
  \begin{equation}\label{eq:LHopitals_Rule}
    \lim\limits_{x \rightarrow a} \frac{f(x)}{g(x)} = \lim\limits_{x \rightarrow a} \frac{f'(x)}{g'(x)}
  \end{equation}
\end{lemma}

\subsection{Fundamental Theorems of Calculus}\label{subsec:Fundamental Theorem of Calculus}
\begin{definition}[First Fundamental Theorem of Calculus]\label{def:1st Fundamental Theorem of Calculus}
  The \emph{first fundamental theorem of calculus} states that, if $f$ is continuous on the closed interval $\left[ a,b \right]$ and $F$ is the indefinite integral of $f$ on $\left[ a,b \right]$, then

  \begin{equation}\label{eq:1st Fundamental Theorem of Calculus}
    \int_{a}^{b}f \left( x \right) dx = F \left( b \right) - F \left( a \right)
  \end{equation}
\end{definition}

\begin{definition}[Second Fundamental Theorem of Calculus]\label{def:2nd Fundamental Theorem of Calculus}
  The \emph{second fundamental theorem of calculus} holds for $f$ a continuous function on an open interval $I$ and $a$ any point in $I$, and states that if $F$ is defined by

  \begin{equation*}
    F \left( x \right) = \int_{a}^{x} f \left( t \right) dt,
  \end{equation*}
  then
  \begin{equation}\label{eq:2nd Fundamental Theorem of Calculus}
    \begin{aligned}
      \frac{d}{dx} \int_{a}^{x} f \left( t \right) dt &= f \left( x \right) \\
      F' \left( x \right) &= f \left( x \right) \\
    \end{aligned}
  \end{equation}
\end{definition}

\begin{definition}[argmax]\label{def:argmax}
  The arguments to the \emph{argmax} function are to be maximized by using their derivatives.
  You must take the derivative of the function, find critical points, then determine if that critical point is a global maxima.
  This is denoted as
  \begin{equation*}\label{eq:argmax}
    \argmax_{x}
  \end{equation*}
\end{definition}

\subsection{Rules of Calculus}\label{subsec:Rules of Calculus}
\subsubsection{Chain Rule}\label{subsubsec:Chain Rule}
\begin{definition}[Chain Rule]\label{def:Chain Rule}
  The \emph{chain rule} is a way to differentiate a function that has 2 functions multiplied together.

  If
  \begin{equation*}
    f(x) = g(x) \cdot h(x)
  \end{equation*}
  then,
  \begin{equation}\label{eq:Chain Rule}
    \begin{aligned}
      f'(x) &= g'(x) \cdot h(x) + g(x) \cdot h'(x) \\
      \frac{df(x)}{dx} &= \frac{dg(x)}{dx} \cdot g(x) + g(x) \cdot \frac{dh(x)}{dx} \\
    \end{aligned}
  \end{equation}
\end{definition}

\subsection{Useful Integrals}\label{subsec:Useful_Integrals}
\begin{equation}\label{eq:Cosine_Indefinite_Integral}
  \int \cos(x) \; dx = \sin(x)
\end{equation}

\begin{equation}\label{eq:Sine_Indefinite_Integral}
  \int \sin(x) \; dx = -\cos(x)
\end{equation}

\begin{equation}\label{eq:x_Cosine_Indefinite_Integral}
  \int x \cos(x) \; dx = \cos(x) + x \sin(x)
\end{equation}
\Cref{eq:x_Cosine_Indefinite_Integral} simplified with Integration by Parts.

\begin{equation}\label{eq:x_Sine_Indefinite_Integral}
  \int x \sin(x) \; dx = \sin(x) - x \cos(x)
\end{equation}
\Cref{eq:x_Sine_Indefinite_Integral} simplified with Integration by Parts.

\begin{equation}\label{eq:x_Squared_Cosine_Indefinite_Integral}
  \int x^{2} \cos(x) \; dx = 2x \cos(x) + (x^{2} - 2) \sin(x)
\end{equation}
\Cref{eq:x_Squared_Cosine_Indefinite_Integral} simplified by using Integration by Parts twice.

\begin{equation}\label{eq:x_Squared_Sine_Indefinite_Integral}
  \int x^{2} \sin(x) \; dx = 2x \sin(x) - (x^{2} - 2) \cos(x)
\end{equation}
\Cref{eq:x_Squared_Sine_Indefinite_Integral} simplified by using Integration by Parts twice.

\begin{equation}\label{eq:Exponential_Cosine_Indefinite_Integral}
  \int e^{\alpha x} \cos(\beta x) \; dx = \frac{e^{\alpha x} \bigl( \alpha \cos(\beta x) + \beta \sin(\beta x) \bigr)}{\alpha^{2} + \beta^{2}} + C
\end{equation}

\begin{equation}\label{eq:Exponential_Sine_Indefinite_Integral}
  \int e^{\alpha x} \sin(\beta x) \; dx = \frac{e^{\alpha x} \bigl( \alpha \sin(\beta x) - \beta \cos(\beta x) \bigr)}{\alpha^{2}+\beta^{2}} + C
\end{equation}

\begin{equation}\label{eq:Exponential_Indefinite_Integral}
  \int e^{\alpha x} \; dx = \frac{e^{\alpha x}}{\alpha}
\end{equation}

\begin{equation}\label{eq:x_Exponential_Indefinite_Integral}
  \int x e^{\alpha x} \; dx = e^{\alpha x} \left( \frac{x}{\alpha} - \frac{1}{\alpha^{2}} \right)
\end{equation}
\Cref{eq:x_Exponential_Indefinite_Integral} simplified with Integration by Parts.

\begin{equation}\label{eq:Inverse_x_Indefinite_Integral}
  \int \frac{dx}{\alpha + \beta x} = \int \frac{1}{\alpha + \beta x} \; dx = \frac{1}{\beta} \ln (\alpha + \beta x)
\end{equation}

\begin{equation}\label{eq:Inverse_x_Squared_Indefinite_Integral}
  \int \frac{dx}{\alpha^{2} + \beta^{2} x^{2}} = \int \frac{1}{\alpha^{2} + \beta^{2} x^{2}} \; dx = \frac{1}{\alpha \beta} \arctan \left( \frac{\beta x}{\alpha} \right)
\end{equation}

\begin{equation}\label{eq:a_Exponential_Indefinite_Integral}
  \int \alpha^{x} \; dx = \frac{\alpha^{x}}{\ln(\alpha)}
\end{equation}

\begin{equation}\label{eq:a_Exponential_Derivative}
  \frac{d}{dx} \alpha^{x} = \frac{d\alpha^{x}}{dx} = \alpha^{x} \ln(x)
\end{equation}

\subsection{Leibnitz's Rule}\label{subsec:Leibnitzs_Rule}
\begin{lemma}[Leibnitz's Rule]\label{lemma:Leibnitzs_Rule}
  Given
  \begin{equation*}
    g(t) = \int_{a(t)}^{b(t)} f(x, t) \, dx
  \end{equation*}
  with $a(t)$ and $b(t)$ differentiable in $t$ and $\frac{\partial f(x, t)}{\partial t}$ continuous in both $t$ and $x$, then
  \begin{equation}\label{eq:Leibnitzs_Rule}
    \frac{d}{dt} g(t) = \frac{d g(t)}{dt} = \int_{a(t)}^{b(t)} \frac{\partial f(x, t)}{\partial t} \, dx + f \bigl[ b(t), t \bigr] \, \frac{d b(t)}{dt} - f \bigl[ a(t), t \bigr] \, \frac{d a(t)}{dt}
  \end{equation}
\end{lemma}



\end{document}