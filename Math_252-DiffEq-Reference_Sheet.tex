\documentclass[10pt,letterpaper,final,twoside,notitlepage]{article}
\usepackage[margin=.5in]{geometry} % 1/2 inch margins on all pages
\usepackage[utf8]{inputenc} % Define the input encoding
\usepackage[USenglish]{babel} % Define language used
\usepackage{amsmath,amsfonts,amssymb}
\usepackage{amsthm} % Gives us plain, definition, and remark to use in \theoremstyle{style}
\usepackage{mathtools} % Allow for text and math in align* environment.
\usepackage{thmtools}
\usepackage{thm-restate}
\usepackage{graphicx}

\usepackage[
backend=biber,
style=alphabetic,
citestyle=authoryear]{biblatex} % Must include citation somewhere in document to print bibliography
\usepackage{hyperref} % Generate hyperlinks to referenced items
\usepackage[nottoc]{tocbibind} % Prints the Reference/Bibliography in TOC as well
\usepackage[noabbrev,nameinlink]{cleveref} % Fancy cross-references in the document everywhere
\usepackage{nameref} % Can make references by name to places
\usepackage{caption} % Allows for greater control over captions in figure, algorithm, table, etc. environments
\usepackage{subcaption} % Allows for multiple figures in one Figure environment
\usepackage[binary-units=true]{siunitx} % Gives us ways to typeset units for stuff
\usepackage{csquotes} % Context-sensitive quotation facilities
\usepackage{enumitem} % Provides [noitemsep, nolistsep] for more compact lists
\usepackage{chngcntr} % Allows us to tamper with the counter a little more
\usepackage{empheq} % Allow boxing of equations in special math environments
\usepackage[x11names]{xcolor} % Gives access to coloring text in environments or just text, MUST be before tikz
\usepackage{tcolorbox} % Allows us to create boxes of various types for examples
\usepackage{tikz} % Allows us to create TikZ and PGF Pictures
\usepackage{ctable} % Greater control over tables and how they look
\usepackage{diagbox} % Allow us to have shared diagonal cells in tables
\usepackage{multirow} % Allow us to have a single cell in a table span multiple rows
\usepackage{titling} % Put document information throughout the document programmatically
\usepackage[linesnumbered,ruled,vlined]{algorithm2e} % Allows us to write algorithms in a nice style.

\counterwithin{figure}{section}
\counterwithin{table}{section}
\counterwithin{equation}{section}
\counterwithin{algocf}{section}
\crefname{algocf}{algorithm}{algorithms}
\Crefname{algocf}{Algorithm}{Algorithms}
\setcounter{secnumdepth}{4}
\setcounter{tocdepth}{4} % Include \paragraph in toc
\crefname{paragraph}{paragraph}{paragraphs}
\Crefname{paragraph}{Paragraph}{Paragraphs}

% Create a theorem environment
\theoremstyle{plain}
\newtheorem{theorem}{Theorem}[section]
% Create a numbered theorem-like environment for lemmas
\newtheorem{lemma}{Lemma}[theorem]

% Create a definition environment
\theoremstyle{definition}
\newtheorem{definition}{Defn}
\newtheorem{corollary}{Corollary}[section]
% \begin{definition}[Term] \label{def:}
%   Make sure the term is emphasized with \emph{term}.
%   This ensures that if \emph is changed, it shows up everywhere
% \end{definition}

% Create a numbered remark environment numbered based on definition
% NOTE: This version of remark MUST go inside a definition environment
\theoremstyle{remark}
\newtheorem{remark}{Remark}[definition]
%\counterwithin{definition}{subsection} % Uncomment to have definitions use section.subsection numbering

% Create an unnumbered remark environment for general use
% NOTE: This version of remark has NO restrictions on placement
\newtheorem*{remark*}{Remark}

% Create a special list that handles properties. It can be broken and restarted
\newlist{propertylist}{enumerate}{1} % {Name}{Template}{Max Depth}
% [newlistname, LevelsToApplyTo]{formatting options}
\setlist[propertylist, 1]{label=\textbf{(\roman*)}, ref=\textbf{(\roman*)}, noitemsep, nolistsep}
\crefname{propertylisti}{property}{properties}
\Crefname{propertylisti}{Property}{Properties}

% Create a special list that handles enumerate starting with lower letters. Breakable/Restartable.
\newlist{boldalphlist}{enumerate}{1} % {Name}{Template}{Max Depth}
% [newlistname, LevelsToApplyTo]{formatting options}
\setlist[boldalphlist, 1]{label=\textbf{(\alph*)}, ref=\alph*, noitemsep, nolistsep} % Set options

\newlist{nocrefenumerate}{enumerate}{1} % {Name}{Template}{Max Depth}
% [newlistname, LevelsToApplyTo]{formatting options}
\setlist[nocrefenumerate, 1]{label=(\arabic*), ref=(\arabic*), noitemsep, nolistsep}

% Create a list that allows for deeper nesting of numbers. By default enumerate only allows depth=4.
\newlist{nestednums}{enumerate}{6}
% [newlistname, LevelsToApplyTo]{formatting options}
\setlist[nestednums]{noitemsep, label*=\arabic*.}

\tcbuselibrary{breakable} % Allow tcolorboxes to be broken across pages
% Create a tcolorbox for examples
% /begin{example}[extra name]{NAME}
% Create a tcolorbox for examples
% Argument #1 is optional, given by [], that is the textbook's problem number
% Argument #2 is mandatory, given by {}, that is the title for the example
% Avoid putting special characters, (), [], {}, ",", etc. in the title.
\newtcolorbox[auto counter,
number within=section,
number format=\arabic,
crefname={example}{examples}, % Define reference format for cref (No Capitals)
Crefname={Example}{Examples}, % Reference format for cleveref (With Capitals)
]{example}[2][]{ % The [2][] Means the first argument is optional
  width=\textwidth,
  title={Example \thetcbcounter: #2. #1}, % Parentheses and commas are not well supported
  fonttitle=\bfseries,
  label={ex:#2},
  nameref=#2,
  colbacktitle=white!100!black,
  coltitle=black!100!white,
  colback=white!100!black,
  upperbox=visible,
  lowerbox=visible,
  sharp corners=all,
  breakable
}

% Create a tcolorbox for general use
\newtcolorbox[% auto counter,
% number within=section,
% number format=\arabic,
% crefname={example}{examples}, % Define reference format for cref (No Capitals)
% Crefname={Example}{Examples}, % Reference format for cleveref (With Capitals)
]{blackbox}{
  width=\textwidth,
  % title={},
  fonttitle=\bfseries,
  % label={},
  % nameref=,
  colbacktitle=white!100!black,
  coltitle=black!100!white,
  colback=white!100!black,
  upperbox=visible,
  lowerbox=visible,
  sharp corners=all
}

% Redefine the 'end of proof' symbol to be a black square, not blank
\renewcommand{\qedsymbol}{$\blacksquare$} % Change proofs to have black square at end

% Common Mathematical Stuff
\newcommand{\Abs}[1]{\ensuremath{\lvert #1 \rvert}}
\newcommand{\DNE}{\ensuremath{\mathrm{DNE}}} % Used when limit of function Does Not Exist

% Complex Numbers functions
\renewcommand{\Re}{\operatorname{Re}} % Redefine to use the command, but not the fraktur version
\renewcommand{\Im}{\operatorname{Im}} % Redefine to use the command, but not the fraktur version
\newcommand{\Real}[1]{\ensuremath{\Re \lbrace #1 \rbrace}}
\newcommand{\Imag}[1]{\ensuremath{\Im \lbrace #1 \rbrace}}
\newcommand{\Conjugate}[1]{\ensuremath{\overline{#1}}}
\newcommand{\Modulus}[1]{\ensuremath{\lvert #1 \rvert}}
\DeclareMathOperator{\PrincipalArg}{\ensuremath{Arg}}

% Math Operators that are useful to abstract the written math away to one spot
% Number Sets
\DeclareMathOperator{\RealNumbers}{\ensuremath{\mathbb{R}}}
\DeclareMathOperator{\AllIntegers}{\ensuremath{\mathbb{Z}}}
\DeclareMathOperator{\PositiveInts}{\ensuremath{\mathbb{Z}^{+}}}
\DeclareMathOperator{\NegativeInts}{\ensuremath{\mathbb{Z}^{-}}}
\DeclareMathOperator{\NaturalNumbers}{\ensuremath{\mathbb{N}}}
\DeclareMathOperator{\ComplexNumbers}{\ensuremath{\mathbb{C}}}
\DeclareMathOperator{\RationalNumbers}{\ensuremath{\mathbb{Q}}}

% Calculus operators
\DeclareMathOperator*{\argmax}{argmax} % Thin Space and subscripts are UNDER in display

% Signal and System Functions
\DeclareMathOperator{\UnitStep}{\mathcal{U}}
\DeclareMathOperator{\sinc}{sinc} % sinc(x) = (sin(pi x)/(pi x))

% Transformations
\DeclareMathOperator{\Lapl}{\mathcal{L}} % Declare a Laplace symbol to be used

% Logical Operators
\DeclareMathOperator{\XOR}{\oplus}

% x86 CPU Registers
\newcommand{\rbpRegister}{\texttt{\%rbp}}
\newcommand{\rspRegister}{\texttt{\%rsp}}
\newcommand{\ripRegister}{\texttt{\%rip}}
\newcommand{\raxRegister}{\texttt{\%rax}}
\newcommand{\rbxRegister}{\texttt{\%rbx}}

%%% Local Variables:
%%% mode: latex
%%% TeX-master: shared
%%% End:


% These packages are more specific to certain documents, but will be availabe in the template
% \usepackage{esint} % Provides us with more types of integral symbols to use
% \usepackage{minted} % Allow us to nicely typeset 300+ programming languages

% \graphicspath{{./Drawings/Course}} % Uncomment this to use pictures in this document
% \addbibresource{./Bibliographies/CourseNum-Name.bib}

% Math Operators that are useful to abstract the written math away to one spot
% These are supposed to be document-specific mathematical operators that will make life easier
% Many fundamental operators are defined in Reference_Sheet_Preamble.tex

\begin{titlepage}
  \title{Math 252: Introduction to Differential Equations --- Reference Sheet \\ Illinois Institute of Technology}
  \author{Karl Hallsby}
  \date{Last Edited: \today}
\end{titlepage}

\begin{document}
\pagenumbering{gobble}
\maketitle
\pagenumbering{roman} % i, ii, iii on beginning pages, that don't have content
\tableofcontents
\clearpage
\pagenumbering{arabic} % 1, 2, 3 on content pages

\section{Introduction} \label{sec:Introduction}
This section will introduce the basic terminology and definitions for solving ordinary differential equations.
\subsection{Definitions and Terminology} \label{subsec:Definitions and Terminology}
\begin{definition}[Differential Equation] \label{def:Differential Equation}
  A \emph{differential equation (DE)} is an equation with 1 or more derivatives.
  \begin{remark}
    The highest differential determines the order of the differential equation.
    This means that the differential equation below is of order 2.
    \begin{align*}
      y'' + y &= 0 \\
      \frac{d^{2}y}{dx^{2}} + y &= 0 \\
    \end{align*}
  \end{remark}
\end{definition}
\begin{definition}{Initial Value Problem} \label{def:Initial Value Problem}
  A differential equation with one or more initial conditions is called an \emph{initial value problem (IVP)}.
  \begin{remark}
    To solve an initial value problem, you must have the same number of initial conditions as the order of the differential equation.
  \end{remark}
  \begin{remark}[Existence of Unique Solution]
    $R$ is a rectangular region on the xy-plane $a \leq x \leq b$, $c \leq y \leq d$ that contains $\left( x_{0}, y_{0} \right)$ interior.
    If $f \left( x,y \right)$ and $\frac{df}{dy}$ are continuous on $R$, then an interval exists $I_{0}$ such that $\left( x_{0}-h, x_{0}+h \right)$ where $h>0$, on the interval $\left[ a,b \right]$, and a unique function $y \left( x \right)$, defined on $I_{0}$ that is a solution of the initial value problem.
  \end{remark}
\end{definition}

\subsection{Confirm If Differential Equation} \label{subsec:Confirm Differential Equation}
You can confirm if the solution $y \left( x \right)$ found for a differential equation $y \left( x \right)'$ is the solution by differentiating the solution and putting that in the solved differential equation and verfiying that the equation holds true.
This is shown in \Cref{ex:Confirm Differential Solution}.
\begin{example}[]{Confirm Differential Solution}
  Given the differential equation, $2y' + y = 0$, is $y = e^{\frac{-x}{2}}$ a solution? \newline

  \tcblower

  \begin{align*}
    y' &= \frac{-1}{2} e^{\frac{-x}{2}} \\
    2 \left( \frac{-1}{2} e^{\frac{-x}{2}} \right) + \left( e^{\frac{-x}{2}} \right) &= 0 \\
    -e^{\frac{-x}{2}} + e^{\frac{-x}{2}} &= 0 \\
    0 &= 0 \text{ \checkmark}
  \end{align*}
\end{example}

\subsection{Separable Differential Equation} \label{subsec:Separable Differential Equation}
\begin{definition}[Separable] \label{def:Separable}
  A \emph{separable} differential equation allows you to move various elements around to solve the equation.
  For example,
  \begin{align*}
    \frac{dP}{dt} &= kP \\
    \frac{1}{P} dP &= k dt \\
    \ln \left( P \right) &= kt + C \\
    P &= Ce^{kt}
  \end{align*}
  \begin{remark}
    These are used extensively in modelling phenomena with differential equations.
    These include: \nameref{subsubsec:Population Growth}, \nameref{subsubsec:Radioactive Decay}, \nameref{subsubsec:Newton Law of Cooling/Heating}, and \nameref{subsubsec:Spread of Disease}.
  \end{remark}
\end{definition}

\subsection{Modeling with Differential Equations} \label{subsec:Modeling with DEs}
\subsubsection{Population Growth} \label{subsubsec:Population Growth}
\begin{definition}[Population Growth]
  \emph{Population growth} can be modelled with a separable differential equation. Namely,
  \begin{equation} \label{eq:Population Growth}
    \frac{dP}{dt} = kP
  \end{equation}
  \begin{remark}[Population Growth Parameters] \label{rmk:Population Growth Parameters}
    The parameters for the \nameref{eq:Population Growth}~equation are given below.
    \begin{itemize}[noitemsep, nolistsep]
    \item $k>0$
    \item $P>0$
    \end{itemize}
  \end{remark}
\end{definition}
\subsubsection{Radioactive Decay} \label{subsubsec:Radioactive Decay}
\begin{definition}[Radioactive Decay] \label{def:Radioactive Decay}
  \emph{Radioactive decay} is the process that some particularly heave atoms undergo to become lighter, more stable atoms.
  \begin{definition}[Half-Life] \label{subdef:Half-Life}
    The \emph{half-life} is the usual reported metric, and is defined as the amount of time required for an element to half its mass through \nameref{def:Radioactive Decay}.
  \end{definition}
  \begin{equation} \label{eq:Radioactive Decay}
    \frac{1}{2} A_{0} = A_{0} e^{kt}
  \end{equation}
  \begin{remark}[Radioactive Decay Parameters] \label{rmk:Radioactive Decay Parameters}
    The parameters for the \nameref{eq:Radioactive Decay}~equation are given below.
    \begin{itemize}[noitemsep, nolistsep]
    \item $k<0$
    \item $A>0$
    \end{itemize}
  \end{remark}
\end{definition}
\subsubsection{Newton's Law of Cooling/Heating} \label{subsubsec:Newton Law of Cooling/Heating}
\begin{definition}[Newton's Law of Cooling/Heating] \label{def:Newton Law of Cooling/Heating}
  \emph{Newton's Law of Cooling/Heating} is the same equation, but some of the parameters change.
  This equation is defined as:
  \begin{equation} \label{eq:Newton Law of Cooling/Heating}
    \frac{dT}{dt} = k \left( T-T_{m} \right)
  \end{equation}
  \begin{remark} \label{rmk:Newton Law of Cooling/Heating Parameters}
    The parameters for the \nameref{eq:Newton Law of Cooling/Heating}~equation are given below.
    \begin{itemize}[noitemsep, nolistsep]
    \item $\frac{dT}{dt}$; The rate of change of temperature in the object per unit time.
    \item $k<0$; The cooling constant and is unique to every object.
    \item $T$; The starting temperature.
    \item $T_{m}$; The temperature of the surrounding medium.
    \end{itemize}
  \end{remark}
\end{definition}
\subsubsection{Spread of Disease} \label{subsubsec:Spread of Disease}
\begin{definition}[Spread of Disease] \label{def:Spread of Disease}
  This is used to model the spread of something throughout a society or group of people.
  \begin{equation} \label{eq:Spread of Disease}
    \frac{dx}{dt} = kxy
  \end{equation}
  \begin{remark}
    The parameters for the \nameref{eq:Spread of Disease}~equation are given below.
    \begin{itemize}[noitemsep, nolistsep]
    \item $\frac{dx}{dt}$; Change in the number of infected per unit time.
    \item $k<0$; Transmission Constant
    \item$x$; Number of Infected
    \item $y$; Number of non-infected, $y$ is really a function of $x$
      \begin{itemize}[noitemsep, nolistsep]
      \item $y = n+1-x$
      \end{itemize}
    \end{itemize}
  \end{remark}
\end{definition}

\subsubsection{Chemical Reactions} \label{subsubsec:Chemical Reactions}
\begin{definition}[Chemical Reactions] \label{def:Chemical Reactions}
  These model how molecules interact in certain proportions to achieve some resultant molecule.
  \begin{equation} \label{eq:Chemical Reactions}
    \frac{dx}{dt} = k \left( \alpha - x \right) \left( \beta - x \right)
  \end{equation}
  \begin{remark}
    The parameters for the \nameref{def:Chemical Reactions}~equation are given below.
    \begin{itemize}[noitemsep, nolistsep]
    \item $x$; Amount of resultant chemical
    \item $k$; Reaction rate, must be greater than 0, $k > 0$
    \item $\frac{dx}{dt}$; Rate of creation of resultant molecule per unit time
    \item $\alpha$; Initial amount of Chemical ``A''
    \item $\beta$; Initial amount of Chemical ``B''
    \item $x \left( 0 \right) = 0$; Initial amount of resultant molecule must be 0 at the start
    \end{itemize}
  \end{remark}
\end{definition}

\subsubsection{Tank Mixture} \label{subsubsec:Tank Mixture}
\begin{definition}[Tank Mixture] \label{def:Tank Mixture}
  A well-mixed dissolved influent ``thing'' is brought into a tank and drained at some rate.
  What is the change in the amount of dissolved ``thing'' at any point in time?
  \begin{equation}\label{eq:Tank Mixture}
    \frac{dA}{dt} = R_{\text{in}} - R_{\text{out}}
  \end{equation}
  \begin{remark}
    The parameters for the \nameref{def:Tank Mixture}~equation are given below.
    \begin{itemize}[noitemsep, nolistsep]
    \item $A$; The amount of dissolved ``thing''
    \item $t$; The time of time the tank has taken
    \item $R_{\text{in}}$; The rate of dissolved ``thing'' into the tank
    \item $R_{\text{out}}$; The rate of dissolved ``thing'' out of the tank
    \end{itemize}
  \end{remark}
\end{definition}

\subsubsection{Torricelli's Law} \label{subsubsec:Torricelli's Law}
\begin{definition}[Torricelli's Law] \label{def:Torricelli's Law}
  This equation relates the rate the volume in a tank changes to the height of the water to the hole in the tank.
  \begin{equation} \label{eq:Torricelli's Law}
    \frac{dV}{dt} = -A_{h} \sqrt{2gh}
  \end{equation}
  \begin{remark}
    The parameters for the \nameref{def:Torricelli's Law}~ equation are given below.
    \begin{itemize}[noitemsep, nolistsep]
    \item $V = A_{w}h$; The volume of water above the hole
    \item $\frac{dV}{dt} = A_{w} \frac{dh}{dt}$; The change in the volume of the water above the hole
    \item $h$; Height of the water
    \item $A_{h}$; Width of the hole
    \item $A_{w}$; Cross-sectional area of the tank
    \end{itemize}
  \end{remark}
\end{definition}

\subsubsection{LRC Circuits} \label{subsubsec:LRC Circuits}
\begin{definition}[LRC Circuits] \label{def:LRC Circuits}
  An \emph{LRC Circuit} is analyzed in terms of the energy moving through the circuit.
  There is a unique relationship for the energy in each element:
  \begin{equation} \label{eq:Energy in Capacitor}
    E \left( t \right) = \frac{q}{C}
  \end{equation}
  \begin{equation} \label{eq:Energy in Resistor}
    E \left( t \right) = RI = R \frac{dq}{dt}
  \end{equation}
  \begin{equation} \label{eq:Energy in Inductor}
    E \left( t \right) = L \frac{dI}{dt} = L \frac{d^{2}q}{dt^{2}}
  \end{equation}
  \begin{remark}
    Depending on the circuit given, you might use a combination of these, but you \emph{\textbf{must}} have at least one capacitor or inductor, otherwise it is not a differential equation.
  \end{remark}
  \begin{remark}
    These equations \emph{add} together when the entire circuit is in series, i.e. the elements are put together back-to-back.
  \end{remark}
\end{definition}

\subsection{Linear and Non-Linear Differential Equations} \label{Linear vs. Non-Linear Differential Equations}
\begin{definition}[Linear Differential Equation] \label{def:Linear Differential Equation}
  A \emph{linear differential equation} is one that satisfies one of the following equations below.
  \begin{equation} \label{eq:Linear Differential Equation}
    \begin{aligned}
      a_{1} \left( x \right) \frac{dy}{dx} + a_{0} \left( x \right) &= g \left( x \right) \\
      a_{2} \left( x \right) \frac{d^{2}y}{dx^{2}} + a_{1} \left( x \right) \frac{dy}{dx} + a_{0} \left( x \right) &= g \left( x \right) \\
    \end{aligned}
  \end{equation}
  \begin{remark}
    The equations in Equation~\eqref{eq:Linear Differential Equation} can be generalized to the $n$th order as shown below.
    \begin{equation} \label{eq:General Linear Differential Equation}
      a_{n} \left( x \right) \frac{d^{n}y}{dx^{n}} + a_{n-1} \left( x \right) \frac{d^{n-1}y}{dx^{n-1}} + \ldots + a_{1} \left( x \right) \frac{dy}{dx} + a_{0} \left( x \right) = g \left( x \right)
    \end{equation}
  \end{remark}
\end{definition}
\begin{definition}[Non-Linear] \label{def:Non-Linear Differential Equation}
  A \emph{non-linear} differential equation is one that does not satisfy the definition of a \nameref{def:Linear Differential Equation}.
  It does not obey Equation~\eqref{eq:General Linear Differential Equation}.
\end{definition}

%%% Local Variables:
%%% mode: latex
%%% TeX-master: "../Math_252-DiffEq-Reference_Sheet"
%%% End: % Section 1

\section{Solving First Degree Differential Equations} \label{sec:Solve First Degree Differential Equations}
This section shows various ways to solve first-degree ordinary differential equations.

\subsection{Solution Curves without a Solution} \label{subsec:Solution Curves without a Solution}
These differential equations are ones that do not have a solution.
Instead, they can be categorized by \nameref{def:Direction Fields}
\begin{definition}[Direction Fields] \label{def:Direction Fields}
  \emph{Direction fields} are similar to vector fields, in that they show how potential solutions could satisfy the equation.
  % \input{./Drawings/Math_252/Direction_Fields.tikz}

  \begin{remark}
    It is important to note that this only applies to first order differential equations.
  \end{remark}
\end{definition}

\subsection{Separable Ordinary Differential Equations} \label{subsec:Separable ODEs}
These are some of the simplest ordinary differential equations to solve.
\begin{definition}[Separable Ordinary Differential Equations] \label{def:Separable ODEs}
  \begin{equation} \label{eq:Separable ODEs}
    \begin{aligned}
      \frac{dy}{dx} &= g \left( x \right) h \left( y \right) \\
      \int \frac{1}{h \left( y \right)} dy &= \int g \left( x \right) dx \\
    \end{aligned}
  \end{equation}

  \begin{remark}
    To be \emph{separable}, all functions of respective variables must be on the same side.
  \end{remark}
\end{definition}

\begin{example}[]{Separable Ordinary Differential Equation-Example 1}
  Solve \[ x \frac{dy}{dx} = 4y \]

  \tcblower

  \begin{align*}
    \frac{1}{y} dy &= \frac{4}{x} dx \\
    \ln \lvert y \rvert &= \left( 4 \ln \lvert x \rvert + C \right) \\
    \lvert y \rvert &= x^{4} \cdot e^{C} \\
    y &= \pm e^{C}x^{4} \\
    y &= C x^{4} \\
  \end{align*}
  Now we have to check our answer.
  \begin{align*}
    \frac{dy}{dx} &= 4 C x^{3} \\
    x \left( 4 C x^{3} \right) &= 4y \\
    4 x^{4} &= 4y \text{, } C = 1 \\
  \end{align*}
\end{example}
\begin{example}[]{Separable Ordinary Differential Equation-Example 2}
  Solve \[ \frac{dP}{dt} = P \left( 1-P \right) \]

  \tcblower

  \begin{align*}
    \frac{1}{P \left( 1-P \right)} dP &= dt \\
    \int \frac{1}{P} + \frac{1}{1-P} dP &= dt \\
    \ln \left( P \right) - \ln \left( 1-P \right) &= t + C \\
    \ln \left( \frac{P}{1-P} \right) &= e^{t+C} \\
    \frac{P}{1-P} &= C e^{t} \\
    P &= C e^{t} \left( 1-P \right) \\
    P + PC e^{t} &= Ce^{t} \\
    P \left( 1+ Ce^{t} \right) &= C e^{t} \\
    P \left( t \right) &= \frac{Ce^{t}}{1+Ce^{t}}
  \end{align*}
\end{example}
\begin{example}[]{Application of Newton's Law of Cooling/Heating}
  Find the function for the constant for \nameref{def:Newton Law of Cooling/Heating}, where $k = -2$ and the temperature of the surrounding medium is $T_{m} = 70$.

  \tcblower

  \begin{align*}
    \frac{dT}{dt} &= k \left( T - T_{m} \right) \\
    \frac{dT}{dt} &= -2 \left( T - 70 \right) \\
    \frac{1}{T-70} dT &= -2 dt \\
    \ln \lvert T-70 \rvert &= -2t + C \\
    \lvert T-70 \rvert &= e^{C}e^{-2t} \\
    T-70 &= \pm e^{C}e^{-2t} \\
    T-70 &= Ce^{-2t} \\
    T \left( t \right) &= Ce^{-2t} +70 \\
    T \left( 0 \right) &= C e^{0} +70 \\
    T \left( 0 \right) &= C + 70 \\
    C &= T \left( 0 \right) -70 \\
  \end{align*}
\end{example}

\subsection{Linear Differential Equations} \label{subsec:Linear ODEs}
\begin{definition}[Linear Differential Equation] \label{def:Linear ODE}
  A \emph{linear differential equation} is one that satisfies the below equation.

  \begin{equation} \label{eq:Linear ODE Equation}
    \begin{aligned}
      a_{1}(x) \frac{dy}{dx} + a_{0}(x)y &= g(x) \\
      a_{1}(x) y' + a_{0}(x)y &= g(x) \\
    \end{aligned}
  \end{equation}

  This can be ``simplified'' down to:
  \begin{equation*}
    \begin{aligned}
      y' + \frac{a_{0}(x)}{a_{1}(x)} y &= \frac{g(x)}{a_{1}(x)} \\
      y' + P(x) y &= f(x) \\
    \end{aligned}
  \end{equation*}

  It will have an integrating factor of:
  \begin{equation} \label{eq:Linear ODE Integrating Factor}
    I(x) = e^{\int \frac{a_{0}(x)}{a_{1}(x)} dx}
  \end{equation}

  If we plug the integrating factor into \Cref{eq:Linear ODE Equation}, the we have:
    \begin{equation*}
    I(x) \frac{dy}{dx} + I(x) \frac{a_{0}(x)}{a_{1}(x)} y = I(x) f(x)
  \end{equation*}

  But, using the \nameref{def:Chain Rule} you get
  \begin{equation*}
    \begin{aligned}
      \left( I(x) y \right)' = I'(x)P(x) + I(x)P'(x) \\
      I'(x) y &= I(x) P(x) y + I(x) y' \\
      I'(x) = I(x) P(x) \\
    \end{aligned}
  \end{equation*}

  This means that the solution to a linear differential equation is simplified to
  \begin{equation} \label{eq:Linear ODE Solution}
    \left( I(x) y \right)' = I(x) f(x)
  \end{equation}

  \begin{remark}
    Each of the terms in \Cref{eq:Linear ODE Equation} can be functions that accept one parameter.
  \end{remark}
\end{definition}

\begin{example}[]{Linear Differential Equation-Example 1}
  Solve the differential equation:
  \begin{equation*}
    \frac{dy}{dx} = 0.2 xy
  \end{equation*}

  \tcblower

\end{example}

\begin{example}[]{Linear Differential Equation-Example 2}
  Solve the differential equation:
  \begin{equation*}
    \frac{dA}{dt} = 6 - \frac{1}{100} A
  \end{equation*}

  \tcblower

\end{example}

\begin{example}[]{Linear Differential Equation-Example 3}
  Solve the differential equation:
  \begin{equation*}
    y' + 3x^{2} y = x^{2}
  \end{equation*}

  \tcblower

\end{example}

\subsection{Exact Differential Equations} \label{subsec:Exact ODEs}
\begin{definition}[Exact Differential Equation] \label{def:Exact ODE}
  An \emph{exact differential equation} is one defined \Cref{eq:Exact ODE}.

  \begin{equation} \label{eq:Exact ODE}
    M(x,y) \, dx + N(x,y) \, dy = 0
  \end{equation}

  For a differential equation to be \emph{exact} there are 2 criteria:
  \begin{enumerate}[noitemsep, nolistsep]
    \item $\frac{\partial M}{\partial y} = \frac{\partial N}{\partial x}$
    \item There exists a function of $f(x,y)$ such that:
      \begin{enumerate}[noitemsep, nolistsep]
        \item $\frac{\partial f}{\partial x} = M(x,y) \rightarrow f(x,y) = \int M(x,y) \, dx$
        \item $\frac{\partial f}{\partial y} = N(x,y) \rightarrow f(x,y) = \int N(x,y) \, dy$
      \end{enumerate}
  \end{enumerate}
\end{definition}

%%% Local Variables:
%%% mode: latex
%%% TeX-master: "../Math_252-DiffEq-Reference_Sheet"
%%% End: % Section 2

\section{Solving Differential Equations with Laplace Transforms}\label{sec:Solve_Diff_Eqns_Laplace}
\begin{example}[]{Solve Differential Equation with Laplace Transform 1}
  Given the ODE
  \begin{equation*}
    y''(t) + 4y(t) = \delta(t-\pi) \; y(0) = 0, \, y'(0) = 0
  \end{equation*}

  \tcblower

  \begin{align*}
    \Lapl \lbrace y''(t) \rbrace + \Lapl \lbrace 4 y(t) \rbrace &= \Lapl \lbrace \delta (t - \pi) \\
    \left(  -y'(0) - sy(0) + s^{2}Y(s) \right) + 4Y(s) &= e^{-s \pi} \\
    \left( s^{2}Y(s) - 0 - s(0) \right) + 4Y(s) &= e^{-s \pi} \\
    Y(s) \left( s^{2} + 4 \right) &= e^{-s \pi} \\
    Y(s) &= \frac{e^{-s \pi}}{s^{2}+4} \\
    y(t) &= \Lapl^{-1} \biggl\lbrace \frac{e^{-s \pi}}{s^{2}+4} \biggr\rbrace = \biggl\lbrace \frac{e^{-s \pi}}{s^{2}+4} + \frac{0}{s^{2}+4} \biggr\rbrace
  \end{align*}
  Now using our Laplace Transform table we receive our answer.
  \begin{equation*}
    y(t) = 0 + \UnitStep(t - \pi) \frac{1}{2}\sin \left( 2 \left( t - \pi \right) \right)
  \end{equation*}
\end{example}

\begin{example}[]{Solve Differential Equation with Laplace Transform 2}
  Given the ODE
  \begin{equation*}
    y''(t) + 2y'(t) + 2y(t) = \delta (t - \pi) \; y(0) = 1, \, y'(0) = 0
  \end{equation*}

  \tcblower

  \begin{align*}
    \Lapl \lbrace y''(t) \rbrace + \Lapl \lbrace 2y'(t) \rbrace + \Lapl \lbrace 2 y(t) \rbrace &= \Lapl \lbrace \delta (t - \pi) \\
    \Lapl \lbrace y''(t) \rbrace + 2 \Lapl \lbrace y'(t) \rbrace + 2 \Lapl \lbrace y(t) \rbrace &= \Lapl \lbrace \delta (t - \pi) \\
    \left( -y'(0) -sy(0) + s^{2}Y(s) \right) + 2 \left( -y(0) + sY(s) \right) + 2Y(s) &= e^{-s \pi} \\
    \left( -0 - 1s + s^{2}Y(s) \right) + 2 \left( -1 + sY(s) \right) + 2Y(s) &= e^{-s \pi} \\
    \left( -s + s^{2}Y(s) \right) + -2 + 2sY(s) + 2Y(s) &= e^{-s \pi} \\
    s^{2}Y(s) + 2sY(s) + 2Y(s) &= e^{-s \pi} + s + 2 \\
    Y(s) \left( s^{2} + 2s + 2 \right) &= e^{-s \pi} + s + 2 \\
    Y(s) &= \frac{e^{-s \pi} + s + 2}{s^{2} + 2s + 2} \\
    Y(s) &= \frac{e^{-s \pi} + s + 2}{{(s + 1)}^{2} + 1}
  \end{align*}
  Now we perform partial fraction decomposition and receive
  \begin{align*}
    Y(s) &= \frac{e^{-s \pi}}{{(s + 1)}^{2} + 1} + \frac{s+1}{{(s + 1)}^{2} + 1} + \frac{1}{{(s + 1)}^{2} + 1} \\
    y(t) &= \Lapl \Biggl\lbrace \frac{e^{-s \pi}}{{(s + 1)}^{2} + 1} \Biggr\rbrace + \Lapl \Biggl\lbrace \frac{s+1}{{(s + 1)}^{2} + 1} \Biggr\rbrace + \Lapl \Biggl\lbrace \frac{1}{{(s + 1)}^{2} + 1} \Biggr\rbrace
  \end{align*}
  Now, using our handy-dandy Laplace Transform table, we receive our answer.
  \begin{equation*}
    y(t) = \UnitStep(t-\pi)e^{-(t-\pi)}\sin(t-\pi) + e^{-t}\cos(t-\pi) + e^{-t}\sin(t-\pi)
  \end{equation*}
\end{example}

%%% Local Variables:
%%% mode: latex
%%% TeX-master: "../Math_252-DiffEq-Reference_Sheet"
%%% End:

%====================================APPENDIX====================================
\appendix
\counterwithin{equation}{section}
\counterwithin{definition}{subsection}

\clearpage
\subsection{Trigonometry} \label{app:Trig}
	\subsubsection{Trigonometric Formulas} \label{subsubsec:Trig Formulas}
		\begin{equation} \label{eq:Sin plus Sin with diff Angles}
			\sin \left( \alpha \right) + \sin \left( \beta \right) = 2 \sin \left( \frac{\alpha + \beta}{2} \right) \cos\left( \frac{\alpha - \beta}{2} \right)  
		\end{equation}
		\begin{equation} \label{eq:Cosine-Sine Product}
			\cos \left( \theta \right) \sin \left( \theta \right) = \frac{1}{2} \sin \left( 2 \theta \right)
		\end{equation}
	
	\subsubsection{Euler Equivalents of Trigonometric Functions} \label{subsubsec:Euler Equivalents}
		\begin{equation} \label{eq:Euler Sin}
			\sin \left( x \right) = \frac{e^{\imath x} + e^{-\imath x}}{2}
		\end{equation}
		\begin{equation} \label{eq:Euler Cos}
			\cos \left( x \right) = \frac{e^{\imath x} - e^{-\imath x}}{2 \imath}
		\end{equation}
		\begin{equation} \label{eq:Euler Sinh}
			\sinh \left( x \right) = \frac{e^{x} - e^{-x}}{2}
		\end{equation}
		\begin{equation} \label{eq:Euler Cosh}
			\cosh \left( x \right) = \frac{e^{x} + e^{-x}}{2}
		\end{equation}

\clearpage
\section{Calculus}\label{app:Calculus}
\subsection{L'Hopital's Rule}\label{subsec:LHopitals_Rule}
L'Hopital's Rule can be used to simplify and solve expressions regarding limits that yield irreconcialable results.
\begin{lemma}[L'Hopital's Rule]\label{lemma:LHopitals_Rule}
  If the equation
  \begin{equation*}
    \lim\limits_{x \rightarrow a} \frac{f(x)}{g(x)} =
    \begin{cases}
      \frac{0}{0} \\
      \frac{\infty}{\infty} \\
    \end{cases}
  \end{equation*}
  then \Cref{eq:LHopitals_Rule} holds.
  \begin{equation}\label{eq:LHopitals_Rule}
    \lim\limits_{x \rightarrow a} \frac{f(x)}{g(x)} = \lim\limits_{x \rightarrow a} \frac{f'(x)}{g'(x)}
  \end{equation}
\end{lemma}

\subsection{Fundamental Theorems of Calculus}\label{subsec:Fundamental Theorem of Calculus}
\begin{definition}[First Fundamental Theorem of Calculus]\label{def:1st Fundamental Theorem of Calculus}
  The \emph{first fundamental theorem of calculus} states that, if $f$ is continuous on the closed interval $\left[ a,b \right]$ and $F$ is the indefinite integral of $f$ on $\left[ a,b \right]$, then

  \begin{equation}\label{eq:1st Fundamental Theorem of Calculus}
    \int_{a}^{b}f \left( x \right) dx = F \left( b \right) - F \left( a \right)
  \end{equation}
\end{definition}

\begin{definition}[Second Fundamental Theorem of Calculus]\label{def:2nd Fundamental Theorem of Calculus}
  The \emph{second fundamental theorem of calculus} holds for $f$ a continuous function on an open interval $I$ and $a$ any point in $I$, and states that if $F$ is defined by

  \begin{equation*}
    F \left( x \right) = \int_{a}^{x} f \left( t \right) dt,
  \end{equation*}
  then
  \begin{equation}\label{eq:2nd Fundamental Theorem of Calculus}
    \begin{aligned}
      \frac{d}{dx} \int_{a}^{x} f \left( t \right) dt &= f \left( x \right) \\
      F' \left( x \right) &= f \left( x \right) \\
    \end{aligned}
  \end{equation}
\end{definition}

\begin{definition}[argmax]\label{def:argmax}
  The arguments to the \emph{argmax} function are to be maximized by using their derivatives.
  You must take the derivative of the function, find critical points, then determine if that critical point is a global maxima.
  This is denoted as
  \begin{equation*}\label{eq:argmax}
    \argmax_{x}
  \end{equation*}
\end{definition}

\subsection{Rules of Calculus}\label{subsec:Rules of Calculus}
\subsubsection{Chain Rule}\label{subsubsec:Chain Rule}
\begin{definition}[Chain Rule]\label{def:Chain Rule}
  The \emph{chain rule} is a way to differentiate a function that has 2 functions multiplied together.

  If
  \begin{equation*}
    f(x) = g(x) \cdot h(x)
  \end{equation*}
  then,
  \begin{equation}\label{eq:Chain Rule}
    \begin{aligned}
      f'(x) &= g'(x) \cdot h(x) + g(x) \cdot h'(x) \\
      \frac{df(x)}{dx} &= \frac{dg(x)}{dx} \cdot g(x) + g(x) \cdot \frac{dh(x)}{dx} \\
    \end{aligned}
  \end{equation}
\end{definition}

\subsection{Useful Integrals}\label{subsec:Useful_Integrals}
\begin{equation}\label{eq:Cosine_Indefinite_Integral}
  \int \cos(x) \; dx = \sin(x)
\end{equation}

\begin{equation}\label{eq:Sine_Indefinite_Integral}
  \int \sin(x) \; dx = -\cos(x)
\end{equation}

\begin{equation}\label{eq:x_Cosine_Indefinite_Integral}
  \int x \cos(x) \; dx = \cos(x) + x \sin(x)
\end{equation}
\Cref{eq:x_Cosine_Indefinite_Integral} simplified with Integration by Parts.

\begin{equation}\label{eq:x_Sine_Indefinite_Integral}
  \int x \sin(x) \; dx = \sin(x) - x \cos(x)
\end{equation}
\Cref{eq:x_Sine_Indefinite_Integral} simplified with Integration by Parts.

\begin{equation}\label{eq:x_Squared_Cosine_Indefinite_Integral}
  \int x^{2} \cos(x) \; dx = 2x \cos(x) + (x^{2} - 2) \sin(x)
\end{equation}
\Cref{eq:x_Squared_Cosine_Indefinite_Integral} simplified by using Integration by Parts twice.

\begin{equation}\label{eq:x_Squared_Sine_Indefinite_Integral}
  \int x^{2} \sin(x) \; dx = 2x \sin(x) - (x^{2} - 2) \cos(x)
\end{equation}
\Cref{eq:x_Squared_Sine_Indefinite_Integral} simplified by using Integration by Parts twice.

\begin{equation}\label{eq:Exponential_Cosine_Indefinite_Integral}
  \int e^{\alpha x} \cos(\beta x) \; dx = \frac{e^{\alpha x} \bigl( \alpha \cos(\beta x) + \beta \sin(\beta x) \bigr)}{\alpha^{2} + \beta^{2}} + C
\end{equation}

\begin{equation}\label{eq:Exponential_Sine_Indefinite_Integral}
  \int e^{\alpha x} \sin(\beta x) \; dx = \frac{e^{\alpha x} \bigl( \alpha \sin(\beta x) - \beta \cos(\beta x) \bigr)}{\alpha^{2}+\beta^{2}} + C
\end{equation}

\begin{equation}\label{eq:Exponential_Indefinite_Integral}
  \int e^{\alpha x} \; dx = \frac{e^{\alpha x}}{\alpha}
\end{equation}

\begin{equation}\label{eq:x_Exponential_Indefinite_Integral}
  \int x e^{\alpha x} \; dx = e^{\alpha x} \left( \frac{x}{\alpha} - \frac{1}{\alpha^{2}} \right)
\end{equation}
\Cref{eq:x_Exponential_Indefinite_Integral} simplified with Integration by Parts.

\begin{equation}\label{eq:Inverse_x_Indefinite_Integral}
  \int \frac{dx}{\alpha + \beta x} = \int \frac{1}{\alpha + \beta x} \; dx = \frac{1}{\beta} \ln (\alpha + \beta x)
\end{equation}

\begin{equation}\label{eq:Inverse_x_Squared_Indefinite_Integral}
  \int \frac{dx}{\alpha^{2} + \beta^{2} x^{2}} = \int \frac{1}{\alpha^{2} + \beta^{2} x^{2}} \; dx = \frac{1}{\alpha \beta} \arctan \left( \frac{\beta x}{\alpha} \right)
\end{equation}

\begin{equation}\label{eq:a_Exponential_Indefinite_Integral}
  \int \alpha^{x} \; dx = \frac{\alpha^{x}}{\ln(\alpha)}
\end{equation}

\begin{equation}\label{eq:a_Exponential_Derivative}
  \frac{d}{dx} \alpha^{x} = \frac{d\alpha^{x}}{dx} = \alpha^{x} \ln(x)
\end{equation}

\subsection{Leibnitz's Rule}\label{subsec:Leibnitzs_Rule}
\begin{lemma}[Leibnitz's Rule]\label{lemma:Leibnitzs_Rule}
  Given
  \begin{equation*}
    g(t) = \int_{a(t)}^{b(t)} f(x, t) \, dx
  \end{equation*}
  with $a(t)$ and $b(t)$ differentiable in $t$ and $\frac{\partial f(x, t)}{\partial t}$ continuous in both $t$ and $x$, then
  \begin{equation}\label{eq:Leibnitzs_Rule}
    \frac{d}{dt} g(t) = \frac{d g(t)}{dt} = \int_{a(t)}^{b(t)} \frac{\partial f(x, t)}{\partial t} \, dx + f \bigl[ b(t), t \bigr] \, \frac{d b(t)}{dt} - f \bigl[ a(t), t \bigr] \, \frac{d a(t)}{dt}
  \end{equation}
\end{lemma}



\clearpage
\section{Laplace Transform}\label{app:Laplace_Transform}
\subsection{Laplace Transform}\label{subsec:Laplace_Transform}
\begin{definition}[Laplace Transform]\label{def:Laplace_Transform}
  The \emph{Laplace transformation} operation is denoted as $\Lapl \lbrace x(t) \rbrace$ and is defined as
  \begin{equation}\label{eq:Laplace_Transform}
    X(s) = \int\limits_{-\infty}^{\infty} x(t) e^{-st} dt
  \end{equation}
\end{definition}

\subsection{Inverse Laplace Transform}\label{subsec:Inverse_Laplace_Transform}
\begin{definition}[Inverse Laplace Transform]\label{def:Inverse_Laplace_Transform}
  The \emph{inverse Laplace transformation} operation is denoted as $\Lapl^{-1} \lbrace X(s) \rbrace$ and is defined as
  \begin{equation}\label{eq:Inverse_Laplace_Transform}
    x(t) = \frac{1}{2j \pi} \int_{\sigma-\infty}^{\sigma+\infty} X(s) e^{st} \, ds
  \end{equation}
\end{definition}

\subsection{Properties of the Laplace Transform}\label{subsec:Laplace_Transform_Properties}
\subsubsection{Linearity}\label{subsubsec:Laplace_Linearity}
The \nameref{def:Laplace_Transform} is a linear operation, meaning it obeys the laws of linearity.
This means \Cref{eq:Laplace_Linearity} must hold.
\begin{subequations}\label{eq:Laplace_Linearity}
  \begin{equation}\label{eq:Laplace_Linearity_Time}
    x(t) = \alpha_{1} x_{1}(t) + \alpha_{2} x_{2}(t)
  \end{equation}
  \begin{equation}\label{eq:Laplace_Linearity_Frequency}
    X(s) = \alpha_{1} X_{1}(s) + \alpha_{2} X_{2}(s)
  \end{equation}
\end{subequations}

\subsubsection{Time Scaling}\label{subsubsec:Laplace_Time_Scaling}
Scaling in the time domain (expanding or contracting) yields a slightly different transform.
However, this only makes sense for $\alpha > 0$ in this case.
This is seen in \Cref{eq:Laplace_Time_Scaling}.
\begin{equation}\label{eq:Laplace_Time_Scaling}
  \Lapl \bigl\lbrace x(\alpha t) \bigr\rbrace = \frac{1}{\alpha} X \left( \frac{s}{\alpha} \right)
\end{equation}

\subsubsection{Time Shift}\label{subsubsec:Laplace_Time_Shift}
Shifting in the time domain means to change the point at which we consider $t=0$.
\Cref{eq:Laplace_Time_Shifting} below holds for shifting both forward in time and backward.
\begin{equation}\label{eq:Laplace_Time_Shifting}
  \Lapl \bigl\lbrace x(t-a) \bigr\rbrace = X(s) e^{-a s}
\end{equation}

\subsubsection{Frequency Shift}\label{subsubsec:Laplace_Frequency_Shift}
Shifting in the frequency domain means to change the complex exponential in the time domain.
\begin{equation}\label{eq:Laplace_Frequency_Shift}
  \Lapl^{-1} \bigl\lbrace X(s-a) \bigr\rbrace = x(t)e^{at}
\end{equation}

\subsubsection{Integration in Time}\label{subsubsec:Laplace_Time_Integration}
Integrating in time is equivalent to scaling in the frequency domain.
\begin{equation}\label{eq:Laplace_Time_Integration}
  \Lapl \left\lbrace \int_{0}^{t} x(\lambda) \, d\lambda \right\rbrace = \frac{1}{s} X(s)
\end{equation}

\subsubsection{Frequency Multiplication}\label{subsubsec:Laplace_Frequency_Multiplication}
Multiplication of two signals in the frequency domain is equivalent to a convolution of the signals in the time domain.
\begin{equation}\label{eq:Laplace_Frequency_Multiplication}
  \Lapl \bigl\lbrace x(t) * v(t) \bigr\rbrace = X(s) V(s)
\end{equation}

\subsubsection{Relation to Fourier Transform}\label{subsubsec:Fourier_Transform_Relation}
The Fourier transform looks and behaves very similarly to the Laplace transform.
In fact, if $X(\omega)$ exists, then \Cref{eq:Fourier_Laplace_Transform_Relation} holds.
\begin{equation}\label{eq:Fourier_Laplace_Transform_Relation}
  X(s) = X(\omega) \vert_{\omega = \frac{s}{j}}
\end{equation}

\subsection{Theorems}\label{subsec:Laplace_Theorems}
There are 2 theorems that are most useful here:
\begin{enumerate}[noitemsep]
\item \nameref{thm:Laplace_Initial_Value_Theorem}
\item \nameref{thm:Laplace_Final_Value_Theorem}
\end{enumerate}


%%% Local Variables:
%%% mode: latex
%%% TeX-master: shared
%%% End:


\end{document}

%%% Local Variables:
%%% mode: latex
%%% TeX-master: t
%%% End:
