\documentclass[10pt,letterpaper,final,twoside,notitlepage]{article}
\usepackage[margin=.5in]{geometry} % 1/2 inch margins on all pages
\usepackage[utf8]{inputenc} % Define the input encoding
\usepackage[USenglish]{babel} % Define language used
\usepackage{amsmath,amsfonts,amssymb}
\usepackage{amsthm} % Gives us plain, definition, and remark to use in \theoremstyle{style}
\usepackage{mathtools} % Allow for text and math in align* environment.
\usepackage{thmtools}
\usepackage{thm-restate}
\usepackage{graphicx}

\usepackage[
backend=biber,
style=alphabetic,
citestyle=authoryear]{biblatex} % Must include citation somewhere in document to print bibliography
\usepackage{hyperref} % Generate hyperlinks to referenced items
\usepackage[nottoc]{tocbibind} % Prints the Reference/Bibliography in TOC as well
\usepackage[noabbrev,nameinlink]{cleveref} % Fancy cross-references in the document everywhere
\usepackage{nameref} % Can make references by name to places
\usepackage{caption} % Allows for greater control over captions in figure, algorithm, table, etc. environments
\usepackage{subcaption} % Allows for multiple figures in one Figure environment
\usepackage[binary-units=true]{siunitx} % Gives us ways to typeset units for stuff
\usepackage{csquotes} % Context-sensitive quotation facilities
\usepackage{enumitem} % Provides [noitemsep, nolistsep] for more compact lists
\usepackage{chngcntr} % Allows us to tamper with the counter a little more
\usepackage{empheq} % Allow boxing of equations in special math environments
\usepackage[x11names]{xcolor} % Gives access to coloring text in environments or just text, MUST be before tikz
\usepackage{tcolorbox} % Allows us to create boxes of various types for examples
\usepackage{tikz} % Allows us to create TikZ and PGF Pictures
\usepackage{ctable} % Greater control over tables and how they look
\usepackage{diagbox} % Allow us to have shared diagonal cells in tables
\usepackage{multirow} % Allow us to have a single cell in a table span multiple rows
\usepackage{titling} % Put document information throughout the document programmatically
\usepackage[linesnumbered,ruled,vlined]{algorithm2e} % Allows us to write algorithms in a nice style.

\counterwithin{figure}{section}
\counterwithin{table}{section}
\counterwithin{equation}{section}
\counterwithin{algocf}{section}
\crefname{algocf}{algorithm}{algorithms}
\Crefname{algocf}{Algorithm}{Algorithms}
\setcounter{secnumdepth}{4}
\setcounter{tocdepth}{4} % Include \paragraph in toc
\crefname{paragraph}{paragraph}{paragraphs}
\Crefname{paragraph}{Paragraph}{Paragraphs}

% Create a theorem environment
\theoremstyle{plain}
\newtheorem{theorem}{Theorem}[section]
% Create a numbered theorem-like environment for lemmas
\newtheorem{lemma}{Lemma}[theorem]

% Create a definition environment
\theoremstyle{definition}
\newtheorem{definition}{Defn}
\newtheorem{corollary}{Corollary}[section]
% \begin{definition}[Term] \label{def:}
%   Make sure the term is emphasized with \emph{term}.
%   This ensures that if \emph is changed, it shows up everywhere
% \end{definition}

% Create a numbered remark environment numbered based on definition
% NOTE: This version of remark MUST go inside a definition environment
\theoremstyle{remark}
\newtheorem{remark}{Remark}[definition]
%\counterwithin{definition}{subsection} % Uncomment to have definitions use section.subsection numbering

% Create an unnumbered remark environment for general use
% NOTE: This version of remark has NO restrictions on placement
\newtheorem*{remark*}{Remark}

% Create a special list that handles properties. It can be broken and restarted
\newlist{propertylist}{enumerate}{1} % {Name}{Template}{Max Depth}
% [newlistname, LevelsToApplyTo]{formatting options}
\setlist[propertylist, 1]{label=\textbf{(\roman*)}, ref=\textbf{(\roman*)}, noitemsep, nolistsep}
\crefname{propertylisti}{property}{properties}
\Crefname{propertylisti}{Property}{Properties}

% Create a special list that handles enumerate starting with lower letters. Breakable/Restartable.
\newlist{boldalphlist}{enumerate}{1} % {Name}{Template}{Max Depth}
% [newlistname, LevelsToApplyTo]{formatting options}
\setlist[boldalphlist, 1]{label=\textbf{(\alph*)}, ref=\alph*, noitemsep, nolistsep} % Set options

\newlist{nocrefenumerate}{enumerate}{1} % {Name}{Template}{Max Depth}
% [newlistname, LevelsToApplyTo]{formatting options}
\setlist[nocrefenumerate, 1]{label=(\arabic*), ref=(\arabic*), noitemsep, nolistsep}

% Create a list that allows for deeper nesting of numbers. By default enumerate only allows depth=4.
\newlist{nestednums}{enumerate}{6}
% [newlistname, LevelsToApplyTo]{formatting options}
\setlist[nestednums]{noitemsep, label*=\arabic*.}

\tcbuselibrary{breakable} % Allow tcolorboxes to be broken across pages
% Create a tcolorbox for examples
% /begin{example}[extra name]{NAME}
% Create a tcolorbox for examples
% Argument #1 is optional, given by [], that is the textbook's problem number
% Argument #2 is mandatory, given by {}, that is the title for the example
% Avoid putting special characters, (), [], {}, ",", etc. in the title.
\newtcolorbox[auto counter,
number within=section,
number format=\arabic,
crefname={example}{examples}, % Define reference format for cref (No Capitals)
Crefname={Example}{Examples}, % Reference format for cleveref (With Capitals)
]{example}[2][]{ % The [2][] Means the first argument is optional
  width=\textwidth,
  title={Example \thetcbcounter: #2. #1}, % Parentheses and commas are not well supported
  fonttitle=\bfseries,
  label={ex:#2},
  nameref=#2,
  colbacktitle=white!100!black,
  coltitle=black!100!white,
  colback=white!100!black,
  upperbox=visible,
  lowerbox=visible,
  sharp corners=all,
  breakable
}

% Create a tcolorbox for general use
\newtcolorbox[% auto counter,
% number within=section,
% number format=\arabic,
% crefname={example}{examples}, % Define reference format for cref (No Capitals)
% Crefname={Example}{Examples}, % Reference format for cleveref (With Capitals)
]{blackbox}{
  width=\textwidth,
  % title={},
  fonttitle=\bfseries,
  % label={},
  % nameref=,
  colbacktitle=white!100!black,
  coltitle=black!100!white,
  colback=white!100!black,
  upperbox=visible,
  lowerbox=visible,
  sharp corners=all
}

% Redefine the 'end of proof' symbol to be a black square, not blank
\renewcommand{\qedsymbol}{$\blacksquare$} % Change proofs to have black square at end

% Common Mathematical Stuff
\newcommand{\Abs}[1]{\ensuremath{\lvert #1 \rvert}}
\newcommand{\DNE}{\ensuremath{\mathrm{DNE}}} % Used when limit of function Does Not Exist

% Complex Numbers functions
\renewcommand{\Re}{\operatorname{Re}} % Redefine to use the command, but not the fraktur version
\renewcommand{\Im}{\operatorname{Im}} % Redefine to use the command, but not the fraktur version
\newcommand{\Real}[1]{\ensuremath{\Re \lbrace #1 \rbrace}}
\newcommand{\Imag}[1]{\ensuremath{\Im \lbrace #1 \rbrace}}
\newcommand{\Conjugate}[1]{\ensuremath{\overline{#1}}}
\newcommand{\Modulus}[1]{\ensuremath{\lvert #1 \rvert}}
\DeclareMathOperator{\PrincipalArg}{\ensuremath{Arg}}

% Math Operators that are useful to abstract the written math away to one spot
% Number Sets
\DeclareMathOperator{\RealNumbers}{\ensuremath{\mathbb{R}}}
\DeclareMathOperator{\AllIntegers}{\ensuremath{\mathbb{Z}}}
\DeclareMathOperator{\PositiveInts}{\ensuremath{\mathbb{Z}^{+}}}
\DeclareMathOperator{\NegativeInts}{\ensuremath{\mathbb{Z}^{-}}}
\DeclareMathOperator{\NaturalNumbers}{\ensuremath{\mathbb{N}}}
\DeclareMathOperator{\ComplexNumbers}{\ensuremath{\mathbb{C}}}
\DeclareMathOperator{\RationalNumbers}{\ensuremath{\mathbb{Q}}}

% Calculus operators
\DeclareMathOperator*{\argmax}{argmax} % Thin Space and subscripts are UNDER in display

% Signal and System Functions
\DeclareMathOperator{\UnitStep}{\mathcal{U}}
\DeclareMathOperator{\sinc}{sinc} % sinc(x) = (sin(pi x)/(pi x))

% Transformations
\DeclareMathOperator{\Lapl}{\mathcal{L}} % Declare a Laplace symbol to be used

% Logical Operators
\DeclareMathOperator{\XOR}{\oplus}

% x86 CPU Registers
\newcommand{\rbpRegister}{\texttt{\%rbp}}
\newcommand{\rspRegister}{\texttt{\%rsp}}
\newcommand{\ripRegister}{\texttt{\%rip}}
\newcommand{\raxRegister}{\texttt{\%rax}}
\newcommand{\rbxRegister}{\texttt{\%rbx}}

%%% Local Variables:
%%% mode: latex
%%% TeX-master: shared
%%% End:


% These packages are more specific to certain documents, but will be availabe in the template
% \usepackage{esint} % Provides us with more types of integral symbols to use
\usepackage[outputdir=./TeX_Output]{minted} % Allow us to nicely typeset 300+ programming languages
\crefname{lstlisting}{listing}{listings}
\Crefname{lstlisting}{Listing}{Listings}
% This document must be compiled with the -shell-escape flag if the packages above are uncommented

% \graphicspath{{./Drawings/Course}} % Uncomment this to use pictures in this document
% \addbibresource{./Bibliographies/CourseNum-Name.bib}

\usemintedstyle{emacs} % Best for use on white backgrounds. Use for inline-code
% This macro creates the 2 minted environments for kernel source code with common options
\def\mintedkernelargs{
  frame=lines, % Surround the source code with lines on top and bottom
  linenos, % We want to show line numbers for each line in the margin
  % Colors used here are xcolor X11 colors.
  % style=fruity, % Use the fruity color scheme. Best for use on black backgrounds. Use for code blocks.
  % bgcolor=black, % Set the background used
  style=emacs,
  bgcolor=white,
  autogobble=true, % Automatically remove shared indentation from files
  breaklines=true, % Break lines that are too long at convenient locations
}
\newcommand{\makenewmintedfiles}[1]{
  \newminted[csource]{c}{#1} % Use with \begin{csource} code \end{csource}

  \newmintedfile[csourcefile]{c}{#1} % Use with \csourcefile[additional-options]{Filename}
}
\expandafter\makenewmintedfiles\expandafter{\mintedkernelargs}
\newmintinline[cinline]{c}{% Use with \cinline{code}
  style=emacs,
  bgcolor=white,
}

% Math Operators that are useful to abstract the written math away to one spot
% These are supposed to be document-specific mathematical operators that will make life easier
% Many fundamental operators are defined in Reference_Sheet_Preamble.tex

\begin{titlepage}
  \title{CS 351: Systems Programming --- Reference Material \\ Illinois Institute of Technology}
  \author{Karl Hallsby}
  \date{Last Edited: \today} % We want to inform people when this document was last edited
\end{titlepage}

\begin{document}
\pagenumbering{gobble}
\maketitle
\pagenumbering{roman} % i, ii, iii on beginning pages, that don't have content
\tableofcontents
\clearpage
\listoftheorems[ignoreall, show={definition, Definition}]
\clearpage
\listoflistings{}
\clearpage
\pagenumbering{arabic} % 1,2,3 on content pages

\section{C Programming}\label{sec:C_Programming}
C is one of the lowest ``high level'' languages you can use today.
It provides very minimal abstractions from hardware and assembly code, but allows you to relatively good typechecked code.

\subsection{Memory Allocation}\label{subsec:Memory_Allocation}
Because C is a language that does not provide many abstractions, it also requires the programmer to remember and manage their memory usage.
So, \textbf{YOU} must be the one to manage the memory, there is \textbf{NO} built-in garbage collector for you to use.

Memory allocation is done on the heap of the program's execution space in memory.
When you allocate memory in your program, you are actually requesting the operating system to give you the memory you want.

\subsubsection{\texttt{malloc}}\label{subsubec:malloc}
This is the simplest function of all possible memory allocation functions.
\texttt{malloc}:
\begin{itemize}
\item Takes one argument:
  \begin{enumerate}
  \item The number of bytes to allocate.
  \end{enumerate}
\item Returns a \textbf{POINTER} to the front of the allocated memory.
\end{itemize}

\texttt{malloc} {\large{\textbf{\emph{DOES NOT}}}} initialize memory, so it will be garbage.

\subsubsection{\texttt{calloc}}\label{subsubsec:calloc}
This is quite similar to malloc.
\texttt{calloc}:
\begin{itemize}
\item Takes 2 arguments:
  \begin{enumerate}
  \item The number of spaces to allocate, for example the number of elements in an array.
  \item The number of bytes to allocate, for the type being stored.
  \end{enumerate}
\item Returns a \textbf{POINTER} to the front of the allocated memory.
\end{itemize}

\texttt{calloc} {\large{\textbf{\emph{ZEROS}}}} memory, so this does have a slight performance penalty.

\subsubsection{\texttt{realloc}}\label{subsubsec:realloc}
\texttt{realloc} is used to \textbf{REALLOCATE} an existing memory location.
\begin{itemize}
\item Takes 2 arguments:
  \begin{enumerate}
  \item The pointer to the memory location previously allocated with either \texttt{malloc} or \texttt{calloc}.
  \item The amount of memory to reallocate, in bytes.
  \end{enumerate}
\item If the \texttt{NULL} pointer is passed to \texttt{realloc}, it will behave exactly like \texttt{malloc}.
\item Returns a \textbf{POINTER} to the front of the reallocated memory
\end{itemize}

\subsubsection{\texttt{free}}\label{subsubsec:free}
\texttt{free} is used to free memory that was previously allocated, removing from the programming space entirely.
\begin{itemize}
\item Takes 1 argument:
  \begin{enumerate}
  \item A pointer to the memory to be deallocated.
  \end{enumerate}
\item Returns \texttt{void}.
\end{itemize}

%%% Local Variables:
%%% mode: latex
%%% TeX-master: "../C_Programming"
%%% End:


%%% Local Variables:
%%% mode: latex
%%% TeX-master: shared
%%% End:


\section{Processes}\label{sec:Processes}
\nameref{def:Process}es are the fundamental unit of computation within an operating system.
\begin{definition}[Process]\label{def:Process}
  A \emph{process} is a \nameref{def:Program} in execution.
  A process carries out the computation that we specify.
  A process contains:
  \begin{itemize}[noitemsep]
  \item Code (\texttt{text}) of your program.
  \item Runtime data (Global, local, dynamic variables)
  \item \nameref{def:Register}s:
    \begin{itemize}[noitemsep]
    \item \nameref{def:Program_Counter} (\texttt{PC})
    \item \nameref{def:Stack_Pointer} (\texttt{SP})
    \item \nameref{def:Frame_Pointer} (\texttt{FP})
    \end{itemize}
  \item \nameref{def:Process_Control_Block}
  \end{itemize}
\end{definition}

\begin{definition}[Program]\label{def:Program}
  A \emph{program} is the binary image stored at some file location on the storage medium.
  The program is read into memory, and then is used to start a \nameref{def:Process} that runs that program.
\end{definition}

\nameref{def:Process}es require both a \textbf{predictable} and \textbf{logical} control flow.
This means:
\begin{itemize}[noitemsep]
\item The \nameref{def:Process} must start somewhere, typically defined to be \texttt{main}.
\item Nothing can disrupt a program mid-execution.
  \begin{itemize}[noitemsep]
  \item This is further discussed in \Cref{subsec:Prevent_Process_Disruption}.
  \end{itemize}
\end{itemize}

\subsection{Prevent Process Disruption}\label{subsec:Prevent_Process_Disruption}
The easiest way to prevent a \nameref{def:Process} from having its control flow being interrupted is for the process to ``own'' the CPU for the entire duration of the process's execution.
However, this means:
\begin{itemize}[noitemsep]
\item No other process can run on this core
\item This prevents efficient multi-\nameref{def:Process}/multitasking systems
\item Malicious or poorly written program can ``take over'' the CPU
\item An idle process (for example, waiting for user input) will underutilize the CPU
\end{itemize}

For the operating system to simulate this seamless logical control flow, we use all of the information used to make a \nameref{def:Process}, and need a \nameref{def:Process_Control_Block}.
\begin{definition}[Process Control Block]\label{def:Process_Control_Block}
  The \emph{Process Control Block} (\emph{PCB}) contains additional metadata about a \nameref{def:Process}.
  This includes:
  \begin{itemize}[noitemsep]
  \item Process ID (\texttt{PID})
  \item CPU Usage
  \item Memory Usage
  \item Pending \nameref{def:Syscall}s
  \end{itemize}
\end{definition}

The \nameref{def:Process_Control_Block} is sued to allow \nameref{def:Process}es to be interrupted, saved, and moved off a core.
This allows the operating system to schedule processes according to some algorithm.

\begin{definition}[Syscall]\label{def:Syscall}
  A \emph{syscall}, short for a \emph{system call} is a way for a user-level \nameref{def:Process} to perform some computation that the operating system kernel restricts.
  Some common syscalls are:
  \begin{itemize}[noitemsep]
  \item Opening a file
  \item Reading a file
  \item Writing a file
  \item Closing a file
  \item Creating a process
  \item Changing the process's binary
  \item Reading from the network
  \item Writing to the network
  \end{itemize}
\end{definition}

\subsection{Required Hardware}\label{subsec:Required_Hardware}
Interrupting the execution of a \nameref{def:Process} requires some hardware support to be possible and efficient.
We need 3 things:
\begin{enumerate}[noitemsep]
\item A hardware mechanism to periodically interrupt the current \nameref{def:Process} to change execution to the operating system.
  \begin{itemize}[noitemsep]
  \item This is usually the \emph{periodic clock interrupt}.
  \end{itemize}
\item An operating system procedure to decide which processes to run, and in which order.
  \begin{itemize}[noitemsep]
  \item This is the operating system's \nameref{def:Process_Scheduler}.
  \end{itemize}
\item A routine for seamlessly switching between processes.
  \begin{itemize}[noitemsep]
  \item This is called a \nameref{def:Context_Switch}.
  \item Relatively speaking, these are expensive to perform.
  \item \textbf{These are external to a \nameref{def:Process}'s logical control flow}.
  \item This forms part of the process of \nameref{def:Exceptional_Control_Flow}.
  \item A \nameref{def:Context_Switch} makes no guarantee about if and/or when this \nameref{def:Process} will start running again.
  \item A \nameref{def:Context_Switch} is the only way to invoke \nameref{def:Syscall}s.
  \end{itemize}
\end{enumerate}

To schedule \nameref{def:Process}es onto one of possibly many CPUs, there are programs called \nameref{def:Process_Scheduler}s.

Every time the \nameref{def:Process_Scheduler} schedules a new \nameref{def:Process}, or when a process needs to perform a \nameref{def:Syscall}, or a kernel-level exception occurs, a \nameref{def:Context_Switch} is made.
\begin{definition}[Context Switch]\label{def:Context_Switch}
  A \emph{context switch} is the process of interrupting a \nameref{def:Process}, saving its current state, and scheduling something else to run on that CPU.\@
  This is done whenever a \nameref{def:Syscall} is made, and happens sometimes during a process's lifetime.
\end{definition}

\nameref{def:Context_Switch}es form a core part of the \nameref{def:Exceptional_Control_Flow}.

\subsection{Process Scheduling}\label{subsec:Process_Scheduling}
Process scheduling is the process by which a process is put ``scheduled'' onto a particular CPU, according to some criteria.
There are any number of scheduling algorithms, each of which optimizes for certain cases, and may yield a different order of process execution.
\begin{definition}[Process Scheduler]\label{def:Process_Scheduler}
  The \emph{process scheduler} is an operating system component that chooses which \nameref{def:Process} to run next.
  It chooses this according to some algorithm, each of which might change the order of process execution.
\end{definition}

One such \nameref{def:Process_Scheduler} is the \nameref{subsubsec:Priority_Scheduling} algorithm.

\subsubsection{Priority Scheduling}\label{subsubsec:Priority_Scheduling}
\emph{Priority scheduling} involves placing a priority on every \nameref{def:Process} that can be scheduled.
Then, the process with the highest priority is chosen first, working our way down to the lowest priority.
This is, partly, the setup most modern operating systems take today.

However, priority scheduling creates new issues that must be dealt with.
One of these is \nameref{def:Starvation}.

\begin{definition}[Starvation]\label{def:Starvation}
  \emph{Starvation} is when something that needs a resource to function does not receive this resource.
\end{definition}

In this case, a lower-priority \nameref{def:Process} can experience \nameref{def:Starvation} if only higher-priority processes are present, and continually steal CPU execution time.

\subsection{Exceptional Control Flow}\label{subsec:Exceptional_Control_Flow}
To illustrate the power of \nameref{def:Exceptional_Control_Flow}, we will use an example piece of code, \Cref{lst:Exceptional_Control_Flow}.

\begin{definition}[Exceptional Control Flow]\label{def:Exceptional_Control_Flow}
  \emph{Exceptional control flow} is designated by the fact that most of the computation involved is done in response to exceptions or special events.
  Which one depends on what ``exceptional'' means in that circumstance.
\end{definition}

\begin{listing}[h!tbp]
\csourcefile{./CS_351-Systems_Programming-Sections/Processes/Code/Exceptional_Control_Flow.c}
\caption{Exceptional Control Flow Example}
\label{lst:Exceptional_Control_Flow}
\end{listing}


%%% Local Variables:
%%% mode: latex
%%% TeX-master: "../CS_351-Systems_Programming-Reference_Sheet"
%%% End:


\section{Process Management}\label{sec:Process_Management}
In almost all modern systems today, there are many, many \nameref{def:Process}es running ``simultaneously''.
That is in quotes because if you have multiple cores/\nameref{def:CPU}s in a single package, you actually \textit{can} run multiple processes at once.
However, we choose to limit our discussion to single core packages, to simplify our discussions and remove a whole class of issues.

By default, there is only one \nameref{def:Process} running at the start of a computer's execution.
We need ways to make more processes, change what \nameref{def:Program}s these processes are executing, and a way to wait for these processes to finish and pick up after them.

\subsection{Making Processes, \texttt{fork}}\label{subsec:Making_Processes-fork}
\texttt{fork} creates a \textbf{copy} of the current \nameref{def:Process}.
This is our \textit{only} method of creating new processes.
The child process is nearly an \textbf{exact} duplicate of the parent process, where only some process metadata in the \nameref{def:Process_Control_Block} is different.
The function prototype for \cinline{fork} is shown in \Cref{lst:PID_Definition_fork_Declaration}.


%%% Local Variables:
%%% mode: latex
%%% TeX-master: "../../CS_351-Systems_Programming-Reference_Sheet."
%%% End:



%%% Local Variables:
%%% mode: latex
%%% TeX-master: "../CS_351-Systems_Programming-Reference_Sheet"
%%% End:


\section{Input/Output (I/O)}\label{sec:IO}
In \textsc{unix}, I/O devices include:
\begin{itemize}[noitemsep]
\item Disk
\item Terminal
\item Shared Memory
\item Printer
\item Network
\end{itemize}

This is mostly because \textsc{unix} made the design decision to try to view every component of a system as a \nameref{def:File}.

Due to the variety of I/O devices that need to be supported, there are a vast number of different mechanisms for using these devices.
But, there are a few common mechanisms, requirements, and activities:
\begin{itemize}[noitemsep]
\item Read/Write Ops
\item Metadata:
  \begin{itemize}[noitemsep]
  \item Name
  \item Position
  \item Directory Name
  \item Creation Date
  \item Last Access Date
  \item IP Address
  \item MAC Address
  \item TCP Packet Sequence Number
  \end{itemize}
\item Robustness
\item Thread-safety
\end{itemize}

There are few general concerns that we need to have about the idea of viewing everything as a \nameref{def:File}.
\begin{itemize}[noitemsep]
\item How are I/O endpoints represented?
  \begin{itemize}[noitemsep]
  \item \nameref{def:File_Descriptor}
  \end{itemize}
\item How do we perform I/O?\@
  \begin{itemize}[noitemsep]
  \item Byte at a time
  \item Give a chunk of memory and later check that the requested I/O completed?
  \end{itemize}
\item How do we perform I/O \emph{efficiently}?
  \begin{itemize}[noitemsep]
  \item Efficiency depends on what we define efficient to be. Essentially, what are we optimizing for?
  \end{itemize}
\end{itemize}

\subsection{I/O Devices}\label{subsec:IO_Devices}
There are 2 major types of I/O devices on \textsc{unix} systems:
\begin{enumerate}[noitemsep]
\item \nameref{def:Block_Device}s
\item \nameref{def:Character_Device}s
\end{enumerate}

\begin{definition}[Block Device]\label{def:Block_Device}
  A \emph{block device} is an I/O device that accesses and stores data in fixed-sized blocks.
  Typically, this also means they have fixed total size as well.
  This means they support seeking through their contents and random access for parts of their contents.

  Some typical devices classifed as this are:
  \begin{itemize}[noitemsep]
  \item Disk
  \item Memory
  \end{itemize}
\end{definition}

\begin{definition}[Character Device]\label{def:Character_Device}
  A \emph{character device} is an I/O device that access and receives data as a stream.
  This means it receives ``characters'' as a stream, one-by-one.
  There is no support for seeking or random access of the stream, because we are getting the data as soon as it is being given.

  Some typical devices in this category are:
  \begin{itemize}[noitemsep]
  \item Network
  \item Mouse
  \item Keyboard
  \end{itemize}
\end{definition}

\subsection{Filesystem}\label{subsec:Filesystem}
The filesystem acts as a namespace for devices, and allows for the efficient storage of data on \nameref{def:Block_Device}s.
A typical file system consists of two types of files:
\begin{enumerate}[noitemsep]
\item \textit{Regular files} consist of ASCII or binary data
  \begin{itemize}[noitemsep]
  \item Directories
  \end{itemize}
\item \textit{Special Files} may represent:
  \begin{itemize}[noitemsep]
  \item In-memory structures
  \item Sockets
  \item Raw Devices
  \end{itemize}
\end{enumerate}

\subsection{Files}\label{subsec:Files}
\begin{definition}[File]\label{def:File}
  A \emph{file} is an operating system abstraction over some other data.
  It allows us to interact with many different file systems and devices over a variety of protocols in a abstract and concise way.
  A file can be accessed by using a \textit{fully qualified path}.

  The only thing each file \textbf{MUST} have is a unique \nameref{def:inode}.
\end{definition}

\begin{definition}[\texttt{inode}]\label{def:inode}
  The \emph{inode} is a filesystem-unique number (Typically, there is one filesystem per device, so this is typically a per-device-unique number).
  The inode tracks:
  \begin{itemize}[noitemsep]
  \item Ownership
  \item Permissions
  \item Size
  \item Type
  \item Location
  \item Number of \nameref{def:Hard_Link}s
  \end{itemize}
\end{definition}

\begin{definition}[Hard Link]\label{def:Hard_Link}
  A \emph{hard link} is a link between \nameref{def:inode}s.
  Thus, they each point to the same data, and must have the same name.
  When one of the links is deleted, the total count for that inode decreases.
  Once the inode reaches \texttt{0} hard links, it is removed (deleted) from the system.
\end{definition}

\begin{definition}[Symlink]\label{def:Symlink}\label{def:Symbolic_Link}\label{def:Soft_Link}
  A \emph{symlink} (\emph{symbolic link}, \emph{soft link}) is a link between \nameref{def:File}s.
  Thus, the link points to the file, which then points to the data.
  The link is not required to have the same name as the original file.
  However, if the file that the symlink is pointing to is deleted, then we are left with a dangling pointer.
\end{definition}


%%% Local Variables:
%%% mode: latex
%%% TeX-master: "../CS_351-Systems_Programming-Reference_Sheet"
%%% End:


\section{Inter-Process Communication}\label{sec:Inter_Process_Communication}
The OS kernel is great at \emph{isolating} \nameref{def:Process}es from each other, but allowing processes to communicate with each other makes them more useful.
Allowing them to communicate enables the processes to exchange data and interact dynamically.

However, this separation is done to make programming easier.
If the OS were to \textbf{not} isolate each process
\begin{itemize}[noitemsep]
\item Any and every \nameref{def:Process} could read and/or write to any other process's memory space.
\item Thus, any process's memory integrity would not be guaranteed
\item In effect, this would make any process's control flow unpredictable.
\end{itemize}

Because the kernel enforces isolation, we need the assistance of the kernel to complete any \nameref{def:IPC}.
Two processes must explicitly request the kernel to allow them to communicate.

\begin{definition}[Inter-Process Communication]\label{def:IPC}
  \emph{Inter-Process Communication} (\emph{IPC}) is the act of two or more \nameref{def:Process}es communicating with one another.
  There are variety of mechanisms for allowing this, explored in \Cref{subsec:IPC_Mechanisms}.
\end{definition}

\subsection{Mechanisms}\label{subsec:IPC_Mechanisms}
There are a variety of mechanisms for \nameref{def:Process}es to interact and communicate with each other.
Predictably, each one of these has an intended use, has certain benefits, and has certain drawbacks.

\subsubsection{Signals}\label{subsubsec:IPC_Mechanism-Signals}
\nameref{def:Signal}s were discussed in more depth in \Cref{subsec:Signals}.
But, in the case of \nameref{def:IPC}, signals are a very limited form of communication, as the signal sends a very well-predefined message.

\subsubsection{(Regular) Files}\label{subsubsec:IPC_Mechanism-Regular_Files}
It is always possible to save the information to a regular file on the system and then have other \nameref{def:Process}es read from and write to this file to communicate.
This does have its place, but for many small reads and writes, the overhead of writing to the storage medium and using the file system will cause greater slow-downs.
The slow-down incurred will typically be 1--2 orders of magnitude slower than memory.

This is very good for \textit{static} \nameref{def:IPC}, such as configuration files.
But files are not typically considered as a regular mechanism for dynamic \nameref{def:IPC}.

\subsubsection{Shared Memory}\label{subsubsec:IPC_Mechanism-Shared_Memory}
\nameref{def:Process}es can share memory regions between each other.
This allows for very fast access, with no direct limitation on the way the information is written and read form the shared area.
However, this lack of uniformity (and potential problems with atomicity) can lead to major headaches.

\paragraph{API}\label{par:Shared_Memory_API}
In the API below, \textbf{YOU}, as the programmer, must explicitly remove this shared memory.
Memory allocated through the functions below are \textbf{NOT} deallocated upon process termination, which can lead to a permanent memory leak.
This is intended because the operating system doesn't necessarily know when a set of \nameref{def:Process}es is done with a chunk of shared memory.

\begin{description}[noitemsep]
\item[\cinline{int shm_open(const char *name, int oflag, mode_t mode)}] Open a named region of memory to share between \nameref{def:Process}es.
\item[\cinline{int shm_unlink(const char *name)}] Unlink a chunk of shared memory from the current \nameref{def:Process}'s memory map.
\end{description}

\paragraph{Synchronizing Shared Memory}\label{par:Syncing_Shared_Memory}
Shared memory is dangerous, because we have no protection from another \nameref{def:Process} writing over what we had just written.
Thus, we need to synchronize access to this memory.

This can be done with \nameref{def:Signal}s, in the following steps:
\begin{enumerate}[noitemsep]
\item Writer sends signal to reader
\item Reader reads from memory
\item Reader signals back to writer that it is done
\item Writer removes the shared memory
\end{enumerate}

But, the operating system cannot queue signals, so certain operations cannot be queued.
In addition, signals do have some overhead, so passing many signals around as a synchronization mechanism can be costly.

\subsubsection{Pipes}\label{subsubsec:IPC_Mechanism-Pipes}
These are similar to the shell pipe \mintinline{bash}{|}.

\begin{definition}[Pipe]\label{def:Pipe}
  \emph{Pipe}s are a data structure and idea that allow us to implement \nameref{def:IPC}.
  A pipe behaves like a queue data structure, with one \nameref{def:Process} writing to the pipe, and another (not necessarily the same one) reading from it.
  A pipe uses explicit \texttt{send}, \texttt{receive}, \texttt{read}, and \texttt{write} functions to utilize the pipe, making it easier to figure out what process is doing what at what time.
  However, only 2 processes can use a pipe at a time, making it difficult to go from one process to many different processes.

  There are 2 kinds of pipes:
  \begin{enumerate}[noitemsep]
  \item \nameref{def:Named_Pipe}
  \item \nameref{def:Unnamed_Pipe}
  \end{enumerate}
\end{definition}

Using \nameref{def:Pipe}s allows us to more easily implement correct \nameref{def:IPC} functionality.
In addition, there is no need to go to the file system, so there are no file system performance implications.
Technically \nameref{def:Pipe}s are implemented using \nameref{subsubsec:IPC_Mechanism-Shared_Memory}, so you get memory-access speeds.

\paragraph{Named Pipes}\label{par:Named_Pipes}
\begin{definition}[Named Pipe]\label{def:Named_Pipe}
  A \emph{named pipe} has some key differences compared to a \nameref{def:Unnamed_Pipe}.
  Both of these types exist in the \textsc{unix} world.
  These differences are:
  \begin{itemize}[noitemsep]
  \item It has a specific name which can be given to it by the programmer, which corresponds to where it is located on the filesystem.
    Named pipe is referred to through this name only by the reader and writer.
    All instances of a named pipe share the same pipe name.
  \item A named pipe can be used for communication between two unnamed process as well.
    \nameref{def:Process}es of different ancestry can share data through a named pipe.
  \item A named pipe exists in the file system.
    After input/output has been performed by the sharing \nameref{def:Process}es, the pipe still exists in the file system independently of the process, and can be used for communcation between some other processes.
  \item Named pipes can be used to provide communication between processes on the same computer or between processes on different computers across a network, as in case of a distributed system.
  \item A named pipe can have multiple process communicating through it, like multiple clients connected to one server.
  \end{itemize}

  In addition to these, one \nameref{def:Process} will block a \texttt{read} operation until another process connects to the \texttt{write} end of the pipe.
\end{definition}

\subparagraph{Named Pipe API}\label{subpar:Named_Pipe_API}
\begin{description}[noitemsep]
\item[\cinline{int mkfifo(const char *path, mode_t perms)}] Create a named pipe at the location specified by \texttt{path}, with permissions \texttt{perms}, and returns that pipe's \nameref{def:File_Descriptor}.
\item[\cinline{close(int fd)}] Close the pipe.
\end{description}

\paragraph{Unnamed Pipes}\label{par:Unnamed_Pipes}
\begin{definition}[Unnamed Pipe]\label{def:Unnamed_Pipe}
  An \emph{unnamed pipe}, sometimes called an \emph{anonymous pipe} has some key differences compared to a \nameref{def:Named_Pipe}.
  Both of these types exist in the \textsc{unix} world.
  The differences are:
  \begin{itemize}[noitemsep]
  \item Unnamed pipes are not given a name.
    It is accessible through two file descriptors that are created through the function \cinline{pipe(fd[2])}, where \texttt{fd[1]} signifies the \textbf{write} \nameref{def:File_Descriptor}, and \texttt{fd[0]} describes the \textbf{read} \nameref{def:File_Descriptor}.
  \item An unnamed pipe is only used for communication between a child and it's parent process.
  \item An unnamed pipe vanishes as soon as it is closed, or one of the \nameref{def:Process} (parent or child) completes execution.
  \item Unnamed pipes are always local; they cannot be used for communication over a network.
  \item An unnamed pipe is a one-way \nameref{def:Pipe} that typically transfers data between a parent process and a child process.
  \end{itemize}
\end{definition}

\subparagraph{Unnamed Pipe API}\label{subpar:Unnamed_Pipe_API}
\begin{description}[noitemsep]
\item[\cinline{int pipe(int fds[2])}] Create an unnamed pipe, which can be referenced by the \nameref{def:File_Descriptor}s \texttt{fds}.
  \begin{description}[noitemsep]
  \item[\cinline{fds[0]}] The read end of the unnamed pipe.
  \item[\cinline{fds[1]}] The write end of the unnamed pipe.
  \end{description}
\item[\cinline{close(int fd)}] Close the given file descriptor.
  All instances of this pipe will need to be closed on all ends to have the pipe deallocated.
\end{description}

\subsubsection{File Locks and Semaphores}\label{subsubsec:IPC_Mechanism-File_Locks_Semaphores}
These two synchronization mechanisms are used to make concurrent systems predictable.

Some mechanisms that are similar to this that we have already used are:
\begin{description}[noitemsep]
\item[\cinline{wait}] But, this has a limited ability to do \textit{things}.
\item[\cinline{kill} and \cinline{signal}] A poor way to communicate, because we cannot queue and cannot handle multiple \nameref{def:Signal}s simultaneously.
\item[Pipes] The synchronization here is actually implicit, and allows for blocking calls to get information on both ends of the pipe.
  From the end-user perspective, this simplifies things, because the kernel is in charge of keeping the pipe synchronized.
  However, we are limited to the byte stream interface of the pipe, meaning we cannot read arbitrarily within the pipe.
\end{description}

\paragraph{File Locks}\label{par:File_Locks}
File \nameref{def:Lock}s control concurrent access/modification of shared files.
\begin{definition}[Lock]\label{def:Lock}
  A \emph{lock} allows \textbf{only one} \nameref{def:Process} to enter the portion of code that is locked.
  While a thread holds this lock no other \nameref{def:Process} can execute on this code portion.

  \begin{remark}[Binary Semaphore]\label{rmk:Binary_Semaphore}
    Locks can be represented as \emph{binary \nameref{def:Semaphore}}s.
  \end{remark}
\end{definition}

These are one of the most common synchronization mechanisms, but they are not the best from a performance perspective.
\begin{itemize}[noitemsep]
\item Concurrently reading a file from multiple processes is allowed.
\item Concurrently modifying a file is not allowed.
  This can have ugly consequences.
\item A file lock prevents other \nameref{def:Process}es from using a file.
\item Locks are \textbf{NOT} preserved across a \texttt{fork}.
\item There is also \textbf{NO} assurance that the filesystem supports file locking, as Linux makes it advisory.
\item Mandatory locking is possible, but the filesystem must have been designed with that in mind.
\item In general, these are not designed for general-purpose synchronization.
\end{itemize}

\begin{definition}[Mutex]\label{def:Mutex}
  A \emph{mutex} (short for \emph{\textbf{mut}ual \textbf{ex}clusion}) is the same as a \nameref{def:Lock}, \textbf{but it can be system wide (shared by multiple processes)}.
  A mutex lock protects critical regions and prevents race conditions.
  That is, a process must acquire the lock before entering a critical section; it releases the lock when it exits the critical section.

  The \cinline{acquire()} function acquires the lock, preventing any other \nameref{def:Process} from using the thing the lock protects.
  Likewise, the \cinline{release()} function releases the lock, allowing another \nameref{def:Process} take acquire the lock and use the resource it protects.
  To perform its function correctly, the lock's \cinline{acquire()} and \cinline{release()} functions must be atomic.

  When a \nameref{def:Process} and/or \nameref{def:Process} attempts \texttt{acquire()} the lock, while it is already owned by someone else, it is put in a \texttt{WAITING} state.
\end{definition}

\subparagraph{File Lock API}\label{subpar:File_Lock_API}
\begin{description}[noitemsep]
\item[\cinline{int fcntl(int fd, int cmd, struct flock)}] Create a file lock for the given \nameref{def:File_Descriptor}.
  \begin{itemize}[noitemsep]
  \item \texttt{cmd} is an enumeration of \texttt{F\_GETLK}, \texttt{F\_SETLK}, \texttt{F\_SETLKW}.
    \begin{description}[noitemsep]
    \item[\cinline{F_GETLK}] Test acquisition of the lock.
    \item[\cinline{F_SETLK}] Acquire the lock.
    \item[\cinline{F_SETLKW}] Release the lock.
    \end{description}
  \item \texttt{struct flock { ... }} is the type of lock and how to get the lock.
  \end{itemize}
\end{description}

\paragraph{Semaphores}\label{par:Semaphores}
\nameref{def:Semaphore}s control shared memory's access and modification.
Essentially, they allow $m$ of $n$ \nameref{def:Process}es to acquire a resource.
If this is supposed to be a mutually exclusive resource, then $m = 1$, and this is called a \nameref{def:Mutex}.

\nameref{def:Semaphore}s control the order in which \nameref{def:Process}es run.
They are typically associated with a counter.
They also require that the wait and release process be atomic, especially when working with the counter.

\begin{definition}[Semaphore]\label{def:Semaphore}
  A \emph{semaphore} regulates the number of things (\nameref{def:Process}es or \nameref{def:Process}s) performing operations on something (Shared Resource).
  Functionally, this is the same as a \nameref{def:Mutex} but allows $x$ \nameref{def:Process}es/\nameref{def:Process}s to enter or use the resource at a time.
  This allows for limits on the number of CPU, I/O or RAM intensive tasks running at the same time.

  A semaphore can only be interacted with through 2 operations \cinline{wait()} and \cinline{signal()}.
  \cinline{wait()} is similar to a \nameref{def:Mutex}'s \cinline{acquire()} function and the \cinline{signal()} function is similar to the \nameref{def:Mutex}'s \cinline{release()} function.
  Here, if a \nameref{def:Process} or \nameref{def:Process} \texttt{wait}s, if there is more of the resource, then the requester gets the resource, and the internal count of the semaphore is decremented.
  If a \nameref{def:Process} \texttt{signal}s, then it is done with the resource, and the requester loses access to the resources, and the internal count of the semaphore is incremented.
  Again, these manipulations \textbf{MUST} be atomic.

  \begin{remark}[Confusion with Mutexes]\label{rmk:Semaphore_Mutex_Confusion}
    Technically, you can create a \nameref{def:Semaphore} that acts like a \nameref{def:Mutex} by giving it a binary value.
    However, this is typically in poor programming taste, because while both function similarly, the \nameref{def:Semaphore} is for signaling the amount of a resource available and the \nameref{def:Mutex} is for signaling if code is capable of execution.
  \end{remark}

  \begin{remark}[Correct Use of Semaphores]\label{rmk:Semaphore_Correct_Usage}
    The correct use of a \nameref{def:Semaphore} is for signaling from one task to another.
    A \nameref{def:Mutex} is meant to be taken and released, always in that order, by each task that uses the shared resource it protects.
    By contrast, tasks that use \nameref{def:Semaphore}s either signal or wait, not both.
  \end{remark}
\end{definition}

\subparagraph{Semaphore API}\label{subpar:Semaphore_API}
\begin{description}[noitemsep]
\item[\cinline{sem_t *sem_open(const char *name, int oflag, mode_t mode, unsigned int value)}] Create a semaphore with \texttt{name}, the flags \texttt{oflag}, with the mode \texttt{mode}, and an initial value of \texttt{value}.
  The semaphore object is returned from the call.
\item[\cinline{int sem_wait(sem_t *sem)}] Decrements \texttt{sem}'s backing counter. If the counter is already \texttt{0}, then this call blocks the calling \nameref{def:Process}.
\item[\cinline{int sem_post(sem_t *sem)}] Icrements \texttt{sem}'s backing counter. If the counter was \texttt{0} and becomes greater than \texttt{0}, and another \nameref{def:Process} is currently waiting on \texttt{sem}, then it one will be woken up and allowed to lock the semaphore.
\end{description}

\paragraph{Spinlocks}\label{par:Spinlocks}
A spinlock is the process of busy-polling a resource until its available.
This is a highly-responsive way to allocate, but wastes a lot of CPU time.

\subsubsection{Sockets}\label{subsubsec:Sockets}
\nameref{def:Socket}s are mainly used for network communication.
However, they can be used on the local computer too.

\begin{definition}[Socket]\label{def:Socket}
  A \texttt{socket} is a way of connecting two nodes on a network or \nameref{def:Process}es to communicate with each other.
  One socket listens, while another socket reaches out to the other to form a connection.

  Network sockets have a high overhead due to the software-defined network stacks.
\end{definition}

Almost all modern computers use this today, except for High Performance Computing, which uses their own hardware solutions to reduce latency.

\subsection{Challenges}\label{subsec:IPC_Challenges}
\subsubsection{Link/Endpoint Creation}\label{subsubsec:Link_Endpoint_Creation}
Some common issues with creating a link and/or endpoint are:
\begin{itemize}[noitemsep]
\item Naming the endpoint
\item Looking up the endpoint
\item Need a registry to keep track of this information
\end{itemize}

\subsubsection{Data Transmission}\label{subsubsec:IPC_Challenge-Data_Transmission}
Data transmission has many questions to answer for it to be effective.
These include:
\begin{itemize}[noitemsep]
\item Unidirectional or bidirectional?
\item Single-sender or multi-sender and/or single-receiver or multi-receiver?
\item Speed of the transmission medium/link?
\item Capacity of the transmission medium?
\item Message packetizing? How does the message stream get converted to packets?
\item How is the transmission routed?
\end{itemize}

\subsubsection{Data Synchronization}\label{subsubsec:IPC_Challenge-Data_Sync}
For the data to be of any use to anyone/anything, must it arrive in a certain order?
These kinds of questions define the data synchronization problem of \nameref{def:IPC}, and some more are included below:
\begin{itemize}[noitemsep]
\item What is the behavior when there are multiple senders and/or receivers?
\item What is the control required to synchronize?
  \begin{itemize}[noitemsep]
  \item Is it done implicitly?
  \item Does it need to be done explicitly?
  \item Is there \textbf{ANY} synchronization?
  \end{itemize}
\end{itemize}

%%% Local Variables:
%%% mode: latex
%%% TeX-master: "../CS_351-Systems_Programming-Reference_Sheet"
%%% End:


\section{Sockets}\label{sec:Sockets}
\subsection{Communication Protocols}\label{subsec:Communication_Protocols}
\begin{itemize}[noitemsep]
\item Protocols are agreement and rules on communication
\item These can be connection-oriented or connectionless
\item These protocols are typically built with a layered architecture-
  \begin{enumerate}
  \item Physical
  \item Data Link
  \item Network
  \item Transport
  \item Session
  \item Presentation
  \item Application
  \end{enumerate}
\item These messages build off each other by wrapping the higher-level protocol in a lower one.
\end{itemize}

\subsubsection{Physical}\label{subsubsec:Physical_Protocols}
\begin{itemize}[noitemsep]
\item How to encode 0s and 1s
\item What voltages are used
\item How long does a bit need to be signaled
\item What does the cable, plug, antenna look like?
\end{itemize}

\subsubsection{Data Link}\label{subsubsec:Data_Link_Protocols}
\begin{itemize}[noitemsep]
\item How big is a frame
\item Can I detect an error
\item What marks an end of a frame
\item How do I control access to a shared channel (Flow Control)
\end{itemize}

\subsubsection{Network}\label{subsubsec:Network_Protocols}
\begin{itemize}[noitemsep]
\item How to route packets
\item Congestion control algorithm
  \begin{itemize}[noitemsep]
  \item Traffic Shaping
  \item Flow Specifications
  \item Bandwidth reservation
  \end{itemize}
\item Accounting
\item Fragment or combine packets
\end{itemize}

\subsubsection{Transport Layer}\label{subsubsec:Transport_Layer_Protocols}
\begin{itemize}[noitemsep]
\item How to order messages and detect duplicates
\item Error detection
\item Retransmission
\item Connection-oriented vs. Connectionless
\end{itemize}

\subsubsection{Session and Presentation}\label{subsubsec:Session_Presentation_Protocols}
\begin{itemize}[noitemsep]
\item
\end{itemize}

\subsubsection{Application}\label{subsubsec:Application_Protocols}
\begin{itemize}[noitemsep]
\item What marks the subject field
\item How to represent cursor movements
\item Services
  \begin{itemize}[noitemsep]
  \item SMTP
  \item FTP
  \item HTTP
  \item SNMP
  \item NFS
  \item NTP
  \item NNTP
  \end{itemize}
\end{itemize}

%%% Local Variables:
%%% mode: latex
%%% TeX-master: "../../CS_351-Systems_Programming-Reference_Sheet"
%%% End:





\subsection{Middleware Protocols}\label{subsubsec:Middleware_Protocols}

%%% Local Variables:
%%% mode: latex
%%% TeX-master: "../CS_351-Systems_Programming-Reference_Sheet"
%%% End:


%====================================APPENDIX====================================
\appendix
\counterwithin{definition}{subsection}

\clearpage
\section{Computer Components}\label{app:Computer_Components}
\subsection{Central Processing Unit}\label{subsec:CPU}
\begin{definition}[Central Processing Unit]\label{def:CPU}
  The \emph{Central Processing Unit}, \emph{CPU}, is a chip that performs all actions in the computer.
  It calculates mathematical and logical values and acts based on them.
  It has several components built onto it, and can be thought of as the ``brain'' of the computer.

  The design of a CPU determines some of the functionality it has.
  Therefore, more specialized processors can be made for special tasks, and more general processors can be built to handle a wide variety of calculations.
\end{definition}

\subsubsection{Registers}\label{subsubsec:Registers}
\begin{definition}[Register]\label{def:Register}
  A \emph{register} is a data storage mechanism built directly onto the \nameref{def:CPU}.
  It is several hundred times faster than the system \nameref{def:Memory}.
  Registers are generally used when the currently running program is performing calculations.
  Since they are so fast, they are used as both source and destination operands in instructions.

  \begin{remark}
    Depending on the \nameref{def:CPU} architecture, there may be cases when \nameref{def:Register}s behave slightly differently between processors.
    This is something that can only be found by checking the \nameref{def:CPU} manufacturer's documentation.
  \end{remark}
\end{definition}

\subsubsection{Program Counter}\label{subsubsec:Program_Counter}
\subsubsection{Arithmetic Logic Unit}\label{subsubsec:ALU}
\subsubsection{Cache}\label{subsubsec:CPU_Cache}

\subsection{Memory}\label{subsec:Memory}
\begin{definition}[Memory]\label{def:Memory}
  \emph{Memory}, or \emph{RAM} (\emph{Random Access Memory}), is a \nameref{def:Volatile} data storage mechanism.
  It is directly connected to the \nameref{def:CPU}.
  This is the location that the \nameref{def:CPU} writes to when it cannot or should not store something in the \nameref{def:CPU}'s \nameref{def:Register}s.

  \begin{remark}[Volatility]
    \nameref{def:Memory} is volatile because each of the cells is a small capacitor.
    In between the clock cycles on the \nameref{def:CPU} and \nameref{def:Memory}, the capacitors discharge.
    On the clock cycle, the capacitors are refreshed with electrical power, which does one of 2 things:
    \begin{enumerate}[noitemsep]
    \item Keep the data bits the same, 1 to 1.
    \item Update the data bits from 0 to 1.
    \end{enumerate}
  \end{remark}
\end{definition}

\begin{definition}[Volatile]\label{def:Volatile}
  If a data storage mechanism is called \emph{volatile}, it means that once the storage mechanism loses power, the data is lost.
  This is in contrast to \nameref{def:Non-Volatile} data storage mechanisms.
\end{definition}

\subsection{Disk}\label{subsec:Disk}
\begin{definition}[Non-Volatile]\label{def:Non-Volatile}
  If a data storage mechanism is called \emph{non-volatile}, it means that once the storage device loses power, the data is still safely stored.
  This is in contrast to \nameref{def:Volatile} data storage mechanisms.
\end{definition}

\subsection{Fetch-Execute Cycle}\label{subsec:Fetch_Execute_Cycle}
%%% Local Variables:
%%% mode: latex
%%% TeX-master: shared
%%% End:


% To make this print, you must include a citation somewhere in the document
\clearpage
\printbibliography{}
\end{document}

%%% Local Variables:
%%% mode: latex
%%% TeX-master: t
%%% End:
