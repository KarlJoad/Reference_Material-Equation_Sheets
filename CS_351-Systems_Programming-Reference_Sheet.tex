\documentclass[10pt,letterpaper,final,twoside,notitlepage]{article}
\usepackage[margin=.5in]{geometry} % 1/2 inch margins on all pages
\usepackage[utf8]{inputenc} % Define the input encoding
\usepackage[USenglish]{babel} % Define language used
\usepackage{amsmath}
\usepackage{mathtools} % Allow for text and math in align* environment.
\usepackage{amsfonts}
\usepackage{amssymb}
\usepackage{amsthm} % Gives us plain, definition, and remark to use in \theoremstyle{style}
\usepackage{thmtools}
\usepackage{thm-restate}
\usepackage{graphicx}

\usepackage[
backend=biber,
style=alphabetic,
citestyle=authoryear]{biblatex} % Must include citation somewhere in document to print bibliography
\usepackage{hyperref} % Generate hyperlinks to referenced items
\usepackage[nottoc]{tocbibind} % Prints the Reference/Bibliography in TOC as well
\usepackage[noabbrev,nameinlink]{cleveref} % Fancy cross-references in the document everywhere
\usepackage{nameref} % Can make references by name to places
\usepackage{caption} % Allows for greater control over captions in figure, algorithm, table, etc. environments
\usepackage{subcaption} % Allows for multiple figures in one Figure environment
\usepackage[binary-units=true]{siunitx} % Gives us ways to typeset units for stuff
\usepackage{csquotes} % Context-sensitive quotation facilities
\usepackage{enumitem} % Provides [noitemsep, nolistsep] for more compact lists
\usepackage{chngcntr} % Allows us to tamper with the counter a little more
\usepackage{empheq} % Allow boxing of equations in special math environments
\usepackage[x11names]{xcolor} % Gives access to coloring text in environments or just text, MUST be before tikz
\usepackage{tcolorbox} % Allows us to create boxes of various types for examples
\usepackage{tikz} % Allows us to create TikZ and PGF Pictures
\usepackage{ctable} % Greater control over tables and how they look
\usepackage{multirow} % Allow us to have a single cell in a table span multiple rows
\usepackage{titling} % Put document information throughout the document programmatically
\usepackage[linesnumbered,ruled,vlined]{algorithm2e} % Allows us to write algorithms in a nice style.

\counterwithin{figure}{section}
\counterwithin{table}{section}
\counterwithin{equation}{section}
\counterwithin{algocf}{section}
\crefname{algocf}{algorithm}{algorithms}
\Crefname{algocf}{Algorithm}{Algorithms}
\setcounter{secnumdepth}{4}
\setcounter{tocdepth}{4} % Include \paragraph in toc
\crefname{paragraph}{paragraph}{paragraphs}
\Crefname{paragraph}{Paragraph}{Paragraphs}

% Create a theorem environment
\theoremstyle{plain}
\newtheorem{theorem}{Theorem}[section]
% Create a numbered theorem-like environment for lemmas
\newtheorem{lemma}{Lemma}[theorem]

% Create a definition environment
\theoremstyle{definition}
\newtheorem{definition}{Defn}
\newtheorem{corollary}{Corollary}[section]
% \begin{definition}[Term] \label{def:}
% 		Make sure the term is emphasized with \emph{term}.
%		This ensures that if \emph is changed, it shows up everywhere
% \end{definition}

% Create a numbered remark environment numbered based on definition
% NOTE: This version of remark MUST go inside a definition environment
\theoremstyle{remark}
\newtheorem{remark}{Remark}[definition]
%\counterwithin{definition}{subsection} % Uncomment to have definitions use section.subsection numbering

% Create an unnumbered remark environment for general use
% NOTE: This version of remark has NO restrictions on placement
\newtheorem*{remark*}{Remark}

% Create a special list that handles properties. It can be broken and restarted
\newlist{propertylist}{enumerate}{1} % {Name}{Template}{Max Depth}
% [newlistname, LevelsToApplyTo]{formatting options}
\setlist[propertylist, 1]{label=\textbf{(\roman*)}, ref=\textbf{(\roman*)}, noitemsep, nolistsep}
\crefname{propertylisti}{property}{properties}
\Crefname{propertylisti}{Property}{Properties}

% Create a special list that handles enumerate starting with lower letters. Breakable/Restartable.
\newlist{boldalphlist}{enumerate}{1} % {Name}{Template}{Max Depth}
% [newlistname, LevelsToApplyTo]{formatting options}
\setlist[boldalphlist, 1]{label=\textbf{(\alph*)}, ref=\alph*, noitemsep, nolistsep} % Set options

\newlist{nocrefenumerate}{enumerate}{1} % {Name}{Template}{Max Depth}
% [newlistname, LevelsToApplyTo]{formatting options}
\setlist[nocrefenumerate, 1]{label=(\arabic*), ref=(\arabic*), noitemsep, nolistsep}

% Create a list that allows for deeper nesting of numbers. By default enumerate only allows depth=4.
\newlist{nestednums}{enumerate}{6}
% [newlistname, LevelsToApplyTo]{formatting options}
\setlist[nestednums]{noitemsep, label*=\arabic*.}

\tcbuselibrary{breakable} % Allow tcolorboxes to be broken across pages
% Create a tcolorbox for examples
% /begin{example}[extra name]{NAME}
% Create a tcolorbox for examples
% Argument #1 is optional, given by [], that is the textbook's problem number
% Argument #2 is mandatory, given by {}, that is the title for the example
% Avoid putting special characters, (), [], {}, ",", etc. in the title.
\newtcolorbox[auto counter,
number within=section,
number format=\arabic,
crefname={example}{examples}, % Define reference format for cref (No Capitals)
Crefname={Example}{Examples}, % Reference format for cleveref (With Capitals)
]{example}[2][]{ % The [2][] Means the first argument is optional
  width=\textwidth,
  title={Example \thetcbcounter: #2. #1}, % Parentheses and commas are not well supported
  fonttitle=\bfseries,
  label={ex:#2},
  nameref=#2,
  colbacktitle=white!100!black,
  coltitle=black!100!white,
  colback=white!100!black,
  upperbox=visible,
  lowerbox=visible,
  sharp corners=all,
  breakable
}

% Create a tcolorbox for general use
\newtcolorbox[% auto counter,
% number within=section,
% number format=\arabic,
% crefname={example}{examples}, % Define reference format for cref (No Capitals)
% Crefname={Example}{Examples}, % Reference format for cleveref (With Capitals)
]{blackbox}{
  width=\textwidth,
  % title={},
  fonttitle=\bfseries,
  % label={},
  % nameref=,
  colbacktitle=white!100!black,
  coltitle=black!100!white,
  colback=white!100!black,
  upperbox=visible,
  lowerbox=visible,
  sharp corners=all
}

% Redefine the 'end of proof' symbol to be a black square, not blank
\renewcommand\qedsymbol{$\blacksquare$} % Change proofs to have black square at end

\renewcommand{\Re}{\operatorname{Re}} % Redefine to use the command, but not the fraktur version
\renewcommand{\Im}{\operatorname{Im}} % Redefine to use the command, but not the fraktur version
% Math Operators that are useful to abstract the written math away to one spot
\DeclareMathOperator{\RealNumbers}{\mathbb{R}}
\newcommand{\TextRealNumbers}{$\RealNumbers$}
\DeclareMathOperator{\AllIntegers}{\mathbb{Z}}
\newcommand{\TextAllIntegers}{$\AllIntegers$}
\DeclareMathOperator{\PositiveInts}{\mathbb{Z}^{+}}
\newcommand{\TextPositiveInts}{$\PositiveInts$}
\DeclareMathOperator{\NegativeInts}{\mathbb{Z}^{-}}
\newcommand{\TextNegativeInts}{$\NegativeInts$}
\DeclareMathOperator{\NaturalNumbers}{\mathbb{N}}
\newcommand{\TextNaturalNumbers}{$\NaturalNumbers$}
\DeclareMathOperator{\ComplexNumbers}{\mathbb{C}}
\newcommand{\TextComplexNumbers}{$\ComplexNumbers$}
\DeclareMathOperator{\RationalNumbers}{\mathbb{Q}}
\newcommand{\TextRationalNumbers}{$\RationalNumbers$}
\DeclareMathOperator*{\argmax}{argmax} % Thin Space and subscripts are UNDER in display
\DeclareMathOperator{\Lapl}{\mathcal{L}} % Declare a Laplace symbol to be used
\DeclareMathOperator{\UnitStep}{\mathcal{U}}
\DeclareMathOperator{\sinc}{sinc} % sinc(x) = (sin(pi x)/(pi x))
\DeclareMathOperator{\XOR}{\oplus}

\newcommand{\rbpRegister}{\texttt{\%rbp}}
\newcommand{\rspRegister}{\texttt{\%rsp}}
\newcommand{\ripRegister}{\texttt{\%rip}}
\newcommand{\raxRegister}{\texttt{\%rax}}
\newcommand{\rbxRegister}{\texttt{\%rbx}}

%%% Local Variables:
%%% mode: latex
%%% TeX-master: shared
%%% End:


% These packages are more specific to certain documents, but will be availabe in the template
% \usepackage{esint} % Provides us with more types of integral symbols to use
\usepackage[outputdir=./TeX_Output]{minted} % Allow us to nicely typeset 300+ programming languages
\crefname{lstlisting}{listing}{listings}
\Crefname{lstlisting}{Listing}{Listings}
% This document must be compiled with the -shell-escape flag if the packages above are uncommented

% \graphicspath{{./Drawings/Course}} % Uncomment this to use pictures in this document
% \addbibresource{./Bibliographies/CourseNum-Name.bib}

\usemintedstyle{emacs} % Best for use on white backgrounds. Use for inline-code
% This macro creates the 2 minted environments for kernel source code with common options
\def\mintedkernelargs{
  frame=lines, % Surround the source code with lines on top and bottom
  linenos, % We want to show line numbers for each line in the margin
  % Colors used here are xcolor X11 colors.
  % style=fruity, % Use the fruity color scheme. Best for use on black backgrounds. Use for code blocks.
  % bgcolor=black, % Set the background used
  style=emacs,
  bgcolor=white,
  autogobble=true, % Automatically remove shared indentation from files
  breaklines=true, % Break lines that are too long at convenient locations
}
\newcommand{\makenewmintedfiles}[1]{
  \newminted[csource]{c}{#1} % Use with \begin{csource} code \end{csource}

  \newmintedfile[csourcefile]{c}{#1} % Use with \csourcefile[additional-options]{Filename}
}
\expandafter\makenewmintedfiles\expandafter{\mintedkernelargs}
\newmintinline[cinline]{c}{% Use with \cinline{code}
  style=emacs,
  bgcolor=white,
}

% Math Operators that are useful to abstract the written math away to one spot
% These are supposed to be document-specific mathematical operators that will make life easier
% Many fundamental operators are defined in Reference_Sheet_Preamble.tex

\begin{titlepage}
  \title{CS 351: Systems Programming --- Reference Material \\ Illinois Institute of Technology}
  \author{Karl Hallsby}
  \date{Last Edited: \today} % We want to inform people when this document was last edited
\end{titlepage}

\begin{document}
\pagenumbering{gobble}
\maketitle
\pagenumbering{roman} % i, ii, iii on beginning pages, that don't have content
\tableofcontents
\clearpage
\listoftheorems[ignoreall, show={definition, Definition}]
\clearpage
\listoflistings{}
\clearpage
\pagenumbering{arabic} % 1,2,3 on content pages

\section{C Programming}\label{sec:C_Programming}
C is one of the lowest ``high level'' languages you can use today.
It provides very minimal abstractions from hardware and assembly code, but allows you to relatively good typechecked code.

\input{./Appendix-Reference_Material/C_Programming/Memory_Allocation}

%%% Local Variables:
%%% mode: latex
%%% TeX-master: shared
%%% End:


\section{Processes}\label{sec:Processes}
\nameref{def:Process}es are the fundamental unit of computation within an operating system.
\begin{definition}[Process]\label{def:Process}
  A \emph{process} is a \nameref{def:Program} in execution.
  A process carries out the computation that we specify.
  A process contains:
  \begin{itemize}[noitemsep]
  \item Code (\texttt{text}) of your program.
  \item Runtime data (Global, local, dynamic variables)
  \item \nameref{def:Register}s:
    \begin{itemize}[noitemsep]
    \item \nameref{def:Program_Counter} (\texttt{PC})
    \item \nameref{def:Stack_Pointer} (\texttt{SP})
    \item \nameref{def:Frame_Pointer} (\texttt{FP})
    \end{itemize}
  \item \nameref{def:Process_Control_Block}
  \end{itemize}
\end{definition}

\begin{definition}[Program]\label{def:Program}
  A \emph{program} is the binary image stored at some file location on the storage medium.
  The program is read into memory, and then is used to start a \nameref{def:Process} that runs that program.
\end{definition}

\nameref{def:Process}es require both a \textbf{predictable} and \textbf{logical} control flow.
This means:
\begin{itemize}[noitemsep]
\item The \nameref{def:Process} must start somewhere, typically defined to be \texttt{main}.
\item Nothing can disrupt a program mid-execution.
  \begin{itemize}[noitemsep]
  \item This is further discussed in \Cref{subsec:Prevent_Process_Disruption}.
  \end{itemize}
\end{itemize}

\subsection{Prevent Process Disruption}\label{subsec:Prevent_Process_Disruption}
The easiest way to prevent a \nameref{def:Process} from having its control flow being interrupted is for the process to ``own'' the CPU for the entire duration of the process's execution.
However, this means:
\begin{itemize}[noitemsep]
\item No other process can run on this core
\item This prevents efficient multi-\nameref{def:Process}/multitasking systems
\item Malicious or poorly written program can ``take over'' the CPU
\item An idle process (for example, waiting for user input) will underutilize the CPU
\end{itemize}

For the operating system to simulate this seamless logical control flow, we use all of the information used to make a \nameref{def:Process}, and need a \nameref{def:Process_Control_Block}.
\begin{definition}[Process Control Block]\label{def:Process_Control_Block}
  The \emph{Process Control Block} (\emph{PCB}) contains additional metadata about a \nameref{def:Process}.
  This includes:
  \begin{itemize}[noitemsep]
  \item Process ID (\texttt{PID})
  \item CPU Usage
  \item Memory Usage
  \item Pending \nameref{def:Syscall}s
  \end{itemize}
\end{definition}


%%% Local Variables:
%%% mode: latex
%%% TeX-master: "../CS_351-Systems_Programming-Reference_Sheet"
%%% End:


%====================================APPENDIX====================================
\appendix
\counterwithin{definition}{subsection}

\clearpage
\section{Computer Components}\label{app:Computer_Components}
\subsection{Central Processing Unit}\label{subsec:CPU}
\begin{definition}[Central Processing Unit]\label{def:CPU}
  The \emph{Central Processing Unit}, \emph{CPU}, is a chip that performs all actions in the computer.
  It calculates mathematical and logical values and acts based on them.
  It has several components built onto it, and can be thought of as the ``brain'' of the computer.

  The design of a CPU determines some of the functionality it has.
  Therefore, more specialized processors can be made for special tasks, and more general processors can be built to handle a wide variety of calculations.
\end{definition}

\subsubsection{Registers}\label{subsubsec:Registers}
\begin{definition}[Register]\label{def:Register}
  A \emph{register} is a data storage mechanism built directly onto the \nameref{def:CPU}.
  It is several hundred times faster than the system \nameref{def:Memory}.
  Registers are generally used when the currently running program is performing calculations.
  Since they are so fast, they are used as both source and destination operands in instructions.

  \begin{remark}
    Depending on the \nameref{def:CPU} architecture, there may be cases when \nameref{def:Register}s behave slightly differently between processors.
    This is something that can only be found by checking the \nameref{def:CPU} manufacturer's documentation.
  \end{remark}
\end{definition}

\subsubsection{Program Counter}\label{subsubsec:Program_Counter}
\subsubsection{Arithmetic Logic Unit}\label{subsubsec:ALU}
\subsubsection{Cache}\label{subsubsec:CPU_Cache}

\subsection{Memory}\label{subsec:Memory}
\begin{definition}[Memory]\label{def:Memory}
  \emph{Memory}, or \emph{RAM} (\emph{Random Access Memory}), is a \nameref{def:Volatile} data storage mechanism.
  It is directly connected to the \nameref{def:CPU}.
  This is the location that the \nameref{def:CPU} writes to when it cannot or should not store something in the \nameref{def:CPU}'s \nameref{def:Register}s.

  \begin{remark}[Volatility]
    \nameref{def:Memory} is volatile because each of the cells is a small capacitor.
    In between the clock cycles on the \nameref{def:CPU} and \nameref{def:Memory}, the capacitors discharge.
    On the clock cycle, the capacitors are refreshed with electrical power, which does one of 2 things:
    \begin{enumerate}[noitemsep]
    \item Keep the data bits the same, 1 to 1.
    \item Update the data bits from 0 to 1.
    \end{enumerate}
  \end{remark}
\end{definition}

\begin{definition}[Volatile]\label{def:Volatile}
  If a data storage mechanism is called \emph{volatile}, it means that once the storage mechanism loses power, the data is lost.
  This is in contrast to \nameref{def:Non-Volatile} data storage mechanisms.
\end{definition}

\subsubsection{Stack}\label{subsubsec:Memory_Stack}
\begin{definition}[Call Stack]\label{def:Call_Stack}
  The \emph{call stack} is an imaginary construct that resembles the traditional stack data structure.
  It is a way to visualize and organize the way memory is used during the execution of a program and its function/methods.
  It is filled with \nameref{def:Stack_Frame}s.
  This \textbf{does not} hold the code that is used in the function, rather it is everything that is needed for the function to be able to run.
\end{definition}

\begin{definition}[Stack Frame]\label{def:Stack_Frame}
  A \emph{stack frame}, \emph{activation frame}, or \emph{activation record} are objects that represent necessary portions of a function.
  These include:
  \begin{itemize}[noitemsep]
  \item A \nameref{def:Dynamic_Link}
  \item \nameref{def:Local_Variable}s
  \item \nameref{def:Temporary_Variable}s
  \item \nameref{def:Static_Link}s
  \item \nameref{def:Function_Argument}s
  \item \nameref{def:Return_Address}
  \end{itemize}

  Additionally, there are 2 registers used as pointers to move around and interact with the stack frame.
  \begin{enumerate}[noitemsep]
  \item \texttt{FP} is in register \rbp{}. It is the \nameref{def:Frame_Pointer}.
  \item \texttt{SP} is in register \rsp{}. It is the \nameref{def:Stack_Pointer}.
  \end{enumerate}
\end{definition}

\begin{definition}[Dynamic Link]\label{def:Dynamic_Link}
  The \emph{dynamic link} or \emph{dynlink} is a memory address pointer and sits at the bottom of a \nameref{def:Stack_Frame}.
  It is a pointer back to the previous function's dynamic link.
  This ensures that any function can find its parent/calling function.

  The dynamic link also serves as a means to access any variable that might be needed by this function.
  To access any variable in \emph{this} function, you can subtract a byte multiple that you need to access the proper value.
  To access any variable in the calling function, you can add a byte multiple that would correspond to the proper variable.

    In the \texttt{x86\textunderscore{} 64} architecture, instruction set, and convention that we used, the register \rbp{} was the dynamic link.
  \begin{remark}
    Note: Due to the conventions we used, when accessing arguments passed to the function, we treated them as local variables, just further down in the call stack.
    This also means that we need to skip over the \nameref{def:Return_Address} block in memory.
  \end{remark}
  
  \begin{remark}
    Because we used the \rbp{} register to store our current \nameref{def:Dynamic_Link}'s address, the dynamic link might also be called the \emph{base pointer}.
  \end{remark}
\end{definition}

\begin{definition}[Local Variable]\label{def:Local_Variable}
  \emph{Local variables} are handled very simply.
  They get an appropriate amount of memory allocated to them on the stack, and that is it.

  There is no way to give a variable a name in assembly. (Usually. Depends on the architecture and instruction set).
  However, there is no way to name something in memory.
  But, because the size of all the objects is known at compile-time, allocating the proper amount of memory required by each variable is possible.

  \begin{remark}
    This holds true for strongly-typed, static, compiled languages, like Java, C, C++, etc.
    However, Python is slightly different in this regard, and handles it differently.
  \end{remark}
\end{definition}

\begin{definition}[Temporary Variable]\label{def:Temporary_Variable}
  A \emph{temporary variable} is one that is allocated on this function's \nameref{def:Stack_Frame} while calculating values.
  Once the calculations are completed, these values are deallocated.
  These temporary variables can also point to objects on the \nameref{def:Heap}.
  When the function has finished running, then these values are deallocated, along with all other \nameref{def:Local_Variable}s in use.
  
  For example, since assembly-level addition only allows for 2 operands, but in general, addition can have more than 2 operands in use, there needs to be a way to store the value used in the addition.
  While we can accumulate and use that value in the addition, the values being added together are \emph{not} modified.
\end{definition}

\begin{definition}[Static Link]\label{def:Static_Link}
  The \emph{static link} is an implicit argument, meaning it \textbf{\emph{ALWAYS}} gets pushed onto the stack as a \nameref{def:Function_Argument}, when appropriate.
  
  Appropriate in this context could mean several things:
  \begin{itemize}[noitemsep]
  \item When an object is alive, and when it is being acted on by a function.
    \begin{itemize}[noitemsep]
    \item In this case, the static link points to an instance of a class on the \nameref{def:Heap}.
    \end{itemize}
  \item When a language allows for nested function declarations.
    \begin{itemize}[noitemsep]
    \item Then the static link points to the \nameref{def:Dynamic_Link} of the outer function
    \item This allows access to the outer function's \nameref{def:Local_Variable}s like normal, and allows us to go back later.
    \end{itemize}
  \end{itemize}
\end{definition}

\begin{definition}[Function Argument]\label{def:Function_Argument}
  \emph{Function argument}s are handled very simply.
  If a function call takes an argument, then the argument is calculated, and then that argument is pushed onto the stack in \emph{this} \nameref{def:Stack_Frame}.

  \begin{remark}
    If more than one argument is passed to a function, there are 2 ways to push values onto the \nameref{def:Stack_Frame}:
    \begin{enumerate}[noitemsep]
    \item In the order they are passed to the function
      \begin{itemize}[noitemsep]
      \item Say a function with 3 arguments is called, then the stack would have arguments in this order
        \begin{enumerate}[noitemsep]
        \item argument0 (Lowest memory address)
        \item argument1
        \item argument2 (Highest memory address)
        \end{enumerate}
      \end{itemize}
    \item In reverse order
      \begin{itemize}[noitemsep]
      \item Say the same function is called with the same 3 arguments, then the stack would have arguments in this order
        \begin{enumerate}[noitemsep]
        \item argument2 (Lowest memory address)
        \item argument1
        \item argument0 (Highest memory address)
        \end{enumerate}
      \end{itemize}
    \end{enumerate}
  \end{remark}

  When the values that were passed need to be accessed, and if the memory sizes of things are known at compile time, then we can calculate how far down we need to go in the stack to find the value.
  This is done by adding a positive value to the \rbp{}
\end{definition}

\begin{definition}[Return Address]\label{def:Return_Address}
  The \emph{return address} is used by the \rip{} register.
  It is calculated and pushed onto the \nameref{def:Stack_Frame} stack when the \texttt{CALL} macro is used.
  It is the thing that allows us to jump around in the code from the \texttt{text} area of our program.

  \begin{remark}
    The \rip{} register is the \emph{register instruction pointer}.
    It holds the value of the \emph{next} instruction to execute.
    Technically, it is the Program Counter's value.
  \end{remark}
\end{definition}

\begin{definition}[Garbage Collection]\label{def:Garbage_Collection}
  \emph{Garbage collection} is the act of deallocating objects that may still be on the heap and organizing the heap.
  Since the heap allocates continuous ``blocks'' of memory required by an object, the heap may have the necessary memory to allocate an object, but in discontinuous locations.
\end{definition}

\begin{definition}[Frame Pointer]\label{def:Frame_Pointer}
  The \emph{frame pointer} is a pointer that \textbf{\emph{ALWAYS}} points to the current function's \nameref{def:Dynamic_Link}.
  The frame pointer is commonly abbreviated as \emph{FP}.
  This is the thing that allows us to access \nameref{def:Local_Variable}s, \nameref{def:Function_Argument}s, and everything else inside of the \nameref{def:Stack_Frame}.
  The value is held in the \rbp{} register.
\end{definition}

\begin{definition}[Stack Pointer]\label{def:Stack_Pointer}
  The \emph{stack pointer} is a pointer that \textbf{\emph{ALWAYS}} points to the top of the current \nameref{def:Stack_Frame}.
  The stack pointer is commonly abbreviated as \emph{SP}.
  This value is held in the \rsp{} register.
  This is the pointer that allows us to push and pop onto this function's \nameref{def:Stack_Frame}.
\end{definition}

\begin{definition}[Class Descriptor]\label{def:Class_Descriptor}
  The \emph{class descriptor} is a portion of memory set aside for the methods that are in an object.
\end{definition}

\subsubsection{Heap}\label{subsubsec:Memory_Heap}
\begin{definition}[Heap]\label{def:Heap}
  The \emph{heap} is a memory organization construct that helps visualize the way memory is used during the execution of a program.
  It is \textbf{\emph{SIGNIFICANTLY}} larger than the \nameref{def:Call_Stack}.

  In an object-oriented language, this is where instances of classes exist.
  These objects are allocated on a first-come first-serve basis.
  When the object is allocated, the amount of memory that the object requires is allocated \textbf{\emph{in a continuous block}}.
  The problem is that if the heap has been extensively used, then there might be enough total memory required to allocate an object, but because it is not in a continuous block, the object cannot be allocated.
  
  This makes deallocation more complex, because when they are deallocated, the program might still view the memory as in-use.
  Additionally, because of the continuous block allocation nature of the heap, it must be organized every once in a while.
  This is called \nameref{def:Garbage_Collection}.

  \begin{remark}
    In an object-oriented language, like Java, when objects are stored on the heap, \textbf{ONLY} their class fields are stored there.
    The class methods are stored elsewhere, in the \nameref{def:Class_Descriptor}.
  \end{remark}
\end{definition}

\subsection{Disk}\label{subsec:Disk}
\begin{definition}[Non-Volatile]\label{def:Non-Volatile}
  If a data storage mechanism is called \emph{non-volatile}, it means that once the storage device loses power, the data is still safely stored.
  This is in contrast to \nameref{def:Volatile} data storage mechanisms.
\end{definition}

\subsection{Fetch-Execute Cycle}\label{subsec:Fetch_Execute_Cycle}
%%% Local Variables:
%%% mode: latex
%%% TeX-master: shared
%%% End:


% To make this print, you must include a citation somewhere in the document
\clearpage
\printbibliography{}
\end{document}

%%% Local Variables:
%%% mode: latex
%%% TeX-master: t
%%% End:
