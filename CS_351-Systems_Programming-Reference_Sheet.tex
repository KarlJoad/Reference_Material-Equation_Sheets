\documentclass[10pt,letterpaper,final,twoside,notitlepage]{article}
\usepackage[margin=.5in]{geometry} % 1/2 inch margins on all pages
\usepackage[utf8]{inputenc} % Define the input encoding
\usepackage[USenglish]{babel} % Define language used
\usepackage{amsmath,amsfonts,amssymb}
\usepackage{amsthm} % Gives us plain, definition, and remark to use in \theoremstyle{style}
\usepackage{mathtools} % Allow for text and math in align* environment.
\usepackage{thmtools}
\usepackage{thm-restate}
\usepackage{graphicx}

\usepackage[
backend=biber,
style=alphabetic,
citestyle=authoryear]{biblatex} % Must include citation somewhere in document to print bibliography
\usepackage{hyperref} % Generate hyperlinks to referenced items
\usepackage[nottoc]{tocbibind} % Prints the Reference/Bibliography in TOC as well
\usepackage[noabbrev,nameinlink]{cleveref} % Fancy cross-references in the document everywhere
\usepackage{nameref} % Can make references by name to places
\usepackage{caption} % Allows for greater control over captions in figure, algorithm, table, etc. environments
\usepackage{subcaption} % Allows for multiple figures in one Figure environment
\usepackage[binary-units=true]{siunitx} % Gives us ways to typeset units for stuff
\usepackage{csquotes} % Context-sensitive quotation facilities
\usepackage{enumitem} % Provides [noitemsep, nolistsep] for more compact lists
\usepackage{chngcntr} % Allows us to tamper with the counter a little more
\usepackage{empheq} % Allow boxing of equations in special math environments
\usepackage[x11names]{xcolor} % Gives access to coloring text in environments or just text, MUST be before tikz
\usepackage{tcolorbox} % Allows us to create boxes of various types for examples
\usepackage{tikz} % Allows us to create TikZ and PGF Pictures
\usepackage{ctable} % Greater control over tables and how they look
\usepackage{diagbox} % Allow us to have shared diagonal cells in tables
\usepackage{multirow} % Allow us to have a single cell in a table span multiple rows
\usepackage{titling} % Put document information throughout the document programmatically
\usepackage[linesnumbered,ruled,vlined]{algorithm2e} % Allows us to write algorithms in a nice style.

\counterwithin{figure}{section}
\counterwithin{table}{section}
\counterwithin{equation}{section}
\counterwithin{algocf}{section}
\crefname{algocf}{algorithm}{algorithms}
\Crefname{algocf}{Algorithm}{Algorithms}
\setcounter{secnumdepth}{4}
\setcounter{tocdepth}{4} % Include \paragraph in toc
\crefname{paragraph}{paragraph}{paragraphs}
\Crefname{paragraph}{Paragraph}{Paragraphs}

% Create a theorem environment
\theoremstyle{plain}
\newtheorem{theorem}{Theorem}[section]
% Create a numbered theorem-like environment for lemmas
\newtheorem{lemma}{Lemma}[theorem]

% Create a definition environment
\theoremstyle{definition}
\newtheorem{definition}{Defn}
\newtheorem{corollary}{Corollary}[section]
% \begin{definition}[Term] \label{def:}
%   Make sure the term is emphasized with \emph{term}.
%   This ensures that if \emph is changed, it shows up everywhere
% \end{definition}

% Create a numbered remark environment numbered based on definition
% NOTE: This version of remark MUST go inside a definition environment
\theoremstyle{remark}
\newtheorem{remark}{Remark}[definition]
%\counterwithin{definition}{subsection} % Uncomment to have definitions use section.subsection numbering

% Create an unnumbered remark environment for general use
% NOTE: This version of remark has NO restrictions on placement
\newtheorem*{remark*}{Remark}

% Create a special list that handles properties. It can be broken and restarted
\newlist{propertylist}{enumerate}{1} % {Name}{Template}{Max Depth}
% [newlistname, LevelsToApplyTo]{formatting options}
\setlist[propertylist, 1]{label=\textbf{(\roman*)}, ref=\textbf{(\roman*)}, noitemsep, nolistsep}
\crefname{propertylisti}{property}{properties}
\Crefname{propertylisti}{Property}{Properties}

% Create a special list that handles enumerate starting with lower letters. Breakable/Restartable.
\newlist{boldalphlist}{enumerate}{1} % {Name}{Template}{Max Depth}
% [newlistname, LevelsToApplyTo]{formatting options}
\setlist[boldalphlist, 1]{label=\textbf{(\alph*)}, ref=\alph*, noitemsep, nolistsep} % Set options

\newlist{nocrefenumerate}{enumerate}{1} % {Name}{Template}{Max Depth}
% [newlistname, LevelsToApplyTo]{formatting options}
\setlist[nocrefenumerate, 1]{label=(\arabic*), ref=(\arabic*), noitemsep, nolistsep}

% Create a list that allows for deeper nesting of numbers. By default enumerate only allows depth=4.
\newlist{nestednums}{enumerate}{6}
% [newlistname, LevelsToApplyTo]{formatting options}
\setlist[nestednums]{noitemsep, label*=\arabic*.}

\tcbuselibrary{breakable} % Allow tcolorboxes to be broken across pages
% Create a tcolorbox for examples
% /begin{example}[extra name]{NAME}
% Create a tcolorbox for examples
% Argument #1 is optional, given by [], that is the textbook's problem number
% Argument #2 is mandatory, given by {}, that is the title for the example
% Avoid putting special characters, (), [], {}, ",", etc. in the title.
\newtcolorbox[auto counter,
number within=section,
number format=\arabic,
crefname={example}{examples}, % Define reference format for cref (No Capitals)
Crefname={Example}{Examples}, % Reference format for cleveref (With Capitals)
]{example}[2][]{ % The [2][] Means the first argument is optional
  width=\textwidth,
  title={Example \thetcbcounter: #2. #1}, % Parentheses and commas are not well supported
  fonttitle=\bfseries,
  label={ex:#2},
  nameref=#2,
  colbacktitle=white!100!black,
  coltitle=black!100!white,
  colback=white!100!black,
  upperbox=visible,
  lowerbox=visible,
  sharp corners=all,
  breakable
}

% Create a tcolorbox for general use
\newtcolorbox[% auto counter,
% number within=section,
% number format=\arabic,
% crefname={example}{examples}, % Define reference format for cref (No Capitals)
% Crefname={Example}{Examples}, % Reference format for cleveref (With Capitals)
]{blackbox}{
  width=\textwidth,
  % title={},
  fonttitle=\bfseries,
  % label={},
  % nameref=,
  colbacktitle=white!100!black,
  coltitle=black!100!white,
  colback=white!100!black,
  upperbox=visible,
  lowerbox=visible,
  sharp corners=all
}

% Redefine the 'end of proof' symbol to be a black square, not blank
\renewcommand{\qedsymbol}{$\blacksquare$} % Change proofs to have black square at end

% Common Mathematical Stuff
\newcommand{\Abs}[1]{\ensuremath{\lvert #1 \rvert}}
\newcommand{\DNE}{\ensuremath{\mathrm{DNE}}} % Used when limit of function Does Not Exist

% Complex Numbers functions
\renewcommand{\Re}{\operatorname{Re}} % Redefine to use the command, but not the fraktur version
\renewcommand{\Im}{\operatorname{Im}} % Redefine to use the command, but not the fraktur version
\newcommand{\Real}[1]{\ensuremath{\Re \lbrace #1 \rbrace}}
\newcommand{\Imag}[1]{\ensuremath{\Im \lbrace #1 \rbrace}}
\newcommand{\Conjugate}[1]{\ensuremath{\overline{#1}}}
\newcommand{\Modulus}[1]{\ensuremath{\lvert #1 \rvert}}
\DeclareMathOperator{\PrincipalArg}{\ensuremath{Arg}}

% Math Operators that are useful to abstract the written math away to one spot
% Number Sets
\DeclareMathOperator{\RealNumbers}{\ensuremath{\mathbb{R}}}
\DeclareMathOperator{\AllIntegers}{\ensuremath{\mathbb{Z}}}
\DeclareMathOperator{\PositiveInts}{\ensuremath{\mathbb{Z}^{+}}}
\DeclareMathOperator{\NegativeInts}{\ensuremath{\mathbb{Z}^{-}}}
\DeclareMathOperator{\NaturalNumbers}{\ensuremath{\mathbb{N}}}
\DeclareMathOperator{\ComplexNumbers}{\ensuremath{\mathbb{C}}}
\DeclareMathOperator{\RationalNumbers}{\ensuremath{\mathbb{Q}}}

% Calculus operators
\DeclareMathOperator*{\argmax}{argmax} % Thin Space and subscripts are UNDER in display

% Signal and System Functions
\DeclareMathOperator{\UnitStep}{\mathcal{U}}
\DeclareMathOperator{\sinc}{sinc} % sinc(x) = (sin(pi x)/(pi x))

% Transformations
\DeclareMathOperator{\Lapl}{\mathcal{L}} % Declare a Laplace symbol to be used

% Logical Operators
\DeclareMathOperator{\XOR}{\oplus}

% x86 CPU Registers
\newcommand{\rbpRegister}{\texttt{\%rbp}}
\newcommand{\rspRegister}{\texttt{\%rsp}}
\newcommand{\ripRegister}{\texttt{\%rip}}
\newcommand{\raxRegister}{\texttt{\%rax}}
\newcommand{\rbxRegister}{\texttt{\%rbx}}

%%% Local Variables:
%%% mode: latex
%%% TeX-master: shared
%%% End:


% These packages are more specific to certain documents, but will be availabe in the template
% \usepackage{esint} % Provides us with more types of integral symbols to use
\usepackage[outputdir=./TeX_Output]{minted} % Allow us to nicely typeset 300+ programming languages
\crefname{lstlisting}{listing}{listings}
\Crefname{lstlisting}{Listing}{Listings}
% This document must be compiled with the -shell-escape flag if the packages above are uncommented

% \graphicspath{{./Drawings/Course}} % Uncomment this to use pictures in this document
% \addbibresource{./Bibliographies/CourseNum-Name.bib}

\usemintedstyle{emacs} % Best for use on white backgrounds. Use for inline-code
% This macro creates the 2 minted environments for kernel source code with common options
\def\mintedkernelargs{
  frame=lines, % Surround the source code with lines on top and bottom
  linenos, % We want to show line numbers for each line in the margin
  % Colors used here are xcolor X11 colors.
  % style=fruity, % Use the fruity color scheme. Best for use on black backgrounds. Use for code blocks.
  % bgcolor=black, % Set the background used
  style=emacs,
  bgcolor=white,
  autogobble=true, % Automatically remove shared indentation from files
  breaklines=true, % Break lines that are too long at convenient locations
}
\newcommand{\makenewmintedfiles}[1]{
  \newminted[csource]{c}{#1} % Use with \begin{csource} code \end{csource}

  \newmintedfile[csourcefile]{c}{#1} % Use with \csourcefile[additional-options]{Filename}
}
\expandafter\makenewmintedfiles\expandafter{\mintedkernelargs}
\newmintinline[cinline]{c}{% Use with \cinline{code}
  style=emacs,
  bgcolor=white,
}

% Math Operators that are useful to abstract the written math away to one spot
% These are supposed to be document-specific mathematical operators that will make life easier
% Many fundamental operators are defined in Reference_Sheet_Preamble.tex

\begin{titlepage}
  \title{CS 351: Systems Programming --- Reference Material \\ Illinois Institute of Technology}
  \author{Karl Hallsby}
  \date{Last Edited: \today} % We want to inform people when this document was last edited
\end{titlepage}

\begin{document}
\pagenumbering{gobble}
\maketitle
\pagenumbering{roman} % i, ii, iii on beginning pages, that don't have content
\tableofcontents
\clearpage
\listoftheorems[ignoreall, show={definition, Definition}]
\clearpage
\listoflistings{}
\clearpage
\pagenumbering{arabic} % 1,2,3 on content pages

\section{C Programming}\label{sec:C_Programming}
C is one of the lowest ``high level'' languages you can use today.
It provides very minimal abstractions from hardware and assembly code, but allows you to relatively good typechecked code.

\subsection{Memory Allocation}\label{subsec:Memory_Allocation}
Because C is a language that does not provide many abstractions, it also requires the programmer to remember and manage their memory usage.
So, \textbf{YOU} must be the one to manage the memory, there is \textbf{NO} built-in garbage collector for you to use.

Memory allocation is done on the heap of the program's execution space in memory.
When you allocate memory in your program, you are actually requesting the operating system to give you the memory you want.

\subsubsection{\texttt{malloc}}\label{subsubec:malloc}
This is the simplest function of all possible memory allocation functions.
\texttt{malloc}:
\begin{itemize}
\item Takes one argument:
  \begin{enumerate}
  \item The number of bytes to allocate.
  \end{enumerate}
\item Returns a \textbf{POINTER} to the front of the allocated memory.
\end{itemize}

\texttt{malloc} {\large{\textbf{\emph{DOES NOT}}}} initialize memory, so it will be garbage.

\subsubsection{\texttt{calloc}}\label{subsubsec:calloc}
This is quite similar to malloc.
\texttt{calloc}:
\begin{itemize}
\item Takes 2 arguments:
  \begin{enumerate}
  \item The number of spaces to allocate, for example the number of elements in an array.
  \item The number of bytes to allocate, for the type being stored.
  \end{enumerate}
\item Returns a \textbf{POINTER} to the front of the allocated memory.
\end{itemize}

\texttt{calloc} {\large{\textbf{\emph{ZEROS}}}} memory, so this does have a slight performance penalty.

\subsubsection{\texttt{realloc}}\label{subsubsec:realloc}
\texttt{realloc} is used to \textbf{REALLOCATE} an existing memory location.
\begin{itemize}
\item Takes 2 arguments:
  \begin{enumerate}
  \item The pointer to the memory location previously allocated with either \texttt{malloc} or \texttt{calloc}.
  \item The amount of memory to reallocate, in bytes.
  \end{enumerate}
\item If the \texttt{NULL} pointer is passed to \texttt{realloc}, it will behave exactly like \texttt{malloc}.
\item Returns a \textbf{POINTER} to the front of the reallocated memory
\end{itemize}

\subsubsection{\texttt{free}}\label{subsubsec:free}
\texttt{free} is used to free memory that was previously allocated, removing from the programming space entirely.
\begin{itemize}
\item Takes 1 argument:
  \begin{enumerate}
  \item A pointer to the memory to be deallocated.
  \end{enumerate}
\item Returns \texttt{void}.
\end{itemize}

%%% Local Variables:
%%% mode: latex
%%% TeX-master: "../C_Programming"
%%% End:


%%% Local Variables:
%%% mode: latex
%%% TeX-master: shared
%%% End:


\section{Processes}\label{sec:Processes}
\nameref{def:Process}es are the fundamental unit of computation within an operating system.
\begin{definition}[Process]\label{def:Process}
  A \emph{process} is a \nameref{def:Program} in execution.
  A process carries out the computation that we specify.
  A process contains:
  \begin{itemize}[noitemsep]
  \item Code (\texttt{text}) of your program.
  \item Runtime data (Global, local, dynamic variables)
  \item \nameref{def:Register}s:
    \begin{itemize}[noitemsep]
    \item \nameref{def:Program_Counter} (\texttt{PC})
    \item \nameref{def:Stack_Pointer} (\texttt{SP})
    \item \nameref{def:Frame_Pointer} (\texttt{FP})
    \end{itemize}
  \item \nameref{def:Process_Control_Block}
  \end{itemize}
\end{definition}

\begin{definition}[Program]\label{def:Program}
  A \emph{program} is the binary image stored at some file location on the storage medium.
  The program is read into memory, and then is used to start a \nameref{def:Process} that runs that program.
\end{definition}

\nameref{def:Process}es require both a \textbf{predictable} and \textbf{logical} control flow.
This means:
\begin{itemize}[noitemsep]
\item The \nameref{def:Process} must start somewhere, typically defined to be \texttt{main}.
\item Nothing can disrupt a program mid-execution.
  \begin{itemize}[noitemsep]
  \item This is further discussed in \Cref{subsec:Prevent_Process_Disruption}.
  \end{itemize}
\end{itemize}

\subsection{Prevent Process Disruption}\label{subsec:Prevent_Process_Disruption}
The easiest way to prevent a \nameref{def:Process} from having its control flow being interrupted is for the process to ``own'' the CPU for the entire duration of the process's execution.
However, this means:
\begin{itemize}[noitemsep]
\item No other process can run on this core
\item This prevents efficient multi-\nameref{def:Process}/multitasking systems
\item Malicious or poorly written program can ``take over'' the CPU
\item An idle process (for example, waiting for user input) will underutilize the CPU
\end{itemize}

For the operating system to simulate this seamless logical control flow, we use all of the information used to make a \nameref{def:Process}, and need a \nameref{def:Process_Control_Block}.
\begin{definition}[Process Control Block]\label{def:Process_Control_Block}
  The \emph{Process Control Block} (\emph{PCB}) contains additional metadata about a \nameref{def:Process}.
  This includes:
  \begin{itemize}[noitemsep]
  \item Process ID (\texttt{PID})
  \item CPU Usage
  \item Memory Usage
  \item Pending \nameref{def:Syscall}s
  \end{itemize}
\end{definition}

The \nameref{def:Process_Control_Block} is sued to allow \nameref{def:Process}es to be interrupted, saved, and moved off a core.
This allows the operating system to schedule processes according to some algorithm.

\begin{definition}[Syscall]\label{def:Syscall}
  A \emph{syscall}, short for a \emph{system call} is a way for a user-level \nameref{def:Process} to perform some computation that the operating system kernel restricts.
  Some common syscalls are:
  \begin{itemize}[noitemsep]
  \item Opening a file
  \item Reading a file
  \item Writing a file
  \item Closing a file
  \item Creating a process
  \item Changing the process's binary
  \item Reading from the network
  \item Writing to the network
  \end{itemize}
\end{definition}

\subsection{Required Hardware}\label{subsec:Required_Hardware}
Interrupting the execution of a \nameref{def:Process} requires some hardware support to be possible and efficient.
We need 3 things:
\begin{enumerate}[noitemsep]
\item A hardware mechanism to periodically interrupt the current \nameref{def:Process} to change execution to the operating system.
  \begin{itemize}[noitemsep]
  \item This is usually the \emph{periodic clock interrupt}.
  \end{itemize}
\item An operating system procedure to decide which processes to run, and in which order.
  \begin{itemize}[noitemsep]
  \item This is the operating system's \nameref{def:Process_Scheduler}.
  \end{itemize}
\item A routine for seamlessly switching between processes.
  \begin{itemize}[noitemsep]
  \item This is called a \nameref{def:Context_Switch}.
  \item Relatively speaking, these are expensive to perform.
  \item \textbf{These are external to a \nameref{def:Process}'s logical control flow}.
  \item This forms part of the process of \nameref{def:Exceptional_Control_Flow}.
  \item A \nameref{def:Context_Switch} makes no guarantee about if and/or when this \nameref{def:Process} will start running again.
  \item A \nameref{def:Context_Switch} is the only way to invoke \nameref{def:Syscall}s.
  \end{itemize}
\end{enumerate}

To schedule \nameref{def:Process}es onto one of possibly many CPUs, there are programs called \nameref{def:Process_Scheduler}s.

Every time the \nameref{def:Process_Scheduler} schedules a new \nameref{def:Process}, or when a process needs to perform a \nameref{def:Syscall}, or a kernel-level exception occurs, a \nameref{def:Context_Switch} is made.
\begin{definition}[Context Switch]\label{def:Context_Switch}
  A \emph{context switch} is the process of interrupting a \nameref{def:Process}, saving its current state, and scheduling something else to run on that CPU.\@
  This is done whenever a \nameref{def:Syscall} is made, and happens sometimes during a process's lifetime.
\end{definition}

\nameref{def:Context_Switch}es form a core part of the \nameref{def:Exceptional_Control_Flow}.

\subsection{Process Scheduling}\label{subsec:Process_Scheduling}
Process scheduling is the process by which a process is put ``scheduled'' onto a particular CPU, according to some criteria.
There are any number of scheduling algorithms, each of which optimizes for certain cases, and may yield a different order of process execution.
\begin{definition}[Process Scheduler]\label{def:Process_Scheduler}
  The \emph{process scheduler} is an operating system component that chooses which \nameref{def:Process} to run next.
  It chooses this according to some algorithm, each of which might change the order of process execution.
\end{definition}

One such \nameref{def:Process_Scheduler} is the \nameref{subsubsec:Priority_Scheduling} algorithm.

\subsubsection{Priority Scheduling}\label{subsubsec:Priority_Scheduling}
\emph{Priority scheduling} involves placing a priority on every \nameref{def:Process} that can be scheduled.
Then, the process with the highest priority is chosen first, working our way down to the lowest priority.
This is, partly, the setup most modern operating systems take today.

However, priority scheduling creates new issues that must be dealt with.
One of these is \nameref{def:Starvation}.

\begin{definition}[Starvation]\label{def:Starvation}
  \emph{Starvation} is when something that needs a resource to function does not receive this resource.
\end{definition}

In this case, a lower-priority \nameref{def:Process} can experience \nameref{def:Starvation} if only higher-priority processes are present, and continually steal CPU execution time.

\subsection{Exceptional Control Flow}\label{subsec:Exceptional_Control_Flow}
To illustrate the power of \nameref{def:Exceptional_Control_Flow}, we will use an example piece of code, \Cref{lst:Exceptional_Control_Flow}.

\begin{definition}[Exceptional Control Flow]\label{def:Exceptional_Control_Flow}
  \emph{Exceptional control flow} is designated by the fact that most of the computation involved is done in response to exceptions or special events.
  Which one depends on what ``exceptional'' means in that circumstance.
\end{definition}

\begin{listing}[h!tbp]
\csourcefile{./CS_351-Systems_Programming-Sections/Processes/Code/Exceptional_Control_Flow.c}
\caption{Exceptional Control Flow Example}
\label{lst:Exceptional_Control_Flow}
\end{listing}


%%% Local Variables:
%%% mode: latex
%%% TeX-master: "../CS_351-Systems_Programming-Reference_Sheet"
%%% End:


%====================================APPENDIX====================================
\appendix
\counterwithin{definition}{subsection}

\clearpage
\section{Computer Components}\label{app:Computer_Components}
\subsection{Central Processing Unit}\label{subsec:CPU}
\begin{definition}[Central Processing Unit]\label{def:CPU}
  The \emph{Central Processing Unit}, \emph{CPU}, is a chip that performs all actions in the computer.
  It calculates mathematical and logical values and acts based on them.
  It has several components built onto it, and can be thought of as the ``brain'' of the computer.

  The design of a CPU determines some of the functionality it has.
  Therefore, more specialized processors can be made for special tasks, and more general processors can be built to handle a wide variety of calculations.
\end{definition}

\subsubsection{Registers}\label{subsubsec:Registers}
\begin{definition}[Register]\label{def:Register}
  A \emph{register} is a data storage mechanism built directly onto the \nameref{def:CPU}.
  It is several hundred times faster than the system \nameref{def:Memory}.
  Registers are generally used when the currently running program is performing calculations.
  Since they are so fast, they are used as both source and destination operands in instructions.

  \begin{remark}
    Depending on the \nameref{def:CPU} architecture, there may be cases when \nameref{def:Register}s behave slightly differently between processors.
    This is something that can only be found by checking the \nameref{def:CPU} manufacturer's documentation.
  \end{remark}
\end{definition}

\subsubsection{Program Counter}\label{subsubsec:Program_Counter}
\subsubsection{Arithmetic Logic Unit}\label{subsubsec:ALU}
\subsubsection{Cache}\label{subsubsec:CPU_Cache}

\subsection{Memory}\label{subsec:Memory}
\begin{definition}[Memory]\label{def:Memory}
  \emph{Memory}, or \emph{RAM} (\emph{Random Access Memory}), is a \nameref{def:Volatile} data storage mechanism.
  It is directly connected to the \nameref{def:CPU}.
  This is the location that the \nameref{def:CPU} writes to when it cannot or should not store something in the \nameref{def:CPU}'s \nameref{def:Register}s.

  \begin{remark}[Volatility]
    \nameref{def:Memory} is volatile because each of the cells is a small capacitor.
    In between the clock cycles on the \nameref{def:CPU} and \nameref{def:Memory}, the capacitors discharge.
    On the clock cycle, the capacitors are refreshed with electrical power, which does one of 2 things:
    \begin{enumerate}[noitemsep]
    \item Keep the data bits the same, 1 to 1.
    \item Update the data bits from 0 to 1.
    \end{enumerate}
  \end{remark}
\end{definition}

\begin{definition}[Volatile]\label{def:Volatile}
  If a data storage mechanism is called \emph{volatile}, it means that once the storage mechanism loses power, the data is lost.
  This is in contrast to \nameref{def:Non-Volatile} data storage mechanisms.
\end{definition}

\subsection{Disk}\label{subsec:Disk}
\begin{definition}[Non-Volatile]\label{def:Non-Volatile}
  If a data storage mechanism is called \emph{non-volatile}, it means that once the storage device loses power, the data is still safely stored.
  This is in contrast to \nameref{def:Volatile} data storage mechanisms.
\end{definition}

\subsection{Fetch-Execute Cycle}\label{subsec:Fetch_Execute_Cycle}
%%% Local Variables:
%%% mode: latex
%%% TeX-master: shared
%%% End:


% To make this print, you must include a citation somewhere in the document
\clearpage
\printbibliography{}
\end{document}

%%% Local Variables:
%%% mode: latex
%%% TeX-master: t
%%% End:
