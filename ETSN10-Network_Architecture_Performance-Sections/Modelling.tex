\section{Modelling}\label{sec:Modelling}
In our discussion of models, we have commonly referred to the OSI model for TCP/IP networking.
This is just one model, we have also used the queuing model to measure system performance.
But, just what is a model?
\begin{definition}[Model]\label{def:Model}
  A \emph{model} is a system that is constructed to have some certain similar properties to the system in reality.
  A model is a way to remove certain factors from the system, or limit them, so the overall system can be tested in a variety of ways.
\end{definition}

Models are used when the real-world experiments would be:
\begin{itemize}[noitemsep]
\item Too big
\item Take too much time
\item No longer exists
\item Dangerous
\item Unethical
\item Not easily reproducible
\item Difficult to analyze
\item Limiting
\end{itemize}

In this course, we will be looking at 2 major models of CSMA-based network traffic.
\begin{enumerate}[noitemsep]
\item \nameref{subsec:Kleinrock_Tobagi_Model}
\item \nameref{subsec:Bianchi_Model}
\end{enumerate}

\subsection{Model Creation}\label{subsec:Model_Creation}
\nameref{def:Model}s must be created by following the steps below.
\begin{enumerate}[noitemsep]
\item Determine the purpose of the model
\item Work out which aspects of reality need to be included, and which can be ignored or simplified
\item Choose the type of model or methodology
\item Build the model
\item Analyze the Model, Compare to reality, make predictions
\item Refine the model
\end{enumerate}

%%% Local Variables:
%%% mode: latex
%%% TeX-master: "../ETSN10-Network_Architecture_Performance-Reference_Sheet"
%%% End:
