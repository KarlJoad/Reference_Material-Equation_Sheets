\section{Modeling}\label{sec:Modeling}
In our discussion of models, we have commonly referred to the OSI model for TCP/IP networking.
This is just one model, we have also used the queuing model to measure system performance.
But, just what is a model?
\begin{definition}[Model]\label{def:Model}
  A \emph{model} is a system that is constructed to have some certain similar properties to the system in reality.
  A model is a way to remove certain factors from the system, or limit them, so the overall system can be tested in a variety of ways.
\end{definition}

Models are used when the real-world experiments would be:
\begin{itemize}[noitemsep]
\item Too big
\item Take too much time
\item No longer exists
\item Dangerous
\item Unethical
\item Not easily reproducible
\item Difficult to analyze
\item Limiting
\end{itemize}

In this course, we will be looking at 2 major models of CSMA-based network traffic.
\begin{enumerate}[noitemsep]
\item \nameref{subsec:Kleinrock_Tobagi_Model}
\item \nameref{subsec:Bianchi_Model}
\end{enumerate}

\subsection{Model Creation}\label{subsec:Model_Creation}
\nameref{def:Model}s must be created by following the steps below.
\begin{enumerate}[noitemsep]
\item Determine the purpose of the model
\item Work out which aspects of reality need to be included, and which can be ignored or simplified
\item Choose the type of model or methodology
\item Build the model
\item Analyze the Model, Compare to reality, make predictions
\item Refine the model
\end{enumerate}

The purpose of the model \textbf{MUST} be correctly determined.
Otherwise, the whole model will not correctly model what you want it to.

\subsection{Analyzing Models}\label{subsec:Model_Analysis}
Once a model has been constructed and has been running, whatever that may mean in its context, it must be analyzed and compared to reality.
\begin{enumerate}[noitemsep]
\item What is being modeled?
\item What is the purpose for developing a model?
\item What aspects of reality are needed? What simplifications and assumptions can be made?
\item What techniques are available to build the model? How can we build it?
\item How is the model evaluated?
\item Can it produce results that can be validated (compared to reality or an already-trusted model)?
\item Can we prove the validity of the model?
\item Can it make predictions that we can test?
\end{enumerate}

\subsection{Simulations}\label{subsec:Simulations}
\begin{definition}[Simulation]\label{def:Simulation}
  A \emph{simulation} is a \nameref{def:Model} that has some dynamic components, which change over time.
  These are typically implemented in software, but some systems are implemented with hardware too.

  \begin{remark}[Stochastic Simulations]\label{rmk:Stochastic_Simulations}
    Simulations do not necessarily behave the same way each time, instead they follow stochastic processes.
  \end{remark}
\end{definition}

\subsection{Kleinrock and Tobagi's Model of Packet-Switched Radio Networks}\label{subsec:Kleinrock_Tobagi_Model}
The Kleinrock and Tobagi model was published in 1975.

\subsubsection{What is Being Modeled?}\label{subsubsec:Kleinrock_Tobagi_Model_What_Modeled}
\begin{itemize}[noitemsep]
\item Packet-switched radio systems using various protocols:
  \begin{itemize}[noitemsep]
  \item \nameref{def:ALOHA}
  \item \nameref{par:Slotted_ALOHA}
  \item \nameref{def:Non_Persistent} CSMA
  \item 1-Persistent CSMA
  \item \nameref{def:p_Persistent} CSMA
\end{itemize}

\item Users in line-of-sight and range of each other, communicating over a broadcast radio channel
\end{itemize}

\subsubsection{What is the Purpose of the Model?}\label{subsubsec:Kleinrock_Tobagi_Model_Purpose}
\begin{itemize}[noitemsep]
\item To investigate the throughput and delay characteristics of the various protocols
\item Find the channel capacity (maximum throughput)
\end{itemize}

\subsubsection{Assumptions and Simplifications}\label{subsubsec:Kleinrock_Tobagi_Model_Assumptions}
\begin{itemize}[noitemsep]
\item Physics: Assume a spherical
  \begin{itemize}[noitemsep]
  \item A terminal cannot transmit and receive simultaneously.
    \begin{itemize}[noitemsep]
    \item This is actually true for wireless LANs today
    \item Radios that can transmit and receive simultaneously tend to be more complicated and expensive.
    \item Own transmission drowns out received transmission
    \end{itemize}
  \item The delay incurred by switching from one mode to the other is negligible.
    \begin{itemize}[noitemsep]
    \item What do we mean by negligible? How small must the delay be in reality for this assumption to be reasonable?
    \end{itemize}
  \item The channel is assumed to be noiseless.
    \begin{itemize}[noitemsep]
    \item This is sometimes true (or almost) and sometimes very much untrue.
    \item This assumption limits the applicability of the model to situations where noise is negligible.
    \end{itemize}
  \item The propagation delay is identical for all sender-receiver pairs.
    \begin{itemize}[noitemsep]
    \item By taking the largest used propagation delay, we can get
      pessimistic bounds from the model.
    \end{itemize}
  \item The time required to detect the carrier is negligible.
  \end{itemize}

\item Protocol: This is out of scope of this work, aka Someone Else’s Problem
  \begin{itemize}[noitemsep]
  \item Any overlap in two (or more) packets causes destructive interference in both (all) packets and they must be retransmitted.
    \begin{itemize}[noitemsep]
    \item The validity of this assumption depends on the error detection and correction schemes used.
    \item We are therefore examining these protocols in the absence of any such schemes to deal with partial destruction of a packet.
    \end{itemize}
  \item The system is synchronised and all transmissions begin at the
    start of a slot.
    \begin{itemize}[noitemsep]
    \item Synchronisation is thus not included in the model, but it
      needs to be part of any real system that the model can be reasonably
      applied to.
    \end{itemize}
  \item There is a positive acknowledgement scheme, i.e. a message is assumed lost unless an acknowledgement is sent.
    \begin{itemize}[noitemsep]
    \item Again, the acknowledgement scheme is not examined in the model, but it needs to be present for the model to be applicable to a real system.
    \end{itemize}

  \item Acknowledgement packets are always correctly received.
  \item The processing time required to check a packet is correct and generate an acknowledgement is negligible.
  \item The average retransmission delay is large compared to the packet transmission time.
    \begin{itemize}[noitemsep]
    \item This reduces the number of cases to deal with later. Retransmissions can essentially be treated just like any other packet.
    \end{itemize}
  \end{itemize}

\item System inputs and outputs: Play nice
  \begin{itemize}[noitemsep]
  \item All packets are of constant length.
    \begin{itemize}[noitemsep]
    \item This is not true in reality.
    \item What do we lose and what do we gain by using this assumption?
    \end{itemize}
  \item The traffic source consists of an infinite number of users who collectively form an independent Poisson source.
    \begin{itemize}[noitemsep]
    \item This is perhaps the biggest simplification in the whole model, and demonstrably false in reality
    \item But we gain a huge amount of analytic tractability
    \item How much does the traffic model affect the results?
    \end{itemize}

  \item Steady state conditions are assumed to exist.
    \begin{itemize}[noitemsep]
    \item Kleinrock and Tobagi note in the paper that the protocols they examine have in fact been shown to be unstable.
    \item However, the results are nonetheless useful in finite cases where a steady state may indeed be reached.
    \end{itemize}
  \end{itemize}
\end{itemize}

\subsubsection{Model Throughputs}\label{subsubsec:Kleinrock_Tobagi_Model_Throughputs}
In all of equations of throughputs, they are true for when $a = 0$.

\nameref{def:ALOHA}
\begin{equation}\label{eq:ALOHA_Throughput}
  S = Ge^{-G}
\end{equation}

\nameref{par:Slotted_ALOHA}
\begin{equation}\label{eq:Slotted_ALOHA_Throughput}
  S = G e^{-2G}
\end{equation}

\nameref{def:Non_Persistent} CSMA
\begin{equation}\label{eq:Non_Persistent_CSMA_Throughput}
  S_{N} = \frac{G}{1+G}
\end{equation}

1-Persistent CSMA
\begin{equation}\label{eq:1_Persistent_CSMA_Throughput}
  S_{1} = \frac{G e^{-G} (1+G)}{G + e^{-G}}
\end{equation}

\nameref{def:p_Persistent} CSMA
\begin{equation}\label{eq:p_Persistent_CSMA_Throughput}
  S_{p} = \frac{G e^{-G} (1+G) Gx}{G+e^{-G}}
\end{equation}

\subsubsection{Model Evaluation}\label{subsubsec:Kleinrock_Tobagi_Model_Eval}
\begin{itemize}[noitemsep]
\item Validity: Is the model correct?
  \begin{itemize}[noitemsep]
  \item Mathematical proof
  \item Simulation results
  \end{itemize}

\item Usefulness: What can we do with the model? Did it achieve
  its aims?
  \begin{itemize}[noitemsep]
  \item The model does indeed produce results for throughput and delay, which can be used in designing packet radio networks.
  \end{itemize}
\end{itemize}

\subsection{Bianchi Model of 802.11}\label{subsec:Bianchi_Model}
This model was published in 2000, after 802.11 was already in use, but not well understood analytically.

\subsubsection{What is Being Modeled}\label{subsubsec:Bianchi_Model_What_Modeled}
\begin{itemize}[noitemsep]
\item 802.11 Distributed Co-ordination Function
  \begin{itemize}[noitemsep]
  \item Both basic access and with RTS/CTS, as well as a combinainon of the two where packets over a certain length use RTS/CTS.
  \end{itemize}
\item Saturation conditions: This means that every node always has
  a packet ready to send.
  \begin{itemize}[noitemsep]
  \item Note that maximum throughput does not occur at saturation!
  \item Too many packets being sent means we lose a lot of packets to collisions. Throughput occurs when there is a balance between offered load and collision probability.
  \end{itemize}
\end{itemize}

\subsubsection{What is the Purpose of the Model?}\label{subsubsec:Bianchi_Model_Purpose}
\begin{itemize}[noitemsep]
\item Determine the saturation throughput
  \begin{itemize}[noitemsep]
  \item Maximum load the system can carry under stable
    conditions
  \end{itemize}
\item Determine the optimal transmission probability each station should adopt for a given network scenario (number of stations)
\item Produce a model that is both accurate and simple, so that it
  is easy to use.
\end{itemize}

\subsubsection{Assumptions and Simplifications}\label{subsubsec:Bianchi_Model_Assumptions}
\begin{itemize}[noitemsep]
\item Ideal channel conditions — no hidden terminals
\item Finite number of terminals
\item Constant and independent collision probability of a packet transmitted by a station, regardless of previous retransmissions
\item All packets have a fixed size
\item Neglect CTS and ACK timeouts after collision
\item Probability of a collision between more than two packets in the same slot is negligible
\item Every station always has a packet waiting to be transmitted
  \begin{itemize}[noitemsep]
  \item This is the key assumption for this model.
  \item The assumption of saturation conditions greatly reduces the
    scope and thus leads to tractable analysis
  \end{itemize}
\end{itemize}

\subsubsection{Model Evaluation}\label{subsubsec:Bianchi_Model_Eval}
\begin{itemize}[noitemsep]
\item Validity
  \begin{itemize}[noitemsep]
  \item Model results compared with results from simulation of 802.11 DCF
  \end{itemize}

\item Usefulness
  \begin{itemize}[noitemsep]
  \item Able to produce performance analysis results with comparatively simple analysis
  \end{itemize}
\end{itemize}

\subsection{Comparing the Two Models}\label{subsec:Comparing_CSMA_Models}
\begin{itemize}[noitemsep]
\item The two models have different purposes, which affects
  \begin{itemize}[noitemsep]
  \item Methodolgy
  \item Assumptions
  \item Evaluation
\end{itemize}

\item The \nameref{subsec:Kleinrock_Tobagi_Model} is more complete and general than Bianchi’s: can model a variety of different protocols under a range of conditions
\item But it is also much more complex and many more assumptions and simplificatinos were required, and they were more drastic than those in the \nameref{subsec:Bianchi_Model}.
\item Both models used simulations in their evaluation. This means they used models to evaluate other models!
\end{itemize}

%%% Local Variables:
%%% mode: latex
%%% TeX-master: "../ETSN10-Network_Architecture_Performance-Reference_Sheet"
%%% End:
