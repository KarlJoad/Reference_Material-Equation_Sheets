\subsection{Random Access}\label{subsec:Random_Access}
\begin{definition}[Random Access]\label{def:Random_Access}
  \emph{Random Access} is the process by which a user can request resources from a system for a duration of time.
  After this time period has passed, the resources are freed from that user and may be passed to another.
  These are also called \emph{Contention-Based protocols}.

  This allows for sharing the same resources among many users, but there may be contention for the resources.
  In addition, because there is not way to arbitrate the delivery of data, there may be packet collisions when multiple transmitters send data, and the receiver cannot decode the data from just one of them.
\end{definition}

\subsubsection{Fundamentals of ALOHA}\label{subsubsec:ALOHA_Fundamentals}
\begin{itemize}[noitemsep]
\item Full duplex communication.
  \begin{itemize}[noitemsep]
  \item Up-link (Upload)
  \item Down-link (Download)
  \end{itemize}

\item Originally UHF (Ultra-High Frequency) communication
\item Incoming traffic (to base station):
  \begin{itemize}[noitemsep]
  \item 9,600 bps (403.500 MHz)
  \item Random access (Pure ALOHA)
  \end{itemize}
\item Outgoing traffic (from base station):
  \begin{itemize}[noitemsep]
  \item 9,600 bps (413.474 MHz)
  \item Broadcast
  \end{itemize}
\end{itemize}

\paragraph{Pure ALOHA}\label{par:Pure_ALOHA}
\begin{itemize}[noitemsep]
\item Whenever a station has something to send, it sends.
\item When a packet is received by the central, an acknowledgment (\texttt{ACK}) is sent back in broadcast.
\item If the sending station does not receive an \texttt{ACK} within a set time, a collision is assumed.
\item When a collision occurs, retransmit within a random time slot, 200--1500 ms.
\end{itemize}

\begin{remark*}
  We wait a random amount of time before transmitting to prevent retransmission collisions.
  If 2 nodes were to experience a collision, and both waited a fixed amount of time, then they would collide a again during their retransmission.
\end{remark*}

\paragraph{Slotted ALOHA}\label{par:Slotted_ALOHA}
\begin{itemize}[noitemsep]
\item Packets may just be transmitted within time slots.
\item If a station has started to transmit in a time slot, other station who wish to transmit within this time slot can not interfere.
\item This principle leads to a much better utilization of the channel.
\end{itemize}

The utilization gets so much better because there is \textbf{no chance} of a collision.
Since each station can only transmit if no other station is transmitting in this time slot, there cannot be collisions part-way through a packet's transmission.

%%% Local Variables:
%%% mode: latex
%%% TeX-master: "../../ETSN10-Network_Architecture_Performance-Reference_Sheet"
%%% End:
