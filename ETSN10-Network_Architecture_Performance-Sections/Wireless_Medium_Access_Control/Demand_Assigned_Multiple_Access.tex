\subsection{Demand Assigned Multiple Access}\label{subsec:DAMA}
The use of time reservation can increase the efficiency of a system up to 80\%.
To do this,
\begin{itemize}[noitemsep]
\item A sender \textbf{reserves} a future timeslot.
\item Sending within this timeslot is possible without any collision.
\item However, the act of having to reserve something increases the delay.
\item This is typically used in satellite links.
\end{itemize}

There are 2 reservation algorithms for requesting a timeslot:
\begin{enumerate}[noitemsep]
\item \nameref{subsubsec:TDMA_Explicit_Reservation}
\item \nameref{subsubsec:TDMA_Reservation_Time}
\end{enumerate}

\subsubsection{Explicit Reservation}\label{subsubsec:TDMA_Explicit_Reservation}
This scheme is similar to the way \nameref{def:ALOHA} works.
Each node requests a time slot on-the-fly.
There are 2 main implementations:
\begin{enumerate}[noitemsep]
\item \nameref{def:ALOHA} mode
  \begin{itemize}[noitemsep]
  \item This means that time slots are assigned as they are needed.
  \item However, if multiple nodes was the same timeslot, there may be a competition.
  \item In addition, multiple timeslots can be reserved at once.
  \item The controlling system will grant access to anyone that asks for it, without question.
  \item However, there may be collision/overlapping timeslot between 2 nodes.
  \end{itemize}

\item Reserved mode
  \begin{itemize}[noitemsep]
  \item Data transmission can only happen inside of successfully reserved slots.
  \item The controlling system will schedule and grant access to these based on some algorithm.
  \item This means that there can be no collisions between timeslots.
  \end{itemize}
\end{enumerate}

\subsubsection{Reservation Time}\label{subsubsec:TDMA_Reservation_Time}
In this scheme, each major time interval is a frame.
Inside this frame, there are $N$ mini-slots and $x$ data-slots, where $N$ is the number of nodes in the network.
Each station gets its own mini-slot, and \textbf{must reserve} up to $k$ data slots, so $x \geq Nk$.

If there are unused data-slots, then they can be assigned to other stations that require the additional bandwidth.

%%% Local Variables:
%%% mode: latex
%%% TeX-master: "../../ETSN10-Network_Architecture_Performance-Reference_Sheet"
%%% End:
