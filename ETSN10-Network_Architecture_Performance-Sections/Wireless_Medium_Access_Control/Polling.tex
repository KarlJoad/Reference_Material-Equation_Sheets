\subsection{Polling}\label{subsec:Polling}
\begin{definition}[Polling]\label{def:Polling}
  \emph{Polling} is the act of having one terminal that can be heard by all others ask for a small amount of information.
  This ``central'' terminal, e.g.\ a base station can poll all other terminals according to a certain scheme.
  This is a typical server-client workflow, where the server is the one polling all the clients regularly.
\end{definition}

We are usually interested in the effective performance of the polling system.
To calculate this, we can use \Cref{eq:Polling_Effective_Performance}.
\begin{equation}\label{eq:Polling_Effective_Performance}
  E = \frac{T_{t}}{T_{t} + T_{\mathrm{Idle}} + T_{\mathrm{Poll}}}
\end{equation}
where
\begin{itemize}[noitemsep]
\item $T_{t}$: Transmission time of the polling packet.
\item $T_{\mathrm{Idle}}$: Time waiting to detect no transmission.
\item $T_{\mathrm{Poll}}$: Time to send the polling message.
\end{itemize}

\subsubsection{Randomly Addressed Polling}\label{subsubsec:Randomly_Addressed_Polling}
To perform randomly addressed polling:
\begin{enumerate}[noitemsep]
\item Base station signals readiness to all mobile terminals
\item Terminals ready to send transmit a random number without collision with the help of \nameref{def:CDMA} or \nameref{def:FDMA}
  \begin{itemize}[noitemsep]
  \item The random number can be seen as dynamic address
  \end{itemize}

\item Base station chooses one address for polling from list of all random numbers
  \begin{itemize}[noitemsep]
  \item Collision if two terminals choose the same address
  \end{itemize}

\item Base station acknowledges correct packets and continues polling the next terminal
\item Cycle starts again after polling all terminals of the list
\end{enumerate}

\subsubsection{Point Coordination Function}\label{subsubsec:Point_Coordination_Function}
\begin{definition}[Point Coordination Function]\label{def:PCF}
  \emph{Point Coordination Function} or \emph{PCF} is a centralized \nameref{def:Polling} mechanism for WiFi (802.11).
  The central poller is the current Access Point (AP).

  In this system, there are 2 types of tranmission periods.
  The first is during the Point Coordinated Function, when the resources on the AP are reserved for users.
  The second is a pure \nameref{def:Random_Access} period.
  Here, the rules of a random access sytem are obeyed.

  \begin{itemize}[noitemsep]
  \item Advantages
    \begin{itemize}[noitemsep]
    \item Resource management. Resources are only given to transmitters who request them.
    \end{itemize}

  \item Disadvantages
    \begin{itemize}[noitemsep]
    \item Traffic Separation. There is no way to separate all the traffic flows.
    \end{itemize}
  \end{itemize}
\end{definition}

\paragraph{Problems with the \nameref*{def:PCF}}\label{par:PCF_Problems}
The performance of \nameref{def:PCF} can be quite poor, especially when used for voice communications.
When a PCF system is running, and a poll occurs, the users that have nothing to send return null packets.
While these packets are empty (contain no information), they are sitll packets that need to be transmitted, thus taking up tranmission time.

Additionally, there is no Quality of Service (QoS) signalling, which can further degrade performance.

There is also no tranmission time limit.
If a user is polled, and it has something to transmit, the transmission will continue until the user is done.
This incurs a cost, because other nodes may not have their chance to transmit during the \nameref{def:PCF} window, and must wait for the \nameref{def:Random_Access} window, or must wait until the next round of polling.

%%% Local Variables:
%%% mode: latex
%%% TeX-master: "../../ETSN10-Network_Architecture_Performance-Reference_Sheet"
%%% End:
