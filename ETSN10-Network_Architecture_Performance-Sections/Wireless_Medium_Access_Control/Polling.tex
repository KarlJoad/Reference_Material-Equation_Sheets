\subsection{Polling}\label{subsec:Polling}
\begin{definition}[Polling]\label{def:Polling}
  \emph{Polling} is the act of having one terminal that can be heard by all others ask for a small amount of information.
  This ``central'' terminal, e.g.\ a base station can poll all other terminals according to a certain scheme.
  This is a typical server-client workflow, where the server is the one polling all the clients regularly.
\end{definition}

We are usually interested in the effective performance of the polling system.
To calculate this, we can use \Cref{eq:Polling_Effective_Performance}.
\begin{equation}\label{eq:Polling_Effective_Performance}
  E = \frac{T_{t}}{T_{t} + T_{\mathrm{Idle}} + T_{\mathrm{Poll}}}
\end{equation}
where
\begin{itemize}[noitemsep]
\item $T_{t}$: Transmission time of the polling packet.
\item $T_{\mathrm{Idle}}$: Time waiting to detect no transmission.
\item $T_{\mathrm{Poll}}$: Time to send the polling message.
\end{itemize}

\subsubsection{Randomly Addressed Polling}\label{subsubsec:Randomly_Addressed_Polling}
To perform randomly addressed polling:
\begin{enumerate}[noitemsep]
\item Base station signals readiness to all mobile terminals
\item Terminals ready to send transmit a random number without collision with the help of \nameref{def:CDMA} or \nameref{def:FDMA}
  \begin{itemize}[noitemsep]
  \item The random number can be seen as dynamic address
  \end{itemize}

\item Base station chooses one address for polling from list of all random numbers
  \begin{itemize}[noitemsep]
  \item Collision if two terminals choose the same address
  \end{itemize}

\item Base station acknowledges correct packets and continues polling the next terminal
\item Cycle starts again after polling all terminals of the list
\end{enumerate}

%%% Local Variables:
%%% mode: latex
%%% TeX-master: "../../ETSN10-Network_Architecture_Performance-Reference_Sheet"
%%% End:
