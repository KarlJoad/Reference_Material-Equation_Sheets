\section{Queuing Theory}\label{sec:Queuing_Theory}
In queuing theory, we view a network as a collection as first-in-first-out (FIFO) data structures.
Queuing theory provides a probabilistic analysis of these queues.

\begin{definition}[Queuing System]\label{def:Queuing_System}
  In a \emph{queuing system} model, there is a FIFO queue with a task arrival rate of $\lambda$.
  The server processes these with a service time of $\mu$.
\end{definition}

\subsection{Little's Law}\label{subsec:Littles_Law}
\begin{definition}[Little's Law]\label{def:Littles_Law}\label{def:Littles_Formula}
  \emph{Little's Law} or \emph{Little's Formula} models the mean number of tasks in a queue-based system.
  This applies to \textbf{any system in equilibrium}, as long as the system does not create or destroy tasks itself.

  \begin{equation}\label{eq:Littles_Law}%\label{eq:Littles_Formula}
    r = \lambda T_{r}
  \end{equation}
  \begin{itemize}[noitemsep]
  \item $r$ --- The mean number of tasks in a queuing system.
  \item $\lambda$ --- The average arrival rate of tasks to the system.
  \item $T_{r}$ --- The mean time for which the task sits in the queue waiting.
  \end{itemize}
\end{definition}

\begin{example}[Lecture 3]{Application of Little's Law}
  If 40 customers visit a pub per hour, and these customers spend an average of 15 minutes in the pub, what is the average number of customers in the pub at any given time?
  \tcblower{}
  Using \Cref{eq:Littles_Law},
  \begin{align*}
    \lambda &= 40 \\
    T_{r} &= \frac{15}{60} = \frac{1}{4} = 0.25
  \end{align*}

  Multiplying these together,
  \begin{equation*}
    r = 40 * \frac{1}{4} = 10
  \end{equation*}
\end{example}

\subsection{Kendall's Notation}\label{subsec:Kendalls_Notation}
\begin{definition}[Kendall's Notation]\label{def:Kendalls_Notation}
  \emph{Kendall's Notation} is a shorthand to specify the characteristics of a \nameref{def:Queuing_System}.
  There are 6 symbols, where the first 2 are distributions, the next 3 are numbers, and the last is the discipline the server uses to process tasks.

  \begin{equation}\label{eq:Kendalls_Notation}
    x_{1}/x_{2}/x_{3}/x_{4}/x_{5}/x_{6}
  \end{equation}
  \begin{enumerate}[noitemsep]
  \item $x_{1}$: The arrival distribution (See \Cref{subsubsec:Distros_Kendalls_Notation}).
  \item $x_{2}$: The server's service distribution (See \Cref{subsubsec:Distros_Kendalls_Notation}).
  \item $x_{3}$: The number of servers.
  \item $x_{4}$: The total capacity of the system (Assumed to be $\infty$ if not specified).
    \begin{itemize}[noitemsep]
    \item The assumption of $\infty$ is not a bad assumption usually.
    \item Packets are usually only a couple of kibibytes, whereas main memory is gibibytes.
    \end{itemize}
  \item $x_{5}$: The population size (the total possible tasks, assumed to be $\infty$ unless specified).
  \item $x_{6}$: Service Discipline (FIFO, LIFO, etc.) (FIFO is assumed, unless specified).
  \end{enumerate}

  \begin{remark}[Shortened Kendall's Notation]\label{rmk:Short_Kendalls_Notation}
    Typically, only the first 3 terms in \nameref{def:Kendalls_Notation} are used.
    The last 3 have assumed conditions if they are not specified.
  \end{remark}
\end{definition}

\subsubsection{Distributions in Kendall's Notation}\label{subsubsec:Distros_Kendalls_Notation}
\begin{itemize}[noitemsep]
\item $M$ --- Exponential, Memoryless, Markovian, Poissonian.
\item $D$ --- Deterministic, Fixed arrival rate.
\item $E_{k}$ --- Erlang with a parameter $k$.
\item $H_{k}$ --- Hyperexponential with a parameter $k$.
\item $G$ --- General (The distribution can be anything).
\end{itemize}

Some examples are:

\begin{example}[Lecture 3]{Kendall's Notation 1}
  What does $M/M/1$ mean in \nameref{def:Kendalls_Notation}?
  \tcblower{}
  \begin{itemize}[noitemsep]
  \item Really $M/M/1/\infty/\infty/FIFO$, since last 3 not specified.
  \item Exponential arrival distribution
  \item Exponential service distribution
  \item 1 server
  \item Infinite capacity
  \item Infinite population
  \item FCFS (FIFO)
  \end{itemize}
\end{example}

\begin{example}[Lecture 3]{Kendall's Notation 2}
  What does $M/M/n$ mean in \nameref{def:Kendalls_Notation}?
  \tcblower{}
  \begin{itemize}[noitemsep]
  \item Really $M/M/n/\infty/\infty/FIFO$, since last 3 notation specified.
  \item Exponential arrival distribution
  \item Exponential service
  \item $n$ servers
  \item Infinite capacity
  \item Infinite population
  \item FCFS (FIFO)
  \end{itemize}
\end{example}
  
\begin{example}[Lecture 3]{Kendall's Notation 3}
  What does $G/G/3/20/1500/SPF$ mean in \nameref{def:Kendalls_Notation}?
  \begin{itemize}[noitemsep]
  \item General arrival distribution
  \item General service distribution
  \item 3 servers
  \item 17 queue slots ($20-3$)
  \item 1500 total jobs
  \item Shortest Packet First
  \end{itemize}
\end{example}

%%% Local Variables:
%%% mode: latex
%%% TeX-master: "../../ETSN10-Network_Architecture_Performance-Reference_Sheet"
%%% End:


\subsection{\texorpdfstring{A General $M/M/1$ Queue}{A General Queue}}\label{subsec:General_MM1_Queue}
If we have a general $M/M/1$ queue, then:
\begin{itemize}[noitemsep]
\item We may assume messages arrive at the channel according to \nameref{def:Poisson_Process} at a rate $\lambda$ messages/sec.
\item We may assume that the message lengths have a \nameref{def:Negative_Exponential_Random_Variable} with a mean of $\dfrac{1}{v}$ bits/message.
\item The channel transmits messages from its buffer at a constant rate $c$ bits/sec.
\item The buffer associated with the channel can be considered to be effectively of infinite length.
\end{itemize}

\begin{definition}[Message Transmission Rate]\label{def:Message_Transmission_Rate}
  The \emph{message tramisssion rate}, $c$, is the rate at which tasks are moved from the queue/buffer to the server.
\end{definition}

\begin{definition}[Message Arrival Rate]\label{def:Message_Arrival_Rate}
  The \emph{message arrival rate}, denoted $\mu$ is defined to be the number of messages that are received per unit time.

  \begin{equation}\label{eq:Message_Arrival_Rate}
    \mu = c v
  \end{equation}
  \begin{itemize}[noitemsep]
  \item $c$: \nameref{def:Message_Transmission_Rate}
  \item $v$: Message length
  \end{itemize}
\end{definition}

\begin{definition}[Occupancy]\label{def:Occupancy}
  The \emph{occupancy} of a system, denoted $\rho$, is the factor to which the rate of message arrivals is greater than the message transmissions after processing.

  \begin{equation}\label{eq:Occupancy}
    \rho = \frac{\lambda}{\mu}
  \end{equation}
  \begin{itemize}[noitemsep]
  \item $\lambda$: The message arrival rate.
  \item $\mu$: The message tramisssion rate after processing.
  \end{itemize}

  \begin{remark}[Steady-State Solution]
    The only steady-state solution for infinite-buffer systems occurs when $\rho < 1$.
    If $\rho \geq 1$, then packets are arriving at the same or greater rate as they are pushed out after processing.
    Meaning, eventually the queue will fill up.
  \end{remark}
\end{definition}

If we want to find the average number of messages in the system, we need to find $\ExpectedValue[R]$.
\begin{equation}\label{eq:Mean_Packets_in_Whole_System}
  \ExpectedValue[R] = \frac{\rho}{1-\rho}
\end{equation}

The average number of packets in \textbf{just} the queue is given by $\ExpectedValue[T_{W}]$.
\begin{equation}\label{eq:Mean_Packets_in_Queue}
  \ExpectedValue[T_{W}] = \frac{1}{\mu} \frac{\rho}{1-\rho}
\end{equation}

The mean delay of a packet due to the system is given by $\ExpectedValue[T_{R}]$.
\begin{equation}\label{eq:Mean_Packet_Delay}
  \ExpectedValue[T_{R}] = \frac{1}{\mu} \frac{1}{1-\rho}
\end{equation}

\begin{definition}[Probability of Delay]\label{def:MM1_Prob_Delay}
  The \emph{probability of delay for an $M/M/1$ queuing system} for a given amount of time is given by
  \begin{equation}\label{eq:MM1_Delay}
    p_{T_{R}}(t) = \mu (1-\rho) e^{-\mu (1-\rho) t}
  \end{equation}

  The probability the delay is within a given range is given by the equation below.
  \begin{equation}
    \label{eq:9}
    \Prob(T_{R} \leq t) = 1 - e^{-\mu (1-\rho) t}
  \end{equation}
\end{definition}

%%% Local Variables:
%%% mode: latex
%%% TeX-master: "../../ETSN10-Network_Architecture_Performance-Reference_Sheet"
%%% End:


\subsection{State-Dependent Queues}\label{subsec:State_Dependent_Queues}
\begin{definition}[State-Dependent Queue]\label{def:State_Dependent_Queue}
  A \emph{state-dependent queue} is a queue that behaves in certain ways depending on the state that it is in.
  It can cycle endlessly \textbf{between} states, but can never cycle on itself.

  There are 2 variables needed for each state.
  \begin{enumerate}[noitemsep]
  \item $\lambda_{i}$: The arrival rate of tasks when the system is in state $i$.
  \item $\mu_{i}$: The servers' service rate when the system is in state $i$.
  \end{enumerate}

  In each state, there needs to be a balance between the packets going to the next stage and leaving the current one and vice versa.
  \begin{equation}\label{eq:State_Dependent_Queue_Balance_Equation}
    \lambda_{i-1} p_{i-1} = \mu_{i} p_{i} \text{, for } i = 1, 2, \ldots
  \end{equation}

  The general solution to \Cref{eq:State_Dependent_Queue_Balance_Equation} is:
  \begin{equation}\label{eq:eq:State_Dependent_Queue_Balance_Equation-Solution}
    p_{i} = \frac{\lambda_{0} \lambda_{1} \cdots \lambda_{i-1}}{\mu_{1} \mu_{2} \cdots \mu_{i}} p_{0} \text{, for } i = 1, 2, \ldots
  \end{equation}
\end{definition}

\subsubsection{\texorpdfstring{A General $M/M/2$ Queue}{Parallel Servers}}\label{subsubsec:General_MM2_Queue}
A general $M/M/2$ \nameref{def:Queuing_System} can be represented as a \nameref{def:State_Dependent_Queue}.

\begin{align*}
  \lambda_{i} &= \lambda \text{, for } i = 1, 2, \ldots \\
  \mu_{i} &= \begin{cases}
    \mu & i = 1 \\
    2 \mu & i = 2, 3, \ldots \\
  \end{cases}
\end{align*}
where $i$ is the number of packets in the system, which represents a different state of the system.

Now, let's compare a $M/M/2$ \nameref{def:Queuing_System} against a $M/M/1$ \nameref{def:Queuing_System}.
\begin{align*}
  {\ExpectedValue[T_{R}]}_{\text{1 server, } 2\mu \text{ rate}} &= \frac{1}{2 \mu} \frac{1}{1-\rho} \\
  {\ExpectedValue[T_{R}]}_{\text{2 servers}} &= \frac{1}{\mu} \frac{1}{(1+\rho) (1-\rho)} \\
  \frac{{\ExpectedValue[T_{R}]}_{\text{1 server, } 2\mu \text{ rate}}}{{\ExpectedValue[T_{R}]}_{\text{2 servers}}} &= \frac{1+\rho}{2} \leq 2
\end{align*}

The conclusion here is concrete.
\begin{center}
  \textbf{\Large A single faster server is always more efficient.}
\end{center}

If we extend our findings about \nameref{def:State_Dependent_Queue}s to $M/M/\infty$ systems, the general solution becomes a \nameref{def:Poisson_Random_Variable}'s distribution.
\begin{equation}\label{eq:MMinfty_Balance_Equation-Solution}
  \begin{aligned}
    p_{i} &= \frac{\lambda_{i}}{\mu^{i} i!} p_{0} \\
    &= e^{-\frac{\lambda}{\mu}} \frac{{\left( \frac{\lambda}{\mu} \right)}^{i}}{i!} \\
    &= e^{-A} \frac{A^{i}}{i!}
  \end{aligned}
\end{equation}

%%% Local Variables:
%%% mode: latex
%%% TeX-master: "../../ETSN10-Network_Architecture_Performance-Reference_Sheet"
%%% End:


\subsection{\texorpdfstring{The Finite Buffer $M/M/1/N$ Queue}{Finite Buffer Queue}}\label{subsec:MM1n_Queue}
The $M/M/1/n$ is a queue with a finite-sized buffer.
The real question with this is what is the probability the buffer will be full and the next packet will have to be blocked.

\begin{definition}[Blocking Probability]\label{def:Blocking_Probability-Packets}
  The \emph{blocking probability} is the probability that the next packet that arrives to the system will be blocked because the queue is full.
  Instead of making the queue longer to keep track of the packets, the system just denies the packet access to the queue.

  \begin{equation}\label{eq:Blocking_Probability-Packets}
    P_{B} = \Prob[\text{Message is blocked}] = p_{n} = \frac{1-\rho}{1-\rho^{n+1}} \rho^{n}
  \end{equation}
\end{definition}

\begin{example}[Lecture 4]{Length of Queue for No Packets Blocked}
  Assume $\rho = 0.5$, and that we want a system that blocks packets with probability $P_{B} \leq 10^{-3}$.
  Find $n$, the minimum length of the queue to satisfy these requirements?
  \tcblower{}
  Using \Cref{eq:Blocking_Probability-Packets},
  \begin{align*}
    \frac{1-\rho}{1-\rho^{n+1}} \rho^{n} &= \frac{1-0.5}{1-0.5^{n+1}} 0.5^{n} \\
                                         &= \frac{0.5^{n+1}}{1-0.5^{n+1}} \leq 10^{-3} \\
    0.5^{n+1} &\leq \frac{10^{-3}}{1 + 10^{-3}} \\
    n + 1 &\geq 9.96 \\
    n &\geq 8.96
  \end{align*}

  So, $n$ should be 9.
\end{example}

\begin{remark*}
  Note that the solution found in \Cref{ex:Length of Queue for No Packets Blocked} is actually the general solution.
  Meaning, that solution for $n$ is valid for all $\rho$.
  This is because the finite buffer copes with overloading by blocking any additional messages from entering the system.
\end{remark*}

\subsubsection{Applying \nameref*{def:Littles_Law}}\label{subsubsec:Applying_Littles_Law-MM1n}
To even use \nameref{def:Littles_Law}, we must account for the blocking of messages by only counting the messages that make it into the system.

\begin{equation}\label{eq:Littles_Law-MM1n}
  \gamma = \lambda (1-P_{B})
\end{equation}

This means our equations change a little bit.
\begin{equation}\label{eq:Occupancy-MM1n}
  \rho = \frac{\lambda}{\mu} \text{, can be $\geq$ 1 r $<1$}
\end{equation}


\begin{subequations}\label{eq:Mean_Packets_in_Whole_System-MM1n}
  \begin{equation}
    \begin{aligned}
      \ExpectedValue[R] &= \gamma \ExpectedValue[T_{R}] \\
      &= \lambda (1-P_{B}) \ExpectedValue[T_{R}] \\
    \end{aligned}
  \end{equation}

  \begin{equation}
    \begin{aligned}
      \ExpectedValue[R] &= \sum\limits_{i=0}^{n} i p_{i} \\
      &= \frac{1-\rho}{1-\rho^{n+1}} \sum\limits_{i=0}^{n}i \rho_{i} \\
    \end{aligned}
  \end{equation}
\end{subequations}

%%% Local Variables:
%%% mode: latex
%%% TeX-master: "../../ETSN10-Network_Architecture_Performance-Reference_Sheet"
%%% End:


\subsection{Modeling Circuit Switching}\label{subsec:Model_Circuit_Switching}
This is easiest to visualize with the old-school telephone communications with human operators physically connecting calls to make a single large continuous circuit.

\begin{itemize}[noitemsep]
\item The rate of call initiation can be modeled as a \nameref{def:Poisson_Random_Variable}, with an arrival rate of $\lambda$.
\item The length of the calls can be modeled as a \nameref{def:Negative_Exponential_Random_Variable}, denoted as $h$, $\mu = \dfrac{1}{h}$.
\item The number of circuits is fixed at $n$.
\item If all circuits are in use, there is no queue to wait in until one is available.
\end{itemize}

This makes \nameref{def:Circuit_Switching} a $M/M/n/n$ \nameref{def:Queuing_System} in \nameref{def:Kendalls_Notation}.

\begin{definition}[Offered Traffic]\label{def:Offered_Traffic}
  \emph{Offered traffic} is the amount of traffic that the end-users are offering the network.

  \begin{equation}\label{eq:Offered_Traffic}
    A = \lambda h
  \end{equation}
\end{definition}

\begin{definition}[Carried Traffic]\label{def:Carried_Traffic}
  \emph{Carried traffic} is the amount of traffic that the network is currently handling.

  \begin{equation}\label{eq:Carried_Traffic}
    A_{C} = A (1-P_{B})
  \end{equation}
\end{definition}

\begin{definition}[Lost Traffic]\label{def:Lost_Traffic}
  \emph{Lost traffic} is traffic that is lost because all the resources available have been allocated and no more are available.
  This cannot be regained later because there is no queue available to handle the things that arrive to the system after all resources have been allocated.

  \begin{equation}\label{eq:Lost_Traffic}
    L = A - A_{C}
  \end{equation}

  \begin{remark}[Validity]\label{rmk:Validity_Lost_Traffic}
    \Cref{eq:Lost_Traffic} is valid for all non-negative exponential distributions.
    However, it is invalidated if repeats occur at all, or if they are allowed to occur, occur to soon.
  \end{remark}
\end{definition}

Let the \nameref{def:Random_Variable} $R$ represent the number of customers currently in the system, meaning $R$ is between $0$ and $n$.
The service rate for this is now
\begin{equation*}
  i \mu \text{, when } R = i
\end{equation*}

\begin{definition}[Blocking Probability]\label{def:Blocking_Probability-Circuit}
  The \emph{blocking probability} is the probability that the next thing that comes to the system will be blocked because all resources have already been allocated and there is no queue.

  \begin{equation}\label{eq:Blocking_Probability-Circuit}
    \begin{aligned}
      P_{B} &= p_{n} \\
      &= \frac{\frac{A^{n}}{n!}}{\sum\limits_{i=0}^{n} \frac{A^{j}}{j!}}
    \end{aligned}
  \end{equation}
\end{definition}

\begin{definition}[Time Congestion]\label{def:Time_Congestion}
  \emph{Time congestion} represents the proportion of time that all the circuits are busy.
  This is only viewed from the system-side, so the end-user will never know about time congestion.

  To find the time congestion, it is simply the expected value of the \nameref{def:Offered_Traffic}, when in state $n$.

  \begin{equation}\label{eq:Time_Congestion}
    \begin{aligned}
      \ExpectedValue_{n}[A] &= P_{B} \\
      &= p_{n} \\
      &= \frac{\frac{A^{N}}{N!}}{\sum\limits_{j=0}^{n}\frac{A^{j}}{j!}}
    \end{aligned}
  \end{equation}
\end{definition}

\begin{definition}[Call Congestion]\label{def:Call_Congestion}
  \emph{Call congestion} is the proportion of calls that find the system busy.
  This is viewed from the end-user side.
\end{definition}

\begin{remark*}
  For arrivals that follow a \nameref{def:Poisson_Random_Variable}'s distribution, \nameref{def:Time_Congestion} = \nameref{def:Call_Congestion}.
  This follows according to the \nameref{thm:PASTA_Theorem}.
\end{remark*}

\begin{theorem}[PASTA Theorem]\label{thm:PASTA_Theorem}
  
\end{theorem}

%%% Local Variables:
%%% mode: latex
%%% TeX-master: "../../ETSN10-Network_Architecture_Performance-Reference_Sheet"
%%% End:


\subsection{Queuing Networks}\label{subsec:Queuing_Networks}
\begin{definition}[Queuing Network]\label{def:Queuing_Network}
  A \emph{queuing network} is a network of nodes in which each node is a queue.
  The output of one queue is connected to the input of another node's queue.

  There are 2 types of queuing networks:
  \begin{enumerate}[noitemsep]
  \item \nameref{def:Closed_Queuing_Network}
  \item \nameref{def:Open_Queuing_Network}
  \end{enumerate}

  \begin{remark}[Limiting Analysis]\label{rmk:Limiting_Queuing_Networks}
    We will only consider $M/M/n$ queues here.
    Analysis of more general queueing networks gets complicated fast.
  \end{remark}
\end{definition}

\begin{definition}[Closed Queuing Network]\label{def:Closed_Queuing_Network}
  A \emph{closed queuing network} is one in which packets (customers, tasks, etc.) can never enter or leave the \nameref{def:Queuing_Network}.
\end{definition}

\begin{definition}[Open Queuing Network]\label{def:Open_Queuing_Network}
  An \emph{open queuing network} is a \nameref{def:Queuing_Network} that is not closed.
  This means that an open queuing network is one in which packets (customers, tasks, etc.) may enter a node's queue from outside the \nameref{def:Queuing_Network}, and after processing, may leave the \nameref{def:Queuing_Network} entirely.
\end{definition}

\begin{theorem}[Burke's Theorem]\label{thm:Burkes_Theorem}
  The interdeparture times from a Markovian ($M/M/n$) queue are exponentially distributed with the same parameter as the interarrival times.
  Meaning
  \begin{equation}\label{Burkes_Theorem}
    \lambda_{\mathrm{Out}} = \lambda_{\mathrm{In}}
  \end{equation}

  \begin{remark}
    This allows us to analyze each queue/node independently.
    We do not need to consider all the nodes at the same time.
  \end{remark}
\end{theorem}

\subsubsection{Jackson Networks}\label{subsubsec:Jackson_Networks}
\begin{definition}[Jackson Network]\label{def:Jackson_Network}
  A \emph{Jackson network} is an \nameref{def:Open_Queuing_Network} of $M/M/n$ queues.
  Each node can receive traffic from other nodes and from outside the network.
  Traffic from each node can go to other nodes, or leave the network entirely.
\end{definition}

\begin{itemize}[noitemsep]
\item $N$: The total number of nodes in a \nameref{def:Jackson_Network}.
\item $\lambda_{i}$: Total incoming average traffic rate to node $i$.
\item $gamma_{i}$: Average rate of traffic entering node $i$ from outside the network
\item $r_{ij}$: Probability a packet leaving node $i$ will then go to node $j$
  \begin{itemize}[noitemsep]
  \item Note r need not be 0 – a packet may immediately return to the node it just left
  \end{itemize}
\item So the probability that a packet leaves the network after leaving node $i$ is given by
  \begin{equation}\label{eq:Prob_Packet_LEaves_Network}
    1 - \sum\limits_{j=1}^{N} r_{ij}
  \end{equation}

\item You can find the total incoming traffic to node $i$ by summing all incoming streams together.
  \begin{equation}\label{eq:Total_Incoming-Traffic}
    \lambda_{i} = \gamma_{i} + \sum\limits_{j=1}^{N} \lambda_{i} r_{ji} \text{, for } i = 1, 2, \ldots, N
  \end{equation}

\item In general, the whole network will not behave like a \nameref{def:Poisson_Process}.
\item Jackson, the namesake of the \nameref{def:Jackson_Network}, showed that each individual node behaves as though it were.
\item The state variable for the entire system of $N$ nodes consists of the vector
  \begin{equation}\label{eq:Queuing_Network_State_Vector}
    \langle k_{1}, k_{2}, \ldots, k_{N} \rangle
  \end{equation}
  where $k_{i}$ is the number of packets (including those currently in service) at the $i$th node.

\item Let the equilibrium probability associated with a given state be denoted
  \begin{equation}\label{eq:Queuing_Network_Equilibrium_Probability}
    \Prob \bigl[ \langle k_{1}, k_{2}, \ldots, k_{N} \rangle \bigr]
  \end{equation}
  and the probability of finding $k_{i}$ customers in the $i$th node be
  \begin{equation*}
    \Prob_{i}[k_{i}]
  \end{equation*}

\item The joint probability distribution for all nodes is the product of the distributions for the individual nodes
  \begin{equation*}
    \Prob \bigl[ \langle k_{1}, k_{2}, \ldots, k_{N} \rangle \bigr] = \Prob_{1}[k_{1}] \Prob_{2}[k_{2}] \cdots \Prob_{N}[k_{N}]
  \end{equation*}
  and each $\Prob_{i}[k_{i}]$ is given by the solution to the classical $M/M/n$ system.
\item So we can treat each node as an $M/M/n$ queue, even though the input is not necessarily Markovian.
\end{itemize}

%%% Local Variables:
%%% mode: latex
%%% TeX-master: "../../ETSN10-Network_Architecture_Performance-Reference_Sheet"
%%% End:


\subsection{Queuing Disciplines}\label{subsec:Queuing_Disciplines}
\begin{definition}[Queuing Discipline]\label{def:Queuing_Discipline}
  A \emph{queuing discipline} is a way to order the packets to send them in a particular order.
  Depending on how we order, namely how we schedule the packets, we achieve different things.
  However, the overarching requirement is to minimize delay, improve throughput, increase busy time, and decrease idle time.

  There are 2 types of queuing disciplines:
  \begin{enumerate}[noitemsep]
  \item \nameref{def:Work_Conserving}
  \item \nameref{def:Non_Work_Conserving}
  \end{enumerate}
\end{definition}

\begin{definition}[Work Conserving]\label{def:Work_Conserving}
  A \emph{work conserving} \nameref{def:Queuing_Discipline} always handles packets right away.
  Meaning, if there is a packet waiting in the queue and the server is sitting idle, it will always serve the packet.

  This is in contrast to a \nameref{def:Non_Work_Conserving} \nameref{def:Queuing_Discipline}.
\end{definition}

\begin{definition}[Non-Work Conserving]\label{def:Non_Work_Conserving}
  A \emph{non-work conserving} \nameref{def:Queuing_Discipline} may not handle packets right away.
  Meaning, if there is a packet waiting in the queue and the server is sitting idle, the server might not server the packet right away.
  Packets are switched only at \textbf{the right time towards the right destination}.

  While this type of \nameref{def:Queuing_Discipline} may not make intuitive sense, it has some advantages.
  \begin{itemize}[noitemsep]
  \item Reduces jitter
  \item It \emph{shapes} the traffic, making it predictable by holding traffic back until a certain time.
  \end{itemize}

  This is contrast to a \nameref{def:Work_Conserving} \nameref{def:Queuing_Discipline}.
\end{definition}

Traditionally, \nameref{def:FIFO} was used.
But, there are some other ones possible that improve the \nameref{def:Queuing_System} in various ways.

\subsubsection{First-In First-Out}\label{subsubsec:FIFO}
\begin{definition}[First-In First-Out]\label{def:FIFO}
  In a \emph{first-in first-out} \nameref{def:Queuing_Discipline}, the first packet that enters the queue is the first one served.
  This is a \nameref{def:Work_Conserving} \nameref{def:Queuing_Discipline}.
  This follows Kleinrock's Conservation Law.
  \begin{equation}\label{eq:Kleinrocks_Conservation_Law}
    C = \sum\limits_{n=1}^{N} \rho_{n} q_{n}
  \end{equation}
  \begin{itemize}[noitemsep]
  \item $C$: A constant value of throughput for the system.
  \item $\rho$: The \nameref{def:Occupancy} for each flow.
  \item $q$: The mean scheduler delay.
  \end{itemize}
  
  However, there are some problems:
  \begin{itemize}[noitemsep]
  \item Small packets are held up by large packets ahead of them in the queue. This leads to:
    \begin{itemize}[noitemsep]
    \item Larger average delays for smaller packets.
    \item Greater throughput for larger packets.
    \item Flows of greedy packets get better service.
    \end{itemize}
  \item Greedy TCP connections crowd out more altruistic connections.
    \begin{itemize}[noitemsep]
    \item If one connection does not back off, others may back off more.
    \item This is a feature of the TCP protocol, because this manages network congestion.
    \end{itemize}
  \end{itemize}
\end{definition}

\subsubsection{Fair Queuing (FQ)}\label{subsubsec:Fair_Queuing}
\begin{definition}[Fair Queuing]\label{def:Fair_Queuing}
  The \emph{fair queuing} \nameref{def:Queuing_Discipline} is a \nameref{def:Work_Conserving} one.
  There are multiple queues for each port, one for each source or flow of data to that port.
  The various queues are serviced in a round-robin fashion, where one packet from a queue is handled in a single cycle, before moving onto the next queue.
  This achieves load balancing among the different flows, meaning there is no advantage for a particular flow to be greedy.

  However, the drawback with this is that short packets are penalized as only one packet from one queue is sent per round.
\end{definition}

\subsubsection{Priority Queuing (PQ)}\label{subsubsec:Priority_Queuing}
\begin{definition}[Priority Queuing]\label{def:Priority_Queuing}
  \emph{Priority queuing} is a \nameref{def:Work_Conserving} \nameref{def:Queuing_Discipline}.
  There are $K$ queues that lead to the server.
  The $k$th queue is constrained by $1 \leq k \leq K$.
  The $k+1$th queue has a higher priority than the $k$th queue.
  The higher priority queue is served first.

  This has a simple implementation with low processing overhead.
  There needs to be an extra step where the packets are sorted into the correct priority queue, but that is quite efficient now.

  However, there is no fairness among the packets.
  The lower priority queues can be starved of service time if the higher queues are always full.
\end{definition}

\subsubsection{Processor Sharing (PS)}\label{subsubsec:Processor_Sharing}
\begin{definition}[Processor Sharing]\label{def:Processor_Sharing}
  A \emph{processor sharing} \nameref{def:Queuing_Discipline} is \nameref{def:Work_Conserving}.
  However, it is not practical, because it is modeled on sending individual bits rather than whole packets.
  It is more of a meta-\nameref{def:Queuing_Discipline}, because it forms the basis for \nameref{def:Bit_Round_Fair_Queuing}.

  In this discipline, there are multiple queues, just like in \nameref{def:Fair_Queuing}.
  One bit from each queue is sent per round of the round-robin.
  This prevents longer packets from getting an advantage.

  The system needs to figure out the virtual start and finish times for a given packet, $i$, of the $\alpha$ queue.
  \begin{equation}\label{eq:PS_Finish_Time}
    F_{i}^{\alpha} = S_{i}^{\alpha} + P_{i}^{\alpha}
  \end{equation}
  where
  \begin{itemize}[noitemsep]
  \item $F_{i}^{\alpha}$: The $\alpha$th queue's $i$th packet's finish time.
  \item $S_{i}^{\alpha}$: The $\alpha$th queue's $i$th packet's start time, which is defined in \Cref{eq:PS_Start_Time}.
  \item $P_{i}^{\alpha}$: The $\alpha$th queue's $i$th packet's time to send.
  \end{itemize}

  \begin{equation}\label{eq:PS_Start_Time}
    S_{i}^{\alpha} = \max \left( F_{i-1}^{\alpha}, A_{i}^{\alpha} \right)
  \end{equation}
\end{definition}

\subsubsection{Bit-Round Fair Queuing (BRFQ)}\label{subsubsec:Bit_Round_Fair_Queuing}
\subsubsection{Generalized Processor Sharing (GPS)}\label{subsubsec:Generalized_Processor_Sharing}
\subsubsection{Weighted Fair Queuing (WFQ)}\label{subsubsec:Weighted_Fair_Queuing}
\subsubsection{Class-Based Queuing (CBQ)}\label{subsubsec:Class_Based_Queuing}
%%% Local Variables:
%%% mode: latex
%%% TeX-master: "../../ETSN10-Network_Architecture_Performance-Reference_Sheet"
%%% End:


\subsection{Actual IP Network Behavior}\label{subsec:Actual_IP_Behavior}
Actual IP traffic doesn't actually follow a \nameref{def:Poisson_Random_Variable}'s distribution very well.
It has:
\begin{itemize}[noitemsep]
\item Self-similarity
\item Lots of dependency (Current traffic depends on the previous traffic).
\item Infinite variance.
\end{itemize}

\begin{definition}[Autocorrelation]\label{def:Autocorrelation}
  \emph{Autocorrelation} is a function that measures the similarity of a function with itself, typically over a period of time.

  \begin{equation}\label{eq:Autocorrelation}
    R_{x, x}(t_{1}, t_{2}) = \ExpectedValue[X(t_{1}), X(t_{2})]
  \end{equation}
\end{definition}

\begin{definition}[Long-Range Dependence]\label{def:Long_Range_Dependence}
  A process that has \emph{long-range dependence} has an autocorrelation that decays slowly as the difference between $t_{1}$ and $t_{2}$ tends towards $\infty$.
  Mathematically, these processes are characterized by an atocorrelation function that decays hyperbolically as $k$ increases.
  This is represented with \Cref{eq:Non_Summability_Correlation}, the \emph{Non-summability of correlation}.

  \begin{equation}\label{eq:Non_Summability_Correlation}
    \sum\limits_{k} r(k) = \infty
  \end{equation}

  \begin{remark}[Uncorrelated]
    If the result from \Cref{eq:Autocorrelation} is 0 when $t_{2} = t_{1} + \Delta t$, i.e.\ when the time difference is very small, then the function $X(t)$ is no correlation.
  \end{remark}

  \begin{remark}[Short-Range Correlation]
    if the result from \Cref{eq:Autocorrelation} decays to 0 very quickly, then the function $X(t)$ has \emph{short-range correlation}.
  \end{remark}
\end{definition}

\subsubsection{The Hurst Parameter}\label{subsubsec:The_Hurst_Parameter}
\begin{definition}[Hurst Parameter]\label{def:Hurst_Parameter}
  The \emph{Hurst parameter}, denoted $H$, is a measure of the ``burstiness'' of a system.
  It is also considered a measure of self-similarity.

  $H$ always lies within the range $0.5 < H < 1.0$.

  The expected value of the Hurst parameter is
  \begin{equation}\label{eq:Hurst_Parameter_Expected_Value}
    \ExpectedValue[x(t)] = \frac{\ExpectedValue[x(at)]}{a^{H}}
  \end{equation}

  The variance of the Hurst parameter is
  \begin{equation}\label{eq:Hurst_Parameter_Variance}
    \Variance[x(t)] = \frac{\Variance[x(at)]}{a^{2H}}
  \end{equation}

  The autocorrelation of the Hurst parameter is
  \begin{equation}\label{eq:Hurst_Parameter_Autocorrelation}
    R_{x, x}(t, s) = \frac{R_{x, x}(at, as)}{a^{2H}}
  \end{equation}
\end{definition}

\subsubsection{Self-Similarity and Autocorrelation}\label{subsubsec:Self_Similarity_Autocorrelation}
For this section, let $X(t)$ be a \nameref{def:Stochastic_Process} with
\begin{itemize}[noitemsep]
\item Constant mean $\mu$
\item Constant variance $\sigma^{2}$
\item An autocorrelation function depending only on $k$.
  \begin{equation*}
    r(k) = \frac{\ExpectedValue[(X(t)-\mu)(X(t+k) - \mu)]}{\ExpectedValue \left[ {\bigl( X(t) - \mu \bigr)}^{2} \right]}
  \end{equation*}
\end{itemize}

Now let $X^{(m)}$ be a new \textbf{aggregate} time series formed by averages of $X$ over non-overlapping blocks in time of size $m$.
This has some intersting implications.
\begin{itemize}[noitemsep]
\item $X$ is exactly self-similar if
  \begin{itemize}[noitemsep]
  \item The aggregated processes have the same autocorrelation structure as $X$.
  \item $r^{(m)}(k) = r(k), k \geq 0$ for all $m = 1, 2, \ldots$
  \end{itemize}

\item $X$ is asymptotically self-similar if the above holds when $r^{(m)}(k) \sim r(k), m \rightarrow \infty$
\item The most striking feature of self-similarity is correlation structures of the aggregated process do not degenerate as $m \rightarrow \infty$
\item This is in contrast to traditional models
  \begin{itemize}[noitemsep]
  \item Correlation structures of their aggregated processes degenerate, i.e.\ $r^{(m)}(k) \rightarrow 0$ as $m \rightarrow \infty, k = 1, 2, 3, \ldots$
  \end{itemize}
\end{itemize}

\subsubsection{Squared Coefficient of Variation}\label{subsubsec:Squared_Coefficient_Variation}
This is a critical factor in determining the type of \nameref{def:Queuing_System} to use.
It is a measure of variability of the system.

\begin{equation}\label{eq:Squared_Coefficient_Variation}
  c^{2} = \frac{\sigma^{2}}{\mu^{2}}
\end{equation}

The squared coefficient of variation for some common \nameref{def:Queuing_System}s are shown below.
\begin{itemize}[noitemsep]
\item $M/M/1 = 1$
\item $M/D/1 = 0$
  \begin{itemize}[noitemsep]
  \item Namely, for any system with a deterministic distribution, the squared coefficient of variation will be 0.
  \end{itemize}
\end{itemize}

%%% Local Variables:
%%% mode: latex
%%% TeX-master: "../../ETSN10-Network_Architecture_Performance-Reference_Sheet"
%%% End:


%%% Local Variables:
%%% mode: latex
%%% TeX-master: "../ETSN10-Network_Architecture_Performance-Reference_Sheet"
%%% End:
