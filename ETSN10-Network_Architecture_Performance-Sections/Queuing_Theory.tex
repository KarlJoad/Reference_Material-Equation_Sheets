\section{Queuing Theory}\label{sec:Queuing_Theory}
In queuing theory, we view a network as a collection as first-in-first-out (FIFO) data structures.
Queuing theory provides a probabilistic analysis of these queues.

\begin{definition}[Queuing System]\label{def:Queuing_System}
  In a \emph{queuing system} model, there is a FIFO queue with a task arrival rate of $\lambda$.
  The server processes these with a service time of $\mu$.
\end{definition}

\subsection{Little's Law}\label{subsec:Littles_Law}
\begin{definition}[Little's Law]\label{def:Littles_Law}\label{def:Littles_Formula}
  \emph{Little's Law} or \emph{Little's Formula} models the mean number of tasks in a queue-based system.
  This applies to \textbf{any system in equilibrium}, as long as the system does not create or destroy tasks itself.

  \begin{equation}\label{eq:Littles_Law}%\label{eq:Littles_Formula}
    r = \lambda T_{r}
  \end{equation}
  \begin{itemize}[noitemsep]
  \item $r$ --- The mean number of tasks in a queuing system.
  \item $\lambda$ --- The average arrival rate of tasks to the system.
  \item $T_{r}$ --- The mean time for which the task sits in the queue waiting.
  \end{itemize}
\end{definition}

\begin{example}[Lecture 3]{Application of Little's Law}
  If 40 customers visit a pub per hour, and these customers spend an average of 15 minutes in the pub, what is the average number of customers in the pub at any given time?
  \tcblower{}
  Using \Cref{eq:Littles_Law},
  \begin{align*}
    \lambda &= 40 \\
    T_{r} &= \frac{15}{60} = \frac{1}{4} = 0.25
  \end{align*}

  Multiplying these together,
  \begin{equation*}
    r = 40 * \frac{1}{4} = 10
  \end{equation*}
\end{example}

\subsection{Kendall's Notation}\label{subsec:Kendalls_Notation}
\begin{definition}[Kendall's Notation]\label{def:Kendalls_Notation}
  \emph{Kendall's Notation} is a shorthand to specify the characteristics of a \nameref{def:Queuing_System}.
  There are 6 symbols, where the first 2 are distributions, the next 3 are numbers, and the last is the discipline the server uses to process tasks.

  \begin{equation}\label{eq:Kendalls_Notation}
    x_{1}/x_{2}/x_{3}/x_{4}/x_{5}/x_{6}
  \end{equation}
  \begin{enumerate}[noitemsep]
  \item $x_{1}$: The arrival distribution (See \Cref{subsubsec:Distros_Kendalls_Notation}).
  \item $x_{2}$: The server's service distribution (See \Cref{subsubsec:Distros_Kendalls_Notation}).
  \item $x_{3}$: The number of servers.
  \item $x_{4}$: The total capacity of the system (Assumed to be $\infty$ if not specified).
    \begin{itemize}[noitemsep]
    \item The assumption of $\infty$ is not a bad assumption usually.
    \item Packets are usually only a couple of kibibytes, whereas main memory is gibibytes.
    \end{itemize}
  \item $x_{5}$: The population size (the total possible tasks, assumed to be $\infty$ unless specified).
  \item $x_{6}$: Service Discipline (FIFO, LIFO, etc.) (FIFO is assumed, unless specified).
  \end{enumerate}

  \begin{remark}[Shortened Kendall's Notation]\label{rmk:Short_Kendalls_Notation}
    Typically, only the first 3 terms in \nameref{def:Kendalls_Notation} are used.
    The last 3 have assumed conditions if they are not specified.
  \end{remark}
\end{definition}

\subsubsection{Distributions in Kendall's Notation}\label{subsubsec:Distros_Kendalls_Notation}
\begin{itemize}[noitemsep]
\item $M$ --- Exponential, Memoryless, Markovian, Poissonian.
\item $D$ --- Deterministic, Fixed arrival rate.
\item $E_{k}$ --- Erlang with a parameter $k$.
\item $H_{k}$ --- Hyperexponential with a parameter $k$.
\item $G$ --- General (The distribution can be anything).
\end{itemize}

%%% Local Variables:
%%% mode: latex
%%% TeX-master: "../ETSN10-Network_Architecture_Performance-Reference_Sheet"
%%% End:
