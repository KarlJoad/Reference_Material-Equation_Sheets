\section{Queuing Theory}\label{sec:Queuing_Theory}
In queuing theory, we view a network as a collection as first-in-first-out (FIFO) data structures.
Queuing theory provides a probabilistic analysis of these queues.

\begin{definition}[Queuing System]\label{def:Queuing_System}
  In a \emph{queuing system} model, there is a FIFO queue with a task arrival rate of $\lambda$.
  The server processes these with a service time of $\mu$.
\end{definition}

\subsection{Little's Law}\label{subsec:Littles_Law}
\begin{definition}[Little's Law]\label{def:Littles_Law}\label{def:Littles_Formula}
  \emph{Little's Law} or \emph{Little's Formula} models the mean number of tasks in a queue-based system.
  This applies to \textbf{any system in equilibrium}, as long as the system does not create or destroy tasks itself.

  \begin{equation}\label{eq:Littles_Law}%\label{eq:Littles_Formula}
    r = \lambda T_{r}
  \end{equation}
  \begin{itemize}[noitemsep]
  \item $r$ --- The mean number of tasks in a queuing system.
  \item $\lambda$ --- The average arrival rate of tasks to the system.
  \item $T_{r}$ --- The mean time for which the task sits in the queue waiting.
  \end{itemize}
\end{definition}

\begin{example}[Lecture 3]{Application of Little's Law}
  If 40 customers visit a pub per hour, and these customers spend an average of 15 minutes in the pub, what is the average number of customers in the pub at any given time?
  \tcblower{}
  Using \Cref{eq:Littles_Law},
  \begin{align*}
    \lambda &= 40 \\
    T_{r} &= \frac{15}{60} = \frac{1}{4} = 0.25
  \end{align*}

  Multiplying these together,
  \begin{equation*}
    r = 40 * \frac{1}{4} = 10
  \end{equation*}
\end{example}

%%% Local Variables:
%%% mode: latex
%%% TeX-master: "../ETSN10-Network_Architecture_Performance-Reference_Sheet"
%%% End:
