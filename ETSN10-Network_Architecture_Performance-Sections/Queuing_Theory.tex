\section{Queuing Theory}\label{sec:Queuing_Theory}
In queuing theory, we view a network as a collection as first-in-first-out (FIFO) data structures.
Queuing theory provides a probabilistic analysis of these queues.

\begin{definition}[Queuing System]\label{def:Queuing_System}
  In a \emph{queuing system} model, there is a FIFO queue with a task arrival rate of $\lambda$.
  The server processes these with a service time of $\mu$.
\end{definition}

\subsection{Little's Law}\label{subsec:Littles_Law}
\begin{definition}[Little's Law]\label{def:Littles_Law}\label{def:Littles_Formula}
  \emph{Little's Law} or \emph{Little's Formula} models the mean number of tasks in a queue-based system.
  This applies to \textbf{any system in equilibrium}, as long as the system does not create or destroy tasks itself.

  \begin{equation}\label{eq:Littles_Law}%\label{eq:Littles_Formula}
    r = \lambda T_{r}
  \end{equation}
  \begin{itemize}[noitemsep]
  \item $r$ --- The mean number of tasks in a queuing system.
  \item $\lambda$ --- The average arrival rate of tasks to the system.
  \item $T_{r}$ --- The mean time for which the task sits in the queue waiting.
  \end{itemize}
\end{definition}

\begin{example}[Lecture 3]{Application of Little's Law}
  If 40 customers visit a pub per hour, and these customers spend an average of 15 minutes in the pub, what is the average number of customers in the pub at any given time?
  \tcblower{}
  Using \Cref{eq:Littles_Law},
  \begin{align*}
    \lambda &= 40 \\
    T_{r} &= \frac{15}{60} = \frac{1}{4} = 0.25
  \end{align*}

  Multiplying these together,
  \begin{equation*}
    r = 40 * \frac{1}{4} = 10
  \end{equation*}
\end{example}

\subsection{Kendall's Notation}\label{subsec:Kendalls_Notation}
\begin{definition}[Kendall's Notation]\label{def:Kendalls_Notation}
  \emph{Kendall's Notation} is a shorthand to specify the characteristics of a \nameref{def:Queuing_System}.
  There are 6 symbols, where the first 2 are distributions, the next 3 are numbers, and the last is the discipline the server uses to process tasks.

  \begin{equation}\label{eq:Kendalls_Notation}
    x_{1}/x_{2}/x_{3}/x_{4}/x_{5}/x_{6}
  \end{equation}
  \begin{enumerate}[noitemsep]
  \item $x_{1}$: The arrival distribution (See \Cref{subsubsec:Distros_Kendalls_Notation}).
  \item $x_{2}$: The server's service distribution (See \Cref{subsubsec:Distros_Kendalls_Notation}).
  \item $x_{3}$: The number of servers.
  \item $x_{4}$: The total capacity of the system (Assumed to be $\infty$ if not specified).
    \begin{itemize}[noitemsep]
    \item The assumption of $\infty$ is not a bad assumption usually.
    \item Packets are usually only a couple of kibibytes, whereas main memory is gibibytes.
    \end{itemize}
  \item $x_{5}$: The population size (the total possible tasks, assumed to be $\infty$ unless specified).
  \item $x_{6}$: Service Discipline (FIFO, LIFO, etc.) (FIFO is assumed, unless specified).
  \end{enumerate}

  \begin{remark}[Shortened Kendall's Notation]\label{rmk:Short_Kendalls_Notation}
    Typically, only the first 3 terms in \nameref{def:Kendalls_Notation} are used.
    The last 3 have assumed conditions if they are not specified.
  \end{remark}
\end{definition}

\subsubsection{Distributions in Kendall's Notation}\label{subsubsec:Distros_Kendalls_Notation}
\begin{itemize}[noitemsep]
\item $M$ --- Exponential, Memoryless, Markovian, Poissonian.
\item $D$ --- Deterministic, Fixed arrival rate.
\item $E_{k}$ --- Erlang with a parameter $k$.
\item $H_{k}$ --- Hyperexponential with a parameter $k$.
\item $G$ --- General (The distribution can be anything).
\end{itemize}

Some examples are:

\begin{example}[Lecture 3]{Kendall's Notation 1}
  What does $M/M/1$ mean in \nameref{def:Kendalls_Notation}?
  \tcblower{}
  \begin{itemize}[noitemsep]
  \item Really $M/M/1/\infty/\infty/FIFO$, since last 3 not specified.
  \item Exponential arrival distribution
  \item Exponential service distribution
  \item 1 server
  \item Infinite capacity
  \item Infinite population
  \item FCFS (FIFO)
  \end{itemize}
\end{example}

\begin{example}[Lecture 3]{Kendall's Notation 2}
  What does $M/M/n$ mean in \nameref{def:Kendalls_Notation}?
  \tcblower{}
  \begin{itemize}[noitemsep]
  \item Really $M/M/n/\infty/\infty/FIFO$, since last 3 notation specified.
  \item Exponential arrival distribution
  \item Exponential service
  \item $n$ servers
  \item Infinite capacity
  \item Infinite population
  \item FCFS (FIFO)
  \end{itemize}
\end{example}
  
\begin{example}[Lecture 3]{Kendall's Notation 3}
  What does $G/G/3/20/1500/SPF$ mean in \nameref{def:Kendalls_Notation}?
  \begin{itemize}[noitemsep]
  \item General arrival distribution
  \item General service distribution
  \item 3 servers
  \item 17 queue slots ($20-3$)
  \item 1500 total jobs
  \item Shortest Packet First
  \end{itemize}
\end{example}

%%% Local Variables:
%%% mode: latex
%%% TeX-master: "../../ETSN10-Network_Architecture_Performance-Reference_Sheet"
%%% End:


\subsection{\texorpdfstring{A General $M/M/1$ Queue}{A General Queue}}\label{subsec:General_MM1_Queue}
If we have a general $M/M/1$ queue, then:
\begin{itemize}[noitemsep]
\item We may assume messages arrive at the channel according to \nameref{def:Poisson_Process} at a rate $\lambda$ messages/sec.
\item We may assume that the message lengths have a \nameref{def:Negative_Exponential_Random_Variable} with a mean of $\dfrac{1}{v}$ bits/message.
\item The channel transmits messages from its buffer at a constant rate $c$ bits/sec.
\item The buffer associated with the channel can be considered to be effectively of infinite length.
\end{itemize}

\begin{definition}[Message Transmission Rate]\label{def:Message_Transmission_Rate}
  The \emph{message tramisssion rate}, $c$, is the rate at which tasks are moved from the queue/buffer to the server.
\end{definition}

\begin{definition}[Message Arrival Rate]\label{def:Message_Arrival_Rate}
  The \emph{message arrival rate}, denoted $\mu$ is defined to be the number of messages that are received per unit time.

  \begin{equation}\label{eq:Message_Arrival_Rate}
    \mu = c v
  \end{equation}
  \begin{itemize}[noitemsep]
  \item $c$: \nameref{def:Message_Transmission_Rate}
  \item $v$: Message length
  \end{itemize}
\end{definition}

\begin{definition}[Occupancy]\label{def:Occupancy}
  The \emph{occupancy} of a system, denoted $\rho$, is the factor to which the rate of message arrivals is greater than the message transmissions after processing.

  \begin{equation}\label{eq:Occupancy}
    \rho = \frac{\lambda}{\mu}
  \end{equation}
  \begin{itemize}[noitemsep]
  \item $\lambda$: The message arrival rate.
  \item $\mu$: The message tramisssion rate after processing.
  \end{itemize}

  \begin{remark}[Steady-State Solution]
    The only steady-state solution for infinite-buffer systems occurs when $\rho < 1$.
    If $\rho \geq 1$, then packets are arriving at the same or greater rate as they are pushed out after processing.
    Meaning, eventually the queue will fill up.
  \end{remark}
\end{definition}

If we want to find the average number of messages in the system, we need to find $\ExpectedValue[R]$.
\begin{equation}\label{eq:Mean_Packets_in_Whole_System}
  \ExpectedValue[R] = \frac{\rho}{1-\rho}
\end{equation}

The average number of packets in \textbf{just} the queue is given by $\ExpectedValue[T_{W}]$.
\begin{equation}\label{eq:Mean_Packets_in_Queue}
  \ExpectedValue[T_{W}] = \frac{1}{\mu} \frac{\rho}{1-\rho}
\end{equation}

The mean delay of a packet due to the system is given by $\ExpectedValue[T_{R}]$.
\begin{equation}\label{eq:Mean_Packet_Delay}
  \ExpectedValue[T_{R}] = \frac{1}{\mu} \frac{1}{1-\rho}
\end{equation}

\begin{definition}[Probability of Delay]\label{def:MM1_Prob_Delay}
  The \emph{probability of delay for an $M/M/1$ queuing system} for a given amount of time is given by
  \begin{equation}\label{eq:MM1_Delay}
    p_{T_{R}}(t) = \mu (1-\rho) e^{-\mu (1-\rho) t}
  \end{equation}

  The probability the delay is within a given range is given by the equation below.
  \begin{equation}
    \label{eq:9}
    \Prob(T_{R} \leq t) = 1 - e^{-\mu (1-\rho) t}
  \end{equation}
\end{definition}

%%% Local Variables:
%%% mode: latex
%%% TeX-master: "../../ETSN10-Network_Architecture_Performance-Reference_Sheet"
%%% End:



%%% Local Variables:
%%% mode: latex
%%% TeX-master: "../ETSN10-Network_Architecture_Performance-Reference_Sheet"
%%% End:
