\section{Wireless Medium Access Control}\label{sec:Wireless_MAC}
We need to separate when devices are transmitting.
Due to the nature of radio waves, we cannot transmit simultaneously on the same frequency from multiple devices.

\input{./ETSN10-Network_Architecture_Performance-Sections/Wireless_Medium_Access_Control/Reservation_Schemes}

\subsection{Polling}\label{subsec:Polling}
\begin{definition}[Polling]\label{def:Polling}
  \emph{Polling} is the act of having one terminal that can be heard by all others ask for a small amount of information.
  This ``central'' terminal, e.g.\ a base station can poll all other terminals according to a certain scheme.
  This is a typical server-client workflow, where the server is the one polling all the clients regularly.
\end{definition}

We are usually interested in the effective performance of the polling system.
To calculate this, we can use \Cref{eq:Polling_Effective_Performance}.
\begin{equation}\label{eq:Polling_Effective_Performance}
  E = \frac{T_{t}}{T_{t} + T_{\mathrm{Idle}} + T_{\mathrm{Poll}}}
\end{equation}
where
\begin{itemize}[noitemsep]
\item $T_{t}$: Transmission time of the polling packet.
\item $T_{\mathrm{Idle}}$: Time waiting to detect no transmission.
\item $T_{\mathrm{Poll}}$: Time to send the polling message.
\end{itemize}

\subsubsection{Randomly Addressed Polling}\label{subsubsec:Randomly_Addressed_Polling}
To perform randomly addressed polling:
\begin{enumerate}[noitemsep]
\item Base station signals readiness to all mobile terminals
\item Terminals ready to send transmit a random number without collision with the help of \nameref{def:CDMA} or \nameref{def:FDMA}
  \begin{itemize}[noitemsep]
  \item The random number can be seen as dynamic address
  \end{itemize}

\item Base station chooses one address for polling from list of all random numbers
  \begin{itemize}[noitemsep]
  \item Collision if two terminals choose the same address
  \end{itemize}

\item Base station acknowledges correct packets and continues polling the next terminal
\item Cycle starts again after polling all terminals of the list
\end{enumerate}

%%% Local Variables:
%%% mode: latex
%%% TeX-master: "../../ETSN10-Network_Architecture_Performance-Reference_Sheet"
%%% End:


\subsection{Random Access}\label{subsec:Random_Access}
\begin{definition}[Random Access]\label{def:Random_Access}
  \emph{Random Access} is the process by which a user can request resources from a system for a duration of time.
  After this time period has passed, the resources are freed from that user and may be passed to another.
  These are also called \emph{Contention-Based protocols}.

  This allows for sharing the same resources among many users, but there may be contention for the resources.
  In addition, because there is not way to arbitrate the delivery of data, there may be packet collisions when multiple transmitters send data, and the receiver cannot decode the data from just one of them.
\end{definition}

\subsubsection{Fundamentals of ALOHA}\label{subsubsec:ALOHA_Fundamentals}
\begin{itemize}[noitemsep]
\item Full duplex communication.
  \begin{itemize}[noitemsep]
  \item Up-link (Upload)
  \item Down-link (Download)
  \end{itemize}

\item Originally UHF (Ultra-High Frequency) communication
\item Incoming traffic (to base station):
  \begin{itemize}[noitemsep]
  \item 9,600 bps (403.500 MHz)
  \item Random access (Pure ALOHA)
  \end{itemize}
\item Outgoing traffic (from base station):
  \begin{itemize}[noitemsep]
  \item 9,600 bps (413.474 MHz)
  \item Broadcast
  \end{itemize}
\end{itemize}

\paragraph{Pure ALOHA}\label{par:Pure_ALOHA}
\begin{itemize}[noitemsep]
\item Whenever a station has something to send, it sends.
\item When a packet is received by the central, an acknowledgment (\texttt{ACK}) is sent back in broadcast.
\item If the sending station does not receive an \texttt{ACK} within a set time, a collision is assumed.
\item When a collision occurs, retransmit within a random time slot, 200--1500 ms.
\end{itemize}

\begin{remark*}
  We wait a random amount of time before transmitting to prevent retransmission collisions.
  If 2 nodes were to experience a collision, and both waited a fixed amount of time, then they would collide a again during their retransmission.
\end{remark*}

\paragraph{Slotted ALOHA}\label{par:Slotted_ALOHA}
\begin{itemize}[noitemsep]
\item Packets may just be transmitted within time slots.
\item If a station has started to transmit in a time slot, other station who wish to transmit within this time slot can not interfere.
\item This principle leads to a much better utilization of the channel.
\end{itemize}

The utilization gets so much better because there is \textbf{no chance} of a collision.
Since each station can only transmit if no other station is transmitting in this time slot, there cannot be collisions part-way through a packet's transmission.

%%% Local Variables:
%%% mode: latex
%%% TeX-master: "../../ETSN10-Network_Architecture_Performance-Reference_Sheet"
%%% End:


\subsection{Energy-Efficient MACs}\label{subsec:Energy_Efficient_MACs}
In some systems, we need to make sure the power usage is minimized.
The biggest energy costs in a network are:
\begin{itemize}[noitemsep]
\item Sending Data
\item Receiving Data
\item Listening to the channel
\item Keeping the node's networking system on when it doesn't need to transmit or receive data.
\end{itemize}

\begin{remark*}
  The costs of both the transmission and reception of data are hard to minimize on a MAC level.
  These would usually be minimized by the design of hardware on the physical level and determining if it is worth it to send information at all in the application layer.
\end{remark*}

%%% Local Variables:
%%% mode: latex
%%% TeX-master: "../ETSN10-Network_Architecture_Performance-Reference_Sheet"
%%% End:
