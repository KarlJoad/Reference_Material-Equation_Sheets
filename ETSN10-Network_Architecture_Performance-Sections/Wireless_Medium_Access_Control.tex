\section{Wireless Medium Access Control}\label{sec:Wireless_MAC}
We need to separate when devices are transmitting.
Due to the nature of radio waves, we cannot transmit simultaneously on the same frequency from multiple devices.

\subsection{Reservation Schemes}\label{subsec:Reservation_Scheme}
To begin our discussion of \nameref{def:Reservation_Scheme}s, we must talk about \nameref{def:ALOHA}.

\begin{definition}[ALOHA]\label{def:ALOHA}
  \emph{ALOHA} was one of the earliest protocols for wireless data transmission.
  It used a purely random access scheme, so when you had a packet to send, it was sent.
  To ensure devices knew if their transmission was successful, an acknoledgement was sent when the next recipient in the chain received the message.

  \begin{remark}
    It is only mentioned here because it is used as a baseline measurement for other protocols.
  \end{remark}
\end{definition}

\begin{definition}[Reservation Scheme]\label{def:Reservation_Scheme}
  A \emph{reservation scheme} is the idea of reserving resources for a node in a wireless network to prevent collisions.
  The resources that can be reserved include: time, frequency, and others.

  We would like to reserve resources because the system will not lose capacity to collisions.
  But, if not \textbf{all} the resources are being fully utilized, then the system loses out on throughput.

  Because of the limitations of the radio spectrum and light being a broadcast medium, we have 4 main solutions for supporting multiple devices that was to transmit.
  \begin{enumerate}[noitemsep]
  \item \nameref{def:SDMA}
  \item \nameref{def:FDMA}
  \item \nameref{def:TDMA}
  \item \nameref{def:CDMA}
  \end{enumerate}
\end{definition}

\begin{table}[h!]
  \centering
  \begin{tabular}{p{2cm}|p{3.6cm}p{3.6cm}p{3.6cm}p{3.6cm}}
    \toprule
    \multicolumn{1}{c|}{\textbf{Approach}} & \multicolumn{1}{c}{\textbf{\nameref{subsubsec:SDMA}}} & \multicolumn{1}{c}{\textbf{\nameref{subsubsec:TDMA}}} & \multicolumn{1}{c}{\textbf{\nameref{subsubsec:FDMA}}} & \multicolumn{1}{c}{\textbf{\nameref{subsubsec:CDMA}}} \\
    \midrule
    Idea & Segment space into cells/sectors & Segment sending time into mutually disjoint time-slots, demand-driven or pattern-driven. & Segment the frequency band into disjoing sub-bands. & Spread the spectrum according to orthogonal codes. \\ \midrule
    Terminals & One one terminal can be active in a single cell/sector. & All terminals active for short time periods on the same frequency. & Every terminal has its own frequency, uninterrupted. & All terminals can be active at the same time at the same place, on the same frequency. \\ \midrule
    Signal Separation & Cell structure, directed antennas. & Synchronization in time domain. & Filtering in frequency domain. & Code plus special receivers. \\ \midrule
    Advantages & Very simple, increases capacity per unit area. & Established, \textbf{fully digital}, flexible. & Simple, established, robust. & Flexible, less frequency planning needed, soft handover. \\ \midrule
    Disadvantages & Inflexible, antennas typically fixed. & Guard time needed (multipath propagation), synchronization difficult. & Inflexible, frequencies are a scarce resource and must be decided from beginning. & Complex receivers, needs more complicated power management from senders. \\ \midrule
    Comments & Only useful in combination with \nameref{subsubsec:TDMA}, \nameref{subsubsec:FDMA}, or \nameref{subsubsec:CDMA}. & Standard in fixed networks. Used together with \nameref{subsubsec:FDMA} and \nameref{subsubsec:SDMA} in mobile networks. & Typically combined with \nameref{subsubsec:TDMA} (frequency hopping patterns) and \nameref{subsubsec:SDMA} (frequency reuse) & Has some problems still. Higher complexity, lower expectations. Needs integration with \nameref{subsubsec:TDMA}/\nameref{subsubsec:FDMA}. \\
    \bottomrule
  \end{tabular}
  \caption{Division Multiple Access Comparison}
  \label{tab:DMA_Comparison}
\end{table}

\subsubsection{SDMA}\label{subsubsec:SDMA}
\begin{definition}[Space Division Multiple Access]\label{def:SDMA}
  In a \emph{space division multiple access} (\emph{SDMA}) setup, the space is segmented into sectors, then addressed with directed antennas.
  These tend to follow a distinct cell structure to handle varying number of people.

  \begin{itemize}[noitemsep]
  \item Multiple users are separated by having separate physical spaces allocated to them.
  \item [+:] This is very simple to implement
  \item [---:] Relatively fixed setup, meaning it is not easy to move or change.
  \item [Example:] This scheme is used in most cellular networks.
\end{itemize}

\end{definition}

\subsubsection{FDMA}\label{subsubsec:FDMA}
\begin{definition}[Frequency Division Multiple Access]\label{def:FDMA}
  In a \emph{frequency division multiple access} (\emph{FDMA}) system, a certain frequency is assigned to a certain transmission channel between the sender and receiver.

  Because of the way FDMA works, it effectively acts like a \nameref{def:Circuit_Switching} system, so we can apply the same equations as \nameref{subsec:Model_Circuit_Switching}.
  \begin{equation}\label{eq:FDMA_Work_Rate}
    \eta = \frac{1}{n v} \sum\limits_{i=1}^{N} \rho_{i}
  \end{equation}
  where
  \begin{itemize}[noitemsep]
  \item $v$: The achievable rate, in bits/Hz/s. This includes:
    \begin{itemize}[noitemsep]
    \item Guard bands
    \item Signaling Overheads
    \end{itemize}
  \item $N$: The number of nodes
  \item $\rho$: The utilized rate
  \item $\eta$: The rate of work
  \end{itemize}

  \begin{itemize}[noitemsep]
  \item Multiple users are given their own unique frequency.
  \item [+:] Simple and robust system.
  \item [---:] The number of possible frequencies are limited, so there may not be enough frequencies for all users.
  \item [Example:]
    \begin{itemize}[noitemsep]
    \item Permanent: Radio broadcasting
    \item Slow Hopping: GSM (2G)
    \item Fast Hopping: FHSS (Frequency Hopping Spread Spectrum)
    \end{itemize}
  \end{itemize}
\end{definition}

\paragraph{OFDM}\label{par:OFDM}
\begin{definition}[Orthogonal Frequency Division Multiplexing]\label{def:OFDM}
  \emph{Orthogonal frequency division multiplexing} or \emph{OFDM} is a way of breaking up the frequency spectrum into orthogonal frequencies into orthogonal subcarriers.
  An orthogonal subcarrier is a frequency that has a waveform that has a peak where all other subcarriers' waveform is zero.
  To achieve this, the equation below is used.
  \begin{equation}\label{eq:Orthogonal_Subcarrier}
    \Delta f = \frac{k}{T_{U}}
  \end{equation}
  where
  \begin{itemize}[noitemsep]
  \item $k \in \NaturalNumbers$
  \item $T_{U}$ is the useful symbol duration
  \end{itemize}

  Because these subcarriers are orthogonal, there do not need to be guard bands between them, allowing for more efficient usage of the transmission spectrum.
  Also, each symbol is spread over a wider frequency range of many slowly-modulated narrowband signals instead of one rapidly-modulating wideband signal.
  If this is paired with error detection and correction, the transmission is even more robust.
\end{definition}

We can use \nameref{def:OFDM} to do 2 things:
\begin{enumerate}[noitemsep]
\item \nameref{par:Subcarrier_Allocation}
\item \nameref{par:Transmission_Power_Control}
\end{enumerate}

\paragraph{Subcarrier Allocation}\label{par:Subcarrier_Allocation}
If a single node does not need the entire frequency spectrum that we have available, we can can share the channel's subcarrier frequencies between multiple nodes.
This flexibility means:
\begin{itemize}[noitemsep]
\item We can assign different numbers of subcarriers to different nodes depending on the node's throughput requirements.
\item We can adjust which subcarriers are allocated to account for adverse channel conditions.
\end{itemize}

\paragraph{Transmission Power Control}\label{par:Transmission_Power_Control}
In general, the greater the power transmission, the greater the data rate.
\begin{remark*}
  This does not hold true for \nameref{def:CDMA} systems.
  That small difference will be discussed there.
\end{remark*}

The flexibility that transmission power control gives us is:
\begin{itemize}[noitemsep]
\item If a node does not require the highest possible data rate for the current SINR, it can reduce transmission power.
\item SINR is lowered but the node changes to a different modulation and coding scheme to compensate.
\item This reduces the data rate, so the transmission power should be chosen to match the required rate.
\item Adjusting the nodes' transmission rate through transmission power creates a form of \nameref{def:SDMA}. Nodes will cause interference to other nodes in a larger area if the transmission power increases.
\end{itemize}

\begin{table}[h!]
  \centering
  \begin{tabular}{cccccc}
    \toprule
    \multicolumn{1}{p{1.6cm}}{Data Rate (mbps)} & Modulation & Coding Rate & \multicolumn{1}{p{2.2cm}}{Coded bits per subcarrier} & \multicolumn{1}{p{2.5cm}}{Coded bits per OFDM symbol} & \multicolumn{1}{p{2.5cm}}{Coded bits per OFDM symbol} \\
    \midrule
    6 & BPSK & 1/2 & 1 & 48 & 25 \\
    9 & BPSK & 3/4 & 1 & 48 & 36 \\
    12 & QPSK & 1/2 & 2 & 96 & 48 \\
    18 & QPSK & 3/4 & 2 & 96 & 72 \\
    24 & 16-QAM & 1/2 & 4 & 192 & 96 \\
    36 & 16-QAM & 3/4 & 4 & 192 & 144 \\
    48 & 64-QAM & 2/3 & 6 & 288 & 192 \\
    54 & 64-QAM & 3/4 & 6 & 288 & 216 \\
    \bottomrule
  \end{tabular}
  \caption{Transmission Power Control}
  \label{tab:Transmission_Power_Control}
\end{table}

\paragraph{Joint Subcarrier Allocation and Transmission Power Control}\label{par:Joint_Subcarrier_Transmission_Power}
We can perform both \nameref{par:Subcarrier_Allocation} and \nameref{par:Transmission_Power_Control} at the same time.

This has the
\begin{itemize}[noitemsep]
\item Advantages of
  \begin{itemize}[noitemsep]
  \item Lots of flexibility with ow we allocate resources.
  \end{itemize}

\item Disadvantages of
  \begin{itemize}[noitemsep]
  \item Being a difficult optimization problem, especially for real-time applications.
  \end{itemize}
\end{itemize}

\subsubsection{TDMA}\label{subsubsec:TDMA}
\begin{definition}[Time Division Multiple Access]\label{def:TDMA}
  A \emph{time division multiple access}  (\emph{TDMA}) system assigns a fixed sending frequency to a transmission channel \textbf{for a specific amount of time}.

  Because of the way FDMA works, it effectively acts like a \nameref{def:Circuit_Switching} system, so we can apply the same equations as \nameref{subsec:Model_Circuit_Switching}.
  \begin{equation}\label{eq:TDMA_Work_Rate}
    \eta = \frac{1}{n v} \sum\limits_{i=1}^{N} \rho_{i}
  \end{equation}
  where
  \begin{itemize}[noitemsep]
  \item $v$: The achievable rate, in bits/Hz/s. This includes:
    \begin{itemize}[noitemsep]
    \item Guard bands
    \item Signaling Overheads
    \end{itemize}
  \item $N$: The number of nodes
  \item $\rho$: The utilized rate
  \item $\eta$: The rate of work
  \end{itemize}

  \begin{itemize}[noitemsep]
  \item Multiple uesers are only allowed to transmit in certain reserved timeslots.
  \item [+:] This is a completely digital system, so it is easy to change and configure.
  \item [---:] All nodes must keep their time synchronized, which is difficult. Multipath propagation of the transmission wave also complicates things.
  \item [Example:] Bluetooth.
  \end{itemize}
\end{definition}

\subsubsection{Demand Assigned Multiple Access}\label{subsubsec:DAMA}
The use of time reservation can increase the efficiency of a system up to 80\%.
To do this,
\begin{itemize}[noitemsep]
\item A sender \textbf{reserves} a future timeslot.
\item Sending within this timeslot is possible without any collision.
\item However, the act of having to reserve something increases the delay.
\item This is typically used in satellite links.
\end{itemize}

There are 2 reservation algorithms for requesting a timeslot:
\begin{enumerate}[noitemsep]
\item \nameref{par:TDMA_Explicit_Reservation}
\item \nameref{par:TDMA_Reservation_Time}
\end{enumerate}

\paragraph{Explicit Reservation}\label{par:TDMA_Explicit_Reservation}
This scheme is similar to the way \nameref{def:ALOHA} works.
Each node requests a time slot on-the-fly.
There are 2 main implementations:
\begin{enumerate}[noitemsep]
\item \nameref{def:ALOHA} mode
  \begin{itemize}[noitemsep]
  \item This means that time slots are assigned as they are needed.
  \item However, if multiple nodes was the same timeslot, there may be a competition.
  \item In addition, multiple timeslots can be reserved at once.
  \item The controlling system will grant access to anyone that asks for it, without question.
  \item However, there may be collision/overlapping timeslot between 2 nodes.
  \end{itemize}

\item Reserved mode
  \begin{itemize}[noitemsep]
  \item Data transmission can only happen inside of successfully reserved slots.
  \item The controlling system will schedule and grant access to these based on some algorithm.
  \item This means that there can be no collisions between timeslots.
  \end{itemize}
\end{enumerate}

\paragraph{Reservation Time}\label{par:TDMA_Reservation_Time}
In this scheme, each major time interval is a frame.
Inside this frame, there are $N$ mini-slots and $x$ data-slots, where $N$ is the number of nodes in the network.
Each station gets its own mini-slot, and \textbf{must reserve} up to $k$ data slots, so $x \geq Nk$.

If there are unused data-slots, then they can be assigned to other stations that require the additional bandwidth.

\subsubsection{CDMA}\label{subsubsec:CDMA}
\begin{definition}[Code Division Multiple Access]\label{def:CDMA}
  A \emph{code division multiple access} (\emph{CDMA}) system works by having all terminals send on the same frequency, likely at the same time.
  These terminals can use the whole bandwidth of the transmission channel.
  To differentiate these terminals,
  \begin{itemize}[noitemsep]
  \item Each sender has a unique random number, the sender XORs the signal with this pseudorandom number
  \item The receiver can ``tune'' into this signal if it knows the pseudorandom number, tuning is done via a correlation function
  \end{itemize}

  \begin{itemize}[noitemsep]
  \item Each user in the system gets an (pseudo-) orthogonal code, which is used to encode their data. The receiver can then decode the signal.
  \item [+:] This is a flexible system, with less management on transmission frequencies required.
  \item [---:] Complex receivers are required to implement this.
  \item [Example:] UMTS
  \end{itemize}
\end{definition}

To handle multiple data streams inside the same user, there needs to be multiple codes that are used.
These are the:
\begin{enumerate}[noitemsep]
\item \nameref{def:Spreading_Code}
\item \nameref{def:Scrambling_Code}
\end{enumerate}

\begin{definition}[Spreading Code]\label{def:Spreading_Code}
  The \emph{spreading code} is used at the end of each data stream, so we can figure out which application/data stream the data belongs to.
  It is only used inside the user's device, and it done before each of the data streams is added together, XORed with the \nameref{def:Scrambling_Code}, and pushed out.
\end{definition}

\begin{definition}[Scrambling Code]\label{def:Scrambling_Code}
  The \emph{scrambling code} is used once all the applications/data streams have been XORed with their \nameref{def:Spreading_Code}s and aggregated.
  This code is then XORed with this output and then sent out to be handled by the receiver.
\end{definition}

There is a concept of the spreading factor.
\begin{equation}\label{eq:Spreading_Factor}
  \frac{R_{c}}{R_{i}} = \frac{\frac{1}{R_{i}}}{\frac{1}{R_{c}}}
\end{equation}

\paragraph{Orthogonal Variable Spreading Factor Codes}\label{par:OVSF_Codes}
The codes used for \nameref{def:CDMA} are generated in a recursive manner.
Each generation of codes halves the overall throughput of the system.

\begin{equation}\label{eq:OVSF_Codes}
  x_{i-1} \rightarrow
  \begin{cases}
    (x_{i}, x_{i}) \\
    (x_{i}, -x_{i}) \\
  \end{cases}
\end{equation}

\begin{table}[h!]
  \centering
  \begin{tabular}{c|p{4cm}p{4cm}}
    \toprule
    & \multicolumn{1}{c}{Down Link} & \multicolumn{1}{c}{Up Link} \\
    \midrule
    \nameref{def:Scrambling_Code} & Identify cells. Code assigned to each cell. Only 512 codes. Assignment done by system designer. & Identify users. Code assigned to each user. $2^{24}$ codes. Assignment done by algorithm. \\
    \nameref{def:Spreading_Code} & Identify channels used by a cell. Code assigned to each user. & Identify channels to be used by users. Code assigned to each channel.
  \end{tabular}
  \caption{How to use OVSF Codes}
  \label{tab:Use_OVSF_Codes}
\end{table}

The codes are not perfectly orthogonal, meaning that some amount of the transmission's value can be viewed as noise (really it's lost information), so each node causes some interference.
Because of the way this system works and interacts with devices near the receiver, there is a general performance degration that closely follows the \nameref{def:Gaussian_Random_Variable}'s distribution.
Meaning, there is only a soft performance degradation as the system gains additional users.

%%% Local Variables:
%%% mode: latex
%%% TeX-master: "../../ETSN10-Network_Architecture_Performance-Reference_Sheet"
%%% End:


\subsection{Polling}\label{subsec:Polling}
\begin{definition}[Polling]\label{def:Polling}
  \emph{Polling} is the act of having one terminal that can be heard by all others ask for a small amount of information.
  This ``central'' terminal, e.g.\ a base station can poll all other terminals according to a certain scheme.
  This is a typical server-client workflow, where the server is the one polling all the clients regularly.
\end{definition}

We are usually interested in the effective performance of the polling system.
To calculate this, we can use \Cref{eq:Polling_Effective_Performance}.
\begin{equation}\label{eq:Polling_Effective_Performance}
  E = \frac{T_{t}}{T_{t} + T_{\mathrm{Idle}} + T_{\mathrm{Poll}}}
\end{equation}
where
\begin{itemize}[noitemsep]
\item $T_{t}$: Transmission time of the polling packet.
\item $T_{\mathrm{Idle}}$: Time waiting to detect no transmission.
\item $T_{\mathrm{Poll}}$: Time to send the polling message.
\end{itemize}

\subsubsection{Randomly Addressed Polling}\label{subsubsec:Randomly_Addressed_Polling}
To perform randomly addressed polling:
\begin{enumerate}[noitemsep]
\item Base station signals readiness to all mobile terminals
\item Terminals ready to send transmit a random number without collision with the help of \nameref{def:CDMA} or \nameref{def:FDMA}
  \begin{itemize}[noitemsep]
  \item The random number can be seen as dynamic address
  \end{itemize}

\item Base station chooses one address for polling from list of all random numbers
  \begin{itemize}[noitemsep]
  \item Collision if two terminals choose the same address
  \end{itemize}

\item Base station acknowledges correct packets and continues polling the next terminal
\item Cycle starts again after polling all terminals of the list
\end{enumerate}

\subsubsection{Point Coordination Function}\label{subsubsec:Point_Coordination_Function}
\begin{definition}[Point Coordination Function]\label{def:PCF}
  \emph{Point Coordination Function} or \emph{PCF} is a centralized \nameref{def:Polling} mechanism for WiFi (802.11).
  The central poller is the current Access Point (AP).

  In this system, there are 2 types of tranmission periods.
  The first is during the Point Coordinated Function, when the resources on the AP are reserved for users.
  The second is a pure \nameref{def:Random_Access} period.
  Here, the rules of a random access sytem are obeyed.
\end{definition}

\paragraph{Problems with the \nameref*{def:PCF}}\label{par:PCF_Problems}
The performance of \nameref{def:PCF} can be quite poor, especially when used for voice communications.
When a PCF system is running, and a poll occurs, the users that have nothing to send return null packets.
While these packets are empty (contain no information), they are sitll packets that need to be transmitted, thus taking up tranmission time.

Additionally, there is no Quality of Service (QoS) signalling, which can further degrade performance.

There is also no tranmission time limit.
If a user is polled, and it has something to transmit, the transmission will continue until the user is done.
This incurs a cost, because other nodes may not have their chance to transmit during the \nameref{def:PCF} window, and must wait for the \nameref{def:Random_Access} window, or must wait until the next round of polling.

%%% Local Variables:
%%% mode: latex
%%% TeX-master: "../../ETSN10-Network_Architecture_Performance-Reference_Sheet"
%%% End:


\subsection{Random Access}\label{subsec:Random_Access}
\begin{definition}[Random Access]\label{def:Random_Access}
  \emph{Random Access} is the process by which a user can request resources from a system for a duration of time.
  After this time period has passed, the resources are freed from that user and may be passed to another.
  These are also called \emph{Contention-Based protocols}.

  This allows for sharing the same resources among many users, but there may be contention for the resources.
  In addition, because there is not way to arbitrate the delivery of data, there may be packet collisions when multiple transmitters send data, and the receiver cannot decode the data from just one of them.
\end{definition}

\subsubsection{Fundamentals of ALOHA}\label{subsubsec:ALOHA_Fundamentals}
\begin{itemize}[noitemsep]
\item Full duplex communication.
  \begin{itemize}[noitemsep]
  \item Up-link (Upload)
  \item Down-link (Download)
  \end{itemize}

\item Originally UHF (Ultra-High Frequency) communication
\item Incoming traffic (to base station):
  \begin{itemize}[noitemsep]
  \item 9,600 bps (403.500 MHz)
  \item Random access (Pure ALOHA)
  \end{itemize}
\item Outgoing traffic (from base station):
  \begin{itemize}[noitemsep]
  \item 9,600 bps (413.474 MHz)
  \item Broadcast
  \end{itemize}
\end{itemize}

\paragraph{Pure ALOHA}\label{par:Pure_ALOHA}
\begin{itemize}[noitemsep]
\item Whenever a station has something to send, it sends.
\item When a packet is received by the central, an acknowledgment (\texttt{ACK}) is sent back in broadcast.
\item If the sending station does not receive an \texttt{ACK} within a set time, a collision is assumed.
\item When a collision occurs, retransmit within a random time slot, 200--1500 ms.
\end{itemize}

\begin{remark*}
  We wait a random amount of time before transmitting to prevent retransmission collisions.
  If 2 nodes were to experience a collision, and both waited a fixed amount of time, then they would collide a again during their retransmission.
\end{remark*}

\paragraph{Slotted ALOHA}\label{par:Slotted_ALOHA}
\begin{itemize}[noitemsep]
\item Packets may just be transmitted within time slots.
\item If a station has started to transmit in a time slot, other station who wish to transmit within this time slot can not interfere.
\item This principle leads to a much better utilization of the channel.
\end{itemize}

The utilization gets so much better because there is \textbf{no chance} of a collision.
Since each station can only transmit if no other station is transmitting in this time slot, there cannot be collisions part-way through a packet's transmission.

\paragraph{ALOHA Performance}\label{par:ALOHA_Performance}
Packet generation process is Poisson distributed with $n$ packets generated during time $t$:
\begin{equation*}
  \Prob(n, t) = \frac{{\left( \lambda_{p}t \right)}^{n} e^{(-\lambda_{p}t)}}{n!}
\end{equation*}

The probability zero packets generated during time $t$ is
\begin{equation*}
  \Prob(n = 0, t) = \frac{{\left( \lambda_{p}t \right)}^{0} e^{-\lambda_{p}t}}{0!} = e^{-\lambda_{p}t}
\end{equation*}

The Effective throughput is equal to the Rate multiplied by the probability of success.
\begin{equation*}
  T_{\mathrm{Put}} = \lambda_{p}T_{p}e^{(-\lambda_{p}T_{p})}
\end{equation*}

The maximum for \nameref{par:Pure_ALOHA} is found at:
\begin{equation*}
  \frac{\partial}{\partial x} T_{\mathrm{Put}} = 0 \Rightarrow \frac{1}{2e}
\end{equation*}

In \nameref{par:Slotted_ALOHA}, $t = T_{\mathrm{P}}$, so its maximum is found at:
\begin{equation*}
  \frac{\partial}{\partial x} T_{\mathrm{Put}} = 0 \Rightarrow \frac{1}{e}
\end{equation*}

\subsubsection{Carrier Sense Multiple Access}\label{subsubsec:CSMA}
\begin{definition}[Carrier Sense Multiple Access]\label{def:CSMA}
  In \emph{Carrier Sense Multiple Access} (\emph{CSMA}), a channel can only transmit its message if the underlying network connection does not have an ongoing transmission.
  This is sometimes called the polite method or Listen Before Talk network communication because of the Carrier Sense that must occur.
  \begin{itemize}[noitemsep]
  \item If the channel is free, it starts to send.
  \item If the device that wants to send senses the channel is busy, it waits to send.
    \begin{itemize}[noitemsep]
    \item The station might be persistent, meaning it might continue to try sensing the network.
    \end{itemize}

  \item Acknowledgement
  \end{itemize}
\end{definition}

\begin{definition}[Persistence]\label{def:Persistence}
  \emph{Persistence} is a property of \nameref{def:CSMA} systems.
  Systems that use persistence are known as persistent systems.

  \begin{remark*}
    It might be helpful to think of persistence as the device's persistence to listen to the channel.
  \end{remark*}

  There are 3 types of persistent systems.
  \begin{enumerate}[noitemsep]
  \item \nameref{def:Non_Persistent}
  \item \nameref{def:p_Persistent}
  \item \nameref{def:Persistent}
  \end{enumerate}
\end{definition}

\begin{definition}[Non-Persistent]\label{def:Non_Persistent}
  A \emph{Non-Persistent} system behaves as follows:
  \begin{enumerate}[noitemsep]
  \item If the channel is busy, the device will wait a random time before performing another carrier sense on the channel.
  \item Once the channel is idle, the packet is sent immediately.
  \end{enumerate}

  The performance of a non-persistent system is:
  \begin{equation}\label{eq:Non_Persistent_Performance}
    S_{1} = \frac{G e^{-G} (1+G)}{G + e^{-G}}
  \end{equation}
  where
  \begin{itemize}[noitemsep]
  \item $G$: Offered Load
  \item $S$: Throughput
  \end{itemize}
\end{definition}

\begin{definition}[$p$-Persistent]\label{def:p_Persistent}
  A \emph{$p$-Persistent} system behaves as follows:
  \begin{enumerate}[noitemsep]
  \item If the channel is busy, the device will continue to perform carrier sense on the channel.
  \item Once the channel is idle, there is a probability $p$ that the device will send.
  \item There is a $1-p$ probability that the deivce will instead wait one time unit before performing carrier sense again.
  \end{enumerate}

  The performance of a $p$-persistent system is:
  \begin{equation}\label{eq:p_Persistent_Performance}
    S_{p} = \frac{G e^{-G} (1+pGx)}{G + e^{-G}}
  \end{equation}
  where
  \begin{itemize}[noitemsep]
  \item $G$: Offered Load
  \item $S$: Throughput
  \item $x$: Is defined by \Cref{eq:p_Persistent_x_Value}
    \begin{equation}\label{eq:p_Persistent_x_Value}
      x = \sum\limits_{k=0}^{\infty} \frac{{(qG)}^{k}}{(1-q^{k+1})!}
    \end{equation}
  \end{itemize}
\end{definition}

\begin{definition}[Persistent]\label{def:Persistent}
  A \emph{Persistent} system behaves as follows:
  \begin{enumerate}[noitemsep]
  \item If the channel is busy, the device will continue to use carrier sense on the channel.
  \item Once the channel is idle, this device will start sending its packet immediately.
  \end{enumerate}

  The performance of a persistent system is:
  \begin{equation}\label{eq:Persistent_Performance}
    S_{n} = \frac{G}{1+G}
  \end{equation}
  where
  \begin{itemize}[noitemsep]
  \item $G$: Offered Load
  \item $S$: Throughput
  \end{itemize}
\end{definition}

\subsubsection{Carrier Sense Multiple Access/Collision Detection}\label{subsubsec:CSMACD}
\begin{definition}[Carrier Sense Multiple Access/Collision Detection]\label{def:CSMACD}
  \emph{Carrier Sense Multiple Access/Collision Detection} is a modification of \nameref{def:CSMA}.
  It functions in much the same way, but:
  \begin{enumerate}[noitemsep]
  \item Perform Carrier Sensing
    \begin{itemize}[noitemsep]
    \item If the channel is free, send.
    \end{itemize}

  \item The station listens when sending
    \begin{itemize}[noitemsep]
    \item When a station detects a collision, it stops sending
    \end{itemize}

  \item Retransmissions
  \end{enumerate}
\end{definition}

If we use a simplistic model:
\begin{itemize}[noitemsep]
\item All stations are the same
\item No exponential backoff of tranmissions
\item All packets are the same size
\item All stations have the same transmission probability
\item $k$ stations
\item $p$ Probability of transmission
\item $A$ Probability 1 station acquires channel in slot
\end{itemize}

\begin{equation}\label{eq:CSMACD_Performance}
  A(k, p) = kp{(1-p)}^{k-1}
\end{equation}

We can find the maximum performance of the system, which occurs when $p=\frac{1}{k}$.
\begin{equation}\label{eq:CSMACD_Max_Performance}
  A_{\mathrm{Max}} = {(1-\frac{1}{k})}^{k-1}
\end{equation}


\subsubsection{Carrier Sense Multiple Access/Collision Avoidance}\label{subsubsec:CSMACA}

%%% Local Variables:
%%% mode: latex
%%% TeX-master: "../../ETSN10-Network_Architecture_Performance-Reference_Sheet"
%%% End:


\subsection{Energy-Efficient MACs}\label{subsec:Energy_Efficient_MACs}
In some systems, we need to make sure the power usage is minimized.
The biggest energy costs in a network are:
\begin{itemize}[noitemsep]
\item Sending Data
\item Receiving Data
\item Listening to the channel
\item Keeping the node's networking system on when it doesn't need to transmit or receive data.
\end{itemize}

\begin{remark*}
  The costs of both the transmission and reception of data are hard to minimize on a MAC level.
  These would usually be minimized by the design of hardware on the physical level and determining if it is worth it to send information at all in the application layer.
\end{remark*}

\begin{remark*}
  Collisions have a massive effect on the energy usage of a networking system.
  A collision means the receiver did not receive the packet correctly, so it must be retransmitted.
  Meaning, 2 transmissions and 2 receptions were needed to correctly read the packet, effectively double the correct amount of power.
\end{remark*}

\begin{remark*}
  \nameref{def:CSMA} is \textbf{not} a power-efficient MAC protocol, because it constantly listens to the channel.
  However, it can be improved by reducing the \nameref{def:Persistence} of the system.
\end{remark*}

%%% Local Variables:
%%% mode: latex
%%% TeX-master: "../ETSN10-Network_Architecture_Performance-Reference_Sheet"
%%% End:
