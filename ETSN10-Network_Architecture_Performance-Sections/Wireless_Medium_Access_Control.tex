\section{Wireless Medium Access Control}\label{sec:Wireless_MAC}
We need to separate when devices are transmitting.
Due to the nature of radio waves, we cannot transmit simultaneously on the same frequency from multiple devices.

\subsection{Reservation Schemes}\label{subsec:Reservation_Scheme}
To begin our discussion of \nameref{def:Reservation_Scheme}s, we must talk about \nameref{def:ALOHA}.

\begin{definition}[ALOHA]\label{def:ALOHA}
  \emph{ALOHA} was one of the earliest protocols for wireless data transmission.
  It used a purely random access scheme, so when you had a packet to send, it was sent.
  To ensure devices knew if their transmission was successful, an acknoledgement was sent when the next recipient in the chain received the message.

  \begin{remark}
    It is only mentioned here because it is used as a baseline measurement for other protocols.
  \end{remark}
\end{definition}

\begin{definition}[Reservation Scheme]\label{def:Reservation_Scheme}
  A \emph{reservation scheme} is the idea of reserving resources for a node in a wireless network to prevent collisions.
  The resources that can be reserved include: time, frequency, and others.

  We would like to reserve resources because the system will not lose capacity to collisions.
  But, if not \textbf{all} the resources are being fully utilized, then the system loses out on throughput.

  Because of the limitations of the radio spectrum and light being a broadcast medium, we have 4 main solutions for supporting multiple devices that was to transmit.
  \begin{enumerate}[noitemsep]
  \item \nameref{def:SDMA}
  \item \nameref{def:FDMA}
  \item \nameref{def:TDMA}
  \item \nameref{def:CDMA}
  \end{enumerate}
\end{definition}

\begin{table}[h!]
  \centering
  \begin{tabular}{p{2cm}|p{3.6cm}p{3.6cm}p{3.6cm}p{3.6cm}}
    \toprule
    \multicolumn{1}{c|}{\textbf{Approach}} & \multicolumn{1}{c}{\textbf{\nameref{subsubsec:SDMA}}} & \multicolumn{1}{c}{\textbf{\nameref{subsubsec:TDMA}}} & \multicolumn{1}{c}{\textbf{\nameref{subsubsec:FDMA}}} & \multicolumn{1}{c}{\textbf{\nameref{subsubsec:CDMA}}} \\
    \midrule
    Idea & Segment space into cells/sectors & Segment sending time into mutually disjoint time-slots, demand-driven or pattern-driven. & Segment the frequency band into disjoing sub-bands. & Spread the spectrum according to orthogonal codes. \\ \midrule
    Terminals & One one terminal can be active in a single cell/sector. & All terminals active for short time periods on the same frequency. & Every terminal has its own frequency, uninterrupted. & All terminals can be active at the same time at the same place, on the same frequency. \\ \midrule
    Signal Separation & Cell structure, directed antennas. & Synchronization in time domain. & Filtering in frequency domain. & Code plus special receivers. \\ \midrule
    Advantages & Very simple, increases capacity per unit area. & Established, \textbf{fully digital}, flexible. & Simple, established, robust. & Flexible, less frequency planning needed, soft handover. \\ \midrule
    Disadvantages & Inflexible, antennas typically fixed. & Guard time needed (multipath propagation), synchronization difficult. & Inflexible, frequencies are a scarce resource and must be decided from beginning. & Complex receivers, needs more complicated power management from senders. \\ \midrule
    Comments & Only useful in combination with \nameref{subsubsec:TDMA}, \nameref{subsubsec:FDMA}, or \nameref{subsubsec:CDMA}. & Standard in fixed networks. Used together with \nameref{subsubsec:FDMA} and \nameref{subsubsec:SDMA} in mobile networks. & Typically combined with \nameref{subsubsec:TDMA} (frequency hopping patterns) and \nameref{subsubsec:SDMA} (frequency reuse) & Has some problems still. Higher complexity, lower expectations. Needs integration with \nameref{subsubsec:TDMA}/\nameref{subsubsec:FDMA}. \\
    \bottomrule
  \end{tabular}
  \caption{Division Multiple Access Comparison}
  \label{tab:DMA_Comparison}
\end{table}

\subsubsection{SDMA}\label{subsubsec:SDMA}
\begin{definition}[Space Division Multiple Access]\label{def:SDMA}
  In a \emph{space division multiple access} (\emph{SDMA}) setup, the space is segmented into sectors, then addressed with directed antennas.
  These tend to follow a distinct cell structure to handle varying number of people.
\end{definition}

\subsubsection{FDMA}\label{subsubsec:FDMA}
\begin{definition}[Frequency Division Multiple Access]\label{def:FDMA}
  In a \emph{frequency division multiple access} (\emph{FDMA}) system, a certain frequency is assigned to a certain transmission channel between the sender and receiver.

  Because of the way FDMA works, it effectively acts like a \nameref{def:Circuit_Switching} system, so we can apply the same equations as \nameref{subsec:Model_Circuit_Switching}.
  \begin{equation}\label{eq:FDMA_Work_Rate}
    \eta = \frac{1}{n v} \sum\limits_{i=1}^{N} \rho_{i}
  \end{equation}
  where
  \begin{itemize}[noitemsep]
  \item $v$: The achievable rate, in bits/Hz/s. This includes:
    \begin{itemize}[noitemsep]
    \item Guard bands
    \item Signaling Overheads
    \end{itemize}
  \item $N$: The number of nodes
  \item $\rho$: The utilized rate
  \item $\eta$: The rate of work
  \end{itemize}
\end{definition}

Some examples of this are:
\begin{itemize}[noitemsep]
\item Permanent: Radio broadcasting
\item Slow Hopping: GSM (2G)
\item Fast Hopping: FHSS (Frequency Hopping Spread Spectrum)
\end{itemize}

\paragraph{OFDM}\label{par:OFDM}
\begin{definition}[Orthogonal Frequency Division Multiplexing]\label{def:OFDM}
  \emph{Orthogonal frequency division multiplexing} or \emph{OFDM} is a way of breaking up the frequency spectrum into orthogonal frequencies into orthogonal subcarriers.
  An orthogonal subcarrier is a frequency that has a waveform that has a peak where all other subcarriers' waveform is zero.
  To achieve this, the equation below is used.
  \begin{equation}\label{eq:Orthogonal_Subcarrier}
    \Delta f = \frac{k}{T_{U}}
  \end{equation}
  where
  \begin{itemize}[noitemsep]
  \item $k \in \NaturalNumbers$
  \item $T_{U}$ is the useful symbol duration
  \end{itemize}

  Because these subcarriers are orthogonal, there do not need to be guard bands between them, allowing for more efficient usage of the transmission spectrum.
  Also, each symbol is spread over a wider frequency range of many slowly-modulated narrowband signals instead of one rapidly-modulating wideband signal.
  If this is paired with error detection and correction, the transmission is even more robust.
\end{definition}

We can use \nameref{def:OFDM} to do 2 things:
\begin{enumerate}[noitemsep]
\item \nameref{par:Subcarrier_Allocation}
\item \nameref{par:Transmission_Power_Control}
\end{enumerate}

\paragraph{Subcarrier Allocation}\label{par:Subcarrier_Allocation}
If a single node does not need the entire frequency spectrum that we have available, we can can share the channel's subcarrier frequencies between multiple nodes.
This flexibility means:
\begin{itemize}[noitemsep]
\item We can assign different numbers of subcarriers to different nodes depending on the node's throughput requirements.
\item We can adjust which subcarriers are allocated to account for adverse channel conditions.
\end{itemize}

\paragraph{Transmission Power Control}\label{par:Transmission_Power_Control}
In general, the greater the power transmission, the greater the data rate.
\begin{remark*}
  This does not hold true for \nameref{def:CDMA} systems.
  That small difference will be discussed there.
\end{remark*}

The flexibility that transmission power control gives us is:
\begin{itemize}[noitemsep]
\item If a node does not require the highest possible data rate for the current SINR, it can reduce transmission power.
\item SINR is lowered but the node changes to a different modulation and coding scheme to compensate.
\item This reduces the data rate, so the transmission power should be chosen to match the required rate.
\item Adjusting the nodes' transmission rate through transmission power creates a form of \nameref{def:SDMA}. Nodes will cause interference to other nodes in a larger area if the transmission power increases.
\end{itemize}

\begin{table}[h!]
  \centering
  \begin{tabular}{cccccc}
    \toprule
    \multicolumn{1}{p{1.6cm}}{Data Rate (mbps)} & Modulation & Coding Rate & \multicolumn{1}{p{2.2cm}}{Coded bits per subcarrier} & \multicolumn{1}{p{2.5cm}}{Coded bits per OFDM symbol} & \multicolumn{1}{p{2.5cm}}{Coded bits per OFDM symbol} \\
    \midrule
    6 & BPSK & 1/2 & 1 & 48 & 25 \\
    9 & BPSK & 3/4 & 1 & 48 & 36 \\
    12 & QPSK & 1/2 & 2 & 96 & 48 \\
    18 & QPSK & 3/4 & 2 & 96 & 72 \\
    24 & 16-QAM & 1/2 & 4 & 192 & 96 \\
    36 & 16-QAM & 3/4 & 4 & 192 & 144 \\
    48 & 64-QAM & 2/3 & 6 & 288 & 192 \\
    54 & 64-QAM & 3/4 & 6 & 288 & 216 \\
    \bottomrule
  \end{tabular}
  \caption{Transmission Power Control}
  \label{tab:Transmission_Power_Control}
\end{table}

\subsubsection{TDMA}\label{subsubsec:TDMA}
\begin{definition}[Time Division Multiple Access]\label{def:TDMA}
  A \emph{time division multiple access}  (\emph{TDMA}) system assigns a fixed sending frequency to a transmission channel \textbf{for a specific amount of time}.
\end{definition}


\subsubsection{CDMA}\label{subsubsec:CDMA}
\begin{definition}[Code Division Multiple Access]\label{def:CDMA}
   (\emph{CDMA})
\end{definition}

%%% Local Variables:
%%% mode: latex
%%% TeX-master: "../ETSN10-Network_Architecture_Performance-Reference_Sheet"
%%% End:
