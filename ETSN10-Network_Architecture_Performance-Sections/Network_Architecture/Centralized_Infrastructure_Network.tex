\subsection{Centralized Infrastructure Networks}\label{subsec:Centralized_Infrastructure_Network}
\begin{definition}[Centralized Infrastructure Network]\label{def:Centralized_Infrastructure_Network}
  \emph{Centralized Infrastructure Networks} have some fixed nodes which have the connection to the wider Internet.
  These gateways are the ones that provide access to the Internet.
  This system works in a hierarchical structure, following a \nameref{def:Star_of_Stars} topology.
\end{definition}

\begin{definition}[Star-of-Stars]\label{def:Star_of_Stars}
  A network with a \emph{star-of-stars} topology is a hierarchical setup, where the leaf nodes in the system are the ones generating data to send.
  The next rows up may generate some data, but mainly serve as aggregators for the lower nodes in the system.
  Eventually, all connections terminate at one head star node.
\end{definition}

\subsubsection{Example Centralized Architectures}\label{subsubsec:Example_Centralized_Architectures}
\begin{enumerate}[noitemsep]
\item 802.11 a/b/g/n/ac
\begin{itemize}[noitemsep]
\item Most common WiFi network type, used in homes, shops, offices, etc.
\item 2.4 or 5 GHz bands, range $\approx 500 \si{\meter}$
\item MAC:\@ 802.11 DCF
\end{itemize}

\item 802.11 ah
  \begin{itemize}[noitemsep]
  \item WiFi for Internet of Things
  \item 800 MHz band, range $\approx 1 \si{\kilo \meter}$
  \item MAC:\@ 802.11 DCF with some modifications:
    \begin{itemize}[noitemsep]
    \item Relay nodes
    \item Target wake times
    \item Contention groups
    \end{itemize}
  \item Aimed at reducing energy usage and providing coverage to more nodes in a single network
\end{itemize}

\item LoRa
  \begin{itemize}[noitemsep]
  \item Low-Power Wide Area Network (LPWAN) for Internet of Things
  \item 800 MHz band, range 10s of km
  \item MAC:\@ ALOHA for the uplink, with receive windows for the downlink
  \end{itemize}
\end{enumerate}

\begin{remark*}
  For the above architectures, standard Internet routing (\nameref{def:Distance_Vector_Routing_Protocol} or \nameref{def:Link_State_Routing_Protocol}) is used.
  \begin{itemize}[noitemsep]
  \item Only one wireless hop, so no specialised routing is needed.
  \end{itemize}
\end{remark*}

\begin{remark*}
  Some wireless-aware protocols can be used at higher layers, e.g. Mobile IP, Wireless TCP (WTCP).
\end{remark*}

%%% Local Variables:
%%% mode: latex
%%% TeX-master: "../../ETSN10-Network_Architecture_Performance-Reference_Sheet"
%%% End:
