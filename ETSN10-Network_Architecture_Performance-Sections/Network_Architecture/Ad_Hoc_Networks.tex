\subsection{Ad-Hoc Networks}\label{subsec:Ad_Hoc_Networks}
\begin{definition}[Ad-Hoc Network]\label{def:Ad_Hoc_Network}
  An \emph{ad-hoc network} is a network that does not rely on \textbf{ANY} fixed infrastructure.
  Many ad-hoc networks tend to have similar properties.
  \begin{propertylist}
  \item Often unplanned: They need to be
    \begin{enumerate}[noitemsep]
    \item Self-configuring
    \item Self-organizing
    \item Self-healing
    \end{enumerate}

  \item The network topology can be highly dynamic because of high \nameref{def:Mobility} or unreliable links.
  \item The network usually has a flat architecture, where most nodes are peers, with few if any gateways/base stations.
  \item The most important performance goal is \textbf{low-energy usage}.
  \end{propertylist}

  This also means that these networks tend to be highly delay-tolerant, with next to no requirements on latency, or data transmission rate.
\end{definition}

\subsubsection{Uses of Ad-Hoc Networks}\label{subsubsec:Uses_Ad_Hoc_Network}
\subsubsection{Example Ad-Hoc Architectures}\label{subsubsec:Ad_Hoc_Architectures}
\subsubsection{Ad-Hoc Energy Usage}\label{subsubsec:Ad_Hoc_Architectures}

%%% Local Variables:
%%% mode: latex
%%% TeX-master: "../../ETSN10-Network_Architecture_Performance-Reference_Sheet"
%%% End:
