\subsection{Ad-Hoc Networks}\label{subsec:Ad_Hoc_Networks}
\begin{definition}[Ad-Hoc Network]\label{def:Ad_Hoc_Network}
  An \emph{ad-hoc network} is a network that does not rely on \textbf{ANY} fixed infrastructure.
  Many ad-hoc networks tend to have similar properties.
  \begin{propertylist}
  \item Often unplanned: They need to be
    \begin{enumerate}[noitemsep]
    \item Self-configuring
    \item Self-organizing
    \item Self-healing
    \end{enumerate}

  \item The network topology can be highly dynamic because of high \nameref{def:Mobility} or unreliable links.
  \item The network usually has a flat architecture, where most nodes are peers, with few if any gateways/base stations.
  \item The most important performance goal is \textbf{low-energy usage}.
  \end{propertylist}

  This also means that these networks tend to be highly delay-tolerant, with next to no requirements on latency, or data transmission rate.
\end{definition}

\subsubsection{Uses of Ad-Hoc Networks}\label{subsubsec:Uses_Ad_Hoc_Network}
\begin{itemize}[noitemsep]
\item Disaster recovery, rescue services
\item Developing areas where there is a lack of infrastructure
\item Personal area networks (e.g. Bluetooth)
\item Wireless sensor networks: monitoring and data gathering, especially in inaccessible or remote areas
\item Vehicular Ad-hoc Networks (VANETs): where vehicles communicate wirelessly
\end{itemize}

\subsubsection{Example Ad-Hoc Architectures}\label{subsubsec:Ad_Hoc_Architectures}
\begin{enumerate}[noitemsep]
\item 802.11p
  \begin{itemize}[noitemsep]
  \item WiFi for vehicular networks
  \item 5 Ghz band, range \textasciitilde{} 500 m
  \item MAC:\@ 802.11 DCF
  \item Ad-hoc mode: no access points, no association
  \item Beacons sent every 100 ms with vehicle location, speed, etc.
  \end{itemize}

\item Z-Wave
  \begin{itemize}[noitemsep]
  \item Mainly used for smart homes
  \item 800 MHz band, range: \textasciitilde{} 100 m
  \item MAC:\@ CSMA/CA
  \end{itemize}
\end{enumerate}

\begin{remark*}
  If multiple hops are required for a transmission, there need to be specialized routing protocols.
  Because nodes may join, or leave the network at any given time.
  Meaning, the network's topology is never truly stable, i.e.\@ you can never rely on any node being present at any given point in time.
\end{remark*}

\subsubsection{Ad-Hoc Energy Usage}\label{subsubsec:Ad_Hoc_Architectures}
For most \nameref{def:Ad_Hoc_Network}s, low energy usage is the \textbf{most} important performance requirement.
Meaning, these networks have few other requirements.
For example, the system may be allowed to have very large delays, so long as the data eventually reaches the end-point.

Some of the things that consume the most energy in a wireless transmission system are:
\begin{enumerate}[noitemsep]
\item Using the radio: both sending and listening
  \begin{itemize}[noitemsep]
  \item CSMA costs extra energy because we must listen to the channel
  \item Collisions and other packet losses cost energy because we must re-transmit
  \item Protocol overheads:
    \begin{itemize}[noitemsep]
    \item Synchronisation (e.g.\@ for TDMA)
    \item Large packet headers, etc.\@ cost energy because we need to transmit and receive information.
    \end{itemize}
  \end{itemize}

\item Keeping the device on: if we can sleep, we can save a lot of
  energy
\item Some types of processing, e.g.\@ encryption is often costly --- but not as much as radio transmission
\end{enumerate}

\begin{remark*}
  It is also important to note that the energy usage is uneven between all the nodes.
  The nodes places along more frequently used routes will need to transmit more that ones on less-frequently used routes.
  So, the routes may have to be reconsidered over time.
\end{remark*}

%%% Local Variables:
%%% mode: latex
%%% TeX-master: "../../ETSN10-Network_Architecture_Performance-Reference_Sheet"
%%% End:
