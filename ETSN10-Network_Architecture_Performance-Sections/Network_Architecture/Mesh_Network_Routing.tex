\subsection{Mesh Network Routing}\label{subsec:Mesh_Network_Routing}
\begin{remark*}
  Most of these can be applied to both \nameref{def:Wireless_Mesh_Network}s and \nameref{def:Ad_Hoc_Network}s.
  However, some of the details relating to the problems due to \nameref{def:Mobility} will \textbf{NOT} apply to \nameref{def:Wireless_Mesh_Network}s.
\end{remark*}

\begin{itemize}[noitemsep]
\item Mobility causes paths to be unstable.
\item Using “traditional” shortest path algorithms is inefficient.
  \begin{itemize}[noitemsep]
  \item Dijkstra
  \item Bellman-Ford
  \item etc.
  \end{itemize}

\item Two main approaches:
  \begin{enumerate}[noitemsep]
  \item \textcolor{red}{Proactive}: Continuously update about topology and changes
    \begin{itemize}[noitemsep]
    \item Advantage: Low latency
    \item Disadvantage: Costly updates
    \end{itemize}
  \item \textcolor{red}{Reactive}: Only determine routing path when sending data
    \begin{itemize}[noitemsep]
    \item Advantage: No constant overhead
    \item Disadvantage: Delay due to route discovery
    \end{itemize}
  \end{enumerate}

\item Some hybrid protocols exist as well, which combine proactive
  and reactive approaches in different ways.
\end{itemize}

Because shortest-path routing algorithms are not usable in \nameref{def:Ad_Hoc_Network}s, what can we use?
We can route based on:
\begin{itemize}[noitemsep]
\item Interference
\item Social Similarity
\item Location
\item Residual Energy
\item Energy Usage
  \begin{itemize}[noitemsep]
  \item Multiple hops using little power vs.\@ single high power transmission.
  \end{itemize}
\end{itemize}

\subsubsection{Ad-Hoc On-Demand Distance Vector Routing}\label{subsubsec:ADOV_Routing}
\begin{definition}[Ad-Hoc On-Demand Distance Vector Routing]\label{def:ADOV_Routing}
  In \emph{Ad-hoc On-demand Distance Vector routing} (\emph{ADOV}), the sender broadcasts a ``Route Request'' packet, which is flooded throughout the whole network, until it reaches the intended receiver node.
  Each node maintains a route cache, which uses the node sequence the route request packet has gone through so far.
  Once the end node has been reached, the correct route state in each node is used to construct the link route.

  This only reacts when routes fail.
  When they do fail, the nodes go through a 2-step process.
  \begin{enumerate}[noitemsep]
  \item Attempt local recovery.
    \begin{itemize}[noitemsep]
    \item Nodes monitor neighbors
    \item Issue error message on detected problem.
    \end{itemize}
  \item If local recovery fails, the original sender repeats the route discovery process by sending another ``Route Request'' packet.
  \end{enumerate}

  The cost of this routing protocol is directly related to the rate of \nameref{def:Mobility} in the network.
  It realy only works for smaller networks with limited \nameref{def:Mobility}.

  It is currently popular in deployed systems.
\end{definition}

\subsubsection{Geographic Routing}\label{subsubsec:Geographic_Routing}
\begin{definition}[Geographic Routing]\label{def:Geographic_Routing}
  \emph{Geographics Routing} is based on the nodes’ \textbf{physical positions}, rather than network addresses or routing tables.
  Once the physically shortest route has been found, messages are then routed towards a destination location.
\end{definition}

However, \nameref{def:Geographic_Routing} has some issues as well.
\begin{itemize}[noitemsep]
\item Route discovery is costly in terms of time and energy.
\item Routing tables quickly become out of date in highly mobile
  networks.
\item Addressing a message to a position may be more useful than
  to a network address.
  \begin{itemize}[noitemsep]
  \item Sending an accident or traffic jam notification to upstream vehicles in a VANET
  \item Sending a request to collect sensor data in a particular region in a WSN
  \item Tracking something through a WSN (e.g.\@ an animal)
  \end{itemize}
\end{itemize}

To make \nameref{def:Geographic_Routing} less costly, we use \nameref{def:Localization}.
\begin{definition}[Localization]\label{def:Localization}
  In \emph{localization}, we find out where each of the nodes are in physical space.
  We can find the absolute locality with GPS, IMU, and many others.
  If we can use the relative locality, then we can use trilateration.
  This is done by having each node broadcast a signal and having each node listen for the same kind of signal.
  This signal is used to locate other beacons by using TDOA.\@
\end{definition}

\subsubsection{Greedy Forwarding}\label{subsubsec:Greedy_Forwarding}
\begin{definition}[Greedy Forwarding]\label{def:Greedy_Forwarding}
  In \emph{greedy forwarding}, we forward this packet to the node \textbf{physically closest} to the destination node.
  This guaranteed there will be no loops in the traversal of the network graph.

  There are various forms and definitions of what ``closest'' means:
  \begin{itemize}[noitemsep]
  \item Distance to target
  \item Distance along the source-destination line
  \item Smallest angle to destination (compass routing)
  \end{itemize}

  \begin{remark}[Greedy Forwarding Failure]\label{rmk:Greedy_Forward_Fail}
    \nameref{def:Greedy_Forwarding} fails when we we hit a local minimum.
    This means that the current node has \textbf{no} neighbors that are closer to the destination that this node.

    However, this can be solved with \nameref{def:Face_Routing}.
  \end{remark}
\end{definition}

\subsubsection{Face Routing}\label{subsubsec:Face_Routing}
\begin{definition}[Face Routing]\label{def:Face_Routing}
  In \emph{face routing}, the route is passed along the faces of a polygon in a set direction.
  The connection links that make the polygon's faces are based off which nodes are along the line between the source and destination nodes.

  The message we are considering is sent around the faces of the polygons using \nameref{def:Greedy_Forwarding} until it either reaches the destination, or you would be required to cross the source-destination line.
  If you have to cross the source-destination line, then you change to the next face.

  \begin{remark}[Planar Graphs]\label{rmk:Face_Routing-Planar_Graphs}
    \nameref{def:Face_Routing} \textbf{requires} that the graph is planar, i.e.\@ there are no crossing links.
    Planarizing the graph can be done by removing crossing edges, using algorithms like Delauney Triangulation, however, this leads to an increase in hop counts.
    This makes the sending of the packet less efficient in terms of delay and energy usage.
  \end{remark}

  \begin{remark}[Localization Errors]\label{rmk:Face_Routing-Localization_Errors}
    \nameref{def:Face_Routing} is sensitive to \nameref{def:Localization} errors, since it relies on geometric information to choose the next node to forward to.
  \end{remark}
\end{definition}

\large{\textbf{We can use BOTH \nameref{def:Greedy_Forwarding} and \nameref{def:Face_Routing}!!}}
\begin{itemize}[noitemsep]
\item Use \nameref{def:Greedy_Forwarding} until we reach a dead-end.
\item Then change to \nameref{def:Face_Routing} to get around the obstacle.
\item Once we are closer to the destination than the dead-end node, we can resume \nameref{def:Greedy_Forwarding}.
\item While using \nameref{def:Greedy_Forwarding}, we can employ all links don’t need a planar graph.
\end{itemize}

\subsubsection{Contention-Based Greedy Forwarding}\label{subsubsec:Contention_Based_Greedy_Forwarding}
With \nameref{def:Greedy_Forwarding}, we want to select the forwarding node closest to the destination.

One way is to keep track of neighbours’ locations, but this increases overhead and breaks down when we have high mobility.

Instead, the source node can send ``Request To Send'' (RTS), and potential forwarding nodes respond with ``Clear To Send'' (CTS) at a time determined by their distance to the destination.

However:
\begin{itemize}[noitemsep]
\item Using this method can increase delay, because you must wait for the timers to expire.
\item We still need to know neighbours’ locations to do \nameref{def:Face_Routing} when recovering from a dead end.
\item Can request locations from neighbours when required.
\end{itemize}

\subsection{Realistic Network Models}\label{subsec:Realistic_Network_Models}
The unit disk that we have been using to approximate the radio transmission of a node is not a good model of the transmission geometry of the device.
\begin{itemize}[noitemsep]
\item The transmission radius is not uniform
\item Three regions:
  \begin{enumerate}[noitemsep]
  \item Good Signal
  \item Degrading Signal (which falls off at a logarithmic rate).
  \item No Signal
\end{enumerate}

\item In the intermediate region (between good and none), we have a link, but it is unreliable.
\end{itemize}

\begin{itemize}[noitemsep]
\item Using \nameref{def:Greedy_Forwarding}, we will select a forwarding node that is far from previous node
  \begin{itemize}[noitemsep]
  \item Likely to land in the region with poor signal
  \item This can lead to increased retransmissions
  \end{itemize}

\item We need to modify our greedy metric to take into account the cost of retransmissions, and balance this against increased hop count as results if we choose a closer node.
\item Nodes can record the number of successfully received packets $S$ and the total number of transmitted packets $T$ on a given link.
\item Packet reception rate (PRR):
  \begin{equation}\label{eq:Packet_Reception_Rate}
    \mathrm{PRR} = \frac{S}{T}
  \end{equation}
  \begin{itemize}[noitemsep]
  \item $S$: Number of successfully received packets.
  \item $T$: The total number of transmitted packets.
  \item $\mathrm{PRR}$: The Packet Reception Rate.
  \end{itemize}
\item The PRR can serve as a reception probability for future transmissions.
\item We can then combine this with the distance gained to select a
  forwarding node,
  \begin{itemize}[noitemsep]
  \item For example, use $\mathrm{PRR} \times \text{distance}$ as forwarding node selection metric, instead of only distance.
  \end{itemize}
\end{itemize}

%%% Local Variables:
%%% mode: latex
%%% TeX-master: "../../ETSN10-Network_Architecture_Performance-Reference_Sheet"
%%% End:
