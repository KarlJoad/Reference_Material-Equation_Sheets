\subsection{Cellular Networks}\label{subsec:Cell_Networks}
Cellular networks are setup in the \nameref{subsubsec:Licensed_Spectrum}s.
They are the major backbone of data infrastructure that is not completely tied to a physical location.

\subsubsection{GSM (2G)}\label{subsubsec:2G}
To reuse the frequencies in GSM, we can subdivide an area into cells.
These cells are idealized to hexagons, and each hexagon gets its own frequency.
Then, any hexagons in adjacent zones cannot have the same frequencies near its border.

The repeating distance of these cells is given in \Cref{eq:GSM_Cell_Repeat_Distance}.
\begin{equation}\label{eq:GSM_Cell_Repeat_Distance}
  D = R \sqrt{3K}
\end{equation}
\begin{itemize}[noitemsep]
\item $R$: Cell radius (How large are the hexagons?)
\item $K$: Cluster size (How many hexagons are allowed?)
\item $D$: The repeating distance (How far between hexagons that repeat the same frequency?)
\end{itemize}

\subsubsection{UMTS (3G)}\label{subsubsec:3G}
Used \nameref{def:CDMA}

\subsubsection{LTE (4G)}\label{subsubsec:4G}
\begin{definition}[LTE]\label{def:LTE}
  \emph{LTE} or \emph{Long Term Evolution} is a standard that was finalized in 2008, and first publicaly available in 2009.
  Its main goals were to increase speed and capacity, while using a simpler IP-based network architecture.
  This was dirven by the large increase in data usage compared to phone calls made.

  The LTE radio interface was incompatible with previous 2G and 3G networks.

  The biggest feature was the concept of \textbf{\nameref{def:Bearer}s}.
  This allows for different Quality of Services for different classes of network traffic.
\end{definition}

\paragraph{Bearers}\label{par:Bearers}
\begin{definition}[Bearer]\label{def:Bearer}
  \emph{Bearer}s are a way to ensure Quality of Service.

  \begin{itemize}[noitemsep]
  \item Minimum Guaranteed Bitrate (GBR)
    \begin{itemize}[noitemsep]
    \item Dedicated resources permanently allocated at bearer establishment
    \item Higher bitrates may be allowed when resources are available
    \end{itemize}

  \item Non-GBR, e.g. FTP, web browsing
    \begin{itemize}[noitemsep]
    \item Best effort service
    \item No resources allocated
    \item QoS class identifier (QCI): priority, packet delay budget, acceptable packet loss rate
    \end{itemize}
  \end{itemize}
\end{definition}

\subsubsection{5G}\label{subsubsec:5G}
\begin{definition}[5G]\label{def:5G}
  \emph{5G} is the next generation of wireless cellular communications.
  It offers many improvements over 4G LTE and will be used nearly everywhere in the digital world.
\end{definition}

\begin{remark*}
  There are a lot of abbreviations in the comming sections.
  I apologize for this, but I have no say in the matter.
  I am including them to make sure there is a good reference for many of these.
\end{remark*}

5G has 3 main use cases:
\begin{enumerate}[noitemsep]
\item Enhanced Mobile Broadband (EMM)
  \begin{itemize}[noitemsep]
  \item The most important aspect is data throughput (How many bits can we send out?).
  \item Quality of Service is also very important, particularly to handle HD video streaming, web browsing, etc.
  \item Energy efficiency \textbf{at the terminal}. This means a longer device battery life.
  \item The main user of this use case are people.
  \end{itemize}
\item Massive Machine Type Communication (MMTC)
  \begin{itemize}[noitemsep]
  \item The most important aspect is the ability to handle \textbf{MANY, MANY} users at once.
  \item In this case, the data rate is not important, as each device has very little data to send.
  \item These packets are either periodic or event-driven.
  \item We also require very low energy usage.
  \item The main user of this use case are Internet of Things devices.
  \end{itemize}
\item Ultra-Reliable Low Latency Communication (URLLC)
  \begin{itemize}[noitemsep]
  \item The most important aspect is the ability to guarantee packet delivery (Absolute or statistical).
  \item There should be bounds on the delay.
  \item The packets are \textbf{VERY} small, as small as a single bit, and event-driven traffic.
  \item This is mainly for vehicular networks and industrial communications.
  \end{itemize}
\end{enumerate}

To support all of these new features with higher data speeds and lower latency, 5G relies on:
\begin{itemize}[noitemsep]
\item \nameref{par:SDN}
\item \nameref{par:NFV}
\item \nameref{par:Network_Slicing}
\item \nameref{par:Cloud_RAN}
\end{itemize}

\paragraph{Software Defined Networking}\label{par:SDN}
\begin{definition}[Software Defined Networking]\label{def:Software_Defined_Networking}\label{def:SDN}
  \emph{Software Defined Networking} (\emph{SDN}) is the concept of centralizing the control of switches and routers in a network.
  This way there is a single \nameref{def:Control_Plane}, and each router/switch can manage its \nameref{def:Data_Plane}.
  By centralizing the \nameref{def:Control_Plane}, there is less work required to configure an entire network, and the system can solve issues itself by reconfiguring the shared \nameref{def:Data_Plane}.

  Previously in 4G, the control functions were located at each entity in the core network.
  In 5G, the control plane and user plane will be separated.

  \begin{remark}[Core 5G Functionality]\label{rmk:SDN_Core_5G}
    \nameref{def:Software_Defined_Networking} is one of the 3 \textbf{MAJOR} functionalities that is enabling 5G to happen.
  \end{remark}
\end{definition}

\begin{definition}[Control Plane]\label{def:Control_Plane}
  The \emph{control plane} of a network's switch and/or router is to define, configure, and handle \textbf{how} the device operates.
  This means the control plane handles:
  \begin{itemize}[noitemsep]
  \item Traffic Management
  \item Admission Policies
  \item Network Management
  \item Routing
  \item Queuing Disciplines
  \end{itemize}

  Some functions of the control plane are:
  \begin{itemize}[noitemsep]
  \item Mobility Management
  \item Policy Management
  \item Subscription Control
  \item End-to-End Path Information
  \end{itemize}
\end{definition}

\begin{definition}[Data Plane]\label{def:Data_Plane}
  The \emph{data plane} of a network's switch and/or router is to handle the transmitted data itself.
  This also means it handles:
  \begin{itemize}[noitemsep]
  \item Forwarding Rules
  \item Queues
  \end{itemize}
\end{definition}

\nameref{def:Software_Defined_Networking} has already been used in fixed networks for a long time.
However, it is just now starting to break into the wireless network space.
As mentioned in \Cref{def:Software_Defined_Networking}, the main idea is to separate the \nameref{def:Control_Plane} from the \nameref{def:Data_Plane}.
\begin{itemize}[noitemsep]
\item Control plane is centralised.
  \begin{itemize}[noitemsep]
  \item Each router only has data plane and northbound interface (NBI).
\end{itemize}
\item NBI is used to report back to controller and update forwarding rules.
\item Can reconfigure the entire network on the fly.
\item Easy to get a view of what is happening in the whole network.
  \begin{itemize}[noitemsep]
  \item Management is easier, generally.
\end{itemize}
\item Adds overhead for signalling.
\item \nameref{def:SDN} avoids the tricky issue of updating rules in such a way that the network is inconsistent.
\end{itemize}

\begin{figure}[h!tbp]
  \centering
  \includegraphics[scale=0.5]{./Drawings/ETSN10-Network_Architecture_Performance/software-defined-networking.png}
  \caption{Software Defined Networking \\ \href{https://www.themetisfiles.com/2012/10/the-future-of-the-network-is-software-defined/}{Source}}
  \label{fig:SDN}
\end{figure}

\paragraph{Network Function Virtualization}\label{par:NFV}
\textbf{TODO!!}

\paragraph{Network Slicing}\label{par:Network_Slicing}
\begin{definition}[Network Slicing]\label{def:Network_Slicing}

\end{definition}
\textbf{TODO!!}

\paragraph{Cloud RAN}\label{par:Cloud_RAN}
\begin{definition}[Cloud Radio Access Network]\label{def:Cloud_RAN}

\end{definition}
\textbf{TODO!!}

\paragraph{Frame Structure}\label{par:Frame_Structure}
\textbf{TODO!!}

\paragraph{Enabling Technologies for 5G}\label{par:5G_Enabling_Technologies}
\textbf{TODO!!}

\subparagraph{Millimeter Wave}\label{subpar:Millimeter_Wave}
\textbf{TODO!!}

\subparagraph{Small Cells}\label{subpar:Small_Cells}
\textbf{TODO!!}

\subparagraph{Massive MIMO}\label{subpar:Massive_MIMO}
\textbf{TODO!!}

%%% Local Variables:
%%% mode: latex
%%% TeX-master: "../../ETSN10-Network_Architecture_Performance-Reference_Sheet"
%%% End:
