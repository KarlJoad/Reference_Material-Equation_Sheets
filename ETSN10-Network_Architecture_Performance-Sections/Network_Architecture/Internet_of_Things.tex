\subsection{Internet of Things}\label{subsec:IoT}
\begin{definition}[Internet of Things]\label{def:IoT}
  The \emph{Internet of Things} (\emph{IoT}) is a type of network where there are small, low-power sensors/devices placed throughout an environment to monitor the condition of the system.
  Some of these devices may also cause a change in the environment as they have been programmed to do.

  The Internet of Things has a hierarchical structure, because of the computing power of the devices.
  Since the actual computational power of any single device in the network is nearly non-existent, the end devices send their data up to the ``fog''.
  This is the place where some data is aggregated, and shared among other devices in the fog for the same user.
  From the fog, it goes up to data centers that form the Cloud.

  \begin{itemize}[noitemsep]
  \item \textbf{Cloud:} Large data centres with resources for computation and storage.
    Typically hosts computation-heavy functions, databases, user interfaces.
  \item \textbf{Fog Nodes:} Can be cellular base stations, local data centres, or local servers.
    Capable of computation and storage, but not as much capacity as in the Cloud.
    Used to host latency-sensitive parts of applications.
  \item \textbf{End Devices:} Sensors and actuators.
    Can be connected to Fog Nodes and backbone network with a range of radio technologies (e.g.\@ cellular, LoRa, WiFi, Bluetooth) or wired connections.
    These can be routed in networks with a single hop (\nameref{def:Star_of_Stars}) or mesh topology.
  \end{itemize}
\end{definition}

\begin{itemize}[noitemsep]
\item Physical objects embedded with sensors, actuators processing power, and network connectivity.
\item “Smart objects” collect and exchange data with each other.
  \begin{itemize}[noitemsep]
  \item Machine-to-machine communication (M2M)
  \end{itemize}

\item Uses a range of different, interacting technologies
  \begin{itemize}[noitemsep]
  \item WSNs, VANETs, WiFi, LoRa, Bluetooth, ZigBee, RFID, $\ldots$
  \end{itemize}

\item Applications in:
  \begin{itemize}[noitemsep]
  \item Healthcare
  \item Energy Management
  \item Transportation
  \item Building Automation
  \item Industrial Processes
  \item $\ldots$ everything!
  \end{itemize}
\end{itemize}

\subsubsection{IoT Performance}\label{subsubsec:IoT_Performance}
The \nameref{def:IoT} is different from previous networks because:
\begin{itemize}[noitemsep]
\item Small, resource-constrained, battery-powered devices
  \begin{itemize}[noitemsep]
  \item Protocols need to be simple for the end device.
  \item Energy efficiency is typically the most important performance metric.
  \end{itemize}

\item Traffic models for IoT:\@ Event-Driven or Periodic.
  \begin{itemize}[noitemsep]
  \item For periodic traffic, scheduled access is usually the most appropriate, e.g. TDMA
  \item For event-driven traffic, random access may be more appropriate, e.g. ALOHA
  \end{itemize}

\item Many devices in the same network
  \begin{itemize}[noitemsep]
  \item If the number of devices is too high, MAC performance starts to degrade.
  \item With many devices, protocol overhead can become significant.
  \item Managing, configuring, and monitoring many devices is difficult.
  \end{itemize}
\end{itemize}

%%% Local Variables:
%%% mode: latex
%%% TeX-master: "../../ETSN10-Network_Architecture_Performance-Reference_Sheet"
%%% End:
