\subsection{Licensed vs. Unlicensed Spectrum}\label{subsec:Licensed_vs_Unlicensed}
There are 2 types of frequencies in use by people and companies today.
\begin{enumerate}[noitemsep]
\item \nameref{subsubsec:Licensed_Spectrum}
\item \nameref{subsubsec:Unlicensed_Spectrum}
\end{enumerate}

\subsubsection{Licensed Spectrum}\label{subsubsec:Licensed_Spectrum}
The licensed spectrum requires that users obtain exclusive rights to use that frequency spectrum.
\begin{itemize}[noitemsep]
\item Solution for QoS-sensitive applications
\item Exclusive right to spectrum use
\item Network engineering possible
  \begin{itemize}[noitemsep]
  \item Predictability
  \item Manageability
  \end{itemize}

\item Complex and costly systems are generally built around these.
\item Typically big players take these.
\end{itemize}

\subsubsection{Unlicensed Spectrum}\label{subsubsec:Unlicensed_Spectrum}
The unlicensed spectrum does not require that users obtain exclusive rights, but they may require that only certain types of data can be used in that frequency.
\begin{itemize}[noitemsep]
\item Inherent best effort systems
\item ``Some'' QoS support possible
\item No right to use spectrum
  \begin{itemize}[noitemsep]
  \item Competition
  \item Collaboration
  \end{itemize}

\item Simple and cheap systems
\item For small/medium players
\end{itemize}

%%% Local Variables:
%%% mode: latex
%%% TeX-master: "../../ETSN10-Network_Architecture_Performance-Reference_Sheet"
%%% End:
