\subsection{Mobility}\label{subsec:Wireless_Mobility}
\begin{definition}[Mobility]\label{def:Mobility}
  \emph{Mobility} is a unique property of wireless networks.
  Since the nodes are not fixed in place by anything, they are free to roam/move.
  This is a big problem, because nodes can enter, leave, and move freely throughout/through the network.
\end{definition}

There are 3 classifications of mobility.
\begin{enumerate}[noitemsep]
\item Pedestrian: $\approx 1.5 \si{\meter/\second}$
\item Vehicle: $5-33 \si{\meter/second}$
\item High-Speed: $> 70 \si{\meter/\second}$
\end{enumerate}

\nameref{def:Mobility} has several effects on a wireless network system.
\begin{enumerate}[noitemsep]
\item Physical Layer
  \begin{itemize}[noitemsep]
  \item Doppler shift
  \item Constantly changing channel $\rightarrow$ higher bit-error rate
  \end{itemize}

\item Data-Link Layer/MAC
  \begin{itemize}[noitemsep]
  \item Which nodes share “links” (and thus can experience collisions) constantly changes.
  \item \nameref{def:Hidden_Node} and \nameref{def:Exposed_Node} problems!
  \end{itemize}

\item Network Layer
  \begin{itemize}[noitemsep]
  \item Constantly changing topology $\rightarrow$ need to constantly change end-to-end path through the network.
  \end{itemize}

\item Transport/Application Layers
  \begin{itemize}[noitemsep]
  \item Not directly affected.
  \item Delays and errors at lower layers can cause problems though.
  \end{itemize}
\end{enumerate}

%%% Local Variables:
%%% mode: latex
%%% TeX-master: "../../ETSN10-Network_Architecture_Performance-Reference_Sheet"
%%% End:
