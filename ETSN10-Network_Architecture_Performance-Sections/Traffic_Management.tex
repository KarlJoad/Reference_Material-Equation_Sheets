\section{Traffic Management}\label{sec:Traffic_Management}
Up to this point, we have covered ways to provide and measure network performance of various layers.
But, isolated mechanisms at different layers in the network will not achieve much on their own.
All of these mechanisms must be coordinated into a single overall strategy for \emph{traffic management}.

\subsection{Steps for Effective Traffic Management}\label{subsec:Steps_Effective_Traffic_Management}
The steps for effective traffic management are below:
\begin{enumerate}[noitemsep]
\item \textbf{Identify} applications' or flows' Quality of Service needs, and \textbf{categorize} the traffic accordingly.
\item \textbf{Enforce} traffic profiles to ensure the network has adequate resources to meet an applications' or flows' Quality of Service needs.
\item \textbf{Monitor} network performance to discover and respond to any problems that may occur.
\item \textbf{Provide} Quality of Service to meet the needs of each flow.
\end{enumerate}

\subsubsection{Identify and Categorize Traffic}\label{subsubsec:Identify_Categorize_Traffic}
Applications need a way to specify their Quality of Service needs and negotiate what kind of service they will receive from the network.
There are some specialized protocols, but they will not be covered in this course.

There are various ways to organize traffic into different classes that receive different Qualities of Service depending on the network's architecture and protocols.
\begin{itemize}[noitemsep]
\item Access Classes in 802.11 EDCA
\item \nameref{def:Bearer}s in \nameref{def:LTE}
\item \nameref{def:Network_Slicing} in \nameref{def:5G}
\end{itemize}

\subsubsection{Enforce Traffic Profiles}\label{subsubsec:Enforce_Traffic_Profiles}
For effective traffic management, the network needs a way to force traffic flows to comply with their agreed-upon traffic profiles, e.g.\ maximum data rate.
\begin{itemize}[noitemsep]
\item Admission Control: Do not allow flows into the network if there are not adequate resources available.
\item \nameref{def:Traffic_Shaping}: For example \nameref{def:Token_Bucket_Scheme}
\item \nameref{def:Flow_Control}: For example a \nameref{def:Credit_Based_Scheme}
\item Some \nameref{def:Queuing_Discipline}s ensure that a single flow cannot take more than its allowed share of bandwidth.
  \begin{itemize}[noitemsep]
  \item \nameref{def:Bit_Round_Fair_Queuing}
  \item \nameref{def:Weighted_Fair_Queuing}
  \item \nameref{def:Class_Based_Queuing}
  \end{itemize}
\end{itemize}

\subsubsection{Monitor Network Performance}\label{subsubsec:Monitor_Network_Performance}
To provide good performance, we have to be able to measure the actual performance currently achieved.
\begin{itemize}[noitemsep]
\item \nameref{def:Congestion_Control} signalling.
\item Monitoring protocols (e.g. Simple Network Management Protocol (SNMP)).
\item Signalling between routers and controller in \nameref{def:Software_Defined_Networking}.
\item Manual monitoring and response by network operators.
\end{itemize}

\subsubsection{Provide Quality of Service}\label{subsubsec:Provide_QoS}
There are many different methods for providing different types of QoS to traffic flows.
Some that are covered in this course include:
\begin{itemize}[noitemsep]
\item \nameref{def:Queuing_Discipline}s
\item Resource allocation and prioritisation in MAC protocols
\item 802.11 EDCA
\item Radio Access Network resource allocation in cellular networks
\item Reservation schemes
\item Random Early Discard (RED)
\item TCP congestion control
\end{itemize}

\subsubsection{Quality of Service Requirements}\label{subsubsec:QoS_Requirements}
QoS requirements for a given traffic flow or set of traffic flows can vary.
Some examples:
\begin{itemize}[noitemsep]
\item Real-time voice and video: low delay, high rate (depending on quality)
\item File transfer: high rate, no packet loss, no bit errors
\item Industrial control system: low delay (even hard deadlines), high reliability (packet loss and errors)
\item Sets of flows: fairness, avoiding starvation of lower priority flows
\end{itemize}


\subsection{Congestion Control}\label{subsec:Congestion_Control}
\defCongestionControl{}
\defCongestion{}

\subsubsection{Token Bucket Scheme}\label{subsubsec:Token_Bucket_Scheme}
\begin{definition}[Token Bucket Scheme]\label{def:Token_Bucket_Scheme}
  In a \emph{token bucket scheme} the transmission sender gets a set of tokens, each of which corresponds to sending a packet, byte, bit, etc.
  To send anything, you must consume one of your tokens, and these are generated over a regular interval.
\end{definition}

To find the maximum number of cells that can depart at any given time, we use \Cref{eq:Token_Bucket_Max_Cells}
\begin{equation}\label{eq:Token_Bucket_Max_Cells}
  R = \rho T + \beta
\end{equation}
where
\begin{description}[noitemsep]
\item $R$ is the Maximum number of cells that can depart from the system with a given number of tokens.
\item $\rho$ is the token generation/arrival rate.
\item $T$ is the amount of time that passes while cells are departing.
\item $\beta$ is the capacity of the token bucket.
\end{description}

To find the amount of time a transmitter can send a burst of information with a given size of token bucket, generation/arrival rate, and transmission speed, we use \Cref{eq:Token_Bucket_Max_Burst_Time}.
\begin{equation}\label{eq:Token_Bucket_Max_Burst_Time}
  T = \frac{\beta}{R_{\mathrm{Max}}-\rho}
\end{equation}
where
\begin{description}[noitemsep]
\item $T$ is the maximum amount of time a transmitter can send cells.
\item $\beta$ is the capacity of the token bucket.
\item $R_{\mathrm{Max}}$ is the maximum cell transmission rate.
\item $\rho$ is the generation/arrival of new tokens.
\end{description}

%%% Local Variables:
%%% mode: latex
%%% TeX-master: "../../ETSN10-Network_Architecture_Performance-Reference_Sheet"
%%% End:


\subsection{Flow Control}\label{subsec:Flow_Control}
\defFlowControl{}
\defSlidingWindow{}

\begin{definition}[Credit-Based Scheme]\label{def:Credit_Based_Scheme}
  In a \emph{credit-based scheme}, the receiver gives the sender ``credit'' that is used when the sender transmits information.
  Unlike the \nameref{def:Token_Bucket_Scheme}, the sender does not ``get credit'' while it is sending information.
  Instead, the sender gets its credit all at once, after sending all it can/wants.
  The rate at which this credit is used is different depending on what is sending, the shape of the network traffic, and other factors.
\end{definition}

\subsubsection{TCP Flow Control}\label{subsubsec:TCP_Flow_Control}
\nameref{def:Transmission_Control_Protocol} uses a form of \nameref{def:Sliding_Window} that differes from windows used in other protocols.
Namely, the \textbf{acknowledgement of received data} and the \textbf{granting of permission} to send more is \textbf{decoupled}.

TCP's flow control is known as a Credit Allocation Scheme, or a \nameref{def:Credit_Based_Scheme}.

\paragraph{TCP Header Fields for Flow Control}\label{par:TCP_Header_Fields_Flow_Control}
For TCP to perform its \nameref{def:Flow_Control}, it must have a way to track which packets have been received.
This way, if any are lost, they can be retransmitted.

\begin{itemize}[noitemsep]
\item This is done with the \emph{Sequence Number} of the packet.
  \begin{itemize}[noitemsep]
  \item Throughout this portion of the document, the sequence number will be represented by $SN$ in mathematical calculations.
  \end{itemize}
\item When the recipient receives the information successfully, it sends an \emph{Acknowledgement (\texttt{ACK}) Number}.
  \begin{itemize}[noitemsep]
  \item Throughout this portion of the document, the acknowledgement number will be represented by $AN$ in mathematical calculations.
  \end{itemize}
\item The control of which $SN$ to send after recieving the $AN$ is determined by the \emph{Window} ($W$).
\end{itemize}

If an \texttt{ACK} contains
\begin{align*}
  AN &= i \\
  W &= j
\end{align*}
then,
\begin{itemize}[noitemsep]
\item $SN$s from $SN = i - 1$ are acknowledged.
\item Permission is granted to send $W = j$ \textbf{more} packets.
  This means packets $i$ to $i + j - 1$ can be sent.
\end{itemize}

\paragraph{Flexible Credit Allocation}\label{par:TCP_Flexible_Credit_Allocation}
The benefit of using this window method in TCP is that the process of allocating credit is flexible, meaning from one round of transmission to the next, different amounts may be sent.

Suppose the last message $B$ sent had
\begin{align*}
  AN &= i \\
  W &= j
\end{align*}

If we want to increase the credit to $k$, i.e.\ $k > j$ when there is no new data, $B$ can issue
\begin{align*}
  AN &= i \\
  W &= k
\end{align*}

If we want to acknowledge a previous segment $i-1$ that contained $m$ packets ($m < j$) \textbf{WITHOUT} allocating more credit, $B$ issues
\begin{align*}
  AN &= i + m \\
  W &= j - m
\end{align*}

\paragraph{Credit Allocation Policy}\label{par:TCP_Credit_Allocation_Policy}
The receive needs a policy for how much credit to give the sender.
There are 2 main approaches:
\begin{enumerate}[noitemsep]
\item Conservative Approach: Grant credit up to the limiit of the current buffer's available space.
  \begin{itemize}[noitemsep]
  \item This may limit throughput in long-delay situations, because the buffer may be nearly full now, but when the data actually arrives to the receiver the queue may be empty and can handle a lot more.
  \end{itemize}
\item Optimistic Approach: Grant credit based on the expectation of freeing space before data arrives.
  \begin{itemize}[noitemsep]
  \item This is based on the fact that there is transmission delay, i.e.\ there is a delay from the time a sender sends their information to the time the receiver receives it.
  \item While the packet is in transmission, the receiver can still process information that is queued, so it grants credit based on its expectation of having free space to handle the packet.
  \item This is dangerous if the space is \textbf{not} available, because lots of packets are discarded.
  \end{itemize}
\end{enumerate}

\paragraph{Effect of Window Size}\label{par:TCP_Window_Size_Effect}
Because the window size is variable in TCP, there are some effects caused by the size of the window.
The \emph{Normalized Throughput} is shown in \Cref{eq:TCP_Normalized_Throughput}.
\begin{equation}\label{eq:TCP_Normalized_Throughput}
  S =
  \begin{cases}
    1 & W \geq 2RD \\
    \frac{W}{2RD} & W < 2RD \\
  \end{cases}
\end{equation}
\begin{description}[noitemsep]
\item $W$ is the TCP window size (bytes/sequence numbers/etc.).
\item $R$ is the data rate (bits/second, bps) at the \textbf{source}.
\item $D$ is the propagation delay (seconds).
  \begin{itemize}[noitemsep]
  \item After the source begins transmitting, it takes $D$ seconds for the first sent packet to arrive at the receiver.
  \item It also takes $D$ seconds for the acknowledgement of the first packet from the receiver to be returned to the sender.
  \item The source could transmit at most $2RD$ bits or $\frac{RD}{4}$ bytes.
    This is called the \textbf{Rate-Delay product}.
  \end{itemize}
\end{description}

\paragraph{Complicating Factors}\label{par:TCP_Complicating_Factors}
There are several factors that complicate the use of TCP's sliding window scheme to perform \nameref{def:Flow_Control}.
\begin{itemize}[noitemsep]
\item Multiple TCP connections are multiplexed over same network interface, reducing available data rate, $R$, and efficiency.
\item For multi-hop connections, $D$ is the sum of delays across \textbf{each} link \textbf{plus} delays at each router.
\item If the sender's data rate, $R$, exceeds the data rate on one of the hops, that hop will be a bottleneck.
\item Lost packets are retransmitted, reducing throughput.
  \begin{itemize}[noitemsep]
  \item The size of the impact depends on retransmission policy.
  \end{itemize}
\end{itemize}

\subsubsection{Retransmission Policies}\label{subsubsec:Retransmission_Policies}
In TCP, and many other protocols, if a packet is lost, it must be retransmitted.
Specifically, in TCP:\@
\begin{itemize}[noitemsep]
\item TCP relies \textbf{exclusively} on positive \texttt{ACK}s, and retransmits on the \texttt{ACK} timing out.
\item There is \textbf{NO} explicit negative \texttt{ACK}.
\item Retransmission is required when:
  \begin{enumerate}[noitemsep]
  \item A segment arrives damaged, as indicated by a checksum error, and the receiver discards the packet.
  \item The segment fails to arrive.
  \end{enumerate}
\end{itemize}

\subsubsection{Timers}\label{subsubsec:Packet_Timers}
\paragraph{Fixed Timers}\label{par:Fixed_Packet_Timers}
\paragraph{Adaptive Timers}\label{par:Adaptive_Packet_Timers}
%%% Local Variables:
%%% mode: latex
%%% TeX-master: "../../ETSN10-Network_Architecture_Performance-Reference_Sheet"
%%% End:


%%% Local Variables:
%%% mode: latex
%%% TeX-master: "../ETSN10-Network_Architecture_Performance-Reference_Sheet"
%%% End:
