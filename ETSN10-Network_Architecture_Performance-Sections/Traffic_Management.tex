\section{Traffic Management}\label{sec:Traffic_Management}
Up to this point, we have covered ways to provide and measure network performance of various layers.
But, isolated mechanisms at different layers in the network will not achieve much on their own.
All of these mechanisms must be coordinated into a single overall strategy for \emph{traffic management}.

\subsection{Steps for Effective Traffic Management}\label{subsec:Steps_Effective_Traffic_Management}
The steps for effective traffic management are below:
\begin{enumerate}[noitemsep]
\item \textbf{Identify} applications' or flows' Quality of Service needs, and \textbf{categorize} the traffic accordingly.
\item \textbf{Enforce} traffic profiles to ensure the network has adequate resources to meet an applications' or flows' Quality of Service needs.
\item \textbf{Monitor} network performance to discover and respond to any problems that may occur.
\item \textbf{Provide} Quality of Service to meet the needs of each flow.
\end{enumerate}


\subsection{Congestion Control}\label{subsec:Congestion_Control}
\defCongestionControl{}
\defCongestion{}

\subsubsection{Congestion Control in TCP}\label{subsubsec:Congestion_Control_TCP}
In \nameref{def:Transmission_Control_Protocol}, there is a problem that occurs because of its greedy backoff procedure.

\begin{definition}[Global Synchronization Problem]\label{def:TCP_Global_Synchronization_Problem}
  \nameref{def:Transmission_Control_Protocol}'s \emph{global synchronization problem} occurs when a sudden burst of traffic causes simultaneous packet loss across many TCP sessions using a single (congested) link.
  Each of the affected TCP session backs off its send rate at the same time, causing link utilization to go way down.
\end{definition}

\subsubsection{Random Early Discard}\label{subsubsec:Random_Early_Discard}
\paragraph{RED Algorithm}\label{par:Random_Early_Discard_Algorithm}
The variables that do not have descriptive names are given below:
\begin{description}[noitemsep]
\item $q\_time$: Start of the queue's idle time.
\item $w_{\mathrm{queue}}$: Weight of the queue.
\item $TH_{\mathrm{Min}}$: Minimum threshold for queue.
\item $TH_{\mathrm{Max}}$: Maximum threshold for queue.
\item $P_{a}$: Current packet-marking probability.
\item $P_{b}$: Temporary Probability used for calculations.
  \begin{itemize}[noitemsep]
  \item Using $P_{b}$ directly tends to penalize bursty traffic prematurely
    \begin{itemize}[noitemsep]
    \item Proportional to $F=(avgQueueSize-TH_{\mathrm{Min}})/(TH_{\mathrm{Max}}-TH_{\mathrm{Min}})$
    \end{itemize}
\item $P_{a}$ remains low and then rises quickly when count approaches $1/(F \times P_{\mathrm{Max}})-1$
\item Result: number of packets allowed to join queue between discards is uniformly distributed in $\left[1, 2, \ldots, \frac{1}{P_{b}} \right]$
  \end{itemize}
\item $P_{\mathrm{Max}}$: Maximum value for $P_{b}$.
\item $f(t)$: A linear function of time $t$.
\end{description}

\begin{algorithm}[H]
  \DontPrintSemicolon{}
  \SetKwInOut{Input}{Input}\SetKwInOut{Output}{Output}
  \SetKwData{AvgQueueSize}{avgQueueSize}
  \SetKwData{CountLastDiscard}{CountLastDiscardPacket}
  \SetKwData{QueueIdleTime}{QueueIdleTime}
  \SetKwData{CurrentTime}{CurrentTime}
  \SetKwData{CurrentQueueSize}{CurrentQueueSize}
  \BlankLine{}

  Initialization: \;
  \AvgQueueSize$\leftarrow 0$ \;
  \CountLastDiscard$\leftarrow -1$ \;

  \ForEach{packet arrival}{
    \textbf{Calculate the average queue size} \;
    \eIf{queue not empty, $q > 0$}{
      \AvgQueueSize$\leftarrow (1-w_{\mathrm{queue}})$\AvgQueueSize$+ w_{\mathrm{queue}}$\CurrentQueueSize \;
    }{
      $m \leftarrow f($\CurrentTime$- q\_time)$ \;
      \AvgQueueSize$\leftarrow$\AvgQueueSize$\times (1-w_{q})m$ \;
    }
    \textbf{Determine Packet Discard} \;
    \uIf{\AvgQueueSize$<TH_{\mathrm{Min}}$}{
      queue packet \;
      \CountLastDiscard$\leftarrow -1$ \;
    }\uElseIf{$TH_{\mathrm{Min}} \leq$\AvgQueueSize$\leq TH_{\mathrm{Max}}$}{
      \CountLastDiscard$\leftarrow$\CountLastDiscard$+1$ \;
      $P_{b} \leftarrow P_{\mathrm{Max}}($\AvgQueueSize$- TH_{\mathrm{Min}}) / (TH_{\mathrm{Max}} - TH_{\mathrm{Min}})$ \;
      $P_{a} \leftarrow P_{b} / (1-$\CountLastDiscard$\times P_{b})$ \;
      \eIf{Probability $P_{a}$}{
        discard packet \;
        \CountLastDiscard$\leftarrow 0$ \;
      }(Probability $1-P_{a}$){
        queue packet \;
      }
    }\Else(\AvgQueueSize$> TH_{\mathrm{Max}}$){
      discard packet \;
      \CountLastDiscard$\leftarrow 0$ \;
    }
  }

  When the queue becomes empty: \;
  \QueueIdleTime$\leftarrow$\CurrentTime
  \caption{Random Early Discard (RED)}
  \label{algo:Random_Early_Discard}
\end{algorithm}

%%% Local Variables:
%%% mode: latex
%%% TeX-master: "../../ETSN10-Network_Architecture_Performance-Reference_Sheet"
%%% End:


\subsection{Flow Control}\label{subsec:Flow_Control}
\defFlowControl{}
\defSlidingWindow{}

\begin{definition}[Traffic Shaping]\label{def:Traffic_Shaping}
  \emph{Traffic Shaping} is used to make \nameref{def:Stochastic_Process} more deterministic.
  During the transfer of data, we force compliance (shape the traffic) by controlling the rate at which data is sent out from the node.
  This can typically be done with \nameref{def:Token_Bucket_System}s, among other options.
\end{definition}

\nameref{def:Traffic_Shaping} is typically done by determining the total capacity for all flows aggregated together.
Then, each of the flows is divided into classes that gives each flow certain properties (voice, video, data, etc.).

\begin{algorithm}[H]
  \DontPrintSemicolon{}
  \SetKwInOut{Input}{Input}\SetKwInOut{Output}{Output}
  \Input{Data Flows and their aggregated output capacity}
  \BlankLine{}

  \For{each flow}{
    \If{used\_capacity + (new flow) $<$ total\_capacity}{
      admit\;
    }\Else{
      no admission\;
    }
  }
\end{algorithm}

\subsubsection{Token Bucket Scheme}\label{subsubsec:Token_Bucket_Scheme}
\begin{definition}[Token Bucket Scheme]\label{def:Token_Bucket_Scheme}
  In a \emph{token bucket scheme} the transmission sender gets a set of tokens, each of which corresponds to sending a packet, byte, bit, etc.
  To send anything, you must consume one of your tokens, and these are generated over a regular interval.
\end{definition}

To find the maximum number of cells that can depart at any given time, we use \Cref{eq:Token_Bucket_Max_Cells}
\begin{equation}\label{eq:Token_Bucket_Max_Cells}
  R = \rho T + \beta
\end{equation}
where
\begin{description}[noitemsep]
\item $R$ is the Maximum number of cells that can depart from the system with a given number of tokens.
\item $\rho$ is the token generation/arrival rate.
\item $T$ is the amount of time that passes while cells are departing.
\item $\beta$ is the capacity of the token bucket.
\end{description}

%%% Local Variables:
%%% mode: latex
%%% TeX-master: "../../ETSN10-Network_Architecture_Performance-Reference_Sheet"
%%% End:


%%% Local Variables:
%%% mode: latex
%%% TeX-master: "../ETSN10-Network_Architecture_Performance-Reference_Sheet"
%%% End:
