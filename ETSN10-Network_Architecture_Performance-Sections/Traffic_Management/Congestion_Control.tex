\subsection{Congestion Control}\label{subsec:Congestion_Control}
\defCongestionControl{}
\defCongestion{}

\subsubsection{Token Bucket Scheme}\label{subsubsec:Token_Bucket_Scheme}
\begin{definition}[Token Bucket Scheme]\label{def:Token_Bucket_Scheme}
  In a \emph{token bucket scheme} the transmission sender gets a set of tokens, each of which corresponds to sending a packet, byte, bit, etc.
  To send anything, you must consume one of your tokens, and these are generated over a regular interval.
\end{definition}

To find the maximum number of cells that can depart at any given time, we use \Cref{eq:Token_Bucket_Max_Cells}
\begin{equation}\label{eq:Token_Bucket_Max_Cells}
  R = \rho T + \beta
\end{equation}
where
\begin{description}[noitemsep]
\item $R$ is the Maximum number of cells that can depart from the system with a given number of tokens.
\item $\rho$ is the token generation/arrival rate.
\item $T$ is the amount of time that passes while cells are departing.
\item $\beta$ is the capacity of the token bucket.
\end{description}

To find the amount of time a transmitter can send a burst of information with a given size of token bucket, generation/arrival rate, and transmission speed, we use \Cref{eq:Token_Bucket_Max_Burst_Time}.
\begin{equation}\label{eq:Token_Bucket_Max_Burst_Time}
  T = \frac{\beta}{R_{\mathrm{Max}}-\rho}
\end{equation}
where
\begin{description}[noitemsep]
\item $T$ is the maximum amount of time a transmitter can send cells.
\item $\beta$ is the capacity of the token bucket.
\item $R_{\mathrm{Max}}$ is the maximum cell transmission rate.
\item $\rho$ is the generation/arrival of new tokens.
\end{description}

%%% Local Variables:
%%% mode: latex
%%% TeX-master: "../../ETSN10-Network_Architecture_Performance-Reference_Sheet"
%%% End:
