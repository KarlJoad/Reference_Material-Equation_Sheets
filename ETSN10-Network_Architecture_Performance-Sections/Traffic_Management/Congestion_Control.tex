\subsection{Congestion Control}\label{subsec:Congestion_Control}
\defCongestionControl{}
\defCongestion{}

In \nameref{def:Transmission_Control_Protocol}, there is a problem that occurs because of its greedy backoff procedure.
If we think back to \nameref{sec:Queuing_Theory}, a data network can be modeled as a network of queues.
Remember, if $\lambda > \mu$ (arrival rate $>$ transmission rate), the queue size grows without bound and packet delay will go to $\infty$.

\begin{definition}[Global Synchronization Problem]\label{def:TCP_Global_Synchronization_Problem}
  \nameref{def:Transmission_Control_Protocol}'s \emph{global synchronization problem} occurs when a sudden burst of traffic causes simultaneous packet loss across many TCP sessions using a single (congested) link.
  Each of the affected TCP session backs off its send rate at the same time, causing link utilization to go way down.
If a network reaches its saturation point, one main strategy is to discard any incoming packet if there is no buffer available.
\begin{itemize}[noitemsep]
\item Saturated nodes exercise some amount of \nameref{def:Flow_Control} over its neighbors.
\item This may cause congestion to propagate throughout the network.
\end{itemize}

\subsubsection{Congestion Control Approaches}\label{subsubsec:Congestion_Control_Approaches}
There are 2 main approaches to controlling congestion:
\begin{enumerate}[noitemsep]
\item Explicit Congestion Signalling (Backpressure)
  \begin{itemize}[noitemsep]
  \item The destination requests the source to reduce its rate.
  \item The Choke Packet: ICMP Source Quench
  \item These can be done in 2 directions:
    \begin{enumerate}[noitemsep]
    \item Backward
    \item Foward
    \end{enumerate}
  \item There are several categories that allow for this:
    \begin{itemize}[noitemsep]
    \item Binary
    \item \nameref{def:Credit_Based_Scheme}
    \item Rate-Based scheme (like \nameref{def:Token_Bucket_Scheme})
    \end{itemize}
  \end{itemize}

\item Implicit Congestion Signalling
  \begin{itemize}[noitemsep]
  \item The source detects congestion from transmission delays and discarded packets, so it reduces its output.
  \end{itemize}
\end{enumerate}

\subsubsection{Packet Discard}\label{subsubsec:Packet_Discard}
Sometimes, you have to drop/discard packets to keep congestion under control.
Systems that do this use \nameref{def:Proactive_Packet_Discard}.

\begin{definition}[Proactive Packet Discard]\label{def:Proactive_Packet_Discard}
  In a \emph{proactive packet discard} system, packets are deliberately discarded before being sent to keep congestion under control.
\end{definition}

One of the most frequently used protocols is \nameref{def:Random_Early_Discard}.

\subsubsection{Random Early Discard}\label{subsubsec:Random_Early_Discard}
\paragraph{RED Motivations}\label{par:Random_Early_Discard_Motivations}
\paragraph{RED Design Goals}\label{par:Random_Early_Discard_Design_Goals}
\paragraph{RED Algorithm}\label{par:Random_Early_Discard_Algorithm}
The variables that do not have descriptive names are given below:
\begin{description}[noitemsep]
\item $q\_time$: Start of the queue's idle time.
\item $w_{\mathrm{queue}}$: Weight of the queue.
\item $TH_{\mathrm{Min}}$: Minimum threshold for queue.
\item $TH_{\mathrm{Max}}$: Maximum threshold for queue.
\item $P_{a}$: Current packet-marking probability.
\item $P_{b}$: Temporary Probability used for calculations.
  \begin{itemize}[noitemsep]
  \item Using $P_{b}$ directly tends to penalize bursty traffic prematurely
    \begin{itemize}[noitemsep]
    \item Proportional to $F=(avgQueueSize-TH_{\mathrm{Min}})/(TH_{\mathrm{Max}}-TH_{\mathrm{Min}})$
    \end{itemize}
\item $P_{a}$ remains low and then rises quickly when count approaches $1/(F \times P_{\mathrm{Max}})-1$
\item Result: number of packets allowed to join queue between discards is uniformly distributed in $\left[1, 2, \ldots, \frac{1}{P_{b}} \right]$
  \end{itemize}
\item $P_{\mathrm{Max}}$: Maximum value for $P_{b}$.
\item $f(t)$: A linear function of time $t$.
\end{description}

\begin{algorithm}[H]
  \DontPrintSemicolon{}
  \SetKwInOut{Input}{Input}\SetKwInOut{Output}{Output}
  \SetKwData{AvgQueueSize}{avgQueueSize}
  \SetKwData{CountLastDiscard}{CountLastDiscardPacket}
  \SetKwData{QueueIdleTime}{QueueIdleTime}
  \SetKwData{CurrentTime}{CurrentTime}
  \SetKwData{CurrentQueueSize}{CurrentQueueSize}
  \BlankLine{}

  Initialization: \;
  \AvgQueueSize$\leftarrow 0$ \;
  \CountLastDiscard$\leftarrow -1$ \;

  \ForEach{packet arrival}{
    \textbf{Calculate the average queue size} \;
    \eIf{queue not empty, $q > 0$}{
      \AvgQueueSize$\leftarrow (1-w_{\mathrm{queue}})$\AvgQueueSize$+ w_{\mathrm{queue}}$\CurrentQueueSize \;
    }{
      $m \leftarrow f($\CurrentTime$- q\_time)$ \;
      \AvgQueueSize$\leftarrow$\AvgQueueSize$\times (1-w_{q})m$ \;
    }
    \textbf{Determine Packet Discard} \;
    \uIf{\AvgQueueSize$<TH_{\mathrm{Min}}$}{
      queue packet \;
      \CountLastDiscard$\leftarrow -1$ \;
    }\uElseIf{$TH_{\mathrm{Min}} \leq$\AvgQueueSize$\leq TH_{\mathrm{Max}}$}{
      \CountLastDiscard$\leftarrow$\CountLastDiscard$+1$ \;
      $P_{b} \leftarrow P_{\mathrm{Max}}($\AvgQueueSize$- TH_{\mathrm{Min}}) / (TH_{\mathrm{Max}} - TH_{\mathrm{Min}})$ \;
      $P_{a} \leftarrow P_{b} / (1-$\CountLastDiscard$\times P_{b})$ \;
      \eIf{Probability $P_{a}$}{
        discard packet \;
        \CountLastDiscard$\leftarrow 0$ \;
      }(Probability $1-P_{a}$){
        queue packet \;
      }
    }\Else(\AvgQueueSize$> TH_{\mathrm{Max}}$){
      discard packet \;
      \CountLastDiscard$\leftarrow 0$ \;
    }
  }

  When the queue becomes empty: \;
  \QueueIdleTime$\leftarrow$\CurrentTime
  \caption{Random Early Discard (RED)}
  \label{algo:Random_Early_Discard}
\end{algorithm}

\subsubsection{Congestion Control in TCP}\label{subsubsec:Congestion_Control_TCP}
%%% Local Variables:
%%% mode: latex
%%% TeX-master: "../../ETSN10-Network_Architecture_Performance-Reference_Sheet"
%%% End:
