\subsection{Flow Control}\label{subsec:Flow_Control}
\defFlowControl{}
\defSlidingWindow{}

\begin{definition}[Traffic Shaping]\label{def:Traffic_Shaping}
  \emph{Traffic Shaping} is used to make \nameref{def:Stochastic_Process} more deterministic.
  During the transfer of data, we force compliance (shape the traffic) by controlling the rate at which data is sent out from the node.
  This can typically be done with \nameref{def:Token_Bucket_System}s, among other options.
\end{definition}

\nameref{def:Traffic_Shaping} is typically done by determining the total capacity for all flows aggregated together.
Then, each of the flows is divided into classes that gives each flow certain properties (voice, video, data, etc.).

\begin{algorithm}[H]
  \DontPrintSemicolon{}
  \SetKwInOut{Input}{Input}\SetKwInOut{Output}{Output}
  \Input{Data Flows and their aggregated output capacity}
  \BlankLine{}

  \For{each flow}{
    \If{used\_capacity + (new flow) $<$ total\_capacity}{
      admit\;
    }\Else{
      no admission\;
    }
  }
\end{algorithm}

\subsubsection{Token Bucket Scheme}\label{subsubsec:Token_Bucket_Scheme}
\begin{definition}[Token Bucket Scheme]\label{def:Token_Bucket_Scheme}
  In a \emph{token bucket scheme} the transmission sender gets a set of tokens, each of which corresponds to sending a packet, byte, bit, etc.
  To send anything, you must consume one of your tokens, and these are generated over a regular interval.
\end{definition}

%%% Local Variables:
%%% mode: latex
%%% TeX-master: "../../ETSN10-Network_Architecture_Performance-Reference_Sheet"
%%% End:
