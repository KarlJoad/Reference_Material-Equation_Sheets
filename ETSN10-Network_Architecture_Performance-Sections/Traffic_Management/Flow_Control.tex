\subsection{Flow Control}\label{subsec:Flow_Control}
\defFlowControl{}
\defSlidingWindow{}

\begin{definition}[Credit-Based Scheme]\label{def:Credit_Based_Scheme}
  In a \emph{credit-based scheme}, the receiver gives the sender ``credit'' that is used when the sender transmits information.
  Unlike the \nameref{def:Token_Bucket_Scheme}, the sender does not ``get credit'' while it is sending information.
  Instead, the sender gets its credit all at once, after sending all it can/wants.
  The rate at which this credit is used is different depending on what is sending, the shape of the network traffic, and other factors.
\end{definition}

\subsubsection{TCP Flow Control}\label{subsubsec:TCP_Flow_Control}
\nameref{def:Transmission_Control_Protocol} uses a form of \nameref{def:Sliding_Window} that differes from windows used in other protocols.
Namely, the \textbf{acknowledgement of received data} and the \textbf{granting of permission} to send more is \textbf{decoupled}.

TCP's flow control is known as a Credit Allocation Scheme, or a \nameref{def:Credit_Based_Scheme}.

\paragraph{TCP Header Fields for Flow Control}\label{par:TCP_Header_Fields_Flow_Control}
For TCP to perform its \nameref{def:Flow_Control}, it must have a way to track which packets have been received.
This way, if any are lost, they can be retransmitted.

\begin{itemize}[noitemsep]
\item This is done with the \emph{Sequence Number} of the packet.
  \begin{itemize}[noitemsep]
  \item Throughout this portion of the document, the sequence number will be represented by $SN$ in mathematical calculations.
  \end{itemize}
\item When the recipient receives the information successfully, it sends an \emph{Acknowledgement (\texttt{ACK}) Number}.
  \begin{itemize}[noitemsep]
  \item Throughout this portion of the document, the acknowledgement number will be represented by $AN$ in mathematical calculations.
  \end{itemize}
\item The control of which $SN$ to send after recieving the $AN$ is determined by the \emph{Window} ($W$).
\end{itemize}

If an \texttt{ACK} contains
\begin{align*}
  AN &= i \\
  W &= j
\end{align*}
then,
\begin{itemize}[noitemsep]
\item $SN$s from $SN = i - 1$ are acknowledged.
\item Permission is granted to send $W = j$ \textbf{more} packets.
  This means packets $i$ to $i + j - 1$ can be sent.
\end{itemize}

\paragraph{Flexible Credit Allocation}\label{par:TCP_Flexible_Credit_Allocation}
The benefit of using this window method in TCP is that the process of allocating credit is flexible, meaning from one round of transmission to the next, different amounts may be sent.

Suppose the last message $B$ sent had
\begin{align*}
  AN &= i \\
  W &= j
\end{align*}

If we want to increase the credit to $k$, i.e.\ $k > j$ when there is no new data, $B$ can issue
\begin{align*}
  AN &= i \\
  W &= k
\end{align*}

If we want to acknowledge a previous segment $i-1$ that contained $m$ packets ($m < j$) \textbf{WITHOUT} allocating more credit, $B$ issues
\begin{align*}
  AN &= i + m \\
  W &= j - m
\end{align*}

\paragraph{Credit Allocation Policy}\label{par:TCP_Credit_Allocation_Policy}
The receive needs a policy for how much credit to give the sender.
There are 2 main approaches:
\begin{enumerate}[noitemsep]
\item Conservative Approach: Grant credit up to the limiit of the current buffer's available space.
  \begin{itemize}[noitemsep]
  \item This may limit throughput in long-delay situations, because the buffer may be nearly full now, but when the data actually arrives to the receiver the queue may be empty and can handle a lot more.
  \end{itemize}
\item Optimistic Approach: Grant credit based on the expectation of freeing space before data arrives.
  \begin{itemize}[noitemsep]
  \item This is based on the fact that there is transmission delay, i.e.\ there is a delay from the time a sender sends their information to the time the receiver receives it.
  \item While the packet is in transmission, the receiver can still process information that is queued, so it grants credit based on its expectation of having free space to handle the packet.
  \item This is dangerous if the space is \textbf{not} available, because lots of packets are discarded.
  \end{itemize}
\end{enumerate}

\paragraph{Effect of Window Size}\label{par:TCP_Window_Size_Effect}
Because the window size is variable in TCP, there are some effects caused by the size of the window.
The \emph{Normalized Throughput} is shown in \Cref{eq:TCP_Normalized_Throughput}.
\begin{equation}\label{eq:TCP_Normalized_Throughput}
  S =
  \begin{cases}
    1 & W \geq 2RD \\
    \frac{W}{2RD} & W < 2RD \\
  \end{cases}
\end{equation}
\begin{description}[noitemsep]
\item $W$ is the TCP window size (bytes/sequence numbers/etc.).
\item $R$ is the data rate (bits/second, bps) at the \textbf{source}.
\item $D$ is the propagation delay (seconds).
  \begin{itemize}[noitemsep]
  \item After the source begins transmitting, it takes $D$ seconds for the first sent packet to arrive at the receiver.
  \item It also takes $D$ seconds for the acknowledgement of the first packet from the receiver to be returned to the sender.
  \item The source could transmit at most $2RD$ bits or $\frac{RD}{4}$ bytes.
    This is called the \textbf{Rate-Delay product}.
  \end{itemize}
\end{description}

\paragraph{Complicating Factors}\label{par:TCP_Complicating_Factors}
\subsubsection{Retransmission Policies}\label{subsubsec:Retransmission_Policies}
\subsubsection{Timers}\label{subsubsec:Packet_Timers}
\paragraph{Fixed Timers}\label{par:Fixed_Packet_Timers}
\paragraph{Adaptive Timers}\label{par:Adaptive_Packet_Timers}
%%% Local Variables:
%%% mode: latex
%%% TeX-master: "../../ETSN10-Network_Architecture_Performance-Reference_Sheet"
%%% End:
