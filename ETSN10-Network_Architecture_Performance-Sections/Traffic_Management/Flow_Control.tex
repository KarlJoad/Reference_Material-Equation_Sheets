\subsection{Flow Control}\label{subsec:Flow_Control}
\defFlowControl{}
\defSlidingWindow{}

\begin{definition}[Credit-Based Scheme]\label{def:Credit_Based_Scheme}
  In a \emph{credit-based scheme}, the receiver gives the sender ``credit'' that is used when the sender transmits information.
  Unlike the \nameref{def:Token_Bucket_Scheme}, the sender does not ``get credit'' while it is sending information.
  Instead, the sender gets its credit all at once, after sending all it can/wants.
  The rate at which this credit is used is different depending on what is sending, the shape of the network traffic, and other factors.
\end{definition}

\subsubsection{TCP Flow Control}\label{subsubsec:TCP_Flow_Control}
\nameref{def:Transmission_Control_Protocol} uses a form of \nameref{def:Sliding_Window} that differes from windows used in other protocols.
Namely, the \textbf{acknowledgement of received data} and the \textbf{granting of permission} to send more is \textbf{decoupled}.

TCP's flow control is known as a Credit Allocation Scheme, or a \nameref{def:Credit_Based_Scheme}.

\paragraph{TCP Header Fields for Flow Control}\label{par:TCP_Header_Fields_Flow_Control}
\paragraph{Flexible Credit Allocation}\label{par:TCP_Flexible_Credit_Allocation}
\paragraph{Credit Allocation Policy}\label{par:TCP_Credit_Allocation_Policy}
\paragraph{Effect of Window Size}\label{par:TCP_Window_Size_Effect}
\paragraph{Complicating Factors}\label{par:TCP_Complicating_Factors}
\subsubsection{Retransmission Policies}\label{subsubsec:Retransmission_Policies}
\subsubsection{Timers}\label{subsubsec:Packet_Timers}
\paragraph{Fixed Timers}\label{par:Fixed_Packet_Timers}
\paragraph{Adaptive Timers}\label{par:Adaptive_Packet_Timers}
%%% Local Variables:
%%% mode: latex
%%% TeX-master: "../../ETSN10-Network_Architecture_Performance-Reference_Sheet"
%%% End:
