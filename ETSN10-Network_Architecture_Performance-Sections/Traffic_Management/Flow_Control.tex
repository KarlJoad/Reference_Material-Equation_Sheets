\subsection{Flow Control}\label{subsec:Flow_Control}
\defFlowControl{}
\defSlidingWindow{}

\begin{definition}[Credit-Based Scheme]\label{def:Credit_Based_Scheme}
  In a \emph{credit-based scheme}, the receiver gives the sender ``credit'' that is used when the sender transmits information.
  Unlike the \nameref{def:Token_Bucket_Scheme}, the sender does not ``get credit'' while it is sending information.
  Instead, the sender gets its credit all at once, after sending all it can/wants.
  The rate at which this credit is used is different depending on what is sending, the shape of the network traffic, and other factors.
\end{definition}

\subsubsection{TCP Flow Control}\label{subsubsec:TCP_Flow_Control}
\nameref{def:Transmission_Control_Protocol} uses a form of \nameref{def:Sliding_Window} that differes from windows used in other protocols.
Namely, the \textbf{acknowledgement of received data} and the \textbf{granting of permission} to send more is \textbf{decoupled}.

TCP's flow control is known as a Credit Allocation Scheme, or a \nameref{def:Credit_Based_Scheme}.

\paragraph{TCP Header Fields for Flow Control}\label{par:TCP_Header_Fields_Flow_Control}
For TCP to perform its \nameref{def:Flow_Control}, it must have a way to track which packets have been received.
This way, if any are lost, they can be retransmitted.

\begin{itemize}[noitemsep]
\item This is done with the \emph{Sequence Number} of the packet.
  \begin{itemize}[noitemsep]
  \item Throughout this portion of the document, the sequence number will be represented by $SN$ in mathematical calculations.
  \end{itemize}
\item When the recipient receives the information successfully, it sends an \emph{Acknowledgement (\texttt{ACK}) Number}.
  \begin{itemize}[noitemsep]
  \item Throughout this portion of the document, the acknowledgement number will be represented by $AN$ in mathematical calculations.
  \end{itemize}
\item The control of which $SN$ to send after recieving the $AN$ is determined by the \emph{Window} ($W$).
\end{itemize}

If an \texttt{ACK} contains
\begin{align*}
  AN &= i \\
  W &= j
\end{align*}
then,
\begin{itemize}[noitemsep]
\item $SN$s from $SN = i - 1$ are acknowledged.
\item Permission is granted to send $W = j$ \textbf{more} packets.
  This means packets $i$ to $i + j - 1$ can be sent.
\end{itemize}

\paragraph{Flexible Credit Allocation}\label{par:TCP_Flexible_Credit_Allocation}
The benefit of using this window method in TCP is that the process of allocating credit is flexible, meaning from one round of transmission to the next, different amounts may be sent.

Suppose the last message $B$ sent had
\begin{align*}
  AN &= i \\
  W &= j
\end{align*}

If we want to increase the credit to $k$, i.e.\ $k > j$ when there is no new data, $B$ can issue
\begin{align*}
  AN &= i \\
  W &= k
\end{align*}

If we want to acknowledge a previous segment $i-1$ that contained $m$ packets ($m < j$) \textbf{WITHOUT} allocating more credit, $B$ issues
\begin{align*}
  AN &= i + m \\
  W &= j - m
\end{align*}

\paragraph{Credit Allocation Policy}\label{par:TCP_Credit_Allocation_Policy}
The receive needs a policy for how much credit to give the sender.
There are 2 main approaches:
\begin{enumerate}[noitemsep]
\item Conservative Approach: Grant credit up to the limiit of the current buffer's available space.
  \begin{itemize}[noitemsep]
  \item This may limit throughput in long-delay situations, because the buffer may be nearly full now, but when the data actually arrives to the receiver the queue may be empty and can handle a lot more.
  \end{itemize}
\item Optimistic Approach: Grant credit based on the expectation of freeing space before data arrives.
  \begin{itemize}[noitemsep]
  \item This is based on the fact that there is transmission delay, i.e.\ there is a delay from the time a sender sends their information to the time the receiver receives it.
  \item While the packet is in transmission, the receiver can still process information that is queued, so it grants credit based on its expectation of having free space to handle the packet.
  \item This is dangerous if the space is \textbf{not} available, because lots of packets are discarded.
  \end{itemize}
\end{enumerate}

\paragraph{Effect of Window Size}\label{par:TCP_Window_Size_Effect}
Because the window size is variable in TCP, there are some effects caused by the size of the window.
The \emph{Normalized Throughput} is shown in \Cref{eq:TCP_Normalized_Throughput}.
\begin{equation}\label{eq:TCP_Normalized_Throughput}
  S =
  \begin{cases}
    1 & W \geq 2RD \\
    \frac{W}{2RD} & W < 2RD \\
  \end{cases}
\end{equation}
\begin{description}[noitemsep]
\item $W$ is the TCP window size (bytes/sequence numbers/etc.).
\item $R$ is the data rate (bits/second, bps) at the \textbf{source}.
\item $D$ is the propagation delay (seconds).
  \begin{itemize}[noitemsep]
  \item After the source begins transmitting, it takes $D$ seconds for the first sent packet to arrive at the receiver.
  \item It also takes $D$ seconds for the acknowledgement of the first packet from the receiver to be returned to the sender.
  \item The source could transmit at most $2RD$ bits or $\frac{RD}{4}$ bytes.
    This is called the \textbf{Rate-Delay product}.
  \end{itemize}
\end{description}

\paragraph{Complicating Factors}\label{par:TCP_Complicating_Factors}
There are several factors that complicate the use of TCP's sliding window scheme to perform \nameref{def:Flow_Control}.
\begin{itemize}[noitemsep]
\item Multiple TCP connections are multiplexed over same network interface, reducing available data rate, $R$, and efficiency.
\item For multi-hop connections, $D$ is the sum of delays across \textbf{each} link \textbf{plus} delays at each router.
\item If the sender's data rate, $R$, exceeds the data rate on one of the hops, that hop will be a bottleneck.
\item Lost packets are retransmitted, reducing throughput.
  \begin{itemize}[noitemsep]
  \item The size of the impact depends on retransmission policy.
  \end{itemize}
\end{itemize}

\subsubsection{Retransmission Policies}\label{subsubsec:Retransmission_Policies}
In TCP, and many other protocols, if a packet is lost, it must be retransmitted.
Specifically, in TCP:\@
\begin{itemize}[noitemsep]
\item TCP relies \textbf{exclusively} on positive \texttt{ACK}s, and retransmits on the \texttt{ACK} timing out.
\item There is \textbf{NO} explicit negative \texttt{ACK}.
\item Retransmission is required when:
  \begin{enumerate}[noitemsep]
  \item A segment arrives damaged, as indicated by a checksum error, and the receiver discards the packet.
  \item The segment fails to arrive.
  \end{enumerate}
\end{itemize}

\subsubsection{Timers}\label{subsubsec:Packet_Timers}
To make a set of retransmission policies work, a set of timers is used.
\begin{itemize}[noitemsep]
\item A timer is associated with each segment as it is sent, i.e.\ retransmission timer ($RTO$).
\item If timer expires before segment \texttt{ACK}ed, sender must retransmit.
\item Key Design Issue:
  \begin{itemize}[noitemsep]
  \item Find a suitable value of retransmission timer?
  \item Too small: many unnecessary retransmissions, wasting network bandwidth.
  \item Too large: delay in handling lost segment
  \end{itemize}
\item The timer should be longer than the Round-Trip Time ($RTT$).
  The time to send the segment and receive the \texttt{ACK} back.
\item Lastly, the delay ($D$) is variable, complicating things.
\end{itemize}

There are 2 main strategies:
\begin{enumerate}[noitemsep]
\item \nameref{par:Fixed_Packet_Timers}
\item \nameref{par:Adaptive_Packet_Timers}
\end{enumerate}

\paragraph{Fixed Timers}\label{par:Fixed_Packet_Timers}
This is exactly what is sounds like.
The Retransmission Timer, $RTO$, has a fixed value for all packets.
This has problems if the network has particularly high variability in delay values.

\paragraph{Adaptive Timers}\label{par:Adaptive_Packet_Timers}
Before we begin with adaptive timers, we should discuss some problems with them.
\begin{itemize}[noitemsep]
\item Peer TCP entity perform cumulative acknowledgements and not \texttt{ACK} immediately.
\item For retransmitted segments, the sender cannot tell whether a given \texttt{ACK} is sent in response to the original transmission or the retransmission.
  \begin{itemize}[noitemsep]
  \item This is because the original packet and the retransmitted one will have the same sequence number, $SN$.
  \item Since \texttt{ACK}s rely on this number, the acknowledgement number $AN$ will have the same value.
  \end{itemize}
\item Network conditions may change suddenly, causing additional problems.
\item However, an adaptive timer is still better than a fixed timer!
\end{itemize}

For this portion of the course, we will use the \emph{Adaptive Retransmission Timer} protocol.
It is based off the Average Round-Trip Time ($ARTT$), whose formula is shown in \Cref{eq:Average_Round_Trip_Time}.
\begin{equation}\label{eq:Average_Round_Trip_Time}
  \begin{aligned}
    ARTT(k+1) &= \frac{1}{k+1} \sum\limits_{i=1}^{k+1} RTT(i) \\
    &= \frac{k}{k+1} ARTT(k) + \frac{1}{k+1}RTT(k+1) \\
  \end{aligned}
\end{equation}
\begin{description}[noitemsep]
\item $k$: The packet ``time'' we are interested in. This is the sending timepoint that we are calculating for.
\item $RTT$: Round-Trip Time. The amount of time needed for a packet to be sent and the \texttt{ACK} to be received.
\item $ARTT$: Average Round-Trip Time. The average of $RTT$ up until this point.
\end{description}

In \Cref{eq:Average_Round_Trip_Time},
\begin{itemize}[noitemsep]
\item Each term is given the same weight ($\frac{1}{k+1}$).
\item This may \textbf{not} adjust properly to recent changes, as this will more likely reflect future behavior.
\item We need to give more weight to more recent values, as they are more likely to predict the network's behavior better.
\end{itemize}

To combat the issues presented by the Average Round-Trip Time ($ARTT$), the \emph{Smoothed Round-Trip Time} ($SRTT$) was created, its formula is shown in \Cref{eq:Smoothed_Round_Trip_Time}.
\begin{equation}\label{eq:Smoothed_Round_Trip_Time}
  SRTT(k+1) = \bigl( \alpha \times SRTT(k) \bigr) + \bigl( (1-\alpha) \times RTT(k+1) \bigr)
\end{equation}
\begin{description}[noitemsep]
\item $k$: The packet ``time'' we are interested in. This is the sending timepoint that we are calculating for.
\item $\alpha$: The weight that we want to attach to more recent events, and disregard that we want to apply to older events.
  \begin{itemize}[noitemsep]
  \item The older the observation, the less it is counted in the Smoothed Round-Trip Time.
  \item $0 < \alpha < 1$
  \item The closer $\alpha$ is to $0$, the more time points that we consider.
  \item The closer $\alpha$ is to $1$, the more we rely \textbf{only} on recent transmissions.
  \end{itemize}
\item $RTT$: Round-Trip Time. The amount of time needed for a packet to be sent and the \texttt{ACK} to be received.
\item $SRTT$: Smoothed Round-Trip Time.
\end{description}

To effectively use the current $SRTT$, the $RTO$ should be set slightly greater than the current $SRTT$.
This is shown in \Cref{eq:RTO_Based_SRTT}.
\begin{equation}\label{eq:RTO_Based_SRTT}
  RTO(k+1) = \min \Bigl( UB, \max \bigl( LB, \beta \times SRTT(k+1) \bigr) \Bigr)
\end{equation}
\begin{description}[noitemsep]
\item $LB$ Lower Bound. Pre-chosen constant that is fixed.
\item $UB$ Upper Bound. Pre-chosen constant that is fixed.
\item $\beta$ Scaling factor used to determine how important the current $SRTT$ is in the choice of $RTO$.
\end{description}

%%% Local Variables:
%%% mode: latex
%%% TeX-master: "../../ETSN10-Network_Architecture_Performance-Reference_Sheet"
%%% End:
