\subsection{The Poisson Process}\label{subsec:Poisson_Process}
\begin{definition}[The Poisson Process]\label{def:Poisson_Process}
  This is a continuous-time, discrete-value \nameref{def:Stochastic_Process}.
  It is also a \nameref{def:Stationary_Process}.
  This process in \textbf{memoryless}, the previous time instant's values do not affect this time instant's value.

  This is very commonly used to describe the arrivals into a queueing system or network.
  There is a parameter $\lambda$ that is the average rate (packets per second, in this case).

  There are 3 different definitions for this process:
  \begin{enumerate}[noitemsep]
  \item Behaviour for a very small interval of time.
    \begin{itemize}[noitemsep]
    \item Approximately a \nameref{def:Bernoulli_Random_Variable}.
    \item The time interval is small enough such that only 1 event occurs with probability $\lambda \Delta t + o(\Delta t)$, which becomes $\lambda \Delta t$ for small $\Delta t$.
    \item 0 events occur with probability $1 -\lambda \Delta t + o(\Delta t)$, which becomes $1 - \lambda \Delta t$ for small $\Delta t$.
    \item Probabilities between non-overlapping intervals are independent
    \end{itemize}

  \item Behaviour over a longer period of time.
    \begin{itemize}[noitemsep]
    \item Receive multiple events, $k$.
    \item Probability of $k$ events is $p_{X}(k) = e^{-\lambda} \frac{\lambda^{k}}{k!}$
    \end{itemize}

  \item Behaviour between events.
    \begin{itemize}[noitemsep]
    \item Approximately a \nameref{def:Negative_Exponential_Random_Variable}.
    \item Time $t$ between events, $p(t) = \lambda e^{-\lambda t}$
    \end{itemize}
  \end{enumerate}
\end{definition}

\begin{proof}[Poisson Process Definition 1/2 Equivalence]\label{proof:Poisson_Process_Defn_1-2}
  \textbf{TODO!}
\end{proof}

\begin{proof}[Poisson Process Definition 2/3 Equivalence]\label{proof:Poisson_Process_Defn_2-3}
  \textbf{TODO!}
\end{proof}

\begin{proof}[Poisson Process Definition 3/1 Equivalence]\label{proof:Poisson_Process_Defn_3-1}
  \textbf{TODO!}
\end{proof}

\subsubsection{Sums of the Poisson Processes}\label{subsubsec:Sums_Poisson_Process}
The superposition of several different Poisson processes.
\begin{itemize}[noitemsep]
\item There are $m$ independent Poisson Processes, with rates $\lambda_{i}$ for $i = 1, 2, \ldots, m$
\item Sum is also a Poisson Process: $\lambda = \sum\limits_{i=1}^{m} \lambda_{i}$
\end{itemize}

%%% Local Variables:
%%% mode: latex
%%% TeX-master: "../ETSN10-Network_Architecture_Performance-Reference_Sheet"
%%% End:
