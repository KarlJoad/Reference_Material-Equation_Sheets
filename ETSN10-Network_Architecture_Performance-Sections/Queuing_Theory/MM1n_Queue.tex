\subsection{\texorpdfstring{The Finite Buffer $M/M/1/N$ Queue}{Finite Buffer Queue}}\label{subsec:MM1n_Queue}
The $M/M/1/n$ is a queue with a finite-sized buffer.
The real question with this is what is the probability the buffer will be full and the next packet will have to be blocked.

\begin{definition}[Blocking Probability]\label{def:Blocking_Probability-Packets}
  The \emph{blocking probability} is the probability that the next packet that arrives to the system will be blocked because the queue is full.
  Instead of making the queue longer to keep track of the packets, the system just denies the packet access to the queue.

  \begin{equation}\label{eq:Blocking_Probability-Packets}
    P_{B} = \Prob[\text{Message is blocked}] = p_{n} = \frac{1-\rho}{1-\rho^{n+1}} \rho^{n}
  \end{equation}
\end{definition}

\begin{example}[Lecture 4]{Length of Queue for No Packets Blocked}
  Assume $\rho = 0.5$, and that we want a system that blocks packets with probability $P_{B} \leq 10^{-3}$.
  Find $n$, the minimum length of the queue to satisfy these requirements?
  \tcblower{}
  Using \Cref{eq:Blocking_Probability-Packets},
  \begin{align*}
    \frac{1-\rho}{1-\rho^{n+1}} \rho^{n} &= \frac{1-0.5}{1-0.5^{n+1}} 0.5^{n} \\
                                         &= \frac{0.5^{n+1}}{1-0.5^{n+1}} \leq 10^{-3} \\
    0.5^{n+1} &\leq \frac{10^{-3}}{1 + 10^{-3}} \\
    n + 1 &\geq 9.96 \\
    n &\geq 8.96
  \end{align*}

  So, $n$ should be 9.
\end{example}

\begin{remark*}
  Note that the solution found in \Cref{ex:Length of Queue for No Packets Blocked} is actually the general solution.
  Meaning, that solution for $n$ is valid for all $\rho$.
  This is because the finite buffer copes with overloading by blocking any additional messages from entering the system.
\end{remark*}

\subsubsection{Applying \nameref*{def:Littles_Law}}\label{subsubsec:Applying_Littles_Law-MM1n}
To even use \nameref{def:Littles_Law}, we must account for the blocking of messages by only counting the messages that make it into the system.

\begin{equation}\label{eq:Littles_Law-MM1n}
  \gamma = \lambda (1-P_{B})
\end{equation}

This means our equations change a little bit.
\begin{equation}\label{eq:Occupancy-MM1n}
  \rho = \frac{\lambda}{\mu} \text{, can be $\geq$ 1 r $<1$}
\end{equation}


\begin{subequations}\label{eq:Mean_Packets_in_Whole_System-MM1n}
  \begin{equation}
    \begin{aligned}
      \ExpectedValue[R] &= \gamma \ExpectedValue[T_{R}] \\
      &= \lambda (1-P_{B}) \ExpectedValue[T_{R}] \\
    \end{aligned}
  \end{equation}

  \begin{equation}
    \begin{aligned}
      \ExpectedValue[R] &= \sum\limits_{i=0}^{n} i p_{i} \\
      &= \frac{1-\rho}{1-\rho^{n+1}} \sum\limits_{i=0}^{n}i \rho_{i} \\
    \end{aligned}
  \end{equation}
\end{subequations}

%%% Local Variables:
%%% mode: latex
%%% TeX-master: "../../ETSN10-Network_Architecture_Performance-Reference_Sheet"
%%% End:
