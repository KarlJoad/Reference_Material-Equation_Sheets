\subsection{Queuing Networks}\label{subsec:Queuing_Networks}
\begin{definition}[Queuing Network]\label{def:Queuing_Network}
  A \emph{queuing network} is a network of nodes in which each node is a queue.
  The output of one queue is connected to the input of another node's queue.

  There are 2 types of queuing networks:
  \begin{enumerate}[noitemsep]
  \item \nameref{def:Closed_Queuing_Network}
  \item \nameref{def:Open_Queuing_Network}
  \end{enumerate}

  \begin{remark}[Limiting Analysis]\label{rmk:Limiting_Queuing_Networks}
    We will only consider $M/M/n$ queues here.
    Analysis of more general queueing networks gets complicated fast.
  \end{remark}
\end{definition}

\begin{definition}[Closed Queuing Network]\label{def:Closed_Queuing_Network}
  A \emph{closed queuing network} is one in which packets (customers, tasks, etc.) can never enter or leave the \nameref{def:Queuing_Network}.
\end{definition}

\begin{definition}[Open Queuing Network]\label{def:Open_Queuing_Network}
  An \emph{open queuing network} is a \nameref{def:Queuing_Network} that is not closed.
  This means that an open queuing network is one in which packets (customers, tasks, etc.) may enter a node's queue from outside the \nameref{def:Queuing_Network}, and after processing, may leave the \nameref{def:Queuing_Network} entirely.
\end{definition}

\begin{theorem}[Burke's Theorem]\label{thm:Burkes_Theorem}
  The interdeparture times from a Markovian ($M/M/n$) queue are exponentially distributed with the same parameter as the interarrival times.
  Meaning
  \begin{equation}\label{Burkes_Theorem}
    \lambda_{\mathrm{Out}} = \lambda_{\mathrm{In}}
  \end{equation}

  \begin{remark}
    This allows us to analyze each queue/node independently.
    We do not need to consider all the nodes at the same time.
  \end{remark}
\end{theorem}

\subsubsection{Jackson Networks}\label{subsubsec:Jackson_Networks}
\begin{definition}[Jackson Network]\label{def:Jackson_Network}
  A \emph{Jackson network} is an \nameref{def:Open_Queuing_Network} of $M/M/n$ queues.
  Each node can receive traffic from other nodes and from outside the network.
  Traffic from each node can go to other nodes, or leave the network entirely.
\end{definition}

\begin{itemize}[noitemsep]
\item $N$: The total number of nodes in a \nameref{def:Jackson_Network}.
\item $\lambda_{i}$: Total incoming average traffic rate to node $i$.
\item $gamma_{i}$: Average rate of traffic entering node $i$ from outside the network
\item $r_{ij}$: Probability a packet leaving node $i$ will then go to node $j$
  \begin{itemize}[noitemsep]
  \item Note r need not be 0 – a packet may immediately return to the node it just left
  \end{itemize}
\item So the probability that a packet leaves the network after leaving node $i$ is given by
  \begin{equation}\label{eq:Prob_Packet_LEaves_Network}
    1 - \sum\limits_{j=1}^{N} r_{ij}
  \end{equation}

\item You can find the total incoming traffic to node $i$ by summing all incoming streams together.
  \begin{equation}\label{eq:Total_Incoming-Traffic}
    \lambda_{i} = \gamma_{i} + \sum\limits_{j=1}^{N} \lambda_{i} r_{ji} \text{, for } i = 1, 2, \ldots, N
  \end{equation}

\item In general, the whole network will not behave like a \nameref{def:Poisson_Process}.
\item Jackson, the namesake of the \nameref{def:Jackson_Network}, showed that each individual node behaves as though it were.
\item The state variable for the entire system of $N$ nodes consists of the vector
  \begin{equation}\label{eq:Queuing_Network_State_Vector}
    \langle k_{1}, k_{2}, \ldots, k_{N} \rangle
  \end{equation}
  where $k_{i}$ is the number of packets (including those currently in service) at the $i$th node.

\item Let the equilibrium probability associated with a given state be denoted
  \begin{equation}\label{eq:Queuing_Network_Equilibrium_Probability}
    \Prob \bigl[ \langle k_{1}, k_{2}, \ldots, k_{N} \rangle \bigr]
  \end{equation}
  and the probability of finding $k_{i}$ customers in the $i$th node be
  \begin{equation*}
    \Prob_{i}[k_{i}]
  \end{equation*}

\item The joint probability distribution for all nodes is the product of the distributions for the individual nodes
  \begin{equation*}
    \Prob \bigl[ \langle k_{1}, k_{2}, \ldots, k_{N} \rangle \bigr] = \Prob_{1}[k_{1}] \Prob_{2}[k_{2}] \cdots \Prob_{N}[k_{N}]
  \end{equation*}
  and each $\Prob_{i}[k_{i}]$ is given by the solution to the classical $M/M/n$ system.
\item So we can treat each node as an $M/M/n$ queue, even though the input is not necessarily Markovian.
\end{itemize}

%%% Local Variables:
%%% mode: latex
%%% TeX-master: "../../ETSN10-Network_Architecture_Performance-Reference_Sheet"
%%% End:
