\subsection{Modeling Circuit Switching}\label{subsec:Model_Circuit_Switching}
This is easiest to visualize with the old-school telephone communications with human operators physically connecting calls to make a single large continuous circuit.

\begin{itemize}[noitemsep]
\item The rate of call initiation can be modeled as a \nameref{def:Poisson_Random_Variable}, with an arrival rate of $\lambda$.
\item The length of the calls can be modeled as a \nameref{def:Negative_Exponential_Random_Variable}, denoted as $h$, $\mu = \dfrac{1}{h}$.
\item The number of circuits is fixed at $n$.
\item If all circuits are in use, there is no queue to wait in until one is available.
\end{itemize}

This makes \nameref{def:Circuit_Switching} a $M/M/n/n$ \nameref{def:Queuing_System} in \nameref{def:Kendalls_Notation}.

\begin{definition}[Offered Traffic]\label{def:Offered_Traffic}
  \emph{Offered traffic} is the amount of traffic that the end-users are offering the network.

  \begin{equation}\label{eq:Offered_Traffic}
    A = \lambda h
  \end{equation}
\end{definition}

\begin{definition}[Carried Traffic]\label{def:Carried_Traffic}
  \emph{Carried traffic} is the amount of traffic that the network is currently handling.

  \begin{equation}\label{eq:Carried_Traffic}
    A_{C} = A (1-P_{B})
  \end{equation}
\end{definition}

\begin{definition}[Lost Traffic]\label{def:Lost_Traffic}
  \emph{Lost traffic} is traffic that is lost because all the resources available have been allocated and no more are available.
  This cannot be regained later because there is no queue available to handle the things that arrive to the system after all resources have been allocated.

  \begin{equation}\label{eq:Lost_Traffic}
    L = A - A_{C}
  \end{equation}

  \begin{remark}[Validity]\label{rmk:Validity_Lost_Traffic}
    \Cref{eq:Lost_Traffic} is valid for all non-negative exponential distributions.
    However, it is invalidated if repeats occur at all, or if they are allowed to occur, occur to soon.
  \end{remark}
\end{definition}

Let the \nameref{def:Random_Variable} $R$ represent the number of customers currently in the system, meaning $R$ is between $0$ and $n$.
The service rate for this is now
\begin{equation*}
  i \mu \text{, when } R = i
\end{equation*}

\begin{definition}[Blocking Probability]\label{def:Blocking_Probability-Circuit}
  The \emph{blocking probability} is the probability that the next thing that comes to the system will be blocked because all resources have already been allocated and there is no queue.

  \begin{equation}\label{eq:Blocking_Probability-Circuit}
    \begin{aligned}
      P_{B} &= p_{n} \\
      &= \frac{\frac{A^{n}}{n!}}{\sum\limits_{i=0}^{n} \frac{A^{j}}{j!}}
    \end{aligned}
  \end{equation}
\end{definition}

\begin{definition}[Time Congestion]\label{def:Time_Congestion}
  \emph{Time congestion} represents the proportion of time that all the circuits are busy.
  This is only viewed from the system-side, so the end-user will never know about time congestion.

  To find the time congestion, it is simply the expected value of the \nameref{def:Offered_Traffic}, when in state $n$.

  \begin{equation}\label{eq:Time_Congestion}
    \begin{aligned}
      \ExpectedValue_{n}[A] &= P_{B} \\
      &= p_{n} \\
      &= \frac{\frac{A^{N}}{N!}}{\sum\limits_{j=0}^{n}\frac{A^{j}}{j!}}
    \end{aligned}
  \end{equation}
\end{definition}

\begin{definition}[Call Congestion]\label{def:Call_Congestion}
  \emph{Call congestion} is the proportion of calls that find the system busy.
  This is viewed from the end-user side.
\end{definition}

\begin{remark*}
  For arrivals that follow a \nameref{def:Poisson_Random_Variable}'s distribution, \nameref{def:Time_Congestion} = \nameref{def:Call_Congestion}.
  This follows according to the \nameref{thm:PASTA_Theorem}.
\end{remark*}

\begin{theorem}[PASTA Theorem]\label{thm:PASTA_Theorem}
  
\end{theorem}

%%% Local Variables:
%%% mode: latex
%%% TeX-master: "../../ETSN10-Network_Architecture_Performance-Reference_Sheet"
%%% End:
