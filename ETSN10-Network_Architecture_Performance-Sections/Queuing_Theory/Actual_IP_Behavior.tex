\subsection{Actual IP Network Behavior}\label{subsec:Actual_IP_Behavior}
Actual IP traffic doesn't actually follow a \nameref{def:Poisson_Random_Variable}'s distribution very well.
It has:
\begin{itemize}[noitemsep]
\item Self-similarity
\item Lots of dependency (Current traffic depends on the previous traffic).
\item Infinite variance.
\end{itemize}

All of this means that all the statistical properties (mean, variance, standard deviation, etc.) of a self-similar process are the same at all timescales.
Meaning that considering a larger timescale does not smooth out variations in the data.

\begin{definition}[Autocorrelation]\label{def:Autocorrelation}
  \emph{Autocorrelation} is a function that measures the similarity of a function with itself, typically over a period of time.

  \begin{equation}\label{eq:Autocorrelation}
    R_{x, x}(t_{1}, t_{2}) = \ExpectedValue[X(t_{1}), X(t_{2})]
  \end{equation}
\end{definition}

\begin{definition}[Long-Range Dependence]\label{def:Long_Range_Dependence}
  A process that has \emph{long-range dependence} has an autocorrelation that decays slowly as the difference between $t_{1}$ and $t_{2}$ tends towards $\infty$.
  Mathematically, these processes are characterized by an atocorrelation function that decays hyperbolically as $k$ increases.
  This is represented with \Cref{eq:Non_Summability_Correlation}, the \emph{Non-summability of correlation}.

  \begin{equation}\label{eq:Non_Summability_Correlation}
    \sum\limits_{k} r(k) = \infty
  \end{equation}

  \begin{remark}[Uncorrelated]
    If the result from \Cref{eq:Autocorrelation} is 0 when $t_{2} = t_{1} + \Delta t$, i.e.\ when the time difference is very small, then the function $X(t)$ is no correlation.
  \end{remark}

  \begin{remark}[Short-Range Correlation]
    if the result from \Cref{eq:Autocorrelation} decays to 0 very quickly, then the function $X(t)$ has \emph{short-range correlation}.
  \end{remark}
\end{definition}

\subsubsection{The Hurst Parameter}\label{subsubsec:The_Hurst_Parameter}
\begin{definition}[Hurst Parameter]\label{def:Hurst_Parameter}
  The \emph{Hurst parameter}, denoted $H$, is a measure of the ``burstiness'' of a system.
  It is also considered a measure of self-similarity.

  $H$ always lies within the range $0.5 < H < 1.0$.

  The expected value of the Hurst parameter is
  \begin{equation}\label{eq:Hurst_Parameter_Expected_Value}
    \ExpectedValue[x(t)] = \frac{\ExpectedValue[x(at)]}{a^{H}}
  \end{equation}

  The variance of the Hurst parameter is
  \begin{equation}\label{eq:Hurst_Parameter_Variance}
    \Variance[x(t)] = \frac{\Variance[x(at)]}{a^{2H}}
  \end{equation}

  The autocorrelation of the Hurst parameter is
  \begin{equation}\label{eq:Hurst_Parameter_Autocorrelation}
    R_{x, x}(t, s) = \frac{R_{x, x}(at, as)}{a^{2H}}
  \end{equation}
\end{definition}

\subsubsection{Self-Similarity and Autocorrelation}\label{subsubsec:Self_Similarity_Autocorrelation}
For this section, let $X(t)$ be a \nameref{def:Stochastic_Process} with
\begin{itemize}[noitemsep]
\item Constant mean $\mu$
\item Constant variance $\sigma^{2}$
\item An autocorrelation function depending only on $k$.
  \begin{equation*}
    r(k) = \frac{\ExpectedValue[(X(t)-\mu)(X(t+k) - \mu)]}{\ExpectedValue \left[ {\bigl( X(t) - \mu \bigr)}^{2} \right]}
  \end{equation*}
\end{itemize}

Now let $X^{(m)}$ be a new \textbf{aggregate} time series formed by averages of $X$ over non-overlapping blocks in time of size $m$.
This has some intersting implications.
\begin{itemize}[noitemsep]
\item $X$ is exactly self-similar if
  \begin{itemize}[noitemsep]
  \item The aggregated processes have the same autocorrelation structure as $X$.
  \item $r^{(m)}(k) = r(k), k \geq 0$ for all $m = 1, 2, \ldots$
  \end{itemize}

\item $X$ is asymptotically self-similar if the above holds when $r^{(m)}(k) \sim r(k), m \rightarrow \infty$
\item The most striking feature of self-similarity is correlation structures of the aggregated process do not degenerate as $m \rightarrow \infty$
\item This is in contrast to traditional models
  \begin{itemize}[noitemsep]
  \item Correlation structures of their aggregated processes degenerate, i.e.\ $r^{(m)}(k) \rightarrow 0$ as $m \rightarrow \infty, k = 1, 2, 3, \ldots$
  \end{itemize}
\end{itemize}

\subsubsection{Squared Coefficient of Variation}\label{subsubsec:Squared_Coefficient_Variation}
This is a critical factor in determining the type of \nameref{def:Queuing_System} to use.
It is a measure of variability of the system.

\begin{equation}\label{eq:Squared_Coefficient_Variation}
  c^{2} = \frac{\sigma^{2}}{\mu^{2}}
\end{equation}

The squared coefficient of variation for some common \nameref{def:Queuing_System}s are shown below.
\begin{itemize}[noitemsep]
\item $M/M/1 = 1$
\item $M/D/1 = 0$
  \begin{itemize}[noitemsep]
  \item Namely, for any system with a deterministic distribution, the squared coefficient of variation will be 0.
  \end{itemize}
\end{itemize}

%%% Local Variables:
%%% mode: latex
%%% TeX-master: "../../ETSN10-Network_Architecture_Performance-Reference_Sheet"
%%% End:
