\subsection{State-Dependent Queues}\label{subsec:State_Dependent_Queues}
\begin{definition}[State-Dependent Queue]\label{def:State_Dependent_Queue}
  A \emph{state-dependent queue} is a queue that behaves in certain ways depending on the state that it is in.
  It can cycle endlessly \textbf{between} states, but can never cycle on itself.

  There are 2 variables needed for each state.
  \begin{enumerate}[noitemsep]
  \item $\lambda_{i}$: The arrival rate of tasks when the system is in state $i$.
  \item $\mu_{i}$: The servers' service rate when the system is in state $i$.
  \end{enumerate}

  In each state, there needs to be a balance between the packets going to the next stage and leaving the current one and vice versa.
  \begin{equation}\label{eq:State_Dependent_Queue_Balance_Equation}
    \lambda_{i-1} p_{i-1} = \mu_{i} p_{i} \text{, for } i = 1, 2, \ldots
  \end{equation}

  The general solution to \Cref{eq:State_Dependent_Queue_Balance_Equation} is:
  \begin{equation}\label{eq:eq:State_Dependent_Queue_Balance_Equation-Solution}
    p_{i} = \frac{\lambda_{0} \lambda_{1} \cdots \lambda_{i-1}}{\mu_{1} \mu_{2} \cdots \mu_{i}} p_{0} \text{, for } i = 1, 2, \ldots
  \end{equation}
\end{definition}

\subsubsection{\texorpdfstring{A General $M/M/2$ Queue}{Parallel Servers}}\label{subsubsec:General_MM2_Queue}
A general $M/M/2$ \nameref{def:Queuing_System} can be represented as a \nameref{def:State_Dependent_Queue}.

\begin{align*}
  \lambda_{i} &= \lambda \text{, for } i = 1, 2, \ldots \\
  \mu_{i} &= \begin{cases}
    \mu & i = 1 \\
    2 \mu & i = 2, 3, \ldots \\
  \end{cases}
\end{align*}
where $i$ is the number of packets in the system, which represents a different state of the system.

Now, let's compare a $M/M/2$ \nameref{def:Queuing_System} against a $M/M/1$ \nameref{def:Queuing_System}.
\begin{align*}
  {\ExpectedValue[T_{R}]}_{\text{1 server, } 2\mu \text{ rate}} &= \frac{1}{2 \mu} \frac{1}{1-\rho} \\
  {\ExpectedValue[T_{R}]}_{\text{2 servers}} &= \frac{1}{\mu} \frac{1}{(1+\rho) (1-\rho)} \\
  \frac{{\ExpectedValue[T_{R}]}_{\text{1 server, } 2\mu \text{ rate}}}{{\ExpectedValue[T_{R}]}_{\text{2 servers}}} &= \frac{1+\rho}{2} \leq 2
\end{align*}

The conclusion here is concrete.
\begin{center}
  \textbf{\Large A single faster server is always more efficient.}
\end{center}

If we extend our findings about \nameref{def:State_Dependent_Queue}s to $M/M/\infty$ systems, the general solution becomes a \nameref{def:Poisson_Random_Variable}'s distribution.
\begin{equation}\label{eq:MMinfty_Balance_Equation-Solution}
  \begin{aligned}
    p_{i} &= \frac{\lambda_{i}}{\mu^{i} i!} p_{0} \\
    &= e^{-\frac{\lambda}{\mu}} \frac{{\left( \frac{\lambda}{\mu} \right)}^{i}}{i!} \\
    &= e^{-A} \frac{A^{i}}{i!}
  \end{aligned}
\end{equation}

%%% Local Variables:
%%% mode: latex
%%% TeX-master: "../../ETSN10-Network_Architecture_Performance-Reference_Sheet"
%%% End:
