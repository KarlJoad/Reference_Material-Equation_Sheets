\subsection{Network Layer}\label{subsec:Network_Layer}
\begin{definition}[Network Layer]\label{def:Network_Layer}
  The \emph{network layer} is the third layer in the \nameref{def:Networking_Stack}.
  This is the last layer that every device must have.
  It is responsible for end-to-end data delivery between two devices (hosts) anywhere on the network.
  It is responsible for routing and forwarding.

  Some examples of this layer are:
  \begin{itemize}[noitemsep]
  \item Internet Protocol (IP)
  \item Internet Control Message Protocol (ICMP)
  \item \nameref{def:Routing_Protocol}s
  \end{itemize}
\end{definition}

\begin{definition}[Routing]\label{def:Routing}
  \emph{Routing} is the process of determining and selecting the best end-to-end path through a network.

  There are algorithms designed to discover the other participants in the network by analyzing the grpah that is created when each participant is considered a node in this graph.
\end{definition}

\begin{definition}[Forwarding]\label{def:Forwarding}
  \emph{Forwarding} is the process of selecting the next hop for the given packet.
\end{definition}

\begin{definition}[Circuit Switching]\label{def:Circuit_Switching}
  In \emph{circuit switching}, dedicated resources are allocated end-to-end for a particular flow of data.
  No other flows may use those resources while the connection is maintained.
  Meaning, that during a connection, there is a single dedicated circuit between the host and the receiver.
\end{definition}

\begin{definition}[Packet Switching]\label{def:Packet_Switching}
  In \emph{packet switching}, no resources are allocated to a flow, and each flow’s data is broken up into discrete packets.
  Each \nameref{def:Packet} is routed and forwarded independently.
\end{definition}

\begin{definition}[Packet]\label{def:Packet}
  A connection from one device to another is a stream.
  However, to effectively implement many things in the \nameref{def:Networking_Stack}, the stream must be broken into several discrete \emph{packets}.
\end{definition}

\subsubsection{Discovering Routing Information}\label{subsubsec:Discovering_Route_Info}
\paragraph{Dijkstra's Algorithm}\label{par:Dijkstras_Algorithm}
\nameref{algo:Dijkstras_Algorithm} starts at the source and builds up the shortest path step-by-step.

There are 2 main problems with \nameref{algo:Dijkstras_Algorithm}:
\begin{enumerate}[noitemsep]
\item The weights along the edges of the graph cannot be negative.
  \begin{itemize}[noitemsep]
  \item If this were to occur, then we would have an infinite cycle going through the negative edge.
  \end{itemize}
\item We \textbf{must} know the entire network's topology before beginning the algorithm.
  \begin{itemize}[noitemsep]
  \item Typically, this is not possible on vast networks, like the Internet.
  \end{itemize}
\end{enumerate}

\begin{algorithm}[H]
  \DontPrintSemicolon{}
  \SetKwInOut{Input}{Input}\SetKwInOut{Output}{Output}
  \SetKw{OR}{\textbf{OR}}
  \Input{A graph with weights on each edge representing ``distance'' between the nodes.}
  \Output{The path through the graph with the least weight from some starting node.}
  \BlankLine{}

   Assign tentative distance estimate to each node: 0 for the initial node, $\infty$ for all other nodes. \;
   Mark all nodes except the initial node as unvisited. \;
   Set the initial node as current node. \;
   \For{The current node's unvisited neighbours}{
     Calculate their distance. \;
     Compare this new distance against the current distance. \;
     Take whichever distance is smaller and use that as the new value. \;
     Mark the current node as visited. \;
     \If{Reached target node \OR{} No more unvisited nodes with distance $< \infty$}{
       Stop and terminate the algorithm. \;
     }{
       Set the unvisited node with the smallest distance as the new current node. \;
     }
   }
  \caption{Dijkstra's Algorithm}
  \label{algo:Dijkstras_Algorithm}
\end{algorithm}

\paragraph{Bellman-Ford Algorithm}\label{par:Bellman_Ford_Algorithm}
The Bellman-Ford algorithm works with graphs that have negative weight values on edges.
It also \textbf{does not} require us to know the entire network's topology before beginning.
However, it is much slower than \nameref{algo:Dijkstras_Algorithm}.

\begin{algorithm}[H]
  \DontPrintSemicolon{}
  \SetKwInOut{Input}{Input}\SetKwInOut{Output}{Output}
  \SetKw{AND}{\textbf{AND}}
  \Input{A graph with weights on each edge representing ``distance'' between the nodes.}
  \Output{The path through the graph with the least weight from some starting node.}
  \BlankLine{}

  Set the distance for the source node to 0 and for all other nodes to $\infty$. \;
  Set the predecessor of all nodes to null. \;
  \For{$n \in N$ where $N$ is the number of nodes \AND{} $n \leq N-1$}{
    We repeat $N-1$ times, because $N-1$ is the maximum length of a non-cyclic path. \;
    \For{Each edge $(u, v)$}{
      \If{distance to $u$, plus the edge weight $<$ current distance to $v$}{
        We have discovered a shorter path to $v$. \;
        Set the distance to $v$ to this new value and the predecessor of $v$ to $u$. \;
      }
    }
  }
  
  \For{each edge $(u, v)$}{
    \If{distance to $u$ plus the edge weight $<$ the distance to $v$}{
      The graph contains a negative-weight cycle. \;
      Terminate. \;
    }{
      We have found the shortest paths to each node from the source. \;
      Terminate. \;
    }
  }
  \caption{Bellman-Ford Algorithm}
  \label{algo:Bellman_Ford_Algorithm}
\end{algorithm}

\subsubsection{Routing Information}\label{subsubsec:Routing_Info}
\begin{definition}[Routing Protocol]\label{def:Routing_Protocol}
  A \emph{routing protocol} is used to construct a \nameref{def:Routing_Table} in each node.
\end{definition}

\begin{definition}[Routing Table]\label{def:Routing_Table}
  The \emph{routing table} is a table that contains entries on the shortest distances through the current network graph.
  This is used to help determine where to next forward packets onto.

  Each node has a routing table that contains the cost to the destination and the next hop for each destination.
  For a valid routing table, we need to have:
  \begin{enumerate}[noitemsep]
  \item Which destination the packets are going to?
  \item What is the next hop after the destination?
  \item The ``metric'' or weight of that particular path through the network graph?
    \begin{itemize}[noitemsep]
    \item We need this term so we can update the routing table as we discover new routes.
    \item This way if a new route with a lower cost presents itself, we can replace the higher-cost one with the lower one.
    \end{itemize}
  \end{enumerate}
\end{definition}

There are 2 ways to collect the routing information necessary to properly route packets through a network.
These 2 routing tables are:
\begin{enumerate}[noitemsep]
\item \nameref{par:Static_Routing_Tables}
\item \nameref{par:Dynamic_Routing_Tables}
\end{enumerate}

\paragraph{Static Routing Tables}\label{par:Static_Routing_Tables}
\begin{definition}[Static Routing Table]\label{def:Static_Routing_Table}
  \emph{Static routing table}s are manually configured by a user or a program before the system begins routing the information.
\end{definition}

This means that there is no overhead to attempt to figure out the network's topology.
The information needed has already been given to the device's \nameref{def:Routing_Table}, so that step is completed.

\paragraph{Dynamic Routing Tables}\label{par:Dynamic_Routing_Tables}
\begin{definition}[Dynamic Routing Table]\label{def:Dynamic_Routing_Table}
  A \emph{dynamic routing table} is a \nameref{def:Routing_Table} that is built automatically by the device.
  To do this, a \nameref{def:Routing_Protocol} is used to build the table while the device is running.
  This is done using either:
  \begin{enumerate}[noitemsep]
  \item \nameref{def:Distance_Vector_Routing_Protocol}
  \item \nameref{def:Link_State_Routing_Protocol}
  \end{enumerate}
\end{definition}

\begin{definition}[Distance Vector Routing Protocol]\label{def:Distance_Vector_Routing_Protocol}
  The \emph{distance vector routing algorithm} is a \nameref{def:Routing_Protocol} that uses \nameref{algo:Bellman_Ford_Algorithm} to construct a \nameref{def:Routing_Table}.
  
  Each node only begins with knowledge of their immediate neighbours and the costs to reach them.
  \begin{itemize}[noitemsep]
  \item Nodes then send this information (the routing table) to their neighbours.
  \item If a neighbour sends us a route that is shorter than one we already have, update our table to reflect this.
  \item After updating, send the new table to our neighbours.
  \item If a node goes down, discard any lines in the routing table that have it as the next hop and follow the above to find a new route.
  \end{itemize}
\end{definition}

\begin{definition}[Link State Routing Protocol]\label{def:Link_State_Routing_Protocol}
  The \emph{link state routing protocol} is a \nameref{def:Routing_Protocol} that uses \nameref{algo:Dijkstras_Algorithm} to construct a \nameref{def:Routing_Table}.
  Thus, we need to know what the whole network looks like.
  
  \begin{itemize}[noitemsep]
  \item Each node floods the network with the list of nodes it can connect to and the costs to them.
  \item Every node builds up a picture of the entire network, then can use \nameref{algo:Dijkstras_Algorithm} to determine the shortest path to each destination.
  \item The \nameref{def:Routing_Table} is then constructed based on the computed shortest paths.
  \end{itemize}
\end{definition}

However, the problem with both \nameref{def:Distance_Vector_Routing_Protocol} and \nameref{def:Link_State_Routing_Protocol} is that they do not scale well.
However, to help with this, both protocols are contained within a single \nameref{def:Autonomous_System}.

\begin{definition}[Autonomous System]\label{def:Autonomous_System}
  An \emph{autonomous system} is a smaller network, like a business or home, that performs a \nameref{def:Routing_Protocol} upon itself.
  It is done this way because both the \nameref{def:Distance_Vector_Routing_Protocol} and \nameref{def:Link_State_Routing_Protocol} do not scale to Internet-sized networks well.

  To combat this, each of the autonomous systems has a \nameref{def:Speaker_Node} that speaks to the outside world.
  This speaker node is used in the \nameref{def:Path_Vector_Routing_Protocol}.
\end{definition}

\begin{definition}[Speaker Node]\label{def:Speaker_Node}
  A \emph{speaker node} is a specially designated node within a single \nameref{def:Autonomous_System} that communicates \textbf{only with other speaker nodes}.
  Then, the \nameref{def:Path_Vector_Routing_Protocol} is performed upon all the speaker nodes to construct a wider network graph consisting only of the \nameref{def:Autonomous_System}s that the speaker nodes belong to.
\end{definition}

\begin{definition}[Path Vector Routing Protocol]\label{def:Path_Vector_Routing_Protocol}
  Between \nameref{def:Autonomous_System}s, we use a variant of \nameref{def:Distance_Vector_Routing_Protocol} called \emph{path vector routing protocol}.
  Each \nameref{def:Autonomous_System} has its own \nameref{def:Speaker_Node}.
  Only the \nameref{def:Speaker_Node}s can communicate across the \nameref{def:Autonomous_System} boundary, and exchange information about which destinations they can reach and the paths to them.
\end{definition}

\subsubsection{IP Addresses}\label{subsubsec:IP_Addresses}
\begin{definition}[IP Address]\label{def:IP_Address}
  \emph{IP Address}es are the unique end-point routing identifiers used on \nameref{def:Packet}s to deliver their data.
  A natural analogy is that of apartment numbers on apartment buildings.

  IPv4 uses 32 bits, split up into 4 8-bit chunks.
  Each chunk is read as its decimal equivalent, for humans.
  IPv6 uses 128 bits, where every 16 bits are interpreted as a hexadecimal number.
\end{definition}

\begin{definition}[Subnet]\label{def:Subnet}
  All hosts that are in the subnet will have \nameref{def:IP_Address}es that match the number of bits that are present in the subnet.
  Keeping with the apartment analogy, the subnet is like the address of the apartment building.

  For example, in \texttt{192.168.2.0/24}, the \texttt{24} means the first 24 bits of the subnet are 1's (\texttt{255.255.255.0}).
  So, the first 24 bits, or 3 \nameref{def:IP_Address} blocks (\texttt{192.168.2}), will remain the same for every client in that subnet.
\end{definition}

\begin{definition}[Classless Inter-Domain Routing]\label{def:Classless_Interdomain_Routing}
  \emph{Classless Inter-Domain Routing} or \emph{CIDR} (pronounced ``cider'') is a hierarchical way to organize \nameref{def:IP_Address}es.
  
  This allowed:
  \begin{itemize}[noitemsep]
  \item \nameref{def:Subnet}s to be of any length.
  \item Destinations in a router’s routing and forwarding tables may be full \nameref{def:IP_Address}es or \nameref{def:Subnet}s.
    \begin{itemize}[noitemsep]
    \item A destination \nameref{def:IP_Address} will then be matched to the most specific destination in the table when making forwarding decisions.
    \end{itemize}
  \end{itemize}
\end{definition}

%%% Local Variables:
%%% mode: latex
%%% TeX-master: "../../ETSN10-Network_Architecture_Performance-Reference_Sheet"
%%% End:
