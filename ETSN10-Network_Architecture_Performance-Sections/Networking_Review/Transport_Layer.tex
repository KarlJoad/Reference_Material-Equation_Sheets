\subsection{Transport Layer}\label{subsec:Transport_Layer}
\begin{definition}[Transport Layer]\label{def:Transport_Layer}
  The \emph{transport layer} is built where the \nameref{def:Network_Layer} has delivered all the data to the end host.
  This means that only the source and destination hosts have this layer.
  So, routers and switches do not have this, but your phone, laptop, and the server you're connecting to do.
  
  This layer may be responsible for:
  \begin{itemize}[noitemsep]
  \item \nameref{def:Connection_Oriented_Communication}
  \item Multiplexing different data flows
  \item Reliable data delivery
  \item Data flow control
  \item Data congestion control
  \end{itemize}
  Some implementation of this layer do not handle all of these functions, but all \textbf{must} handle the multiplexing of different data flows.

  There are 2 main transport layers in-use today:
  \begin{enumerate}[noitemsep]
  \item \nameref{def:Transmission_Control_Protocol}
  \item \nameref{def:User_Datagram_Protocol}
  \end{enumerate}
\end{definition}

\begin{definition}[Connection-Oriented Communication]\label{def:Connection_Oriented_Communication}
  A \emph{connection-oriented communication} system means that a connection must be established before any data is transferred.
  In addition, this connection must be maintained throughout the transmission of data.

  \begin{remark}[Connectionless Communication]\label{rmk:Connectionless_Communication}
    In a \emph{connectionless communication} system, hosts can send data at any time without prior connection being made.
  \end{remark}
\end{definition}

\begin{definition}[Transmission Control Protocol]\label{def:Transmission_Control_Protocol}
  \textbf{TODO!}
\end{definition}

\begin{definition}[User Datagram Protocol]\label{def:User_Datagram_Protocol}
  \textbf{TODO!}
\end{definition}

%%% Local Variables:
%%% mode: latex
%%% TeX-master: "../../ETSN10-Network_Architecture_Performance-Reference_Sheet"
%%% End:
