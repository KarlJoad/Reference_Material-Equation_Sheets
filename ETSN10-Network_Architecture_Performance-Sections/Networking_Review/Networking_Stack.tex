\subsection{Networking Stack}\label{subsec:Networking_Stack}
\begin{definition}[Networking Stack]\label{def:Networking_Stack}
  The \emph{networking stack} is a set of layers that are built on top of another that are used to implement inter-computer communication.
  There are 5 layers:
  \begin{enumerate}[noitemsep]
  \item \nameref{def:Physical_Layer}
  \item \nameref{def:Data_Link_Layer}
  \item \nameref{def:Network_Layer}
  \item \nameref{def:Transport_Layer}
  \item \nameref{def:Application_Layer}
  \end{enumerate}

  Each of these fulfill different roles, allowing for different implementations to be used interchangeably.
  Additionally, not all points in the network graph require the entire networking stack.
  For example, a router or switch does not require the \nameref{def:Transport_Layer} or the \nameref{def:Application_Layer}, since the job of this hardware is to just move packets around.
\end{definition}

Each layer of the networking stack add a header and possibly a footer.
Information in headers and footers includes source and destination addresses, checksums, packet size, and protocol identifiers.

\begin{definition}[Encapsulation]\label{def:Encapsulation}
  The process of adding headers and footers is called \emph{encapsulation}.
\end{definition}

\begin{definition}[Decapsulation]\label{def:Decapsulation}
  The process of removing the headers and footers of a \nameref{def:Packet} at the other end is called \emph{decapsulation}.
\end{definition}

%%% Local Variables:
%%% mode: latex
%%% TeX-master: "../../ETSN10-Network_Architecture_Performance-Reference_Sheet"
%%% End:
