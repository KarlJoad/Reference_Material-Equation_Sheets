\subsection{Physical Layer}\label{subsec:Physical_Layer}
\begin{definition}[Physical Layer]\label{def:Physical_Layer}
  The \emph{physical layer} in the \nameref{def:Networking_Stack} is the lowest layer in the stack.
  It consists of the physical connection that is made between computers and the modulation and coding to represent data as a signal on that connection.
  For obvious reasons, all devices \textbf{must} implement this layer.

  Some examples of this layer are:
  \begin{itemize}[noitemsep]
  \item Copper twisted-pair cables
  \item Fiber optic wires
  \item Radio transmission
  \end{itemize}

  \begin{remark}
    The design of different \nameref{def:Physical_Layer}s is more a hardware design question than that of a software implementation issue.
    So, it is not discussed in this course, beyond the use of cellular networks.
  \end{remark}
\end{definition}

%%% Local Variables:
%%% mode: latex
%%% TeX-master: "../../ETSN10-Network_Architecture_Performance-Reference_Sheet"
%%% End:
