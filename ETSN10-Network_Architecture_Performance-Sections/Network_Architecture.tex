\section{Network Architecture}\label{sec:Network_Architecture}
\begin{definition}[Network Architecture]\label{def:Network_Architecture}
  \emph{Network Architecture} is:
  \begin{itemize}[noitemsep]
  \item How the overall system is designed: what components there are and how they interact.
  \item Protocols and technologies for all the layers of the network stack.
  \item Physical infrastructure: base stations, cables, antennas, routers, etc.
  \item How everything works together to meet the overall performance goals for the whole system.
  \end{itemize}
\end{definition}

\subsection{Wireless vs. Wired Networks}\label{subsec:Wireless_vs_Wired}
Wireless and wired (fixed) networks operate under the same underlying network principles.
However, wireless systems have a very different set of challenges to also contend with, because of the nature of wireless radio communication, and it being a broadcast medium.

The biggest problems for wireless networks that wired networks do not have are:
\begin{itemize}[noitemsep]
\item Interference, \Cref{subsec:Wireless_Interference}
\item Mobility, \Cref{subsec:Wireless_Mobility}
\item Wireless channels:
  \begin{itemize}[noitemsep]
  \item Path loss
  \item Shadowing
  \item Fading
  \item These will not be handled in this course.
  \end{itemize}
\end{itemize}

\subsection{Interference in Wireless Networks}\label{subsec:Wireless_Interference}
The biggest difference between wired and wireless networks when it comes to interference is because of the transportation medium.
In a wired system, discrete connections form isolated links between nodes.
The properties of any single link do not directly affect the properties of any other node.

However, in wireless networks, each of the nodes is broadcasting into space.
Any of these nodes' broadcasts can affect any of the other nodes.
You can think of it as if each node has a large sphere extending radially outwards, where multiple nodes can cover the same zone.
Additionally, this may also mean that some nodes are dependent on other nodes.

\subsubsection{Performance in Interfering Networks}\label{subsubsec:Performance_Interfering_Networks}
In the work done by \citeauthor{Wireless_Network_Capacity}, it was found there is a closed-form solution for the performance of interfering wireless networks.

Place $n$ nodes on a unit disk.
Each node as the capacity $W$ bits per second.
Then, the upper bound on the capacity for the network as $n \rightarrow \infty$ is
\begin{equation}\label{eq:Wireless_Network_Capacity_Upper_Bound}
  \frac{W}{\sqrt{n \log(n)}}
\end{equation}

If the nodes are placed optimally, i.e.\ each node is placed such that its transmission range never crosses another node's range.
Then the capacity of the network is given in \Cref{eq:Wireless_Network_Capacity_Optimal}
\begin{equation}\label{eq:Wireless_Network_Capacity_Optimal}
  \frac{W}{\sqrt{n}}
\end{equation}

\subsection{Mobility}\label{subsec:Wireless_Mobility}
\begin{definition}[Mobility]\label{def:Mobility}
  \emph{Mobility} is a unique property of wireless networks.
  Since the nodes are not fixed in place by anything, they are free to roam/move.
  This is a big problem, because nodes can enter, leave, and move freely throughout/through the network.
\end{definition}

There are 3 classifications of mobility.
\begin{enumerate}[noitemsep]
\item Pedestrian: $\approx 1.5 \si{\meter/\second}$
\item Vehicle: $5-33 \si{\meter/second}$
\item High-Speed: $> 70 \si{\meter/\second}$
\end{enumerate}

\nameref{def:Mobility} has several effects on a wireless network system.
\begin{enumerate}[noitemsep]
\item Physical Layer
  \begin{itemize}[noitemsep]
  \item Doppler shift
  \item Constantly changing channel $\rightarrow$ higher bit-error rate
  \end{itemize}

\item Data-Link Layer/MAC
  \begin{itemize}[noitemsep]
  \item Which nodes share “links” (and thus can experience collisions) constantly changes.
  \item \nameref{def:Hidden_Node} and \nameref{def:Exposed_Node} problems!
  \end{itemize}

\item Network Layer
  \begin{itemize}[noitemsep]
  \item Constantly changing topology $\rightarrow$ need to constantly change end-to-end path through the network.
  \end{itemize}

\item Transport/Application Layers
  \begin{itemize}[noitemsep]
  \item Not directly affected.
  \item Delays and errors at lower layers can cause problems though.
  \end{itemize}
\end{enumerate}

\subsection{Licensed vs. Unlicensed Spectrum}\label{subsec:Licensed_vs_Unlicensed}
There are 2 types of frequencies in use by people and companies today.
\begin{enumerate}[noitemsep]
\item \nameref{subsubsec:Licensed_Spectrum}
\item \nameref{subsubsec:Unlicensed_Spectrum}
\end{enumerate}

\subsubsection{Licensed Spectrum}\label{subsubsec:Licensed_Spectrum}
The licensed spectrum requires that users obtain exclusive rights to use that frequency spectrum.
\begin{itemize}[noitemsep]
\item Solution for QoS-sensitive applications
\item Exclusive right to spectrum use
\item Network engineering possible
  \begin{itemize}[noitemsep]
  \item Predictability
  \item Manageability
  \end{itemize}

\item Complex and costly systems are generally built around these.
\item Typically big players take these.
\end{itemize}

\subsubsection{Unlicensed Spectrum}\label{subsubsec:Unlicensed_Spectrum}
The unlicensed spectrum does not require that users obtain exclusive rights, but they may require that only certain types of data can be used in that frequency.
\begin{itemize}[noitemsep]
\item Inherent best effort systems
\item ``Some'' QoS support possible
\item No right to use spectrum
  \begin{itemize}[noitemsep]
  \item Competition
  \item Collaboration
  \end{itemize}

\item Simple and cheap systems
\item For small/medium players
\end{itemize}

\subsection{Data Rate vs. Range}\label{subsec:Data_Rate_vs_Range}
Higher data rates means lower tranmission range.
Higher data rates also means greater amounts of power required to transmit the message.

\subsection{Centralized Infrastructure Networks}\label{subsec:Centralized_Infrastructure_Network}
\begin{definition}[Centralized Infrastructure Network]\label{def:Centralized_Infrastructure_Network}
  \emph{Centralized Infrastructure Networks} have some fixed nodes which have the connection to the wider Internet.
  These gateways are the ones that provide access to the Internet.
  This system works in a hierarchical structure, following a \nameref{def:Star_of_Stars} topology.
\end{definition}

\begin{definition}[Star-of-Stars]\label{def:Star_of_Stars}
  A network with a \emph{star-of-stars} topology is a hierarchical setup, where the leaf nodes in the system are the ones generating data to send.
  The next rows up may generate some data, but mainly serve as aggregators for the lower nodes in the system.
  Eventually, all connections terminate at one head star node.
\end{definition}

\subsubsection{Example Centralized Architectures}\label{subsubsec:Centralized_Architectures}
\begin{enumerate}[noitemsep]
\item 802.11 a/b/g/n/ac
\begin{itemize}[noitemsep]
\item Most common WiFi network type, used in homes, shops, offices, etc.
\item 2.4 or 5 GHz bands, range $\approx 500 \si{\meter}$
\item MAC:\@ 802.11 DCF
\end{itemize}

\item 802.11 ah
  \begin{itemize}[noitemsep]
  \item WiFi for Internet of Things
  \item 800 MHz band, range $\approx 1 \si{\kilo \meter}$
  \item MAC:\@ 802.11 DCF with some modifications:
    \begin{itemize}[noitemsep]
    \item Relay nodes
    \item Target wake times
    \item Contention groups
    \end{itemize}
  \item Aimed at reducing energy usage and providing coverage to more nodes in a single network
\end{itemize}

\item LoRa
  \begin{itemize}[noitemsep]
  \item Low-Power Wide Area Network (LPWAN) for Internet of Things
  \item 800 MHz band, range 10s of km
  \item MAC:\@ ALOHA for the uplink, with receive windows for the downlink
  \end{itemize}
\end{enumerate}

\begin{remark*}
  For the above architectures, standard Internet routing (\nameref{def:Distance_Vector_Routing_Protocol} or \nameref{def:Link_State_Routing_Protocol}) is used.
  \begin{itemize}[noitemsep]
  \item Only one wireless hop, so no specialised routing is needed.
  \end{itemize}
\end{remark*}

\subsection{Internet of Things}\label{subsec:IoT}
\begin{definition}[Internet of Things]\label{def:IoT}
  The \emph{Internet of Things} (\emph{IoT}) is a type of network where there are small, low-power sensors/devices placed throughout an environment to monitor the condition of the system.
  Some of these devices may also cause a change in the environment as they have been programmed to do.
\end{definition}

%%% Local Variables:
%%% mode: latex
%%% TeX-master: "../ETSN10-Network_Architecture_Performance-Reference_Sheet"
%%% End:
