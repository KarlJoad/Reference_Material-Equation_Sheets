\documentclass[10pt,letterpaper,final,twoside,notitlepage]{article}
\usepackage[margin=.5in]{geometry}
\usepackage[utf8]{inputenc}
\usepackage[english]{babel}
\usepackage{amsmath}
\usepackage{amsfonts}
\usepackage{amssymb}
\usepackage{graphicx}

\usepackage{subfigure} % Float environment to place multiple figures in one figure environment
\usepackage{nameref} % \nameref{label} lets you reference things by name
\usepackage{hyperref} % Hyperlinks between references
\usepackage{enumitem} % Provides [noitemsep, nolistsep] for more compact lists

%\graphicspath{{./Drawings/Math_374}} % Uncomment this to use pictures in this document

\author{Karl Hallsby}
\title{Equations}
\begin{document}
\section{Relative Frequency} \label{sec:Relative Frequency}
\begin{itemize}[noitemsep, nolistsep]
	\item $f_k (n) = \frac{N_k (n)}{n}$ $\leftarrow$ {\large \textbf{Relative Frequency}}
	\begin{itemize}[noitemsep, nolistsep]
		\item $k$ is the outcome
		\item $N_k (n)$ is the number of times outcome $k$
	\end{itemize}
	\item $\lim\limits_{n \rightarrow \infty} f_k (n) = p_k$ $\leftarrow$ {\large \textbf{Statistical Regularity}}
	\begin{itemize}[noitemsep, nolistsep]
		\item $p_k$ is the probability of event $k$ occurring
	\end{itemize}
\end{itemize}

	\subsection{Properties of Relative Frequencies} \label {subsec:Properties Relative Frequency}
	\begin{enumerate}[noitemsep, nolistsep]
		\item $f_k (n) = \frac{N_k (n)}{n}$
		\item $0 \leq N_k (n) \leq n$
		\item $0 \leq f_k (n) \leq 1 = \frac{0}{n} \leq \frac{N_k (n)}{n} \leq \frac{n}{n}$
		\item $\sum_{k=1}^{k} f_k (n) = \sum_{k=1}^{k} \frac{N_k (n)}{n} = \frac{\sum_{k=1}^{k} N_k (n)}{n} = \frac{n}{n} = 1$
		\item $\sum_{k=1}^{k} f_k (n) = 1$
		\item If events A and B are disjoint and event C is "A or B", then $F_C = F_A (n) + F_B (n)$
	\end{enumerate}
\end{document}