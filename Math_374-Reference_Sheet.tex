\documentclass[10pt,letterpaper,final,twoside,notitlepage]{article}
\usepackage[margin=.5in]{geometry}
\usepackage[utf8]{inputenc}
\usepackage[english]{babel}
\usepackage{amsmath}
\usepackage{amsfonts}
\usepackage{amssymb}
\usepackage{amsthm} % Gives us plain, definition, and remark to use in \theoremstyle{style}
\usepackage{graphicx}

\usepackage{hyperref} % Generate hyperlinks to referenced items
\usepackage{nameref} % Can make references by name to places
\usepackage{ctable} % Greater control over tables and how they look
\usepackage{subcaption} % Allows for multiple figures in one Figure environment
\usepackage{textcomp} % Forcibly loads a package that gensymb relies on
\usepackage{gensymb} % Gives access to some characters, for example, the degree symbol
\usepackage{enumitem} % Provides [noitemsep, nolistsep] for more compact lists
\usepackage{chngcntr} % Allows us to tamper with the counter a little more
\usepackage{empheq} % Allow boxing of equations in special math environments

%\graphicspath{{./Drawings/Math_374}} % Uncomment this to use pictures in this document
\counterwithin{equation}{section} % Uncomment to number eqns with sec nums too

\theoremstyle{plain}
\newtheorem{theorem}{Theorem}

\theoremstyle{definition}
\newtheorem{definition}{Defn}
\newtheorem{corollary}{Corollary}[section]

\theoremstyle{remark}
\newtheorem{remark}{Remark}[definition]
%\counterwithin{definition}{subsection} % Uncomment to have definitions use section.subsection numbering

\newcounter{example}[section]


\author{Karl Hallsby}
\title{Reference Material}

\begin{document}
\section{Relative Frequency} \label{sec:Relative Frequency}
\begin{itemize}[noitemsep, nolistsep]
	\item $f_k (n) = \frac{N_k (n)}{n}$ $\leftarrow$ {\large \textbf{Relative Frequency}}
	\begin{itemize}[noitemsep, nolistsep]
		\item $k$ is the outcome
		\item $N_k (n)$ is the number of times outcome $k$
	\end{itemize}
	\item $\lim\limits_{n \rightarrow \infty} f_k (n) = p_k$ $\leftarrow$ {\large \textbf{Statistical Regularity}}
	\begin{itemize}[noitemsep, nolistsep]
		\item $p_k$ is the probability of event $k$ occurring
	\end{itemize}
\end{itemize}

	\subsection{Properties of Relative Frequencies} \label {subsec:Properties Relative Frequency}
	\begin{enumerate}[noitemsep, nolistsep]
		\item $f_k (n) = \frac{N_k (n)}{n}$
		\item $0 \leq N_k (n) \leq n$
		\item $0 \leq f_k (n) \leq 1 = \frac{0}{n} \leq \frac{N_k (n)}{n} \leq \frac{n}{n}$
		\item $\sum_{k=1}^{k} f_k (n) = \sum_{k=1}^{k} \frac{N_k (n)}{n} = \frac{\sum_{k=1}^{k} N_k (n)}{n} = \frac{n}{n} = 1$
		\item $\sum_{k=1}^{k} f_k (n) = 1$
		\item If events A and B are disjoint and event C is "A or B", then $F_C = F_A (n) + F_B (n)$
	\end{enumerate}

\section{Set Theory} \label{sec:Set Theory}
\begin{enumerate}[noitemsep, nolistsep]
	\item A \emph{set} is a collection of objects, denoted by capital letters
	\item Denote the \emph{universal set, $U$}; consisting of all possible objects of interest in a given setting/application
	\item For any set $A$, we say that \emph{``$x$ is an element of $A$''}, denoted $x \in A$ if object $x$ of the universal set $U$ is contained in $A$
	\item We say that \emph{``$x$ is not an element of $A$''}, denoted $x \notin A$ if object $x$ of the universal set $U$ is not contained in $A$
	\item We say that \emph{``$A$ is a subset of $B$''}, denoted $A \subset B$ if every element in $A$ also belongs to $B$, $x \in A \rightarrow x \in B$
	\item The \emph{empty set, $\emptyset$} is defined as the set with no elements
		\begin{itemize}[noitemsep, nolistsep]
			\item The empty set is a subset of every set
		\end{itemize}
	\item Sets \emph{$A$ and $B$ are equal} if they contain the same elements. To show this:
		\begin{enumerate}[noitemsep, nolistsep]
			\item Enumerate the elements of each set
			\item Thm: $A=B \iff A \subset B$ AND $B \subset A$
		\end{enumerate}
	\item The \emph{union of 2 sets $A$, $B$}, denoted $A \cup B$ is defined as the set of outcomes that are either in $A$, or in $B$, or both
	\item The \emph{intersection fo 2 sets, $A$, $B$}, denoted $A \cap B$ is defined as the set of outcomes in $A$ and $B$
	\item The 2 sets $A$, $B$ are said to be \emph{disjoint or mutually exclusive} if $A \cap B = \emptyset$
	\item The \emph{complement of a set $A$}, denoted $A^{C}$ is defined as the set of elements of $U$ not in $A$
		\begin{itemize}[noitemsep, nolistsep]
			\item $A^{C} = \lbrace x \in U \vert x \notin A \rbrace$
		\end{itemize}
	\item \emph{Relative complement} or \emph{difference}, denoted $A-B$, is the set of elements in $A$ that are not in $B$
		\begin{itemize}[noitemsep, nolistsep]
			\item $A-B = A \cap B^{C}$
			\item $A^{C} = U - A$
		\end{itemize}
\end{enumerate}

	\subsection{Properties of Set Operations} \label{subsec:Properties of Set Ops}
	Set Operators are:
	\begin{enumerate}
		\item Commutative, Equation~\eqref{eq:Set Ops-Commutative}
			\begin{equation} % Commutative
				\begin{aligned}
					A \cup B &= B \cup A \\
					A \cap B &= B \cap A \\
				\end{aligned}
				\label{eq:Set Ops-Commutative}
			\end{equation}
			
			\item Associative, Equation~\eqref{eq:Set Ops-Associative}
				\begin{equation} % Associative
					\begin{aligned}
						A \cup \left( B \cup C \right) &= \left( A \cup B \right) \cup C \\
						A \cap \left( B \cap C \right) &= \left( A \cap B \right) \cap C \\
					\end{aligned}
					\label{eq:Set Ops-Associative}
				\end{equation}
			
			\item Distributive, Equation~\eqref{eq:Set Ops-Distributive}
				\begin{equation} % Distributive
					\begin{aligned}
						A \cup \left( B \cap C \right) &= \left( A \cup B \right) \cap \left( A \cup C \right) \\
						A \cap \left( B \cup C \right) &= \left( A \cap B \right) \cup \left( A \cap C \right) \\
					\end{aligned}
					\label{eq:Set Ops-Distributive}
				\end{equation}
			
			\item Set Operations obey De Morgan's Laws, Equation~\eqref{eq:Set Ops-De Morgan's}
				\begin{equation} % De Morgan's
					\begin{aligned}
						\left( A \cup B \right)^{C} &= A^{C} \cap B^{C} \\
						\left( A \cap B \right)^{C} &= A^{C} \cup B^{C} \\
					\end{aligned}
					\label{eq:Set Ops-De Morgan's}
				\end{equation}
			
	\end{enumerate}
	Additionally, 
	\begin{definition}[Union of $n$ Sets] \label{def:Union of n Sets}
		The \emph{union of $n$ sets} $\bigcup\limits_{k=1}^{n} A_{k} = A_{1} \cup A_{2} \cup A_{3} \cup \ldots \cup A_{n}$ is the set consisting of all elements such that $x \in A_{k}$ for some $1 \leq k \leq n$.
		\begin{itemize}[noitemsep, nolistsep]
			\item All sets need to be empty to make $\bigcup\limits_{k=1}^{n} A_{k} = \emptyset$
		\end{itemize}
	\end{definition}
	\begin{definition}[Intersection of $n$ Sets] \label{def:Intersection of n Sets}
		The \emph{intersection of $n$ sets} $\bigcap\limits_{k=1}^{n} A_{k} = A_{1} \cap A_{2} \cap A_{3} \cap \ldots \cap A_{n}$ is the set consisting of all elements such that $x \in a_{k}$ for all $1 \leq k \leq n$
		\begin{itemize}[noitemsep, nolistsep]
			\item Just one set needs to be empty to make $\bigcap\limits_{k=1}^{n} A_{k} = \emptyset$
		\end{itemize}
	\end{definition}

\section{Probability Theory} \label{sec:Probability Theory}
There are 3 main components to \nameref{sec:Probability Theory}.
\begin{enumerate}[noitemsep, nolistsep]
	\item \nameref{sec:Set Theory}
	\item \nameref{subsec:Probability Law Corollary}
	\item \nameref{subsec:Conditional Probability}~and~\nameref{subsec:Event Independence}
\end{enumerate}

	\subsection{Random Experiments} \label{subsec:Random Experiments}
	\begin{definition}[Random Experiment] \label{def:Random Experiment}
		A \emph{random experiment} is an experiment whose outcome varies in an unpredictable fashion when performed under the same conditions.
	\end{definition}
	\begin{definition}[Sample Space] \label{def:Sample Space}
		A \emph{sample space, $S$} of a random experiment is the set of all possible experiments.
	\end{definition}
	\begin{definition}[Outcome/Sample Point] \label{def:Outcome}
		An \emph{outcome}, or \emph{sample point} of a random experiment is a result that cannot be decomposed into other results.
	\end{definition}
	\begin{definition}[Event] \label{def:Event}
		An \emph{event} corresponds to a subset of the sample space. We say an event occurs if and only if (iff) the outcome of the experiment is in the subset representing the event.
	\end{definition}
	\begin{definition}[Event Classes] \label{def:Event Classes}
		An \emph{event class} $\EventClass$ is the collection of the all the events' sets. $\EventClass$ should be closed under unions, intersections, and complements.
		\begin{itemize}[noitemsep, nolistsep]
			\item For $S$ finite, or countably infinite, then we can let $\EventClass$ be all subsets of $S$.
			\item For $S$ uncountably infinite, instead we can let $\EventClass$ consist of the subsets that can be obtained as countable unions and intersections of some sets of $\EventClass$.
		\end{itemize}
	\end{definition}
	\begin{definition}[Probability Law] \label{def:Probability Law}
		A \emph{probability law} for a random experiment $E$, with sample space $S$, and an event class $\EventClass$ is a rule that assigns to each event $A \in \EventClass$ a number $P \left[A \right]$, called the probability of $A$ that satisfies the axioms:
		\begin{enumerate}[label=Axiom~\Roman*:, align=left, noitemsep, nolistsep] \label{subdef:Probability Law Axioms}
			\item $0 \leq P\left[ A \right]$
			\item $P \left[ S \right] = 1$
			\item If $A \cap B = \emptyset$, then $P \left[ A \cup B \right] = P \left[ A \right] + P \left[ B \right]$
			\item[Axiom III':] If $A_{1}$, $A_{2}$, $\ldots$ is a sequence of events such that $A_{i} \cap A_{j} = \emptyset$ for all $i \neq j$, then $P \left[ \bigcup_{k=1}^{\infty} A_{k} \right] = \sum_{k=1}^{\infty} P \left[ A_{k} \right]$
		\end{enumerate}
	\end{definition}
	
	\subsection{Probability Law Corollaries} \label{subsec:Probability Law Corollary}
		\begin{enumerate}[label=Axiom~\Roman*:, align=left, noitemsep, nolistsep] % Probability Law Axioms
			\item $0 \leq P\left[ A \right]$
			\item $P \left[ S \right] = 1$
			\item If $A \cap B = \emptyset$, then $P \left[ A \cup B \right] = P \left[ A \right] + P \left[ B \right]$
			\item[Axiom III':] If $A_{1}$, $A_{2}$, $\ldots$ is a sequence of events such that $A_{i} \cap A_{j} = \emptyset$ for all $i \neq j$, then $P \left[ \bigcup_{k=1}^{\infty} A_{k} \right] = \sum_{k=1}^{\infty} P \left[ A_{k} \right]$
		\end{enumerate}
		\begin{corollary} \label{cor:Probability Parts}
			$P \left[ A^{C} \right] = 1 - P \left[ A \right]$
		\end{corollary}
		\begin{corollary} \label{cor:Probability of Event}
			$P \left[ A \right] \leq 1$
		\end{corollary}
		\begin{corollary} \label{cor:Probability of Empty Set}
			$P \left[ \emptyset \right] = 0$
		\end{corollary}
		\begin{corollary} \label{cor: Probability Addition of Disjoint Pairs}
			If $A_{1}$, $A_{2}$, $\ldots$, $A_{n}$ are pairwise mutually exclusive ($A_{1} \cap A_{2} \cap \ldots \cap A_{n} = \emptyset$), then $P \left[ \bigcup_{k=1}^{n} \right] = \sum_{k=1}^{n} P \left[ A_{k} \right]$ for $n \geq 2$
		\end{corollary}
		\begin{corollary} \label{cor:Inclusion-Exclusion Principle to 2 Sets}
			$P \left[ A \cup B \right] = P \left[ A \right] + P \left[ B \right] - P \left[ A \cap B \right]$
		\end{corollary}
		\begin{corollary} \label{cor:Inclusion-Exclusion Principle to n Sets}
			$P \left[ A \cup B \right] = \sum\limits_{j=1}^{n} P \left[ A_{j} \right] - \sum_{j<k} P \left[A_{j} \cap A_{k} \right] + \ldots + \left( -1 \right)^{n+1} P \left[ A_{1} \cap \ldots \cap A_{n} \right]$
		\end{corollary}
		\begin{corollary} \label{cor:Subset Probability to Superset}
			If $A \subset B$, then $P \left[ A \right] \leq P \left[ B \right]$
		\end{corollary}
	
	\subsection{Conditional Probability} \label{subsec:Conditional Probability}
		\begin{definition}[Conditional Probability] \label{def:Conditional Probability}
			The \emph{conditional probability} of event $A$ \textbf{GIVEN THAT} event $B$ occurred is denoted $P \left[ A \vert B \right]$ and is defined as
			\begin{equation} \label{eq:Conditional Probability}
				P \left[ A \vert B \right] = \frac{P \left[ A \cap B \right]}{P \left[ B \right]}
			\end{equation}
		\end{definition}
		\begin{theorem}[Theorem of Total Probability] \label{thm:Theorem of Total Probability}
			Let $B_{1}$, $B_{2}$, $\ldots$, $B_{n}$ be mutually exclusive events whose union equals the sample space $S$, i.e. $B_{1}$, $B_{2}$, $\ldots$, $B_{n}$ is a partition of $S$.
		\end{theorem}
		\begin{definition}[Baye's Rule] \label{def:Baye's Rule}
			Let $B_{1}$, $B_{2}$, $\ldots$, $B_{n}$ be a partition of sample space $S$.
			\begin{equation}
				P \left[ B_{j} \vert A \right] = \frac{P \left[ A \cap B_{j} \right]}{P \left[ A \right]}
				= \frac{P \left[ A \vert B_{j} \right] * P \left[ B_{j} \right]}{\sum\limits_{k=1}^{n} P \left[ A \vert B_{k} \right] * P \left[ B_{k} \right]}
			\end{equation}
		\end{definition}
		\begin{example}[Exam 1, Problem 5]{Baye's Rule}
			An urn contains 9 balls, identical in every way, except that they are labeled with numbers 1 through 9.
			Two balls are selected at random, without replacement, and the sequence of labels observed are recorded.
			\begin{enumerate}[label=(\alph*), noitemsep, nolistsep]
				\item Give the formula for the conditional probability of event $A$ given that event $B$ occurred (where $A$ and $B$ are arbitrary events).
				\item What is the probability that the label of the second ball is even?
				\item What is the probability that the label of the first ball was odd given that the second was even?
			\end{enumerate}
		\end{example}
	
	\subsection{Event Independence} \label{subsec:Event Independence}
		\begin{definition}[Independent] \label{def:Event Independence}
			Two events $A$ and $B$ are \emph{independent} if 
			\begin{equation} \label{eq:Event Independence}
				P \left[ A \cap B \right] = P \left[ A \right] * P \left[ B \right], P\left[ A \right] \neq 0, P\left[ B \right] \neq 0
			\end{equation}
			\begin{itemize}[noitemsep, nolistsep]
				\item If $A \cap B = \emptyset$, the $A$ and $B$ are \textbf{dependent}.
				\item If checking for independence between more than 2 events, you must check each pair, each triple, etc. until you check the independence of each event against each other. For 3 events, $A$, $B$, $C$:
					\begin{itemize}[noitemsep, nolistsep]
						\item Check $P \left[ A \cap B \cap C \right] = P \left[ A \right] * P \left[ B \right] * P \left[ C \right]$
						\item Also need to check:
							\begin{enumerate}[noitemsep, nolistsep]
								\item $P \left[ A \cap B \right] = P \left[ A \right] * P \left[ B \right]$
								\item $P \left[ B \cap C \right] = P \left[ B \right] * P \left[ C \right]$
								\item $P \left[ A \cap C \right] = P \left[ A \right] * P \left[ C \right]$
							\end{enumerate}
					\end{itemize}
			\end{itemize}
		\end{definition}
		\begin{example}[Exam 1, Problem 4]{Event Independence}
			Let $S=\lbrace 1,2,3,4 \rbrace$, and $A = \lbrace 1,2 \rbrace$, $B=\lbrace 1,3 \rbrace$, $C = \lbrace 1,4 \rbrace$, $D = \lbrace 3,4 \rbrace$.
			Assume the outcomes are equiprobable.
			Are the following events independent?
				\begin{enumerate}[noitemsep, nolistsep]
					\item $A$ and $B$
					\item $A$ and $D$
					\item $A$, $B$, and $C$
				\end{enumerate}
		\end{example}
	If 2 events $A$ and $B$ are independent, then their complements are also independent. This is shown in \nameref{proof:Independence of Complements of Events}.
		\begin{proof}[Independence of Complements of Events] \label{proof:Independence of Complements of Events}
			We assumed that $A$ and $B$ were independent, so $P \left[ A \cap B \right] = P \left[ A \right] \cdot P \left[ B \right]$.
			There are 2 more facts we will need:
			\begin{enumerate}[leftmargin=1.0in, label=Fact \arabic*: , ref=Fact \arabic*, noitemsep, nolistsep]
				% leftmargin sets a distance for left margin
				% ref sets the way items will be cross-referenced, and can differ from the label.
				\item $P \left[ B \right] + P \left[ B^{C} \right] = 1$ \label{proof:Independence of Complements of Events:Fact 1}
				\item $P \left[ A \cap B^{C} \right] + P \left[ A \cap B \right] = P \left[ A \right]$ \label{proof:Independence of Complements of Events:Fact 2}
			\end{enumerate}
			From \ref{proof:Independence of Complements of Events:Fact 1}, we have:
			\begin{equation*}
				P \left[ A \cap B \right] = P \left[ A \right] \cdot \left( 1-P \left[ B^{C} \right] \right)
			\end{equation*}
			From \ref{proof:Independence of Complements of Events:Fact 2}, we have $P \left[ A \cap B \right] = P \left[ A \right] - P \left[ A \cap B^{C} \right]$.
			Substituting these into the equation above:
			\begin{align*}
				P \left[ A \right] - P \left[ A \cap B^{C} \right] &= P \left[ A \right] \cdot \left( 1-P \left[ B^{C} \right] \right)\\
				P \left[ A \right] - P \left[ A \cap B^{C} \right] &= P \left[ A \right] - P \left[ A \right] \cdot P \left[ B^{C} \right] \\
				- P \left[ A \cap B^{C} \right] &= -P \left[ A \right] \cdot P \left[ B^{C} \right] \\
				P \left[ A \cap B^{C} \right] &= P \left[ A \right] \cdot P \left[ B^{C} \right] \\
			\end{align*}
			$\therefore$ $A$ and $B^{C}$ are independent, according to the definition of \nameref{def:Event Independence}~events in \Cref{eq:Event Independence}.
		\end{proof}

\section{Counting} \label{sec:Counting}
	\subsection[Permutations]{Ordered Sampling with Replacement} \label{subsec:Ordered Sampling with Replacement} \label{subsec:Permutations}
		\begin{definition}[Permutations] \label{def:Ordered Sampling with Replacement}
			The number of distinct outcomes of an experiment, where the elements being samples are replaced between each sampling.
			\begin{equation*}
				\text{If } k = n \\
			\end{equation*}
			\begin{equation} \label{eq:Ordered Sampling with Replacement}
				\frac{n}{First} * \frac{n-1}{Second} * \frac{n-2}{Third} * \ldots * \frac{n-k-1}{kth \text{ Item}} = n!
			\end{equation}
		\end{definition}
	\subsection{Ordered Sampling without Replacement} \label{subsec:Ordered Sampling without Replacement}
		\begin{definition} \label{def:Ordered Sampling without Replacement}
			Choose $k$ elements in succession wihtout replacement from a population of $n$ distinct objects, where $k \leq n$
			\begin{equation} \label{eq:Ordered Sampling without Replacement}
				\frac{n}{First} * \frac{n-1}{Second} * \frac{n-2}{Third} * \ldots * \frac{n-k-1}{kth \text{ Item}}
			\end{equation}
		\end{definition}
	\subsection{Unordered Sampling with Replacement} \label{subsec:Unordered Sampling with Replacement}
		
	\subsection{Unordered Sampling without Replacement} \label{subsec:Unordered Sampling without Replacement}		
		\begin{definition} \label{def:Unordered Sampling with Replacement}
			The number of ways to choose $k$ items out of $n$ items.
			Said $n$ choose $k$:
			\begin{equation} \label{eq:Unordered Sampling with Replacement}
			\binom{n}{k} = \frac{n * (n-1) * (n-2) * \ldots * (n-k+1)}{k!} = \frac{n!}{k! \left( n-k \right)!}
			\end{equation}
		\end{definition}
		\begin{equation}
			\binom{n}{k} = \binom{n}{n-k}
		\end{equation}
		
\section{Single Discrete Random Variables} \label{sec:Single Discrete Random Variables}
	\begin{definition}[Random Variable]
		A \emph{random variable} $X$ is a function that assigns a real number $X \left( \zeta \right)$ to each outcome $\zeta$ in the sample space of the random experiment.
	\end{definition}
\section{Single Continuous Random Variables} \label{sec:Single Continuous Random Variables}

\section{Multiple Random Variables} \label{sec:Multiple Random Variables}
	\subsection[Joint PMF]{Joint Probability Mass Function} \label{subsec:Joint PMF}
		\begin{definition}[Joint Probability Mass Function] \label{def:Joint PMF}
			The \emph{joint probability mass function (joint PMF)} of 2 discrete random variables $X$, $Y$ is defined as:
			\begin{equation} \label{eq:Joint PMF}
				p_{X,Y} = P \left[ \lbrace X=x \rbrace \cap \lbrace Y=y \rbrace \right] \text{ for all } x,y \in S_{X,Y}
			\end{equation}
			\begin{itemize}[noitemsep, nolistsep]
				\item This satisfies ALL properties of single random variable PMFs
			\end{itemize}
		\end{definition}
	
		\subsubsection[Marginal PMF]{Marginal Probability Mass Function} \label{subsubsec:Marginal PMF}
			\begin{definition}[Marginal Probability Mass Function] \label{def:Marginal PMF}
				Given a joint PMF of discrete random variables $X$, $Y$, the \emph{Marginal Probability Mass Function (Marginal PMF)} of $X$ is defined as:
				\begin{equation} \label{eq:Marginal PMF}
					p_{X} \left( x_{i} \right) = P \left[ X = x_{i} \right] \text{ for } x_{i} \in S_{X}
				\end{equation}
				and is calculated as:
				\begin{equation} \label{eq:Calculate Marginal PMF}
					p \left( x_{i} \right) = \sum_{y \in S_{Y}} p_{X,Y} \left( x_{i}, y \right)
				\end{equation}
			\end{definition}
		
	\subsection[Joint CDF]{Joint Cumulative Distribution Function} \label{subsec:Joint CDF}
		\begin{definition}[Joint Cumulative Distribution Function] \label{def:Joint CDF}
			The \emph{Joint Cumulative Distribution Function (Joint CDF)} of $X$ and $Y$ is defined as the probability of the event $ \lbrace X \leq x \rbrace \cap \lbrace Y \leq y \rbrace $
			\begin{equation} \label{eq:Joint CDF}
				\begin{aligned}
					F_{X,Y} \left( x, y \right) &= P \left[ \lbrace X \leq x \rbrace \cap \lbrace Y \leq y \rbrace \right] \text{ for all } \left( x,y \right) \in \mathbb{R}^2 \\
					&= P \left[ \lbrace X \leq x \rbrace , \lbrace Y \leq y \rbrace \right]
				\end{aligned}
			\end{equation}
		\end{definition}
		
		\subsubsection{Properties of Joint Cumulative Distribution Functions} \label{subsubsec:Properties of Joint Cumulative Distribution Functions}
			\begin{enumerate}[label=\textbf{(\roman*)}, noitemsep, nolistsep]
				\item $F_{X,Y} \left( x,y \right)$ is non decreasing.
					\begin{equation} \label{eq:Joint CDF Property 1}
						F_{X,Y} \left( x_{1},y_{1} \right) \leq F_{X,Y} \left( x_{2},y_{2} \right) \text{ if } x_{1} \leq x_{2} \text{ and } y_{1} \leq y_{2}
					\end{equation}
				\item \begin{equation} \label{eq:Joint CDF Property 2}
						\begin{aligned}
							\lim\limits_{y \rightarrow -\infty} F_{X,Y} \left( x,y \right) &= 0 \\
							\lim\limits_{x \rightarrow -\infty} F_{X,Y} \left( x,y \right) &= 0 \\
							\lim\limits_{\left( x,y \right) \rightarrow \left( \infty, \infty \right)} F_{X,Y} \left( x,y \right) &= 1 \\
						\end{aligned}
					\end{equation}
				\item The Marginal CDFs can be obtained from the Joint CDF by removing restrictions for all but one variable.
					\begin{equation} \label{eq:Joint CDF Property 3}
						\begin{aligned}
							F_{X} \left( x \right) &= P \left[ \lbrace X \leq x \rbrace, \lbrace Y \text{ is anything} \rbrace \right] \\
													   &= P \left[ \lbrace X \leq x \rbrace, \lbrace -\infty \leq y \leq \infty \rbrace \right] \\
													   &= \lim\limits_{y \rightarrow \infty} F_{X,Y} \left( x,y \right) \\
							F_{Y} \left( y \right) &= \lim\limits_{x \rightarrow \infty} F_{X,Y} \left( x,y \right) \\
						\end{aligned}
					\end{equation}
				\item The Joint CDF is continuous from $\infty$ to $-\infty$.
					\begin{equation} \label{eq:Joint CDF Property 4}
						\begin{aligned}
							\lim\limits_{x \rightarrow a^{+}} F_{X,Y} \left( x,y \right) &= F_{X,Y} \left( a,y \right) \\
							\lim\limits_{y \rightarrow b^{+}} F_{X,Y} \left( x,y \right) &= F_{X,Y} \left( x,b \right) \\
						\end{aligned}
					\end{equation}
				\item The probability of the ``rectangle'' $\lbrace x_{1} \leq X \leq x_{2}, y_{1} \leq Y \leq y_{2} \rbrace$
					\begin{equation} \label{eq:Joint CDF Property 5}
						\begin{aligned}
							P \left[ \lbrace x_{1} \leq X \leq x_{2}, y_{1} \leq Y \leq y_{2} \rbrace \right] &= P \left[ \lbrace X \leq x_{2}, Y \leq y_{2} \rbrace \right] - P \left[ \lbrace X \leq x_{1}, Y \leq y_{2} \rbrace \right] - \\
							&P \left[ \lbrace X \leq x_{2}, Y \leq y_{1} \rbrace \right] + P \left[ \lbrace X \leq x_{1}, Y \leq y_{1} \rbrace \right] \\
							&= F_{X,Y} \left( x_{2}, y_{2} \right) - F_{X,Y} \left( x_{1}, y_{2} \right) - F_{X,Y} \left( x_{2}, y_{1} \right) + F_{X,Y} \left( x_{1}, y_{1} \right)
						\end{aligned}
					\end{equation}
			\end{enumerate}
	
		\subsubsection[Marginal CDF]{Marginal Cumulative Distribution Function} \label{subsubsec:Marginal CDF}
			\begin{definition}[Marginal Cumulative Distribution Function] \label{def:Marginal CDF}
				We obtain the \emph{Marginal Cumulative Distribution Functions (Marginal CDFs)} by removing the constraint on one of the variables. 
				\begin{equation} \label{eq:Marginal CDF}
					\begin{aligned}
						F_{X} \left( x \right) &= P \left[ \lbrace X \leq x \rbrace, \lbrace Y \text{ is anything} \rbrace \right] \\
						&= P \left[ \lbrace X \leq x \rbrace, \lbrace -\infty \leq y \leq \infty \rbrace \right] \\
						&= \lim\limits_{y \rightarrow \infty} F_{X,Y} \left( x,y \right) \\
						F_{Y} \left( y \right) &= \lim\limits_{x \rightarrow \infty} F_{X,Y} \left( x,y \right) \\
					\end{aligned}
				\end{equation}
			\end{definition}
	\subsection[Joint PDF]{Joint Probability Density Function} \label{subsec:Joint PDF}
		\begin{definition}[Joint Probability Density Function] \label{def:Joint PDF}
			We say that $X$, $Y$ are jointly continuous if the probabilities of events involving $X$ and $Y$ can be expressed as an integral of a \emph{Joint Probability Density Function (Joint PDF)}. \newline
			i.e. THere exists soem nonnegative function $f_{X,Y} \left( x,y \right)$, which we call the joint PDF, that is defined on the real plane such that tfor every event $B$ which is a subset of the xy plane
			\begin{equation}\label{eq:Joint PDF}
				P \left[ \left( X,Y \right) \text{in } B \right] = \iint_{B} f_{X,Y} \left( x,y \right) dx dy
			\end{equation}
			\begin{remark}
				The probability mass of an event is found by integrating the PDF over the region in the xy plane corresponding to your event.
			\end{remark}
		\end{definition}
	
		\subsubsection{Properties of Joint Probability Density Functions} \label{subsubsec:Joint PDF Properties}
			\begin{gather}
				\iint_{B} f_{X,Y} \left( x,y \right) = 1 \\
				x \geq 0, y \geq 0 \forall x \forall y 
			\end{gather}
			
		\subsubsection{Facts about Joint PDFs} \label{subsubsec:Joint PDF Facts}
			\begin{align}
				\int_{-\infty}^{\infty} \int_{-\infty}^{\infty} &f_{X,Y} \left( x,y \right) = 1 \\
				F_{X,Y} \left( x,y \right) &= \int_{-\infty}^{x} \int_{-\infty}^{y} f_{X,Y} \left( s,t \right) dt ds \\
				f_{X,Y} &= \frac{\partial^{2} f_{X,Y} \left( x,y \right)}{\partial x \partial y} 
			\end{align}
			
		\subsubsection{Marginal PDF} \label{subsubsec:Marginal PDF}
			\begin{definition}[Marginal Probability Density Function] \label{def:Marginal PDF}
				The \emph{Marginal Probability Density Functions (Marginal PDFs)} $f_{X} \left( x \right)$ and $f_{Y} \left( y \right)$ are obtained by taking the derivative of the marginal CDFs.
				\begin{equation}
					\begin{aligned}
						f_{X} \left( x \right) &= \frac{d}{dx} F_{X} \left( x \right) \\
						&= \frac{d}{dx} \int_{-\infty}^{x} \left[ \int_{-\infty}^{\infty} f_{X,Y} \left( s,t \right) dt ds \right] \\
						&= \frac{d}{dx} \int_{-\infty}^{x} \int_{-\infty}^{\infty} f_{X,Y} \left( s,t \right) dt ds \\
						&\text{Simplified with \nameref{def:2nd Fundamental Theorem of Calculus}} \\
						&= \int_{-\infty}^{\infty} f_{X,Y} \left( x,t \right) dt \\
						f_{X} &= \int_{-\infty}^{\infty} f_{X,Y} \left( x,t \right) dt \\
					\end{aligned}
				\end{equation}
			\end{definition}
	\subsection{Independence of Multiple Random Variables} \label{subsec:Independence of Multiple Random Variables}
		\begin{definition}[Independent Random Variables] \label{def:Independence of Multiple Random Variables}
			\emph{$X$ and $Y$ are independent random variables} if \emph{\textbf{ANY}} event $A_{1}$ defined in terms of $S$ is independent of \emph{\textbf{ANY}} event $A_{2}$ defined in terms of $Y$.
			\begin{equation} \label{eq:Independence of Multiple Random Variables}
				P \left[ X \in A_{1}, Y \in A_{2} \right] = P \left[ X \in A_{1} \right] * P \left[ Y \in A_{2} \right]
			\end{equation}
		\end{definition}
	There are 3 ways to phrase this:
	\begin{enumerate}[noitemsep, nolistsep]
		\item For discrete random variables $X$ and $Y$, $X$ and $Y$ are independent if and only if:
			\begin{equation} \label{eq:Independence of Multiple Discrete Random Variables Using PMF}
				p_{X,Y} \left( x,y \right) = p_{X} \left( x \right) * p_{Y} \left( y \right)
			\end{equation}
		\item For general random variables $X$ and $Y$, $X$ and $Y$ are independent if and only if:
			\begin{equation} \label{eq:Independence of Multiple General Random Variables Using CDF}
				F_{X,Y} \left( x,y \right) = F_{X} \left( x \right) * F_{Y} \left( y \right)
			\end{equation}
		\item For (continuous) random variables $X$ and $Y$, $X$ and $Y$ are independent if and only if:
			\begin{equation} \label{eq:Independence of Multiple Continuous Random Variables Using PDF}
				f_{X,Y} \left( x,y \right) = f_{X} \left( x \right) * f_{Y} \left( y \right)
			\end{equation}
	\end{enumerate}
	You can prove \nameref{eq:Independence of Multiple Discrete Random Variables Using PMF}, \Cref{eq:Independence of Multiple Discrete Random Variables Using PMF}.
	\begin{proof}[Independence of Discrete Random Variables with PMF] \label{proof:Independence of Discrete Random Variables with PMF}
		
	\end{proof}
	\begin{theorem}[Independence of Random Functions] \label{thm:Independence of Random Functions}
		If random variables $X$, $Y$ are independent, then $g\left( X \right)$ and $h \left( Y \right)$ are also independent.
	\end{theorem}

	\subsection{Expected Value of Functions with 2 Random Variables} \label{subsec:Expected Value of Functions with 2 Random Variables}
		\begin{definition}[Expectation of a Function with 2 Random Variables] \label{def:Expectation of a Function with 2 Random Variables}
			Let $Z$ be a random variable described by the function $Z = g \left( X,Y \right)$.
			\begin{equation} \label{eq:Expected Value of a Function with 2 Random Variables}
				\ExpectedValue = 
				\begin{cases}
					\int_{-\infty}^{\infty} \int_{-\infty}^{\infty} g \left( x,y \right) \cdot f_{X,Y} \left( x,y \right) dx dy &
						\text{if $X$ and $Y$ are jointly continuous} \\
					\sum\limits_{i \in S_{X}} \sum\limits_{j \in S_{Y}} g \left( x_{i}, y_{j} \right) \cdot p_{X,Y} \left( x,y \right) &
						\text{if $X$ and $Y$ are both discrete} \\
				\end{cases}
			\end{equation}
			\begin{remark}[Expected Value of Sum of Random Variables] \label{rmk:Expected Value of Sum of Random Variables}
				You \emph{\textbf{do not}} need to assume independence to say:
				\begin{equation} \label{eq:Expected Value of Sum of Random Variables}
					\ExpectedValue \left[ X_{1}+X_{2}+\ldots+X_{n} \right] = \ExpectedValue \left[ X_{1} \right] + \ExpectedValue \left[ X_{2} \right] + \ldots + \ExpectedValue \left[ X_{n} \right]
				\end{equation}
			\end{remark}
			\begin{remark}[Expected Value of Product of Random Variables] \label{rmk:Expected Value of Product of Random Variables}
				If $X$ and $Y$ are independent, then
				\begin{equation} \label{eq:Expected Value of Product of Random Variables}
					\ExpectedValue \left[ g \left( X \right) h \left( Y \right) \right] = \ExpectedValue \left[ g \left( X \right) \right] \cdot \ExpectedValue \left[ h \left( Y \right) \right]
				\end{equation}
			\end{remark}
		\end{definition}

	\subsection{Joint Moments, Correlation, and Covariance} \label{subsec:Joint Moments, Correlation, and Covariance}
		\subsubsection{Joint Moments} \label{subsubsec:Joint Moments}
			\begin{definition}[The j,kth Moment] \label{def:jkth Moment}
				The \emph{j,kth moment of $X$ and $Y$} is:
				\begin{equation} \label{eq:jkth Moment}
					\ExpectedValue \left[ X^{j} Y^{k} \right] =
					\begin{cases}
						\int_{-\infty}^{\infty} \int_{-\infty}^{\infty} x^{j} y^{k} \cdot f_{X,Y} \left( x,y \right) dx dy &
							\text{if $X$, $Y$ are jointly continuous} \\
						\sum\limits_{i \in S_{X}} \sum\limits_{\ell \in S_{Y}} x_{i}^{j} y_{l}^{k} \cdot p_{X,Y} \left( x_{i},y_{\ell} \right) & 
							\text{if $X$, $Y$ are discrete} \\
					\end{cases}
				\end{equation}
			\end{definition} 
		
		\subsubsection{Correlation} \label{subsubsec:Correlation}
			\begin{definition}[Correlation] \label{def:Correlation}
				The \emph{Correlation of $X$ and $Y$} is defined as the $1,1$ moment, i.e. $\ExpectedValue \left[ X^{1} Y^{1} \right]$.
				\begin{remark}
					If $X$, $Y$ are such that $\ExpectedValue \left[ X^{1} Y^{1} \right] = 0$, then we say that $X$, $Y$ are \emph{orthogonal}.
				\end{remark}
				\begin{remark}[Uncorrelated] \label{rmk:Uncorrelated}
					If $X$, $Y$ are such that $\ExpectedValue \left[ XY \right] = \ExpectedValue \left[ X \right] \ExpectedValue \left[ Y \right]$, then $X$ and $Y$ are \emph{uncorrelated}.
				\end{remark}
				\begin{remark}
					If $X$, $Y$ are independent, then they are uncorrelated; but if $X$ and $Y$ are uncorrelated, \emph{\textbf{they are not always independent}}.
				\end{remark}
			\end{definition}
			\begin{definition}[Correlation Coefficient] \label{def:Correlation Coefficient}
				The \emph{correlation coefficient of $X$, $Y$} is defined as
				\begin{equation} \label{eq:Correlation Coefficient}
					\rho_{X,Y} = \frac{\text{Cov} \left[ X,Y \right]}{\sigma_{X} \sigma_{Y}}
				\end{equation}
				\begin{remark}
					$\rho_{X,Y}$ only ranges $-1 \leq \rho_{X,Y} \leq 1$
				\end{remark}
				\begin{remark}
					The closer $\rho_{X,Y}$ is to $+1$, the closer $X$ and $Y$ are to having a positive linear relationship (Positive slope). \newline
					The closer $\rho_{X,Y}$ is to $-1$, the closer $X$ and $Y$ are to having a negative linear relationship (Negative slope). \newline
					If $\rho_{X,Y} = 0$, the $\text{Cov}\left[ X,Y \right] = 0$, which means that $X$ and $Y$ are \emph{uncorrelated}.
				\end{remark}
			\end{definition}
	
		\subsubsection{Covariance} \label{subsubsec:Covariance}
			\begin{definition}[Covariance] \label{def:Covariance}
				The \emph{covariance of $X$ and $Y$} is denoted:
				\begin{equation} \label{eq:Covariance-Form 1}
					\text{Cov} \left[ X,Y \right] = \ExpectedValue \left[ \left( X - \ExpectedValue \left[ X \right] \right) \left( Y - \ExpectedValue \left[ Y \right] \right)\right]
				\end{equation}
				\begin{equation} \label{eq:Covariance-Form 2}
					\text{Cov} \left[ X,Y \right] = \ExpectedValue \left[ XY \right] - \ExpectedValue \left[ X \right] \ExpectedValue \left[ Y \right]
				\end{equation}
			\end{definition}
		
		\subsection{Conditional Probability Functions} \label{subsec:Multiple Variable Conditional Probability Functions}
		There are 3 major cases for these:
			\begin{enumerate}[noitemsep, nolistsep]
				\item \nameref{subsubsec:2 Discrete Random Variables}
				\item \nameref{subsubsec:1 Discrete 1 Continuous Random Variables}
				\item \nameref{subsubsec:2 Continuous Random Variables}
			\end{enumerate}
					
			\subsubsection{2 Discrete Random Variables} \label{subsubsec:2 Discrete Random Variables}
				\begin{definition}[Conditional Probability Mass Function] \label{def:2 Discrete-Conditional PMF}
					The \emph{conditional Probability Mass Function (Conditional PMF)} of $Y$ given that $X=x$ is:
					\begin{equation} \label{eq:2 Discrete-Conditional PMF}
						p_{Y} \left( y \Given x \right)
							= \frac{P \left[ \lbrace Y=y \rbrace \cap \lbrace X=x \rbrace \right]}{P \left[ X=x \right]} 
							= \frac{p_{X,Y} \left( x,y \right)}{p_{X} \left( x \right)}
					\end{equation}
					\begin{remark}
						This also implies that
						\begin{equation} \label{eq:2 Discrete-Joint PMF}
							p_{X,Y} \left( x,y \right) = p_{Y} \left( y \Given x \right) \cdot p_{X} \left( x \right)
						\end{equation}
					\end{remark}
					\begin{remark}
						If $X$ and $Y$ are \emph{independent}, then:
						\begin{equation} \label{eq:2 Discrete-Independent Conditional PMF}
							p_{X} \left( y \Given x \right)
							= \frac{p_{X,Y} \left( x,y \right)}{p_{X} \left( x \right)}
							= \frac{p_{X} \left( x \right) p_{Y} \left( y \right)}{p_{X} \left( x \right)}
							= p_{Y} \left( y \right)
						\end{equation}
					\end{remark}
					\begin{remark}
						The \nameref{def:2 Discrete-Conditional PMF} of 2 discrete random variables satisfies all \nameref{subsubsec:Properties of Probability Mass Functions}.
					\end{remark}
				\end{definition}

			\subsubsection{1 Discrete and 1 Continuous Random Variable} \label{subsubsec:1 Discrete 1 Continuous Random Variables}
			For this section, let $X$ be a discrete random variable and $Y$ a continuous random variable.
				\begin{definition}[Conditional Cumulative Distribution Function] \label{def:1 Discrete 1 Continuous-Conditional CDF}
					The \emph{conditional Cumulative Distribution Function (Conditional CDF)} of $Y$ given that $X=x$ is:
					\begin{equation} \label{eq:1 Discrete 1 Continuous-Conditional CDF}
						F_{Y} \left( y \Given x \right) = P \left[ Y \leq y \Given X=x \right]
						= \frac{P \left[ \lbrace Y \leq y \rbrace \cap \lbrace X=x \rbrace \right]}{P \left[ X=x \right]}
					\end{equation}
					\begin{remark}
						If $X$ and $Y$ are \emph{independent}, then:
						\begin{equation} \label{eq:1 Discrete 1 Continuous-Independent Conditional CDF}
							F_{Y} \left( y \Given x \right)
							= \frac{F_{X,Y} \left( x,y \right)}{p_{X} \left( x \right)}
							= \frac{F_{Y} \left( y \right) p_{X} \left( x \right)}{p_{X} \left( x \right)}
							= F_{Y} \left( y \right)
						\end{equation}
						This also means that:
						\begin{equation*}
							P \left[ Y \leq y \Given X=x \right]
							= P \left[ Y \leq y \right] \cdot P \left[ X=x \right]
						\end{equation*}
					\end{remark}
					\begin{remark}
						The similar relations for independent random variables with their conditional and marginal probability functions does not hold true with this.
					\end{remark}
					\begin{remark}
						The \nameref{def:1 Discrete 1 Continuous-Conditional CDF} of 1 discrete random variable and 1 continuous random variable satisfies all \nameref{subsubsec:Properties of Cumulative Distribution Functions}.
					\end{remark}
				\end{definition}
				\begin{definition}[Conditional Probability Density Function] \label{def:1 Discrete 1 Continuous-Conditional PDF}
					The \emph{conditional Probability Distribution Function (Conditional PDF)} of $Y$ given $X=x$ is
					\begin{equation} \label{eq:1 Discrete 1 Continuous-Conditional PDF}
						f_{Y} \left( y \Given x \right) = \frac{d}{dy} F_{Y} \left( y \Given x \right)
					\end{equation}
					This also means,
					\begin{equation*}
						P \left[ Y \leq y \Given X=x \right]
						= \int_{y \in A} f_{Y} \left( y \Given x \right) dy
					\end{equation*}
					\begin{remark}
						The \nameref{def:1 Discrete 1 Continuous-Conditional PDF} of 1 discrete random variable and 1 continuous random variable satisfies all \nameref{subsubsec:Properties of Probability Density Functions}.
					\end{remark}
				\end{definition}
			
			\subsubsection{2 Continuous Random Variables} \label{subsubsec:2 Continuous Random Variables}
				\begin{definition}[Conditional Cumulative Distribution Function] \label{def:2 Continuous-Conditional CDF}
					The \emph{conditional Cumulative Distribution Function (Conditional CDF)} of $Y$ given $X=x$ for $X$ and $Y$ continuous random variables is:
					\begin{equation} \label{eq:2 Continuous-Conditional CDF}
						F_{Y} \left(y \Given x \right)
						= \lim\limits_{h\rightarrow 0} F_{Y} \left( y \Given x < X \leq \left( x+h \right) \right)
						= \frac{\int_{-\infty}^{y} f_{X,Y} \left( x,v \right) dv}{f_{X} \left( x \right)}
					\end{equation}
					\begin{remark}
						The \nameref{def:2 Continuous-Conditional CDF} of 2 continuous random variables satisfies all \nameref{subsubsec:Properties of Cumulative Distribution Functions}.
					\end{remark}
					\begin{remark}
						The similar relations for the conditional and marginal probability functions do not hold up for 2 continuous random variables too well.
					\end{remark}
				\end{definition}
			
				\begin{definition}[Conditional Probability Density Function] \label{def:2 Continuous-Conditional PDF}
					The \emph{conditional Probability Density Function (Conditional PDF)} of $Y$ given $X=x$ for $X$ and $Y$ continuous random variables is:
					\begin{equation} \label{eq:2 Continuous-Conditional PDF}
						f_{Y} \left( y \Given x \right)
						= \frac{d}{dy} F_{Y} \left( y \Given x \right)
						= \frac{f_{X,Y} \left( x,y \right)}{f_{X} \left( x \right)}
					\end{equation}
					\begin{remark}
						If $X$ and $Y$ are independent, then:
						\begin{equation} \label{eq:2 Continuous-Independent Conditional PDF}
							f_{X} \left( y \Given x \right)
							= \frac{f_{X,Y} \left( x,y \right)}{f_{X} \left( x \right)}
							= \frac{f_{X} \left( x \right) f_{Y} \left( y \right)}{f_{X} \left( x \right)}
							= f_{Y} \left( y \right)
						\end{equation}
					\end{remark}
					\begin{remark}
						The \nameref{def:2 Continuous-Conditional PDF} of 2 continuous random variables satisfies all \nameref{subsubsec:Properties of Probability Density Functions}.
					\end{remark}
				\end{definition}
			
	\subsection{Conditional Expectation of Multiple Random Variables} \label{subsec:Conditional Expectation of Multiple Variables}
		\begin{definition}[Conditional Expectation] \label{def:Conditional Expectation of Multiple Variables}
			The \emph{conditional expectation} of $Y$ given $X$ is:
			\begin{equation} \label{eq:Conditional Expectation of Multiple Variables}
				\ExpectedValue \left[ Y \Given X=x \right] = \int_{-\infty}^{\infty} y \cdot f_{Y} \left( y \Given x \right) dy
			\end{equation}
			\begin{remark}[Special Case]
				There is a special case when \emph{\textbf{both}} $X$ and $Y$ are discrete random variables.
				\begin{equation} \label{eq:Conditional Expectation of Multiple Discrete Variables}
					\ExpectedValue \left[ Y \Given X=x \right] = \sum_{y \in S_{Y}} y \cdot p_{Y} \left( y \Given x \right)
				\end{equation}
			\end{remark}
			\begin{remark}
				When calculating the \nameref{subsec:Conditional Expectation of Multiple Variables}, and they as for $\ExpectedValue \left[ Y \Given X=x \right]$, that means you \emph{\textbf{must}} consider all possible values that $X$ can take.
				This can be generalized to the equation below.
				\begin{equation} \label{eq:General Conditional Expectation of Multiple Variables}
					\ExpectedValue \left[ Y \Given X=x \right] = \sum_{x \in S_{X}} \left( \sum_{y \in S_{Y}} y \cdot p_{Y} \left( y \Given x \right) \right)
				\end{equation}
				This can be described. You must take a single value for $x$, and take it over all $y$'s, then take the next value for $x$, until you have exhausted all values in both $S_{X}$ and $X_{Y}$. \newline
				This can also be translated into the continuous case, but the discrete case is a little simpler to understand this generality.
			\end{remark}
			\begin{remark}
				$\ExpectedValue \left[ Y \Given X=x \right]$ is a function of $X$, so it can be written as $g \left( x \right) = \ExpectedValue \left[ Y \Given X=x \right]$.
				Thus, we can also say
				\begin{equation} \label{eq:Expected Value of Conditional Expected Value of Multiple Variables}
					\ExpectedValue \left[ g \left( x \right) \right] = \ExpectedValue \left[ \ExpectedValue \left[ Y \Given X \right] \right] = \ExpectedValue \left[ Y \right]
				\end{equation}
				\begin{subequations}
					\begin{align} \label{eq:Joint PDF of Multiple Continuous Random Variables}
						\ExpectedValue \left[ Y \right]
						= \ExpectedValue \left[ \ExpectedValue \left[ Y \Given X \right] \right]
						&= \int_{-\infty}^{\infty} \ExpectedValue \left[ Y \Given x \right] f_{X} \left( x \right) dx
						= \int_{-\infty}^{\infty} \int_{-\infty}^{\infty} y f_{Y} \left( y \Given x \right) dy f_{X} \left( x \right) dx \\
					 \label{eq:Joint PDF of Multiple Discrete Random Variables}
						\ExpectedValue \left[ Y \right]
						= \ExpectedValue \left[ \ExpectedValue \left[ Y \Given X \right] \right]
						&= \sum_{x \in S_{X}} \ExpectedValue \left[ Y \Given x \right] p_{X} \left( x \right)
						= \sum_{x_{j} \in S_{X}} \sum_{y_{i} \in S_{Y}} y_{i} p_{Y} \left( y_{i} \Given x_{k} \right) p_{X} \left( x_{j} \right)
					\end{align}
				\end{subequations}
			\end{remark}
		\end{definition}
		\begin{proof}[Prove Expectation of Conditional Expected Value] \label{proof:Expected Value of Conditional Expected Value of Multiple Variables}
		\end{proof}

\section{Random Vectors} \label{sec:Random Vectors}
Random Vectors are usually denoted:
	\begin{equation} \label{eq:Random Vector Notation}
		\vec{X} = \langle X_{1}, X_{2} X_{3}, \ldots, X_{n} \rangle
	\end{equation}
	\begin{definition}[Random Vector] \label{def:Random Vector}
		A \emph{random vector} is a list of \nameref{def:Random Variable, Full}s.
		\begin{remark}
			Almost all of the material for \nameref{sec:Multiple Random Variables} is applicable here.
			However, the 2 random variable equations and definitions must be generalized to $n$ random variables.
		\end{remark}
	\end{definition}
	
	\subsection{Joint CDF of a Random Vector} \label{subsec:Joint CDF of Random Vector}
		\begin{equation} \label{eq:Joint CDF of Random Vector}
			\begin{aligned}
				F_{\vec{X}} \left( \vec{x} \right) 
					&= F_{X_{1}, X_{2}, X_{3}, \ldots, X_{n}} \left( x_{1}, x_{2}, x_{3}, \ldots, x_{n} \right) \\
					&= P \left[ X_{1} \leq x_{1}, X_{2} \leq x_{2}, X_{3} \leq x_{3}, \ldots, X_{n} \leq x_{n} \right] \\
			\end{aligned}
		\end{equation}
		
	\subsection{Joint PDF of a Random Vector} \label{subsec:Joint PDF of Random Vector}
		\begin{equation} \label{eq:Joint PDF of Random Vector}
			f_{\vec{X}} \left( \vec{x} \right) = \frac{\partial^{n} F_{\vec{X}} \left( \vec{x} \right)}{\partial x_{1} \partial x_{2} \partial x_{3} \cdots \partial x_{n}}
		\end{equation}
		
		\subsubsection{Marginal PDF of a Random Vector} \label{subsubsec:Marginal PDF of Random Vector}
		Integrate out the terms that you're not interested in.
		\begin{equation} \label{eq:Marginal PDF of Random Vector}
			f_{\vec{X}} = \int_{-\infty}^{\infty} \cdots \int_{-\infty}^{\infty} f_{\vec{X}} \left( \vec{x} \right) \partial x_{2} \partial x_{3} \cdots \partial x_{n}
		\end{equation}
		For instance, say we want the marginal PDF of some function with respect to $X_{1}$, $X_{3}$, and $X_{4}$.
		\begin{equation} \label{eq:Marginal PDF of Random Vector Multiple Variables}
			f_{X_{1}, X_{3}, X_{4}} \left( x_{1}, x_{3}, x_{4} \right) = \int_{-\infty}^{\infty} \cdots \int_{-\infty}^{\infty} f_{\vec{X}} \left( \vec{x} \right) \partial x_{2} \partial x_{5} \partial x_{6} \cdots \partial x_{n}
		\end{equation}
	
	\subsection{Conditional Probability Functions of Random Vectors} \label{subsec:Random Vector Conditional Probability Functions}
	This section is just an extension of Section~\ref{subsec:Multiple Variable Conditional Probability Functions}, \nameref{subsec:Multiple Variable Conditional Probability Functions}.
	There are 3 major cases for these:
		\begin{enumerate}[noitemsep, nolistsep]
			\item \nameref{subsubsec:Conditional Probability Discrete Random Vectors}
			\item \nameref{subsubsec:Conditional Probability Mixed Random Vectors}
			\item \nameref{subsubsec:Conditional Probability Continuous Random Vectors}
		\end{enumerate}
	
		\begin{remark*} \label{rmk:Define Random Vector Y for Example}
			\begin{large}
				For the sections below, let $\vec{Y}= \langle Y_{1},Y_{2},Y_{3} \rangle$ and $\vec{y}= \langle y_{1},y_{2},y_{3} \rangle$.
			\end{large} \newline
			While I am using $\vec{Y}$ and $\vec{y}$, these equations can be further generalized to higher dimensions.
			All that would be required for this is to keep track of everything.
		\end{remark*}
	
		\subsubsection{Discrete Random Vectors} \label{subsubsec:Conditional Probability Discrete Random Vectors}
			\begin{definition}[Conditional Probability Mass Function] \label{def:Discrete Random Vector-Conditional PMF}
				The \emph{conditional Probability Mass Function (Conditional PMF)} of $Y_{3}$ given that $Y=y$ is:
				\begin{equation} \label{eq:Discrete Random Vector-Conditional PMF}
					p_{Y_{3}} \left( y_{3} \Given y_{1},y_{2} \right)
					= \frac{P \left[ \lbrace Y_{3}=y_{3} \rbrace \cap \left( \lbrace Y_{1}=y_{1} \rbrace \cap \lbrace Y_{2}=y_{2} \rbrace \right) \right]}{P \left[ \lbrace Y_{1}=y_{1} \rbrace \cap \lbrace Y_{2}=y_{2} \rbrace \right]} 
					= \frac{p_{\vec{Y}} \left( \vec{y} \right)}{p_{Y_{1},Y_{2}} \left( y_{1},y_{2} \right)}
				\end{equation}
				\begin{remark}
					This also implies that
					\begin{equation} \label{eq:Discrete Random Vector-Joint PMF}
						p_{\vec{Y}} \left( \vec{y} \right) = p_{Y_{3}} \left( y_{3} \Given y_{1},y_{2} \right) \cdot p_{Y_{2}} \left( y_{2} \Given y_{1} \right) \cdot p_{Y_{1}} \left( y_{1} \right)
					\end{equation}
				\end{remark}
				\begin{remark}
					If all elements of $\vec{Y}$ are \emph{independent} (Remember that you need to check each subgroup too, like shown in Section~\ref{subsec:Event Independence}), then:
					\begin{equation} \label{eq:Discrete Random Vector-Independent Conditional PMF}
						p_{Y_{3}} \left( y_{3} \Given y_{1},y_{2} \right)
						= \frac{p_{\vec{Y}} \left( \vec{y} \right)}{p_{Y_{1},Y_{2}} \left( y_{1},y_{2} \right)}
						= \frac{p_{Y_{1},Y_{2}} \left( y_{1},y_{2} \right) p_{Y_{3}} \left( y_{3} \right)}{p_{Y_{1},Y_{2}} \left( y_{1},y_{2} \right)}
						= p_{Y_{3}} \left( y_{3} \right)
					\end{equation}
				\end{remark}
				\begin{remark}
					The \nameref{def:Discrete Random Vector-Conditional PMF} of 2 discrete random variables satisfies all \nameref{subsubsec:Properties of Probability Mass Functions}.
				\end{remark}
			\end{definition}
		
		\subsubsection{Mixed Random Vectors} \label{subsubsec:Conditional Probability Mixed Random Vectors}
		
		
		\subsubsection{Continuous Random Vectors} \label{subsubsec:Conditional Probability Continuous Random Vectors}
		
	\subsection{Mean Vector} \label{subsec:Mean Vector}
		\begin{definition}[Mean Vector] \label{def:Mean Vector}
			For $\vec{X} = \langle X_{1},X_{2},\ldots,X_{n} \rangle$, the \emph{mean vector} is defined as the column vector of expected values of the components of $X_{k}$:
			\begin{equation} \label{eq:Mean Vector}
				\mathbf{m_{X}}
				= \ExpectedValue \left[ \vec{X} \right]
				= \begin{bmatrix}
					X_{1} \\
					X_{2} \\
					\vdots \\
					X_{n}
				\end{bmatrix}
				\triangleq \begin{bmatrix}
					\ExpectedValue \left[ X_{1} \right] \\
					\ExpectedValue \left[ X_{2} \right] \\
					\vdots \\
					\ExpectedValue \left[ X_{n} \right] \\
				\end{bmatrix}
			\end{equation}
			\begin{remark}
				Note that we defined the vector of expected values as a column vector.
				Other texts will use row vectors for other things, but the use of column vectors here is intentional.
			\end{remark}
		\end{definition}
	
	\subsection{Correlation and Covariance Matrix} \label{subsec:Correlation and Correlation Matrix}
		\begin{definition}[Correlation Matrix]
			The \emph{correlation matrix} has the second moments of $\bar{X}$ as its entries:
			\begin{equation} \label{eq:Correlation Matrix}
				\mathbf{\bar{R}_{X}}
				= \begin{bmatrix}
					\ExpectedValue \left[ X_{1}^{2} \right] & \ExpectedValue \left[ X_{1}X_{2} \right] & \cdots & \ExpectedValue \left[ X_{1}X_{n} \right] \\
					\ExpectedValue \left[ X_{2}X_{1} \right] & \ExpectedValue \left[ X_{2}^{2} \right] & \cdots & \ExpectedValue \left[ X_{2}X_{n} \right] \\
					\vdots & \vdots & \ddots & \vdots \\
					\ExpectedValue \left[ X_{n}X_{1} \right] & \ExpectedValue \left[ X_{n}X_{2} \right] & \cdots & \ExpectedValue \left[ X_{n}^{2} \right] \\
				\end{bmatrix} \\
			\end{equation}
			\begin{remark}
				$\bar{R}_{X}$ is a $n \times n$ symmetric matrix.
			\end{remark}
		\end{definition}
		\begin{definition}[Covariance Matrix] \label{def:Covariance Matrix}
			The \emph{covariance matrix} has the second-order central moments as its entries:
			\begin{equation} \label{eq:Covariance Matrix}
				\begin{aligned}
				\mathbf{\bar{K}_{X}}
				&= \begin{bmatrix}
					\ExpectedValue \left[ \left( X_{1} - m_{1} \right)^{2} \right] & \ExpectedValue \left[ \left( X_{1}-m_{1} \right) \left( X_{2}-m_{2} \right) \right] & \cdots & \ExpectedValue \left[ \left( X_{1}-m_{1} \right) \left( X_{n}-m_{n} \right) \right] \\
					\ExpectedValue \left[ \left( X_{2}-m_{2} \right) \left( X_{1}-m_{1} \right) \right] & \ExpectedValue \left[ \left( X_{2}-m_{2} \right)^{2} \right] & \cdots & \ExpectedValue \left[ \left( X_{2}-m_{2} \right) \left( X_{n}-m_{n} \right) \right] \\
					\vdots & \vdots & \ddots & \vdots \\
					\ExpectedValue \left[ \left( X_{n}-m_{n} \right) \left( X_{1}-m_{1} \right) \right] & \ExpectedValue \left[ \left( X_{n}-m_{n} \right) \left( X_{2}-m_{2} \right) \right] & \cdots & \ExpectedValue \left[ \left( X_{n}-m_{n} \right)^{2} \right] \\
				\end{bmatrix} \\
				&= \begin{bmatrix}
				\Variance \left[ X_{1} \right] & \Covariance \left[X_{1},X_{2} \right] & \cdots & \Covariance \left[ X_{1},X_{n} \right] \\
				\Covariance \left[ X_{2},X_{1} \right] & \Variance \left[ X_{2} \right]  & \cdots & \Covariance \left[ X_{2},X_{n} \right] \\
				\cdots & \cdots & \ddots & \cdots \\
				\Covariance \left[ X_{n},X_{1} \right] & \Covariance \left[ X_{2},X_{n} \right] & \cdots & \Variance \left[ X_{n} \right] \\
				\end{bmatrix}
				\end{aligned}
			\end{equation}
			\begin{remark}
				$\bar{K}_{X}$ is a $n \times n$ symmetric matrix.
			\end{remark}
			\begin{remark}
				The diagonal elements of $\bar{K}_{X}$ are given by the variances $\Variance \left[ X_{k} \right] = \ExpectedValue \left[ \left( X_{k}-m_{k} \right)^{2} \right]$ of the elements of $\vec{X}$.
			\end{remark}
			\begin{remark}
				If the diagonal elements of $\bar{K}_{X}$ are \nameref{rmk:Uncorrelated}, then $\Covariance \left[ X_{j}, X_{k} \right] = 0$ for $j \neq k$, and $\bar{K}_{X}$, the \nameref{def:Covariance Matrix} is a diagonal matrix.
			\end{remark}
			\begin{remark}
				If the random variables $X_{1},X_{2},\cdots,X_{n}$ are independent, then they are uncorrelated and $\bar{K}_{X}$ is diagonal.
			\end{remark}
			\begin{remark}
				If the \nameref{def:Mean Vector} is $\bar{0}$, that is, $m_{k} = \ExpectedValue \left[ X_{k} \right] = 0$ for all $k$, then $\bar{R}_{X} = \bar{K}_{X}$.
			\end{remark}
		\end{definition}

%====================================APPENDIX====================================
\clearpage
\appendix
\counterwithin{equation}{section}

\section{Reference Material} \label{sec:Reference Material}
\subsection{Trigonometry} \label{app:Trig}
	\subsubsection{Trigonometric Formulas} \label{subsubsec:Trig Formulas}
		\begin{equation} \label{eq:Sin plus Sin with diff Angles}
			\sin \left( \alpha \right) + \sin \left( \beta \right) = 2 \sin \left( \frac{\alpha + \beta}{2} \right) \cos\left( \frac{\alpha - \beta}{2} \right)  
		\end{equation}
		\begin{equation} \label{eq:Cosine-Sine Product}
			\cos \left( \theta \right) \sin \left( \theta \right) = \frac{1}{2} \sin \left( 2 \theta \right)
		\end{equation}
	
	\subsubsection{Euler Equivalents of Trigonometric Functions} \label{subsubsec:Euler Equivalents}
		\begin{equation} \label{eq:Euler Sin}
			\sin \left( x \right) = \frac{e^{\imath x} + e^{-\imath x}}{2}
		\end{equation}
		\begin{equation} \label{eq:Euler Cos}
			\cos \left( x \right) = \frac{e^{\imath x} - e^{-\imath x}}{2 \imath}
		\end{equation}
		\begin{equation} \label{eq:Euler Sinh}
			\sinh \left( x \right) = \frac{e^{x} - e^{-x}}{2}
		\end{equation}
		\begin{equation} \label{eq:Euler Cosh}
			\cosh \left( x \right) = \frac{e^{x} + e^{-x}}{2}
		\end{equation}
\section{Calculus}\label{app:Calculus}
\subsection{L'Hopital's Rule}\label{subsec:LHopitals_Rule}
L'Hopital's Rule can be used to simplify and solve expressions regarding limits that yield irreconcialable results.
\begin{lemma}[L'Hopital's Rule]\label{lemma:LHopitals_Rule}
  If the equation
  \begin{equation*}
    \lim\limits_{x \rightarrow a} \frac{f(x)}{g(x)} =
    \begin{cases}
      \frac{0}{0} \\
      \frac{\infty}{\infty} \\
    \end{cases}
  \end{equation*}
  then \Cref{eq:LHopitals_Rule} holds.
  \begin{equation}\label{eq:LHopitals_Rule}
    \lim\limits_{x \rightarrow a} \frac{f(x)}{g(x)} = \lim\limits_{x \rightarrow a} \frac{f'(x)}{g'(x)}
  \end{equation}
\end{lemma}

\subsection{Fundamental Theorems of Calculus}\label{subsec:Fundamental Theorem of Calculus}
\begin{definition}[First Fundamental Theorem of Calculus]\label{def:1st Fundamental Theorem of Calculus}
  The \emph{first fundamental theorem of calculus} states that, if $f$ is continuous on the closed interval $\left[ a,b \right]$ and $F$ is the indefinite integral of $f$ on $\left[ a,b \right]$, then

  \begin{equation}\label{eq:1st Fundamental Theorem of Calculus}
    \int_{a}^{b}f \left( x \right) dx = F \left( b \right) - F \left( a \right)
  \end{equation}
\end{definition}

\begin{definition}[Second Fundamental Theorem of Calculus]\label{def:2nd Fundamental Theorem of Calculus}
  The \emph{second fundamental theorem of calculus} holds for $f$ a continuous function on an open interval $I$ and $a$ any point in $I$, and states that if $F$ is defined by

  \begin{equation*}
    F \left( x \right) = \int_{a}^{x} f \left( t \right) dt,
  \end{equation*}
  then
  \begin{equation}\label{eq:2nd Fundamental Theorem of Calculus}
    \begin{aligned}
      \frac{d}{dx} \int_{a}^{x} f \left( t \right) dt &= f \left( x \right) \\
      F' \left( x \right) &= f \left( x \right) \\
    \end{aligned}
  \end{equation}
\end{definition}

\begin{definition}[argmax]\label{def:argmax}
  The arguments to the \emph{argmax} function are to be maximized by using their derivatives.
  You must take the derivative of the function, find critical points, then determine if that critical point is a global maxima.
  This is denoted as
  \begin{equation*}\label{eq:argmax}
    \argmax_{x}
  \end{equation*}
\end{definition}

\subsection{Rules of Calculus}\label{subsec:Rules of Calculus}
\subsubsection{Chain Rule}\label{subsubsec:Chain Rule}
\begin{definition}[Chain Rule]\label{def:Chain Rule}
  The \emph{chain rule} is a way to differentiate a function that has 2 functions multiplied together.

  If
  \begin{equation*}
    f(x) = g(x) \cdot h(x)
  \end{equation*}
  then,
  \begin{equation}\label{eq:Chain Rule}
    \begin{aligned}
      f'(x) &= g'(x) \cdot h(x) + g(x) \cdot h'(x) \\
      \frac{df(x)}{dx} &= \frac{dg(x)}{dx} \cdot g(x) + g(x) \cdot \frac{dh(x)}{dx} \\
    \end{aligned}
  \end{equation}
\end{definition}

\subsection{Useful Integrals}\label{subsec:Useful_Integrals}
\begin{equation}\label{eq:Cosine_Indefinite_Integral}
  \int \cos(x) \; dx = \sin(x)
\end{equation}

\begin{equation}\label{eq:Sine_Indefinite_Integral}
  \int \sin(x) \; dx = -\cos(x)
\end{equation}

\begin{equation}\label{eq:x_Cosine_Indefinite_Integral}
  \int x \cos(x) \; dx = \cos(x) + x \sin(x)
\end{equation}
\Cref{eq:x_Cosine_Indefinite_Integral} simplified with Integration by Parts.

\begin{equation}\label{eq:x_Sine_Indefinite_Integral}
  \int x \sin(x) \; dx = \sin(x) - x \cos(x)
\end{equation}
\Cref{eq:x_Sine_Indefinite_Integral} simplified with Integration by Parts.

\begin{equation}\label{eq:x_Squared_Cosine_Indefinite_Integral}
  \int x^{2} \cos(x) \; dx = 2x \cos(x) + (x^{2} - 2) \sin(x)
\end{equation}
\Cref{eq:x_Squared_Cosine_Indefinite_Integral} simplified by using Integration by Parts twice.

\begin{equation}\label{eq:x_Squared_Sine_Indefinite_Integral}
  \int x^{2} \sin(x) \; dx = 2x \sin(x) - (x^{2} - 2) \cos(x)
\end{equation}
\Cref{eq:x_Squared_Sine_Indefinite_Integral} simplified by using Integration by Parts twice.

\begin{equation}\label{eq:Exponential_Cosine_Indefinite_Integral}
  \int e^{\alpha x} \cos(\beta x) \; dx = \frac{e^{\alpha x} \bigl( \alpha \cos(\beta x) + \beta \sin(\beta x) \bigr)}{\alpha^{2} + \beta^{2}} + C
\end{equation}

\begin{equation}\label{eq:Exponential_Sine_Indefinite_Integral}
  \int e^{\alpha x} \sin(\beta x) \; dx = \frac{e^{\alpha x} \bigl( \alpha \sin(\beta x) - \beta \cos(\beta x) \bigr)}{\alpha^{2}+\beta^{2}} + C
\end{equation}

\begin{equation}\label{eq:Exponential_Indefinite_Integral}
  \int e^{\alpha x} \; dx = \frac{e^{\alpha x}}{\alpha}
\end{equation}

\begin{equation}\label{eq:x_Exponential_Indefinite_Integral}
  \int x e^{\alpha x} \; dx = e^{\alpha x} \left( \frac{x}{\alpha} - \frac{1}{\alpha^{2}} \right)
\end{equation}
\Cref{eq:x_Exponential_Indefinite_Integral} simplified with Integration by Parts.

\begin{equation}\label{eq:Inverse_x_Indefinite_Integral}
  \int \frac{dx}{\alpha + \beta x} = \int \frac{1}{\alpha + \beta x} \; dx = \frac{1}{\beta} \ln (\alpha + \beta x)
\end{equation}

\begin{equation}\label{eq:Inverse_x_Squared_Indefinite_Integral}
  \int \frac{dx}{\alpha^{2} + \beta^{2} x^{2}} = \int \frac{1}{\alpha^{2} + \beta^{2} x^{2}} \; dx = \frac{1}{\alpha \beta} \arctan \left( \frac{\beta x}{\alpha} \right)
\end{equation}

\begin{equation}\label{eq:a_Exponential_Indefinite_Integral}
  \int \alpha^{x} \; dx = \frac{\alpha^{x}}{\ln(\alpha)}
\end{equation}

\begin{equation}\label{eq:a_Exponential_Derivative}
  \frac{d}{dx} \alpha^{x} = \frac{d\alpha^{x}}{dx} = \alpha^{x} \ln(x)
\end{equation}

\subsection{Leibnitz's Rule}\label{subsec:Leibnitzs_Rule}
\begin{lemma}[Leibnitz's Rule]\label{lemma:Leibnitzs_Rule}
  Given
  \begin{equation*}
    g(t) = \int_{a(t)}^{b(t)} f(x, t) \, dx
  \end{equation*}
  with $a(t)$ and $b(t)$ differentiable in $t$ and $\frac{\partial f(x, t)}{\partial t}$ continuous in both $t$ and $x$, then
  \begin{equation}\label{eq:Leibnitzs_Rule}
    \frac{d}{dt} g(t) = \frac{d g(t)}{dt} = \int_{a(t)}^{b(t)} \frac{\partial f(x, t)}{\partial t} \, dx + f \bigl[ b(t), t \bigr] \, \frac{d b(t)}{dt} - f \bigl[ a(t), t \bigr] \, \frac{d a(t)}{dt}
  \end{equation}
\end{lemma}




\end{document}