\documentclass[10pt,letterpaper,final,twoside,notitlepage]{article}
\usepackage[margin=.5in]{geometry}
\usepackage[utf8]{inputenc}
\usepackage[english]{babel}
\usepackage{amsmath}
\usepackage{amsfonts}
\usepackage{amssymb}
\usepackage{amsthm} % Gives us plain, definition, and remark to use in \theoremstyle{style}
\usepackage{graphicx}

\usepackage{hyperref} % Generate hyperlinks to referenced items
\usepackage{nameref} % Can make references by name to places
\usepackage{indentfirst} % Indents the first line of new paragraphs
\usepackage{ctable} % Greater control over tables and how they look
\usepackage{subcaption} % Allows for multiple figures in one Figure environment
\usepackage{gensymb} % Gives access to some characters, for example, the degree symbol
\usepackage{enumitem} % Provides [noitemsep, nolistsep] for more compact lists
\usepackage{chngcntr} % Allows us to tamper with the counter a little more
\usepackage{empheq} % Allow boxing of equations in special math environments

%\graphicspath{{./Drawings/Math_374}} % Uncomment this to use pictures in this document
%\numberwithin{equation}{section} % Uncomment this to number equations with section numbers too

\theoremstyle{plain}
\newtheorem{theorem}{Theorem}

\theoremstyle{definition}
\newtheorem{definition}{Defn}
%\counterwithin{definition}{subsection} % Uncomment to have definitions use section/subsection numbering
	

\author{Karl Hallsby}
\title{Reference Material}

\begin{document}
\section{Relative Frequency} \label{sec:Relative Frequency}
\begin{itemize}[noitemsep, nolistsep]
	\item $f_k (n) = \frac{N_k (n)}{n}$ $\leftarrow$ {\large \textbf{Relative Frequency}}
	\begin{itemize}[noitemsep, nolistsep]
		\item $k$ is the outcome
		\item $N_k (n)$ is the number of times outcome $k$
	\end{itemize}
	\item $\lim\limits_{n \rightarrow \infty} f_k (n) = p_k$ $\leftarrow$ {\large \textbf{Statistical Regularity}}
	\begin{itemize}[noitemsep, nolistsep]
		\item $p_k$ is the probability of event $k$ occurring
	\end{itemize}
\end{itemize}

	\subsection{Properties of Relative Frequencies} \label {subsec:Properties Relative Frequency}
	\begin{enumerate}[noitemsep, nolistsep]
		\item $f_k (n) = \frac{N_k (n)}{n}$
		\item $0 \leq N_k (n) \leq n$
		\item $0 \leq f_k (n) \leq 1 = \frac{0}{n} \leq \frac{N_k (n)}{n} \leq \frac{n}{n}$
		\item $\sum_{k=1}^{k} f_k (n) = \sum_{k=1}^{k} \frac{N_k (n)}{n} = \frac{\sum_{k=1}^{k} N_k (n)}{n} = \frac{n}{n} = 1$
		\item $\sum_{k=1}^{k} f_k (n) = 1$
		\item If events A and B are disjoint and event C is "A or B", then $F_C = F_A (n) + F_B (n)$
	\end{enumerate}

\section{Set Theory} \label{sec:Set Theory}
\begin{itemize}[noitemsep, nolistsep]
	\item A \emph{set} is a collection of objects, denoted by capital letters
	\item Denote the \emph{universal set, $U$}; consisting of all possible objects of interest in a given setting/application
	\item For any set $A$, we say that \emph{``$x$ is an element of $A$''}, denoted $x \in A$ if object $x$ of the universal set $U$ is contained in $A$
	\item We say that \emph{``$x$ is not an element of $A$''}, denoted $x \notin A$ if object $x$ of the universal set $U$ is not contained in $A$
	\item We say that \emph{``$A$ is a subset of $B$''}, denoted $A \subset B$ if every element in $A$ also belongs to $B$, $x \in A \rightarrow x \in B$
	\item The \emph{empty set, $\emptyset$} is defined as the set with no elements
		\begin{itemize}[noitemsep, nolistsep]
			\item The empty set is a subset of every set
		\end{itemize}
	\item Sets \emph{$A$ and $B$ are equal} if they contain the same elements. To show this:
		\begin{enumerate}[noitemsep, nolistsep]
			\item Enumerate the elements of each set
			\item Thm: $A=B \iff A \subset B$ AND $B \subset A$
		\end{enumerate}
	\item The \emph{union of 2 sets $A$, $B$}, denoted $A \cup B$ is defined as the set of outcomes that are either in $A$, or in $B$, or both
	\item The \emph{intersection fo 2 sets, $A$, $B$}, denoted $A \cap B$ is defined as the set of outcomes in $A$ and $B$
	\item The 2 sets $A$, $B$ are said to be \emph{disjoint or mutually exclusive} if $A \cap B = \emptyset$
	\item The \emph{complement of a set $A$}, denoted $A^{C}$ is defined as the set of elements of $U$ not in $A$
		\begin{itemize}[noitemsep, nolistsep]
			\item $A^{C} = \lbrace x \in U \vert x \notin A \rbrace$
		\end{itemize}
	\item \emph{Relative complement} or \emph{difference}, denoted $A-B$, is the set of elements in $A$ that are not in $B$
		\begin{itemize}[noitemsep, nolistsep]
			\item $A-B = A \cap B^{C}$
			\item $A^{C} = U - A$
		\end{itemize}
\end{itemize}

	\subsection{Properties of Set Operations} \label{subsec:Properties of Set Ops}
	\begin{equation}
		\begin{aligned}
			A \cup B &= B \cup A \\
			A \cap B &= B \cap A \\
		\end{aligned}
	\end{equation}

\section{Probability Theory} \label{sec:Probability Theory}
There are 3 main components to \nameref{sec:Probability Theory}.
\begin{enumerate}[noitemsep, nolistsep]
	\item Set Theory
	\item Axioms of Probability
	\item Conditional Probability and Independence
\end{enumerate}

	\subsection{Random Experiments} \label{subsec:Random Experiments}
	\begin{definition}[Random Experiment] \label{def:Random Experiment}
		A \emph{random experiment} is an experiment whose outcome varies in an unpredictable fashion when performed under the same conditions.
	\end{definition}
	\begin{definition}[Sample Space] \label{def:Sample Space}
		A \emph{sample space, $S$} of a random experiment is the set of all possible experiments.
	\end{definition}
	\begin{definition}[Outcome/Sample Point] \label{def:Outcome}
		An \emph{outcome}, or \emph{sample point} of a random experiment is a result that cannot be decomposed into other results.
	\end{definition}
	\begin{definition}[Event] \label{def:Event}
		An \emph{event} corresponds to a subset of the sample space. We say an event occurs if and only if (iff) the outcome of the experiment is in the subset representing the event.
	\end{definition}


\end{document}