1\documentclass[10pt,letterpaper,final,twoside,notitlepage]{article}
\usepackage[margin=.5in]{geometry}
\usepackage[utf8]{inputenc}
\usepackage[english]{babel}
\usepackage{amsmath}
\usepackage{amsfonts}
\usepackage{amssymb}
\usepackage{amsthm} % Gives us plain, definition, and remark to use in \theoremstyle{style}
\usepackage{graphicx}

\usepackage{hyperref} % Generate hyperlinks to referenced items
\usepackage[noabbrev,nameinlink]{cleveref} % Fancy cross-references in the document everywhere
\usepackage{nameref} % Can make references by name to places
\usepackage{subcaption} % Allows for multiple figures in one Figure environment
\usepackage{siunitx} % Gives us ways to typeset units for stuff
\usepackage{enumitem} % Provides [noitemsep, nolistsep] for more compact lists
\usepackage{chngcntr} % Allows us to tamper with the counter a little more
\usepackage{empheq} % Allow boxing of equations in special math environments
\usepackage{tcolorbox} % Allows us to create boxes of various types for examples
\usepackage{tikz} % Allows us to create TikZ and PGF Pictures
\usetikzlibrary{trees}
%\usepackage{ctable} % Greater control over tables and how they look

%\graphicspath{{./Drawings/Math_374}} % Uncomment this to use pictures in this document
\counterwithin{equation}{section} % Uncomment to number eqns with sec nums too

\theoremstyle{plain}
\newtheorem{theorem}{Theorem}
\counterwithin{theorem}{section}

\theoremstyle{definition}
\newtheorem{definition}{Defn}
\newtheorem{corollary}{Corollary}[section]

\theoremstyle{remark}
\newtheorem{remark}{Remark}[definition]
\newtheorem*{remark*}{Remark}
%\counterwithin{definition}{subsection} % Uncomment to have definitions use section.subsection numbering

% Create a special list that handles properties. It can be broken and restarted
\newlist{propertylist}{enumerate}{1} % {Name}{Template}{Max Depth}
\setlist[propertylist, 1]{label=\textbf{(\roman*)}, noitemsep, nolistsep} % Set options

% Create a special list that handles enumerate starting with lower letters. Breakable/Restartable.
\newlist{boldalphlist}{enumerate}{1} % {Name}{Template}{Max Depth}
\setlist[boldalphlist, 1]{label=\textbf{(\alph*)}, ref=\alph*, noitemsep, nolistsep} % Set options

% Create a tcolorbox for examples
% Argument #1 is optional, given by [], that is the textbook's problem number
% Argument #2 is mandatory, given by {}, that is the title for the example
\newtcolorbox[auto counter,
	number within=section,
	number format=\arabic,
	crefname={example}{examples}, % Define reference format for cref (No Capitals)
	Crefname={Example}{Examples}, % Reference format for cleveref (With Capitals)
]{example}[2][]{ % The [2][] Means the first argument is optional
	width=\textwidth,
	title={Example \thetcbcounter: #2. #1},
	fonttitle=\bfseries,
	label={ex:#2},
	nameref=#2,
	colbacktitle=white!100!black,
	coltitle=black!100!white,
	colback=white!100!black,
	upperbox=visible,
	lowerbox=visible,
	sharp corners=all
}

% Redefine the 'end of proof' symbol to be a black square, not blank
\renewcommand\qedsymbol{$\blacksquare$} % Change proofs to have black square at end

% Math Operators that are useful to abstract the written math away to one spot
\DeclareMathOperator{\EventClass}{\mathcal{F}}
\DeclareMathOperator{\RealNums}{\mathbb{R}}
\DeclareMathOperator{\Prob}{\operatorname{P}}
\DeclareMathOperator{\ExpectedValue}{\operatorname{\mathbb{E}}}
\DeclareMathOperator{\Variance}{\operatorname{VAR}}
\DeclareMathOperator{\StdDev}{\operatorname{STD}}
\DeclareMathOperator{\Covariance}{\operatorname{Cov}}
\DeclareMathOperator{\DrawnIID}{\overset{\text{iid}}{\DrawnFrom}}
\DeclareMathOperator{\Bias}{\operatorname{B}}
\DeclareMathOperator{\MeanSqErr} {\operatorname{MSE}}
\DeclareMathOperator{\Likelihood}{\mathcal{L}}
\DeclareMathOperator*{\argmax}{argmax} % Thin Space and subscripts are UNDER in display
\DeclareMathOperator{\MaxLikeEstim}{\operatorname{MLE}}

\DeclareMathOperator{\Given}{\vert}
\DeclareMathOperator{\DrawnFrom}{\sim}

\author{Karl Hallsby}
\title{Reference Material}

\begin{document}
\pagenumbering{roman} % i, ii, iii on beginning pages, that don't have content
\tableofcontents
\clearpage
\pagenumbering{arabic} % 1, 2, 3 on content pages

\section{Probability Models} \label{sec:Probability Models}
	\subsection{Relative Frequency} \label{subsec:Relative Frequency}
		\begin{definition}[Relative Frequency] \label{def:Relative Frequency}
			\emph{Relative frequency} is defined in Equation~\eqref{eq:Relative Frequency}:
			\begin{equation} \label{eq:Relative Frequency}
				f_k (n) = \frac{N_k (n)}{n}
			\end{equation}
			\begin{itemize}[noitemsep, nolistsep]
				\item $k$ is the outcome
				\item $N_k (n)$ is the number of times outcome $k$
			\end{itemize}
		\end{definition}
	
		\subsubsection{Properties of Relative Frequencies} \label {subsec:Properties Relative Frequency}
		\begin{enumerate}[label=\textbf{(\roman*)}, noitemsep, nolistsep]
			\item 
				\begin{equation}
					f_k (n) = \frac{N_k (n)}{n}
				\end{equation}
			\item
				\begin{equation}
					0 \leq N_k (n) \leq n
				\end{equation}
			\item
				\begin{equation}
					0 \leq f_k (n) \leq 1 = \frac{0}{n} \leq \frac{N_k (n)}{n} \leq \frac{n}{n}
				\end{equation}
			\item
				\begin{equation}
					\sum\limits_{k=1}^{k} f_k (n) = \sum\limits_{k=1}^{k} \frac{N_k (n)}{n} = \frac{\sum\limits_{k=1}^{k} N_k (n)}{n} = \frac{n}{n} = 1
				\end{equation}
			\item
				\begin{equation}
					\sum\limits_{k=1}^{k} f_k (n) = 1
				\end{equation}
			\item If events $A$ and $B$ are disjoint and event $C$ is "$A$ or $B$", then 
				\begin{equation}
					F_C = F_A (n) + F_B (n)
				\end{equation}
		\end{enumerate}
	\subsection{Statistical Regularity} \label{subsec:Statistical Regularity}
		\begin{definition} \label{def:Statistical Regularity}
			The averages obtained in long sequences of trials that lead to approximately the same value have a property called \emph{statistical regularity}.
			This is defined in Equation~\eqref{eq:Statistical Regularity}.
			\begin{equation} \label{eq:Statistical Regularity}
				\lim\limits_{n \rightarrow \infty} f_k (n) = p_k
			\end{equation}
			\begin{itemize}[noitemsep, nolistsep]
				\item $p_k$ is the probability of event $k$ occurring
			\end{itemize}
		\end{definition} % Section 1

\section{Set Theory} \label{sec:Set Theory}
\begin{enumerate}[noitemsep, nolistsep]
	\item A \emph{set} is a collection of objects, denoted by capital letters
	\item Denote the \emph{universal set, $U$}; consisting of all possible objects of interest in a given setting/application
	\item For any set $A$, we say that \emph{``$x$ is an element of $A$''}, denoted $x \in A$ if object $x$ of the universal set $U$ is contained in $A$
	\item We say that \emph{``$x$ is not an element of $A$''}, denoted $x \notin A$ if object $x$ of the universal set $U$ is not contained in $A$
	\item We say that \emph{``$A$ is a subset of $B$''}, denoted $A \subset B$ if every element in $A$ also belongs to $B$, $x \in A \rightarrow x \in B$
	\item The \emph{empty set, $\emptyset$} is defined as the set with no elements
		\begin{itemize}[noitemsep, nolistsep]
			\item The empty set is a subset of every set
		\end{itemize}
	\item Sets \emph{$A$ and $B$ are equal} if they contain the same elements. To show this:
		\begin{enumerate}[noitemsep, nolistsep]
			\item Enumerate the elements of each set
			\item Thm: $A=B \iff A \subset B$ AND $B \subset A$
		\end{enumerate}
	\item The \emph{union of 2 sets $A$, $B$}, denoted $A \cup B$ is defined as the set of outcomes that are either in $A$, or in $B$, or both
	\item The \emph{intersection fo 2 sets, $A$, $B$}, denoted $A \cap B$ is defined as the set of outcomes in $A$ and $B$
	\item The 2 sets $A$, $B$ are said to be \emph{disjoint or mutually exclusive} if $A \cap B = \emptyset$
	\item The \emph{complement of a set $A$}, denoted $A^{C}$ is defined as the set of elements of $U$ not in $A$
		\begin{itemize}[noitemsep, nolistsep]
			\item $A^{C} = \lbrace x \in U \vert x \notin A \rbrace$
		\end{itemize}
	\item \emph{Relative complement} or \emph{difference}, denoted $A-B$, is the set of elements in $A$ that are not in $B$
		\begin{itemize}[noitemsep, nolistsep]
			\item $A-B = A \cap B^{C}$
			\item $A^{C} = U - A$
		\end{itemize}
\end{enumerate}

	\subsection{Properties of Set Operations} \label{subsec:Properties of Set Ops}
	Set Operators are:
	\begin{enumerate}
		\item Commutative, Equation~\eqref{eq:Set Ops-Commutative}
			\begin{equation} % Commutative
				\begin{aligned}
					A \cup B &= B \cup A \\
					A \cap B &= B \cap A \\
				\end{aligned}
				\label{eq:Set Ops-Commutative}
			\end{equation}
			
			\item Associative, Equation~\eqref{eq:Set Ops-Associative}
				\begin{equation} % Associative
					\begin{aligned}
						A \cup \left( B \cup C \right) &= \left( A \cup B \right) \cup C \\
						A \cap \left( B \cap C \right) &= \left( A \cap B \right) \cap C \\
					\end{aligned}
					\label{eq:Set Ops-Associative}
				\end{equation}
			
			\item Distributive, Equation~\eqref{eq:Set Ops-Distributive}
				\begin{equation} % Distributive
					\begin{aligned}
						A \cup \left( B \cap C \right) &= \left( A \cup B \right) \cap \left( A \cup C \right) \\
						A \cap \left( B \cup C \right) &= \left( A \cap B \right) \cup \left( A \cap C \right) \\
					\end{aligned}
					\label{eq:Set Ops-Distributive}
				\end{equation}
			
			\item Set Operations obey De Morgan's Laws, Equation~\eqref{eq:Set Ops-De Morgan's}
				\begin{equation} % De Morgan's
					\begin{aligned}
						\left( A \cup B \right)^{C} &= A^{C} \cap B^{C} \\
						\left( A \cap B \right)^{C} &= A^{C} \cup B^{C} \\
					\end{aligned}
					\label{eq:Set Ops-De Morgan's}
				\end{equation}
			
	\end{enumerate}
	Additionally, 
	\begin{definition}[Union of $n$ Sets] \label{def:Union of n Sets}
		The \emph{union of $n$ sets} $\bigcup\limits_{k=1}^{n} A_{k} = A_{1} \cup A_{2} \cup A_{3} \cup \ldots \cup A_{n}$ is the set consisting of all elements such that $x \in A_{k}$ for some $1 \leq k \leq n$.
		\begin{itemize}[noitemsep, nolistsep]
			\item All sets need to be empty to make $\bigcup\limits_{k=1}^{n} A_{k} = \emptyset$
		\end{itemize}
	\end{definition}
	\begin{definition}[Intersection of $n$ Sets] \label{def:Intersection of n Sets}
		The \emph{intersection of $n$ sets} $\bigcap\limits_{k=1}^{n} A_{k} = A_{1} \cap A_{2} \cap A_{3} \cap \ldots \cap A_{n}$ is the set consisting of all elements such that $x \in a_{k}$ for all $1 \leq k \leq n$
		\begin{itemize}[noitemsep, nolistsep]
			\item Just one set needs to be empty to make $\bigcap\limits_{k=1}^{n} A_{k} = \emptyset$
		\end{itemize}
	\end{definition} % Section 2

\section{Probability Theory} \label{sec:Probability Theory}
There are 3 main components to \nameref{sec:Probability Theory}.
\begin{enumerate}[noitemsep, nolistsep]
	\item \nameref{sec:Set Theory}
	\item \nameref{subsec:Probability Law Corollary}
	\item \nameref{subsec:Conditional Probability}~and~\nameref{subsec:Event Independence}
\end{enumerate}

	\subsection{Random Experiments} \label{subsec:Random Experiments}
	\begin{definition}[Random Experiment] \label{def:Random Experiment}
		A \emph{random experiment} is an experiment whose outcome varies in an unpredictable fashion when performed under the same conditions.
	\end{definition}
	\begin{definition}[Sample Space] \label{def:Sample Space}
		A \emph{sample space, $S$} of a random experiment is the set of all possible experiments.
	\end{definition}
	\begin{definition}[Outcome/Sample Point] \label{def:Outcome}
		An \emph{outcome}, or \emph{sample point} of a random experiment is a result that cannot be decomposed into other results.
	\end{definition}
	\begin{definition}[Event] \label{def:Event}
		An \emph{event} corresponds to a subset of the sample space. We say an event occurs if and only if (iff) the outcome of the experiment is in the subset representing the event.
	\end{definition}
	\begin{definition}[Event Classes] \label{def:Event Classes}
		An \emph{event class} $\EventClass$ is the collection of the all the events' sets. $\EventClass$ should be closed under unions, intersections, and complements.
		\begin{itemize}[noitemsep, nolistsep]
			\item For $S$ finite, or countably infinite, then we can let $\EventClass$ be all subsets of $S$.
			\item For $S$ uncountably infinite, instead we can let $\EventClass$ consist of the subsets that can be obtained as countable unions and intersections of some sets of $\EventClass$.
		\end{itemize}
	\end{definition}
	\begin{definition}[Probability Law] \label{def:Probability Law}
		A \emph{probability law} for a random experiment $E$, with sample space $S$, and an event class $\EventClass$ is a rule that assigns to each event $A \in \EventClass$ a number $P \left[A \right]$, called the probability of $A$ that satisfies the axioms:
		\begin{enumerate}[label=Axiom~\Roman*:, align=left, noitemsep, nolistsep] \label{subdef:Probability Law Axioms}
			\item $0 \leq P\left[ A \right]$
			\item $P \left[ S \right] = 1$
			\item If $A \cap B = \emptyset$, then $P \left[ A \cup B \right] = P \left[ A \right] + P \left[ B \right]$
			\item[Axiom III':] If $A_{1}$, $A_{2}$, $\ldots$ is a sequence of events such that $A_{i} \cap A_{j} = \emptyset$ for all $i \neq j$, then $P \left[ \bigcup_{k=1}^{\infty} A_{k} \right] = \sum_{k=1}^{\infty} P \left[ A_{k} \right]$
		\end{enumerate}
	\end{definition}
	
	\subsection{Probability Law Corollaries} \label{subsec:Probability Law Corollary}
		\begin{enumerate}[label=Axiom~\Roman*:, align=left, noitemsep, nolistsep] % Probability Law Axioms
			\item $0 \leq P\left[ A \right]$
			\item $P \left[ S \right] = 1$
			\item If $A \cap B = \emptyset$, then $P \left[ A \cup B \right] = P \left[ A \right] + P \left[ B \right]$
			\item[Axiom III':] If $A_{1}$, $A_{2}$, $\ldots$ is a sequence of events such that $A_{i} \cap A_{j} = \emptyset$ for all $i \neq j$, then $P \left[ \bigcup_{k=1}^{\infty} A_{k} \right] = \sum_{k=1}^{\infty} P \left[ A_{k} \right]$
		\end{enumerate}
		\begin{corollary} \label{cor:Probability Parts}
			$P \left[ A^{C} \right] = 1 - P \left[ A \right]$
		\end{corollary}
		\begin{corollary} \label{cor:Probability of Event}
			$P \left[ A \right] \leq 1$
		\end{corollary}
		\begin{corollary} \label{cor:Probability of Empty Set}
			$P \left[ \emptyset \right] = 0$
		\end{corollary}
		\begin{corollary} \label{cor: Probability Addition of Disjoint Pairs}
			If $A_{1}$, $A_{2}$, $\ldots$, $A_{n}$ are pairwise mutually exclusive ($A_{1} \cap A_{2} \cap \ldots \cap A_{n} = \emptyset$), then $P \left[ \bigcup_{k=1}^{n} \right] = \sum_{k=1}^{n} P \left[ A_{k} \right]$ for $n \geq 2$
		\end{corollary}
		\begin{corollary} \label{cor:Inclusion-Exclusion Principle to 2 Sets}
			$P \left[ A \cup B \right] = P \left[ A \right] + P \left[ B \right] - P \left[ A \cap B \right]$
		\end{corollary}
		\begin{corollary} \label{cor:Inclusion-Exclusion Principle to n Sets}
			$P \left[ A \cup B \right] = \sum\limits_{j=1}^{n} P \left[ A_{j} \right] - \sum_{j<k} P \left[A_{j} \cap A_{k} \right] + \ldots + \left( -1 \right)^{n+1} P \left[ A_{1} \cap \ldots \cap A_{n} \right]$
		\end{corollary}
		\begin{corollary} \label{cor:Subset Probability to Superset}
			If $A \subset B$, then $P \left[ A \right] \leq P \left[ B \right]$
		\end{corollary}
	
	\subsection{Conditional Probability} \label{subsec:Conditional Probability}
		\begin{definition}[Conditional Probability] \label{def:Conditional Probability}
			The \emph{conditional probability} of event $A$ \textbf{GIVEN THAT} event $B$ occurred is denoted $P \left[ A \vert B \right]$ and is defined as
			\begin{equation} \label{eq:Conditional Probability}
				P \left[ A \vert B \right] = \frac{P \left[ A \cap B \right]}{P \left[ B \right]}
			\end{equation}
		\end{definition}
		\begin{theorem}[Theorem of Total Probability] \label{thm:Theorem of Total Probability}
			Let $B_{1}$, $B_{2}$, $\ldots$, $B_{n}$ be mutually exclusive events whose union equals the sample space $S$, i.e. $B_{1}$, $B_{2}$, $\ldots$, $B_{n}$ is a partition of $S$.
		\end{theorem}
		\begin{definition}[Baye's Rule] \label{def:Baye's Rule}
			Let $B_{1}$, $B_{2}$, $\ldots$, $B_{n}$ be a partition of sample space $S$.
			\begin{equation}
				P \left[ B_{j} \vert A \right] = \frac{P \left[ A \cap B_{j} \right]}{P \left[ A \right]}
				= \frac{P \left[ A \vert B_{j} \right] * P \left[ B_{j} \right]}{\sum\limits_{k=1}^{n} P \left[ A \vert B_{k} \right] * P \left[ B_{k} \right]}
			\end{equation}
		\end{definition}
		\begin{example}[Exam 1, Problem 5]{Baye's Rule}
			An urn contains 9 balls, identical in every way, except that they are labeled with numbers 1 through 9.
			Two balls are selected at random, without replacement, and the sequence of labels observed are recorded.
			\begin{enumerate}[label=(\alph*), noitemsep, nolistsep]
				\item Give the formula for the conditional probability of event $A$ given that event $B$ occurred (where $A$ and $B$ are arbitrary events).
				\item What is the probability that the label of the second ball is even?
				\item What is the probability that the label of the first ball was odd given that the second was even?
			\end{enumerate}
		\end{example}
	
	\subsection{Event Independence} \label{subsec:Event Independence}
		\begin{definition}[Independent] \label{def:Event Independence}
			Two events $A$ and $B$ are \emph{independent} if 
			\begin{equation} \label{eq:Event Independence}
				P \left[ A \cap B \right] = P \left[ A \right] * P \left[ B \right], P\left[ A \right] \neq 0, P\left[ B \right] \neq 0
			\end{equation}
			\begin{itemize}[noitemsep, nolistsep]
				\item If $A \cap B = \emptyset$, the $A$ and $B$ are \textbf{dependent}.
				\item If checking for independence between more than 2 events, you must check each pair, each triple, etc. until you check the independence of each event against each other. For 3 events, $A$, $B$, $C$:
					\begin{itemize}[noitemsep, nolistsep]
						\item Check $P \left[ A \cap B \cap C \right] = P \left[ A \right] * P \left[ B \right] * P \left[ C \right]$
						\item Also need to check:
							\begin{enumerate}[noitemsep, nolistsep]
								\item $P \left[ A \cap B \right] = P \left[ A \right] * P \left[ B \right]$
								\item $P \left[ B \cap C \right] = P \left[ B \right] * P \left[ C \right]$
								\item $P \left[ A \cap C \right] = P \left[ A \right] * P \left[ C \right]$
							\end{enumerate}
					\end{itemize}
			\end{itemize}
		\end{definition}
		\begin{example}[Exam 1, Problem 4]{Event Independence}
			Let $S=\lbrace 1,2,3,4 \rbrace$, and $A = \lbrace 1,2 \rbrace$, $B=\lbrace 1,3 \rbrace$, $C = \lbrace 1,4 \rbrace$, $D = \lbrace 3,4 \rbrace$.
			Assume the outcomes are equiprobable.
			Are the following events independent?
				\begin{enumerate}[noitemsep, nolistsep]
					\item $A$ and $B$
					\item $A$ and $D$
					\item $A$, $B$, and $C$
				\end{enumerate}
		\end{example}
	If 2 events $A$ and $B$ are independent, then their complements are also independent. This is shown in \nameref{proof:Independence of Complements of Events}.
		\begin{proof}[Independence of Complements of Events] \label{proof:Independence of Complements of Events}
			We assumed that $A$ and $B$ were independent, so $P \left[ A \cap B \right] = P \left[ A \right] \cdot P \left[ B \right]$.
			There are 2 more facts we will need:
			\begin{enumerate}[leftmargin=1.0in, label=Fact \arabic*: , ref=Fact \arabic*, noitemsep, nolistsep]
				% leftmargin sets a distance for left margin
				% ref sets the way items will be cross-referenced, and can differ from the label.
				\item $P \left[ B \right] + P \left[ B^{C} \right] = 1$ \label{proof:Independence of Complements of Events:Fact 1}
				\item $P \left[ A \cap B^{C} \right] + P \left[ A \cap B \right] = P \left[ A \right]$ \label{proof:Independence of Complements of Events:Fact 2}
			\end{enumerate}
			From \ref{proof:Independence of Complements of Events:Fact 1}, we have:
			\begin{equation*}
				P \left[ A \cap B \right] = P \left[ A \right] \cdot \left( 1-P \left[ B^{C} \right] \right)
			\end{equation*}
			From \ref{proof:Independence of Complements of Events:Fact 2}, we have $P \left[ A \cap B \right] = P \left[ A \right] - P \left[ A \cap B^{C} \right]$.
			Substituting these into the equation above:
			\begin{align*}
				P \left[ A \right] - P \left[ A \cap B^{C} \right] &= P \left[ A \right] \cdot \left( 1-P \left[ B^{C} \right] \right)\\
				P \left[ A \right] - P \left[ A \cap B^{C} \right] &= P \left[ A \right] - P \left[ A \right] \cdot P \left[ B^{C} \right] \\
				- P \left[ A \cap B^{C} \right] &= -P \left[ A \right] \cdot P \left[ B^{C} \right] \\
				P \left[ A \cap B^{C} \right] &= P \left[ A \right] \cdot P \left[ B^{C} \right] \\
			\end{align*}
			$\therefore$ $A$ and $B^{C}$ are independent, according to the definition of \nameref{def:Event Independence}~events in \Cref{eq:Event Independence}.
		\end{proof} % Section 3

\section{Counting} \label{sec:Counting}
	\subsection{Sampling \emph{with} Replacement \emph{with} Order} \label{subsec:Ordered Sampling with Replacement}
		\begin{definition} \label{def:Ordered Sampling with Replacement}
			Choose $k$ elements in succession with replacement between selections, from a population of $n$ distinct objects, where $k$ needs to have no relation to $n$.
			\begin{equation} \label{eq:Ordered Sampling with Replacement}
				\frac{n}{First} * \frac{n}{Second} * \frac{n}{Third} * \ldots * \frac{n}{kth \text{ Item}} = n^{k}
			\end{equation}
		\end{definition}
		\begin{example}[Problem 2.42]{Sampling with Replacement with Order}
			A lock has two buttons: a ``0'' button and a ``1'' button.
			To open a door you need to push the buttons according to a preset 8-bit sequence.
			How many sequences are there?
			Suppose you press an arbitrary 8-bit sequence; what is the probability that the door opens? If the first try does not succeed in opening the door, you try another number; what is the probability of success?
			
			\tcblower
			
			Solution.
		\end{example}
	
	\subsection{Sampling \emph{without} Replacement \emph{with} Order} \label{subsec:Ordered Sampling without Replacement}
		\begin{definition} \label{def:Ordered Sampling without Replacement}
			Choose $k$ elements in succession without replacement from a population of $n$ distinct objects, where $k \leq n$
			\begin{equation} \label{eq:Ordered Sampling without Replacement}
				\frac{n}{First} * \frac{n-1}{Second} * \frac{n-2}{Third} * \ldots * \frac{n-k+1}{kth \text{ Item}}
			\end{equation}
		\end{definition}
		
		\subsubsection{Permutations} \label{subsubsec:Permutations}
			\begin{definition}[Permutation] \label{def:Permutation}
				\emph{Permutations} are special cases of \nameref{subsec:Ordered Sampling without Replacement}, where $k = n$
				\begin{equation} \label{eq:Permutation}
					\frac{n}{First} * \frac{n-1}{Second} * \frac{n-2}{Third} * \ldots * \frac{2}{} * \frac{1}{} * \ldots * \frac{n-k-1}{kth \text{ Item}} = n!
				\end{equation}
			\end{definition}
			\begin{example}[Exam 1, Problem 2]{Password Permutations}
				\begin{enumerate}[noitemsep, nolistsep]
					\item How many unique case-sensitive passswords, of length 8 characters, can be constructed using number (0-9), lower and upper-case letters (a-z, A-Z), and a set of 15 special characters and no spaces?
					\item How many unique passwords if the user must also use at least one integer?
				\end{enumerate}
			
				\tcblower
				
				Solution.
			\end{example}
		
	\subsection{Sampling \emph{with} Replacement \emph{without} Order} \label{subsec:Unordered Sampling with Replacement}
		\begin{definition} \label{def:Unordered Sampling without Replacement}
			Pick $k$ objects from a set of $n$ distinct object with replacement.
			Record the result without order.
			The total number of ways to do this is given in \Cref{eq:Unordered Sampling without Replacement}.
			\begin{equation} \label{eq:Unordered Sampling without Replacement}
				\binom{n+k-1}{k} = \binom{n+k-1}{n-1}
			\end{equation}
		\end{definition}
	
	\subsection{Sampling \emph{without} Replacement \emph{without} Ordering} \label{subsec:Unordered Sampling without Replacement}		
		\begin{definition} \label{def:Unordered Sampling with Replacement}
			Pick $k$ objects from a set of $n$ distinct objects without replacement.
			Record the results with without order.
			We call the resulting subset of $k$ selected objects a ``combination of size $k$.''
			The number of ways to choose $k$ items out of $n$ items is given in \Cref{eq:Unordered Sampling with Replacement}.
			Also said $n$ choose $k$:
			\begin{equation} \label{eq:Unordered Sampling with Replacement}
			\binom{n}{k} = \frac{n * (n-1) * (n-2) * \ldots * (n-k+1)}{k!} = \frac{n!}{k! \left( n-k \right)!}
			\end{equation}
		\end{definition}
		\begin{equation}
			\binom{n}{k} = \binom{n}{n-k}
		\end{equation} % Section 4

\section{Single Discrete Random Variables} \label{sec:Single Discrete Random Variables}
	\begin{definition}[Random Variable] \label{def:Random Variable, Simple}
		A \emph{random variable} $X$ is a function that assigns a real number $X \left( \zeta \right)$ to each outcome $\zeta$ in the sample space of the random experiment.
	\end{definition}
	\begin{definition}[Discrete Random Variable] \label{def:Discrete Random Variable}
		A \emph{discrete random variable} is a random variable that assumes values in a countable set. For example, the number of heads in 3 coin flips is a discrete random variable.
	\end{definition}

	\subsection{Probability Mass Function  (PMF)} \label{subsec:Probability Mass Function}
		\begin{definition}[Probability Mass Function] \label{def:Probability Mass Function}
			The \emph{probability mass function (PMF)} of a discrete random variable $X$ is defined as:
			\begin{equation} \label{eq:Probability Mass Function}
				p_{X} \left( x \right) = P \left[ X=x \right]
			\end{equation}
			Using the coin example from the definition of a~\nameref{def:Discrete Random Variable},
			\begin{equation}
				p_{X} \left( x \right) = 
				\begin{cases}
					\frac{1}{8} & x=0 \\
					\frac{3}{8} & x=1 \\
					\frac{3}{8} & x=2 \\
					\frac{1}{8} & x=3 \\
				\end{cases}
			\end{equation}
		\end{definition}
		\begin{example}[Problem 3.2]{Probability Mass Function}
		A die is tossed and the random variable $X$ is defined as the number of full pairs of dots on the fact showing up.
			\begin{boldalphlist}[label=\textbf{(\alph*)}, ref=(\alph*)]
				\item Describe the underlying space $S$ of this random experiment and specify the probabilities of the elementary events. \label{ex:3.2a}
				\item Show the mapping from $S$ to $S_{X}$, the range of $S$ \label{ex:3.2b}
				\item Find the probabilities for the various values of $X$. \label{ex:3.2c}
				\item Repeat parts \ref{ex:3.2a}, \ref{ex:3.2b}, and \ref{ex:3.2c} if $Y$ is the number of full or partial pairs of dots in the face showing up.
				\item Explain why $\Prob \left[ X = 0 \right]$ and $\Prob \left[ Y = 0 \right]$ are not equal.
			\end{boldalphlist}
		
		\tcblower
		
		Solution from Homework 3.
		\end{example}
		
		\subsubsection{Properties of Probability Mass Functions} \label{subsubsec:Properties of Probability Mass Functions}
			\begin{propertylist}%[label=\textbf{(\roman*)}, noitemsep, nolistsep]
				\item
					\begin{equation}
						p_{X} \left( x \right) \geq 0 \text{, } \forall x \in \RealNums
					\end{equation}
				\item
					\begin{equation}
						\sum\limits_{x \in S_{X}} p_{X} \left( x \right) = 1
					\end{equation}
			\end{propertylist}
					\begin{example}[Problem 3.13]{Find Normalizing Constant}
					Let $X$ be a random varaible with PMF $p_{k} = \frac{c}{k^{2}}$ for $k = 1,2,\ldots$
						\begin{enumerate}[label=\textbf{(\alph*)}]
							\item Estimate the value of $c$ numerically. Note that the series converges.
							\item Find $\Prob \left[ X > 4 \right]$.
							\item Find $\Prob \left[ 6 \leq X \leq 8 \right]$.
						\end{enumerate}
					
					\tcblower
					
						Solution from Homework 4.
					\end{example}
			\begin{propertylist}[resume]
				\item
					\begin{equation}
						P \left[ x \in B \right] = \sum\limits_{x \in B} p_{X} \left( x \right) \text{, where } B \subset S_{X}
					\end{equation}
			\end{propertylist}
		
	\subsection{Expected Value/Mean of Single Discrete Random Variable} \label{subsec:Expected Value of Single Discrete}
		\begin{definition}[Expected Value/Mean of Single Discrete Random Variable] \label{def:Expected Value of Single Discrete}
			The \emph{expected value} or \emph{mean} of a single discrete random variable $X$ is defined by
			\begin{equation} \label{eq:Expected Value of Single Discrete}
				m_{X} = \ExpectedValue \left[ X \right] = \sum_{x \in S_{X}} x \cdot p_{X} \left( x \right)
			\end{equation}
			\begin{remark} \label{rmk:Expected Value of Single Discrete Countably Infinite}
				If $X$ is countably infinite, you will have an infinite series that exists only if
				\begin{equation} \label{eq:Expected Value of Single Discrete Countably Infinite}
					\sum_{s \in S_{X}} \lvert x \rvert \cdot p_{X} \left( x \right)
				\end{equation}
				is absolutely convergent.
			\end{remark}
		\end{definition}
		\begin{example}[Problem 3.27]{Expectation of Discrete Random Variable}
			Find the expectation of $X$ where the PMF of $X$ is \[p_{k} = \frac{\frac{pi^{2}}{6}}{k^{2}}\]. (Note that this PMF is the same as in \Cref{ex:Find Normalizing Constant}).
			
			\tcblower
			
			Solution from Homework 4. ONLY THE Expectation PART.
		\end{example}
	
		\subsubsection{Properties of Expected Values} \label{subsubsec:Properties of Discrete Expected Value}
			\begin{definition}[Linearity of Expectation] \label{def:Linearity of Expectation}
				Let $Y = X_{1} + X_{2}$
				\begin{equation}
					\ExpectedValue \left[ X \right] = \ExpectedValue \left[ X_{1} \right] + \ExpectedValue \left[ X_{2} \right]
				\end{equation}
				This can be generalized to
				\begin{equation} \label{eq:General Linearity of Expectation}
					\ExpectedValue \left[ \sum_{i=1}^{k} x_{i} \right] = \sum_{i=1}^{k} \ExpectedValue \left[ X_{i} \right]
				\end{equation}
			\end{definition}
			\begin{propertylist}
				\item
					\begin{equation}
						\ExpectedValue \left[ X_{1} + X_{2} \right] = \ExpectedValue \left[ X_{1} \right] + \ExpectedValue \left[ X_{2} \right]
					\end{equation}
				\item
					\begin{equation}
						\ExpectedValue \left[ g \left( X \right) \right] = \sum\limits_{s \in S_{X}} g \left( x \right) \cdot p_{X} \left[ X \right]
					\end{equation}
				\item
					\begin{equation}
						\ExpectedValue \left[ c g\left( X \right) \right] = c \ExpectedValue \left[ g\left( X \right) \right]
					\end{equation}
				\item
					\begin{equation}
						\ExpectedValue \left[ g_{1} \left( X \right) + g_{2} \left( X \right) + \ldots + g_{m} \left( X \right) \right] = \sum\limits_{i=1}^{m} \ExpectedValue \left[ g_{i} \left( X \right) \right]
					\end{equation}
			\end{propertylist}
		
		\subsubsection{Moments of Random Variable} \label{subsubsec:Moments of a Random Variable}
			\begin{definition}[Moment]
				The \emph{moment} of a random variable, $X$ is defined as the expectation of the random variable raised to the moment.
				\begin{equation} \label{eq:Moments of a Random Variable}
					\begin{aligned}
						\ExpectedValue \left[ X^{1} \right] &= \text{First Moment} \\
						\ExpectedValue \left[ X^{2} \right] &= \text{Second Moment} \\
						&\vdots \\
						\ExpectedValue \left[ X^{k} \right] &= \text{kth Moment} \\
					\end{aligned}
				\end{equation}
			\end{definition}					
	
	\subsection{Variance of Single Discrete Random Variable} \label{subsec:Variance of Single Discrete}
		\begin{definition}[Variance] \label{def:Variance of Single Discrete}
			The \emph{variance} of a single discrete random variable $X$ is defined as:
			\begin{equation} \label{eq:Variance of Single Discrete-Form 1}
				\ExpectedValue \left[ \left( X - \ExpectedValue \left[ X \right] \right)^{2} \right]
			\end{equation}
			\begin{equation} \label{eq:Variance of Single Discrete-Form 2}
				\Variance \left[ X \right] = \ExpectedValue \left[ X^{2} \right] - \left( \ExpectedValue \left[ X \right] \right)^{2}
			\end{equation}
			and is denoted as $\sigma_{X}^{2}$, or as the operator $\Variance \left[ X \right]$.
			\begin{remark} \label{rmk:Constant in Variance}
				If $X$ is a random variable, and $c$ is some constant coefficient, then:
				\begin{equation}
					\Variance \left[ cX \right] = c^{2} \Variance \left[ X \right]
				\end{equation}
			\end{remark}
		\end{definition}
		\begin{example}[Problem 3.27]{Variance of Discrete Random Variable}
			Find the variance of $X$ where the PMF of $X$ is \[p_{k} = \frac{\frac{pi^{2}}{6}}{k^{2}}\]. (Note that this PMF is the same as in \Cref{ex:Find Normalizing Constant}).
			
			\tcblower
			
			Solution from Homework 4. ONLY THE Variance PART.
		\end{example}
		\begin{definition}[Standard Deviation] \label{def:Standard Deviation}
			The standard deviation of a random variable $X$ is:
			\begin{equation} \label{eq:Standard Deviation}
				\sigma_{X} = \sqrt{\Variance \left[ X \right]}
			\end{equation}
		\end{definition}
	
	\subsection{Conditional Probability Mass Function} \label{subsec:Conditional Probability Mass Function}
		\begin{definition}[Conditional Probability Mass of Function] \label{def:Conditional Probability Mass Function}
			Let $X$ be a discrete random variable, with PMF $p_{X} \left( x \right)$ and let $C$ be the event with non-zero probability, i.e. $P \left[ C \right] > 0$.
			The \emph{conditional probability mass function of $X$ given $C$ (Conditional PMF)} is defined as:
			\begin{equation} \label{eq:Conditional Probability Mass Function}
				p_{X \Given C} \left( x \Given C \right) = P \left[ X=x \Given C \right] \text{ for } x \in \RealNums
			\end{equation}
			\begin{remark} \label{rmk:Properties of Conditional Probability Mass Functions}
				The conditional PMF, $p_{X \Given C} \left( x \Given C \right)$, satisfies \emph{\textbf{all}} properties of \nameref{subsubsec:Properties of Probability Density Functions}.
			\end{remark}
		\end{definition}
	
	\subsection{Conditional Expected Value of Single Discrete Random Variable} \label{subsec:Conditional Expected Value of Single Discrete}
		\begin{definition}[Conditional Expected Value of Discrete Random Variable] \label{def:Conditional Expected Value of Single Discrete}
			The \emph{conditional expected value of the discrete random variable} $X$ given $B$ is defined as:
			\begin{equation} \label{eq:Conditional Expected Value of Single Discrete}
				m_{X \Given B} = \ExpectedValue \left[ X \Given B \right] = \sum\limits_{x \in S_{X}} s \cdot p_{X} \left( x \Given B \right)
			\end{equation}
		\end{definition}
		\begin{example}[Problem 3.39]{Conditional Expected Value}
			\begin{boldalphlist}
				\item Find the conditional expected value of $X$ in Problem 3.5 given that no message gets through in the first time slow. SHow that $\ExpectedValue \left[ X \Given X>1 \right] = \ExpectedValue \left[ X \right] + 1$. \label{ex:3.39a}
				\item Find the conditional expected value of $X$ in problem 3.5 given that a message gets through in the first time slot. \label{ex:3.39b}
				\item Find $\ExpectedValue \left[ X \right]$ by using the results of Parts \ref{ex:3.39a} and \ref{ex:3.39b}. \label{ex:3.39c}
				\item Find $\ExpectedValue \left[ X^{2} \right]$ and $\Variance \left[ X \right]$ using the approach in parts \ref{ex:3.39b} and \ref{ex:3.39c}.
			\end{boldalphlist}
		
			\tcblower
			
			Solution from Homework 5.
		\end{example}
	
	\subsection{Conditional Variance of Single Discrete Random Variable} \label{subsec:Conditional Variance of Single Discrete}
		\begin{definition}[Conditional Variance of Discrete Random Variable] \label{def:Conditional Variance of Single Discrete}
			The \emph{conditional variance of a discrete random variable} $X$ given event $B$ as defined as:
			\begin{equation} \label{eq:Conditional Variance of Single Discrete}
				\begin{aligned}
					\sigma_{X \Given B}^{2} &= \Variance \left[ X \Given B \right] \\
						&= \ExpectedValue \left[ \left( X - \ExpectedValue \left[ X \Given B \right] \right)^{2} \Given B \right] \\
						&= \sum\limits_{x \in S_{X}} \left( x - m_{X \Given B} \right)^{2} \cdot p_{X} \left( x \Given B \right) \\
					\Variance \left[ X \Given B \right] &= \ExpectedValue \left[ X^{2} \Given B \right] - \left( \ExpectedValue \left[ X \Given B \right] \right)^{2} \\
				\end{aligned}
			\end{equation}
		\end{definition} % Section 5

\section{Single Continuous Random Variables} \label{sec:Single Continuous Random Variables}
	\begin{definition}[Random Variable] \label{def:Random Variable, Full}
		Consider a random experiment with sample space $S$ and event class $\EventClass$.
		A \emph{random variable} $X$ is a function from the sample space $S$ to the real line $\RealNums$ with the property the set $A_{b} = \lbrace \zeta: X \vert \zeta \leq b \rbrace$ is in $\EventClass$ for every $b$ in $\RealNums$.
	\end{definition}
	\begin{definition}[Continuous Random Variable] \label{def:Continuous Random Variable}
		A \emph{continuous random variable} is a random variable whose \nameref{subsec:Cumulative Distribution Function} is continuous everywhere.
	\end{definition}

	\subsection{Cumulative Distribution Function (CDF)} \label{subsec:Cumulative Distribution Function}
		\begin{definition}[Cumulative Distribution Function] \label{def:Cumulative Distribution Function}
			\emph{Cumulative Distribution Function (CDF)} of a random variable $X$ is defined as the probability of the event $\lbrace X \leq x \rbrace$.
			\begin{equation} \label{eq:Cumulative Distribution Function}
				F_{X} \left( x \right) = \Prob \left[ X \leq x \right] \text{ for } -\infty < x < \infty
			\end{equation}
		\end{definition}
	
		\subsubsection{Properties of Cumulative Distribution Functions} \label{subsubsec:Properties of Cumulative Distribution Functions}
			\begin{propertylist}
				\item 
					\begin{equation}
						0 \leq F_{X} \left( x \right) \leq 1
					\end{equation}
				\item If you include the whole sample space, you should end up with $1$.
					\begin{equation}
						\lim\limits_{x \rightarrow \infty} F_{X} \left( x \right) = 1
					\end{equation}
				\item If you exclude the whole sample space, you should end up with $0$.
					\begin{equation}
						\lim\limits_{x \rightarrow -\infty} F_{X} \left( x \right) = 0
					\end{equation}
				\item $F_{X} \left( x \right)$ is non-decreasing.
					\begin{equation}
						F_{X} \left( a \right) \leq F_{X} \left( b \right) \text{ if } a \leq b
					\end{equation}
				\item The CDF is continuous from the right.
					\begin{equation}
				 		F_{X} \left( b \right) = \lim\limits_{h \rightarrow 0} F_{X} \left( b+h \right) \text{ where } h>0
					\end{equation}
				\item
					\begin{equation}
						\Prob \left[ a<X \leq b \right] = F_{X} \left( b \right) - F_{X} \left( a \right)
					\end{equation}
				\item The probability at a point in a CDF. (This usually ends up being $0$).
					\begin{equation}
						\Prob \left[ X=b \right] = F_{X} \left( b \right) - F_{X} \left( b^{-} \right)
					\end{equation}
				\item The probability of the event \emph{\textbf{not}} occurring.
					\begin{equation}
						\Prob \left[ X>x \right] = 1 - \Prob \left[ X \leq x \right] =  1 - F_{X} \left( x \right)
					\end{equation}
			\end{propertylist}
			\begin{example}[Problem 4.29]{Properties of Cumulative Distribution Functions}
				Let $C$ be an event for which $\Prob \left[ C \right] > 0$. Show that $F_{X} \left( X \Given C \right)$ satisfies the 8 properties of a Cumulative Distribution Function.
				\begin{propertylist}
					\item $0 \leq F_{X} \left( x \right) \leq 1$
					\item $\lim \limits_{x \rightarrow \infty} F_{X} \left( x \right) = 1$
					\item $\lim \limits_{x \rightarrow -\infty} F_{X} \left( x \right) = 0$
					\item For $a < b$, $F_{X} \left( a \right) \leq F_{X} \left( b \right)$
					\item $h>0$, $F_{X} \left( b \right) = \lim\limits_{h \rightarrow 0^{+}} F_{X} \left( b+h \right) = F_{X} \left( b^{+} \right)$
					\item $\Prob \left[ a < X \leq b \right] = F_{X} \left( b \right) - F_{X} \left( a \right)$
					\item $\Prob \left[ X = a \right] = F_{X} \left( a \right) - F_{X} \left( a^{-} \right)$
					\item $\Prob \left[ X > x \right] = 1 - F_{X} \left( x \right)$
				\end{propertylist}
				
				\tcblower
				
			\end{example}
		
		\subsubsection{Conditional Cumulative Distribution Function} \label{subsubsec:Conditional Cumulative Distribution Fuction}
			\begin{definition} [Conditional Cumulative Distribution Function] \label{def:Conditional Cumulative Distribution Function}
				The \emph{conditional cumulative distribution function (Conditional CDF)} of $X$ given $C$ is defined by:
				\begin{equation} \label{eq:Conditional Cumulative Distribution Function}
					F_{X \Given C} \left( x \Given C \right) = \frac{P \left[ \lbrace X = x \rbrace \Given C \right]}{\Prob \left[ C \right]}
				\end{equation}
				\begin{remark}
					The conditional CDF, $F_{X \Given C} \left( x \Given C \right)$ satisfies \emph{\textbf{all}} \nameref{subsubsec:Properties of Cumulative Distribution Functions}.
				\end{remark}
			\end{definition}
			\begin{example}[Problem 4.38]{Conditional Cumulative Distribution Function}
				A binary transmission system sends a ``0'' bit using a -1 voltage signal and a ``1'' by transmitting a +1.
				The received signal is corrupted by noise $N$ that has a Laplacian distribution with parameter $\alpha$.
				Assume that ``0'' and ``1'' bits are equiprobable.
				\begin{boldalphlist}
					\item Find the PDF of the received signal $Y = X+N$ where $X$ is the transmitted signal, given that ``0'' was transmitted; that a ``1'' was transmitted?
					\item Suppose that the receiver decides that a ``0'' was sent if $Y<0$ and a ``1'' was sent if $Y \geq 0$.
					What is the probability that the receiver makes an error given that a +1 was transmitted? A -1 was tramsitted?
					\item What is the overall probability of error?
				\end{boldalphlist}
				
				\tcblower
				
				Solution to Problem 4.38 from Homework 7.
			\end{example}
		
	\subsection{Probability Density Function (PDF)} \label{subsec:Probability Density Function}
		\begin{definition}[Probability Density Function] \label{def:Probability Density Function}
			The \emph{probability density function (PDF)} of a random variable $X$, if it exists, is defined as the derivative of the CDF of $X$.
			\begin{equation} \label{eq:Probability Density Function}
				f_{X} \left( x \right) = \frac{d}{dx} f_{X} \left( x \right)
			\end{equation}
			\begin{remark} \label{rmk:Probability Density Function}
				Both discrete and continuous random variables can have PDFs, however, the discrete random variable will have a discontinuous PDF.
			\end{remark}
			\begin{remark} \label{rmk:Probability Density Function Construction}
				It is possible to construct a random variable that has a \nameref{subsec:Cumulative Distribution Function}, but an undefined \nameref{subsec:Probability Density Function}.
			\end{remark}
			\begin{remark}
				This is an alternate, more useful way to specify the probability law described by the \nameref{subsec:Cumulative Distribution Function}.
			\end{remark}
		\end{definition}
		\begin{example}[Problem 4.25]{Find Probability Density Function}
			Find the PDF of the Weibull random variable where $\beta = 0.5$, $1$, and $2$.
			\begin{equation*}
				F_{X} \left( x \right) = \begin{cases}
					0 & \text{for } x<0 \\
					1-e^{-\left(\frac{x}{\lambda}\right)^{\beta}} & \text{for } x \geq 0
				\end{cases}
			\end{equation*}
			
			\tcblower
			
			Solution to Problem 4.25 from Homework 6.
		\end{example}
	
		\subsubsection{Properties of Probability Density Functions} \label{subsubsec:Properties of Probability Density Functions}
			These properties apply to PDFs of continuous random variables, and may not hold true for other types of random variables.
			\begin{propertylist}
				\item The associated CDF is non-decreasing, a \nameref{subsubsec:Properties of Cumulative Distribution Functions}.
					\begin{equation}
						f_{X} \left( x \right) \geq 0
					\end{equation}
				\item Since the definition of the PDF is that it's the derivative of the CDF, integrating the space over the PDF will yield the CDF.
					\begin{equation}
						\Prob \left[ a \leq X \leq b \right] = \int_{a}^{b} f_{X} \left( x \right) dx = F_{X} \left( b \right) - F_{X} \left( a \right)
					\end{equation}
				\item The value of a location in CDF is the integral of the PDF over the area.
					\begin{equation}
						F_{X} \left( x \right) = \int_{-\infty}^{x} f_{X} \left( t \right) dt
					\end{equation}
				\item Including the whole sample space should yield $1$.
					\begin{equation}
						\int_{-\infty}^{\infty} f_{X} \left( x \right) dx = 1
					\end{equation}
			\end{propertylist}
					\begin{example}[Final Exam Practice Problem 4]{Find Normalizing Constant $c$}
						A random variable $X$ has Probability Distribution Function (PDF):
						\begin{equation*}
							f_{X}\left( x \right) = \begin{cases}
								cx \left( 1- x^{3} \right) & 0 \leq x \leq 1 \\
								0 & \text{otherwise} \\
								\end{cases}
							\end{equation*}
						\begin{enumerate}[noitemsep, nolistsep]
							\item Find the normalizing constant $c$. (5 pts)
							\item Find $P \left[ X = 0.5 \right]$. (3 pts)
							\item Find $P \left[ X > 0.5 \right]$. (7 pts)
						\end{enumerate}
						
						\tcblower
						
						Solution to Final Exam Practice Problem 4.
					\end{example}
			\begin{remark*}
				Any non-negative, piecewise continuous function $g \left( x \right)$ with finite $\int_{-\infty}^{\infty} g \left( x \right) dx = C$ can be used to form a PDF.
			\end{remark*}
		
		\subsubsection{Conditional Probability Density Function} \label{subsubsec:Conditional Probability Density Function}
			\begin{definition}[Conditional Probability Density Function] \label{def:Conditional Probability Density Function}
				The \emph{conditional probability density function (Conditional PDF)} of $X$ given $C$ is defined by:
				\begin{equation} \label{eq:Conditional Probability Density Function}
					f_{X \Given C} \left( x \Given C \right) = \frac{d}{dx} F_{X \Given C} \left( x \Given C \right)
				\end{equation}
				\begin{remark}
					The conditional PDF, $f_{X \Given C} \left( x \Given C \right)$ satisfies \emph{\textbf{all}} \nameref{subsubsec:Properties of Probability Density Functions}.
				\end{remark}
			\end{definition}
	
	\subsection{Expected Value of Single Continuous Random Variable} \label{subsec:Expected Value of Single Continuous}
		\begin{definition}[Expected Value/Mean of Random Variable] \label{def:Expected Value of Single Continuous}
			The \emph{expected value of a random variable} $X$, denoted $\ExpectedValue \left[ X \right]$ is defined as:
			\begin{equation} \label{eq:Expected Value of Single Continuous}
				\ExpectedValue \left[ X \right] = \int_{-\infty}^{\infty} t f_{X} \left( t \right) dt
			\end{equation}
			\begin{remark}
				This works with \emph{\textbf{all}} random variables, or general random variables.
			\end{remark}
			\begin{remark}
				$\ExpectedValue \left[ X \right]$ is defined if the integral in \Cref{eq:Expected Value of Single Continuous} converges absolutely.
				This means:
				\begin{equation*}
					\ExpectedValue \left[ X \right] = \int_{-\infty}^{\infty} t f_{X} \left( t \right) dt < \infty
				\end{equation*}
			\end{remark}
		\end{definition}
		\begin{example}[Problem 4.57]{Conditional Expected Value of Continuous Random Variable}
			Find the $n$th moment of $U$, the uniform random variable in the unit interval.
			Repeat for $X$ uniform in $\left[ a,b \right]$.
			
			\tcblower
			
			Solution to Problem 4.57 from Homework 7.
		\end{example}
	
		\subsubsection{Properties of Expected Value} \label{subsubsec:Properties of Continuous Expected Value}
			\begin{propertylist}
				\item The expected value of a function of a random variable.
					\begin{equation}
						\ExpectedValue \left[ h \left( X \right) \right] = \int_{-\infty}^{\infty} h \left( t \right) \cdot f_{X} \left( t \right) dt
					\end{equation}
				\item Expectation of a constant, $c$, should be the constant itself.
					\begin{equation}
						\ExpectedValue \left[ c \right] = c
					\end{equation}
				\item Sum of a random variable, $X$, and a constant, $c$, is the same as the sum of the expectation of the random variable and the constant.
					\begin{equation}
						\ExpectedValue \left[ X+c \right] = \ExpectedValue \left[ X \right] + \ExpectedValue \left[ c \right]
					\end{equation}
				\item Linearity of Expectations for random variables
					\begin{equation}
						\ExpectedValue \left[ a_{0} + a_{1}X + a_{2}X^{2} + \ldots + a_{n}X^{n} \right] = a_{0} + a_{1}\ExpectedValue \left[ X \right] + a_{2}\ExpectedValue \left[ X^{2} \right] + \ldots + a_{n}\ExpectedValue \left[ X^{n} \right]
					\end{equation}
			\end{propertylist}
		
	\subsection{Variance of Single Continuous Random Variable} \label{subsec:Variance of Single Continuous}
		\begin{definition}[Variance of Random Variable] \label{def:Variance of Single Continuous}
			The \emph{variance} of the random variable $X$ is defined by:
			\begin{equation} \label{eq:Variance of Single Continuous}
				\sigma^{2} = \Variance \left[ X \right] = \ExpectedValue \left[ \left( X - \ExpectedValue \left[ X \right] \right)^{2} \right]
			\end{equation}
			\begin{remark}
				This holds true for \emph{\textbf{all}} types of random variables; discrete, continuous, and mixed.
			\end{remark}
		\end{definition}
		\begin{definition}[Standard Deviation] \label{def:Standard Deviation of Single Continuous}
			The \emph{standard deviation} of a random variable $X$, denoted by:
			\begin{equation} \label{eq:Standard Deviation of Single Continuous}
				\sigma = \StdDev \left[ X \right] = \sqrt{\Variance \left[ X \right]}
			\end{equation}
			\begin{remark}
				This holds true for \emph{\textbf{all}} types of random variables; discrete, continuous, and mixed.
			\end{remark}
		\end{definition}
	
	\subsection{Gaussian/Normal Random Variable} \label{subsec:Gaussian Random Variable}
		\begin{definition}[Gaussian/Normal Random Variable] \label{def:Gaussian Random Variable}
			The \emph{Gaussian or normal random variable} is the classic ``bell curve'' probability distribution.
			It is usually described as $X \DrawnFrom N \left( \mu, \sigma^{2} \right)$.
			$\mu$ is $\ExpectedValue \left[ X \right] $ and $\sigma^{2}$ is how narrow/sharp the bell is.
			A Gaussian Random Variable has a PDF of:
			\begin{equation} \label{eq:PDF of Gaussian Random Variable}
				f_{X} \left( x \right) = \frac{1}{\sqrt{2 \pi} \sigma} e^{-\frac{\left( x - \mu \right)^{2}}{2 \sigma^{2}}} \text{, } x \in \RealNums
			\end{equation}
		\end{definition}
		\begin{definition}[Standard Normal Distribution] \label{def:Standard Normal Distribution}
			The \emph{standard normal distribution} is just a specific \nameref{def:Gaussian Random Variable}.
			The standard normal distribution is a \nameref{def:Gaussian Random Variable} with $\mu = 0, \sigma^{2} = 1$.
			\begin{remark} \label{rmk:CDF of Standard Normal Distribution}
				The CDF of the \nameref{def:Standard Normal Distribution} is denoted with $\Phi$.
			\end{remark}
		\end{definition}
	
	To find the probability of something for a Gaussian Random Variable, you would end up converting it to the \nameref{def:Standard Normal Distribution}.
	If $X \DrawnFrom N \left( \mu, \sigma^{2} \right)$ and $Y \DrawnFrom N \left( 0,1 \right)$,
		\begin{equation} \label{eq:Probability of Gaussian Random Variable}
			\begin{aligned}
				\Prob \left[ a \leq x \leq b \right] &= \frac{1}{\sqrt{2 \pi} \sigma} \int_{-\infty}^{\infty} e^{\frac{-1}{2} \left( \frac{x-\mu}{\sigma} \right)^{2}} dx \\
				&= \frac{1}{\sqrt{2 \pi}} \int_{\frac{a-\mu}{\sigma}}^{\frac{b-\mu}{\sigma}} e^{\frac{-1}{2}y}dy \\
				&= P \left[ \frac{a-\mu}{\sigma} \leq Y \leq \frac{b-\mu}{\sigma} \right] \\
				&= F_{Y} \left( \frac{b-\mu}{\sigma} \right) - F_{Y} \left( \frac{a-\mu}{\sigma} \right) \\
				&= \Phi \left( \frac{b-\mu}{\sigma} \right) - \Phi \left( \frac{a-\mu}{\sigma} \right) \\
			\end{aligned}
		\end{equation}
		
		\subsubsection{Q-Function} \label{subsubsec:Q-Function}
			\begin{definition}[Q-Function] \label{def:Q-Function}
				The \emph{Q-Function} is primarily used in electrical engineering.
				It is defined as:
				\begin{equation} \label{eq:Q-Function}
					\begin{aligned}
						Q &= 1 - \Phi \left( x \right) \\
						&= \frac{1}{\sqrt{2 \pi}} \int_{x}^{\infty} e^{\frac{-t^{2}}{2}} dt \\
					\end{aligned}
				\end{equation}
				\begin{remark}
					\begin{equation}
						Q \left( Z \right) = 1-f_{Z} \left( z \right)
					\end{equation}
				\end{remark}
			\end{definition}
			\begin{example}[Problem 4.62]{Q-Function Application}
				The $r$th percentile, $\pi \left( r \right)$, of a random variable $X$ is defined by $\Prob \left[ X \leq \pi \left( r \right) \right] = \frac{r}{100}$.
				\begin{boldalphlist}
					\item Find the 90\%, 95\%, and 99\% percentiles of the exponential random variable with parameter $\lambda$.
					\item Repeat part a for the Gaussian random variable with parameters $m=0$ and $\sigma^{2}$.
				\end{boldalphlist}
				
				\tcblower
				
				Solution to Problem 4.62 from Homework 7.
			\end{example}
	
	\subsection{Markov Inequality} \label{subsec:Markov Inequality}
		\begin{definition}[Markov Inequality] \label{def:Markov Inequality}
			Let $X$ be a non-negative random variable with $\ExpectedValue \left[ X \right] < \infty$.
			The \emph{Markov Inequality} states that:
			\begin{equation} \label{eq:Markov Inequality}
				\Prob \left[ X \geq a \right] \leq \frac{\ExpectedValue\left[ X \right]}{a}
			\end{equation}
		\end{definition}
		\begin{proof}[Proving the Markov Inequality] \label{proof:Markov Inequality}
			\begin{equation*}
				\ExpectedValue \left[ X \right] = \int_{-\infty}^{\infty} x \cdot f_{X} \left( x \right) dx 
			\end{equation*}
			Because we defined $X \geq 0$, we change the lower bound to $0$.
			\begin{equation*}
				\ExpectedValue \left[ X \right] = \int_{0}^{\infty} x f_{X} \left( x \right) dx 
			\end{equation*}
			We then split the integral up around some point, $a$.
			\begin{equation*}
				\ExpectedValue \left[ X \right] = \int_{0}^{a} x f_{X} \left( x \right) dx + \int_{a}^{\infty} x f_{X} \left( x \right) dx
			\end{equation*}
			Since the first integral is integrating over a non-negative function, the integral is also non-negative.
			\begin{equation*}
				\int_{0}^{a} x f_{X} \left( x \right) dx + \int_{a}^{\infty} x f_{X} \left( x \right) dx \geq \int_{a}^{\infty} x f_{X} \left( x \right) dx
			\end{equation*}
			\begin{equation*}
				\ExpectedValue \left[ X \right] \geq \int_{a}^{\infty} x f_{X} \left( x \right) dx
			\end{equation*}
			Because $x>a$, we can pull a term out of $f_{X} \left( x \right)$
			\begin{equation*}
				\ExpectedValue \left[ X \right] \geq \int_{a}^{\infty} a f_{X} \left(x \right) dx
			\end{equation*}
			Because $a$ is a constant, we pull it out of the integral,
			\begin{equation*}
				\ExpectedValue \left[ X \right] \geq a \int_{a}^{\infty} f_{X} \left( x \right) dx
			\end{equation*}
			Then, we end up with an integral that is the definition of the probability of a continuous random variable.
			\begin{equation*}
				\ExpectedValue \left[ X \right] \geq a \Prob \left[ X \geq a \right]
			\end{equation*}
			\begin{equation*} \label{eq:Proved Markov Inequality}
				\therefore
				\ExpectedValue \left[ X \right] \geq a \Prob \left[ X \geq a \right]
			\end{equation*}
		\end{proof}
		\begin{example}[Problem 4.97]{Markov Inequality}
			Compare the \nameref{eq:Markov Inequality} and the exact probability for th event $\lbrace X > c \rbrace$ as a function of $c$ for:
			\begin{boldalphlist}
				\item $X$ is a uniform random variable in the interval $\left[ 0,b \right]$.
				\item $X$ is an exponential random variable with parameter $\lambda$.
				\item $X$ is a Pareto random varaible with $\alpha >1$.
				\item $X$ is a Rayleigh random variable.
			\end{boldalphlist}
			
			\tcblower
			
			Solution to Problem 4.97 from Homework 7.
		\end{example}
	
	\subsection{Chebychev Inequality} \label{subsec:Chebychev Inequality}
		\begin{definition}[Chebychev Inequality] \label{def:Chebychev Inequality}
			Let $X$ be a non-negative random variable with $\ExpectedValue \left[ X \right] < \infty$.
			The \emph{Chebychev Inequality} states that:
			\begin{equation} \label{eq:Chebychev Inequality}
				P \left[ \lvert X-\mu \rvert \geq a \right] \leq \frac{\sigma^{2}}{a^{2}}
			\end{equation}
			\begin{proof}[Proving the Chebychev Inequality] \label{proof:Chebychev Inequality}
				\begin{equation*}
					P \left[ \left( X-\mu \right)^{2} \geq a^{2} \right] \leq \frac{\ExpectedValue \left[ \left( X-\mu \right)^{2} \right]}{a^{2}}
				\end{equation*}
				Because $X-\mu = \sigma$, we replace it.
				\begin{equation*}
					P \left[ \left( X-\mu \right)^{2} \geq a^{2} \right] \leq \frac{\ExpectedValue \left[ \sigma^{2} \right]}{a^{2}}
				\end{equation*}
			\end{proof}
		\end{definition}
		\begin{example}[Problem 4.100]{Chebychev Inequality}
			Let $X$ be the number of successes in $n$ Bernoulli trials were the probability of success is $p$.
			Let $Y = \frac{X}{n}$ be the average number of successes per trial.
			Apply the \nameref{eq:Chebychev Inequality} to the event $\lbrace \lvert Y-p \rvert > a \rbrace$.
			What happens as $n \rightarrow \infty$?
		
			\tcblower
			
			Solution to Problem 4.100 from Homework 7.
		\end{example} % Section 6

\section{Multiple Random Variables} \label{sec:Multiple Random Variables}
	\subsection[Joint PMF]{Joint Probability Mass Function} \label{subsec:Joint PMF}
		\begin{definition}[Joint Probability Mass Function] \label{def:Joint PMF}
			The \emph{joint probability mass function (joint PMF)} of 2 discrete random variables $X$, $Y$ is defined as:
			\begin{equation} \label{eq:Joint PMF}
				p_{X,Y} = P \left[ \lbrace X=x \rbrace \cap \lbrace Y=y \rbrace \right] \text{ for all } x,y \in S_{X,Y}
			\end{equation}
			\begin{itemize}[noitemsep, nolistsep]
				\item This satisfies ALL properties of single random variable PMFs
			\end{itemize}
		\end{definition}
	
		\subsubsection[Marginal PMF]{Marginal Probability Mass Function} \label{subsubsec:Marginal PMF}
			\begin{definition}[Marginal Probability Mass Function] \label{def:Marginal PMF}
				Given a joint PMF of discrete random variables $X$, $Y$, the \emph{Marginal Probability Mass Function (Marginal PMF)} of $X$ is defined as:
				\begin{equation} \label{eq:Marginal PMF}
					p_{X} \left( x_{i} \right) = P \left[ X = x_{i} \right] \text{ for } x_{i} \in S_{X}
				\end{equation}
				and is calculated as:
				\begin{equation} \label{eq:Calculate Marginal PMF}
					p \left( x_{i} \right) = \sum_{y \in S_{Y}} p_{X,Y} \left( x_{i}, y \right)
				\end{equation}
			\end{definition}
		
	\subsection[Joint CDF]{Joint Cumulative Distribution Function} \label{subsec:Joint CDF}
		\begin{definition}[Joint Cumulative Distribution Function] \label{def:Joint CDF}
			The \emph{Joint Cumulative Distribution Function (Joint CDF)} of $X$ and $Y$ is defined as the probability of the event $ \lbrace X \leq x \rbrace \cap \lbrace Y \leq y \rbrace $
			\begin{equation} \label{eq:Joint CDF}
				\begin{aligned}
					F_{X,Y} \left( x, y \right) &= P \left[ \lbrace X \leq x \rbrace \cap \lbrace Y \leq y \rbrace \right] \text{ for all } \left( x,y \right) \in \mathbb{R}^2 \\
					&= P \left[ \lbrace X \leq x \rbrace , \lbrace Y \leq y \rbrace \right]
				\end{aligned}
			\end{equation}
		\end{definition}
		
		\subsubsection{Properties of Joint Cumulative Distribution Functions} \label{subsubsec:Properties of Joint Cumulative Distribution Functions}
			\begin{enumerate}[label=\textbf{(\roman*)}, noitemsep, nolistsep]
				\item $F_{X,Y} \left( x,y \right)$ is non decreasing.
					\begin{equation} \label{eq:Joint CDF Property 1}
						F_{X,Y} \left( x_{1},y_{1} \right) \leq F_{X,Y} \left( x_{2},y_{2} \right) \text{ if } x_{1} \leq x_{2} \text{ and } y_{1} \leq y_{2}
					\end{equation}
				\item \begin{equation} \label{eq:Joint CDF Property 2}
						\begin{aligned}
							\lim\limits_{y \rightarrow -\infty} F_{X,Y} \left( x,y \right) &= 0 \\
							\lim\limits_{x \rightarrow -\infty} F_{X,Y} \left( x,y \right) &= 0 \\
							\lim\limits_{\left( x,y \right) \rightarrow \left( \infty, \infty \right)} F_{X,Y} \left( x,y \right) &= 1 \\
						\end{aligned}
					\end{equation}
				\item The Marginal CDFs can be obtained from the Joint CDF by removing restrictions for all but one variable.
					\begin{equation} \label{eq:Joint CDF Property 3}
						\begin{aligned}
							F_{X} \left( x \right) &= P \left[ \lbrace X \leq x \rbrace, \lbrace Y \text{ is anything} \rbrace \right] \\
													   &= P \left[ \lbrace X \leq x \rbrace, \lbrace -\infty \leq y \leq \infty \rbrace \right] \\
													   &= \lim\limits_{y \rightarrow \infty} F_{X,Y} \left( x,y \right) \\
							F_{Y} \left( y \right) &= \lim\limits_{x \rightarrow \infty} F_{X,Y} \left( x,y \right) \\
						\end{aligned}
					\end{equation}
				\item The Joint CDF is continuous from $\infty$ to $-\infty$.
					\begin{equation} \label{eq:Joint CDF Property 4}
						\begin{aligned}
							\lim\limits_{x \rightarrow a^{+}} F_{X,Y} \left( x,y \right) &= F_{X,Y} \left( a,y \right) \\
							\lim\limits_{y \rightarrow b^{+}} F_{X,Y} \left( x,y \right) &= F_{X,Y} \left( x,b \right) \\
						\end{aligned}
					\end{equation}
				\item The probability of the ``rectangle'' $\lbrace x_{1} \leq X \leq x_{2}, y_{1} \leq Y \leq y_{2} \rbrace$
					\begin{equation} \label{eq:Joint CDF Property 5}
						\begin{aligned}
							P \left[ \lbrace x_{1} \leq X \leq x_{2}, y_{1} \leq Y \leq y_{2} \rbrace \right] &= P \left[ \lbrace X \leq x_{2}, Y \leq y_{2} \rbrace \right] - P \left[ \lbrace X \leq x_{1}, Y \leq y_{2} \rbrace \right] - \\
							&P \left[ \lbrace X \leq x_{2}, Y \leq y_{1} \rbrace \right] + P \left[ \lbrace X \leq x_{1}, Y \leq y_{1} \rbrace \right] \\
							&= F_{X,Y} \left( x_{2}, y_{2} \right) - F_{X,Y} \left( x_{1}, y_{2} \right) - F_{X,Y} \left( x_{2}, y_{1} \right) + F_{X,Y} \left( x_{1}, y_{1} \right)
						\end{aligned}
					\end{equation}
			\end{enumerate}
	
		\subsubsection[Marginal CDF]{Marginal Cumulative Distribution Function} \label{subsubsec:Marginal CDF}
			\begin{definition}[Marginal Cumulative Distribution Function] \label{def:Marginal CDF}
				We obtain the \emph{Marginal Cumulative Distribution Functions (Marginal CDFs)} by removing the constraint on one of the variables. 
				\begin{equation} \label{eq:Marginal CDF}
					\begin{aligned}
						F_{X} \left( x \right) &= P \left[ \lbrace X \leq x \rbrace, \lbrace Y \text{ is anything} \rbrace \right] \\
						&= P \left[ \lbrace X \leq x \rbrace, \lbrace -\infty \leq y \leq \infty \rbrace \right] \\
						&= \lim\limits_{y \rightarrow \infty} F_{X,Y} \left( x,y \right) \\
						F_{Y} \left( y \right) &= \lim\limits_{x \rightarrow \infty} F_{X,Y} \left( x,y \right) \\
					\end{aligned}
				\end{equation}
			\end{definition}
	\subsection[Joint PDF]{Joint Probability Density Function} \label{subsec:Joint PDF}
		\begin{definition}[Joint Probability Density Function] \label{def:Joint PDF}
			We say that $X$, $Y$ are jointly continuous if the probabilities of events involving $X$ and $Y$ can be expressed as an integral of a \emph{Joint Probability Density Function (Joint PDF)}. \newline
			i.e. THere exists soem nonnegative function $f_{X,Y} \left( x,y \right)$, which we call the joint PDF, that is defined on the real plane such that tfor every event $B$ which is a subset of the xy plane
			\begin{equation}\label{eq:Joint PDF}
				P \left[ \left( X,Y \right) \text{in } B \right] = \iint_{B} f_{X,Y} \left( x,y \right) dx dy
			\end{equation}
			\begin{remark}
				The probability mass of an event is found by integrating the PDF over the region in the xy plane corresponding to your event.
			\end{remark}
		\end{definition}
	
		\subsubsection{Properties of Joint Probability Density Functions} \label{subsubsec:Joint PDF Properties}
			\begin{gather}
				\iint_{B} f_{X,Y} \left( x,y \right) = 1 \\
				x \geq 0, y \geq 0 \forall x \forall y 
			\end{gather}
			
		\subsubsection{Facts about Joint PDFs} \label{subsubsec:Joint PDF Facts}
			\begin{align}
				\int_{-\infty}^{\infty} \int_{-\infty}^{\infty} &f_{X,Y} \left( x,y \right) = 1 \\
				F_{X,Y} \left( x,y \right) &= \int_{-\infty}^{x} \int_{-\infty}^{y} f_{X,Y} \left( s,t \right) dt ds \\
				f_{X,Y} &= \frac{\partial^{2} f_{X,Y} \left( x,y \right)}{\partial x \partial y} 
			\end{align}
			
		\subsubsection{Marginal PDF} \label{subsubsec:Marginal PDF}
			\begin{definition}[Marginal Probability Density Function] \label{def:Marginal PDF}
				The \emph{Marginal Probability Density Functions (Marginal PDFs)} $f_{X} \left( x \right)$ and $f_{Y} \left( y \right)$ are obtained by taking the derivative of the marginal CDFs.
				\begin{equation}
					\begin{aligned}
						f_{X} \left( x \right) &= \frac{d}{dx} F_{X} \left( x \right) \\
						&= \frac{d}{dx} \int_{-\infty}^{x} \left[ \int_{-\infty}^{\infty} f_{X,Y} \left( s,t \right) dt ds \right] \\
						&= \frac{d}{dx} \int_{-\infty}^{x} \int_{-\infty}^{\infty} f_{X,Y} \left( s,t \right) dt ds \\
						&\text{Simplified with \nameref{def:2nd Fundamental Theorem of Calculus}} \\
						&= \int_{-\infty}^{\infty} f_{X,Y} \left( x,t \right) dt \\
						f_{X} &= \int_{-\infty}^{\infty} f_{X,Y} \left( x,t \right) dt \\
					\end{aligned}
				\end{equation}
			\end{definition}
	\subsection{Independence of Multiple Random Variables} \label{subsec:Independence of Multiple Random Variables}
		\begin{definition}[Independent Random Variables] \label{def:Independence of Multiple Random Variables}
			\emph{$X$ and $Y$ are independent random variables} if \emph{\textbf{ANY}} event $A_{1}$ defined in terms of $S$ is independent of \emph{\textbf{ANY}} event $A_{2}$ defined in terms of $Y$.
			\begin{equation} \label{eq:Independence of Multiple Random Variables}
				P \left[ X \in A_{1}, Y \in A_{2} \right] = P \left[ X \in A_{1} \right] * P \left[ Y \in A_{2} \right]
			\end{equation}
		\end{definition}
	There are 3 ways to phrase this:
	\begin{enumerate}[noitemsep, nolistsep]
		\item For discrete random variables $X$ and $Y$, $X$ and $Y$ are independent if and only if:
			\begin{equation} \label{eq:Independence of Multiple Discrete Random Variables Using PMF}
				p_{X,Y} \left( x,y \right) = p_{X} \left( x \right) * p_{Y} \left( y \right)
			\end{equation}
		\item For general random variables $X$ and $Y$, $X$ and $Y$ are independent if and only if:
			\begin{equation} \label{eq:Independence of Multiple General Random Variables Using CDF}
				F_{X,Y} \left( x,y \right) = F_{X} \left( x \right) * F_{Y} \left( y \right)
			\end{equation}
		\item For (continuous) random variables $X$ and $Y$, $X$ and $Y$ are independent if and only if:
			\begin{equation} \label{eq:Independence of Multiple Continuous Random Variables Using PDF}
				f_{X,Y} \left( x,y \right) = f_{X} \left( x \right) * f_{Y} \left( y \right)
			\end{equation}
	\end{enumerate}
	You can prove \nameref{eq:Independence of Multiple Discrete Random Variables Using PMF}, \Cref{eq:Independence of Multiple Discrete Random Variables Using PMF}.
	\begin{proof}[Independence of Discrete Random Variables with PMF] \label{proof:Independence of Discrete Random Variables with PMF}
		
	\end{proof}
	\begin{theorem}[Independence of Random Functions] \label{thm:Independence of Random Functions}
		If random variables $X$, $Y$ are independent, then $g\left( X \right)$ and $h \left( Y \right)$ are also independent.
	\end{theorem}

	\subsection{Expected Value of Functions with 2 Random Variables} \label{subsec:Expected Value of Functions with 2 Random Variables}
		\begin{definition}[Expectation of a Function with 2 Random Variables] \label{def:Expectation of a Function with 2 Random Variables}
			Let $Z$ be a random variable described by the function $Z = g \left( X,Y \right)$.
			\begin{equation} \label{eq:Expected Value of a Function with 2 Random Variables}
				\ExpectedValue = 
				\begin{cases}
					\int_{-\infty}^{\infty} \int_{-\infty}^{\infty} g \left( x,y \right) \cdot f_{X,Y} \left( x,y \right) dx dy &
						\text{if $X$ and $Y$ are jointly continuous} \\
					\sum\limits_{i \in S_{X}} \sum\limits_{j \in S_{Y}} g \left( x_{i}, y_{j} \right) \cdot p_{X,Y} \left( x,y \right) &
						\text{if $X$ and $Y$ are both discrete} \\
				\end{cases}
			\end{equation}
			\begin{remark}[Expected Value of Sum of Random Variables] \label{rmk:Expected Value of Sum of Random Variables}
				You \emph{\textbf{do not}} need to assume independence to say:
				\begin{equation} \label{eq:Expected Value of Sum of Random Variables}
					\ExpectedValue \left[ X_{1}+X_{2}+\ldots+X_{n} \right] = \ExpectedValue \left[ X_{1} \right] + \ExpectedValue \left[ X_{2} \right] + \ldots + \ExpectedValue \left[ X_{n} \right]
				\end{equation}
			\end{remark}
			\begin{remark}[Expected Value of Product of Random Variables] \label{rmk:Expected Value of Product of Random Variables}
				If $X$ and $Y$ are independent, then
				\begin{equation} \label{eq:Expected Value of Product of Random Variables}
					\ExpectedValue \left[ g \left( X \right) h \left( Y \right) \right] = \ExpectedValue \left[ g \left( X \right) \right] \cdot \ExpectedValue \left[ h \left( Y \right) \right]
				\end{equation}
			\end{remark}
		\end{definition}

	\subsection{Joint Moments, Correlation, and Covariance} \label{subsec:Joint Moments, Correlation, and Covariance}
		\subsubsection{Joint Moments} \label{subsubsec:Joint Moments}
			\begin{definition}[The j,kth Moment] \label{def:jkth Moment}
				The \emph{j,kth moment of $X$ and $Y$} is:
				\begin{equation} \label{eq:jkth Moment}
					\ExpectedValue \left[ X^{j} Y^{k} \right] =
					\begin{cases}
						\int_{-\infty}^{\infty} \int_{-\infty}^{\infty} x^{j} y^{k} \cdot f_{X,Y} \left( x,y \right) dx dy &
							\text{if $X$, $Y$ are jointly continuous} \\
						\sum\limits_{i \in S_{X}} \sum\limits_{\ell \in S_{Y}} x_{i}^{j} y_{l}^{k} \cdot p_{X,Y} \left( x_{i},y_{\ell} \right) & 
							\text{if $X$, $Y$ are discrete} \\
					\end{cases}
				\end{equation}
			\end{definition} 
		
		\subsubsection{Correlation} \label{subsubsec:Correlation}
			\begin{definition}[Correlation] \label{def:Correlation}
				The \emph{Correlation of $X$ and $Y$} is defined as the $1,1$ moment, i.e. $\ExpectedValue \left[ X^{1} Y^{1} \right]$.
				\begin{remark}
					If $X$, $Y$ are such that $\ExpectedValue \left[ X^{1} Y^{1} \right] = 0$, then we say that $X$, $Y$ are \emph{orthogonal}.
				\end{remark}
				\begin{remark}[Uncorrelated] \label{rmk:Uncorrelated}
					If $X$, $Y$ are such that $\ExpectedValue \left[ XY \right] = \ExpectedValue \left[ X \right] \ExpectedValue \left[ Y \right]$, then $X$ and $Y$ are \emph{uncorrelated}.
				\end{remark}
				\begin{remark}
					If $X$, $Y$ are independent, then they are uncorrelated; but if $X$ and $Y$ are uncorrelated, \emph{\textbf{they are not always independent}}.
				\end{remark}
			\end{definition}
			\begin{definition}[Correlation Coefficient] \label{def:Correlation Coefficient}
				The \emph{correlation coefficient of $X$, $Y$} is defined as
				\begin{equation} \label{eq:Correlation Coefficient}
					\rho_{X,Y} = \frac{\text{Cov} \left[ X,Y \right]}{\sigma_{X} \sigma_{Y}}
				\end{equation}
				\begin{remark}
					$\rho_{X,Y}$ only ranges $-1 \leq \rho_{X,Y} \leq 1$
				\end{remark}
				\begin{remark}
					The closer $\rho_{X,Y}$ is to $+1$, the closer $X$ and $Y$ are to having a positive linear relationship (Positive slope). \newline
					The closer $\rho_{X,Y}$ is to $-1$, the closer $X$ and $Y$ are to having a negative linear relationship (Negative slope). \newline
					If $\rho_{X,Y} = 0$, the $\text{Cov}\left[ X,Y \right] = 0$, which means that $X$ and $Y$ are \emph{uncorrelated}.
				\end{remark}
			\end{definition}
	
		\subsubsection{Covariance} \label{subsubsec:Covariance}
			\begin{definition}[Covariance] \label{def:Covariance}
				The \emph{covariance of $X$ and $Y$} is denoted:
				\begin{equation} \label{eq:Covariance-Form 1}
					\text{Cov} \left[ X,Y \right] = \ExpectedValue \left[ \left( X - \ExpectedValue \left[ X \right] \right) \left( Y - \ExpectedValue \left[ Y \right] \right)\right]
				\end{equation}
				\begin{equation} \label{eq:Covariance-Form 2}
					\text{Cov} \left[ X,Y \right] = \ExpectedValue \left[ XY \right] - \ExpectedValue \left[ X \right] \ExpectedValue \left[ Y \right]
				\end{equation}
			\end{definition}
		
		\subsection{Conditional Probability Functions} \label{subsec:Multiple Variable Conditional Probability Functions}
		There are 3 major cases for these:
			\begin{enumerate}[noitemsep, nolistsep]
				\item \nameref{subsubsec:2 Discrete Random Variables}
				\item \nameref{subsubsec:1 Discrete 1 Continuous Random Variables}
				\item \nameref{subsubsec:2 Continuous Random Variables}
			\end{enumerate}
					
			\subsubsection{2 Discrete Random Variables} \label{subsubsec:2 Discrete Random Variables}
				\begin{definition}[Conditional Probability Mass Function] \label{def:2 Discrete-Conditional PMF}
					The \emph{conditional Probability Mass Function (Conditional PMF)} of $Y$ given that $X=x$ is:
					\begin{equation} \label{eq:2 Discrete-Conditional PMF}
						p_{Y} \left( y \Given x \right)
							= \frac{P \left[ \lbrace Y=y \rbrace \cap \lbrace X=x \rbrace \right]}{P \left[ X=x \right]} 
							= \frac{p_{X,Y} \left( x,y \right)}{p_{X} \left( x \right)}
					\end{equation}
					\begin{remark}
						This also implies that
						\begin{equation} \label{eq:2 Discrete-Joint PMF}
							p_{X,Y} \left( x,y \right) = p_{Y} \left( y \Given x \right) \cdot p_{X} \left( x \right)
						\end{equation}
					\end{remark}
					\begin{remark}
						If $X$ and $Y$ are \emph{independent}, then:
						\begin{equation} \label{eq:2 Discrete-Independent Conditional PMF}
							p_{X} \left( y \Given x \right)
							= \frac{p_{X,Y} \left( x,y \right)}{p_{X} \left( x \right)}
							= \frac{p_{X} \left( x \right) p_{Y} \left( y \right)}{p_{X} \left( x \right)}
							= p_{Y} \left( y \right)
						\end{equation}
					\end{remark}
					\begin{remark}
						The \nameref{def:2 Discrete-Conditional PMF} of 2 discrete random variables satisfies all \nameref{subsubsec:Properties of Probability Mass Functions}.
					\end{remark}
				\end{definition}

			\subsubsection{1 Discrete and 1 Continuous Random Variable} \label{subsubsec:1 Discrete 1 Continuous Random Variables}
			For this section, let $X$ be a discrete random variable and $Y$ a continuous random variable.
				\begin{definition}[Conditional Cumulative Distribution Function] \label{def:1 Discrete 1 Continuous-Conditional CDF}
					The \emph{conditional Cumulative Distribution Function (Conditional CDF)} of $Y$ given that $X=x$ is:
					\begin{equation} \label{eq:1 Discrete 1 Continuous-Conditional CDF}
						F_{Y} \left( y \Given x \right) = P \left[ Y \leq y \Given X=x \right]
						= \frac{P \left[ \lbrace Y \leq y \rbrace \cap \lbrace X=x \rbrace \right]}{P \left[ X=x \right]}
					\end{equation}
					\begin{remark}
						If $X$ and $Y$ are \emph{independent}, then:
						\begin{equation} \label{eq:1 Discrete 1 Continuous-Independent Conditional CDF}
							F_{Y} \left( y \Given x \right)
							= \frac{F_{X,Y} \left( x,y \right)}{p_{X} \left( x \right)}
							= \frac{F_{Y} \left( y \right) p_{X} \left( x \right)}{p_{X} \left( x \right)}
							= F_{Y} \left( y \right)
						\end{equation}
						This also means that:
						\begin{equation*}
							P \left[ Y \leq y \Given X=x \right]
							= P \left[ Y \leq y \right] \cdot P \left[ X=x \right]
						\end{equation*}
					\end{remark}
					\begin{remark}
						The similar relations for independent random variables with their conditional and marginal probability functions does not hold true with this.
					\end{remark}
					\begin{remark}
						The \nameref{def:1 Discrete 1 Continuous-Conditional CDF} of 1 discrete random variable and 1 continuous random variable satisfies all \nameref{subsubsec:Properties of Cumulative Distribution Functions}.
					\end{remark}
				\end{definition}
				\begin{definition}[Conditional Probability Density Function] \label{def:1 Discrete 1 Continuous-Conditional PDF}
					The \emph{conditional Probability Distribution Function (Conditional PDF)} of $Y$ given $X=x$ is
					\begin{equation} \label{eq:1 Discrete 1 Continuous-Conditional PDF}
						f_{Y} \left( y \Given x \right) = \frac{d}{dy} F_{Y} \left( y \Given x \right)
					\end{equation}
					This also means,
					\begin{equation*}
						P \left[ Y \leq y \Given X=x \right]
						= \int_{y \in A} f_{Y} \left( y \Given x \right) dy
					\end{equation*}
					\begin{remark}
						The \nameref{def:1 Discrete 1 Continuous-Conditional PDF} of 1 discrete random variable and 1 continuous random variable satisfies all \nameref{subsubsec:Properties of Probability Density Functions}.
					\end{remark}
				\end{definition}
			
			\subsubsection{2 Continuous Random Variables} \label{subsubsec:2 Continuous Random Variables}
				\begin{definition}[Conditional Cumulative Distribution Function] \label{def:2 Continuous-Conditional CDF}
					The \emph{conditional Cumulative Distribution Function (Conditional CDF)} of $Y$ given $X=x$ for $X$ and $Y$ continuous random variables is:
					\begin{equation} \label{eq:2 Continuous-Conditional CDF}
						F_{Y} \left(y \Given x \right)
						= \lim\limits_{h\rightarrow 0} F_{Y} \left( y \Given x < X \leq \left( x+h \right) \right)
						= \frac{\int_{-\infty}^{y} f_{X,Y} \left( x,v \right) dv}{f_{X} \left( x \right)}
					\end{equation}
					\begin{remark}
						The \nameref{def:2 Continuous-Conditional CDF} of 2 continuous random variables satisfies all \nameref{subsubsec:Properties of Cumulative Distribution Functions}.
					\end{remark}
					\begin{remark}
						The similar relations for the conditional and marginal probability functions do not hold up for 2 continuous random variables too well.
					\end{remark}
				\end{definition}
			
				\begin{definition}[Conditional Probability Density Function] \label{def:2 Continuous-Conditional PDF}
					The \emph{conditional Probability Density Function (Conditional PDF)} of $Y$ given $X=x$ for $X$ and $Y$ continuous random variables is:
					\begin{equation} \label{eq:2 Continuous-Conditional PDF}
						f_{Y} \left( y \Given x \right)
						= \frac{d}{dy} F_{Y} \left( y \Given x \right)
						= \frac{f_{X,Y} \left( x,y \right)}{f_{X} \left( x \right)}
					\end{equation}
					\begin{remark}
						If $X$ and $Y$ are independent, then:
						\begin{equation} \label{eq:2 Continuous-Independent Conditional PDF}
							f_{X} \left( y \Given x \right)
							= \frac{f_{X,Y} \left( x,y \right)}{f_{X} \left( x \right)}
							= \frac{f_{X} \left( x \right) f_{Y} \left( y \right)}{f_{X} \left( x \right)}
							= f_{Y} \left( y \right)
						\end{equation}
					\end{remark}
					\begin{remark}
						The \nameref{def:2 Continuous-Conditional PDF} of 2 continuous random variables satisfies all \nameref{subsubsec:Properties of Probability Density Functions}.
					\end{remark}
				\end{definition}
			
	\subsection{Conditional Expectation of Multiple Random Variables} \label{subsec:Conditional Expectation of Multiple Variables}
		\begin{definition}[Conditional Expectation] \label{def:Conditional Expectation of Multiple Variables}
			The \emph{conditional expectation} of $Y$ given $X$ is:
			\begin{equation} \label{eq:Conditional Expectation of Multiple Variables}
				\ExpectedValue \left[ Y \Given X=x \right] = \int_{-\infty}^{\infty} y \cdot f_{Y} \left( y \Given x \right) dy
			\end{equation}
			\begin{remark}[Special Case]
				There is a special case when \emph{\textbf{both}} $X$ and $Y$ are discrete random variables.
				\begin{equation} \label{eq:Conditional Expectation of Multiple Discrete Variables}
					\ExpectedValue \left[ Y \Given X=x \right] = \sum_{y \in S_{Y}} y \cdot p_{Y} \left( y \Given x \right)
				\end{equation}
			\end{remark}
			\begin{remark}
				When calculating the \nameref{subsec:Conditional Expectation of Multiple Variables}, and they as for $\ExpectedValue \left[ Y \Given X=x \right]$, that means you \emph{\textbf{must}} consider all possible values that $X$ can take.
				This can be generalized to the equation below.
				\begin{equation} \label{eq:General Conditional Expectation of Multiple Variables}
					\ExpectedValue \left[ Y \Given X=x \right] = \sum_{x \in S_{X}} \left( \sum_{y \in S_{Y}} y \cdot p_{Y} \left( y \Given x \right) \right)
				\end{equation}
				This can be described. You must take a single value for $x$, and take it over all $y$'s, then take the next value for $x$, until you have exhausted all values in both $S_{X}$ and $X_{Y}$. \newline
				This can also be translated into the continuous case, but the discrete case is a little simpler to understand this generality.
			\end{remark}
			\begin{remark}
				$\ExpectedValue \left[ Y \Given X=x \right]$ is a function of $X$, so it can be written as $g \left( x \right) = \ExpectedValue \left[ Y \Given X=x \right]$.
				Thus, we can also say
				\begin{equation} \label{eq:Expected Value of Conditional Expected Value of Multiple Variables}
					\ExpectedValue \left[ g \left( x \right) \right] = \ExpectedValue \left[ \ExpectedValue \left[ Y \Given X \right] \right] = \ExpectedValue \left[ Y \right]
				\end{equation}
				\begin{subequations}
					\begin{align} \label{eq:Joint PDF of Multiple Continuous Random Variables}
						\ExpectedValue \left[ Y \right]
						= \ExpectedValue \left[ \ExpectedValue \left[ Y \Given X \right] \right]
						&= \int_{-\infty}^{\infty} \ExpectedValue \left[ Y \Given x \right] f_{X} \left( x \right) dx
						= \int_{-\infty}^{\infty} \int_{-\infty}^{\infty} y f_{Y} \left( y \Given x \right) dy f_{X} \left( x \right) dx \\
					 \label{eq:Joint PDF of Multiple Discrete Random Variables}
						\ExpectedValue \left[ Y \right]
						= \ExpectedValue \left[ \ExpectedValue \left[ Y \Given X \right] \right]
						&= \sum_{x \in S_{X}} \ExpectedValue \left[ Y \Given x \right] p_{X} \left( x \right)
						= \sum_{x_{j} \in S_{X}} \sum_{y_{i} \in S_{Y}} y_{i} p_{Y} \left( y_{i} \Given x_{k} \right) p_{X} \left( x_{j} \right)
					\end{align}
				\end{subequations}
			\end{remark}
		\end{definition}
		\begin{proof}[Prove Expectation of Conditional Expected Value] \label{proof:Expected Value of Conditional Expected Value of Multiple Variables}
		\end{proof} % Section 7

\section{Random Vectors} \label{sec:Random Vectors}
Random Vectors are usually denoted:
	\begin{equation} \label{eq:Random Vector Notation}
		\vec{X} = \langle X_{1}, X_{2} X_{3}, \ldots, X_{n} \rangle
	\end{equation}
	\begin{definition}[Random Vector] \label{def:Random Vector}
		A \emph{random vector} is a list of \nameref{def:Random Variable, Full}s.
		\begin{remark}
			Almost all of the material for \nameref{sec:Multiple Random Variables} is applicable here.
			However, the 2 random variable equations and definitions must be generalized to $n$ random variables.
		\end{remark}
	\end{definition}
	
	\subsection{Joint CDF of a Random Vector} \label{subsec:Joint CDF of Random Vector}
		\begin{equation} \label{eq:Joint CDF of Random Vector}
			\begin{aligned}
				F_{\vec{X}} \left( \vec{x} \right) 
					&= F_{X_{1}, X_{2}, X_{3}, \ldots, X_{n}} \left( x_{1}, x_{2}, x_{3}, \ldots, x_{n} \right) \\
					&= P \left[ X_{1} \leq x_{1}, X_{2} \leq x_{2}, X_{3} \leq x_{3}, \ldots, X_{n} \leq x_{n} \right] \\
			\end{aligned}
		\end{equation}
		
	\subsection{Joint PDF of a Random Vector} \label{subsec:Joint PDF of Random Vector}
		\begin{equation} \label{eq:Joint PDF of Random Vector}
			f_{\vec{X}} \left( \vec{x} \right) = \frac{\partial^{n} F_{\vec{X}} \left( \vec{x} \right)}{\partial x_{1} \partial x_{2} \partial x_{3} \cdots \partial x_{n}}
		\end{equation}
		
		\subsubsection{Marginal PDF of a Random Vector} \label{subsubsec:Marginal PDF of Random Vector}
		Integrate out the terms that you're not interested in.
		\begin{equation} \label{eq:Marginal PDF of Random Vector}
			f_{\vec{X}} = \int_{-\infty}^{\infty} \cdots \int_{-\infty}^{\infty} f_{\vec{X}} \left( \vec{x} \right) \partial x_{2} \partial x_{3} \cdots \partial x_{n}
		\end{equation}
		For instance, say we want the marginal PDF of some function with respect to $X_{1}$, $X_{3}$, and $X_{4}$.
		\begin{equation} \label{eq:Marginal PDF of Random Vector Multiple Variables}
			f_{X_{1}, X_{3}, X_{4}} \left( x_{1}, x_{3}, x_{4} \right) = \int_{-\infty}^{\infty} \cdots \int_{-\infty}^{\infty} f_{\vec{X}} \left( \vec{x} \right) \partial x_{2} \partial x_{5} \partial x_{6} \cdots \partial x_{n}
		\end{equation}
	
	\subsection{Conditional Probability Functions of Random Vectors} \label{subsec:Random Vector Conditional Probability Functions}
	This section is just an extension of Section~\ref{subsec:Multiple Variable Conditional Probability Functions}, \nameref{subsec:Multiple Variable Conditional Probability Functions}.
	There are 3 major cases for these:
		\begin{enumerate}[noitemsep, nolistsep]
			\item \nameref{subsubsec:Conditional Probability Discrete Random Vectors}
			\item \nameref{subsubsec:Conditional Probability Mixed Random Vectors}
			\item \nameref{subsubsec:Conditional Probability Continuous Random Vectors}
		\end{enumerate}
	
		\begin{remark*} \label{rmk:Define Random Vector Y for Example}
			\begin{large}
				For the sections below, let $\vec{Y}= \langle Y_{1},Y_{2},Y_{3} \rangle$ and $\vec{y}= \langle y_{1},y_{2},y_{3} \rangle$.
			\end{large} \newline
			While I am using $\vec{Y}$ and $\vec{y}$, these equations can be further generalized to higher dimensions.
			All that would be required for this is to keep track of everything.
		\end{remark*}
	
		\subsubsection{Discrete Random Vectors} \label{subsubsec:Conditional Probability Discrete Random Vectors}
			\begin{definition}[Conditional Probability Mass Function] \label{def:Discrete Random Vector-Conditional PMF}
				The \emph{conditional Probability Mass Function (Conditional PMF)} of $Y_{3}$ given that $Y=y$ is:
				\begin{equation} \label{eq:Discrete Random Vector-Conditional PMF}
					p_{Y_{3}} \left( y_{3} \Given y_{1},y_{2} \right)
					= \frac{P \left[ \lbrace Y_{3}=y_{3} \rbrace \cap \left( \lbrace Y_{1}=y_{1} \rbrace \cap \lbrace Y_{2}=y_{2} \rbrace \right) \right]}{P \left[ \lbrace Y_{1}=y_{1} \rbrace \cap \lbrace Y_{2}=y_{2} \rbrace \right]} 
					= \frac{p_{\vec{Y}} \left( \vec{y} \right)}{p_{Y_{1},Y_{2}} \left( y_{1},y_{2} \right)}
				\end{equation}
				\begin{remark}
					This also implies that
					\begin{equation} \label{eq:Discrete Random Vector-Joint PMF}
						p_{\vec{Y}} \left( \vec{y} \right) = p_{Y_{3}} \left( y_{3} \Given y_{1},y_{2} \right) \cdot p_{Y_{2}} \left( y_{2} \Given y_{1} \right) \cdot p_{Y_{1}} \left( y_{1} \right)
					\end{equation}
				\end{remark}
				\begin{remark}
					If all elements of $\vec{Y}$ are \emph{independent} (Remember that you need to check each subgroup too, like shown in Section~\ref{subsec:Event Independence}), then:
					\begin{equation} \label{eq:Discrete Random Vector-Independent Conditional PMF}
						p_{Y_{3}} \left( y_{3} \Given y_{1},y_{2} \right)
						= \frac{p_{\vec{Y}} \left( \vec{y} \right)}{p_{Y_{1},Y_{2}} \left( y_{1},y_{2} \right)}
						= \frac{p_{Y_{1},Y_{2}} \left( y_{1},y_{2} \right) p_{Y_{3}} \left( y_{3} \right)}{p_{Y_{1},Y_{2}} \left( y_{1},y_{2} \right)}
						= p_{Y_{3}} \left( y_{3} \right)
					\end{equation}
				\end{remark}
				\begin{remark}
					The \nameref{def:Discrete Random Vector-Conditional PMF} of 2 discrete random variables satisfies all \nameref{subsubsec:Properties of Probability Mass Functions}.
				\end{remark}
			\end{definition}
		
		\subsubsection{Mixed Random Vectors} \label{subsubsec:Conditional Probability Mixed Random Vectors}
		
		
		\subsubsection{Continuous Random Vectors} \label{subsubsec:Conditional Probability Continuous Random Vectors}
		
	\subsection{Mean Vector} \label{subsec:Mean Vector}
		\begin{definition}[Mean Vector] \label{def:Mean Vector}
			For $\vec{X} = \langle X_{1},X_{2},\ldots,X_{n} \rangle$, the \emph{mean vector} is defined as the column vector of expected values of the components of $X_{k}$:
			\begin{equation} \label{eq:Mean Vector}
				\mathbf{m_{X}}
				= \ExpectedValue \left[ \vec{X} \right]
				= \begin{bmatrix}
					X_{1} \\
					X_{2} \\
					\vdots \\
					X_{n}
				\end{bmatrix}
				\triangleq \begin{bmatrix}
					\ExpectedValue \left[ X_{1} \right] \\
					\ExpectedValue \left[ X_{2} \right] \\
					\vdots \\
					\ExpectedValue \left[ X_{n} \right] \\
				\end{bmatrix}
			\end{equation}
			\begin{remark}
				Note that we defined the vector of expected values as a column vector.
				Other texts will use row vectors for other things, but the use of column vectors here is intentional.
			\end{remark}
		\end{definition}
	
	\subsection{Correlation and Covariance Matrix} \label{subsec:Correlation and Correlation Matrix}
		\begin{definition}[Correlation Matrix]
			The \emph{correlation matrix} has the second moments of $\bar{X}$ as its entries:
			\begin{equation} \label{eq:Correlation Matrix}
				\mathbf{\bar{R}_{X}}
				= \begin{bmatrix}
					\ExpectedValue \left[ X_{1}^{2} \right] & \ExpectedValue \left[ X_{1}X_{2} \right] & \cdots & \ExpectedValue \left[ X_{1}X_{n} \right] \\
					\ExpectedValue \left[ X_{2}X_{1} \right] & \ExpectedValue \left[ X_{2}^{2} \right] & \cdots & \ExpectedValue \left[ X_{2}X_{n} \right] \\
					\vdots & \vdots & \ddots & \vdots \\
					\ExpectedValue \left[ X_{n}X_{1} \right] & \ExpectedValue \left[ X_{n}X_{2} \right] & \cdots & \ExpectedValue \left[ X_{n}^{2} \right] \\
				\end{bmatrix} \\
			\end{equation}
			\begin{remark}
				$\bar{R}_{X}$ is a $n \times n$ symmetric matrix.
			\end{remark}
		\end{definition}
		\begin{definition}[Covariance Matrix] \label{def:Covariance Matrix}
			The \emph{covariance matrix} has the second-order central moments as its entries:
			\begin{equation} \label{eq:Covariance Matrix}
				\begin{aligned}
				\mathbf{\bar{K}_{X}}
				&= \begin{bmatrix}
					\ExpectedValue \left[ \left( X_{1} - m_{1} \right)^{2} \right] & \ExpectedValue \left[ \left( X_{1}-m_{1} \right) \left( X_{2}-m_{2} \right) \right] & \cdots & \ExpectedValue \left[ \left( X_{1}-m_{1} \right) \left( X_{n}-m_{n} \right) \right] \\
					\ExpectedValue \left[ \left( X_{2}-m_{2} \right) \left( X_{1}-m_{1} \right) \right] & \ExpectedValue \left[ \left( X_{2}-m_{2} \right)^{2} \right] & \cdots & \ExpectedValue \left[ \left( X_{2}-m_{2} \right) \left( X_{n}-m_{n} \right) \right] \\
					\vdots & \vdots & \ddots & \vdots \\
					\ExpectedValue \left[ \left( X_{n}-m_{n} \right) \left( X_{1}-m_{1} \right) \right] & \ExpectedValue \left[ \left( X_{n}-m_{n} \right) \left( X_{2}-m_{2} \right) \right] & \cdots & \ExpectedValue \left[ \left( X_{n}-m_{n} \right)^{2} \right] \\
				\end{bmatrix} \\
				&= \begin{bmatrix}
				\Variance \left[ X_{1} \right] & \Covariance \left[X_{1},X_{2} \right] & \cdots & \Covariance \left[ X_{1},X_{n} \right] \\
				\Covariance \left[ X_{2},X_{1} \right] & \Variance \left[ X_{2} \right]  & \cdots & \Covariance \left[ X_{2},X_{n} \right] \\
				\cdots & \cdots & \ddots & \cdots \\
				\Covariance \left[ X_{n},X_{1} \right] & \Covariance \left[ X_{2},X_{n} \right] & \cdots & \Variance \left[ X_{n} \right] \\
				\end{bmatrix}
				\end{aligned}
			\end{equation}
			\begin{remark}
				$\bar{K}_{X}$ is a $n \times n$ symmetric matrix.
			\end{remark}
			\begin{remark}
				The diagonal elements of $\bar{K}_{X}$ are given by the variances $\Variance \left[ X_{k} \right] = \ExpectedValue \left[ \left( X_{k}-m_{k} \right)^{2} \right]$ of the elements of $\vec{X}$.
			\end{remark}
			\begin{remark}
				If the diagonal elements of $\bar{K}_{X}$ are \nameref{rmk:Uncorrelated}, then $\Covariance \left[ X_{j}, X_{k} \right] = 0$ for $j \neq k$, and $\bar{K}_{X}$, the \nameref{def:Covariance Matrix} is a diagonal matrix.
			\end{remark}
			\begin{remark}
				If the random variables $X_{1},X_{2},\cdots,X_{n}$ are independent, then they are uncorrelated and $\bar{K}_{X}$ is diagonal.
			\end{remark}
			\begin{remark}
				If the \nameref{def:Mean Vector} is $\bar{0}$, that is, $m_{k} = \ExpectedValue \left[ X_{k} \right] = 0$ for all $k$, then $\bar{R}_{X} = \bar{K}_{X}$.
			\end{remark}
		\end{definition} % Section 8

\section{Sums of Random Variables} \label{sec:Sums of Random Variables}
	\begin{definition}[Sum of Random Variables] \label{def:Sum of Random Variables}
		The definition of a \emph{sum of random variables} is given in \Cref{eq:Sum of Random Variables} below.
		Where $X_{i}$ is a random variable,
		\begin{equation} \label{eq:Sum of Random Variables}
			S_{n} = \sum_{i=1}^{n} X_{i} = X_{1} + X_{2} + \ldots + X_{n}
		\end{equation}
	\end{definition}
	
	\subsection{Means and Variances of Sums of Random Variables} \label{subsec:Means and Variances of Sums of Random Variables}
		\begin{definition}[Mean of Sums of Random Variables] \label{def:Mean of Sums of Random Variables}
			The \emph{mean of sums of random variables} is the same as the \emph{expected value of sums of random variables}.
			\begin{equation}
				\ExpectedValue \left[ S_{n} \right] = \sum_{i=1}^{n} \ExpectedValue \left[ X_{i} \right]
			\end{equation}
			\begin{remark}
				All the properties of \nameref{subsubsec:Properties of Discrete Expected Value} and/or \nameref{subsubsec:Properties of Continuous Expected Value} hold true here as well..
			\end{remark}
		\end{definition}
		\begin{example}[Problem 7.1]{Mean of Sum of Random Variables}
                  Let $W = X + Y + Z$, where $X$, $Y$, and $Z$ are zero-mean, unit variance random variables with $\Covariance \left[ X,Y \right] = \frac{1}{2}$, $\Covariance \left[ Y,Z \right] = \frac{-1}{4}$, and $\Covariance \left[ X,Z \right] = \frac{1}{2}$.
                  Find the mean of $W$.

                  \tcblower

                  Solution to Problem 7.1, Part a, only Mean from Homework 10.
		\end{example}
		\begin{definition}[Variance of Sums of Random Variables] \label{def:Variance of Sums of Random Variables}
			The defintion of the \emph{variance of sums of random variables} is the same as we have been using them previously, \nameref{subsec:Variance of Single Discrete} and \nameref{subsec:Variance of Single Continuous}.
			\begin{equation} \label{eq:Variance of Sums of Random Variables}
				\Variance \left[ S_{n} \right] = \Variance \left[ \sum_{i=1}^{n} X_{i} \right] = \sum_{i=1}^{n} \Variance \left[ X_{i} \right] + \sum_{j=1}^{n} \sum_{\substack{k=1 \\ j \neq k}}^{n} \Covariance \left[ X_{j}, X_{k} \right]
			\end{equation}
			\begin{remark} \label{rmk:Variance of Sums of Independent Random Variables}
				If $X_{1}, X_{2}, \ldots , X_{n}$ are independent, then:
				\begin{equation} \label{eq:Variance of Sums of Independent Random Variables}
					\Variance \left[ \sum_{i=1}^{n} X_{i} \right] = \sum_{i=1}^{n} \Variance \left[ X_{i} \right]
				\end{equation}
			\end{remark}
		\end{definition}
		\begin{example}[Problem 7.1]{Variance of Sum of Random Variables}
                  Let $W = X + Y + Z$, where $X$, $Y$, and $Z$ are zero-mean, unit variance random variables with $\Covariance \left[ X,Y \right] = \frac{1}{2}$, $\Covariance \left[ Y,Z \right] = \frac{-1}{4}$, and $\Covariance \left[ X,Z \right] = \frac{1}{2}$.
                  Find the variance of $W$.

                  \tcblower

                  Solution to Problem 7.1, Part a, only Variance from Homework 10.
		\end{example}
		\begin{example}[Problem 7.1]{Mean and Variance of Sum of Uncorrelated Random Variables}
                  Let $W = X + Y + Z$, where $X$, $Y$, and $Z$ are zero-mean, unit variance random variables with $\Covariance \left[ X,Y \right] = \frac{1}{2}$, $\Covariance \left[ Y,Z \right] = \frac{-1}{4}$, and $\Covariance \left[ X,Z \right] = \frac{1}{2}$.
                  Find the mean and variance of $W$ assuming that $X$, $Y$< and $Z$ are uncorrelated random variables.

                  \tcblower

                  Solution to Problem 7.2, Part b, from Homework 10.
		\end{example}
		\begin{definition}[Independent and Identically Distributed] \label{def:Independent and Identically Distributed}
			We say that $X_{1},X_{2},\ldots,X_{n}$ are \emph{Independent and Identically Distributed (iid)} random variables if $X_{i}$ are drawn independently from the same population/probability distribution.
			\begin{subequations}
				\begin{equation} \label{eq:Mean of Independent and Identically Distributed}
					\sum_{i=1}^{n} \ExpectedValue \left[ X_{i} \right] = n \mu
				\end{equation}
				\begin{equation} \label{eq:Variance of Independent and Identically Distributed}
					\Variance \left[ S_{n} \right] = n \sigma^{2}
				\end{equation}
			\end{subequations}
			\begin{itemize}[noitemsep, nolistsep]
				\item $\mu$ is the mean of a random variable $X_{i}$
				\item $\sigma^{2}$ is the variance of a random variable $X_{i}$.
			\end{itemize}
		\end{definition}

%%% Local Variables:
%%% mode: latex
%%% TeX-master: "../Math_374-Reference_Sheet"
%%% End:
 % Section 9

\section{Statistics} \label{sec:Statistics}
In applying probability models to real situations, we perform experiments and collect data to answer questions such as:
	\begin{enumerate}[noitemsep, nolistsep]
		\item What are the values of the parameters of the distribution of a random variable of interest?
			\begin{itemize}[noitemsep, nolistsep]
				\item Mean or Expected value
				\item Variance
			\end{itemize}
		
		\item Is the data set consistent with some model?
			\begin{itemize}[noitemsep, nolistsep]
				\item Some assumed distribution, which must be true, otherwise the model is wrong.
			\end{itemize}
		
		\item Is the data set consistent with some parameter value of the assumed value?
	\end{enumerate}

	\begin{definition}[Random Sample] \label{def:Random Sample}
		A \emph{random sample} is a set of $n$ \nameref{def:Random Variable, Full} or \nameref{def:Statistic} that are drawn with \nameref{def:Independent and Identically Distributed}.
		\begin{equation} \label{eq:Random Sample}
			\mathbf{X}_{n} = \left( X_{1},X_{2},\ldots,X_{n} \right)
		\end{equation}
		\begin{remark}
			This is \emph{similar} to the definition of a \nameref{def:Random Vector}.
			The difference here is that the values in a \nameref{def:Random Sample} must be \nameref{def:Independent and Identically Distributed} and \emph{may} be related to each other somehow.
		\end{remark}
		\begin{remark}[Random Sample Parameters] \label{rmk:Random Sample Parameters}
			These are an additional variable that is added onto the \nameref{def:Probability Density Function} or \nameref{def:Probability Mass Function}.
			When we were using these functions in the previous sections, these parameters were either constant or assumed to be constant.
			When considering these samples in \nameref{sec:Statistics}, you must also account for the \nameref{rmk:Random Sample Parameters}.
		\end{remark}
	\end{definition}
	\begin{definition}[Statistic] \label{def:Statistic}
		A \emph{statistic} $W (\mathbf{X}_{n})$ is a function of the random sample $X_{1},X_{2},\ldots,X_{n}$.
		\begin{equation} \label{eq:Statistic}
			W \left( \mathbf{X}_{n} \right) = g \left( X_{1},X_{2},\ldots,X_{n} \right)
		\end{equation}
	\end{definition}
	\begin{definition}[Unit Variance] \label{def:Unit Variance}
		The \emph{unit variance} means that the standard deviation, $\sigma$ of a sample, as well as the variance, $\sigma^{2}$ will tend towards 1 as the sample size increases to infinity.
	\end{definition}
	
	\subsection{Sample Mean} \label{subsec:Sample Mean}
		\begin{definition}[Sample Mean] \label{def:Sample Mean}
			The \emph{sample mean} of a sequence is denoted as,
			\begin{equation} \label{eq:Sample Mean}
				\bar{X} = M_{n} = \frac{\sum_{i=1}^{n} X_{i}}{n}
			\end{equation}
		\end{definition}
		\begin{definition}[Expected Value of Sample Mean] \label{def:Expected Value of Sample Mean}
			The \emph{expected value of the sample mean} is defined as:
			\begin{equation} \label{eq:Expected Value of Sample Mean}
				\ExpectedValue \left[ \bar{X} \right]
				= \ExpectedValue \left[ M_{n} \right]
				= \frac{\ExpectedValue \left[ S_{n} \right]}{n}
				= \frac{n \mu}{n}
				= \mu
			\end{equation}
			\begin{remark}
				The sample mean $M_{n}$ is an \emph{\nameref{def:Unbiased Estimator}} of population mean $\mu$.
			\end{remark}
		\end{definition}
		\begin{definition}[Variance of Sample Mean] \label{def:Variance of Sample Mean}
			The \emph{variance of the sample mean} is denoted as:
			\begin{equation} \label{eq:Variance of Sample Mean}
				\Variance \left[ \bar{X} \right]
				= \Variance \left[ M_{n} \right]
				= \Variance \left[ \frac{S_{n}}{n} \right]
				= \frac{1}{n^{2}} \Variance \left[ S_{n} \right]
				= \frac{\sigma^{2}}{n}
			\end{equation}
			\begin{remark}
				The larger $n$ gets, the smaller $\Variance \left[ M_{n} \right]$ gets, and the closer $M_{n}$ gets to $\mu$.
			\end{remark}
		\end{definition}
			
	Also, we can use the \nameref{eq:Chebychev Inequality} to approximate many values. In this case, we change the Chebychev Inequality from Equation~\eqref{eq:Chebychev Inequality} to Equation~\eqref{eq:Statistics Chebychev Inequality} like so:
		\begin{equation} \label{eq:Statistics Chebychev Inequality}
			P \left[ \lvert M_{n} - \ExpectedValue \left[ M_{n} \right] \rvert \geq \varepsilon \right] \leq \frac{\Variance \left[ M_{n} \right]}{\varepsilon^{2}}
		\end{equation}
		
	\subsection{Important Probability and Statistics Theorems} \label{subsec:Important Probability and Statistics Theorems}
		There are 3 very import theorems that are used quite frequently in both \nameref{sec:Probability Theory} and \nameref{sec:Statistics}.
		\begin{enumerate}[noitemsep, nolistsep]
			\item \nameref{thm:Weak Law of Large Numbers}
			\item \nameref{thm:Strong Law of Large Numbers}
			\item \nameref{thm:Central Limit Theorem}
		\end{enumerate}
		\begin{theorem}[Weak Law of Large Numbers] \label{thm:Weak Law of Large Numbers}
			Let $X_{1},X_{2},\ldots,X_{n}$ be a sequence of \nameref{def:Independent and Identically Distributed} random variables form a population with mean $\ExpectedValue \left[ X \right] = \mu$, then for $\varepsilon > 0$,
			\begin{equation} \label{eq:Weak Law of Large Numbers}
				\lim\limits_{n \rightarrow \infty} P \left[ \lvert M_{n} - \mu \rvert < \varepsilon \right] = 1
			\end{equation}
			\begin{remark*}
				In words this means, for large enough fixed values of $n$, $M_{n}$ is close to $\mu$ with high probability.
			\end{remark*}
		\end{theorem}
		\begin{theorem}[Strong Law of Large Numbers] \label{thm:Strong Law of Large Numbers}
			Let $X_{1},X_{2},\ldots,X_{n}$ be a sequence of \nameref{def:Independent and Identically Distributed} random variables form a population with mean $\ExpectedValue \left[ X \right] = \mu$ and finite variance, then
			\begin{equation} \label{eq:Strong Law of Large Numbers}
				P \left[ \lim\limits_{n \rightarrow \infty} M_{n} = \mu \right] = 1
			\end{equation}
			\begin{remark*}
				With probability 1, every sequence of sample mean calculations will eventually approach and stay close to the population mean.
			\end{remark*}
		\end{theorem}
		\begin{theorem}[Central Limit Theorem] \label{thm:Central Limit Theorem}
			Let $X_{1},X_{2},\ldots,X_{n}$ be a sequence of \nameref{def:Independent and Identically Distributed} random variables form a population with mean $\ExpectedValue \left[ X \right] = \mu < \infty$ and finite variance $\sigma^{2}$ and let
			\begin{equation*}
				Z_{n} = \frac{S_{n} - n\mu}{\sigma \sqrt{n}} = \frac{\bar{X} - \mu}{\frac{\sigma}{\sqrt{n}}}
			\end{equation*}
			then,
			\begin{equation} \label{eq:Central Limit Theorem}
				\lim\limits_{n \rightarrow \infty} P \left[ Z_{n} \leq z \right] = \frac{1}{\sqrt{2 \pi}} \int_{-\infty}^{z} e^{-\frac{x^{2}}{2}} dx
			\end{equation}
			\begin{itemize} % I preferred the way it looked without noitemsep/nolistsep
				\item $S_{n}=X_{1}+X_{2}+\ldots+X_{n}$, The sum of the random variables
				\item $\mu=\ExpectedValue \left[ X_{1} \right]$, The mean for an individual random variable
				\item $\sigma=\sqrt{\Variance \left[ X_{1} \right]}$, The variance of an individual random variable
				\item $n$ is the number of trials/recordings/samples/etc.
				\item $\bar{X} = \frac{1}{n} \sum\limits_{i=1}^{n} X_{i}$, The \nameref{def:Sample Mean}
			\end{itemize}
			\begin{remark*}
				This means that over time, as you gain more and more sample means, they will start to resemble the \nameref{def:Gaussian Random Variable}, or the Normal Random Variable.
			\end{remark*}
		\end{theorem}
		\begin{tcolorbox}[title=\textbf{\nameref{thm:Central Limit Theorem} Example}, colbacktitle=white!100!black, coltitle=black!100!white, colback=white!100!black, sharp corners=all] \label{ex:Central Limit Theorem}
			Problem 7.25 from Homework 11 (Extra Credit).
		\end{tcolorbox}
	
	\subsection{Estimators} \label{subsec:Estimators}
		\begin{itemize}[noitemsep, nolistsep]
			\item A \nameref{def:Statistic} is a function of the data $X_{1},X_{2},\ldots,X_{n}$
			\item An \emph{estimator} for a parameter, $\theta$, usually denoted $\hat{\theta}$, is also a statistic
		\end{itemize}
		\begin{definition}[Unbiased Estimator] \label{def:Unbiased Estimator}
			In general we say that a \nameref{def:Statistic} $\Theta (X)$ (a function of data $X_{1},X_{2},\ldots,X_{n}$) is an \emph{unbiased estimator} of a parameter $\theta$ if $\ExpectedValue \left[ W \left( \mathbf{X} \right) \right] = \theta$.
			\begin{remark}[What makes a good estimator of any parameter, $\theta$?]
				A \emph{good estimator} of any parameter, $\theta$, should:
				\begin{itemize}[noitemsep, nolistsep]
					\item Give the correct value of $\theta$
					\item Not vary too much around $\theta$
				\end{itemize}
			\end{remark}
			\begin{remark}
				This is the definition of \emph{unbiased}, drawn from the definition of \nameref{def:Bias}
			\end{remark}
		\end{definition}
	
		\subsubsection{Goodness of an Estimator} \label{subsubsec:Estimator Goodness}
		There are 4 measures we use to determine how good our estimator is.
			\begin{enumerate}[noitemsep, nolistsep]
				\item \nameref{def:Bias}
				\item \nameref{def:Variance of Sample Mean}
				\item \nameref{def:Mean Squared Error}
				\item \nameref{def:Consistency}
			\end{enumerate}
		If our estimator is an \nameref{def:Unbiased Estimator}, then:
			\begin{itemize}
				\item Accuracy is defined as $\Bias [\hat{\theta} ] = \ExpectedValue [\hat{\theta}] - \theta$
				\item Precision is defined as $\Variance [\hat{\theta}]$
			\end{itemize}
			\begin{definition}[Bias] \label{def:Bias}
				\emph{Bias} is defined as:
				\begin{equation} \label{eq:Accuracy}
					\Bias [ \hat{\Theta} ] = \ExpectedValue [ \hat{\Theta} ] - \theta
				\end{equation}
				\begin{remark} \label{rmk:Unbiased}
					The estimator $\hat{\Theta}$ is \emph{unbiased} for $\theta$ if
					\begin{equation} \label{Unbiased}
						\ExpectedValue [ \hat{\Theta} ] = \theta
					\end{equation}
				\end{remark}
			\end{definition}
					
			\begin{definition}[Mean Squared Error] \label{def:Mean Squared Error}
				The \emph{Mean Squared Error} of an estimator for parameter $\hat{\theta}$ is:
				\begin{equation} \label{eq:Mean Squared Error}
					\MeanSqErr [ \hat{\theta} ]
					= \ExpectedValue \left[ \left( \hat{\Theta} - \theta \right)^{2} \right]
					= \Variance \left[ \hat{\theta} \right] + \left( \Bias \left[ \hat{\theta} \right] \right)^{2}
				\end{equation}
			\end{definition}
			\begin{remark*}
				When doing statistical analysis, there is something called the \emph{Bias-Variance Tradeoff}.
				When doing the analysis, if you try to minimize bias, your variance will increase and vice-versa.
				There is a happy medium, which is not discussed in this class.
			\end{remark*}
			\begin{definition}[Consistency] \label{def:Consistency}
				$\hat{\theta}$ is a \emph{consistent estimator} for $\theta$ if $\hat{\Theta}$ converges to $\theta$ in probability.
				\begin{equation} \label{eq:Consistency}
					\lim\limits_{n \rightarrow \infty} \Prob \left[ \lvert \hat{\Theta} - \theta \rvert > \varepsilon \right] = 0
				\end{equation}
			\end{definition}
	
	\subsection{How to Find a Good Estimator} \label{subsec:Find Good Estimator}
	There are several methods, two of which are:
		\begin{enumerate}[noitemsep, nolistsep]
			\item \nameref{subsubsec:Method of Moments}
				\begin{itemize}[noitemsep, nolistsep]
					\item Sample Moments and Population Moments, $\bar{X}_{n} = \mu$
					\item You needs as many moments as parameters to get enough equations
				\end{itemize}
			\item \nameref{def:Maximum Likelihood Estimation}
		\end{enumerate}
	
		\subsubsection{Method of Moments} \label{subsubsec:Method of Moments}
			\begin{tcolorbox}[title=\textbf{\nameref{subsubsec:Method of Moments} Example}, colbacktitle=white!100!black, coltitle=black!100!white, colback=white!100!black, sharp corners=all] \label{ex:Method of Moments}
				Problem 8.6 from Homework 11 (Extra Credit).
			\end{tcolorbox}
		
		\subsubsection{Maximum Likelihood Estimation} \label{subsubsec:Maximum Likelihood Estimation}
			\begin{definition}[Maximum Likelihood Estimation] \label{def:Maximum Likelihood Estimation}
				Let $ X_{1},X_{2},\ldots,X_{n} \DrawnIID f \left( x \Given \theta \right) $.
				\begin{equation} \label{eq:Maximum Likelihood Estimation}
					\hat{\Theta}_{\MaxLikeEstim} = \argmax_{\theta \in \Theta} \Likelihood \left( \theta \Given x_{1},x_{2},\ldots,x_{n} \right)
				\end{equation}
				\begin{remark} \label{rmk:Likelihood}
					Likelihood, denoted $\Likelihood$ is defined as the \nameref{def:Joint PDF} of the \nameref{def:Random Sample} and its \nameref{rmk:Random Sample Parameters}.
					\begin{equation} \label{eq:Likelihood}
						\Likelihood \left( \theta \Given x_{1},x_{2},\ldots,x_{n} \right) = f_{X_{1}} \left( x_{1} \Given \theta \right) \cdot f_{X_{2}} \left( x_{2} \Given \theta \right) \cdot \ldots \cdot f_{X_{n}} \left( x_{n} \Given \theta \right)
					\end{equation}
				\end{remark}
				\begin{remark}
					It is often easier to maximize $\hat{\Theta}_{\MaxLikeEstim}$ over the log-likelihood.
					\begin{equation*}
						\hat{\Theta}_{\MaxLikeEstim} = \argmax_{\theta \in \Theta} \log \Likelihood \left( \theta \Given x_{1},x_{2},\ldots,x_{n} \right)
					\end{equation*}
				\end{remark}
				\begin{remark}
					\[ \argmax_{\theta \in \Theta} \] is the global maxima of the function. This is further described in Definition~\ref{def:argmax}.
				\end{remark}
			\end{definition}
	
	\subsection{Confidence Intervals} \label{subsec:Confidence Interval}
	Because certain estimators are not discrete, but continuous, the estimators expected value might not be quite right.
	This is where \ref{def:Confidence Interval}s come in.
		\begin{definition}[Confidence Interval] \label{def:Confidence Interval}
			A \emph{confidence interval} is an interval or set of values that is highly likely to contain the true value of the parameter.
		\end{definition}
		\begin{definition}[$1-\alpha$ Confidence Interval] \label{def:1-alpha Confidence Interval}
			The \emph{$1-\alpha$ confidence interval} is a \nameref{def:Confidence Interval} where we estimate the probability of the parameter, $\theta$, being in the random interval to $1-\alpha$.
			The problem is, find a random interval $\left[ \ell \left( \mathbf{X} \right), u \left( \mathbf{X} \right) \right]$ such that
			\begin{equation} \label{eq:1-alpha Confidence Interval}
				\Prob \left[ \ell \left( \mathbf{X}_{n} \right) , u \left( \mathbf{X}_{n} \right) \right] = 1-\alpha
			\end{equation}
			\begin{remark}
				$\ell \left( \mathbf{X} \right)$ is the \emph{lower bound of the random interval}.
				$u \left( \mathbf{X} \right)$ is the \emph{upper bound of the random interval}.
				\textbf{Only ONE of these may be $\infty$ at a time.} Otherwise, you're including the whole sample space, making the confidence interval useless.
			\end{remark}
			\begin{remark}
				If the problem says, with a confidence interval of 95\%, that is the $1-\alpha$ portion; i.e. $1-\alpha = 0.95 \%$.
			\end{remark}
		\end{definition}
	
	\subsection{Hypothesis Testing} \label{subsec:Hypothesis Testing}
	There are 4 parts to \nameref{subsec:Hypothesis Testing}
		\begin{enumerate}[noitemsep, nolistsep]
			\item Identify the hypotheses. $H_{0}$ is the \emph{null hypothesis}. It is usually compared against another, complementary hypothesis, $H_{1}$.
			\item The Rejection Region, which is evidence against the null hypothesis that we may or may not choose to include. $R \subset \mathcal{X}$, where $\mathcal{X}$ is the sample space.
			\item The Decision Rule:
				\begin{enumerate}[noitemsep, nolistsep]
					\item Reject $H_{0}$ if $X \in R$
					\item Accept/Do not reject $H_{0}$ if $X \notin R$
				\end{enumerate}
			\item The Test Statistic
		\end{enumerate}
	There are 2 errors that can occur in \nameref{subsec:Hypothesis Testing}.
		\begin{enumerate}[label=\textbf{Type \Roman* Error: }, ref=Hypothesis Testing Type \Roman* Error, align=left, noitemsep, nolistsep]
			\item Reject $H_{0}$ if $H_{0}$ is true.
			\item Accept $H_{0}$ if $H_{0}$ is false.
		\end{enumerate} % Section 10

%====================================APPENDIX====================================
\appendix
\counterwithin{equation}{section}
\counterwithin{definition}{subsection}

\clearpage
\subsection{Trigonometry} \label{app:Trig}
	\subsubsection{Trigonometric Formulas} \label{subsubsec:Trig Formulas}
		\begin{equation} \label{eq:Sin plus Sin with diff Angles}
			\sin \left( \alpha \right) + \sin \left( \beta \right) = 2 \sin \left( \frac{\alpha + \beta}{2} \right) \cos\left( \frac{\alpha - \beta}{2} \right)  
		\end{equation}
		\begin{equation} \label{eq:Cosine-Sine Product}
			\cos \left( \theta \right) \sin \left( \theta \right) = \frac{1}{2} \sin \left( 2 \theta \right)
		\end{equation}
	
	\subsubsection{Euler Equivalents of Trigonometric Functions} \label{subsubsec:Euler Equivalents}
		\begin{equation} \label{eq:Euler Sin}
			\sin \left( x \right) = \frac{e^{\imath x} + e^{-\imath x}}{2}
		\end{equation}
		\begin{equation} \label{eq:Euler Cos}
			\cos \left( x \right) = \frac{e^{\imath x} - e^{-\imath x}}{2 \imath}
		\end{equation}
		\begin{equation} \label{eq:Euler Sinh}
			\sinh \left( x \right) = \frac{e^{x} - e^{-x}}{2}
		\end{equation}
		\begin{equation} \label{eq:Euler Cosh}
			\cosh \left( x \right) = \frac{e^{x} + e^{-x}}{2}
		\end{equation}

\clearpage
\section{Calculus}\label{app:Calculus}
\subsection{L'Hopital's Rule}\label{subsec:LHopitals_Rule}
L'Hopital's Rule can be used to simplify and solve expressions regarding limits that yield irreconcialable results.
\begin{lemma}[L'Hopital's Rule]\label{lemma:LHopitals_Rule}
  If the equation
  \begin{equation*}
    \lim\limits_{x \rightarrow a} \frac{f(x)}{g(x)} =
    \begin{cases}
      \frac{0}{0} \\
      \frac{\infty}{\infty} \\
    \end{cases}
  \end{equation*}
  then \Cref{eq:LHopitals_Rule} holds.
  \begin{equation}\label{eq:LHopitals_Rule}
    \lim\limits_{x \rightarrow a} \frac{f(x)}{g(x)} = \lim\limits_{x \rightarrow a} \frac{f'(x)}{g'(x)}
  \end{equation}
\end{lemma}

\subsection{Fundamental Theorems of Calculus}\label{subsec:Fundamental Theorem of Calculus}
\begin{definition}[First Fundamental Theorem of Calculus]\label{def:1st Fundamental Theorem of Calculus}
  The \emph{first fundamental theorem of calculus} states that, if $f$ is continuous on the closed interval $\left[ a,b \right]$ and $F$ is the indefinite integral of $f$ on $\left[ a,b \right]$, then

  \begin{equation}\label{eq:1st Fundamental Theorem of Calculus}
    \int_{a}^{b}f \left( x \right) dx = F \left( b \right) - F \left( a \right)
  \end{equation}
\end{definition}

\begin{definition}[Second Fundamental Theorem of Calculus]\label{def:2nd Fundamental Theorem of Calculus}
  The \emph{second fundamental theorem of calculus} holds for $f$ a continuous function on an open interval $I$ and $a$ any point in $I$, and states that if $F$ is defined by

  \begin{equation*}
    F \left( x \right) = \int_{a}^{x} f \left( t \right) dt,
  \end{equation*}
  then
  \begin{equation}\label{eq:2nd Fundamental Theorem of Calculus}
    \begin{aligned}
      \frac{d}{dx} \int_{a}^{x} f \left( t \right) dt &= f \left( x \right) \\
      F' \left( x \right) &= f \left( x \right) \\
    \end{aligned}
  \end{equation}
\end{definition}

\begin{definition}[argmax]\label{def:argmax}
  The arguments to the \emph{argmax} function are to be maximized by using their derivatives.
  You must take the derivative of the function, find critical points, then determine if that critical point is a global maxima.
  This is denoted as
  \begin{equation*}\label{eq:argmax}
    \argmax_{x}
  \end{equation*}
\end{definition}

\subsection{Rules of Calculus}\label{subsec:Rules of Calculus}
\subsubsection{Chain Rule}\label{subsubsec:Chain Rule}
\begin{definition}[Chain Rule]\label{def:Chain Rule}
  The \emph{chain rule} is a way to differentiate a function that has 2 functions multiplied together.

  If
  \begin{equation*}
    f(x) = g(x) \cdot h(x)
  \end{equation*}
  then,
  \begin{equation}\label{eq:Chain Rule}
    \begin{aligned}
      f'(x) &= g'(x) \cdot h(x) + g(x) \cdot h'(x) \\
      \frac{df(x)}{dx} &= \frac{dg(x)}{dx} \cdot g(x) + g(x) \cdot \frac{dh(x)}{dx} \\
    \end{aligned}
  \end{equation}
\end{definition}

\subsection{Useful Integrals}\label{subsec:Useful_Integrals}
\begin{equation}\label{eq:Cosine_Indefinite_Integral}
  \int \cos(x) \; dx = \sin(x)
\end{equation}

\begin{equation}\label{eq:Sine_Indefinite_Integral}
  \int \sin(x) \; dx = -\cos(x)
\end{equation}

\begin{equation}\label{eq:x_Cosine_Indefinite_Integral}
  \int x \cos(x) \; dx = \cos(x) + x \sin(x)
\end{equation}
\Cref{eq:x_Cosine_Indefinite_Integral} simplified with Integration by Parts.

\begin{equation}\label{eq:x_Sine_Indefinite_Integral}
  \int x \sin(x) \; dx = \sin(x) - x \cos(x)
\end{equation}
\Cref{eq:x_Sine_Indefinite_Integral} simplified with Integration by Parts.

\begin{equation}\label{eq:x_Squared_Cosine_Indefinite_Integral}
  \int x^{2} \cos(x) \; dx = 2x \cos(x) + (x^{2} - 2) \sin(x)
\end{equation}
\Cref{eq:x_Squared_Cosine_Indefinite_Integral} simplified by using Integration by Parts twice.

\begin{equation}\label{eq:x_Squared_Sine_Indefinite_Integral}
  \int x^{2} \sin(x) \; dx = 2x \sin(x) - (x^{2} - 2) \cos(x)
\end{equation}
\Cref{eq:x_Squared_Sine_Indefinite_Integral} simplified by using Integration by Parts twice.

\begin{equation}\label{eq:Exponential_Cosine_Indefinite_Integral}
  \int e^{\alpha x} \cos(\beta x) \; dx = \frac{e^{\alpha x} \bigl( \alpha \cos(\beta x) + \beta \sin(\beta x) \bigr)}{\alpha^{2} + \beta^{2}} + C
\end{equation}

\begin{equation}\label{eq:Exponential_Sine_Indefinite_Integral}
  \int e^{\alpha x} \sin(\beta x) \; dx = \frac{e^{\alpha x} \bigl( \alpha \sin(\beta x) - \beta \cos(\beta x) \bigr)}{\alpha^{2}+\beta^{2}} + C
\end{equation}

\begin{equation}\label{eq:Exponential_Indefinite_Integral}
  \int e^{\alpha x} \; dx = \frac{e^{\alpha x}}{\alpha}
\end{equation}

\begin{equation}\label{eq:x_Exponential_Indefinite_Integral}
  \int x e^{\alpha x} \; dx = e^{\alpha x} \left( \frac{x}{\alpha} - \frac{1}{\alpha^{2}} \right)
\end{equation}
\Cref{eq:x_Exponential_Indefinite_Integral} simplified with Integration by Parts.

\begin{equation}\label{eq:Inverse_x_Indefinite_Integral}
  \int \frac{dx}{\alpha + \beta x} = \int \frac{1}{\alpha + \beta x} \; dx = \frac{1}{\beta} \ln (\alpha + \beta x)
\end{equation}

\begin{equation}\label{eq:Inverse_x_Squared_Indefinite_Integral}
  \int \frac{dx}{\alpha^{2} + \beta^{2} x^{2}} = \int \frac{1}{\alpha^{2} + \beta^{2} x^{2}} \; dx = \frac{1}{\alpha \beta} \arctan \left( \frac{\beta x}{\alpha} \right)
\end{equation}

\begin{equation}\label{eq:a_Exponential_Indefinite_Integral}
  \int \alpha^{x} \; dx = \frac{\alpha^{x}}{\ln(\alpha)}
\end{equation}

\begin{equation}\label{eq:a_Exponential_Derivative}
  \frac{d}{dx} \alpha^{x} = \frac{d\alpha^{x}}{dx} = \alpha^{x} \ln(x)
\end{equation}

\subsection{Leibnitz's Rule}\label{subsec:Leibnitzs_Rule}
\begin{lemma}[Leibnitz's Rule]\label{lemma:Leibnitzs_Rule}
  Given
  \begin{equation*}
    g(t) = \int_{a(t)}^{b(t)} f(x, t) \, dx
  \end{equation*}
  with $a(t)$ and $b(t)$ differentiable in $t$ and $\frac{\partial f(x, t)}{\partial t}$ continuous in both $t$ and $x$, then
  \begin{equation}\label{eq:Leibnitzs_Rule}
    \frac{d}{dt} g(t) = \frac{d g(t)}{dt} = \int_{a(t)}^{b(t)} \frac{\partial f(x, t)}{\partial t} \, dx + f \bigl[ b(t), t \bigr] \, \frac{d b(t)}{dt} - f \bigl[ a(t), t \bigr] \, \frac{d a(t)}{dt}
  \end{equation}
\end{lemma}



\end{document}