\documentclass[10pt,letterpaper,final,twoside,notitlepage]{article}
\usepackage[margin=.5in]{geometry} % 1/2 inch margins on all pages
\usepackage[utf8]{inputenc} % Define the input encoding
\usepackage[USenglish]{babel} % Define language used
\usepackage{amsmath}
\usepackage{mathtools} % Allow for text and math in align* environment.
\usepackage{amsfonts}
\usepackage{amssymb}
\usepackage{amsthm} % Gives us plain, definition, and remark to use in \theoremstyle{style}
\usepackage{thmtools}
\usepackage{thm-restate}
\usepackage{graphicx}

\usepackage[
backend=biber,
style=alphabetic,
citestyle=authoryear]{biblatex} % Must include citation somewhere in document to print bibliography
\usepackage{hyperref} % Generate hyperlinks to referenced items
\usepackage[nottoc]{tocbibind} % Prints the Reference/Bibliography in TOC as well
\usepackage[noabbrev,nameinlink]{cleveref} % Fancy cross-references in the document everywhere
\usepackage{nameref} % Can make references by name to places
\usepackage{caption} % Allows for greater control over captions in figure, algorithm, table, etc. environments
\usepackage{subcaption} % Allows for multiple figures in one Figure environment
\usepackage[binary-units=true]{siunitx} % Gives us ways to typeset units for stuff
\usepackage{csquotes} % Context-sensitive quotation facilities
\usepackage{enumitem} % Provides [noitemsep, nolistsep] for more compact lists
\usepackage{chngcntr} % Allows us to tamper with the counter a little more
\usepackage{empheq} % Allow boxing of equations in special math environments
\usepackage[x11names]{xcolor} % Gives access to coloring text in environments or just text, MUST be before tikz
\usepackage{tcolorbox} % Allows us to create boxes of various types for examples
\usepackage{tikz} % Allows us to create TikZ and PGF Pictures
\usepackage{ctable} % Greater control over tables and how they look
\usepackage{multirow} % Allow us to have a single cell in a table span multiple rows
\usepackage{titling} % Put document information throughout the document programmatically
\usepackage[linesnumbered,ruled,vlined]{algorithm2e} % Allows us to write algorithms in a nice style.

\counterwithin{figure}{section}
\counterwithin{table}{section}
\counterwithin{equation}{section}
\counterwithin{algocf}{section}
\crefname{algocf}{algorithm}{algorithms}
\Crefname{algocf}{Algorithm}{Algorithms}
\setcounter{secnumdepth}{4}
\setcounter{tocdepth}{4} % Include \paragraph in toc
\crefname{paragraph}{paragraph}{paragraphs}
\Crefname{paragraph}{Paragraph}{Paragraphs}

% Create a theorem environment
\theoremstyle{plain}
\newtheorem{theorem}{Theorem}[section]
% Create a numbered theorem-like environment for lemmas
\newtheorem{lemma}{Lemma}[theorem]

% Create a definition environment
\theoremstyle{definition}
\newtheorem{definition}{Defn}
\newtheorem{corollary}{Corollary}[section]
% \begin{definition}[Term] \label{def:}
% 		Make sure the term is emphasized with \emph{term}.
%		This ensures that if \emph is changed, it shows up everywhere
% \end{definition}

% Create a numbered remark environment numbered based on definition
% NOTE: This version of remark MUST go inside a definition environment
\theoremstyle{remark}
\newtheorem{remark}{Remark}[definition]
%\counterwithin{definition}{subsection} % Uncomment to have definitions use section.subsection numbering

% Create an unnumbered remark environment for general use
% NOTE: This version of remark has NO restrictions on placement
\newtheorem*{remark*}{Remark}

% Create a special list that handles properties. It can be broken and restarted
\newlist{propertylist}{enumerate}{1} % {Name}{Template}{Max Depth}
% [newlistname, LevelsToApplyTo]{formatting options}
\setlist[propertylist, 1]{label=\textbf{(\roman*)}, ref=\textbf{(\roman*)}, noitemsep, nolistsep}
\crefname{propertylisti}{property}{properties}
\Crefname{propertylisti}{Property}{Properties}

% Create a special list that handles enumerate starting with lower letters. Breakable/Restartable.
\newlist{boldalphlist}{enumerate}{1} % {Name}{Template}{Max Depth}
% [newlistname, LevelsToApplyTo]{formatting options}
\setlist[boldalphlist, 1]{label=\textbf{(\alph*)}, ref=\alph*, noitemsep, nolistsep} % Set options

\newlist{nocrefenumerate}{enumerate}{1} % {Name}{Template}{Max Depth}
% [newlistname, LevelsToApplyTo]{formatting options}
\setlist[nocrefenumerate, 1]{label=(\arabic*), ref=(\arabic*), noitemsep, nolistsep}

% Create a list that allows for deeper nesting of numbers. By default enumerate only allows depth=4.
\newlist{nestednums}{enumerate}{6}
% [newlistname, LevelsToApplyTo]{formatting options}
\setlist[nestednums]{noitemsep, label*=\arabic*.}

\tcbuselibrary{breakable} % Allow tcolorboxes to be broken across pages
% Create a tcolorbox for examples
% /begin{example}[extra name]{NAME}
% Create a tcolorbox for examples
% Argument #1 is optional, given by [], that is the textbook's problem number
% Argument #2 is mandatory, given by {}, that is the title for the example
% Avoid putting special characters, (), [], {}, ",", etc. in the title.
\newtcolorbox[auto counter,
number within=section,
number format=\arabic,
crefname={example}{examples}, % Define reference format for cref (No Capitals)
Crefname={Example}{Examples}, % Reference format for cleveref (With Capitals)
]{example}[2][]{ % The [2][] Means the first argument is optional
  width=\textwidth,
  title={Example \thetcbcounter: #2. #1}, % Parentheses and commas are not well supported
  fonttitle=\bfseries,
  label={ex:#2},
  nameref=#2,
  colbacktitle=white!100!black,
  coltitle=black!100!white,
  colback=white!100!black,
  upperbox=visible,
  lowerbox=visible,
  sharp corners=all,
  breakable
}

% Create a tcolorbox for general use
\newtcolorbox[% auto counter,
% number within=section,
% number format=\arabic,
% crefname={example}{examples}, % Define reference format for cref (No Capitals)
% Crefname={Example}{Examples}, % Reference format for cleveref (With Capitals)
]{blackbox}{
  width=\textwidth,
  % title={},
  fonttitle=\bfseries,
  % label={},
  % nameref=,
  colbacktitle=white!100!black,
  coltitle=black!100!white,
  colback=white!100!black,
  upperbox=visible,
  lowerbox=visible,
  sharp corners=all
}

% Redefine the 'end of proof' symbol to be a black square, not blank
\renewcommand\qedsymbol{$\blacksquare$} % Change proofs to have black square at end

\renewcommand{\Re}{\operatorname{Re}} % Redefine to use the command, but not the fraktur version
\renewcommand{\Im}{\operatorname{Im}} % Redefine to use the command, but not the fraktur version
% Math Operators that are useful to abstract the written math away to one spot
\DeclareMathOperator{\RealNumbers}{\mathbb{R}}
\newcommand{\TextRealNumbers}{$\RealNumbers$}
\DeclareMathOperator{\AllIntegers}{\mathbb{Z}}
\newcommand{\TextAllIntegers}{$\AllIntegers$}
\DeclareMathOperator{\PositiveInts}{\mathbb{Z}^{+}}
\newcommand{\TextPositiveInts}{$\PositiveInts$}
\DeclareMathOperator{\NegativeInts}{\mathbb{Z}^{-}}
\newcommand{\TextNegativeInts}{$\NegativeInts$}
\DeclareMathOperator{\NaturalNumbers}{\mathbb{N}}
\newcommand{\TextNaturalNumbers}{$\NaturalNumbers$}
\DeclareMathOperator{\ComplexNumbers}{\mathbb{C}}
\newcommand{\TextComplexNumbers}{$\ComplexNumbers$}
\DeclareMathOperator{\RationalNumbers}{\mathbb{Q}}
\newcommand{\TextRationalNumbers}{$\RationalNumbers$}
\DeclareMathOperator*{\argmax}{argmax} % Thin Space and subscripts are UNDER in display
\DeclareMathOperator{\Lapl}{\mathcal{L}} % Declare a Laplace symbol to be used
\DeclareMathOperator{\UnitStep}{\mathcal{U}}
\DeclareMathOperator{\sinc}{sinc} % sinc(x) = (sin(pi x)/(pi x))
\DeclareMathOperator{\XOR}{\oplus}

\newcommand{\rbpRegister}{\texttt{\%rbp}}
\newcommand{\rspRegister}{\texttt{\%rsp}}
\newcommand{\ripRegister}{\texttt{\%rip}}
\newcommand{\raxRegister}{\texttt{\%rax}}
\newcommand{\rbxRegister}{\texttt{\%rbx}}

%%% Local Variables:
%%% mode: latex
%%% TeX-master: shared
%%% End:


% These packages are more specific to certain documents, but will be availabe in the template
% \usepackage{esint} % Provides us with more types of integral symbols to use
% \usepackage[outputdir=./TeX_Output]{minted} % Allow us to nicely typeset 300+ programming languages
% This document must be compiled with the -shell-escape flag if the packages above are uncommented

% \graphicspath{{./Drawings/EDIN01-Cryptography}} % Uncomment this to use pictures in this document
% \addbibresource{./Bibliographies/EDIN01-Cryptography.bib}

% Math Operators that are useful to abstract the written math away to one spot
% These are supposed to be document-specific mathematical operators that will make life easier
% Many fundamental operators are defined in Reference_Sheet_Preamble.tex

\DeclareMathOperator{\divides}{\vert}
\DeclareMathOperator{\DIV}{\text{div}}
\DeclareMathOperator{\lcm}{\text{lcm}}
\DeclareMathOperator{\ElementOrder}{\text{ord}}
\newcommand{\SetOrder}[1]{\lvert #1 \rvert}
\newcommand{\TextSetOrder}[1]{$\SetOrder{#1}$}

% REQUIRES THAN \AllIntegers BE DEFINED BEFORE MAKING THE NEW COMMAND
% \input-ing the Reference Preamble before here handles this for you.
% \IntsMod{#} takes something and subscripts it with the all integers symbol
% These commands MUST be placed inside of a math environment to work
\newcommand{\IntsMod}[1]{\AllIntegers_{#1}}
% Wrapper command. You don't need to specify anything because n is specified as the modulus for you
\newcommand{\IntsModN}{\IntsMod{n}}
\newcommand{\TextIntsMod}[1]{$\IntsMod{#1}$}
% Wrapper command. You don't need to specify anything because n is specified as the modulus for you
\newcommand{\TextIntsModN}{$\IntsModN{}$}
\newcommand{\MultiplicativeGroup}[1]{\IntsMod{#1}^{*}}
\newcommand{\MultiplicativeGroupN}{\IntsModN^{*}}
\newcommand{\TextMultiplicativeGroup}[1]{$\IntsMod{#1}^{*}$}
\newcommand{\TextMultiplicativeGroupN}{$\IntsModN^{*}$}

\begin{titlepage}
  \title{EDIN01: Cryptography - Reference Sheet}
  \author{Karl Hallsby}
  \date{Last Edited: \today} % We want to inform people when this document was last edited
\end{titlepage}

\begin{document}
\pagenumbering{gobble}
\maketitle
\pagenumbering{roman} % i, ii, iii on beginning pages, that don't have content
\tableofcontents
\clearpage
\pagenumbering{arabic} % 1,2,3 on content pages

\section{Cryptography Introduction}\label{sec:Intro_Cryptography}
\begin{definition}[Cryptographic Primitive]\label{def:Cryptographic_Primitive}
  A \emph{cryptographic primitive} is an algorithm with basic cryptographic properties.
  These are solutions to different problems where cryptography is required.
\end{definition}

\begin{definition}[Cryptographic Protocol]\label{def:Cryptographic_Protocol}
  A \emph{cryptographic protocol} involves the back-and-forth communication among two or more parties.

  \begin{remark}[Bob and Alice]\label{rmk:Bob_and_Alice}
    Typically, the parties are named Bob and Alice.
    These are arbitrary names, but these are the most commonly used ones.
  \end{remark}
\end{definition}

There are have been several \nameref{def:Cryptographic_Protocol}s.
\begin{enumerate}[noitemsep]
\item \textit{Symmetric-key cryptography} - Methods in which both the sender and receiver share the same key
  \begin{enumerate}[noitemsep]
  \item Block ciphers
  \item Stream ciphers
  \item MAC algorithms
  \end{enumerate}
\item \textit{Public-key cryptography}: 2 different, but mathematically related keys are used.
  A public key and a private key.
  \begin{enumerate}[noitemsep]
  \item The public key cannot decrypt something that was encrypted with the private key.
  \item The public key can be shared freely, because the private key cannot be generated from the public key.
  \end{enumerate}
\item \textit{Cryptographic hash functions} are a related and important class of cryptographic algorithms.
  \begin{enumerate}[noitemsep]
  \item This is a keyless \nameref{def:Cryptographic_Primitive}.
  \item Takes an arbitrary length input and produces a fixed-length output.
  \item The mapping between the input and output is such that the output cannot generate the input, therefore making it cryptographic.
  \end{enumerate}
\end{enumerate}

\subsection{Historical Cryptography}\label{subsec:Historical_Cryptography}
\begin{definition}[Cryptology]\label{def:Cryptology}
  \emph{Cryptology} was the science of secret writing.
\end{definition}

\begin{definition}[Cryptography]\label{def:Cryptography}
  \emph{Cryptography} dealt with the development of systems for secret writing.
\end{definition}

\begin{definition}[Cryptanalysis]\label{def:Cryptanalysis}
  \emph{Cryptanalysis} was the analysis of existing cryptographic systems to break them.
\end{definition}

Just to give a super quick background on how we've gotten to where we are today when it comes to cryptography.

\subsubsection{Monoalphabetic Ciphers}\label{subsubsec:Monoalphabetic_Ciphers}
\begin{definition}[Monoalphabetic Cipher]\label{def:Monoalphabetic_Cipher}
  In a \emph{monoalphabetic cipher} a single letter is replaced by the cipher's mapping.
  Since the cipher can do this to arbitrary letters, this could continue indefinitely for any single letter.

  These were some of the first ciphers developed by Man.
  These include simple substitute ciphers, and letter shifting ciphers.
  However, these can be broken with \emph{frequency analysis}.
\end{definition}

\subsubsection{Polyalphabetic Ciphers}\label{subsubsec:Polyalphabetic_Ciphers}
\begin{definition}[Polyalphabetic Cipher]\label{def:Polyalphabetic_Cipher}
  In a \emph{polyalphabetic cipher} multiple letters are replaced by the cipher's mapping.
  Additionally, since the cipher can output multiple letters, the ciphered letters could be run through the cipher again.

  These were developed in response to \nameref{def:Monoalphabetic_Cipher}s being broken.
  However, these can also be broken, with \emph{extended frequency analysis}.
\end{definition}

Eventually, it was realized that the secrecy of the cipher is not sensible/possible.
This leads us to the conclusion that \textbf{any cryptographic scheme should remain secure even if the adversary understands the cipher algorithm itself}.

\subsubsection{Cryptographic Keys}\label{subsubsec:Cryptographic_Keys}
The use of keys as ciphers is a slightly more modern occurrence.
\begin{definition}[Kerckhoff's Principle]\label{def:Kerckhoffs_Principle}
  \emph{Kerckhoff's Principle} states that the security of the key used should alone be sufficient for a good cipher to maintain confidentiality under an attack.
  Essentially, the security of the key used should be sufficient such that the cipher can be maintained confidently while under attack.
\end{definition}

However, only since the mid-1970s, has public key cryptography has been possible.

\begin{table}[h!]
  \centering
  \begin{tabular}{cc}
    \toprule
    Symmetric-Key Cryptography & Public-Key Cryptography \\
    \midrule
    Block ciphers & Public-Key encryption \\
    Stream ciphers & Digital Signature Schemes \\
    Cryptographic Hash Functions & Key exchange protocols \\
                               & Electronic Cash/Cryptocurrency \\
                               & Interactive Proof Systems \\
    \bottomrule
  \end{tabular}
  \caption{Uses of Key-Based Cryptography}
  \label{tab:Key_Cryptography_Uses}
\end{table}

Computers can efficiently encrypt, given the following constraints:
\begin{enumerate}[noitemsep]
\item Some modern techniques can only keep the keys secret if certain mathematical problems are \nameref{def:Intractable}.
  \begin{enumerate}[noitemsep]
  \item Integer factorization
  \item Discrete logarithm problems
  \end{enumerate}
\item However, there are no absolute proofs that a cryptographic technique is secure.
\end{enumerate}

\begin{definition}[Intractable]\label{def:Intractable}
  An \emph{intractable} problem is one in which there are no \textbf{efficient} algorithms to solve them.
\end{definition}

%%% Local Variables:
%%% mode: latex
%%% TeX-master: "../EDIN01-Cryptography-Reference_Sheet"
%%% End:


\section{Number Theory}\label{sec:Number_Theory}
Before we can start with any of the deeper cryptography stuff, we need to start with some basic number theory.
\begin{definition}[Number Theory]\label{def:Number_Theory}
  \emph{Number theory} is a branch of pure mathematics devoted primarily to the study of the integers and integer-valued functions.
  Number theorists study prime numbers as well as the properties of objects made out of integers (for example, rational numbers) or defined as generalizations of the integers (for example, algebraic integers).
\end{definition}

\begin{definition}[Divides]\label{def:Divides}
  For $a, b \in \AllIntegers$, we say that $a$ \emph{divides} $b$ (written $a \Divides b$) if there exists an integer $c$ such that $b = ac$.

  Properties:
  \begin{propertylist}
  \item $a \Divides a$
  \item If $a \Divides b$ and $b \Divides c$, then $a \Divides c$.
  \item If $a \Divides b$ and $a \Divides c$, then $a \Divides (bx + cy)$ for any $x, y \in \AllIntegers$.
  \item If $a \Divides b$ and $b \Divides a$, then $a = \pm b$.
  \end{propertylist}
\end{definition}

\subsection{Integer Long Division}\label{subsec:Integer_Long_Division}
For $a, b \in \AllIntegers$, with $b \geq 1$.
Then an ordinary long division of $a$ by $b$, i.e. $a \div b$ yields two integers $q$ and $r$ such that
\begin{equation}\label{eq:Integer_Long_Division}
  a = qb + r, \text{where } 0 \leq r < b
\end{equation}

$q$ and $r$ are called the \nameref{def:Integer_Quotient} and \nameref{def:Integer_Remainder}, respectively, and are \textbf{unique}.

\begin{definition}[Quotient]\label{def:Integer_Quotient}
  The \emph{quotient}, $q$, of $a$ divided by $b$ is denoted $a \DIV b$.
\end{definition}

\begin{definition}[Remainder]\label{def:Integer_Remainder}
  The \emph{remainder}, $r$, of $a$ divided by $b$ is denoted $a \bmod b$.
\end{definition}

\begin{example}[]{Integer Long Division}
  If $a=53$ and $b=9$, what is $a \bmod b$?

  \tcblower{}

  \begin{align*}
    53 &= q9 + r \\
    q &= 5 \\
    r &= 8
  \end{align*}

  Thus, $53 \bmod 9 = 8$.
\end{example}

\subsection{Greatest Common Divisor}\label{subsec:Greatest_Common_Divisor}
\begin{definition}[Common Divisor]\label{def:Common_Divisor}
  An integer $c$ is a \emph{common divisor} of $a$ and $b$ if $c \Divides a$ and $c \Divides b$.
\end{definition}

\begin{definition}[Greatest Common Divisor]\label{def:GCD}
  A non-negative integer $d$ is called the \emph{greatest common divisor} (\emph{GCD}) of integers $a$ and $b$ if:
  \begin{enumerate}[noitemsep]
  \item $d$ is a \nameref{def:Common_Divisor} of $a$ and $b$.
  \item For every other common divisor $c$ it holds that $c \Divides d$.
  \end{enumerate}

  The greatest common divisor is denoted
  \begin{equation}\label{eq:GCD}
    \gcd(a, b)
  \end{equation}

  $\gcd(a,b)$ is the \textbf{largest positive} integer dividing both $a$ and $b$ (except for $\gcd(0,0)=0$).
  
  \begin{remark}
    If $a, b \in \PositiveInts$, then $\lcm(a, b) \cdot \gcd(a, b) = a \cdot b$
  \end{remark}
\end{definition}

\begin{example}[]{Greatest Common Divisor}
  What is the $\gcd(18, 24)$?

  \tcblower{}

  Common Divisors = $\lbrace \pm 1, \pm 2, \pm 4, \pm 6 \rbrace$.

  Since we can only allow positive integers,
  \begin{equation*}
    \gcd(18, 24) = +6
  \end{equation*}
\end{example}

\subsection{Least Common Multiple}\label{subsec:Least_Common_Multiple}
\begin{definition}[Least Common Multiple]\label{def:LCM}
  A non-negative integer $d$ is called the \emph{least common multiple} (\emph{LCM}) of integers $a$ and $b$ if:
  \begin{enumerate}[noitemsep]
  \item $a \Divides d$ and $b \Divides d$
  \item For every integer $c$ such that $a \Divides c$ and $b \Divides c$, we have $d \Divides c$.
  \end{enumerate}

  The least common multiple is denoted
  \begin{equation}\label{eq:LCM}
    \lcm(a, b)
  \end{equation}

  $\lcm(a, b)$ is the \textbf{smallest positive} integer divisible by both $a$ and $b$.

  \begin{remark}
    If $a, b \in \PositiveInts$, then $\lcm(a, b) \cdot \gcd(a, b) = a \cdot b$
  \end{remark}
\end{definition}

\subsection{Primality}\label{subsec:Primality}
\begin{definition}[Relatively Prime]\label{def:Relatively_Prime}
  $a, b$ are called \emph{relatively prime} if $\gcd(a, b) = 1$.
\end{definition}

\begin{definition}[Prime]\label{def:Prime}
  An integer $p \geq 2$ is called \emph{prime} if its only positive divisors are $1$ and $p$.
  Otherwise, $p$ is called a \emph{\nameref{def:Composite}}.
\end{definition}

\begin{definition}[Composite]\label{def:Composite}
  An integer $p \geq 2$ is called \emph{composite} if it has more positive divisors than just $1$ and $p$.
  Otherwise, $p$ is called a \emph{\nameref{def:Prime}}.
\end{definition}

\subsubsection{Number of Primes}\label{subsubsec:Number_of_Primes}
The number of primes $\leq x$ is denoted
\begin{equation}\label{eq:Number_of_Primes}
  \pi(x)
\end{equation}
\begin{enumerate}[noitemsep]
\item There are infinitely many primes
\item $\lim\limits_{x \rightarrow \infty} \frac{\pi(x)}{\frac{x}{\ln(x)}} = 1$
\item For $x \geq 17$, $\frac{x}{\ln(x)} < \pi(x) < \frac{1.25506x}{\ln(x)}$
\end{enumerate}

\subsection{Unique Factorization}\label{subsec:Unique_Factorization}
\begin{theorem}[Unique Factorization Theorem]\label{thm:Unique_Factorization_Theorem}
  Every integer $n \geq 2$ can be written as a product of prime powers,
  \begin{equation*}
    n = p_{1}^{e_{1}} p_{2}^{e_{2}} \cdots p_{k}^{e_{k}}
  \end{equation*}
  where $p_{1}, p_{2}, \ldots p_{k}$ are distinct primes and $e_{1}, e_{2}, \ldots e_{k}$ are positive integers.
  Furthermore, the factorization is unique up to rearrangement of the factors.
\end{theorem}

\subsubsection{\nameref*{def:GCD} and \nameref*{def:LCM} with Unique Factors}\label{subsubsec:GCD_LCM_Unique_Factors}
If $a = p_{1}^{e_{1}} p_{2}^{e_{2}} \cdots p_{k}^{e_{k}}$ and $b = p_{1}^{f_{1}} p_{2}^{f_{2}} \cdots p_{k}^{e_{k}}$, where $e_{i}, f_{i}, i = 1, 2, \ldots k$ are non-negative integers, then
\begin{equation}\label{eq:GCD_Unique_Factors}
  \gcd(a, b) = p_{1}^{\min(e_{1}, f_{1})} p_{2}^{\min(e_{2}, f_{2})} \cdots p_{k}^{\min(e_{k}, f_{k})}
\end{equation}
and
\begin{equation}\label{eq:LCM_Unique_Factors}
  \lcm(a, b) = p_{1}^{\max(e_{1}, f_{1})} p_{2}^{\max(e_{2}, f_{2})} \cdots p_{k}^{\max(e_{k}, f_{k})}
\end{equation}

\subsection{Euler Phi Function}\label{subsec:Euler_Phi_Function}
\begin{definition}[Euler Phi Function]\label{def:Euler_Phi_Function}
  For $n \geq 1$, let $\phi(n)$ denote the number of integers in the interval $[1, n]$, which are \nameref{def:Relatively_Prime} to $n$.
  This function is called the \emph{Euler Phi Function}.
  \begin{equation}\label{eq:Euler_Phi_Function}
    \phi(n) = \left( p_{1}^{e_{1}} - p_{1}^{e_{1}-1} \right) \left( p_{2}^{e_{2}} - p_{2}^{e_{2}-1} \right) \cdots \left( p_{k}^{e_{k}} - p_{k}^{e_{k}-1} \right)
  \end{equation}
  \begin{remark}
    The \nameref{def:Euler_Phi_Function} is closely related to \nameref{def:Set_Order}.
  \end{remark}
\end{definition}

\begin{theorem}[Euler Phi Function]\label{thm:Euler_Phi_Function}
  There are a few properties of the \nameref{def:Euler_Phi_Function} that we will treat as true because of this theorem.
  \begin{propertylist}
  \item If $p$ is a \nameref{def:Prime}, then $\phi(p) = p - 1$.\label{prop:Euler_Phi_Function_Properties-Prime_Number}
  \item If $\gcd(a, b) = 1$, then $\phi(ab) = \phi(a) \phi(b)$.\label{prop:Euler_Phi_Function_Properties-Split_Multiplication}
  \item If $n = p_{1}^{e_{1}} p_{2}^{e_{2}} \cdots p_{k}^{e_{k}}$, then
    \begin{equation*}
      \phi(n) = \left( p_{1}^{e_{1}} - p_{1}^{e_{1}-1} \right) \left( p_{2}^{e_{2}} - p_{2}^{e_{2}-1} \right) \cdots \left( p_{k}^{e_{k}} - p_{k}^{e_{k}-1} \right)
    \end{equation*}\label{prop:Euler_Phi_Function_Properties-Product_of_Primes}
  \end{propertylist}
\end{theorem}

\begin{example}[Exercise 1, Question 1.2a]{Euler Phi Function}
  Find the value of $\phi(36)$?
  \tcblower{}
  First, we note that 36 is not a \nameref{def:Prime} number, thus we need to find a set of primes that are equal to 36.
  The divisors of 36 are
  \begin{equation*}
    \lbrace \pm 1, \pm 2, \pm 3, \pm 4, \pm 9, \pm 12, \pm 18, \pm 36 \rbrace
  \end{equation*}
  And all possible prime numbers present as divisors of 36 are
  \begin{equation*}
    \lbrace 2, 3 \rbrace
  \end{equation*}

  So, there must be some combination of $2^{x}\cdot 3^{y}$ that yields 36.
  In fact, $2^{2} \cdot 3^{2} = 4 \cdot 9 = 36$.
  Since 36 can be broken up as a product of 2 \nameref{def:Prime} numbers raised to some power, we can use \Cref{prop:Euler_Phi_Function_Properties-Product_of_Primes} of the \nameref{def:Euler_Phi_Function} to simplify this.
  But first, we need to separate the two values from each other using \Cref{prop:Euler_Phi_Function_Properties-Split_Multiplication} of the \nameref{def:Euler_Phi_Function} to apply \Cref{prop:Euler_Phi_Function_Properties-Product_of_Primes}.

  We need to check $\gcd \left( 2^{2}, 3 ^{2} \right) = 1$.
  \begin{align*}
    \gcd \left( 2^{2}, 3^{2} \right) &= \gcd(4, 9) \\
    9 &= a4 + b \\
                                     &= 2 \cdot 4 + 1 \\
    4 &= a \cdot 1 + b \\
                                     &= 4 \cdot 1 + 0
  \end{align*}
  So, $\gcd \left( 2^{2}, 3^{2} \right) = 1$, so we can use \Cref{prop:Euler_Phi_Function_Properties-Split_Multiplication}.
  \begin{equation*}
    \phi \left( 2^{2} \cdot 3^{2} \right) = \phi \left( 2^{2} \right) \phi \left( 3^{2} \right)
  \end{equation*}
  And now we can apply \Cref{prop:Euler_Phi_Function_Properties-Product_of_Primes} to find our answer.

  \begin{align*}
    \phi \left( 2^{2} \right) \phi \left( 3^{2} \right) &= \left( 2^{2} - 2^{2-1} \right) \left( 3^{2} - 3^{2-1} \right) \\
                                                        &= (4-2)(9-3) \\
                                                        &= (2)(6) \\
                                                        &= 12
  \end{align*}
  Thus, $\phi(36) = 12$.
\end{example}

\begin{lemma}[Computing the \nameref{def:GCD}]\label{lemma:Compute_GCD}
  If $a$ and $b$ are positive integers where $a > b$, then
  \begin{equation}\label{eq:Compute_GCD}
    \gcd(a, b) = \gcd(b, a \bmod b)
  \end{equation}
  \begin{remark*}
    This can be repeated to efficiently calculate the $\gcd(a, b)$.
    This is called the \nameref{def:Euclidean_Algorithm}.
  \end{remark*}
\end{lemma}

\begin{definition}[Euclidean Algorithm]\label{def:Euclidean_Algorithm}
  The \emph{euclidean algorithm} is a way to efficiently calculate the $\gcd(a, b)$.
  
  \begin{algorithm}[H]
    \DontPrintSemicolon{}
    \SetKwInOut{Input}{Input}\SetKwInOut{Output}{Output}

    \Input{Two non-negative integers $a, b$ where $a \geq b$.}
    \Output{$\gcd(a, b)$.}
    \BlankLine{}
    Set $r_{0} \leftarrow a$, $r_{1} \leftarrow b$, $i \leftarrow 1$ \;
    \While{$r_{i} \neq 0$}{
      Set $r_{i+1} \leftarrow r_{i-1} \bmod r_{i}$ \;
      $i \leftarrow i+1$ \;
    }
    \Return{$r_{i}$}
    \caption{Euclidean Algorithm}
    \label{algo:Euclidean_Algorithm}
  \end{algorithm}
\end{definition}

\begin{example}[Exercise 1, Question 1.1a]{Euclidean Algorithm}
  Find the \nameref{def:GCD} of 222 and 1870?

  \tcblower{}

  \begin{align*}
    1870 &= a 222 + b \\
         &= 8 \cdot 222 + 94 \\
    222 &= a 94 + b \\
         &= 2 \cdot 94 + 34 \\
    94 &= a 34 + b \\
         &= 2 \cdot 34 + 26 \\
    34 &= a 26 + b \\
         &= 1 \cdot 26 + 8 \\
    26 &= a 8 + b \\
         &= 3 \cdot 8 + 2 \\
    8 &= a 2 + b \\
         &= 4 \cdot 2 + 0
  \end{align*}

  Thus, since $4 \cdot 2 = 8$, 2 is the \nameref{def:GCD} of 222 and 1870.
\end{example}

\begin{theorem}
  There exist integers $x, y$ such that $\gcd(a, b)$ can be written as
  \begin{equation}\label{eq:Extended_Euclidean_Algorithm_Basis}
    \gcd(a, b) = ax + by
  \end{equation}
\end{theorem}
\begin{proof}
  \begin{align*}
    \gcd(a, b) &= r_{i} \\
               &= r_{i-2} - q_{i-1}r_{i-1} \\
               &= r_{i-2} - q_{i-1}(r_{i-3} - q_{i-2}r_{i-2}) \\
               &\vdots \\
               &= r_{0}x + r_{1}y \\
               &= ax + by
  \end{align*}
  for some integers $x, y \in \AllIntegers$.
\end{proof}

This means that the \nameref{def:Euclidean_Algorithm} can be extended to return the values of $x$ and $y$ from \Cref{eq:Extended_Euclidean_Algorithm_Basis}.

\begin{definition}[Extended Euclidean Algorithm]\label{def:Extended_Euclidean_Algorithm}
  The \emph{extended euclidean algorithm} is a way to efficiently calculate the linear pair of integers ($x, y \in \AllIntegers$) that satisfy \Cref{eq:Extended_Euclidean_Algorithm_Basis}.
  \begin{equation*}
    \gcd(a, b) = ax + by
  \end{equation*}

  \begin{algorithm}[H]
    \DontPrintSemicolon{}
    \SetKwInOut{Input}{Input}\SetKwInOut{Output}{Output}

    \Input{Two non-negative integers $a, b$ where $a \geq b$.}
    \Output{$d = \gcd(a, b)$ and two integers $x, y$ such that $d = ax + by$.}
    \BlankLine{}

    \If{$b=0$}{
      \Return{$a$, $x \leftarrow 1$, $y \leftarrow 0$}
    }
    Set $x_{2} \leftarrow 1$, $x_{1} \leftarrow 0$, $y_{2} \leftarrow 0$, and $y_{1} \leftarrow 1$. \;
    \While{$b > 0$}{
      $q \leftarrow a \div b$. \;
      $r \leftarrow a-qb$, $x \leftarrow x_{2} - qx_{1}$, $y \leftarrow y_{2} - qy_{1}$. \;
      $a \leftarrow b$
      $b \leftarrow r$
      $x_{2} \leftarrow x_{1}$, $x_{1} \leftarrow x$, $y_{2} \leftarrow y_{1}$, and $y_{1} \leftarrow y$. \;
    }
    Set $d \leftarrow a$, $x \leftarrow x_{2}$, $y \leftarrow y_{2}$. \;
    \Return{$d, x, y$}
    \caption{Extended Euclidean Algorithm (Bezout's Theorem)}
    \label{algo:Extended_Euclidean_Algorithm}
  \end{algorithm}
\end{definition}

\begin{example}[Exercise 1, Question 1.1b]{Extended Euclidean Algorithm}
  Find the integers $x$ and $y$ such that $\gcd(222, 1870) = 222x + 1870y$?
  \tcblower{}
  From \Cref{ex:Euclidean Algorithm}, we know $\gcd(222, 1870) = 2$, so we can plug that in.
  We now know that
  \begin{equation*}
    2 = 222x + 1870y
  \end{equation*}

  Now, we essentially run the \nameref{def:Euclidean_Algorithm} backwards.
  \begin{align*}
    2 &= 26 - 3 \cdot 8 \\
      &= 26 - 3(34 - 1 \cdot 26) = 26 - 3 \cdot 34 + 3 \cdot 26 \\
      &= 4 \cdot 26 - 3 \cdot 34 \\
      &= 4(94 - 2 \cdot 34) - 3 \cdot 34 = 4 \cdot 94 - 8 \cdot 34 - 3 \cdot 34\\
      &= 4 \cdot 94 - 11 \cdot 34 \\
      &= 4 \cdot 94 - 11(222 - 2 \cdot 94) = 4 \cdot 94 - 11 \cdot 222 + 22 \cdot 94 \\
      &= 26 \cdot 94 - 22 \cdot 222 \\
      &= 26(1870 - 8 \cdot 222) - 22 \cdot 222 = 26 \cdot 1870 - 208 \cdot 222 - 11 \cdot 222 \\
      &= -219 \cdot 222 + 26 \cdot 1870
  \end{align*}
  Now we need to check the solution we might have found
  \begin{align*}
    -219 \cdot 222 + 26 \cdot 1870 &= 2 \\
    -48618 + 48620 &= 2 \\
    2 &= 2
  \end{align*}

  Thus,
  \begin{align*}
    x &= -219 \\
    y &= 26
  \end{align*}
\end{example}

\subsection{\texorpdfstring{The Integers modulo $n$}{The Integers modulo n}}\label{subsec:Integer_Modulo_n}
Let $n$ be a positive integer.
\begin{definition}[Congruence]\label{def:Congruence}
  If $a$ and $b$ are integers, then \emph{$a$ is said to be congruent to $b$ modulo $n$}, which is written as
  \begin{equation}\label{eq:A_Congruent_B}
    a \equiv b \pmod{n}
  \end{equation}

  If $n$ divides $(a-b)$, i.e. $n \Divides (a-b)$, then we call $n$ the \emph{modulus} of the congruence.

\end{definition}

\begin{theorem}
  For $a, a_{1}, b, b_{1}, c\in \AllIntegers$, we have
  \begin{propertylist}
  \item $a \equiv b \pmod{n}$ \emph{if and only if} $a$ and $b$ leave the same \nameref{def:Integer_Remainder} when divided by $n$.
  \item $a \equiv a \pmod{n}$ \label{prop:A_Congruent_B_Reflexivity}
  \item If $a \equiv b \pmod{n}$, then $b \equiv a \pmod{n}$ \label{prop:A_Congruent_B_Symmetry}
  \item If $a \equiv b \pmod{n}$ adn $b \equiv c \pmod{n}$, then $a \equiv c \pmod{n}$ \label{prop:A_Congruent_B_Transitivity}
  \item If $a \equiv a_{1} \pmod{n}$ and $b \equiv b_{1} \pmod{n}$, then $a+b = a_{1} + b_{1} \pmod{n}$ and $ab = a_{1}b_{1} \pmod{n}$.
  \end{propertylist}

  \Crefrange{prop:A_Congruent_B_Reflexivity}{prop:A_Congruent_B_Transitivity} are called \emph{reflexivity}, \emph{symmetry}, and \emph{transitivity}, respectively.
\end{theorem}

\begin{example}[Exercise 1, Question 1.2b]{Integers modulo n}
  Write all the units (\nameref{def:Invertible_Element}) in \TextIntsMod{36}?
  \tcblower{}
  First, we start by constructing our set of integers modulo $n$.
  \begin{equation*}
    \IntsMod{36} = \bigl\lbrace [0], [1], [2], [3], [4], \ldots, [35] \bigr\rbrace
  \end{equation*}

  Since we are only worried about the units of \TextIntsMod{36}, we need to find the integers that satisfy \Cref{eq:Invertible_Element}.
  This is done by finding an $a$ value that has a \nameref{def:Multiplicative_Inverse}, which requires that \Cref{eq:Invertible} be true, namely
  \begin{equation*}
    \gcd(a, n) = 1
  \end{equation*}
  This leaves us with
  \begin{equation*}
    \IntsMod{36} = \bigl\lbrace [1], [5], [7], [11], [13], [17], [19], [23], [25], [29], [31], [35] \bigr\rbrace
  \end{equation*}
  which is the solution.
\end{example}

\subsection{Equivalence Classes}\label{subsec:Equivalence_Classes}
\begin{definition}[Equivalence Class]\label{def:Equivalence_Class}
  \nameref{def:Congruence} modulo $n$ partitions $\AllIntegers$ into $n$ sets, called \emph{equivalence class}es, where each integer belongs to exactly one equivalence class.

  For example, these are all congruent to each other modulo $n$:
  \begin{subequations}\label{eq:Equivalence_Class}
    \begin{equation}\label{subeq:Equivalence_Class_Remainder_0}
      [0] = \lbrace \ldots, -2n, -n,\, 0, n, 2n, \ldots \rbrace
    \end{equation}
    \begin{equation}\label{subeq:Equivalence_Class_Remainder_1}
      [1] = \lbrace \ldots -2n + 1, -n+1,\, 1, n+1, 2n+1 \ldots \rbrace
    \end{equation}
    \begin{equation}\label{eq:General_Equivalent_Class_Remainder}
      [r] = \IntsMod{r} = (x \bmod r) + n\AllIntegers
    \end{equation}
  \end{subequations}

  Since all elements in an equivalent class have the same \nameref{def:Integer_Remainder}, $r$, we use $r$ as a \emph{represenatative} for the equivalence class.
  \begin{remark}
    In this case, the representatives of the equivalence classes shown in \Crefrange{subeq:Equivalence_Class_Remainder_0}{subeq:Equivalence_Class_Remainder_1} are 0 and 1, respectively, and consist of all integers that are mod 0 or mod 1, respectively.
  \end{remark}
\end{definition}

%%% Local Variables:
%%% mode: latex
%%% TeX-master: "../EDIN01-Cryptography-Reference_Sheet"
%%% End:


\section{\nameref*{sec:Number_Theory} on Sets}\label{sec:Number_Theory_on_Sets}
While this section is not technically different than \Cref{sec:Number_Theory}, it is worth it to split these up, since \Cref{sec:Number_Theory} does not deal with sets.
However, using what we learned in \Cref{sec:Number_Theory}, \nameref{sec:Number_Theory}, it is natural to extend these to sets of numbers.

\subsection{\texorpdfstring{\TextIntsModN{}}{Sets of Integers Modulo n}}\label{subsec:Z_mod_n}
\begin{definition}[\TextIntsModN{}]\label{def:Z_mod_n}
  \nameref{subsec:Integer_Modulo_n}, denoted \TextIntsModN{}, is the set of (\nameref{def:Equivalence_Class}es of) integers $\lbrace [0], [1], \ldots , [n-1] \rbrace$.
  Addition, subtraction, and multiplication are all performed with modulo $n$.
  \Crefrange{ex:Addition on Integers mod n}{ex:Multiplication on Integers mod n} demonstrate this.
\end{definition}

\begin{example}[]{Addition on Integers mod n}
  When dealing with the set of integers \TextIntsMod{15}, what is the sum of 5 and 9?
  \tcblower{}
  \begin{equation*}
    5 \bmod 15 + 9 \bmod 15 = 11 \bmod 15
  \end{equation*}

  Thus, the answer is 11.
\end{example}

\begin{example}[]{Subtraction on Integers mod n}
  When dealing with the set of integers \TextIntsMod{15}, what is 5 minus 9?
  \tcblower{}
  \begin{align*}
    5 \bmod 15 - 9 \bmod 15 &= 5 \bmod 15 + (-9 \bmod 15) \\
                            &= 5 \bmod 15 + \underbrace{-9 \bmod 15}_{-9 + 15 = 6} \\
                            &= 5 \bmod 15 + 6 \bmod 15 \\
                            &= 11 \bmod 15
  \end{align*}

  Thus, the answer is, again, 11.
\end{example}

\begin{example}[]{Multiplication on Integers mod n}
  When dealing with the set of integers \TextIntsMod{15}, what is the product of 5 and 9?
  \tcblower{}
  \begin{align*}
    5 \bmod 15 \cdot 9 \bmod 15 &= 45 \bmod 15 \\
                                &= 0
  \end{align*}

  Thus, the answer is 0, because $45 = 3 \cdot 15$.
\end{example}

\subsection{\texorpdfstring{Inverse in \TextIntsModN{}}{Inverse in Integers Modulo n}}\label{subsec:Inverse_Z_mod_n}
Addition, subtraction, and multiplication can be performed trivially in \TextIntsModN{}, as shown in \Crefrange{ex:Addition on Integers mod n}{ex:Multiplication on Integers mod n}.
However, the concept of division is a little bit more difficult.
\begin{definition}[Multiplicative Inverse]\label{def:Multiplicative_Inverse}
  Let $a \in \IntsModN{}$.
  The \emph{multiplicative inverse} of $a$ is an integer $x \in \IntsModN{}$, such that $ax = 1$.
  If such an integer, $x$, exists, then $a$ is said to be \emph{invertible} and $x$ is called the inverse of $a$, denoted as $a^{-1}$.
\end{definition}

\begin{definition}[Division in \TextIntsModN{}]\label{def:Division_Z_mod_n}
  \emph{Division} of $a$ by an element $b \in \IntsModN{}$ (written $a/b$) is defined as $ab^{-1}$, and is only defined if $b$ has a \nameref{def:Multiplicative_Inverse}.
\end{definition}

\begin{definition}[Invertible]\label{def:Invertible}
  Let $a \in \IntsModN{}$.
  Then $a$ is \emph{invertible} if and only iff
  \begin{equation}\label{eq:Invertible}
    \gcd(a, n) = 1
  \end{equation}
\end{definition}

\begin{proof}
  Assume that $\gcd(a, n) = 1$.
  We know that $1 = \gcd(a, n) = xa + yn$ for some $x, y \in \AllIntegers$.
  Since $yn$ is a multiple of $n$, namely $yn \bmod n = 0$, it is removed from the equation.
  Then $x \bmod n$ is an inverse to $a$.

  Now assume $\gcd(a, n) > 1$.
  If $a$ has an inverse $x$, then $ax = 1 \bmod n$, which means $1 = ax + ny$ for some $x, y \in \AllIntegers$, directly contradicting the assumption that $\gcd(a, n) = 1$.
\end{proof}

The two possible cases of division, i.e.\ possible and impossible, are shown in \Crefrange{ex:Possible Division on Integers mod n}{ex:Impossible Division on Integers mod n}.

\begin{example}[]{Possible Division on Integers mod n}
  When dealing with the set of integers \TextIntsMod{15}, what is the result from the division of 5 by 11?
  \tcblower{}
  \begin{align*}
    5 \bmod 15 \div 11 \bmod 15 &= 5 \cdot 11^{-1}
  \end{align*}
  Now we need to find the \nameref{def:Multiplicative_Inverse} of $1$.
  \begin{equation*}
    11^{-1} = \gcd(11, 15)
  \end{equation*}
  We can compute the \nameref{def:GCD} efficiently with the \nameref{def:Euclidean_Algorithm}.
  \begin{align*}
    15 &= 1 \cdot 11 + 4 \\
    11 &= 2 \cdot 4 + 3 \\
    4 &= 1 \cdot 3 + 1 \\
    3 &= 3 \cdot 1 + 0 \\
  \end{align*}
  Thus, the \nameref{def:Euclidean_Algorithm} gives us $\gcd(11, 15) = 1$.
  Since $\gcd(11, 15) = 1 = 1$, 11 \textbf{does} have a \nameref{def:Multiplicative_Inverse}, making the division possible.
  Now we need to run through the \nameref{def:Extended_Euclidean_Algorithm}, to find the values $x, y \in \AllIntegers$.
  \begin{align*}
    1 &= 4 - 1 \cdot 3 \\
      &= 4 - 1 \cdot (11 - 2 \cdot 4) = 3 \cdot 4 - 1 \cdot 11 \\
      &= 3 (15 - 1 \cdot 11) - 1 \cdot 11 \\
      &= 3 \cdot 15 - 3 \cdot 11 - 1 \cdot 11 \\
      &= 3 \cdot 15 - 4 \cdot 11
  \end{align*}
  Thus,
  \begin{align*}
    x &= -4 \\
    y &= 3
  \end{align*}
  Now we know
  \begin{equation*}
    {(11 \bmod 15)}^{-1} = -4 \bmod 15
  \end{equation*}
  Since $-4$ is not part of \TextIntsMod{15}, we need to find the additive inverse.
  $-4 + 15 = 11$.
  Thus,
  \begin{equation*}
    {(11 \bmod 15)}^{-1} = 11 \bmod 15
  \end{equation*}
  Now, we perform a simple substitution.
  \begin{align*}
    5 \bmod 15 / 11 \bmod 15 &= 5 \bmod 15 \cdot {(11 \bmod 15)}^{-1} \\
                             &= 5 \bmod 15 \cdot 11 \bmod 15 \\
                             &= 55 \bmod 15 \\
                             &= 10
  \end{align*}

  So, the result of the division of $5$ by $11$ is $10$.
\end{example}

\begin{example}[]{Impossible Division on Integers mod n}
  When dealing with the set of integers \TextIntsMod{15}, what is the result from the division of 5 by 9?
  \tcblower{}
  \begin{equation*}
    5 \bmod 15 \div 9 \bmod 15 = 5 \cdot 9^{-1}
  \end{equation*}
  Now we need to find the \nameref{def:Multiplicative_Inverse} of $9$.
  \begin{equation*}
    9^{-1} = \gcd(9, 15)
  \end{equation*}
  We can compute the \nameref{def:GCD} efficiently with the \nameref{def:Euclidean_Algorithm}.
  \begin{align*}
    15 &= 1 \cdot 9 + 6 \\
    9 &= 1 \cdot 6 + 3 \\
    3 &= 1 \cdot 3 + 0 \\
  \end{align*}
  Thus, the \nameref{def:Euclidean_Algorithm} gives us $\gcd(9, 15) = 3$.
  Since $\gcd(9, 15) = 3 \neq 1$, 9 does \textbf{not} have a \nameref{def:Multiplicative_Inverse}, making the division impossible.
\end{example}

\subsection{Chinese Remainder Theorem}\label{subsec:Chinese_Remainder_Theorem}
\begin{theorem}[Chinese Remainder Theorem]\label{thm:Chinese_Remainder_Theorem}
  Let the integers $n_{1}, n_{2}, \ldots, n_{k}$ be pairwise \nameref{def:Relatively_Prime}.
  Then the system of \nameref{def:Congruence}s
  \begin{align*}
    x &\equiv a_{1} \pmod{n_{1}} \\
    x &\equiv a_{2} \pmod{n_{2}} \\
      &\vdots \\
    x &\equiv a_{k} \pmod{n_{k}}
  \end{align*}
  has a unique solution modulo $n = n_{1}n_{2} \cdots n_{k}$.
\end{theorem}

\begin{definition}[Gauss's Algorithm]\label{def:Gauss_Algorithm}
  The solution $x$ to the system of \nameref{def:Congruence}s promised by the \nameref{thm:Chinese_Remainder_Theorem} can be calculated as
  \begin{equation}\label{eq:Gauss_Algorithm}
    x = \Biggl( \sum\limits_{i=1}^{k}a_{i} N_{i} M_{i} \Biggr) \bmod n
  \end{equation}
  where $N_{i} = \frac{n}{n_{i}}$ and $M_{i} = N_{i}^{-1} = {\left( \frac{n}{n_{i}} \right)}^{-1} \bmod n_{i}$ ($M_{i}$ is the \nameref{def:Multiplicative_Inverse} of $N_{i} \bmod n_{i}$).
  
  This simplifies to
  \begin{equation}\label{eq:Gauss_Algorithm_Simplified}
    x = \sum\limits_{i=1}^{k}a_{i} \frac{n}{n_{i}} \Biggl( \frac{n_{i}}{n} \bmod n \Biggr)
  \end{equation}
\end{definition}

\begin{definition}[Chinese Remainder Theorem]\label{def:Chinese_Remainder_Theorem}
  The \emph{\nameref{thm:Chinese_Remainder_Theorem}} allows us to change the way we represent elements of our set, \TextIntsModN{}.
  
  The integers modulo $n$, \TextIntsModN{}, where $n = n_{1}n_{2}$ and $\gcd(n_{1}, n_{2}) = 1$.
  An element $a \in$ \TextIntsModN{} has a unique representation: $(a \bmod n_{1}, a \bmod n_{2})$.
  We can denote this mapping by $\gamma : \IntsModN{} \rightarrow \IntsMod{n_{1}} \times \IntsMod{n_{2}}$.
  \begin{propertylist}
  \item $\gamma(a) = \gamma(b)$ if and only if $a = b$. \label{prop:Chinese_Remainder_Theorem_Property-Equivalence}
  \item For all $(a_{1}, a_{2}) \in \IntsMod{n_{1}} \times \IntsMod{n_{2}}$, there exists an $a$ such that $\gamma(a) = (a_{1}, a_{2})$.
  \item $\gamma(a+b) = \gamma(a) + \gamma(b)$
  \item $\gamma(ab) = \gamma(a) \gamma(b)$ \label{prop:Chinese_Remainder_Theorem_Property-Multiplication}
  \end{propertylist}
  These properties (\Crefrange{prop:Chinese_Remainder_Theorem_Property-Equivalence}{prop:Chinese_Remainder_Theorem_Property-Multiplication}) make $\gamma$ an \emph{\nameref{def:Isomorphism}}.

  \begin{remark}
    In the case of large integers for cryptography, knowing just one part of the number can ehlp get the other part.
    However, if the number is very large, 2048 bits for instance, these calculations start becoming \nameref{def:Intractable}.
  \end{remark}
\end{definition}

\begin{example}[]{Chinese Remainder Theorem Mapping}
  Find the mapping of $7$ when in \TextIntsMod{15}?
  \tcblower{}
  Since 7 is an element in \TextIntsMod{15},
  \begin{equation*}
    7 \Leftrightarrow (7 \bmod 3, 7 \bmod 5) = (1, 2)
  \end{equation*}
\end{example}

\subsection{\texorpdfstring{Multiplicative Groups, \TextMultiplicativeGroupN{}}{Multiplicative Groups}}\label{Multiplicative_Groups}
\begin{definition}[Multiplicative Group, \TextMultiplicativeGroupN{}]\label{def:Multiplicative_Group}
  Define the \emph{multiplicative group} of \TextIntsModN{}, denoted \TextMultiplicativeGroupN{} as the set of all elements in \TextIntsModN{} with \nameref{def:Multiplicative_Inverse}s.
  \begin{equation}\label{eq:Multiplicative_Group}
    \MultiplicativeGroupN{} = \lbrace a \in \IntsModN{} \vert \gcd(a, b) = 1 \rbrace
  \end{equation}
\end{definition}

\begin{definition}[Set Order]\label{def:Set_Order}
  The \emph{order a set}, for example, \TextMultiplicativeGroupN{}, is the number of elements in \TextMultiplicativeGroupN{} (\TextSetOrder{\MultiplicativeGroupN{}}).
  From the definition of the \nameref{def:Euler_Phi_Function}
  \begin{equation}\label{eq:Set_Order_Euler_Phi_Function}
    \SetOrder{\MultiplicativeGroupN{}} = \phi(n)
  \end{equation}

  \begin{remark}[Closed Under Multiplication]\label{rmk:Set_Order_Closed_Multiplication}
    Since the produce of two elements with \nameref{def:Multiplicative_Inverse}s is another element with a \nameref{def:Multiplicative_Inverse}, we say that \TextSetOrder{\MultiplicativeGroupN{}} is \emph{closed under multiplication}.
  \end{remark}
\end{definition}

\begin{definition}[Element Order]\label{def:Element_Order}
  The \emph{order of an element} $a \in \MultiplicativeGroupN{}$, denoted $\ElementOrder(a)$ is defined as the least positive integer $t$ ($t \in \AllIntegers$) such that
  \begin{equation}\label{eq:Element_Order}
    a^{t} \bmod n = 1
  \end{equation}
\end{definition}

\begin{lemma}[Element Order]\label{lemma:Element_Order}
  Let $a \in \MultiplicativeGroupN{}$.
  If $a^{s}$ for some $s$, then $\ElementOrder(a) \Divides s$.
  In particular, $\ElementOrder(a) \Divides \phi(n)$ must be true.
\end{lemma}

\begin{example}[Exercise 1, Problem 1.6b]{Element Order}
  Find the $\ElementOrder(5)$ in \TextMultiplicativeGroup{8}?
  \tcblower{}
  \textbf{TODO}
  \textit{Remember to check with $\Divides \phi(n)$, since $5 \in \MultiplicativeGroup{8}$.}
\end{example}

\begin{proof}[Element Order]\label{proof:Element_Order}
  Let $t = \ElementOrder(a)$.
  By long division, $s = qt + r$, where $r < t$.
  Then $a^{s} = a^{qt + r} = a^{qt}a^{r}$ and since $a^{t} = 1$, from \Cref{eq:Element_Order}, we have $a^{s} = a^{r}$ and $a^{r} = 1$.
  This reduction is shown below:
  \begin{align*}
    a^{s} &= a^{qt + r} \\
          &= a^{qt}a^{r} \\
          &= {\left( a^{t} \right)}^{q} a^{r} \\
          &= {\left( 1 \right)}^{q} a^{r} \\
          &= 1^{q} a^{r} \\
          &= 1 a^{r} \\
          &= a^{r}
  \end{align*}

  But, $r<t$, so we must have $r=0$, and so $\ElementOrder(a) \Divides s$.
\end{proof}

\subsection{Euler's Theorem}\label{subsec:Eulers_Theorem}
\begin{theorem}[Euler's Theorem]\label{thm:Eulers_Theorem}
  If $a \in \MultiplicativeGroupN{}$, then
  \begin{equation}\label{eq:Eulers_Theorem}
    a^{\phi(n)} \equiv 1 \pmod{n}
  \end{equation}
\end{theorem}

\begin{proof}[Euler's Theorem]\label{proof:Eulers_Theorem}
  Let $\MultiplicativeGroupN{} = \lbrace a_{1}, a_{2}, \ldots, a_{\phi(n)} \rbrace$.
  Looking at the set of elements $A = \lbrace aa_{1}, aa_{2}, \ldots, aa_{\phi(n)} \rbrace$, it is easy to see that $A = a \MultiplicativeGroupN{}$.
  So we have 2 ways of writing the product of all of the elements, i.e.
  \begin{equation*}
    \prod\limits_{i=1}^{\phi(n)} a a_{i} = \prod\limits_{i=1}^{\phi(n)} a_{i}
  \end{equation*}
  
  This leads to
  \begin{equation*}
    \prod\limits_{i=1}^{\phi(n)} a = a^{\phi(n)} = 1
  \end{equation*}
  which is the same as what we said in \Cref{eq:Eulers_Theorem}.
\end{proof}

\subsection{Fermat's Little Theorem}\label{subsec:Fermats_Little_Theorem}
\begin{theorem}[Fermat's Little Theorem]\label{thm:Fermats_Little_Theorem}
  Let $p$ be a \nameref{def:Prime} number.
  If $\gcd(a, p) = 1$, then
  \begin{equation}\label{eq:Fermats_Little_Theorem}
    a^{p-1} \equiv 1 \pmod{p}
  \end{equation}
\end{theorem}
\begin{remark*}
  This ties in with \nameref{thm:Eulers_Theorem}, because working in \TextIntsModN{}, all exponents can be reduced by modulo $\phi(n)$.
\end{remark*}

\subsection{Generators}\label{subsec:Generators}
\begin{definition}[Generator]\label{def:Generator}
  Let $a \in \MultiplicativeGroupN{}$.
  If $\ElementOrder(a) = \phi(n)$, then $a$ is said to be a \emph{generator} (or a \emph{primitive element}) of \TextMultiplicativeGroupN{}.
  Furthermore, if \TextMultiplicativeGroupN{} has a generator, then \TextMultiplicativeGroupN{} is said to be\emph{\nameref{def:Cyclic}}.
  \begin{remark}
    It is clear that if $a \in \MultiplicativeGroupN{}$ is a \nameref{def:Generator}, then every element in \TextMultiplicativeGroupN{} can be expressed as $a^{i}$ for some integer $i$ ($i \in \AllIntegers$).
    So, we can write
    \begin{equation}\label{eq:Generator_in_Multiplicative_Group}
      \MultiplicativeGroupN{} = \lbrace a^{i} \vert 0 \leq i \leq \phi(n) - 1 \rbrace
    \end{equation}
  \end{remark}
\end{definition}

\begin{example}[Exercise 1, Question 1.6c]{Cyclic Group}
  Is \TextMultiplicativeGroup{8} a \nameref{def:Cyclic} \nameref{def:Group}?
  \tcblower{}
  \textbf{TODO}
\end{example}

\subsection{Quadratic Residues}\label{subsec:Quadratic_Residues}
\begin{definition}[Quadratic Residue]\label{def:Quadratic_Residue}
  An element $a \in \MultiplicativeGroupN{}$ is said to be a \emph{quadratic residue} modulo $n$ (or a \emph{square}) if there exists an $x \in \MultiplicativeGroupN{}$ such that $x^{2} = a$.
  \begin{subequations}\label{eq:Quadratic_Residue}
    \begin{equation}\label{subeq:Quadratic_Residue_1}
      a \in \IntsModN{} \exists x \in \MultiplicativeGroupN{}\:\: x^{2} = a \pmod{n}
    \end{equation}
    \begin{equation}\label{subeq:Quadratic_Residue_2}
      a \in \IntsModN{} \exists x \in \MultiplicativeGroupN{}\:\: a \equiv x^{2} \pmod{n}
    \end{equation}
  \end{subequations}

  \begin{remark}[Square Root]\label{rmk:Square_Root}
    If $x^{2} = a$, then $x$ is called the \emph{square root} of $a \bmod n$.
  \end{remark}

  Otherwise, $a$ is called a \emph{\nameref{def:Quadratic_Non_Residue} modulo $n$}.
\end{definition}

\begin{definition}[Quadratic Non-Residue]\label{def:Quadratic_Non_Residue}
  An element $a \in \MultiplicativeGroupN{}$ is said to be a \emph{quadratic non-residue} modulo $n$ if there does not exist an $x \in \MultiplicativeGroupN{}$ such that $x^{2} = a$.
  \begin{equation*}
    a \in \IntsModN{} \nexists x \in \MultiplicativeGroupN{}\:\: x^{2} = a \pmod{n}
  \end{equation*}
  Otherwise, $a$ is called a \emph{\nameref{def:Quadratic_Residue} modulo $n$}.
\end{definition}
%%% Local Variables:
%%% mode: latex
%%% TeX-master: "../EDIN01-Cryptography-Reference_Sheet"
%%% End:

%====================================APPENDIX====================================
\appendix
\counterwithin{definition}{subsection}

\clearpage
\section{Complex Numbers}
	\begin{equation} \label{eq:Exponential to Rectangular}
		A e^{-ix} = A \left[ \cos \left( x \right) + i\sin \left( x \right) \right]
	\end{equation}

\clearpage
\subsection{Trigonometry} \label{app:Trig}
	\subsubsection{Trigonometric Formulas} \label{subsubsec:Trig Formulas}
		\begin{equation} \label{eq:Sin plus Sin with diff Angles}
			\sin \left( \alpha \right) + \sin \left( \beta \right) = 2 \sin \left( \frac{\alpha + \beta}{2} \right) \cos\left( \frac{\alpha - \beta}{2} \right)  
		\end{equation}
		\begin{equation} \label{eq:Cosine-Sine Product}
			\cos \left( \theta \right) \sin \left( \theta \right) = \frac{1}{2} \sin \left( 2 \theta \right)
		\end{equation}

\clearpage
\subsection{Calculus} \label{app:Calculus}
	\subsubsection{Fundamental Theorems of Calculus} \label{subsubsec:Fundamental Theorem of Calculus}
		\begin{definition}[First Fundamental Theorem of Calculus] \label{def:1st Fundamental Theorem of Calculus}
			The \emph{first fundamental theorem of calculus} states that, if $f$ is continuous on the closed interval $\left[ a,b \right]$ and $F$ is the indefinite integral of $f$ on $\left[ a,b \right]$, then 
			\begin{equation} \label{eq:1st Fundamental Theorem of Calculus}
				\int_{a}^{b}f \left( x \right) dx = F \left( b \right) - F \left( a \right)
			\end{equation}
		\end{definition}
		\begin{definition}[Second Fundamental Theorem of Calculus] \label{def:2nd Fundamental Theorem of Calculus}
			The \emph{second fundamental theorem of calculus} holds for $f$ a continuous function on an open interval $I$ and $a$ any point in $I$, and states that if $F$ is defined by
			\begin{equation*}
				F \left( x \right) = \int_{a}^{x} f \left( t \right) dt,
			\end{equation*}
			then
			\begin{equation} \label{eq:2nd Fundamental Theorem of Calculus}
				\begin{aligned}
					\frac{d}{dx} \int_{a}^{x} f \left( t \right) dt &= f \left( x \right) \\
					F' \left( x \right) &= f \left( x \right) \\
				\end{aligned}
			\end{equation}
		\end{definition}

\clearpage
\section{Laplace Transform}\label{app:Laplace_Transform}
\subsection{Laplace Transform}\label{subsec:Laplace_Transform}
\begin{definition}[Laplace Transform]\label{def:Laplace_Transform}
  The \emph{Laplace transformation} operation is denoted as $\Lapl \lbrace x(t) \rbrace$ and is defined as
  \begin{equation}\label{eq:Laplace_Transform}
    X(s) = \int\limits_{-\infty}^{\infty} x(t) e^{-st} dt
  \end{equation}
\end{definition}

\subsection{Inverse Laplace Transform}\label{subsec:Inverse_Laplace_Transform}
\begin{definition}[Inverse Laplace Transform]\label{def:Inverse_Laplace_Transform}
  The \emph{inverse Laplace transformation} operation is denoted as $\Lapl^{-1} \lbrace X(s) \rbrace$ and is defined as
  \begin{equation}\label{eq:Inverse_Laplace_Transform}
    x(t) = \frac{1}{2j \pi} \int_{\sigma-\infty}^{\sigma+\infty} X(s) e^{st} \, ds
  \end{equation}
\end{definition}



 % To make this print, you must include a citation somewhere in the document
\printbibliography{}
\end{document}
%%% Local Variables:
%%% mode: latex
%%% TeX-master: t
%%% End:
