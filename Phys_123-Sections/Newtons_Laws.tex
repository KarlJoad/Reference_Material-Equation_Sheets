\section{Newton's Laws} \label{sec:Newtons Laws}
There are 3 fundamental laws of classical mechanics.

\begin{enumerate}[noitemsep, nolistsep]
  \item An object in motion/at reast stays as such, unless acted upon by an outide force(s).
  \item Force is equal to the change in momentum ($p=mv$) per change in time. For a constant mass, force equals mass times acceleration ($\vec{F} = m \vec{a}$).
  \item For every force there is an equal and opposite force.
\end{enumerate}

There are a few forces that are fundamental in our universe.

\begin{itemize}[noitemsep, nolistsep]
  \item Gravitational Force
  \item Magnetic Force
  \item Electric Force
  \item Frictional Force
  \item Normal Force
  \item Tension Force
\end{itemize}

\begin{remark*}
  \textbf{\emph{REMEMBER}} that forces are vectors!!
\end{remark*}

To solve any force problem, you should construct a free-body diagram.

\subsection{Newton's Laws in 1-D} \label{subsec:Newtons Laws 1-D}
\begin{example}[]{Atwood Machine}
  Find the acceleration that is the result of dropping one side of the machine?
  Neglect any friction that may occur because of the pulley.

  \tcblower

  \begin{align*}
    m_{2} \vec{a}_{2} &= m_{2} \vec{g} + \vec{T}_{2} \\
    m_{1} \vec{a}_{1} &= m_{1} \vec{g} + \vec{T}_{1}
  \end{align*}

  One thing to remember is that both masses will have the same acceleration, because the pulley has no friction.
  So because $\vec{a}_{2} = \vec{a}_{1}$, we can solve for a single $\vec{a}$ variable.

  \begin{align*}
    m_{2} \vec{a} &= m_{2} \vec{g} + \vec{T}_{2} \\
    m_{1} \vec{a} &= m_{1} \vec{g} + \vec{T}_{1} \\
    \vec{a} \left( m_{2} + m_{1} \right) &= \vec{g} \left( m_{2} - m_{1} \right) \\
    \vec{a} &= \frac{g \left( m_{2} - m_{1} \right)}{m_{2}+m_{1}}
  \end{align*}
\end{example}

\subsection{Newton's Laws in Multi-D} \label{subsec:Newtons Laws Multi-D}
For a multi-dimensional situation, you break the free-body problem's vectors down into their components.

\begin{example}[]{Sliding Block}
  If a block of mass $m$ slides down a ramp of angle $\theta$, what is the coefficient of static and kinetic friction?

  \tcblower

\end{example}
