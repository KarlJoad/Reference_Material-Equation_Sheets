\section{Newton's Laws}\label{sec:Newtons Laws}
There are 3 fundamental laws of classical mechanics.

\begin{enumerate}[noitemsep, nolistsep]
  \item An object in motion/at reast stays as such, unless acted upon by an outide force(s).
  \item Force is equal to the change in momentum ($p=mv$) per change in time. For a constant mass, force equals mass times acceleration ($\vec{F} = m \vec{a}$).
  \item For every force there is an equal and opposite force.
\end{enumerate}
There are a few forces that are fundamental in our universe.

\begin{itemize}[noitemsep, nolistsep]
  \item \nameref{subsubsec:Gravitational Force}
  \item \nameref{subsubsec:Magnetic Force}
  \item \nameref{subsubsec:Electric Force}
  \item \nameref{subsubsec:Frictional Force}
  \item \nameref{subsubsec:Normal Force}
  \item Tension Force
\end{itemize}

\begin{remark*}
  \textbf{\emph{REMEMBER}} that forces are vectors!!
\end{remark*}

To solve any force problem, you should construct a free-body diagram.

\subsection{Newton's Laws in 1-D}\label{subsec:Newtons Laws 1-D}
\begin{example}[]{Atwood Machine}
  Find the acceleration that is the result of dropping one side of the machine?
  Neglect any friction that may occur because of the pulley.

  \tcblower

  \begin{align*}
    m_{2} \vec{a}_{2} &= m_{2} \vec{g} + \vec{T}_{2} \\
    m_{1} \vec{a}_{1} &= m_{1} \vec{g} + \vec{T}_{1}
  \end{align*}

  One thing to remember is that both masses will have the same acceleration, because the pulley has no friction.
  So because $\vec{a}_{2} = \vec{a}_{1}$, we can solve for a single $\vec{a}$ variable.

  \begin{align*}
    m_{2} \vec{a} &= m_{2} \vec{g} + \vec{T}_{2} \\
    m_{1} \vec{a} &= m_{1} \vec{g} + \vec{T}_{1} \\
    \vec{a} \left( m_{2} + m_{1} \right) &= \vec{g} \left( m_{2} - m_{1} \right) \\
    \vec{a} &= \frac{g \left( m_{2} - m_{1} \right)}{m_{2}+m_{1}}
  \end{align*}
\end{example}

\subsection{Newton's Laws in Multi-D}\label{subsec:Newtons Laws Multi-D}
For a multi-dimensional situation, you break the free-body problem's vectors down into their components.

\begin{example}[]{Sliding Block}
  If a block of mass $m$ slides down a ramp of angle $\theta$, what is the coefficient of static and kinetic friction?

  \tcblower

\end{example}

\subsection{Common Forces}\label{subsec:Common Forces}
This section will discuss various forces in greater detail.

\subsubsection{Gravitational Force}\label{subsubsec:Gravitational Force}
\begin{definition}[Gravitational Force]\label{def:Gravitational Force}
  \emph{Gravitational force} is a force that arises from \nameref{def:Gravity}.
  \nameref{def:Gravity} will be covered more in \Cref{sec:Gravitation}, \nameref{sec:Gravitation}.
  This force arises when one object is being pulled towards another due to \nameref{def:Gravity}.
  \nameref{def:Gravitational Force} is generally the acceleration vector due to Earth's gravity.

  \begin{equation}\label{eq:Gravitational Force}
    \vec{g} = 9.81 \si{\meter / \second^{2}}
  \end{equation}
\end{definition}

\subsubsection{Magnetic Force}\label{subsubsec:Magnetic Force}
This force is discussed in much greater detail in the Physics 221: Electromagnetics and Optics course.

\subsubsection{Electric Force}\label{subsubsec:Electric Force}
This force is discussed in much greater detail in the Physics 221: Electromagnetics and Optics course.

\subsubsection{Frictional Force}\label{subsubsec:Frictional Force}
Frictional force is the common culprit for the ``Equal and Opposite Force'' in classical mechanics.

\begin{definition}[Frictional Force]\label{def:Frictional Force}
  Frictional force is defined as

  \begin{equation}\label{eq:Frictional Force}
    \vec{F}_{f} = \mu \vec{N}
  \end{equation}
  \begin{itemize}[noitemsep, nolistsep]
    \item $\mu$ - The coefficient of the type of friction you are dealing with (Either the \nameref{def:Coefficient of Static Friction} or \nameref{def:Coefficient of Kinetic Friction})
    \item $\vec{N}$ - The \nameref{def:Normal Force}
  \end{itemize}
\end{definition}

There are 2 types of frictional force:
\begin{enumerate}[noitemsep, nolistsep]
  \item \nameref{def:Static Friction}
  \item \nameref{def:Kinetic Friction}
\end{enumerate}

\begin{definition}[Static Friction]\label{def:Static Friction}
  \emph{Static friction} is friction that arises when an object is starting from a static position, and is being moved.
  This force tends to be stronger than \nameref{def:Kinetic Friction} because of electrostatic bonds between the object and it's supporting surface, along with other reasons.
  However, static friction is drawn from the \nameref{def:Coefficient of Static Friction}.
  This is a \nameref{def:Scalar} number that represents how much the object \emph{does not} want to move.

  \begin{equation}\label{eq:Static Friction}
    \vec{F}_{f,s} = \mu_{s} \vec{N}
  \end{equation}
  \begin{itemize}[noitemsep, nolistsep]
    \item $\mu_{s}$ - The \nameref{def:Coefficient of Static Friction}
    \item $\vec{N}$ - The \nameref{def:Normal Force}
  \end{itemize}
\end{definition}

\begin{definition}[Coefficient of Static Friction]\label{def:Coefficient of Static Friction}
  \emph{Coefficient of static friction} is a \nameref{def:Scalar} number that represents how much the object does not want to \textbf{START} moving.
  This value tends to be greater than the \nameref{def:Coefficient of Kinetic Friction}.
  This is because there are additional bonds and forces at play that resist the start of an objects motion.
  The \nameref{def:Coefficient of Kinetic Friction} is denoted as such:
  \begin{equation}\label{eq:Coefficient of Static Friction}
    \mu_{s}
  \end{equation}
\end{definition}

\begin{definition}[Kinetic Friction]\label{def:Kinetic Friction}
  \emph{Kinetic friction} is a frction that arises when an object is in motion.
  This force is weaker than \nameref{def:Static Friction}.
  The value we may use for kinetic friction is drawn from the \nameref{def:Coefficient of Kinetic Friction}.

  \begin{equation}\label{eq:Kinetic Friction}
    \vec{F}_{f,k} = -\mu_{k} \vec{N}
  \end{equation}
  \begin{itemize}[noitemsep, nolistsep]
    \item $\mu_{k}$ - The \nameref{def:Coefficient of Kinetic Friction}
    \item $\vec{N}$ - The \nameref{def:Normal Force} that is going \textbf{\emph{IN}} the direction of motion. You might need to break the normal force's vector down into its components.
  \end{itemize}
\end{definition}

\begin{definition}[Coefficient of Kinetic Friction]\label{def:Coefficient of Kinetic Friction}
  \emph{Coefficient of kinetic friction} is \nameref{def:Scalar} number that represents how much the object does not want to \textbf{CONTINUE} moving.
  This value tends to be smaller than its \nameref{def:Coefficient of Static Friction} counterpart.
  The \nameref{def:Coefficient of Kinetic Friction} is denoted as such:
  \begin{equation}\label{eq:Coefficient of Kinetic Friction}
    \mu_{k}
  \end{equation}
\end{definition}

\subsubsection{Normal Force}\label{subsubsec:Normal Force}
\begin{definition}[Normal Force]\label{def:Normal Force}
  The \emph{normal force} is, as the name implies, a normalizing force.
  This does not necessarily make it special, as both \nameref{def:Kinetic Friction} and \nameref{def:Static Friction} can be considered normalizing forces as well.
  However, the normal force is generally considered whenever something is being held against something.
  For example, a book on a table.
  The table is exerting a \nameref{def:Normal Force} on the book to prevent it from falling through the table.
  Likewise, the ground is exerting a \nameref{def:Normal Force} on the table to prevent it from falling through the ground.

  This normal force is defined as

  \begin{equation}\label{eq:Normal Force}
    \vec{N} = -m \vec{g}
  \end{equation}
\end{definition}

%%% Local Variables:
%%% mode: latex
%%% TeX-master: "../Phys_123-Reference_Sheet"
%%% End:
