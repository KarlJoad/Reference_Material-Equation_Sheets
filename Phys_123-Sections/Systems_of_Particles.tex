\section{Systems of Particles}\label{sec:Systems of Particles}
Up until now we have only been considering the objects that we work with to be single, equally distributed masses.
However, in the real world, we have strangely shaped things and objects that have multiple materials inside of them for balance and strength.

To account for oddities in these objects, we calculate something called the \nameref{def:Center of Mass}.

\subsection{Center of Mass}\label{subsec:Center of Mass}
\begin{definition}[Center of Mass]\label{def:Center of Mass}
  The \emph{center of mass} of an object is a weighted average of all particles in a system.
  Center of mass takes the mass of a point and the distance from the center the point is into account, then normalizes by mass.

  \begin{equation}\label{eq:Center of Mass}
    \vec{R} = \frac{\sum\limits_{i=1}^{n} m_{i} \vec{r}_{i}}{\sum\limits_{i=1}^{n} m_{i}}
  \end{equation}
\end{definition}

\begin{example}[]{Center of Mass of Atomic Bond}
  Take a carbon monoxide molecule, CO.
  Carbon has a mass of 12 \si{\atomicmassunit} and Oxygen has a mass of 16 \si{\atomicmassunit}.
  The bond between them is $a$ long.
  What is the center of mass of this system?

  \tcblower

  We need to start by having a reference point, which we can define to be the center of the carbon atom.
  So, $r_{\text{C}} = 0$ and $r_{\text{O}} = a$.
  \begin{align*}
    \vec{R} &= \frac{m_{\text{C}} (0) + m_{\text{O}} (a)}{m_{\text{C}} + m_{\text{O}}} \\
    \vec{R} &= \frac{12 (0) + 16 (a)}{12 + 16} \\
    \vec{R} &= \frac{16}{28} a \\
    \vec{R} &= \frac{4}{7} a
  \end{align*}

  So, the center of mass of this carbon monoxide molecule is $\frac{4}{7}$ of the way towards the Oxygen atom.
\end{example}

\subsubsection{Center of Mass in Multiple Dimensions}\label{subsubsec:Multi-D Center of Mass}
\nameref{def:Center of Mass} can be applied to systems in multiple dimensions.
All that must be done is that the distances be calculated for each dimension separately.

\begin{example}[]{Center of Mass of Atomic Bond in 2-D}
  Given a water atom, $\text{H}_{2} \text{0}$, what is its center of mass?
  \begin{itemize}[noitemsep, nolistsep]
    \item Mass of Hydrogen is 1 \si{\atomicmassunit}
    \item Mass of Oxygen is 16 \si{\atomicmassunit}
    \item Distance between hydrogen bonds is $a$
  \end{itemize}

  \tcblower

  We can start by placing the hydrogen atoms on the $x$-axis of an $xy$-plane.
  The oxygen atom will sit on the $y$-axis.
  The variable $h$ stands for the distance between the oxygen atom and the hydrogen bond location. \\

  The center of mass, $\vec{R}$ will be broken down into its components.
  The $x$ portion of the center of mass will be given the variable $\vec{X}$.
  The $y$ portion of the center of mass will be given the variable $\vec{Y}$.

  \begin{align*}
    \vec{X} &= \frac{m_{\text{H}} \left( \frac{-a}{2} \right) + m_{\text{H}} \left( \frac{-a}{2} \right) + m_{\text{O}}(0)}{m_{\text{H}} + m_{\text{H}} + m_{\text{O}}} = \frac{0}{2m_{\text{H}} + m_{\text{O}}} \\
    \vec{X} &= 0
  \end{align*}
  Since the oxygen atom is at $x=0$, it does not contribute to the center of mass in the $x$ direction.

  \begin{align*}
    \vec{Y} &= \frac{m_{\text{O}} (h) + m_{\text{H}}(0) + m_{\text{H}}(0)}{m_{\text{O}} + m_{\text{H}} + m_{\text{H}}} \\
    \vec{Y} &= \frac{m_{\text{O}}}{m_{\text{O}} + 2m_{\text{H}}} h \\
    \vec{Y} &= \frac{16}{16 + 2(1)} h = \frac{16}{18} h \\
    \vec{Y} &= \frac{8}{9} h
  \end{align*}
  Since the hydrogen atoms are on $y=0$, they do not contribute to the center of mass in the $y$ direction.

  Combining both the $x$ and $y$ solutions, we come up with
  \begin{equation*}
    \vec{R} = \biggl \langle 0, \frac{8}{9} h \biggr \rangle
  \end{equation*}
\end{example}

%%% Local Variables:
%%% mode: latex
%%% TeX-master: "../Phys_123-Reference_Sheet"
%%% End: