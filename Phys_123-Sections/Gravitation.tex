\section{Gravitation} \label{sec:Gravitation}
\begin{definition}[Gravity] \label{def:Gravity}
  \emph{Gravity} is one of \nameref{sec:Fundamental Interactions}.
  It is also one of the least understood.
  However, it it the thing that is keeping us ``glued'' to this plant.

  There is an equation that describes the force caused by gravity on multiple objects.

  \begin{equation} \label{eq:Gravitation}
    \vec{F}_{g} = G \frac{m_{1}m_{2}}{r^{2}} \cdot \frac{\vec{r}}{\hat{r}}
  \end{equation}
  \begin{itemize}[noitemsep, nolistsep]
    \item $G$ - The universal gravitational constant. The actual value is in \Cref{app:Physical Constants}.
    \item $m_{1}$ - The mass of one of the objects.
    \item $m_{2}$ - The mass of the other object.
    \item $r^{2}$ - Distance between the objects.
    \item $\vec{r}$ - Vector describing the direction of the distance.
    \item $\hat{r}$ - Normalizing unit vector
  \end{itemize}

  \begin{remark}
    \Cref{eq:Gravitation} only applies if:
    \begin{itemize}[noitemsep, nolistsep]
      \item Relativistic effects are small
      \item Relatively low density, compared to a black hole for instance
      \item Relatively low velocity, compared to light speed
    \end{itemize}
  \end{remark}
\end{definition}

\begin{definition}[Kepler's Laws] \label{def:Keplers Laws}
  Johannes Kepler developed these three laws for gravitation and planetary motion.

  \begin{enumerate}[noitemsep, nolistsep]
    \item Planets move in ellipses with one focus at the sun.
    \item Planets move fastest when close to the sun, and slowest when furthest from the sun, in the same time. The area of sectors in the ellipse are the same.
    \item The sun's side is perihelion, the far side is the aphelion. $\text{Period}^{2} \propto \text{Semi-Major}^{3}$. \begin{equation}
        \frac{T_{1}^{2}}{a_{1}^{3}} = \frac{T_{2}^{2}}{a_{2}^{3}}
      \end{equation}
    \end{enumerate}

    Using the information from \Cref{def:Gravity}, we can construct an updated set of \nameref{def:Keplers Laws}.
    \begin{enumerate}[noitemsep, nolistsep]
      \item All trajectories are conic sections.
      \item Angular momentum, $\vec{\ell}$ is conserved.
      \item $\frac{T^{2}}{a^{3}} = \frac{4 \pi^{2}}{G \left( m + M \right)}$.
    \end{enumerate}
\end{definition}
