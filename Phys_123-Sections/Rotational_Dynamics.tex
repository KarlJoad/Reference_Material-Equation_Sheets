\section{Rotational Dynamics}\label{sec:Rotational Dynamics}
\emph{Rotational dynamics} applies the concepts from \nameref{sec:Newtons Laws}, \nameref{sec:Energy}, \nameref{sec:Systems of Particles}, and \nameref{sec:Momentum} to rotating objects.

\subsection{Moment of Inertia}\label{subsec:Moment of Inertia}
\begin{definition}[Moment of Inertia]\label{def:Moment of Inertia}
  A \emph{moment of inertia} or \emph{moment} is quite similar to \nameref{def:Center of Mass}.
  It is the point on the object where the mass and distance from this point are equal throughout the object.
  It can be thought of as the ``balancing'' point for rotation.

  \begin{equation}\label{eq:Moment of Inertia}
    I = \sum\limits_{i=1}^{n} m_{i} r_{i}^{2}
  \end{equation}

  \begin{remark}
    Because we have made the constraint that all the objects we will be dealing with are \nameref{def:Rigid Body} objects, that means our angular velocity, $\omega$ is constant everywhere.
  \end{remark}
  \begin{remark}
    Moments of Inertia are additive.
  \end{remark}
\end{definition}

\begin{theorem}[Parallel Axis Theorem]\label{thm:Parallel Axis Theorem}
  Suppose a body of mass $m$ is made to rotate about an axis $z$ passing throught the body's \nameref{def:Center of Mass}.
  The body has a \nameref{def:Moment of Inertia} $I_{\text{CoM}}$ with respect to this axis.
  The \nameref{thm:Parallel Axis Theorem} states that if the body is made to rotate about a new axis $z'$ which is parallel to the first axis and displaced from it by a distance $d$, then the \nameref{def:Moment of Inertia} $I$ with respect to the new axis is related to $I_{\text{CoM}}$ by
  \begin{equation}\label{eq:Parallel Axis Theorem}
    I = I_{\text{CoM}} + md^{2}
  \end{equation}
\end{theorem}

\subsubsection{Common Moments of Inertia}\label{subsubsec:Common Moments of Inertia}
\begin{table}[h!]
  \centering
  \renewcommand{\arraystretch}{1.5}
  \begin{tabular}{cc}
    Object Type & \nameref{def:Moment of Inertia} Formula \\ \hline
    Slender Rod, Axis Through Center (Not Length-wise Axis) & $I = \frac{1}{12} m h^{2}$ \\ \hline
    Slender Rod, Axis at End & $I = \frac{1}{3} m h^{2}$ \\ \hline
    Rectangular Plate, Axis Through Center & $I = \frac{1}{12} m \left( a^{2} + b^{2} \right)$ \\ \hline
    Rectangular Plate, Axis along Edge & $I = \frac{1}{3} \left( \perp \text{Side} \right)^{2}$ \\ \hline
    Hollow Cylinder & $I = \frac{1}{2} m \left( r_{\text{Outer}}^{2} - r_{\text{Inner}}^{2} \right)$ \\ \hline
    Solid Cylinder & $I = m r^{2}$ \\ \hline
    Thin Walled Cylinder (Soda can thin) & $I = m r^{2}$ \\ \hline
    Solid Sphere & $\frac{2}{5} m r^{2}$ \\ \hline
    Thin Walled Sphere & $I = \frac{2}{3} m r^{2}$ \\ \hline
  \end{tabular}
\end{table}

\subsection{Torque}\label{subsec:Torque}
\begin{definition}[Torque]\label{def:Torque}
  \emph{Torque} is the force that is applied when making an object rotate.
  There are two portions to it.
  The force that is being applied, $\vec{F}$ at some radius $r$ from the point of rotation.

  There are 2 equations that can be applied for \nameref{def:Torque}.

  \begin{equation}\label{eq:Torque Cross Product}
    \vec{\tau} = \vec{r} \cross \vec{F}
  \end{equation}

  \begin{equation}\label{eq:Torque Magnitudes}
    \vec{\tau} = \lVert \vec{r} \rVert \lVert \vec{F} \rVert \sin (\theta)
  \end{equation}
\end{definition}

\subsection{Angular Momentum}\label{subsec:Angular Momentum}
\begin{definition}[Angular Momentum]\label{def:Angular Momentum}
  \emph{Angular momentum} is similar to Linear \nameref{def:Momentum}, but we define it in terms of rotation components.

  \begin{equation}\label{eq:Angular Momentum}
    \vec{\ell} = \sum\limits_{i=1}^{n} \vec{r}_{i} \cross \vec{p}_{i}
  \end{equation}
\end{definition}

\begin{definition}[Conservation of Angular Momentum]\label{def:Conservation of Angular Momentum}
  The \emph{conservation of angular momentum} states that if $\vec{\tau}_{\text{Net}} = \vec{0}$, then $\vec{\ell} = \text{Constant}$.
\end{definition}
