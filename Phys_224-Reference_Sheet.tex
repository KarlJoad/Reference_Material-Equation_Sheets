\documentclass[10pt,letterpaper,final,twoside,notitlepage]{article}
\usepackage[margin=.5in]{geometry}
\usepackage[utf8]{inputenc}
\usepackage[english]{babel}
\usepackage{amsmath}
\usepackage{amsfonts}
\usepackage{amssymb}
\usepackage{graphicx}

\usepackage{subfigure} % Float environment to place multiple figures in one figure environment
\usepackage{nameref} % \nameref{label} lets you reference things by name
\usepackage{hyperref} % Hyperlinks between references
\usepackage{enumitem} % Provides [noitemsep, nolistsep] for more compact lists

\graphicspath{{./Drawings/Phys_224}} % Uncomment this to use pictures in this document
\numberwithin{equation}{section} % Uncomment this to number equations with section numbers too

\author{Karl Hallsby}
\title{Phys 224 Reference Sheet}

\begin{document}
\section{General Stuff} \label{sec:General}
	\begin{itemize}[noitemsep]
		\item Density - $\rho = \frac{\Delta m}{\Delta V}$
			\begin{itemize}
				\item Uniform Density - $\rho = \frac{m}{V}$
			\end{itemize}
		
		\item Pressure - $p = \frac{\Delta F}{\Delta A}$
			\begin{itemize}
				\item Uniform Force on Flat Area - $\rho = \frac{F}{A}$
				\item Conversions - $1 atm = 1.01 \times 10^5 Pa = 760 torr = 14.7 lb/in^2$
			\end{itemize}
		\item Trig Relations are in \nameref{subsec:Trig Formulas}, Section~\ref{subsec:Trig Formulas}.
	\end{itemize}

\section{Fluids} \label{sec:Fluids}
We must satisfy several parameters to make life easier, and to use most of these formulae.
	\begin{enumerate}[noitemsep]
		\item Incompressible - Density of the fluid is constant
		\item Non-turbulent Flow - Think of fluids swirling around an object
		\item Isostatic Pressure - Pressure inside the fluid is the same in all directions
	\end{enumerate}

	\begin{itemize}[noitemsep]
		\item Pressure at Some Depth - $p_{2} = p_{1} + \rho g \left( y_{1}-y_{2} \right)$
			\begin{itemize}
				\item Pressure at Depth $h \rightarrow p = p_{0} + \rho gh$
			\end{itemize}
		\item Pascal's Principle - 2 Parts
			\begin{enumerate}[noitemsep]
				\item $\vec{F_{o}} = \vec{F}_{i}\ \frac{A_{o}}{A_{i}}$
				\item $d_{o} = d_{i} \frac{A_{i}}{A_{o}}$
			\end{enumerate}
		\item Archimede's Principle - $\vec{F}_{Up} = \vec{F}_{Down}$
			\begin{itemize}[noitemsep]
				\item $\vec{F}_{Bouyant} = m_{Floating}g$
			\end{itemize}
		\item Continuity - $A_{1}v_{1} = A_{2}v_{2}$
		\item Bernoulli's Equation - $p_{1} + \frac{1}{2} \rho v_{1}^{2} + \rho gy_{1} = p_{2} + 	\frac{1}{2} \rho v_{2}^{2} + \rho gy_{2}$
			\begin{itemize}[noitemsep]
				\item Fluids at Rest - $p_{2} = p_{1} + \rho g \left( y_{1}-y_{2} \right)$
				\item Fluids not Changing Height - $p_{1} + \frac{1}{2} \rho v_{1}^{2} = p_{2} + 	\frac{1}{2} \rho v_{2}^{2}$
			\end{itemize}
	\end{itemize}

\section{Waves} \label{sec:Waves}
{\Large Usually of form $y = y_{m} \sin \left( kx \pm \omega t \right)$}
	\begin{itemize}[noitemsep]
		\item $y_{m}$ - Amplitude, $m$
		\item $k$ - Angular Wave Number, $rad/m$
			\begin{itemize}[noitemsep, nolistsep]
				\item $k = \frac{2 \pi}{\lambda}$
				\item $\lambda$ is wavelength, $m$
			\end{itemize}
		\item $\omega$ - Angular Frequency, $rad/s$
			\begin{itemize}[noitemsep, nolistsep]
				\item $\omega = 2 \pi f$
				\item $f$ is frequency, $Hz$
				\item Sign of this goes the opposite the direction the wave is going
					\begin{enumerate}[noitemsep]
						\item Wave going in positive direction $\left( + \right)$, then the sign should be negative $\left( - \right)$
						\item Wave going in negative direction $\left( - \right)$, then the sign should be positive $\left( + \right)$
					\end{enumerate}
			\end{itemize}
		\item $v = \lambda f$, Wave Velocity, $m/s$
			\begin{itemize}[noitemsep]
				\item $v = \frac{\omega}{2 \pi} * \frac{2 \pi}{k} = \frac{\omega}{k}$
			\end{itemize}
	\end{itemize}

	\subsection*{Wave Interference} \label{subsec:Wave Interference}
	Waves are nice, and they just sum when they interfere. Let: 
	\begin{align}
		y_{1} \left( x, t \right) &= y \sin \left( kx - \omega t \right) \\
		y_{2} \left( x, t \right) &= y \sin \left( kx + \omega t + \varphi \right) \\
		Y \left( x, t \right) &= y\left[ \sin \left( kx - \omega t \right) + \sin \left( kx + \omega t + \varphi \right) \right] \label{eq:Sum of 2 Waves}
	\end{align}
	You can usually use Equation~\ref{eq:Sin plus Sin with diff Angles} to simplify Equation~\ref{eq:Sum of 2 Waves}.

	\subsection*{Standing Waves} \label{subsec:Standing Waves}
	This is actually the superposition of 2 waves, traveling in opposite directions, on a medium that is fixed at both ends, i.e. a taut string held by a wall.
		\subsubsection*{Location of Nodes and Antinodes} \label{subsubsec:Node/Antinode Location}
		\begin{itemize}[noitemsep]
			\item Nodes - $x = n \frac{\lambda}{2}$ for $n = 0, 1, 2, \ldots$
			\item Antinodes - $x = \left( n + \frac{1}{2} \right) \frac{\lambda}{2}$ for $n = 0, 1, 2, \ldots$
		\end{itemize}

		\subsubsection*{Resonant Frequencies/Harmonics} \label{subsubsec:Resonant Frequencies/Harmonics}
		These can also be called harmonics. There is a resonant frequency for every number of nodes/antinodes on tthe standing wave. \newline
		$f = \frac{v}{\lambda} = n \frac{v}{2L}$
		\begin{itemize}[noitemsep]
			\item $L$ is the length of the medium (The String).
			\item $\lambda$ is the wavelength of the wave formed.
		\end{itemize}
		This can be extended to find the base resonant frequency, if you know how many node levels are between the two resonant frequencies given, i.e. they say that the \textbf{NEXT} frequency, means $n+1$. \newline
		$f_{n+m}-f_{n} = \left( n+m \right) \frac{v}{2L} - n \frac{v}{2L} = m \frac{v}{2L}$
		
	\subsection*{Reflecting Sound} \label{subsec:Reflecting Sound}
		\begin{itemize}[noitemsep]
			\item $D = \left( n + 1 \right)d = vt$
			\begin{itemize}[noitemsep,nolistsep]
				\item $n$ is the number of reflections that occurred
			\end{itemize}
		\end{itemize}

	\subsection*{Doppler Effect} \label{subsec:Doppler Effect}
	{\Large $f' = f \frac{v \pm v_{D}}{v \pm v_{S}}$}
	\begin{itemize}[noitemsep]
		\item Moving \textit{\textbf{TOWARDS}} each other: Frequency Increase
		\item Moving \textit{\textbf{AWAY}} from each other: Frequency Decrease
		\item $f$ - Initial Frequency, $Hz$
		\item $v$ - Sound Speed, $m/s$
		\item $v_{D}$ - Detector Speed, $m/s$
		\item $v_{S}$ - Source Speed, $m/s$
		\item \textbf{For Numerator}:
			\begin{itemize}[noitemsep,nolistsep]
				\item If detector is moving towards the source, $+$
				\item If detector is moving away from the source, $-$
			\end{itemize}
		\item \textbf{For Denominator}:
			\begin{itemize}[noitemsep,nolistsep]
				\item If source is moving away from detector, $+$
				\item If source is moving towards the detector, $-$
			\end{itemize}
		\end{itemize}

\section{Thermodynamics} \label{sec:Thermo}
\section{Quantum Mechanics} \label{sec:Quantum Mech}
\section{Reference Material} \label{sec:Reference Material}
	\subsection{Trigonometric Formulas} \label{subsec:Trig Formulas}
	\begin{align}
		\sin \left( \alpha \right) + \sin \left( \beta \right) &= 2 \sin \left( \frac{\alpha + \beta}{2} \right) \cos\left( \frac{\alpha - \beta}{2} \right)  \label{eq:Sin plus Sin with diff Angles}
\end{align}
\end{document}