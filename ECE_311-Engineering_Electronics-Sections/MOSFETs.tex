\section{MOS Field-Effect Transistors}\label{sec:MOSFETs}
In this section, we start studying \nameref{def:Transistor}s.

\begin{definition}[Transistor]\label{def:Transistor}
  A \emph{transistor} is a \nameref{def:Semiconductor} device used to amplify or switch electronic signals and electrical power.
  Transistors are one of the basic building blocks of modern electronics.
  It is composed of semiconductor material usually with at least three terminals for connection to an external circuit.
\end{definition}

The \nameref{def:MOSFET} is the first \nameref{def:Transistor} we will be studying in this course.
It is the second oldest transistor, but is the most frequently used one today, particularly for digital applications.

\begin{definition}[MOSFET]\label{def:MOSFET}
  \emph{MOSFET}, short for \emph{Metal-Oxide-\nameref{def:Semiconductor} Field-Effect \nameref{def:Transistor}}, is an insulated-gate field-effect transistor.

  The insulated-gate portion of its name implies that the gate is completely isolated from the rest of the circuit.
  This is achieved with the metal oxide layer (seen in \Cref{fig:MOSFET-Physical_Structure-Cross_Section}) acting as an insulator, preventing any current from entering the transistor from that terminal.

  The field-effect portion of the name implies that the electric field of the gate is the driving force in this circuit.
  Because no current is allowed through the gate terminal of the transistor, this only has a voltage applied, causing an electric field to form.

  The typical circuit symbol (for \nameref{def:NMOS}) is shown in \Cref{fig:MOSFET-Symbol-NMOS}.
\end{definition}


%%% Local Variables:
%%% mode: latex
%%% TeX-master: "../ECE_311-Engineering_Electronics-Reference_Sheet"
%%% End:
