\section{Semiconductors}\label{sec:Semiconductors}
\subsection{Intrinsic Semiconductors}\label{subsec:Intrinsic_Semiconductors}
\begin{definition}[Semiconductor]\label{def:Semiconductor}
  \emph{Semiconductor}s are materials whose conductivity is somewhere between that of true conductors, like copper, and insulators, such as glass.
  Because semiconductors are somewhere between conductors and insulators, they have electrical properties that are easily manipulated through \nameref{def:Doping}.
\end{definition}


\subsection{Doped Semiconductors}\label{subsec:Doped_Semiconductors}

\subsection{Current Flow in Semiconductors}\label{subsec:Semiconductors_Current_Flow}
\subsubsection{Drift Current}\label{subsubsec:Drift_Current}

\subsubsection{Diffusion Current}\label{subsubsec:Diffusion_Current}
\subsection{The \PNJunction{} Junction}\label{subsec:The_pn_Junction}

\subsection{The \PNJunction{} Junction with Applied Voltage}\label{subsec:The_pn_Junction-Voltage_Applied}

\subsection{Depletion Layer}\label{subsec:Depletion_Layer}
\begin{definition}[Depletion Layer]\label{def:Depletion_Layer}
  The \emph{depletion layer} is the location in the \PNJunction{} where the two differently-doped sides meet.
  Here, there is a barrier of the opposing carrier on each side.
  This is visualized in \Cref{fig:Depletion_Layer}.
\end{definition}

\begin{figure}[h!tbp]
  \centering
  \includegraphics[scale=0.5]{./Depletion_Layer.png}
  \caption{Depletion Layer (\cite[p.~150]{sedraTextbook7})}
  \label{fig:Depletion_Layer}
\end{figure}

We can find the width of the \nameref{def:Depletion_Layer} using \Cref{eq:Depletion_Layer_Width}.
\begin{equation}\label{eq:Depletion_Layer_Width}
  \DepletionDistance = \sqrt{\frac{2 \SiElectricPermittivity}{\eCharge} \left( \frac{1}{\AcceptorConcentration} + \frac{1}{\DonorConcentration} \right) \JunctionBuiltInVoltage}
\end{equation}

The depletion layer ``bleeds'' into each side of the \PNJunction{}.
We can find the distance the depletion layer falls into each side with \Cref{eq:Depletion_Layer_Directions-Electron,eq:Depletion_Layer_Directions-Hole}.

\begin{subequations}\label{eq:Depletion_Layer_Directions}
  \begin{equation}\label{eq:Depletion_Layer_Directions-Electron}
    x_{\Electron} = \DepletionDistance \left( \frac{\AcceptorConcentration}{\AcceptorConcentration + \DonorConcentration} \right)
  \end{equation}
  \begin{equation}\label{eq:Depletion_Layer_Directions-Hole}
    x_{\Hole} = \DepletionDistance \left( \frac{\DonorConcentration}{\AcceptorConcentration + \DonorConcentration} \right)
  \end{equation}
\end{subequations}

Lastly, the sum of the ``bleed'' in both directions is equal to the width of the entire \nameref{def:Depletion_Layer}.
\begin{equation}\label{eq:Depletion_Layer-Directions_Sum}
  \DepletionDistance = x_{\Electron} + x_{\Hole}
\end{equation}

%%% Local Variables:
%%% mode: latex
%%% TeX-master: "../ECE_311-Engineering_Electronics-Reference_Sheet"
%%% End:
