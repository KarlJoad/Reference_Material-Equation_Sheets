\section{Semiconductors}\label{sec:Semiconductors}
\subsection{Intrinsic Semiconductors}\label{subsec:Intrinsic_Semiconductors}
\begin{definition}[Semiconductor]\label{def:Semiconductor}
  \emph{Semiconductor}s are materials whose conductivity is somewhere between that of true conductors, like copper, and insulators, such as glass.
  Because semiconductors are somewhere between conductors and insulators, they have electrical properties that are easily manipulated through \nameref{def:Doping}.
\end{definition}

\begin{definition}[Electron]\label{def:Electron}
  An \emph{electron} in this scenario is a \textbf{free electron}.
  This means the electron is not bound to any particular atomic nucleus.
  Such an electron is free to conduct electric current if an electric field is applied.

  If an atom is missing electrons due to an electron being free, a \nameref{def:Hole} can be thought of in its place.
\end{definition}

The concentration of free \nameref{def:Electron}s in a material is given the symbol shown in \Cref{eq:Concentration_Free_Electrons}.

\begin{equation}\label{eq:Concentration_Free_Electrons}
  \ElectronConcentration
\end{equation}

\begin{definition}[Hole]\label{def:Hole}
  A \emph{hole} is the lack of an \nameref{def:Electron} being attached to an atom.
  These have the same, but opposite charge of an electron.

  \textbf{These are NOT particles in any physical sense}.
  Holes are useful abstractions to use when thinking about current flow in a semiconductor's crystal lattice.
\end{definition}

The concentration of free \nameref{def:Hole}s in a material is given the symbol shown in \Cref{eq:Concentration_Holes}.

\begin{equation}\label{eq:Concentration_Holes}
  \HoleConcentration
\end{equation}

\begin{definition}[Intrinsic]\label{def:Intrinsic}
  An \emph{intrinsic} material is one that is pure.
  It has a regular lattice structure, where atoms are held in place by covalent bonds.
\end{definition}

In an \nameref{def:Intrinsic} material, the concentration of free \nameref{def:Electron}s and \nameref{def:Hole}s are equal.
This is represented by the relation in \Cref{eq:Electron_Hole_Concentration-Intrinsic}.

\begin{equation}\label{eq:Electron_Hole_Concentration-Intrinsic}
  \ElectronConcentration = \HoleConcentration = \HoleElectronConcentration
\end{equation}

Typically, we express the product of \nameref{def:Hole} and free \nameref{def:Electron} concentrations as a product, shown in \Cref{eq:Electron_Hole_Concentration}.

\begin{equation}\label{eq:Electron_Hole_Concentration}
  \HoleConcentration \ElectronConcentration = \HoleElectronConcentration^{2}
\end{equation}

Semiconductor physics tells us that \HoleElectronConcentration{} is defined by \Cref{eq:Hole_Electron_Concentration}.
\begin{equation}\label{eq:Hole_Electron_Concentration}
  \HoleElectronConcentration = B \Temp^{\frac{3}{2}} e^{\frac{-\BandgapEnergy}{2 \BoltzmannConstant \Temp}}
\end{equation}

where
\begin{description}[noitemsep]
\item $B$ is a material-dependent parameter.
\item $\Temp$ is the temperature
\item $\BoltzmannConstant$ is Boltzmann's constant.
\end{description}

\subsection{Doped Semiconductors}\label{subsec:Doped_Semiconductors}
\begin{definition}[Doping]\label{def:Doping}
  \emph{Doping} is the process of deliberately adding atomic impurities to alter the electrical/conductivity characteristics of \nameref{def:Semiconductor}s.
  This is done by substantially increasing the concentration of \nameref{def:Electron}s or \nameref{def:Hole}s, but without changing the crystal properties of the original semiconductor.
\end{definition}

There are two kinds of doping:
\begin{enumerate}[noitemsep]
\item \ElectronConcentration{} Type.
  This is done by doping with an element with 5 valence electrons, typically phosphorus.
\item \HoleConcentration{} Type.
  This is done by doping with an element with 3 valence electrons, typically boron.
\end{enumerate}

If the concentration of donor atoms in an \ElectronConcentration{}-type doped \nameref{def:Semiconductor} is \DonorConcentration{}, where \DonorConcentration{} is much greater than \HoleElectronConcentration{} ($\DonorConcentration \gg \HoleConcentration$), then the concentration of \textbf{free \nameref{def:Electron}s} is:
\begin{equation}\label{eq:n-Type_Donor_Concentration}
  \ElectronConcentration_{\NType} \simeq \DonorConcentration
\end{equation}

By substituting \Cref{eq:n-Type_Donor_Concentration} into \Cref{eq:Hole_Electron_Concentration}, we can find the \nameref{def:Hole} concentration for an \ElectronConcentration{}-type \nameref{def:Semiconductor}.
\begin{equation}\label{eq:n-Type_Acceptor_Concentration}
  \HoleConcentration_{\NType} \simeq \frac{\HoleElectronConcentration^{2}}{\DonorConcentration}
\end{equation}

In an \NType-type doped \nameref{def:Semiconductor}, the free \nameref{def:Electron}s have a significantly larger concentration, and are said to be the \textbf{majority charge carriers}.
The \nameref{def:Hole}s are the \textbf{minority charge carriers}.

\begin{blackbox}
  The complete opposite of this holds true for \PType{}-type doped \nameref{def:Semiconductor}s.
  Meaning,
  \begin{equation*}
    \HoleConcentration_{\PType} \simeq \AcceptorConcentration
  \end{equation*}
  \begin{equation*}
    \ElectronConcentration_{\PType} \simeq \frac{\HoleElectronConcentration^{2}}{\AcceptorConcentration}
  \end{equation*}

  Similarly oppositely, in an \PType-type doped \nameref{def:Semiconductor}, the free \nameref{def:Hole}s have a significantly larger concentration, and are said to be the \textbf{majority charge carriers}.
The free \nameref{def:Electron}s are the \textbf{minority charge carriers}.
\end{blackbox}

\subsection{Current Flow in Semiconductors}\label{subsec:Semiconductors_Current_Flow}
\subsubsection{Drift Current}\label{subsubsec:Drift_Current}
\begin{definition}[Drift Current]\label{def:Drift_Current}
  \emph{Drift current} arises when an electric field \EField{} is applied to a \nameref{def:Semiconductor}.
  The \nameref{def:Hole}s are accelerated \textbf{in} the direction of \EField{}, and the free \nameref{def:Electron}s are accelerated in the \textbf{opposite} direction of \EField{}.
\end{definition}

In the presence of an electric field, because the drift current is made of of \nameref{def:Electron}s and \nameref{def:Hole}s moving due to a force field, they have a velocity.
This velocity is the \nameref{def:Drift_Velocity}.

\begin{definition}[Drift Velocity]\label{def:Drift_Velocity}
  \emph{Drift velocity} is the velocity that \nameref{def:Hole}s or \nameref{def:Electron}s gain when an electric field (voltage) is applied to the \nameref{def:Semiconductor}.
  There are two separate equations for drift velocity, one for \nameref{def:Hole}s and one for \nameref{def:Electron}s, shown in \Cref{eq:Drift_Velocity-Hole,eq:Drift_Velocity-Electron}, respectively.
\end{definition}

\begin{equation}\label{eq:Drift_Velocity-Hole}
  \DriftVelocity_{\HoleConcentration-\text{drift}} = \HoleMobility \EField
\end{equation}

\begin{equation}\label{eq:Drift_Velocity-Electron}
  \DriftVelocity_{\ElectronConcentration-\text{drift}} = -\ElectronMobility \EField
\end{equation}

\Cref{eq:Drift_Velocity-Electron} is negative because \nameref{def:Electron}s move in the opposite direction of the electric field \EField{}.

In \Cref{eq:Drift_Velocity-Hole,eq:Drift_Velocity-Electron}, the constants \HoleMobility{} and \ElectronMobility{} are used.
\HoleMobility{} is the \nameref{def:Hole_Mobility}.
\ElectronMobility{} is the \nameref{def:Electron_Mobility}.

\begin{definition}[Hole Mobility]\label{def:Hole_Mobility}
  \emph{Hole mobility}, \HoleMobility{} is a value representing how ``easy'' it is for \nameref{def:Hole}s to move through the \nameref{def:Semiconductor}'s crystal structure in response to an applied electric field, \EField{}.

  The hole mobility for \nameref{def:Intrinsic} silicon is a known constant, and is shown in \Cref{eq:Hole_Mobility-Intrinsic_Silicon}.
  \begin{equation}\label{eq:Hole_Mobility-Intrinsic_Silicon}
    \HoleMobility = \SI{480}{\centi\meter\squared\per\volt\per\second}
  \end{equation}
\end{definition}

\begin{definition}[Electron Mobility]\label{def:Electron_Mobility}
  \emph{Electron mobility}, \ElectronMobility{} is a value representing how ``easy'' it is for \nameref{def:Electron}s to move through the \nameref{def:Semiconductor}'s crystal structure in response to an applied electric field, \EField{}.

  The electron mobility for \nameref{def:Intrinsic} silicon is a known constant, and is shown in \Cref{eq:Electron_Mobility-Intrinsic_Silicon}.
  \begin{equation}\label{eq:Electron_Mobility-Intrinsic_Silicon}
    \ElectronMobility = \SI{1350}{\centi\meter\squared\per\volt\per\second}
  \end{equation}

  \begin{remark}
    \nameref{def:Electron}s move through a semiconductor's crystal lattice much more easily than \nameref{def:Hole}s do.
    This can be seen by $\ElectronMobility \approx 2.5 \HoleMobility$.
  \end{remark}
\end{definition}

We are interested in the current flowing through an object due to an applied electric field, so we use:
\begin{equation}\label{eq:Drift_Current-Hole}
  \DCCurrent{\HoleConcentration} = \Area \eCharge \HoleConcentration \DriftVelocity_{\HoleConcentration-\text{drift}}
\end{equation}

If we substitute $\DriftVelocity_{\HoleConcentration-\text{drift}}$ with our knowledge from \Cref{eq:Drift_Velocity-Hole}, then we have the equation below.

\begin{equation*}
  \DCCurrent{\HoleConcentration} = \Area \eCharge \HoleConcentration \HoleMobility \EField
\end{equation*}

If we divide this equation by the cross-sectional area, $\Area$, then we have \Cref{eq:Drift_Current_Density-Hole}.

\begin{equation}\label{eq:Drift_Current_Density-Hole}
  \CurrentDensity{\HoleConcentration} = \frac{\DCCurrent{\HoleConcentration}}{\Area} = \eCharge \HoleConcentration \HoleMobility \EField
\end{equation}

Similarly, we can find the drift current equation for $\DCCurrent{\ElectronConcentration}$ and the current density equation.

\begin{equation}\label{eq:Drift_Current-Electron}
  \DCCurrent{\ElectronConcentration} = - \Area \eCharge \ElectronConcentration \DriftVelocity_{\HoleConcentration-\text{drift}}
\end{equation}

\begin{equation}\label{eq:Drift_Current_Density-Electron}
  \CurrentDensity{\ElectronConcentration} = \eCharge \ElectronConcentration \ElectronMobility \EField
\end{equation}

Then the total drift current is just the addition of the two separate drift currents.

\begin{equation}\label{eq:Total_Drift_Current_Density}
  \CurrentDensity{} = \CurrentDensity{\HoleConcentration} + \CurrentDensity{\ElectronConcentration}
\end{equation}

By factoring out the constant terms, we end up with:
\begin{align*}
  \CurrentDensity{} &= \CurrentDensity{\HoleConcentration} + \CurrentDensity{\ElectronConcentration} \\
                    &= \eCharge \HoleConcentration \HoleMobility \EField + \eCharge \ElectronConcentration \ElectronMobility \EField \\
                    &= \eCharge (\HoleConcentration \HoleMobility + \ElectronConcentration \ElectronMobility) \EField \\
  \CurrentDensity{} &= \Conductivity \EField
\end{align*}

This leads to two important equations, \Cref{eq:Total_Current_Density-Conductivity,eq:Total_Current_Density-Resistivity}.

\begin{subequations}
  \begin{equation}\label{eq:Total_Current_Density-Conductivity}
    \CurrentDensity{} = \Conductivity \EField
  \end{equation}
  \begin{equation}\label{eq:Total_Current_Density-Resistivity}
    \CurrentDensity{} = \frac{\EField}{\Resistivity}
  \end{equation}
\end{subequations}

\begin{definition}[Conductivity]\label{def:Conductivity}
  \emph{Conductivity}, typically represented with $\Conductivity$, is how good a conductor an object is.
  \begin{equation}\label{eq:Conductivity}
    \Conductivity = \eCharge (\HoleConcentration \HoleMobility + \ElectronConcentration \ElectronMobility)
  \end{equation}
\end{definition}

\begin{definition}[Resistivity]\label{def:Resistivity}
  \emph{Resistivity}, typically represented with $\Resistivity$, is how good an object is at preventing the flow of current.
  \begin{equation}\label{eq:Conductivity}
    \Resistivity \equiv \frac{1}{\Conductivity} = \frac{1}{\eCharge (\HoleConcentration \HoleMobility + \ElectronConcentration \ElectronMobility)} \: \si{\ohm\centi\meter}
  \end{equation}
\end{definition}

\subsubsection{Diffusion Current}\label{subsubsec:Diffusion_Current}
\begin{definition}[Diffusion Current]\label{def:Diffusion_Current}
  \emph{Diffusion current} arises due to concentration differences in \nameref{def:Electron}s and \nameref{def:Hole}s in a semiconductor.
  It travels from the \HoleConcentration{} side to the \ElectronConcentration{} side.
\end{definition}

The \nameref{def:Diffusion_Current} density is proportional to the slope of the concentration gradient at any given point in the \nameref{def:Semiconductor}.
For example, if a \nameref{def:Semiconductor} has \textbf{holes} added to it such that there is a larger amount of holes at one side of a block than the other, we end up with \Cref{eq:Diffusion_Current_Density-Hole}.

\begin{equation}\label{eq:Diffusion_Current_Density-Hole}
  \CurrentDensity{\HoleConcentration} = - \eCharge \HoleDiffusivity \frac{d \HoleConcentration(x)}{dx}
\end{equation}

Similarly, for a free \nameref{def:Electron} gradient, we have \Cref{eq:Diffusion_Current_Density-Electron}.

\begin{equation}\label{eq:Diffusion_Current_Density-Electron}
  \CurrentDensity{\ElectronConcentration} = \eCharge \ElectronDiffusivity \frac{d \ElectronConcentration(x)}{dx}
\end{equation}

\begin{definition}[Hole Diffusivity]\label{def:Hole_Diffusivity}
  \emph{Hole diffusivity} or the \emph{hole diffusion constant} is a constant representing how easy it is for a \nameref{def:Hole} to diffuse through the substrate's crystal lattice.
  It is given the symbol $\HoleDiffusivity$.
\end{definition}

\begin{definition}[Electron Diffusivity]\label{def:Electron_Diffusivity}
  \emph{Electron diffusivity} or the \emph{electron diffusion constant} is a constant representing how easy it is for a free \nameref{def:Electron} to diffuse through the substrate's crystal lattice.
  It is given the symbol $\ElectronDiffusivity$.
\end{definition}

There exists a relationship between the diffusivity constants (\Cref{def:Hole_Diffusivity,def:Electron_Diffusivity}) and the mobility constants (\Cref{def:Hole_Mobility,def:Electron_Mobility}), as seen in \Cref{eq:Diffusivity_Mobility_Constants_Relation}.

\begin{equation}\label{eq:Diffusivity_Mobility_Constants_Relation}
  \frac{\ElectronDiffusivity}{\ElectronMobility} = \frac{\HoleDiffusivity}{\HoleMobility} = \ThermalVoltage
\end{equation}

\subsection{The \PNJunction{} Junction}\label{subsec:The_pn_Junction}

\subsection{The \PNJunction{} Junction with Applied Voltage}\label{subsec:The_pn_Junction-Voltage_Applied}

\subsection{Depletion Layer}\label{subsec:Depletion_Layer}
\begin{definition}[Depletion Layer]\label{def:Depletion_Layer}
  The \emph{depletion layer} is the location in the \PNJunction{} where the two differently-doped sides meet.
  Here, there is a barrier of the opposing carrier on each side.
  This is visualized in \Cref{fig:Depletion_Layer}.
\end{definition}

\begin{figure}[h!tbp]
  \centering
  \includegraphics[scale=0.5]{./Depletion_Layer.png}
  \caption{Depletion Layer (\cite[p.~150]{sedraTextbook7})}
  \label{fig:Depletion_Layer}
\end{figure}

We can find the width of the \nameref{def:Depletion_Layer} using \Cref{eq:Depletion_Layer_Width}.
\begin{equation}\label{eq:Depletion_Layer_Width}
  \DepletionDistance = \sqrt{\frac{2 \SiElectricPermittivity}{\eCharge} \left( \frac{1}{\AcceptorConcentration} + \frac{1}{\DonorConcentration} \right) \JunctionBuiltInVoltage}
\end{equation}

The depletion layer ``bleeds'' into each side of the \PNJunction{}.
We can find the distance the depletion layer falls into each side with \Cref{eq:Depletion_Layer_Directions-Electron,eq:Depletion_Layer_Directions-Hole}.

\begin{subequations}\label{eq:Depletion_Layer_Directions}
  \begin{equation}\label{eq:Depletion_Layer_Directions-Electron}
    x_{\ElectronConcentration} = \DepletionDistance \left( \frac{\AcceptorConcentration}{\AcceptorConcentration + \DonorConcentration} \right)
  \end{equation}
  \begin{equation}\label{eq:Depletion_Layer_Directions-Hole}
    x_{\HoleConcentration} = \DepletionDistance \left( \frac{\DonorConcentration}{\AcceptorConcentration + \DonorConcentration} \right)
  \end{equation}
\end{subequations}

Lastly, the sum of the ``bleed'' in both directions is equal to the width of the entire \nameref{def:Depletion_Layer}.
\begin{equation}\label{eq:Depletion_Layer-Directions_Sum}
  \DepletionDistance = x_{\ElectronConcentration} + x_{\HoleConcentration}
\end{equation}

%%% Local Variables:
%%% mode: latex
%%% TeX-master: "../ECE_311-Engineering_Electronics-Reference_Sheet"
%%% End:
