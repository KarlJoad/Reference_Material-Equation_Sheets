\section{Transistor Amplifiers}\label{sec:Transistor_Amps}
Now, we are going to apply \emph{both} AC and DC signals to a \nameref{def:Transistor}, and watch the outputs.
There is some important terminology to be aware of here:
\begin{description}
\item[DC] Uses all uppercase letters, such as $\DCVoltage{\Base}$ or $\DCCurrent{\Drain}$.
\item[AC] Uses all lowercase letters, such as $\ACVoltage{\Gate\Source}$ or $\ACCurrent{\Drain}$.
\item[Mixed] Uses lowercase letters, with capital subscripts, such as $\SignalVoltage{GS}$.
  These are typically defined as $\SignalVoltage{GS} = \ACVoltage{GS} + \DCVoltage{GS}$.
\end{description}

To solve a transistor amplifier problem, there are a few steps:
\begin{enumerate}[noitemsep]
\item Find the DC bias point of the \nameref{def:Transistor}.
  This involves performing DC analysis of the transistor, like was done in \Cref{sec:MOSFETs} and \Cref{sec:BJTs}.
  Zero \emph{all} AC sources and treat all capacitors as open circuits.
  \begin{enumerate}[noitemsep]
  \item When solving, assume the \nameref{def:Transistor} is in the useful region for amplification.
  \end{enumerate}
\item Find the transconductance gain, $\TransconductanceGain$ of the circuit (\Cref{eq:MOSFET-Transconductance_Gain,eq:BJT-Transconductance_Gain}).
\item Find the AC operation of the \nameref{def:Transistor}.
  This involves performing AC analysis of the transistor.
  Zero \emph{all} DC sources, and treat all capacitors are short circuits.
\item Make use of a small-signal equivalent circuit.
\end{enumerate}

\subsection{Biasing}\label{subsec:Biasing}
Biasing of a \nameref{def:Transistor} is required so that it is operating in its active region (\nameref{subsubsec:MOSFET_Saturation_Region} for \nameref{def:MOSFET}s, and \nameref{subsubsec:BJT_Active_Region} for \nameref{def:BJT}s).

If we view the $\ACVoltage{\Gate\Source}$-$\ACVoltage{\Drain\Source}$ voltage transfer characteristics of a \nameref{def:MOSFET} (\Cref{fig:MOSFET-Bias_Graph}) or the $\ACVoltage{\Base\Emitter}$-$\ACVoltage{\Collector\Emitter}$ voltage transfer characteristic of a \nameref{def:BJT} (\Cref{fig:BJT-Bias_Graph}), we can see why this is necessary.

\subsection{MOSFET Amplifiers}\label{subsec:MOSFET_Amps}
\begin{subequations}\label{eq:MOSFET-Transconductance_Gain}
  \begin{equation}\label{eq:MOSFET-Transconductance_Gain-Overdrive_Voltage}
    \TransconductanceGain = \ElectronMobility \OxideCapacitivity \frac{\Width}{\Length} \OverdriveVoltage
  \end{equation}
  \begin{equation}\label{eq:MOSFET-Transconductance_Gain-Drain_Current}
    \TransconductanceGain = \sqrt{2 \ElectronMobility \OxideCapacitivity \frac{\Width}{\Length} \DCCurrent{\Drain}}
  \end{equation}
  \begin{equation}\label{eq:MOSFET-Transconductance_Gain-Drain_Current_Overdrive_Voltage}
    \TransconductanceGain = \frac{2 \DCCurrent{\Drain}}{\OverdriveVoltage}
  \end{equation}
\end{subequations}

The small-signal open circuit voltage gain of a \nameref{def:MOSFET} is given in \Cref{eq:MOSFET-Small_Signal_Voltage_Gain}, and the maximum value is found in \Cref{eq:MOSFET-Small_Signal_Voltage_Gain-Max}.
\begin{equation}\label{eq:MOSFET-Small_Signal_Voltage_Gain}
  \Amp{\Voltage} = -\frac{\DCVoltage{DD} - \DCVoltage{DS}}{\frac{\OverdriveVoltage}{2}}
\end{equation}

\begin{equation}\label{eq:MOSFET-Small_Signal_Voltage_Gain-Max}
  \Abs{\Amp{\Voltage,Max}} = \frac{\DCVoltage{DD} - \left. \OverdriveVoltage \right\rvert_{B}}{\frac{\left. \OverdriveVoltage \right\rvert_{B}}{2}}
\end{equation}

In addition to \Cref{eq:MOSFET-Small_Signal_Voltage_Gain}, the small-signal amplification of a \nameref{def:MOSFET} is also dependent on several factors of the circuit, as shown in \Cref{eq:MOSFET-Small_Signal_Voltage_Gain-Overdrive_Voltage}.
\begin{equation}\label{eq:MOSFET-Small_Signal_Voltage_Gain-Overdrive_Voltage}
  \Amp{\Voltage} = -k_{\NType} R_{\Drain} \OverdriveVoltage
\end{equation}

The resistance at the output \emph{due to the inefficiencies of the transistor} is defined by \Cref{eq:MOSFET-Transistor_Resistance}.
\begin{equation}\label{eq:MOSFET-Transistor_Resistance}
  r_{0} = \frac{\DCVoltage{A}}{\DCCurrent{\Drain}}
\end{equation}
This resistance goes in \emph{parallel} with the voltage-dependent current source that the \nameref{def:Transistor} can be modeled as.

\subsection{BJT Amplifiers}\label{subsec:BJT_Amps}
\begin{equation}\label{eq:BJT-Transconductance_Gain}
  \TransconductanceGain = \frac{\DCCurrent{\Collector}}{\ThermalVoltage}
\end{equation}

The small-signal voltage gain of a \nameref{def:BJT} is shown in \Cref{eq:BJT-Small_Signal_Voltage_Gain}.
\begin{equation}\label{eq:BJT-Small_Signal_Voltage_Gain}
  \Amp{V} = - \frac{\DCCurrent{\Collector} R_{\Collector}}{\ThermalVoltage}
\end{equation}

\subsection{Amplification}\label{subsec:Amplification}
The \emph{amplification} of a circuit is the ratio of the measured output against its input.
\Cref{eq:Amplification} illustrates this.

\begin{equation}\label{eq:Amplification}
  \Amp{} = \frac{\ACVoltage{\Out}}{\ACVoltage{\In}}
\end{equation}

\subsection{Gain}\label{subsec:Gain}
The \emph{gain} of a circuit is determined by the ratio of the output signal when a load is attached to the input signal.
This is distinctly different than \nameref{subsec:Amplification} because a load resistance has been added, which alters the output signal.

\Cref{eq:Gain} gives a general equation for the gain of a circuit.
\textbf{Remember that $\ACVoltage{\Out}$ MUST include the load resistance!}
\begin{equation}\label{eq:Gain}
  \Gain{} = \frac{\ACVoltage{\Out}}{\ACVoltage{\In}}
\end{equation}

The gain of a circuit can be calculated by multiplying the amplification of various subcomponents in the circuit with each other.
Thus, \Cref{eq:Gain-Broken} can be put together.
\begin{equation}\label{eq:Gain-Broken}
  \Gain{} = \Amp{1 \to 2} \Amp{2 \to 3} \cdots \Amp{n-1 \to n}
\end{equation}

%%% Local Variables:
%%% mode: latex
%%% TeX-master: "../ECE_311-Engineering_Electronics-Reference_Sheet"
%%% End:
