\section{Transistor Amplifiers}\label{sec:Transistor_Amps}
Now, we are going to apply \emph{both} AC and DC signals to a \nameref{def:Transistor}, and watch the outputs.
There is some important terminology to be aware of here:
\begin{description}
\item[DC] Uses all uppercase letters, such as $\DCVoltage{\Base}$ or $\DCCurrent{\Drain}$.
\item[AC] Uses all lowercase letters, such as $\ACVoltage{\Gate\Source}$ or $\ACCurrent{\Drain}$.
\item[Mixed] Uses lowercase letters, with capital subscripts, such as $\SignalVoltage{GS}$.
  These are typically defined as $\SignalVoltage{GS} = \ACVoltage{GS} + \DCVoltage{GS}$.
\end{description}

To solve a transistor amplifier problem, there are a few steps:
\begin{enumerate}[noitemsep]
\item Find the DC bias point of the \nameref{def:Transistor}.
  This involves performing DC analysis of the transistor, like was done in \Cref{sec:MOSFETs} and \Cref{sec:BJTs}.
  Zero \emph{all} AC sources and treat all capacitors as open circuits.
  \begin{enumerate}[noitemsep]
  \item When solving, assume the \nameref{def:Transistor} is in the useful region for amplification.
  \end{enumerate}
\item Find the transconductance gain, $\TransconductanceGain$ of the circuit (\Cref{eq:MOSFET-Transconductance_Gain,eq:BJT-Transconductance_Gain}).
\item Find the AC operation of the \nameref{def:Transistor}.
  This involves performing AC analysis of the transistor.
  Zero \emph{all} DC sources, and treat all capacitors are short circuits.
\item Make use of a small-signal equivalent circuit.
\end{enumerate}

\subsection{MOSFET Amplifiers}\label{subsec:MOSFET_Amps}

\subsection{BJT Amplifiers}\label{subsec:BJT_Amps}

\subsection{Amplification}\label{subsec:Amplification}

\subsection{Gain}\label{subsec:Gain}

%%% Local Variables:
%%% mode: latex
%%% TeX-master: "../ECE_311-Engineering_Electronics-Reference_Sheet"
%%% End:
