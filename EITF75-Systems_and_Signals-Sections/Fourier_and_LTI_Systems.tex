\section{The Fourier Transform and LTI Systems}\label{sec:Fourier_Transform-LTI_Systems}
\subsection{Magnitude Response}\label{subsec:Fourier_Magnitude_Response}
\begin{definition}[Magnitude Reponse]\label{def:Fourier_Magnitude_Response}
  The \emph{magnitude response} of a Fourier Transform is commonly denoted as $\lvert H(\omega) \rvert$ or $\lvert H(f) \rvert$.
  However, in this text, it is denoted as $\lVert H(\omega) \rVert$ or $\lVert H(f) \rVert$.

  The equation that defines a \nameref{def:FourierTransform}'s \nameref{def:Fourier_Magnitude_Response} is
  \begin{equation}\label{eq:Fourier_Magnitude_Response}
    \begin{aligned}
      \lVert H(\omega) \rVert &= \sqrt{{\Bigl( \Re \lbrace H(\omega) \rbrace \Bigr)}^{2} + {\Bigl( \Im \lbrace H(\omega) \rbrace \Bigr)}^{2}} \\
      \lVert H(f) \rVert &= \sqrt{{\Bigl( \Re \lbrace H(f) \rbrace \Bigr)}^{2} + {\Bigl( \Im \lbrace H(f) \rbrace \Bigr)}^{2}} \\
    \end{aligned}
  \end{equation}

  \begin{remark}
    It is important to note that the numerator will \textbf{\emph{NOT}} have any imaginary terms ($i$ or $j$) in it!
  \end{remark}
\end{definition}

\begin{remark*}
  It is important to note that in this reference guide, the magnitude is denoted with double bars.
  For example, if a complex function $a(n)$ exists, then I will denote its magnitude as $\lVert a(n) \rVert$.
  This helps distinguish between magnitude of a function and its absolute value.
  These sometimes have different values, so it is useful to differentiate between the two.
\end{remark*}

\subsubsection{Methods of Finding Magnitude Response}\label{subsubsec:Fourier_Magnitude_Response_Methods}
There are 2 easy ways to find the \nameref{def:Fourier_Magnitude_Response}.
\begin{enumerate}[noitemsep]
\item Solve $H(\omega)$ or $H(f)$ and make sinusoids
\item Multiply $H(\omega)$ or $H(f)$ by $e^{-j \omega}$ or $e^{-j 2\pi f}$ and cancel terms out
\end{enumerate}

\begin{example}[]{Find Magnitude Response-Method 2}
  Compute the amplitude function $\lVert H(f) \rVert$?

  \begin{equation*}
    H(z) = \frac{1}{5} \left( 1 + z^{-1} + z^{-2} + z^{-3} + z^{-4} \right)
  \end{equation*}

  \tcblower{}

  Substitute $z=e^{j 2\pi f}$.
  \begin{equation*}
    H(f) = \frac{1}{5} \left( 1 + e^{-1 j 2\pi f} + e^{-2 j 2\pi f} + e^{-3 j 2\pi f} + e^{-4 j 2\pi f} \right)
  \end{equation*}

  You could try to find a common exponential term(s) to factor out, but there this second method is easier for series of exponentials.
  \begin{align*}
    H(f) &= \frac{1}{5} \left( 1 + e^{-1 j 2\pi f} + e^{-2 j 2\pi f} + e^{-3 j 2\pi f} + e^{-4 j 2\pi f} \right) \\
    e^{-j 2\pi f} H(f) &= \frac{1}{5} \left( e^{-1 j 2\pi f} + e^{-2 j 2\pi f} + e^{-3 j 2\pi f} + e^{-4 j 2\pi f} + e^{-5 j 2\pi f} \right) \\
    H(f) - e^{-j 2\pi f} H(f) &= \Bigl( \frac{1}{5} \left( 1 + e^{-1 j 2\pi f} + e^{-2 j 2\pi f} + e^{-3 j 2\pi f} + e^{-4 j 2\pi f} \right) \Bigr) \\
    &- \Bigl( \frac{1}{5} \left( e^{-1 j 2\pi f} + e^{-2 j 2\pi f} + e^{-3 j 2\pi f} + e^{-4 j 2\pi f} + e^{-5 j 2\pi f} \right) \Bigr) \\
    H(f) \left( 1-e^{-j 2\pi f} \right) &= \frac{1}{5} \left( 1-e^{-5 j 2\pi f} \right) \\
    H(f) &= \frac{1}{5} \frac{1-e^{-5 j 2\pi f}}{1-e^{-j 2\pi f}} \\
  \end{align*}
  Now we can factor terms out to make complex exponentials that can form sinusoids.
  \begin{align*}
    H(f) &= \frac{1}{5} \frac{e^{-\frac{5}{2} j 2\pi f} \left( e^{\frac{5}{2} j 2\pi f} - e^{-\frac{5}{2} j 2\pi f} \right)}{e^{-\frac{1}{2} j 2\pi f} \left( e^{\frac{1}{2} j 2\pi f} - e^{-\frac{1}{2} j 2\pi f} \right)} \\
         &= \frac{1}{5} \frac{e^{-\frac{5}{2} j 2\pi f}}{e^{-\frac{1}{2} j 2\pi f}} \frac{2 \cos \left( \frac{5}{2} \cdot 2\pi f \right)}{2 \cos \left( \frac{1}{2} \cdot 2\pi f \right)} \\
         &= \frac{1}{5} e^{-2 j 2\pi f} \frac{\cos(5\pi f)}{\cos(\pi f)} \\
    \lVert H(f) \rVert &= \Biggl\lVert \frac{1}{5} e^{-2 j 2\pi f} \frac{\cos(5\pi f)}{\cos(\pi f)} \Biggr\rVert \\
         &= \frac{1}{5} (1) \Biggl\lVert \frac{\cos(5\pi f)}{\cos(\pi f)} \Biggr\rVert \\
         &= \frac{1}{5} \Biggl\lVert \frac{\cos(5\pi f)}{\cos(\pi f)} \Biggr\rVert \\
  \end{align*}

  Our solution is
  \begin{equation*}
    \lVert H(f) \rVert = \frac{1}{5} \Biggl\lVert \frac{\cos(5\pi f)}{\cos(\pi f)} \Biggr\rVert
  \end{equation*}
\end{example}

\subsection{Phase Response}\label{subsec:Fourier_Phase_Response}
\begin{definition}[Phase Response]\label{def:Fourier_Phase_Response}
  The \emph{phase response} of a Fourier Transform is commonly denoted as $\measuredangle H(\omega)$ or $\measuredangle H(f)$.
  This is a function that defines a \nameref{def:FourierTransform}'s \nameref{def:Fourier_Phase_Response} is defined in the equation below.
  \begin{equation}\label{eq:Fourier_Phase_Response}
    \begin{aligned}
      \measuredangle H(\omega) = \Theta (\omega) &= \tan^{-1} \Biggl( \frac{\Im \lbrace H(\omega) \rbrace}{\Re \lbrace H(\omega) \rbrace} \Biggr) \\
      \measuredangle H(f) = \Theta (f) &= \tan^{-1} \Biggl( \frac{\Im \lbrace H(f) \rbrace}{\Re \lbrace H(f) \rbrace} \Biggr) \\
    \end{aligned}
  \end{equation}

  \begin{remark}
    It is important to note that the numerator will \textbf{\emph{NOT}} have any imaginary terms ($i$ or $j$) in it!
  \end{remark}

  \begin{remark}
    Note that real positive values have argument $=0$
    \begin{equation}\label{eq:Fourier_Phase_Response-Real_Positive_Values}
      \Theta(\omega) = 0,\:\:\: \Theta(f) = 0
    \end{equation}
    Real negative values have argument $=\pm \pi$
    \begin{equation}\label{eq:Fourier_Phase_Response-Real_Negative_Values}
      \Theta(\omega) = \pm \pi,\:\:\: \Theta(f) = \pm \pi
    \end{equation}
  \end{remark}

  \begin{remark}[Complex Exponential to Unit Circle]\label{rmk:Complex_Exp-Unit_Circle_Relation}
    \textbf{\emph{REMEMBER:}}
    \begin{equation}\label{eq:Complex_Exp-Unit_Circle_Relation}
      \begin{aligned}
        e^{\pm j\omega} &= \cos (\omega) \pm j \sin (\omega) \\
        e^{\pm j 2 \pi f} &= \cos (2 \pi f) \pm j \sin (2 \pi f) \\
      \end{aligned}
    \end{equation}
    This is also defined in \Cref{eq:Euler Complex} on \Cpageref{eq:Euler Complex}.
  \end{remark}
\end{definition}

\subsection{Frequency Response}\label{subsec:Fourier_Frequency_Response}
The \nameref{subsec:Fourier_Frequency_Response} of a function is defined in 2 parts.
\begin{enumerate}[noitemsep]
\item \nameref{subsec:Fourier_Magnitude_Response}
\item \nameref{subsec:Fourier_Phase_Response}
\end{enumerate}
\begin{equation}\label{eq:Fourier_Frequency_Response}
  \begin{aligned}
    H(\omega) &= \lVert H(\omega) \rVert \, \Theta(\omega) \\
    H(f) &= \lVert H(f) \rVert \, \Theta(f) \\
  \end{aligned}
\end{equation}

\begin{remark*}
  This is similar to the \nameref{subsec:Rectangular to Polar} conversion shown on \Cpageref{subsec:Rectangular to Polar}.
\end{remark*}

%%% Local Variables:
%%% mode: latex
%%% TeX-master: "../EITF75-Systems_and_Signals-Reference_Sheet"
%%% End:
