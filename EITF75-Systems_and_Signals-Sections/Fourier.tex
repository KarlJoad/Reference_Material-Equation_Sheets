\section{The Fourier Transform and Fourier Series}\label{sec:FourierTransformSeries}
When a signal is decomposed with either the \nameref{def:FourierTransform} or the \nameref{def:FourierSeries}, you receive either sinusoids or complex-valued exponentials.
This decomposition is said to be represented in the \emph{frequency domain}.

\begin{definition}[Fourier Transform]\label{def:FourierTransform}
  When decomposing the class of signals with finite energy, you perform a \emph{Fourier transform}.
  This is generally shown as the function
  \begin{equation*}
    c_{k} = \FourierTransform \lbrace x(t) \rbrace
  \end{equation*}

  There are 2 possible equations for the \nameref{def:FourierTransform}, depending of the function is continuous-time or discrete-time.
  \begin{enumerate}[noitemsep]
  \item Continuous-Time: \Cref{eq:FourierTransform-Continuous-Frequency}
  \item Discrete-Time: \Cref{eq:FourierTransform-Discrete-Frequency}
  \end{enumerate}
  
  The \nameref{def:FourierTransform} is defined as
  \begin{equation}\label{eq:FourierTransform-Continuous-Frequency}
    X(F) = \int\limits_{-\infty}^{\infty} x(t) e^{-j 2 \pi F t} dt
  \end{equation}

  \begin{equation}\label{eq:FourierTransform-Discrete-Frequency}
    X(f) = \sum\limits_{n=-\infty}^{\infty} x(n) e^{-j 2 \pi f n}
  \end{equation}

  \begin{remark}
    Sometimes $X(F)$ and $X(f)$ will be denoted with $\Omega$ and $\omega$ ($X(\Omega)$ and $X(\omega)$) respectively.
    In both cases, $\Omega$ and $\omega$ mean something similar.
    \begin{align*}
      \Omega &= 2 \pi F \\
      \omega &= 2 \pi f
    \end{align*}

    This means that we can rewrite \Crefrange{eq:FourierTransform-Continuous-Frequency}{eq:FourierTransform-Discrete-Frequency} as
    \begin{equation}\label{eq:FourierTransform-Continuous-Omega}
      X(\Omega) = \int\limits_{-\infty}^{\infty} x(t) e^{-j \Omega t} dt
    \end{equation}
    \begin{equation}\label{eq:FourierTransform-Discrete-Omega}
      X(\omega) = \sum\limits_{n=-\infty}^{\infty} x(n) e^{-j \omega n}
    \end{equation}
  \end{remark}

  \begin{remark}
    Generally, when people say the \nameref{def:FourierTransform}, they are referring to the transform on \nameref{def:Continuous-Time Signals}.
    There is a distinction that occurs with the \emph{DTFT} or \emph{\nameref{def:Discrete-Time_Fourier_Transform}}.

    This document explains them side-by-side, but will primarily focus on the \nameref{def:Discrete-Time_Fourier_Transform}.
  \end{remark}
\end{definition}

\begin{definition}[Fourier Series]\label{def:FourierSeries}
  When decomposing the class of periodic signals, you are returned a \emph{Fourier series}.
  This is generally shown as the function
  \begin{equation*}
    X(F) = \FourierTransform \lbrace x(t) \rbrace
  \end{equation*}
\end{definition}

\begin{definition}[Discrete-Time Fourier Transform]\label{def:Discrete-Time_Fourier_Transform}
  The \emph{Discrete-Time \nameref{def:FourierTransform}, DTFT} is a special case of the \nameref{def:FourierTransform} that occurs when the input function $x(n)$ is a case of \nameref{def:Discrete-Time Signals}.

  The transformation (analysis) equations are:
  \begin{subequations}
    \begin{equation}\label{eq:Discrete-Time_Fourier_Transform-Discrete-Frequency}
      X(f) = \sum\limits_{n=-\infty}^{\infty} x(n) e^{-j 2 \pi f n}
    \end{equation}
    \begin{equation*}
      \omega = 2 \pi f
    \end{equation*}
    \begin{equation}\label{eq:FourierTransform-Discrete-Omega}
      X(\omega) = \sum\limits_{n=-\infty}^{\infty} x(n) e^{-j \omega n}
    \end{equation}
  \end{subequations}

  The reverse (synthesis) equations are:
  \begin{subequations}
    \begin{equation}\label{eq:InverseFourierTransform-Frequency}
      x(n) = \int\limits_{-\infty}^{\infty} X(f) e^{j 2\pi fn} df
    \end{equation}
    \begin{equation}\label{eq:InverseFourierTransform-Omega}
      x(n) = \frac{1}{2\pi} \int\limits_{-\infty}^{\infty} X(\omega) e^{j \omega n} d\omega
    \end{equation}
  \end{subequations}
  These equations are expanded more upon in \Cref{subsec:InverseFourierTransform}, \nameref{subsec:InverseFourierTransform}.
\end{definition}

\subsection{\nameref{def:FourierTransform} Relations}\label{subsec:Fourier_Transform_Relations}
Each of these relations is just a side-note, the only relation of real importance is \Cref{eq:ZTransformFourierTransformRelation}.
The \nameref{def:FourierTransform} is just a special case in each of these scenarios.
The \nameref{def:FourierTransform} is evaluated around the unit circle on the real-imaginary plane.

\subsubsection{\nameref{app:Laplace_Transform} \nameref{def:FourierTransform} Relation}\label{subsubsec:Laplace-Transform_Fourier_Transform_Relation}
There is a correlation between the \nameref{app:Laplace_Transform} and the \nameref{def:FourierTransform}.
The \nameref{def:FourierTransform} is a more specific case of the \nameref{app:Laplace_Transform}, when
\begin{equation*}
  e^{-st} = e^{-j 2\pi ft}
\end{equation*}

\subsubsection{\texorpdfstring{\nameref{def:Z-Transform} \nameref{def:Discrete-Time_Fourier_Transform} Relation}{Z-Transform Fourier Transform Relation}}\label{subsubsec:ZTransformFourierTransformRelation}
There is a relationship between the $\ZTransform$-Transform and the \nameref{def:Discrete-Time_Fourier_Transform}.
\begin{equation}\label{eq:ZTransformFourierTransformRelation}
  \begin{aligned}
    z &= e^{2\pi f} \\
    z &= e^{2\pi n} \\
  \end{aligned}
\end{equation}

\subsection{The Inverse Fourier Transform}\label{subsec:InverseFourierTransform}
\begin{definition}[Inverse Fourier Transform]\label{def:InverseFourierTransform}
  Since the \nameref{def:FourierTransform} is a ``lossless'' function (the definition of a transformation), the \emph{inverse fourier transform} is just the opposite setup of~\Crefrange{eq:FourierTransform-Continuous-Frequency}{eq:FourierTransform-Discrete-Frequency}.

  In both cases, a Continuous-Time signal and a Discrete-Time signal, you use the below synthesis equations (\Crefrange{eq:InverseFourierTransform-Frequency}{eq:InverseFourierTransform-Omega}).
  
  \begin{equation}\label{eq:InverseFourierTransform-Frequency}
    \begin{aligned}
      x(t) &= \int\limits_{-\infty}^{\infty} X(F) e^{j 2\pi Ft} dF \\
      x(n) &= \int\limits_{-\infty}^{\infty} X(f) e^{j 2\pi fn} df \\
    \end{aligned}
  \end{equation}

  If you're calculating with $\Omega$ or $\omega$ instead of $F$ or $f$, then use these synthesis equations.
  \begin{equation}\label{eq:InverseFourierTransform-Omega}
    \begin{aligned}
      x(t) &= \frac{1}{2\pi} \int\limits_{-\infty}^{\infty} X(\Omega) e^{j \Omega t} d\Omega \\
      x(n) &= \frac{1}{2\pi} \int\limits_{-\infty}^{\infty} X(\omega) e^{j \omega n} d\omega \\
    \end{aligned}
  \end{equation}
\end{definition}

\subsection{Fourier Transform Properties for Discrete-Time Signals}\label{subsec:FourierTransformProperties-Discrete}
\begin{table}[h!]
  \centering
  \begin{tabular}{lll}
    \toprule
    Property & Time Domain & Frequency Domain \\
             & $x(n)$ & $X(f)$ or $X(\omega)$ \\
    \midrule
    \multirow{3}{*}{Notation} & $x(n)$ & $X(\omega)$ \\ %\multirow{numrows}{cellwidth (* is natural width)}{value}
             & $x_{1}(n)$ & $X_{1}(\omega)$ \\
             & $x_{2}(n)$ & $X_{2}(\omega)$ \\
    \nameref{subsubsec:FourierTransformProperties-Linearity} & $a_{1}x_{1}(n) + a_{2}x_{2}(n)$ & $a_{1}X_{1}(\omega)+a_{2}X_{2}(\omega)$ \\
    \nameref{subsubsec:FourierTransformProperties-TimeShifting} & $x(n-k)$ & $e^{-j\omega k}X(\omega)$ \\
    \nameref{subsubsec:FourierTransformProperties-TimeReversal} & $x(-n)$ & $X(-\omega)$ \\
    \nameref{subsubsec:FourierTransformProperties-Convolution} & $x_{1}(n) * x_{2}(n)$ & $X_{1}(\omega)X_{2}(\omega)$ \\
    \nameref{subsubsec:FourierTransformProperties-Correlation} & $r_{x_{1},x_{2}}(l) = x_{1}(l) * x_{2}(-l)$ & \multicolumn{1}{r}{$S_{x_{1},x_{2}}(\omega) = X_{1}(\omega)X_{2}(\omega)$} \\
             && \multicolumn{1}{r}{$= X_{1}(\omega)X_{2}^{*}(\omega)$} \\
             && \multicolumn{1}{r}{[if $x_{2}(n)$ is real]} \\
    \nameref{subsubsec:FourierTransformProperties-WienerKhintchineTheorem} &$ r_{xx}(l)$ & $S_{xx}(\omega)$ \\
    \nameref{subsubsec:FourierTransformProperties-FrequencyShifting} & $e^{j \omega_{0} n} x(n)$ & $X(\omega - \omega_{0})$ \\
    \nameref{subsubsec:FourierTransformProperties-Modulation} & $x(n) \cos \left( \omega_{0} n \right)$ & $\frac{1}{2} X(\omega + \omega_{0}) + \frac{1}{2} X(\omega - \omega_{0})$ \\
    \nameref{subsubsec:FourierTransformProperties-MultiplicationTimeDomain} & $ x_{1}(n)x_{2}(n)$ & $\frac{1}{2 \pi} \int_{-\pi}^{\pi} X_{1}(\lambda) x_{2}(\omega - \lambda) d\lambda$ \\
    \nameref{subsubsec:FourierTransformProperties-DifferentiationFrequencyDomain} & $n x(n)$ & $j \frac{dX(\omega)}{d\omega}$ \\
    Conjugation & $x^{*}(n)$ & $X^{*}(-\omega)$ \\
    \nameref{subsubsec:FourierTransformProperties-ParsevalsTheorem} & \multicolumn{2}{c}{$\sum_{n=-\infty}^{\infty} x_{1}(n)x_{2}^{*}(n) = \frac{1}{2 \pi} \int_{-\pi}^{\pi}X_{1}(\omega)X_{2}^{*}(\omega) d\omega$} \\
    \bottomrule
  \end{tabular}
  \caption{Properties of the \nameref{def:FourierTransform} for \nameref{def:Discrete-Time Signals}}
  \label{tab:FourierTransformProperties}
\end{table}

\subsubsection{Linearity}\label{subsubsec:FourierTransformProperties-Linearity}
If
\begin{equation*}
  \begin{aligned}
    x_{1}(n) &\FourierTransformRelation X_{1}(f) \\
    x_{2}(n) &\FourierTransformRelation X_{2}(f) \\
  \end{aligned}
\end{equation*}
then
\begin{equation}\label{eq:FourierTransformProperties-Linearity}
  a_{1}x_{1}(n) + a_{2}x_{2}(n) \FourierTransformRelation a_{1}X_{1}(f) + a_{2}X_{2}(f)
\end{equation}

\subsubsection{Time Shifting}\label{subsubsec:FourierTransformProperties-TimeShifting}
\subsubsection{Time Reversal}\label{subsubsec:FourierTransformProperties-TimeReversal}
\subsubsection{Convolution}\label{subsubsec:FourierTransformProperties-Convolution}
\subsubsection{Correlation}\label{subsubsec:FourierTransformProperties-Correlation}
\subsubsection{Wiener-Khintchine Theorem}\label{subsubsec:FourierTransformProperties-WienerKhintchineTheorem}
\subsubsection{Frequency Shifting}\label{subsubsec:FourierTransformProperties-FrequencyShifting}
\subsubsection{Modulation}\label{subsubsec:FourierTransformProperties-Modulation}
\subsubsection{Multiplication in Time Domain}\label{subsubsec:FourierTransformProperties-MultiplicationTimeDomain}
\subsubsection{Differentiation in Frequency Domain}\label{subsubsec:FourierTransformProperties-DifferentiationFrequencyDomain}
\subsubsection{Conjugation}\label{subsubsec:FourierTransformProperties-Conjugation}
\subsubsection{Parseval's Theorem}\label{subsubsec:FourierTransformProperties-ParsevalsTheorem}
If
\begin{equation*}
  \begin{aligned}
    x_{1}(n) &\FourierTransformRelation X_{1}(f) \\
    x_{2}(n) &\FourierTransformRelation X_{2}(f) \\
  \end{aligned}
\end{equation*}
then

\begin{equation}\label{eq:FourierTransformProperties-ParsevalsTheorem-Frequency}
  \sum\limits_{n=-\infty}^{\infty} x_{1}(n) x_{2}^{*}(n) = \int\limits_{-\pi}^{\pi} X_{1}(f) X_{2}^{*}(f) df
\end{equation}
\begin{equation}\label{eq:FourierTransformProperties-ParsevalsTheorem-Omega}
  \sum\limits_{n=-\infty}^{\infty} x_{1}(n) x_{2}^{*}(n) = \frac{1}{2 \pi} \int\limits_{-\pi}^{\pi} X_{1}(\omega) X_{2}^{*}(\omega) d\omega
\end{equation}

Both \Crefrange{eq:FourierTransformProperties-ParsevalsTheorem-Frequency}{eq:FourierTransformProperties-ParsevalsTheorem-Omega} can be expressed in another format.
\begin{equation}\label{eq:FourierTransformProperties-ParsevalsTheorem-Frequency-Absolute}
  \sum\limits_{n=-\infty}^{\infty} \lvert x_{1}(n) \rvert^{2} = \int\limits_{-\pi}^{\pi} \lvert X_{1}(f) \rvert^{2} df
\end{equation}
\begin{equation}\label{eq:FourierTransformProperties-ParsevalsTheorem-Omega-Absolute}
  \sum\limits_{n=-\infty}^{\infty} \lvert x_{1}(n) \rvert^{2} = \frac{1}{2 \pi} \int\limits_{-\pi}^{\pi} \lvert X_{1}(\omega) \rvert^{2} d\omega
\end{equation}
%%% Local Variables:
%%% mode: latex
%%% TeX-master: "../EITF75-Systems_and_Signals-Reference_Sheet"
%%% End:
