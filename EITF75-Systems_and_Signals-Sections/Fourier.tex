\section{The Fourier Transform and Fourier Series} \label{sec:Fourier Transform and Series}
When a signal is decomposed with either the \nameref{def:Fourier Transform} or the \nameref{def:Fourier Series}, you receive either sinusoids or complex-valued exponentials.
This decomposition is said to be represented in the \emph{frequency domain}.
\begin{definition}[Fourier Transform]\label{def:Fourier Transform}
  When decomposing the class of signals with finite energy, you perform a \emph{Fourier transform}.
  This is generally shown as the function
  \begin{equation*}
    c_{k} = \FourierTransform \lbrace x(t) \rbrace
  \end{equation*}
\end{definition}
\begin{definition}[Fourier Series]\label{def:Fourier Series}
  When decomposing the class of periodic signals, you are returned a \emph{Fourier series}.
  This is generally shown as the function
  \begin{equation*}
    X(F) = \FourierTransform \lbrace x(t) \rbrace
  \end{equation*}
\end{definition}
%%% Local Variables:
%%% mode: latex
%%% TeX-master: "../EITF75-Systems_and_Signals-Reference_Sheet"
%%% End:
