\section{The Fourier Transform and Fourier Series}\label{sec:FourierTransformSeries}
When a signal is decomposed with either the \nameref{def:FourierTransform} or the \nameref{def:FourierSeries}, you receive either sinusoids or complex-valued exponentials.
This decomposition is said to be represented in the \emph{frequency domain}.

\begin{definition}[Fourier Transform]\label{def:FourierTransform}
  When decomposing the class of signals with finite energy, you perform a \emph{Fourier transform}.
  This is generally shown as the function
  \begin{equation*}
    c_{k} = \FourierTransform \lbrace x(t) \rbrace
  \end{equation*}

  There are 2 possible equations for the \nameref{def:FourierTransform}, depending of the function is continuous-time or discrete-time.
  \begin{enumerate}[noitemsep]
  \item Continuous-Time: \Cref{eq:FourierTransform-Continuous}
  \item Discrete-Time: \Cref{eq:FourierTransform-Discrete}
  \end{enumerate}
  
  The \nameref{def:FourierTransform} is defined as
  \begin{equation}\label{eq:FourierTransform-Continuous}
    X(F) = \int\limits_{-\infty}^{\infty} x(t) e^{-j 2 \pi F t} dt
  \end{equation}
  \begin{equation}\label{eq:FourierTransform-Discrete}
    X(f) = \sum\limits_{n=-\infty}^{\infty} x(n) e^{-j 2 \pi f n}
  \end{equation}
  \begin{remark}
    Sometimes $X(F)$ and $X(f)$ will be denoted with $\Omega$ and $\omega$ ($X(\Omega)$ and $X(\omega)$) respectively.
    In both cases, $\Omega$ and $\omega$ mean something similar.
    \begin{align*}
      \Omega &= 2 \pi F \\
      \omega &= 2 \pi f
    \end{align*}
  \end{remark}
\end{definition}
\begin{definition}[Fourier Series]\label{def:FourierSeries}
  When decomposing the class of periodic signals, you are returned a \emph{Fourier series}.
  This is generally shown as the function
  \begin{equation*}
    X(F) = \FourierTransform \lbrace x(t) \rbrace
  \end{equation*}
\end{definition}

\subsection{\texorpdfstring{$\ZTransform$-Transform Fourier Transform Relation}{Z-Transform Fourier Transform Relation}}\label{subsec:ZTransformFourierTransformRelation}
There is a relationship between the $\ZTransform$-Transform and the \nameref{def:FourierTransform}.
\begin{equation}\label{eq:ZTransformFourierTransformRelation}
  \begin{aligned}
    z &= e^{2\pi f} \\
    z &= e^{2\pi n} \\
  \end{aligned}
\end{equation}

\subsection{The Inverse Fourier Transform}\label{subsec:InverseFourierTransform}
\begin{definition}[Inverse Fourier Transform]\label{def:InverseFourierTransform}
  Since the \ref{def:FourierTransform} is a ``lossless'' function (the definition of a transformation), the \emph{inverse fourier transform} is just the opposite setup of~\Crefrange{eq:FourierTransform-Continuous}{eq:FourierTransform-Discrete}.

  In both cases, a Continuous-Time signal and a Discrete-Time signal, you use the below synthesis equation.
  
  \begin{equation}\label{eq:InverseFourierTransform-Continuous}
    x(t) = \frac{1}{2\pi} \int\limits_{-\infty}^{\infty} X(F) e^{j 2\pi Ft} dF
  \end{equation}
\end{definition}
%%% Local Variables:
%%% mode: latex
%%% TeX-master: "../EITF75-Systems_and_Signals-Reference_Sheet"
%%% End:
