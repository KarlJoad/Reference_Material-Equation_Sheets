\section{Discrete-Time Systems}\label{sec:Discrete-Time Systems}
As discussed in \Cref{subsec:Discrete-Time Signals}, $x(n)$ is a function of an independent variable that is an integer.
It is important to note that a discrete-time signal is \emph{not defined} at instants between the samples.
Also, if $n$ is not an integer, $x(n)$ is not defined.

Besides graphical representation of a discrete-time system, there are 3 ways to represent a discrete-time signal.
\begin{enumerate}[noitemsep]
\item \nameref{subsubsec:Functional Representation}
\item \nameref{subsubsec:Tabular Representation}
\item \nameref{subsubsec:Sequence Representation}
\end{enumerate}

\subsection{Representing Discrete-Time Systems}\label{subsec:Representing Discrete-Time Systems}
\subsubsection{Functional Representation}\label{subsubsec:Functional Representation}
This representation of a discrete-time system is done as a mathematical function.
\begin{equation}\label{eq:Functional Representation}
  x(n) = \begin{cases}
    1 ,& \text{for } n = 1,3 \\
    4 ,& \text{for } n = 2 \\
    0 ,& \text{elsewhere}
  \end{cases}
\end{equation}

\subsubsection{Tabular Representation}\label{subsubsec:Tabular Representation}
This representation of a discrete-time sysem is done as a table of corresponding values.
\begin{table}[h!]
  \centering
  \begin{tabular}{c|cccccccccc}
    $n$ & $\ldots$ & -2 & -1 & 0 & 1 & 2 & 3 & 4 & 5 & $\ldots$ \\ \midrule
    $x(n)$ & $\ldots$ & 0 & 0 & 0 & 1 & 4 & 1 & 0 & 0 & $\ldots$
  \end{tabular}
\end{table}

\subsubsection{Sequence Representation}\label{subsubsec:Sequence Representation}
There are 2 methods of representation for this.
The first includes all values for $-\infty < n < \infty$.
In all cases, $n=0$ is marked in the sequence, somehow.
I will do this with an underline.
\begin{equation}\label{eq:Infinite Sequence Representation}
  x(n) = \lbrace \ldots, 0, \underline{0}, 1, 4, 1, 0, 0, \ldots \rbrace
\end{equation}

The second only works if all $x(n)$ values for $n < 0$ are 0.
\begin{equation}\label{eq:Zero Sequence Representation}
  x(n) = \lbrace \underline{0}, 1, 4, 1, 0, 0, \ldots \rbrace
\end{equation}

A finite-duration sequence can be represented as
\begin{equation}\label{eq:Finite Sequence Representation}
  x(n) = \lbrace 3, -1, \underline{-2}, 5, 0, 4, -1 \rbrace
\end{equation}
This is identified as a seven-point sequence.

A finite-duration sequence where $x(n)=0$ for all $n<0$ is represented as
\begin{equation}\label{eq:Zero Finite Sequence Representation}
  x(n) = \lbrace \underline{0}, 1, 4, 1 \rbrace
\end{equation}
This is identified as a four-point sequence.

\subsection{Elementary Discrete-Time Signals}\label{subsec:Elementary Discrete-Time Signals}
The following signals are basic signals that appear often and play an important role in signal processing.

\subsubsection{Unit Impulse Signal}\label{subsubsec:Unit Impulse Signal}
\begin{definition}[Unit Impulse Signal]\label{def:Unit Impulse Signal}
  The \emph{unit impulse signal} or \emph{unit sample sequence} is denoted as $\delta(n)$ and is defined as
  \begin{equation}\label{eq:Unit Impulse Signal}
    \delta(n) \equiv = \begin{cases}
      1, & \text{for } n = 0 \\
      0, & \text{for } n \neq 0
    \end{cases}
  \end{equation}

  This function is a signal that is zero everywhere, except at $n=0$, where its value is $1$.

  \begin{remark}
    This signal is different that the analog signal $\delta (t)$, which is also called a unit impulse, and is defined to be 0 everywhere except $t=0$.
    The discrete unit impulse sequence is much less mathematically complicated.
  \end{remark}
\end{definition}

\subsubsection{Unit Step Signal}\label{subsubsec:Unit Step Signal}
\begin{definition}[Unit Step Signal]\label{def:Unit Step Signal}
  The \emph{unit step signal} is denoted as $u(n)$ and is defined as
  \begin{equation}\label{eq:Unit Step Signal}
    u(n) \equiv \begin{cases}
      1, & \text{for } n \geq 0 \\
      0, & \text{for } n < 0
    \end{cases}
  \end{equation}
\end{definition}

\subsubsection{Unit Ramp Signal}\label{subsubsec:Unit Ramp Signal}
\begin{definition}[Unit Ramp Signal]\label{def:Unit Ramp Signal}
  The \emph{unit ramp signal} is denoted as $u_{r}(n)$ and is defined as
  \begin{equation}\label{eq:Unit Ramp Signal}
    u_{r}(n) \equiv \begin{cases}
      n, & \text{for } n \geq 0 \\
      0, & \text{for } n < 0
    \end{cases}
  \end{equation}
\end{definition}

\subsubsection{Exponential Signal}\label{subsubsec:Exponential Signal}
\begin{definition}[Exponential Signal]\label{def:Exponential Signal}
  The \emph{exponential signal} is a sequence of the form
  \begin{equation}\label{eq:Exponential Signal}
    x(n) = a^{n} \>\> \text{for all } n
  \end{equation}

  If $a$ is real, then $x(n)$ is a real signal.
  When $a$ is complex valued ($a \equiv b \pm c j$), it can be expressed as
   \begin{equation}\label{eq:Complex Exponential Signal}
     \begin{aligned}
       x(n) &= r^{n} e^{j \theta n} \\
       &= r^{n} \left( \cos \theta n + j \sin \theta n \right)
     \end{aligned}
   \end{equation}
  This can be expressed by graphing the real and imaginary parts
  \begin{equation}\label{eq:Real Imaginary Complex Exponential Signal}
    \begin{aligned}
      x_{R}(n) &\equiv r^{n} \cos \theta n \\
      x_{I}(n) &\equiv r^{n} j \sin \theta n
    \end{aligned}
  \end{equation}
  or by graphing the amplitude function and phase function.
  \begin{equation}\label{eq:Amplitude Phase Complex Exponential Signal}
    \begin{aligned}
      \lvert x(n) \rvert &= A(n) \equiv r^{n} \\
      \angle x(n) &= \phi(n) \equiv \theta n
    \end{aligned}
  \end{equation}
\end{definition}

\subsection{Classification of Discrete-Time Signals}\label{subsec:Classification Discrete-Time Signals}
\subsubsection{Energy Signal}\label{subsubsec:Energy Signal}
\subsubsection{Power Signal}\label{subsubsec:Power Signal}
\subsubsection{Periodic and Aperiodic Signals}\label{subsubsec:Periodic Aperiodic Signals}
\subsubsection{Symmetric and Antisymmetric Signals}\label{subsubsec:Symmetric and Antisymmetric Signals}
%%% Local Variables:
%%% mode: latex
%%% TeX-master: "../EITF75-Systems_and_Signals-Reference_Sheet"
%%% End:
