\section{Convolutions}\label{sec:Convolutions}
\begin{definition}[Convolution]\label{def:Convolution}
  The \emph{convolution} operator.

  \begin{equation}\label{eq:Convolution}
    y(t) = \sum\limits_{k=-\infty}^{\infty} x(k) * h(n-k)
  \end{equation}
\end{definition}

If there is a relaxed \nameref{def:Linear_Time-Invariant} system to an input $x(n)$, then the output can be found by computing the \nameref{def:Convolution} of the input with the sample response on the system.
This results in the equation shown below.
\begin{equation}\label{eq:LTI_System_Convolution}
  y(n) = x(n) * h(n)
\end{equation}

\subsection{Correlation}\label{subsec:Correlation}
\begin{definition}[Correlation]\label{def:Correlation}
  \emph{Correlation} measures the similarity between two signals.
  The greater the correlation, the more similar they are.

  There are 2 types of \nameref{def:Correlation}.
  \begin{enumerate}[noitemsep]
  \item \nameref{subsubsec:Cross_Correlation}
  \item \nameref{subsubsec:Auto_Correlation}
  \end{enumerate}
\end{definition}

\subsubsection{Cross Correlation}\label{subsubsec:Cross_Correlation}
\begin{definition}[Cross Correlation]\label{def:Cross_Correlation}
  \emph{Cross correlation} measures the similarity between time shifted versions of \textbf{\emph{different}} signals.

  The defining equation is shown below:
  \begin{equation}\label{eq:Cross_Correlation}
    r_{y,x}(k) = \sum\limits_{n=-\infty}^{\infty} y(n) x(n-k)
  \end{equation}

  However, there is a way to express \Cref{eq:Cross_Correlation} in terms of a \nameref{def:Convolution}.
  \begin{equation}\label{eq:Cross_Correlation-Convolution}
    r_{y,x}(k) = y(n) * x(-n)
  \end{equation}
\end{definition}

\subsubsection{Auto Correlation}\label{subsubsec:Auto_Correlation}
\begin{definition}[Auto Correlation]\label{def:Auto_Correlation}
  \emph{Auto correlation} measures the similarity between time shifted version of the \textbf{\emph{same}} signal.

  The defining equation is:
  \begin{equation}\label{eq:Auto_Correlation}
    r_{x,x}(k) = \sum\limits_{n=-\infty}^{\infty} x(n) x(n-k)
  \end{equation}

  However, there is a way to express \Cref{eq:Auto_Correlation} in terms of a \nameref{def:Convolution}.
  \begin{equation}\label{eq:Auto_Correlation-Convolution}
    r_{x,x}(k) = x(n) * x(-n)
  \end{equation}

  \begin{remark}\label{rmk:Cross-Auto_Correlation_Relation}
    It is good to note that the \nameref{def:Auto_Correlation} is technically a type of \nameref{def:Cross_Correlation} where the second function is the same as the first.
  \end{remark}
\end{definition}

\subsection{Properties of the Convolution}\label{subsec:Convolution_Properties}
\begin{table}[h!]
  \centering
  \begin{tabular}{cc}
    \toprule
    \nameref{subsubsec:Convolution_Property-Identity} & $y(n) = x(n) * \delta(n) = x(n)$ \\
    \nameref{subsubsec:Convolution_Property-Shifting} & $x(n) * \delta(n-k) = y(n-k) = x(n-k)$ \\
    \nameref{subsubsec:Convolution_Property-Commutative} & $x(n) * h(n) = h(n) * x(n)$ \\
    \nameref{subsubsec:Convolution_Property-Associative} & $\left[ x(n) * h_{1}(n) \right] * h_{2}(n) = x(n) * \left[ h_{1}(n) * h_{2}(n) \right]$\\
    \nameref{subsubsec:Convolution_Property-Distributive} & $x(n) * \left[ h_{1}(n) + h_{2}(n) \right] = x(n) * h_{1}(n) + x(n) * h_{2}(n)$ \\
    \bottomrule
  \end{tabular}
  \caption{\nameref{subsec:Convolution_Properties}}
  \label{tab:Convolution_Properties}
\end{table}

\subsubsection{Identity Property}\label{subsubsec:Convolution_Property-Identity}
\begin{definition}[Identity Property]\label{def:Convolution_Property-Identity}
  The \nameref{def:Unit Impulse Signal} is the identity element for the \nameref{def:Convolution}.
  \begin{equation}\label{eq:Convolution_Property-Identity}
    y(n) = x(n) * \delta(n) = x(n)
  \end{equation}
\end{definition}

\subsubsection{Shifting Property}\label{subsubsec:Convolution_Property-Shifting}
\begin{definition}[Shifting Property]\label{def:Convolution_Property-Shifting}
  Since the $\delta(n)$ function is the Identity function, if we shift $\delta(n)$ by $k$, the convolution sequence is also shifted by $k$.
  \begin{equation}\label{eq:Convolution_Property-Shifting}
    x(n) * \delta(n-k) = y(n-k) = x(n-k)
  \end{equation}
\end{definition}

\subsubsection{Commutative Law}\label{subsubsec:Convolution_Property-Commutative}
\begin{definition}[Commutative Law for Convolutions]\label{def:Convolution_Property-Commutative}
  The \emph{commutative law for \nameref{def:Convolution}s} is just like many other operations.
  \begin{equation}\label{eq:Convolution_Property-Commutative}
    x(n) * h(n) = h(n) * x(n)
  \end{equation}
\end{definition}

\subsubsection{Associative Law}\label{subsubsec:Convolution_Property-Associative}
\begin{definition}[Associative Law for Convolutions]\label{def:Convolution_Property-Associative}
  The \emph{associative law for \nameref{def:Convolution}s} is just like many other operations.
  \begin{equation}\label{eq:Convolution_Property-Associative}
    \left[ x(n) * h_{1}(n) \right] * h_{2}(n) = x(n) * \left[ h_{1}(n) * h_{2}(n) \right]
  \end{equation}
\end{definition}

\subsubsection{Distributive Law}\label{subsubsec:Convolution_Property-Distributive}
\begin{definition}[Distributive Law for Convolutions]\label{def:Convolution_Property-Distributive}
  The \emph{distributive law for \nameref{def:Convolution}s} is just like many other operations.
  \begin{equation}\label{eq:Convolution_Property-Distributive}
    x(n) * \left[ h_{1}(n) + h_{2}(n) \right] = x(n) * h_{1}(n) + x(n) * h_{2}(n)
  \end{equation}
\end{definition}

%%% Local Variables:
%%% mode: latex
%%% TeX-master: "../EITF75-Systems_and_Signals-Reference_Sheet"
%%% End:
