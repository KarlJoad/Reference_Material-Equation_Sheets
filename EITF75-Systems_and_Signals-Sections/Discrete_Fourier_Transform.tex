\section{Discrete Fourier Transform}\label{sec:DFT}
\begin{definition}[Discrete Fourier Transform]\label{def:DFT}
  The \emph{Discrete Fourier Transform} or \emph{DFT} can be the \nameref{def:Discrete-Time_Fourier_Transform} sampled at certain values.
  This is only true if $N > \text{Length of Signal DTFT}$.

  The $N$-point DFT is shown as:
  \begin{equation}\label{eq:DFT}
    X_{DFT}(k) = \sum\limits_{n=0}^{N-1} x(n) e^{-j 2\pi \frac{k}{N} n} \:\: \text{for } k=0, 1, 2, \ldots, N-1
  \end{equation}

  If $N$ is specified, then replace all occurrences of $N$ in \Cref{eq:DFT} with that value.

  \begin{remark}
    If the length, $N$ of the DFT is not specified, it is assumed that $N = \text{length of the signal}$.
    If the length of the DFT $N$ is greater than the length of the signal, you are sampling the \nameref{def:Discrete-Time_Fourier_Transform} of the signal.
  \end{remark}

  \begin{remark}[Discrete Fourier Transform Resolution]\label{rmk:DFT_Resolution}
    The \emph{resolution} of the \nameref{def:DFT} is tied to the length of the signal used.
    The longer the signal is, the greater the resolution of the \nameref{def:DFT}.
    \begin{itemize}[noitemsep]
    \item Padding a signal with 0s will increase the resolution of a signal without changing it too much
    \end{itemize}
  \end{remark}

  \begin{remark}[Time Complexity to Compute]\label{rmk:DFT_Time_Complexity}
    If we are interested in the time complexity (Big-O) $O(n)$ of the \nameref{def:DFT}, it is $O(n^{2})$.
    \begin{enumerate}[noitemsep]
    \item $N$ values of $X_{DFT}(k)$ to be computed
    \item Each value of $X_{DFT}(k)$ requires $N$ multiplications of $x(n) e^{-j 2\pi \frac{k}{N} n}$
    \item This means the time complexity is $O(n^{2})$.
    \end{enumerate}
  \end{remark}
\end{definition}

The \nameref{def:DFT} is used for many reasons, some of which are listed below:
\begin{enumerate}[noitemsep]
\item Computers have limited memory, and the \nameref{def:Discrete-Time_Fourier_Transform} of a \nameref{def:Discrete-Time Signals} is a \nameref{def:Continuous-Time Signals}. Thus, the \nameref{def:Discrete-Time_Fourier_Transform} cannot be stored in memory.
\item The \nameref{def:FFT} can be used to compute \nameref{def:Circular_Convolution}s relatively quickly.
\end{enumerate}

\begin{definition}[Fast Fourier Transform]\label{def:FFT}
  This is a special case of the \nameref{def:DFT} that is only useful for computers.
  If $N=2^{k}$, where $k$ is an integer, then the \nameref{def:DFT} can be performed in $O(n \log_{2}(n))$ time.
  This can be used to calculate \nameref{def:Linear_Convolution}s relatively quickly, especially when the number of terms in the sequence is quite large.

  These 2 pieces of MATLAB/GNU Octave source produce the same output, but through different methods.
  \begin{matlabsource}
    x = [1 2 3 4]
    h = [2 2 1 1]
    y = conv(x, h)
    y = 2 6 11 17 13 7 4
  \end{matlabsource}

  \begin{matlabsource}
    x = [1 2 3 4 0 0 0 0]
    h = [2 2 1 1 0 0 0 0]
    Y_k = ifft(fft(x) .* fft(h))
    Y_k = 2.0000 6.0000 11.0000 17.0000 13.0000 7.0000 4.0000 -0.0000
  \end{matlabsource}
  \begin{remark}[Zero-Padding]
    Note that the padding with zeros to a length greater than the output length of the \nameref{def:Linear_Convolution} is required.
    Also, to take advantage of the \nameref{def:FFT}'s quick calculation property, the length of the inputs \textbf{MUST} be a power of 2, $2^{k}$.

    So, the below code produces something different because it calculates the \nameref{def:Circular_Convolution} directly.
    \begin{matlabsource}
      x = [1 2 3 4]
      h = [2 2 1 1]
      Y_k_bad = ifft(fft(x) .* fft(h))
      Y_k_bad = 15 13 15 17
    \end{matlabsource}
  \end{remark}
\end{definition}

\begin{remark*}
  We now have several very different possible representations of the same signal with only slight variations in the function description.
  We will list them out to ensure clarity.
  \begin{itemize}[noitemsep]
  \item $x(t)$, \nameref{def:Continuous-Time Signals}
  \item $x(n)$, \nameref{def:Discrete-Time Signals}
  \item $X(z)$, the \nameref{subsec:Z-Transform} of $x(n)$
  \item $X(F)$, the \nameref{def:FourierTransform} of $x(t)$
  \item $X(f)$, the \nameref{def:Discrete-Time_Fourier_Transform} of $x(n)$
  \item $X(k)$, the \nameref{def:DFT} of $x(n)$
  \end{itemize}
\end{remark*}

\subsection{\nameref{sec:DFT} of Sinusoids}\label{subsec:DFT_of_Sinusoids}
The 2 equations presented below are important for performing the \nameref{def:DFT} on sinusoids.
Both \Crefrange{eq:DFT_of_Cosine}{eq:DFT_of_Sine} work in both directions.

\begin{equation}\label{eq:DFT_of_Cosine}
  \begin{aligned}
    x(n) &= A \cos \left( 2\pi \frac{k_{0}}{N} n \right),\; 0 < k_{0} \leq N-1 \\
    &= \frac{A}{2} \left( e^{j \frac{2\pi k_{0}}{N} n} + e^{-j \frac{2\pi k_{0}}{N} n} \right) \\
    X(k) &= \sum\limits_{n=0}^{N-1} \frac{A}{2} e^{j \frac{2\pi k_{0} n}{N}} e^{-j \frac{2\pi k n}{N}} + \sum\limits_{n=0}^{N-1} \frac{A}{2} e^{-j \frac{2\pi k_{0} n}{N}} e^{-j \frac{2\pi k n}{N}} \\
    &= \frac{A}{2} \sum\limits_{n=0}^{N-1} e^{\frac{-j 2\pi (k-1)}{N}} + \frac{A}{2} \sum\limits_{n=0}^{N-1} e^{\frac{-j 2\pi (k+1)}{N}} \\
    &\text{This is a Geometric Series} \\
    X(k) &= \frac{AN}{2} \biggl[ (\delta(k-k_{0}) \bmod N) + (\delta(k+k_{0}) \bmod N) \biggr]
  \end{aligned}
\end{equation}

\begin{equation}\label{eq:DFT_of_Sine}
  \begin{aligned}
    x(n) &= A \sin \left( 2\pi \frac{k_{0}}{N} n \right),\; 0 < k_{0} \leq N-1 \\
    &= \frac{A}{2j} \left( e^{j \frac{2\pi k_{0}}{N} n} - e^{-j \frac{2\pi k_{0}}{N} n} \right) \\
    X(k) &= \sum\limits_{n=0}^{N-1} \frac{A}{2j} e^{j \frac{2\pi k_{0} n}{N}} e^{-j \frac{2\pi k n}{N}} - \sum\limits_{n=0}^{N-1} \frac{A}{2j} e^{-j \frac{2\pi k_{0} n}{N}} e^{-j \frac{2\pi k n}{N}} \\
    &= \frac{A}{2j} \sum\limits_{n=0}^{N-1} e^{\frac{-j 2\pi (k-1)}{N}} - \frac{A}{2j} \sum\limits_{n=0}^{N-1} e^{\frac{-j 2\pi (k+1)}{N}} \\
    &\text{This is a Geometric Series} \\
    X(k) &= \frac{AN}{2j} \biggl[ (\delta(k-k_{0}) \bmod N) - (\delta(k+k_{0}) \bmod N) \biggr]
  \end{aligned}
\end{equation}

\subsection{Inverse Discrete Fourier Transform}\label{subsec:IDFT}
\begin{definition}[Inverse Discrete Fourier Transform]\label{def:IDFT}
  The \emph{Inverse Discrete Fourier Transform (IDFT)} is the inverse of the \nameref{def:DFT}.

  \begin{equation}\label{eq:IDFT}
    x_{IDFT}(n) = \frac{1}{N} \sum\limits_{k=0}^{N-1} X(k) e^{j 2\pi \frac{k}{N} n} \:\: \text{for } n = 0, 1, \ldots, N-1
  \end{equation}
\end{definition}

\subsection{The Discrete Fourier Transform Expressed with Matrices}\label{subsec:DFT_Matrix}
We start with \Cref{eq:DFT_W_Variable} to simplify the writing of our equations.
\begin{equation}\label{eq:DFT_W_Variable}
  W_{N} = e^{-j \frac{2\pi}{N}}
\end{equation}

$W_{N}$ is a complex variable used to replace the exponential function.
You will see $W_{N}^{kn}$, which evaluates to $e^{-j \frac{2\pi}{N}kn}$.

If we define 2 matrices, where $\mathbf{x}(n)$ is a discretely-valued function and $\mathbf{X}_{N}(k)$ is its \nameref{def:DFT}:
\begin{equation}\label{eq:DFT_Function_Matrices}
  \mathbf{x}_{N}(n) =
  \begin{bmatrix}
    x(0) \\
    x(1) \\
    \vdots \\
    x(N -1)
  \end{bmatrix},\:\:\:
  \mathbf{X}_{N}(k) =
  \begin{bmatrix}
    X(0) \\
    X(1) \\
    \vdots \\
    X(N-1) \\
  \end{bmatrix}
\end{equation}

Now, we extend our definition of $W_{N}$ from \Cref{eq:DFT_W_Variable} to this matrix format as well.
\begin{equation}\label{eq:DFT_W_Matrix}
  \mathbf{W}_{N} =
  \begin{bmatrix}
    1 & 1 & 1 & \cdots & 1 \\
    1 & W_{N}^{kn} \vert_{k=1, n=1} & W_{N}^{kn} \vert_{k=1, n=2} & \cdots & W_{N}^{kn} \vert_{k=1, n=N-1} \\
    \vdots & W_{N}^{kn} \vert_{k=2, n=1} & W_{N}^{kn} \vert_{k=2, n=2} & \cdots & W_{N}^{kn} \vert_{k=2, n=N-1} \\
    \vdots & \vdots & \vdots & \ddots & \vdots \\
    1 & W_{N}^{kn} \vert_{k=N-1, n=1} & W_{N}^{kn} \vert_{k=N-1, n=2} & \cdots & W_{N}^{kn} \vert_{k=N-1, n=N-1} \\
  \end{bmatrix}
\end{equation}

With these matrices defined, we can express the $N$-point \nameref{def:DFT} as
\begin{equation}\label{eq:DFT_Matrix}
  \mathbf{X}_{N}(k) = \mathbf{W}_{N} \mathbf{x}_{n}(n)
\end{equation}

and, because of matrix inversion, we can express the $N$-point \nameref{def:IDFT} as
\begin{equation}\label{eq:IDFT_Matrix_Inverted_W_Matrix}
  \mathbf{x}_{N}(n) = \mathbf{W}_{N}^{-1} \mathbf{X}_{N}(k)
\end{equation}

$\mathbf{x}_{N}(n)$ can also be expressed as the \nameref{def:IDFT} with \Cref{eq:IDFT_Matrix_Complex_Conjugate}
\begin{equation}\label{eq:IDFT_Matrix_Complex_Conjugate}
  \mathbf{x}_{N}(n) = \frac{1}{N} \mathbf{W}_{N}^{*} \mathbf{X}_{N}(k)
\end{equation}

The duality presented from \Crefrange{eq:IDFT_Matrix_Inverted_W_Matrix}{eq:IDFT_Matrix_Complex_Conjugate} means
\begin{equation}\label{eq:DFT_W_Matrix_Unitary}
  \begin{aligned}
    \mathbf{W}_{N}^{-1} = \frac{1}{N} \mathbf{W}_{N}^{*} \\
    \mathbf{W}_{N} \mathbf{W}_{N}^{*} = N \mathbf{I}_{N} \\
  \end{aligned}
\end{equation}
where $\mathbf{I}_{N}$ is an $N \times N$ identity matrix.

\subsection{Properties of the Discrete Fourier Transform}\label{subsec:DFT_Properties}
With the \nameref{def:DFT}, the usual properties do not apply.
\begin{itemize}[noitemsep]
\item $x(n) * y(n) \neq X(k)Y(k)$
\item $x(n-n_{0}) \nleftrightarrow X(k) e^{j 2\pi \frac{k}{N} n_{0}}$
\end{itemize}

These have to be modified instead.
\begin{table}[h!]
  \centering
  \begin{tabular}{ccc}
    \toprule
    Property & Time Domain $x(n)$ & DFT Domain $X(k)$ \\
    \midrule
    Notation & $x(n)$, $y(n)$ & $X(k)$, $Y(k)$ \\
    \nameref{subsubsec:DFT_Properties-Periodicity} & $x(n) = x(n+N)$ & $X(k) = X(k+N)$ \\
    \nameref{subsubsec:DFT_Properties-Linearity} & $a_{1}x_{1}(n) + a_{2}x_{2}(n)$ & $a_{1}X_{1}(k) + a_{2}X_{2}(k)$ \\
    \nameref{subsubsec:DFT_Properties-Time_Reversal} & $x(N-n)$ & $X(N-k)$ \\
    \nameref{subsubsec:DFT_Properties-Circular_Time_Shifting} & $x(n - n_{0} \bmod N)$ & $X(k) e^{-j 2\pi \frac{k}{N} n_{0}} $ \\
    \nameref{subsubsec:DFT_Properties-Circular_Frequency_Shift} & $x(n)e^{j2\pi l n/N}$ & $X(k-l \bmod N)$ \\
    \nameref{subsubsec:DFT_Properties-Complex_Conjugate} & $X^{*}(n)$ & $X^{*}(N-k)$ \\
    \nameref{subsubsec:DFT_Properties-Circular_Convolution} & $x(n) \CircularConvolution y(n)$ & $X(k)Y(k)$ \\
    \nameref{subsubsec:DFT_Properties-Circular_Correlation} & $x(n) \CircularConvolution y^{*}(-n)$ & $X(k)Y^{*}(k)$ \\
    \nameref{subsubsec:DFT_Properties-2_Sequence_Multiplication} & $x_{1}(n)x_{2}(n)$ & $\frac{1}{N} X_{1}(k) \CircularConvolution X_{2}(k)$ \\
    \nameref{subsubsec:DFT_Properties-Parsevals_Theorem} & $\sum\limits_{n=0}^{N-1} x(n) y^{*}(n)$ & $\frac{1}{N} \sum\limits_{k=0}^{N-1} X(k)Y^{*}(k)$ \\
    \bottomrule
  \end{tabular}
  \caption{Properties of the \nameref{def:DFT}}
  \label{tab:DFT_Properties}
\end{table}

\subsubsection{Periodicity}\label{subsubsec:DFT_Properties-Periodicity}
If $x(n)$ and $X(k)$ are an $N$-point \nameref{def:DFT} pair, then
\begin{equation}\label{eq:DFT_Properties-Periodicity-Time}
  x(n+N) = x(n)
\end{equation}
\begin{equation}\label{eq:DFT_Properties-Periodicity-Frequency}
  X(k+N) = X(k)
\end{equation}

The periodicities in $x(n)$ and $X(k)$ follow immediately from \Cref{eq:DFT} and \Cref{eq:IDFT}, respectively.

\subsubsection{Linearity}\label{subsubsec:DFT_Properties-Linearity}
If
\begin{align*}
  x_{1}(n) &\DFTRelation X_{1}(k) \\
  x_{2}(n) &\DFTRelation X_{2}(k)
\end{align*}
then for any real-valued or complex-valued constants $a_{1}$ and $a_{2}$,

\begin{equation}\label{eq:DFT_Properties-Linearity}
  a_{1}x_{1}(n) + a_{2}x_{2}(n) \DFTRelation a_{1}X_{1}(k) + a_{2}X_{2}(k)
\end{equation}

\subsubsection{Time Reversal}\label{subsubsec:DFT_Properties-Time_Reversal}
If
\begin{equation*}
  x(n) \DFTRelation X(k)
\end{equation*}
then
\begin{equation}\label{eq:DFT_Properties-Time_Reversal}
  x(-n \bmod N) = x(N-n) \DFTRelation X(-k \bmod N) = X(N-k)
\end{equation}

Hence, reversing the $N$-point sequence in time is equivalent to reversing the \nameref{def:DFT} values.

\subsubsection{Circular Time Shifting}\label{subsubsec:DFT_Properties-Circular_Time_Shifting}
The definition of time shifting needs to be modified from the definition of time shifting in~\Cref{subsubsec:FourierTransformProperties-TimeShifting}.

\begin{definition}[Discrete Fourier Transform Time Shifting]\label{def:DFT_Properties-Circular_Time_Shifting}
  If a signal $x(n)$ is shifted by $n_{0}$, then the \nameref{def:DFT} is also shifted.

  \begin{equation}\label{eq:DFT_Properties-Circular_Time_Shifting}
    x(n - n_{0} \bmod N) = X(k) e^{-j 2\pi \frac{k}{N} n_{0}}
  \end{equation}

  The modulus is because the signal is now circular.
\end{definition}

\subsubsection{Circular Frequency Shift}\label{subsubsec:DFT_Properties-Circular_Frequency_Shift}
If
\begin{equation*}
  x(n) \DFTRelation X(k)
\end{equation*}
then
\begin{equation}\label{eq:DFT_Properties-Circular_Frequency_Shift}
  x(n)e^{j 2\pi l n/N} \DFTRelation X((k-l) \bmod N)
\end{equation}

Hence, the multiplication of the sequence $x(n)$ with the complex exponential sequence $e^{k 2\pi k n/N}$ is equivalent to the circular shift of the \nameref{def:DFT} by $l$ unitus in frequency.
This is the dual to the \nameref{subsubsec:DFT_Properties-Circular_Time_Shifting} property.

\subsubsection{Complex Conjugate}\label{subsubsec:DFT_Properties-Complex_Conjugate}
If
\begin{equation*}
  x(n) \DFTRelation X(k)
\end{equation*}
then
\begin{equation}\label{eq:DFT_Properties-Complex_Conjugate}
  x^{*}(n) \DFTRelation X^{*}(-k \bmod N) = X^{*}(N-k)
\end{equation}

The \nameref{def:IDFT} of $X^{*}(k)$ is
\begin{equation*}
  \frac{1}{N} \sum\limits_{k=0}^{N-1} X^{*}(k)e^{j 2\pi k n/N} = \Biggl[ \frac{1}{N} \sum\limits_{k=0}^{N-1} X(k) e^{j 2\pi k (N-n)/N} \Biggr]
\end{equation*}
therefore,
\begin{equation}\label{eq:DFT_Properties-IDFT_Complex_Conjugate}
  x^{*}(-n \bmod N) = x^{*}(N-n) \DFTRelation X^{*}(k)
\end{equation}

\subsubsection{Circular Convolution}\label{subsubsec:DFT_Properties-Circular_Convolution}
\begin{definition}[Circular Convolution]\label{def:Circular_Convolution}
  The \emph{circular convolution} is similar to the \nameref{def:Linear_Convolution}.
  The key difference is that the circular convolution repeats its sequence indefinitely.
  Computing a \nameref{def:Circular_Convolution} is shown in \Cref{ex:Circular Convolution}.

  Mathematically, this type of convolution is represented as
  \begin{equation}\label{eq:Circular_Convolution}
    x_{N}(n) \CircularConvolution h(n) = \sum\limits_{m=-\infty}^{\infty} h(m) \sum\limits_{k=-\infty}^{\infty} x(n-m-kN)
  \end{equation}

  This can be simplified a little bit to
  \begin{equation}\label{eq:Circular_Convolution-Simplified}
    x_{1}(n) \CircularConvolution x_{2}(n) = \sum\limits_{k=0}^{N-1} x_{1}(k) x_{2}(n-k \bmod N)
  \end{equation}
  It is important to remember that the modulus (mod) operator yields 0 when the input is a multiple of the divisor.

  \begin{remark}[Alternate Circular Convolution Symbol]
    There is no single defined symbol for \nameref{def:Circular_Convolution}s and \nameref{def:Linear_Convolution}s.
    In this text, and personally, I use the $\CircularConvolution$ symbol.
    However, other texts may use:
    \begin{enumerate}[noitemsep]
    \item $\otimes$
    \item $\textcircled{N}$
    \item $*$
    \item $\cdot$ (Centered Dot)
    \item $\bullet$
    \item etc.
    \end{enumerate}
  \end{remark}
  
  \begin{remark}[Circular Convolution Length]\label{rmk:Circular_Convolution_Length}
    The length of the resulting sequence from a \nameref{def:Circular_Convolution} is
    \begin{equation}\label{eq:Circular_Convolution_Length}
      L
    \end{equation}
    where $L$ is the length of the input sequences.
  \end{remark}

  \begin{remark}[Linear vs. Circular Convolution Length]
    The length of a \nameref{def:Linear_Convolution} is $2L-1$, whereas the length of a \nameref{def:Circular_Convolution} is $L$; given that the input signal lengths are $L$.
  \end{remark}
\end{definition}

\begin{example}[Lecture 10, Slide 90]{Circular Convolution}
  Perform a \nameref{def:Circular_Convolution} on these signals.
  \begin{align*}
    x(n) &= \lbrace \underline{1}, 2, 3, 4 \rbrace \\
    h(n) &= \lbrace \underline{2}, 2, 1, 1 \rbrace
  \end{align*}

  \tcblower{}

  There are 2 realistic ways to solve this by hand.
  \begin{enumerate}[noitemsep]
  \item Perform a convolution where one of the signals is repeated periodically
  \item Perform a normal convolution, but pad with 0s to get input signals that are longer than $2L-1$ of the original signals. Then, add the first term where 0s were padded to the first where they weren't.
    \begin{itemize}[noitemsep]
    \item This is the basis of performing a \nameref{def:Linear_Convolution} with the \nameref{def:FFT} and \nameref{def:Circular_Convolution}s
    \end{itemize}
  \end{enumerate}

  For both methods, I will perform a \nameref{par:Folding} on $h(n)$ to get $h(-n) = \lbrace 1, 1, 2, \underline{2} \rbrace$.

  \textbf{Method 1} \\
  \begin{center}
    \begin{tabular}{ccccccccccc}
      \toprule
      $h(k)$ & 1 & 1 & 2 & \underline{2} & $\rightarrow$ & & & & & \\
      $x(k)$ & \textcolor{red}{2} & \textcolor{red}{3} & \textcolor{red}{4} & \underline{1} & 2 & 3 & 4 & \textcolor{red}{1} & \textcolor{red}{2} & \textcolor{red}{3} \\
      \midrule
      $y(k)$ & & & & \underline{15} & 13 & 15 & 17 & & \\
      \bottomrule
    \end{tabular}
  \end{center}

  \textbf{Method 2}
  \begin{center}
    \begin{tabular}{ccccccccccccccc}
      \toprule
      $h(k)$ & 1 & 1 & 2 & \underline{2} & $\rightarrow$ & & & & & & & & & \\
      $x(k)$ & \textcolor{red}{0} & \textcolor{red}{0} & \textcolor{red}{0} & \underline{1} & 2 & 3 & 4 & 0 & 0 & 0 & 0 & \textcolor{red}{1} & \textcolor{red}{2} & \textcolor{red}{3} \\
      \midrule
      $y(k)$ & & & & \underline{2} & 6 & 11 & 17 & 13 & 7 & 4 & 0 & & & \\
      \bottomrule
    \end{tabular}
  \end{center}
  Then, $\underline{2} + 13 = 15$, $6 + 7 = 13$, $11 + 4 = 15$, $17 + 0 = 17$.
  \newline

  Both methods yield $y(k) = \lbrace \underline{15}, 13, 15, 17 \rbrace$.

  \begin{remark*}
    You can also solve this with MATLAB/GNU Octave by following the code examples in the definition of the \nameref{def:FFT},~\Cref{def:FFT}.
  \end{remark*}
\end{example}

\begin{remark*}
  Personally, I use Method 2 shown in \Cref{ex:Circular Convolution} to calculate \nameref{def:Circular_Convolution}s by hand.
\end{remark*}

\subsubsection{Circular Correlation}\label{subsubsec:DFT_Properties-Circular_Correlation}
In general, for complex-valued sequences $x(n)$ and $y(n)$, if
\begin{align*}
  x(n) &\DFTRelation X(k) \\
  y(n) &\DFTRelation Y(k)
\end{align*}
then
\begin{equation}\label{eq:DFT_Properties-Circular_Correlation}
  x(n) \CircularConvolution y^{*}(n) = \tilde{r}_{xy}(l) \DFTRelation \tilde{R}_{xy}(k) = X(k)Y^{*}(k)
\end{equation}
\begin{remark*}
  This is closely related to the linear \nameref{def:Cross_Correlation}.
\end{remark*}

\paragraph{Circular Autocorrelation}\label{par:DFT_Properties-Circular_Autocorrelation}
If $x(n) = y(n)$, then an autocorrelation is performed.
This is closely related to the linear \nameref{def:Auto_Correlation}.
\begin{equation}\label{eq:DFT_Properties-Circular_Autocorrelation}
  x(n) \CircularConvolution x^{*}(n) = \tilde{r}_{xx}(l) \DFTRelation \tilde{R}_{XX}(k) = X(k)X^{*}(k)
\end{equation}

\subsubsection{Multiplication of 2 Sequences}\label{subsubsec:DFT_Properties-2_Sequence_Multiplication}
If
\begin{align*}
  x_{1}(n) &\DFTRelation X_{1}(k) \\
  x_{2}(n) &\DFTRelation X_{2}(K)
\end{align*}
then
\begin{equation}\label{eq:2_Sequence_Multiplication}
  x_{1}(n)x_{2}(n) \DFTRelation \frac{1}{N} X_{1}(k) \CircularConvolution X_{2}(k)
\end{equation}

\subsubsection{Parseval's Theorem}\label{subsubsec:DFT_Properties-Parsevals_Theorem}
For complex-valued sequences $x(n)$ and $y(n)$, in general, if
\begin{align*}
  x(n) &\DFTRelation X(k) \\
  y(n) &\DFTRelation Y(k)
\end{align*}
then
\begin{equation}\label{eq:DFT_Properties-Parsevals_Theorem}
  \sum\limits_{n=0}^{N-1}x(n)y^{*}(n) = \frac{1}{N} \sum\limits_{k=0}^{N-1} X(k)Y^{*}(k)
\end{equation}

If $x(n) = y(n)$
\begin{equation}\label{eq:DFT_Properties-Parsevals_Theorem-Magnitude}
  \sum\limits_{n=0}^{N-1}x(n) x^{*}(n) = \sum\limits_{n=0}^{N-1} \lVert x(n) \rVert^{2} = \frac{1}{N} \sum\limits_{k=0}^{N-1} X(k)X^{*}(k) = \frac{1}{N} \sum\limits_{k=0}^{N-1} \lVert X(k) \rVert^{2}
\end{equation}

\subsection{Application of the Discrete Fourier Transform}\label{subsec:DFT_Application}
If the signal that is being filtered by a finite filter with equation $h(n)$ has length $M$ is generating an infinite number of discrete values, then the standard \nameref{def:Linear_Convolution} does not work.
This is the case with things like data streams, where there is a constant flow of data that is discretely valued.

We need to have a way to perform convolutions and other operations on, what is essentially, an infinite signal.
There are 3 ways discussed in the textbook, but I will only heavily discuss the first 2.
These \textbf{do not} work for \nameref{def:IIR} filters.
\begin{enumerate}[noitemsep]
\item \nameref{subsubsec:DFT_Application-Overlap_Add}
\item \nameref{subsubsec:DFT_Application-Overlap_Save}
\item \nameref{subsubsec:DFT_Application-Overlap_Discard}
\end{enumerate}

\subsubsection{Overlap-Add}\label{subsubsec:DFT_Application-Overlap_Add}
This is a way to perform a \nameref{def:Linear_Convolution} on an infinite discretely-valued signal with a filter $h(n)$ that has a finite length of $M$.

The steps to perform \nameref{subsubsec:DFT_Application-Overlap_Add} are:
\begin{enumerate}[noitemsep]
\item Partition your ``infinite'' discrete signal into blocks of length $L$
  \begin{itemize}[noitemsep]
  \item This will give you $A$ blocks, where $0 \leq A < \infty$.
  \item This means you have blocks with sequences $x_{a}(n), \: a \in A$ that are of length $L$.
  \end{itemize}
\item Compute the \nameref{def:DFT} of your filter $h(n)$ to get $H(k)$.
  \begin{enumerate}[noitemsep]
  \item Zero-pad $h(n)$ to a length of $N=L+M-1$. Remember that $h(n)$ has length $L$.
  \item Perform the \nameref{def:DFT} on the zero-padded sequence to get $H(k)$
  \item You can disregard $h(n)$ now
    \begin{itemize}[noitemsep]
    \item If you're programming this, this means you can precalculate $H(k)$ somewhere else.
    \end{itemize}
  \end{enumerate}
\item Compute the \nameref{def:DFT} of one of the blocks from the partitioned input signal
  \begin{enumerate}[noitemsep]
  \item Zero-Pad $x_{a}(n)$ to a length of $N=L+M-1$
  \item Perform the \nameref{def:DFT} on the zero padded sequence to get $X_{a}(k)$
  \end{enumerate}
\item Multiply $X_{a}(k)$ and $H(k)$ to get $Y_{a}(k)$
  \begin{itemize}[noitemsep]
  \item $Y_{a}(k) = X_{a}(k) H(k)$
  \end{itemize}
\item Perform the \nameref{def:IDFT} on $Y_{a}(k)$ to get $y_{a}(n)$.
  \begin{itemize}[noitemsep]
  \item $y_{a}(n)$ is of length $N=L+M-1$
  \end{itemize}
\item Take the first $L$ terms of $y(n)$ and make that your real output $z(n)$
\item Save the other $M-1$ terms in memory (leftovers), call this $\ell_{a}(n)$
\item Take the next block, find $y_{a+1}(n)$, and add $\ell_{a}(n)$ to the output of the next block's $y_{a+1}(n)$
  \begin{enumerate}[noitemsep]
  \item The first term of $\ell_{a}(n)$, while not the first term from block $a$, is now the first term when adding it to $y_{a+1}(n)$.
  \item Perform element-wise addition.
  \item $z_{a+1}(n) = y_{a+1}(n) + \ell_{a}(n)$
  \end{enumerate}
\item Repeat this for the entire length of the infinitely long input signal.
\end{enumerate}

\begin{remark*}
  In case the steps above were confusing, the variables used and their meanings are below.
  \begin{itemize}[noitemsep]
  \item $x(n)$, the infinitely long input signal
  \item $a$, the block index from the partitioning of the input signal
  \item $A$, the total number of partitions of $x(n)$ that are present
  \item $L$, the length of each block
  \item $x_{a}(n)$, the signal sequence in block $a$
  \item $X_{a}(k)$, the \nameref{def:DFT} of the signal sequence in block $a$
  \item $h(n)$, the finite length filter's sequence
  \item $H(k)$, the \nameref{def:DFT} of the filter's sequence
  \item $M$, the length of the filter's sequence
  \item $N$, the length used in the circular convolution, equal to $L+M-1$
  \item $Y_{a}(k)$, the output of the multiplication of $X_{a}(k)$ and $H(k)$
  \item $y_{a}(n)$, the \nameref{def:IDFT} of $Y_{a}(k)$
  \item $\ell_{a}(n)$, the part of $y_{a}(n)$ after pulling off the first $L$ terms
  \item $z_{a}(n)$, the final output of performing the convolution of block $a$ and adding $\ell_{a-1}(n)$
  \end{itemize}
\end{remark*}

The trick for \nameref{subsubsec:DFT_Application-Overlap_Add} is to get each block's convolution and multiplication fast enough, and it becomes efficient.
To make it fast enough, $N$ should be a power of 2 ($2^{k}$).
This turns the calculation of the \nameref{def:DFT}s into \nameref{def:FFT}s, which are really fast.
Then multiplication, while lengthy, is a singular operation, unlike convolutions which are many.
Because we zero-pad smartly, while we are technically calculating the \nameref{def:Circular_Convolution} of the signal, we are actually performing the \nameref{def:Linear_Convolution}.

\subsubsection{Overlap-Save}\label{subsubsec:DFT_Application-Overlap_Save}
This is another way to perform a \nameref{def:Linear_Convolution} on an infinitely long discretely-valued signal with a filter $h(n)$ that has finite length of $M$.

The steps to perform \nameref{subsubsec:DFT_Application-Overlap_Save} are:
\begin{enumerate}[noitemsep]
\item Partition your infinite discrete input signal, $x(n)$, into blocks of length $L$ to get $x_{a}(n)$
  \begin{itemize}[noitemsep]
  \item This will give you $A$ block, where $0 \leq A < \infty$.
  \item This means you have blocks with sequences $x_{a}(n), \: a \in A$ that are of length $L$.
  \end{itemize}
\item Compute the \nameref{def:DFT} of your filter $h(n)$ to get $H(k)$.
  \begin{enumerate}[noitemsep]
  \item Perform the \nameref{def:DFT} on the sequence
  \item You can disregard $h(n)$ now
    \begin{itemize}[noitemsep]
    \item If you're programming this, this means you can precalculate $H(k)$ somewhere else.
    \end{itemize}
  \end{enumerate}
\item Compute the \nameref{def:DFT} of one of the blocks from the partitioned input signal to get $X_{a}(k)$
  \begin{enumerate}[noitemsep]
  \item Zero-pad the \textbf{\emph{FRONT}} of the block until the length of the block's sequences with the padding is $N=L+M-1$.
  \end{enumerate}
\item Perform the \nameref{def:Circular_Convolution} of $h(n)$ and zero-padded block signal $x_{a}(n)$ by multiplying $X_{a}(k)$ and $H(k)$, which yields $Y_{a}(k)$
\item Make your output $Z_{a}(k)$ the last $M-1$ terms in the $Y_{a}(k)$ sequences. (Leave out the first $L$ terms).
\item Take the \nameref{def:IDFT} of $Z_{a}(k)$ to get $z_{a}(n)$.
\item Move onto the $(a+1)$th block, and prepend (attach to the front) the last $L-M$ terms from $x_{a}(n)$, to $x_{a+1}(n)$.
\item Perform the \nameref{def:DFT} of this sequence, and take the last $M-1$ terms from $Y_{a+1}(k)$ to the output $Z(k)$
\end{enumerate}

\subsubsection{Overlap-Discard}\label{subsubsec:DFT_Application-Overlap_Discard}

%%% Local Variables:
%%% mode: latex
%%% TeX-master: "../EITF75-Systems_and_Signals-Reference_Sheet"
%%% End:
