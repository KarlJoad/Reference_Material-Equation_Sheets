\section{The $Z$-Transform}\label{sec:Z-Transform}
The \nameref{def:Z-Transform} plays the same role in the analysis of discrete-time signals and LTI systems as the \nameref{def:Laplace Transform} does in the analysis of continuous time-signals and LTI systems.

\subsection{The $Z$-Transform}\label{subsec:Z-Transform}
\begin{definition}[$Z$-Transform]\label{def:Z-Transform}
  The \emph{$z$-transform} is defined as the power series
  \begin{equation}\label{eq:Z-Transform}
    X(z) \equiv \sum_{n=-\infty}^{\infty} x(n)z^{-n}
  \end{equation}
\end{definition}

\subsubsection{The One-Sided $Z$-Transform}\label{subsubsec:One-Sided Z-Transform}
\begin{definition}[One-Sided $Z$-Transform]\label{def:One-Sided Z-Transform}
  The \emph{one-sided $z$-transform} is the same as the \nameref{def:Z-Transform}, but is only defined at $n$ values greater than or equal to 0.
  \begin{equation}\label{eq:One-Sided Z-Transform}
    X(z) \equiv \sum_{n=0}^{\infty} x(n)z^{-n}
  \end{equation}
\end{definition}

The \nameref{def:One-Sided Z-Transform} is generally used when there are initial conditions on a causal signal.
This captures the normal causal portion of the signal, while also showing the effect of the initial condition.

\subsection{The Inverse $Z$-Transform}\label{subsec:Inverse Z-Transform}
This is the formal definition of \nameref{subsec:Inverse Z-Transform}.
\begin{equation}\label{eq:Inverse Z-Transform}
  x(n) = \frac{1}{2 \pi j} \oint_{C} X(z) z^{n-1} \, dz
\end{equation}
where the integrals is a contour integral over a closed path C that encloses the origin and lies within the region of convergence of $X(z)$.

There are 3 methods that are often used for the evaluation of the inverse $z$-transform in practice:
\begin{enumerate}[noitemsep]
\item Direct evaluation of~\eqref{eq:Inverse Z-Transform}.
\item Expansion into a series of terms, in the variable s$z$ and $z^{-1}$.
\item Partial-fraction expansion and table lookup.
\end{enumerate}

\subsubsection{The Inverse $Z$-Transform by Contour Integration}\label{subsubsec:Inverse Z-Transform by Contour Integration}
\begin{definition}[Cauchy's Integral Theorem]\label{def:Cauchy Integral Theorem}
  Let $f(z)$ be a function of the complex variable $z$ and $C$ be a closed path in the $z$-plane.
  If the derivative $\frac{\mathrm{d}f(z)}{\mathrm{d}z}$ exists on and inside the contour $C$ and if $f(z)$ has no poles at $z = z_{0}$, then
  \begin{equation}\label{eq:Cauchy Integral Theorem Specfic}
    \frac{1}{2 \pi j} \oint_{C} \frac{f(z)}{z-z_{0}} \, \mathrm{d}z = \begin{cases}
      f(z_{0}), & \text{ if $z_{0}$ is inside $C$} \\
      0, & \text{ if $z_{0}$ is outside $C$}
    \end{cases}
  \end{equation}

  More generally, if the $(k+1)$-order derivative of $f(z)$ exists and $f(z)$ has no poles at $z = z_{0}$, then
  \begin{equation}\label{eq:Cauchy Integral THeorem General}
    \frac{1}{2 \pi j} \oint_{C} \frac{f(z)}{\left( z-z_{0} \right)^{k}} \, \mathrm{d}z = \begin{cases}
      \frac{1}{\left( k-1 \right)!} \frac{\mathrm{d}^{k-1}f(z)}{\mathrm{d}z^{k-1}} \bigg \vert_{z=z_{0}}, & \text{ if $z_{0}$ is inside $C$} \\
      0, & \text{ if $z_{0}$ is outside $C$}
    \end{cases}
  \end{equation}
\end{definition}


\subsubsection{The Inverse $Z$-Transform by Power Series Expansion}\label{subsubsec:Inverse Z-Transform by Power Series Expansion}
\subsubsection{The Inverse $Z$-Transform by Partial-Fraction Expansion}\label{subsubsec:Inverse Z-Transform by Partial-Fraction Expansion}
\subsection{Properties of the $Z$-Transform}\label{subsec:Z-Transform Properties}
\begin{propertylist}
\item 
\end{propertylist}

\subsubsection{Properties of the One-Sided $Z$-Transform}\label{subsubsec:One-Sided Z-Transform Properties}
\begin{propertylist}
\item 
\end{propertylist}

\subsection{Rational $Z$-Transforms}\label{subsec:Rational Z-Transforms}
\subsubsection{Decomposition of Rational $Z$-Transforms}\label{subsubsec:Decompose Rational Z-Transforms}

\subsection{Analysis of LTI Systems in the $Z$-Domain}\label{subsec:Analysis LTI Systems Z-Domain}

%%% Local Variables:
%%% mode: latex
%%% TeX-master: "../EITF75-Systems_and_Signals-Reference_Sheet"
%%% End:
