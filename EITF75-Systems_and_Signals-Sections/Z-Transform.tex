\section{The \texorpdfstring{$\ZTransform$-Transform}{Z-Transform}}\label{sec:Z-Transform}
The \nameref{def:Z-Transform} plays the same role in the analysis of \nameref{def:Discrete-Time Signals} and LTI systems as the \nameref{def:Laplace_Transform} does in the analysis of \nameref{def:Continuous-Time Signals} and LTI systems.

\subsection{The \texorpdfstring{$\ZTransform$-Transform}{Z-Transform}}\label{subsec:Z-Transform}
\begin{definition}[$\ZTransform$-Transform]\label{def:Z-Transform}
  The \emph{$z$-transform} is defined as the power series
  \begin{equation}\label{eq:Z-Transform}
    X(z) \equiv \sum_{n=-\infty}^{\infty} x(n)z^{-n}
  \end{equation}

  \begin{remark}
    For convenience, the $z$-transform of a signal $x(n)$ is denoted by
    \begin{equation}\label{eq:Z-Transform Representation}
      X(z) \equiv \ZTransform \lbrace x(n) \rbrace
    \end{equation}
    and the relationship between $x(n)$ and $X(z)$ is indicated by
    \begin{equation}\label{eq:Z-Transform Relationship}
      x(n) \ZTransformRelation X(z)
    \end{equation}
  \end{remark}
\end{definition}

\subsubsection{Region of Convergence}\label{subsubsec:ROC}
\begin{definition}[ROC]\label{def:ROC}
  The \emph{ROC} or \emph{region of convergence} is the region for which the infinite power series in the $z$-transform has a convergent solution.
  \begin{remark}
    \textbf{\emph{Any time we cite a $z$-transform, we should also indicate its \nameref{def:ROC}}}
  \end{remark}
\end{definition}

\begin{example}[]{Simple $Z$-Transform}
  Determine the $z$-transform of the signal
  \begin{equation*}
    x(n) = \left( \frac{1}{2} \right)^{n} \UnitStep(n)
  \end{equation*}

  \tcblower

  The $z$-transform is the infinite power series
  \begin{equation*}
    \begin{aligned}
      X(z) &= 1 + \frac{1}{2}z^{-1} + {\left( \frac{1}{2} \right)}^{-2} + \cdots + {\left( \frac{1}{2} \right)}^{n}z^{-n} + \cdots \\
      &= \sum_{n=0}^{\infty} {\left( \frac{1}{2} \right)}^{n}z^{-n} = \sum_{n=0}^{\infty} {\left( \frac{1}{2}z^{-1} \right)}^{n}
    \end{aligned}
  \end{equation*}
  Because this is an infinite geometric series, we can solve with with our equivalency:
  \begin{equation*}
    1 + A + A^{2} + \cdots + A^{n} + \cdots = \frac{1}{1-A} \> \text{if } \lvert A \rvert < 1
  \end{equation*}

  Thus, $X(z)$ converges to
  \begin{equation*}
    X(z) = \frac{1}{1-\frac{1}{2}z^{-1}}, \quad \ROC: \lvert z \rvert > \frac{1}{2}
  \end{equation*}
\end{example}

\begin{table}[h!]
  \centering
  \begin{tabular}{cc}
    \toprule
    Signal & ROC \\
    \midrule
    \multicolumn{2}{c}{Finite-Duration Signals} \\
    \midrule
    Causal & Entire $z$-plane except $z=0$ \\
    Anticausal & Entire $z$-plane except $z=\infty$ \\
    Two-Sided & Entire $z$-plane except $z=0$ and $z=\infty$ \\
    \midrule
    \multicolumn{2}{c}{Infinite-Duration Signals} \\
    \midrule
    Causal & $\lvert z \rvert > r_{2}$ \\
    Anticausal & $\lvert z \rvert < r_{1}$ \\
    Two-Sided & $r_{2} < \lvert z \rvert < r_{1}$ \\
    \bottomrule
  \end{tabular}
  \caption{Characteristic Familes of Signals with Their Corresponding ROCs}
  \label{tab:Z-Transform ROC Correspondence}
\end{table}

\subsubsection{The One-Sided \texorpdfstring{$\ZTransform$-Transform}{Z-Transform}}\label{subsubsec:One-Sided Z-Transform}
\begin{definition}[One-Sided $\ZTransform$-Transform]\label{def:One-Sided Z-Transform}
  The \emph{one-sided $z$-transform} is the same as the \nameref{def:Z-Transform}, but is only defined at $n$ values greater than or equal to 0.
  \begin{equation}\label{eq:One-Sided Z-Transform}
    X(z) \equiv \sum_{n=0}^{\infty} x(n)z^{-n}
  \end{equation}
\end{definition}

The \nameref{def:One-Sided Z-Transform} is generally used when there are initial conditions on a causal signal.
This captures the normal causal portion of the signal, while also showing the effect of the initial condition.

\subsection{The Inverse \texorpdfstring{$\ZTransform$-Transform}{Z-Transform}}\label{subsec:Inverse Z-Transform}
This is the formal definition of \nameref{subsec:Inverse Z-Transform}.
\begin{equation}\label{eq:Inverse Z-Transform}
  x(n) = \frac{1}{2 \pi j} \oint_{C} X(z) z^{n-1} \, dz
\end{equation}
where the integrals is a contour integral over a closed path C that encloses the origin and lies within the region of convergence of $X(z)$.

There are 3 methods that are often used for the evaluation of the inverse $z$-transform in practice:
\begin{enumerate}[noitemsep]
\item Direct evaluation of~\eqref{eq:Inverse Z-Transform}.
\item Expansion into a series of terms, in the variable s$z$ and $z^{-1}$.
\item Partial-fraction expansion and table lookup.
\end{enumerate}

\subsubsection{The Inverse \texorpdfstring{$\ZTransform$-Transform}{Z-Transform} by Contour Integration}\label{subsubsec:Inverse Z-Transform by Contour Integration}
\begin{definition}[Cauchy's Integral Theorem]\label{def:Cauchy Integral Theorem}
  Let $f(z)$ be a function of the complex variable $z$ and $C$ be a closed path in the $z$-plane.
  If the derivative $\frac{\mathrm{d}f(z)}{\mathrm{d}z}$ exists on and inside the contour $C$ and if $f(z)$ has no poles at $z = z_{0}$, then
  \begin{equation}\label{eq:Cauchy Integral Theorem Specfic}
    \frac{1}{2 \pi j} \oint_{C} \frac{f(z)}{z-z_{0}} \, \mathrm{d}z = \begin{cases}
      f(z_{0}), & \text{ if $z_{0}$ is inside $C$} \\
      0, & \text{ if $z_{0}$ is outside $C$}
    \end{cases}
  \end{equation}

  More generally, if the $(k+1)$-order derivative of $f(z)$ exists and $f(z)$ has no poles at $z = z_{0}$, then
  \begin{equation}\label{eq:Cauchy Integral THeorem General}
    \frac{1}{2 \pi j} \oint_{C} \frac{f(z)}{\left( z-z_{0} \right)^{k}} \, \mathrm{d}z = \begin{cases}
      \frac{1}{\left( k-1 \right)!} \frac{\mathrm{d}^{k-1}f(z)}{\mathrm{d}z^{k-1}} \bigg \vert_{z=z_{0}}, & \text{ if $z_{0}$ is inside $C$} \\
      0, & \text{ if $z_{0}$ is outside $C$}
    \end{cases}
  \end{equation}
\end{definition}


\subsubsection{The Inverse \texorpdfstring{$\ZTransform$-Transform}{Z-Transform} by Power Series Expansion}\label{subsubsec:Inverse Z-Transform by Power Series Expansion}
\subsubsection{The Inverse \texorpdfstring{$\ZTransform$-Transform}{Z-Transform} by Partial-Fraction Expansion}\label{subsubsec:Inverse Z-Transform by Partial-Fraction Expansion}
\subsection{Properties of the \texorpdfstring{$\ZTransform$-Transform}{Z-Transform}}\label{subsec:Z-Transform Properties}
\begin{table}[h!]
  \centering
  \begin{tabular}{p{4cm}ccp{5cm}}
    \toprule
    Property & Time Domain & $z$-Domain & $\ROC$ \\
    \midrule
    \multirow{3}{*}{Notation} & $x(n)$ & $X(z)$ & $\ROC: r_{2} < \lvert z \rvert < r_{1}$ \\
             & $x_{1}(n)$ & $X_{1}(z)$ & $\ROC_{1}$ \\
             & $x_{2}(n)$ & $X_{2}(z)$ & $\ROC_{2}$ \\
    \nameref{subsubsec:Z-Transform Linearity} & $a_{1}x_{1}(n) + a_{2}x_{2}(n)$ & $a_{1}X_{1}(z) + a_{2}X_{2}(z)$ & At least the intersection of $\ROC_{1}$ and $\ROC_{2}$ \\
    \nameref{subsubsec:Z-Transform Time Shifting} & $x(n-k)$ & $z^{-k}X(z)$ & That of $X(z)$, except $z=0$ if $k>0$ and $z=\infty$ if $k<0$ \\
    \nameref{subsubsec:Z-Domain Scaling} & $a^{n}x(n)$ & $X(a^{-1}z)$ & $\lvert a \rvert r_{2} < \lvert z \rvert < \lvert a \rvert r_{1}$ \\
    \nameref{subsubsec:Z-Transform Time Reversal} & $x(-n)$ & $X(z^{-1})$ & $\frac{1}{r_{1}} < \lvert z \rvert < \frac{1}{r_{2}}$ \\
    Conjugation & $x^{*}(n)$ & $X^{*}(z^{*})$ & $\ROC$ \\
    Real Part & $\Re \lbrace x(n) \rbrace$ & $\frac{1}{2} \left[ X(z) + X^{*}(z^{*}) \right]$ & Includes $\ROC$ \\p
    Imaginary Part & $\Im \lbrace x(n) \rbrace$ & $\frac{1}{2} j \left[ X(z) - X^{*}(z^{*}) \right]$ & Includes $\ROC$ \\
    \nameref{subsubsec:Z-Domain Differentiation} & $nx(n)$ & $-z \frac{dX(z)}{dz}$ & $r_{2} < \lvert z \rvert r_{1}$ \\
    \nameref{subsubsec:Z-Domain Convolutions} & $x_{1} * x_{2}$ & $X_{1}(z)X_{2}(z)$ & At least, the intersection of $\ROC_{1}$ and $\ROC_{2}$ \\
    \nameref{subsubsec:Z-Transform 2 Sequence Correlation} & $r_{x_{1}x_{2}}(l) = x_{1}(l) * x_{2}(-l)$ & $R_{x_{1}x_{2}}(z) = X_{1}(z)x_{2}(z^{-1})$ & At least, the intersection of $\ROC$ of $X_{1}(z)$ and $X_{2}(z^{-1})$ \\
    \nameref{subsubsec:Initial Value Theorem for Z-Transform} & If $x(n)$ causal & $x(0) = \lim\limits_{z \rightarrow \infty} X(z)$ & \\
    \nameref{subsubsec:Z-Transform 2 Sequence Multiplication} & $x_{1}(n)x_{2}(n)$ & $\frac{1}{2 \pi j} \oint_{C} X_{1}(v)X_{2}(\frac{z}{v}) v^{-1} dv$ & At least, $r_{1l}r_{2l} < \lvert a \rvert < r_{1u}r_{2u}$ \\
    \nameref{subsubsec:Parsevals Relation for Z-Transform} & $\sum\limits_{n=-\infty}^{\infty} x_{1}(n)x_{2}^{*}(n)$ &= $\frac{1}{2 \pi j} \oint_{C} X_{1}(v)X_{2}^{*}(\frac{1}{v^{*}})v^{-1} dv$ & \\
    \bottomrule
  \end{tabular}
  \caption{\nameref{subsec:Z-Transform Properties}}
  \label{tab:Z-Transform Properties}
\end{table}

\subsubsection{\texorpdfstring{$\ZTransform$-Transform}{Z-Transform} Linearity}\label{subsubsec:Z-Transform Linearity}
If
\begin{equation*}
  \begin{aligned}
    x_{1}(n) &\ZTransformRelation X_{1}(z) \\
    x_{2}(n) &\ZTransformRelation X_{2}(z) \\
  \end{aligned}
\end{equation*}
then
\begin{equation}\label{eq:Z-Transform Linearity}
  x(n) = a_{1}x_{1}(n) + a_{2}x_{2}(n) \ZTransformRelation X(z) = a_{1}X_{1}(z) + a_{2}X_{2}(z)
\end{equation}
for any constants $a_{1}$ and $a_{2}$.

The linearity property can be generalized to an arbitrary number of signals.

\begin{example}[Example 3.2.1]{Simple \texorpdfstring{$Z$-Transform}{Z-Transform} Linearity Problem}
  Deermine the $z$-transform and the ROC of the signal
  \begin{equation*}
    x(n) = \left[ 3 \left( 2^{n} \right) - 4 \left( 3^{n} \right) \right] \UnitStep(n)
  \end{equation*}

  \tcblower

  Solution on Page 158.
\end{example}

\begin{example}[Example 3.2.2]{\texorpdfstring{$Z$-Transform}{Z-Transform} Linearity on Trig Functions}
  Determine the $z$-transform of the signals
  \begin{boldalphlist}
  \item $x(n) = \left( \cos \omega_{0}n \right) \UnitStep(n)$
  \item $x(n) = \left( \sin \omega_{0}n \right) \UnitStep(n)$
  \end{boldalphlist}

  \tcblower

  Solution on Pages 158-159.
\end{example}

\subsubsection{\texorpdfstring{$\ZTransform$-Transform}{Z-Transform} Time Shifting}\label{subsubsec:Z-Transform Time Shifting}
If
\begin{equation*}
  x(n) \ZTransformRelation X(z)
\end{equation*}
then
\begin{equation}\label{eq:Z-Transform Time Shifting}
  x(n-k) \ZTransformRelation z^{-k}X(z)
\end{equation}
The ROC of $z^{-k}X(z)$ is the same as that of $X(z)$ except for $z=0$ if $k > 0$ and $z = \infty$ if $k < 0$.

\begin{example}[Example 3.2.3]{\texorpdfstring{$Z$-Transform}{Z-Transform} Time Shifting}
  By applying the time-shifting property, determine the $z$-transform of the signals
  \begin{equation*}
    \begin{aligned}
      x_{1}(n) &= \lbrace 1, 2, \underline{5}, 7, 0, 1 \rbrace \\
      x_{2}(n) &= \lbrace \underline{0}, 0, 1, 2, 5, 7, 0, 1 \rbrace \\
    \end{aligned}
  \end{equation*}
  from the $z$-transform of
  \begin{equation*}
    \begin{aligned}
      x_{0}(n) &= \lbrace \underline{1}, 2, 5, 7, 0, 1 \rbrace \\
        X_{0}(z) &= 1 + 2z^{-1} + 5z^{-2} + 7z^{-3} + z^{-5}, \ROC: \text{entire $z$-plane except } z=0 \\
      \end{aligned}
    \end{equation*}

    \tcblower

    Solution on Page 160.
\end{example}

\subsubsection{\texorpdfstring{$\ZTransform$-Domain}{Z-Domain} Scaling}\label{subsubsec:Z-Domain Scaling}
If
\begin{equation*}
  x(n) \ZTransformRelation X(z), \> \ROC: r_{1}< \lvert z \rvert < r_{2}
\end{equation*}
then
\begin{equation}\label{eq:Z-Domain Scaling}
  a^{n}x(n) \ZTransformRelation X \left( a^{-1}z \right), \> \ROC: \lvert a \rvert r_{1} < \lvert z \rvert < \lvert a \rvert r_{2}
\end{equation}

\subsubsection{\texorpdfstring{$\ZTransform$-Transform}{Z-Transform} Time Reversal}\label{subsubsec:Z-Transform Time Reversal}
If
\begin{equation*}
  x(n) \ZTransformRelation X(z), \> \ROC: r_{1}< \lvert z \rvert < r_{2}
\end{equation*}
then
\begin{equation}\label{eq:Z-Transform Time Reversal}
  x(-n) \ZTransformRelation X(z^{-1}), \ROC: \frac{1}{r_{2}} < \lvert z \rvert < \frac{1}{r_{1}}
\end{equation}

\begin{example}[Example 3.2.6]{\texorpdfstring{$Z$-Transform}{Z-Transform} Time Reversal}
  Determine the $z$-transform of the signal
  \begin{equation*}
    x(n) = \UnitStep(-n)
  \end{equation*}

  \tcblower

  The transform for $\UnitStep(n)$ is given in \Cref{tab:Common Z-Transforms}.
  \begin{equation*}
    \UnitStep(n) \ZTransformRelation \frac{1}{1-x^{-1}}, \> \ROC: \lvert z \rvert > 1
  \end{equation*}
  By using~\eqref{eq:Z-Transform Time Reversal}, we obtain
  \begin{equation*}
    \UnitStep(-n) \ZTransformRelation \frac{1}{1-z}, \ROC: \lvert z \rvert < 1
  \end{equation*}
\end{example}

\subsubsection{\texorpdfstring{$\ZTransform$-Domain}{Z-Domain} Differentiation}\label{subsubsec:Z-Domain Differentiation}
If
\begin{equation*}
  x(n) \ZTransformRelation X(z)
\end{equation*}
then
\begin{equation}\label{eq:Z-Domain Differentiation}
  nx(n) \ZTransformRelation -z \frac{dX(z)}{dz}
\end{equation}

\begin{example}[Example 3.2.7]{\texorpdfstring{$Z$-Domain}{Z-Domain} Differentiation}
  Determine the $z$-transform of the signal
  \begin{equation*}
    x(n) = na^{n}\UnitStep(n)
  \end{equation*}

  \tcblower

  The signal $x(n)$ can be expressed as $nx_{1}(n)$, where $x_{1}(n) = a^{n}\UnitStep(n)$.
  By passing this through the $z$-transform, we have
  \begin{equation*}
    x_{1}(n) = a^{n}\UnitStep(n) \ZTransformRelation X_{1}(z) = \frac{1}{1-az^{-1}}, \> \ROC: \lvert z \rvert > \lvert a \rvert
  \end{equation*}
  Then by using~\eqref{eq:Z-Domain Differentiation}, we obtain
  \begin{equation*}
    na^{n} \UnitStep(n) \ZTransformRelation X(z) = -z \frac{dX_{1}(z)}{dz} = \frac{az^{-1}}{{\left( 1-az^{-1}\right)}^{2}}
  \end{equation*}
\end{example}

\subsubsection{\texorpdfstring{$\ZTransform$-Domain}{Z-Domain} Convolutions}\label{subsubsec:Z-Domain Convolutions}
If
\begin{equation*}
  \begin{aligned}
    x_{1}(n) &\ZTransformRelation X_{1}(z) \\
    x_{2}(n) &\ZTransformRelation X_{2}(z) \\
  \end{aligned}
\end{equation*}
then
\begin{equation}\label{eq:Z-Domain Convolutions}
  x(n) = x_{1}(n) * x_{2}(n) \ZTransformRelation X(z) = X_{1}(z)X_{2}(z)
\end{equation}
The ROC of $X(z)$ is, at least, the intersection of that for $X_{1}(z)$ and $X_{2}(z)$.

\begin{example}[Example 3.2.9]{\texorpdfstring{$Z$-Domain}{Z-Domain} Convolutions}
  Compute the convolution $x(n)$ of the signals
  \begin{align*}
    x_{1}(n) &= \lbrace 1, -2, 1 \rbrace \\
    x_{2}(n) &= \begin{cases}
      1, & 0 \leq n \leq 6 \\
      0, & \text{elsewhere}
    \end{cases}
  \end{align*}
  When
  \begin{align*}
    X_{1}(z) &= 1-2z^{-1}+z^{-2} \\
    X_{2}(z) &= 1 + z^{-1} + z^{-2} + z^{-3} + z^{-4} + z^{-5}
  \end{align*}

  \tcblower

  According to~\eqref{eq:Z-Domain Convolutions} we carry out the multiplication of $X_{1}(z)$ and $X_{2}(z)$.
  Thus
  \begin{equation*}
    X(z) = X_{1}(z)X_{2}(z) = 1-z^{-1}-z^{-6}+z^{-7}
  \end{equation*}
  Hence
  \begin{equation*}
    x(n) = \lbrace \underline{1}, -1, 0, 0, 0, 0, -1, 1 \rbrace
  \end{equation*}
\end{example}

\subsubsection{\texorpdfstring{$\ZTransform$-Transform}{Z-Transform} 2 Sequence Correlation}\label{subsubsec:Z-Transform 2 Sequence Correlation}
If
\begin{equation*}
  \begin{aligned}
    x_{1}(n) &\ZTransformRelation X_{1}(z) \\
    x_{2}(n) &\ZTransformRelation X_{2}(z) \\
  \end{aligned}
\end{equation*}
then
\begin{equation}\label{eq:Z-Transform 2 Sequence Correlation}
  r_{x_{1}x_{2}}(l) = \sum\limits_{n=-\infty}^{\infty} x_{1}(n)x_{2}(n-l) \ZTransformRelation R_{x_{1}x_{2}}(z) = X_{1}(z)X_{2}(z^{-1})
\end{equation}

\begin{example}[Example 3.2.10]{\texorpdfstring{$Z$-Transform}{Z-Transform} 2 Sequence Correlation}
  Determine the autocorrelation of the signal
  \begin{equation*}
    x(n) = a^{n}\UnitStep(n), \> -1<a<1
  \end{equation*}

  \tcblower

  Solution on Page 166.
\end{example}

\subsubsection{\texorpdfstring{$\ZTransform$-Transform}{Z-Transform} 2 Sequence Multiplication}\label{subsubsec:Z-Transform 2 Sequence Multiplication}
If
\begin{equation*}
  \begin{aligned}
    x_{1}(n) &\ZTransformRelation X_{1}(z) \\
    x_{2}(n) &\ZTransformRelation X_{2}(z) \\
  \end{aligned}
\end{equation*}
then
\begin{equation}\label{eq:Z-Transform 2 Sequence Multiplication}
  x(n) = x_{1}(n)x_{2}(n) \ZTransformRelation X_{z} = \frac{1}{2 \pi j} \oint_{C} X_{1}(v)X_{2}\left( \frac{z}{v} \right) v^{-1} dv
\end{equation}
where $C$ is a closed contour that encloses the origin and lies within th eregion of convergence common to both $X_{1}(v)$ and $X_{2}(\frac{1}{v})$.

\subsubsection{Parsevals Relation for \texorpdfstring{$\ZTransform$-Transform}{Z-Transform}}\label{subsubsec:Parsevals Relation for Z-Transform}
If $x_{1}(n)$ and $x_{2}(n)$ are complex-valued sequences, then
\begin{equation}\label{eq:Parsevals Relation for Z-Transform}
  \sum\limits_{n=-\infty}^{\infty} x_{1}(n)x_{2}^{*}(n) = \frac{1}{2 \pi j} \oint_{C} X_{1}(v)X_{2}^{*} \left( \frac{1}{v^{*}} \right) v^{-1} dv
\end{equation}

\subsubsection{Initial Value Theorem for \texorpdfstring{$\ZTransform$-Transform}{Z-Transform}}\label{subsubsec:Initial Value Theorem for Z-Transform}
If $x(n)$ is \emph{causal} [i.e., $x(n)=0$ for $n<0$], then
\begin{equation}\label{eq:Initial Value Theorem for Z-Transform}
  x(0) = \lim\limits_{z\rightarrow\infty}X(z)
\end{equation}

\subsection{Properties of the One-Sided \texorpdfstring{$\ZTransform$-Transform}{Z-Transform}}\label{subsec:One-Sided Z-Transform Properties}
\begin{propertylist}
\item 
\end{propertylist}

\subsection{Rational \texorpdfstring{$\ZTransform$-Transforms}{Z-Transform}}\label{subsec:Rational Z-Transforms}
An important family of $z$-transforms are those for which $X(z)$ is a rational function, a ratio of two polynomials in $z^{-1}$ (or $z$).

\subsubsection{Poles and Zeros of a \texorpdfstring{$\ZTransform$-Transform}{Z-Transform}}\label{subsubsec:Poles and Zeros of Z-Transform}
\begin{definition}[Zeros]\label{def:Zeros of a Z-Transform}
  The \emph{zeros} of a $z$-transform $X(z)$ are the values of $z$ for which $X(z)=0$.

  This is analogous to ``setting the numerator equal to zero.''
\end{definition}
\begin{definition}[Poles]\label{def:Poles of a Z-Transform}
  The \emph{poles} of a $z$ transform $X(z)$ are the values of $z$ for which $X(z)=\infty$.

  This is analogous to ``setting the denominator equal to zero.''
\end{definition}

If $X(z)$ is a rational function, then
\begin{equation*}
  X(z) = \frac{B(z)}{A(z)} = \frac{b_{0} + b_{1}z^{-1} + \cdots + b_{M}z^{-M}}{a_{0} + a_{1}z^{-1} + \cdots + a_{N}z^{-N}} = \frac{\sum_{k=0}^{M} b_{k}z^{-k}}{\sum_{k=0}^{N} a_{k}z^{-k}}
\end{equation*}
If $a_{0} \neq 0$ and $b_{0} \neq 0$, we can avoid negative powers of $z$ by factoring out the terms $z^{-M}$ and $z^{-N}$.
\begin{equation*}
  X(z) = \frac{B(z)}{A(z)} = \frac{z^{-M}}{z^{-N}} \frac{b_{0}z^{M} + b_{1}z^{M-1} + \cdots + b_{M}}{a_{0}z^{N} + a_{1}z^{N-1} + \cdots + a_{N}}
\end{equation*}
Since $B(z)$ and $A(z)$ are polynomials in $z$, they can be expressed in factored form as
\begin{equation}\label{eq:Rational Z-Transform Poles and Zeros Factored}
  X(z) = \frac{B(z)}{A(z)} = \frac{z^{-M}}{z^{-N}} \frac{(z-z_{1})(z-z_{2}) \cdots (z-z_{M})}{(z-p_{1})(z-p_{2}) \cdots (z-p_{N})}
\end{equation}
Thus, $X(z)$ has $M$ finite \nameref{def:Zeros of a Z-Transform} at $z=z_{1},z_{2}, \ldots, z_{M}$ (the roots of the numerator polynomial), $N$ finite \nameref{def:Poles of a Z-Transform} at $z=p_{1},p_{2}, \ldots, p_{N}$ (the roots of the denominator polynomial, and $\lvert N-M \rvert$ zeros (if $N>M$) or poles (if $N<M$) at the origin $z=0$.
Poles and zeroes may occur at $z=\infty$.
A zero exists at $z=\infty$ if $X(\infty) = 0$ and a pole exists at $z=\infty$ if $X(\infty) = \infty$.

\begin{definition}[Pole-Zero Plot]\label{def:Pole-Zero Plot of Z-Transform}
  Poles and Zeros of a $z$-transform can be shown graphically by a \emph{pole-zero plot} in the complex plane, which shows the location of poles by crosses ($\times$) and the location of zeros by circles ($\circ$).
  Multiplicity is shown by a number close to the corresponding cross or circle.
  The ROC of a $z$-transform should not contain any poles, by definition.
\end{definition}

\subsubsection{Decomposition of Rational \texorpdfstring{$\ZTransform$-Transforms}{Z-Transform}}\label{subsubsec:Decompose Rational Z-Transforms}
The short end of this story is that you should group complex-conjugate pairs together.

\subsection{Analysis of LTI Systems in the \texorpdfstring{$\ZTransform$-Domain}{Z-Domain}}\label{subsec:Analysis LTI Systems Z-Domain}
There are a few steps to move from the time-domain to the $\ZTransform$-domain to perform analysis.
\begin{enumerate}[noitemsep]
\item Convert all your time-based to terms to the $\ZTransform$-domain.
  \begin{itemize}[noitemsep]
  \item $y(n-k) \rightarrow z^{-k}Y(z)$
  \item $x(n-k) \rightarrow z^{-k}X(z)$
  \end{itemize}
\item Express $H(z)$ as $\frac{Y(z)}{X(z)}$, the \nameref{def:System_Function}.
\item Find the roots of the numerator and the denominator.
  \begin{itemize}[noitemsep]
  \item When solving for the roots, you should solve in terms of $z^{k}$, not $z^{-k}$. Factor our $z^{-k}$ to achieve this.
  \item If the degree of the numerator is greater than or equal to the degree of the numerators, you have to reduce the degree of the numerator.
    \begin{enumerate}[noitemsep]
    \item Use Long Polynomial Division
    \item Use Partial-Fraction Expansion on the Remainder
    \end{enumerate}
  \end{itemize}
  \begin{enumerate}[noitemsep]
  \item The roots of the numerator is/are \nameref{def:Zeros of a Z-Transform}. This is where the $\ZTransform$-plane becomes 0.
  \item The roots of the denominator is/are \nameref{def:Poles of a Z-Transform}. This is where the $\ZTransform$-plane tends towards $\infty$.
  \end{enumerate}
\end{enumerate}

\begin{definition}[System Function]\label{def:System_Function}
  The \emph{system function} or \emph{system equation} is the $\ZTransform$-transform of the filter response.

  \begin{equation}\label{eq:System_Function}
     H(z) = \ZTransform \lbrace h(n) \rbrace
   \end{equation}

   Because of~\Cref{eq:Z-Domain Convolutions}, and the relation shown in~\Cref{eq:LTI_System_Convolution}, we can write the system equation like so
   \begin{equation}\label{eq:System_Function-LTI_System}
     \begin{aligned}
       Y(z) &= X(z) H(z) \\
       H(z) &= \frac{Y(z)}{X(z)} \\
     \end{aligned}
   \end{equation}

   This forms the basis for \nameref{subsec:Rational Z-Transforms} and \nameref{subsec:Analysis LTI Systems Z-Domain}
\end{definition}

\subsection{Common \texorpdfstring{$\ZTransform$-Transforms}{Z-Transforms}}\label{subsubsec:Common Z-Transforms}
The most common $\ZTransform$-transforms are shown in~\Cref{tab:Common Z-Transforms}.

\begin{table}[h!]
  \centering
  \renewcommand{\arraystretch}{1.4}
  \begin{tabular}{ccc}
    \toprule
    Signal, $x(n)$ & $z$-Transform, $X(z)$ & $\ROC$ \\
    \midrule
    $\UnitImpulse(n)$ & 1 & All $z$ \\
    $\UnitStep(n)$ & $\frac{1}{1-z^{-1}}$ & $\lvert z \rvert > 1$ \\
    $a^{n}\UnitStep(n)$ & $\frac{1}{1-az^{-1}}$ & $\lvert z \rvert > \lvert a \rvert$ \\
    $na^{n}\UnitStep(n)$ & $\frac{az^{-1}}{{\left( 1-az^{-1} \right)}^{2}}$ & $\lvert z \rvert > \lvert a \rvert$ \\
    $-a^{n}\UnitStep(-n-1)$ & $\frac{1}{1-az^{-1}}$ & $\lvert z \rvert < \lvert a \rvert$ \\
    $-na^{n}\UnitStep(-n-1)$ & $\frac{az^{-z}}{{\left( 1-az^{-1} \right)}^{2}}$ & $\lvert z \rvert < \lvert a \rvert$ \\
    $\left( \cos \omega_{0}n \right) \UnitStep(n)$ & $\frac{1-z^{-1}\cos \omega_{0}}{1-2z^{-1}\cos \omega_{0} + z^{-2}}$ & $\lvert z \rvert > 1$ \\
    $\left( \sin \omega_{0}n \right) \UnitStep(n)$ & $\frac{z^{-1}\sin \omega_{0}}{1-2z^{-1}\cos \omega_{0} + z^{-2}}$ & $\lvert z \rvert > 1$ \\
    $\left( a^{n} \cos \omega_{0}n \right) \UnitStep(n)$ & $\frac{1-az^{-1}\cos \omega_{0}}{1-2az^{-1} \cos \omega_{0} + a^{2}z^{-2}}$ & $\lvert z \rvert > \lvert a \rvert$ \\
        $\left( a^{n} \sin \omega_{0}n \right) \UnitStep(n)$ & $\frac{az^{-1}\sin \omega_{0}}{1-2az^{-1} \cos \omega_{0} + a^{2}z^{-2}}$ & $\lvert z \rvert > \lvert a \rvert$ \\
    \bottomrule
  \end{tabular}
  \caption{Common $\ZTransform$-Transforms}
  \label{tab:Common Z-Transforms}
\end{table}

%%% Local Variables:
%%% mode: latex
%%% TeX-master: "../EITF75-Systems_and_Signals-Reference_Sheet"
%%% End:
