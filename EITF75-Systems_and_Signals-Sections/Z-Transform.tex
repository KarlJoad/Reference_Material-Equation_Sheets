\section{The \texorpdfstring{$Z$-Transform}{Z-Transform}}\label{sec:Z-Transform}
The \nameref{def:Z-Transform} plays the same role in the analysis of discrete-time signals and LTI systems as the \nameref{def:Laplace Transform} does in the analysis of continuous time-signals and LTI systems.

\subsection{The \texorpdfstring{$Z$-Transform}{Z-Transform}}\label{subsec:Z-Transform}
\begin{definition}[$Z$-Transform]\label{def:Z-Transform}
  The \emph{$z$-transform} is defined as the power series
  \begin{equation}\label{eq:Z-Transform}
    % X(z) \equiv \sum_{n=-\infty}^{\infty} x(n)z^{-n}
  \end{equation}

  \begin{remark}
    For convenience, the $z$-transform of a signal $x(n)$ is denoted by
    \begin{equation}\label{eq:Z-Transform Representation}
      X(z) \equiv \ZTransform \lbrace x(n) \rbrace
    \end{equation}
    and the relationship between $x(n)$ and $X(z)$ is indicated by
    \begin{equation}\label{eq:Z-Transform Relationship}
      x(n) \overset{z}{\leftrightarrow} X(z)
    \end{equation}
  \end{remark}
\end{definition}

\subsubsection{Region of Convergence}\label{subsubsec:ROC}
\begin{definition}[ROC]\label{def:ROC}
  The \emph{ROC} or \emph{region of convergence} is the region for which the infinite power series in the $z$-transform has a convergent solution.
  \begin{remark}
    \textbf{\emph{Any time we cite a $z$-transform, we should also indicate its \nameref{def:ROC}}}
  \end{remark}
\end{definition}

\begin{example}[]{Simple $Z$-Transform}
  Determine the $z$-transform of the signal
  \begin{equation*}
    x(n) = \left( \frac{1}{2} \right)^{n} \UnitStep(n)
  \end{equation*}

  \tcblower

  The $z$-transform is the infinite power series
  \begin{equation*}
    \begin{aligned}
      X(z) &= 1 + \frac{1}{2}z^{-1} + {\left( \frac{1}{2} \right)}^{-2} + \cdots + {\left( \frac{1}{2} \right)}^{n}z^{-n} + \cdots \\
      &= \sum_{n=0}^{\infty} {\left( \frac{1}{2} \right)}^{n}z^{-n} = \sum_{n=0}^{\infty} {\left( \frac{1}{2}z^{-1} \right)}^{n}
    \end{aligned}
  \end{equation*}
  Because this is an infinite geometric series, we can solve with with our equivalency:
  \begin{equation*}
    1 + A + A^{2} + \cdots + A^{n} + \cdots = \frac{1}{1-A} \> \text{if } \lvert A \rvert < 1
  \end{equation*}

  Thus, $X(z)$ converges to
  \begin{equation*}
    X(z) = \frac{1}{1-\frac{1}{2}z^{-1}}, \quad \ROC: \lvert z \rvert > \frac{1}{2}
  \end{equation*}
\end{example}

\begin{table}[h!]
  \centering
  \begin{tabular}{cc}
    \toprule
    Signal & ROC \\
    \midrule
    \multicolumn{2}{c}{Finite-Duration Signals} \\
    Causal & Entire $z$-plane except $z=0$ \\
    Anticausal & Entire $z$-plane except $z=\infty$ \\
    Two-Sided & Entire $z$-plane except $z=0$ and $z=\infty$ \\
    \midrule
    \multicolumn{2}{c}{Infinite-Duration Signals} \\
    Causal & $\lvert z \rvert > r_{2}$ \\
    Anticausal & $\lvert z \rvert < r_{1}$ \\
    Two-Sided & $r_{2} < \lvert z \rvert < r_{1}$ \\
    \bottomrule
  \end{tabular}
  \caption{Characteristic Familes of Signals with Their Corresponding ROCs}
  \label{tab:Z-Transform ROC Correspondence}
\end{table}

\subsubsection{The One-Sided \texorpdfstring{$Z$-Transform}{Z-Transform}}\label{subsubsec:One-Sided Z-Transform}
\begin{definition}[One-Sided $Z$-Transform]\label{def:One-Sided Z-Transform}
  The \emph{one-sided $z$-transform} is the same as the \nameref{def:Z-Transform}, but is only defined at $n$ values greater than or equal to 0.
  \begin{equation}\label{eq:One-Sided Z-Transform}
    X(z) \equiv \sum_{n=0}^{\infty} x(n)z^{-n}
  \end{equation}
\end{definition}

The \nameref{def:One-Sided Z-Transform} is generally used when there are initial conditions on a causal signal.
This captures the normal causal portion of the signal, while also showing the effect of the initial condition.

\subsection{The Inverse \texorpdfstring{$Z$-Transform}{Z-Transform}}\label{subsec:Inverse Z-Transform}
This is the formal definition of \nameref{subsec:Inverse Z-Transform}.
\begin{equation}\label{eq:Inverse Z-Transform}
  x(n) = \frac{1}{2 \pi j} \oint_{C} X(z) z^{n-1} \, dz
\end{equation}
where the integrals is a contour integral over a closed path C that encloses the origin and lies within the region of convergence of $X(z)$.

There are 3 methods that are often used for the evaluation of the inverse $z$-transform in practice:
\begin{enumerate}[noitemsep]
\item Direct evaluation of~\eqref{eq:Inverse Z-Transform}.
\item Expansion into a series of terms, in the variable s$z$ and $z^{-1}$.
\item Partial-fraction expansion and table lookup.
\end{enumerate}

\subsubsection{The Inverse \texorpdfstring{$Z$-Transform}{Z-Transform} by Contour Integration}\label{subsubsec:Inverse Z-Transform by Contour Integration}
\begin{definition}[Cauchy's Integral Theorem]\label{def:Cauchy Integral Theorem}
  Let $f(z)$ be a function of the complex variable $z$ and $C$ be a closed path in the $z$-plane.
  If the derivative $\frac{\mathrm{d}f(z)}{\mathrm{d}z}$ exists on and inside the contour $C$ and if $f(z)$ has no poles at $z = z_{0}$, then
  \begin{equation}\label{eq:Cauchy Integral Theorem Specfic}
    \frac{1}{2 \pi j} \oint_{C} \frac{f(z)}{z-z_{0}} \, \mathrm{d}z = \begin{cases}
      f(z_{0}), & \text{ if $z_{0}$ is inside $C$} \\
      0, & \text{ if $z_{0}$ is outside $C$}
    \end{cases}
  \end{equation}

  More generally, if the $(k+1)$-order derivative of $f(z)$ exists and $f(z)$ has no poles at $z = z_{0}$, then
  \begin{equation}\label{eq:Cauchy Integral THeorem General}
    \frac{1}{2 \pi j} \oint_{C} \frac{f(z)}{\left( z-z_{0} \right)^{k}} \, \mathrm{d}z = \begin{cases}
      \frac{1}{\left( k-1 \right)!} \frac{\mathrm{d}^{k-1}f(z)}{\mathrm{d}z^{k-1}} \bigg \vert_{z=z_{0}}, & \text{ if $z_{0}$ is inside $C$} \\
      0, & \text{ if $z_{0}$ is outside $C$}
    \end{cases}
  \end{equation}
\end{definition}


\subsubsection{The Inverse \texorpdfstring{$Z$-Transform}{Z-Transform} by Power Series Expansion}\label{subsubsec:Inverse Z-Transform by Power Series Expansion}
\subsubsection{The Inverse \texorpdfstring{$Z$-Transform}{Z-Transform} by Partial-Fraction Expansion}\label{subsubsec:Inverse Z-Transform by Partial-Fraction Expansion}
\subsection{Properties of the \texorpdfstring{$Z$-Transform}{Z-Transform}}\label{subsec:Z-Transform Properties}
\begin{propertylist}
\item \nameref{subsubsec:Linearity}
\item \nameref{subsubsec:Time Shifting}
\item \nameref{subsubsec:Z-Domain Scaling}
\item \nameref{subsubsec:Time Reversal}
\item \nameref{subsubsec:Z-Domain Differentiation}
\item \nameref{subsubsec:Z-Domain Convolutions}
\item \nameref{subsubsec:2 Sequence Correlation}
\item \nameref{subsubsec:2 Sequence Multiplication}
\item \nameref{subsubsec:Parsevals Relation}
\item \nameref{subsubsec:Initial Value Theorem}
\end{propertylist}

\begin{table}[h!]
  \centering
  \begin{tabular}{p{4cm}ccp{5cm}}
    \toprule
    Property & Time Domain & $z$-Domain & $\ROC$ \\
    \midrule
    Notation & $x(n)$ & $X(z)$ & $\ROC: r_{2} < \lvert z \rvert < r_{1}$ \\
             & $x_{1}(n)$ & $X_{1}(z)$ & $\ROC_{1}$ \\
             & $x_{2}(n)$ & $X_{2}(z)$ & $\ROC_{2}$ \\
    \nameref{subsubsec:Linearity} & $a_{1}x_{1}(n) + a_{2}x_{2}(n)$ & $a_{1}X_{1}(z) + a_{2}X_{2}(z)$ & At least the intersection of $\ROC_{1}$ and $\ROC_{2}$ \\
    \nameref{subsubsec:Time Shifting} & $x(n-k)$ & $z^{-k}X(z)$ & That of $X(z)$, except $z=0$ if $k>0$ and $z=\infty$ if $k<0$ \\
    \nameref{subsubsec:Z-Domain Scaling} & $a^{n}x(n)$ & $X(a^{-1}z)$ & $\lvert a \rvert r_{2} < \lvert z \rvert < \lvert a \rvert r_{1}$ \\
    \nameref{subsubsec:Time Reversal} & $x(-n)$ & $X(z^{-1})$ & $\frac{1}{r_{1}} < \lvert z \rvert < \frac{1}{r_{2}}$ \\
    Conjugation & $x^{*}(n)$ & $X^{*}(z^{*})$ & $\ROC$ \\
    Real Part & $\Re \lbrace x(n) \rbrace$ & $\frac{1}{2} \left[ X(z) + X^{*}(z^{*}) \right]$ & Includes $\ROC$ \\
    Imaginary Part & $\Im \lbrace x(n) \rbrace$ & $\frac{1}{2} j \left[ X(z) - X^{*}(z^{*}) \right]$ & Includes $\ROC$ \\
    \nameref{subsubsec:Z-Domain Differentiation} & $nx(n)$ & $-z \frac{dX(z)}{dz}$ & $r_{2} < \lvert z \rvert r_{1}$ \\
    \nameref{subsubsec:Z-Domain Convolutions} & $x_{1} * x_{2}$ & $X_{1}(z)X_{2}(z)$ & At least, the intersection of $\ROC_{1}$ and $\ROC_{2}$ \\
    \nameref{subsubsec:2 Sequence Correlation} & $r_{x_{1}x_{2}}(l) = x_{1}(l) * x_{2}(-l)$ & $R_{x_{1}x_{2}}(z) = X_{1}(z)x_{2}(z^{-1})$ & At least, the intersection of $\ROC$ of $X_{1}(z)$ and $X_{2}(z^{-1})$ \\
    \nameref{subsubsec:Initial Value Theorem} & If $x(n)$ causal & $x(0) = \lim\limits_{z \rightarrow \infty} X(z)$ & \\
    \nameref{subsubsec:2 Sequence Multiplication} & $x_{1}(n)x_{2}(n)$ & $\frac{1}{2 \pi j} \oint_{C} X_{1}(v)X_{2}(\frac{z}{v}) v^{-1} dv$ & At least, $r_{1l}r_{2l} < \lvert a \rvert < r_{1u}r_{2u}$ \\
    \nameref{subsubsec:Parsevals Relation} & $\sum\limits_{n=-\infty}^{\infty} x_{1}(n)x_{2}^{*}(n)$ &= $\frac{1}{2 \pi j} \oint_{C} X_{1}(v)X_{2}^{*}(\frac{1}{v^{*}})v^{-1} dv$ & \\
    \bottomrule
  \end{tabular}
  \caption{\nameref{subsec:Z-Transform Properties}}
  \label{tab:Z-Transform Properties}
\end{table}

\subsubsection{Linearity}\label{subsubsec:Linearity}
\subsubsection{Time Shifting}\label{subsubsec:Time Shifting}
\subsubsection{\texorpdfstring{$Z$-Domain}{Z-Domain} Scaling}\label{subsubsec:Z-Domain Scaling}
\subsubsection{Time Reversal}\label{subsubsec:Time Reversal}
\subsubsection{\texorpdfstring{$Z$-Domain}{Z-Domain} Differentiation}\label{subsubsec:Z-Domain Differentiation}
\subsubsection{\texorpdfstring{$Z$-Domain}{Z-Domain} Convolutions}\label{subsubsec:Z-Domain Convolutions}
\subsubsection{2 Sequence Correlation}\label{subsubsec:2 Sequence Correlation}
\subsubsection{2 Sequence Multiplication}\label{subsubsec:2 Sequence Multiplication}
\subsubsection{Parsevals Relation}\label{subsubsec:Parsevals Relation}
\subsubsection{Initial Value Theorem}\label{subsubsec:Initial Value Theorem}

\subsection{Properties of the One-Sided \texorpdfstring{$Z$-Transform}{Z-Transform}}\label{subsec:One-Sided Z-Transform Properties}
\begin{propertylist}
\item 
\end{propertylist}

\subsection{Rational \texorpdfstring{$Z$-Transforms}{Z-Transform}}\label{subsec:Rational Z-Transforms}
\subsubsection{Decomposition of Rational \texorpdfstring{$Z$-Transforms}{Z-Transform}}\label{subsubsec:Decompose Rational Z-Transforms}

\subsection{Analysis of LTI Systems in the \texorpdfstring{$Z$-Domain}{Z-Transform}}\label{subsec:Analysis LTI Systems Z-Domain}

%%% Local Variables:
%%% mode: latex
%%% TeX-master: "../EITF75-Systems_and_Signals-Reference_Sheet"
%%% End:
