\section{The $Z$-Transform}\label{sec:Z-Transform}
The \nameref{def:Z-Transform} plays the same role in the analysis of discrete-time signals and LTI systems as the \nameref{def:Laplace Transform} does in the analysis of continuous time-signals and LTI systems.

\subsection{The $Z$-Transform}\label{subsec:Z-Transform}
\begin{definition}[$Z$-Transform]\label{def:Z-Transform}
  The \emph{$z$-transform} is defined as the power series
  \begin{equation}\label{eq:Z-Transform}
    X(z) \equiv \sum_{n=-\infty}^{\infty} x(n)z^{-n}
  \end{equation}
\end{definition}

\subsubsection{The One-Sided $Z$-Transform}\label{subsubsec:One-Sided Z-Transform}
\subsection{The Inverse $Z$-Transform}\label{subsec:Inverse Z-Transform}
\subsubsection{The Inverse $Z$-Transform by Contour Integration}\label{subsubsec:Inverse Z-Transform by Contour Integration}
\subsubsection{The Inverse $Z$-Transform by Power Series Expansion}\label{subsubsec:Inverse Z-Transform by Power Series Expansion}
\subsubsection{The Inverse $Z$-Transform by Partial-Fraction Expansion}\label{subsubsec:Inverse Z-Transform by Partial-Fraction Expansion}
\subsection{Properties of the $Z$-Transform}\label{subsec:Z-Transform Properties}
\begin{propertylist}
\item 
\end{propertylist}

\subsection{Rational $Z$-Transforms}\label{subsec:Rational Z-Transforms}
\subsubsection{Decomposition of Rational $Z$-Transforms}\label{subsubsec:Decompose Rational Z-Transforms}

\subsection{Analysis of LTI Systems in the $Z$-Domain}\label{subsec:Analysis LTI Systems Z-Domain}

%%% Local Variables:
%%% mode: latex
%%% TeX-master: "../EITF75-Systems_and_Signals-Reference_Sheet"
%%% End:
