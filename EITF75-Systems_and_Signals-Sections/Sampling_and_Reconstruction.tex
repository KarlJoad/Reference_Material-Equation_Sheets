\section{Sampling and Reconstruction}\label{sec:Sampling_and_Reconstruction}
\begin{definition}[Optimal Sampling]\label{def:Optimal_Sampling}
  \emph{Optimal sampling} (\emph{lossless sampling}) of a signal $x(t)$ occurs when
  \begin{equation}\label{eq:Optimal_Sampling}
    x_{a}\left( \frac{n}{F_{S}} \right) = x(n)
  \end{equation}

  If there is a noisy signal $x(t) = s(t) + n(t)$, then so long as we can recover the interesting signal $s(t)$, then the sampling was lossless or optimal.

  We can recover $x(t)$ from $x(n)$ if $X_{a}(F)$ looks the same as $X(f)$.
  (If the \nameref{def:FourierTransform} of the analog signal is the same as the \nameref{def:Discrete-Time_Fourier_Transform} of the discrete signal).
  \begin{equation}\label{eq:Recover_Signal}
    X_{a}(F) \approx X(f)
  \end{equation}
  \begin{remark}[Shape of $X(f)$]
    The sampling of $X(f)$ is optimal even if $X(f)$ is:
    \begin{itemize}[noitemsep]
    \item Flipped
    \item Scaled
    \item Negative
    \item Compressed
    \item Expanded
    \item etc.
    \end{itemize}
  \end{remark}

  In summary, the sampling is \emph{optimal} or \emph{lossless} if:
  \begin{enumerate}[noitemsep]
  \item $X_{a}(F)$ contains all information about $x(t)$
  \item $A_{a}(F)$ is aperiodic, but has finite bandwidth
  \item $x(n)$ is a sampled version of $x(t)$
  \item $X(f)$ is sufficient to recover $X_{a}(F)$ $\longrightarrow$ $x(n)$ sufficient to recover $x(t)$
  \end{enumerate}
\end{definition}

There exists a relationship between $x(n)$ and $x\left( \frac{n}{F_{S}} \right)$ in the frequency domain.
\begin{subequations}
  \begin{equation}\label{eq:DTFT_of_xn}
    x(n) = \frac{1}{F_{s}} \int_{\frac{-F_{S}}{2}}^{\frac{F_{S}}{2}} X\left( \frac{F}{F_{S}} \right) e^{j 2\pi n \frac{F}{F_{S}}} \, dF
  \end{equation}
  \begin{equation}\label{eq:Fourier_Representation_of_Sampled_xt}
    x(n) = x(t \vert t = \frac{n}{F_{S}}) = \int_{\frac{-F_{S}}{2}}^{\frac{F_{S}}{2}} \left[ \sum\limits_{k=-\infty}^{\infty} X_{a}(F-kF_{S}) \right] e^{j 2\pi n \frac{F}{F_{S}}} \, dF
  \end{equation}
\end{subequations}
\begin{itemize}[noitemsep]
\item \Cref{eq:DTFT_of_xn} is the \nameref{def:Discrete-Time_Fourier_Transform} of $x(n)$, the discrete signal
\item \Cref{eq:Fourier_Representation_of_Sampled_xt} is the Fourier representation of the sampled continuous-time signal, $x(t \vert t = \frac{n}{F_{S}}) = x(n)$
\end{itemize}

\Crefrange{eq:DTFT_of_xn}{eq:Fourier_Representation_of_Sampled_xt} can be combined and simplified to form \Cref{eq:Optimally_Sampled_Signal}.

\begin{equation}\label{eq:Optimally_Sampled_Signal}
  X(f) = F_{S} \sum\limits_{k=-\infty}^{\infty} X_{a}\Bigl( (f-k) F_{S} \Bigr)
\end{equation}

\subsection{Sampling}\label{subsec:Sampling}
There exists a sampling theory presented by Shannon in 1948 that describes a sufficient condition for complete signal recovery.

\begin{theorem}[Sampling Theorem (Shannon 1948)]\label{thm:Shannon_Sampling_Theorem}
  If $F_{S} > 2B$, where $B$ is the highest frequency of the analog signal, then the analog signal can be recovered from its sampled version.
\end{theorem}

\begin{remark*}
  Note that \nameref{thm:Shannon_Sampling_Theorem} is a \emph{sufficient} condition.
  It does not necessarily represent the smallest sampling frequency possible.
  If the conditions frequency conditions are correct, and later signal processing possible, you can fold signals into empty bands.
\end{remark*}

\subsubsection{Aliasing}\label{subsubsec:Aliasing}
\begin{definition}[Aliasing]\label{def:Aliasing}
  \emph{Aliasing} occurs when the sampling frequency $2 F_{S}< F$.
  When this happens, some of the values on each side of the origin ($X_{a}(0)$ in \Cref{eq:Optimally_Sampled_Signal}) modify each other.

  If \Cref{eq:Optimally_Sampled_Signal} is applied directly, the values for each point in the sampled frequency domain are found directly.
  If done graphically, you can employ \nameref{def:Folding}.

  \begin{remark}[\nameref{def:Optimal_Sampling}]
    \nameref{def:Optimal_Sampling} occurs when there is \emph{no} aliasing.
  \end{remark}
\end{definition}

\begin{definition}[Folding]\label{def:Folding}
  \emph{Folding} is a method of realizing an \nameref{def:Aliasing}.
  There are several steps to perform a folding:
  \begin{enumerate}[noitemsep]
  \item Identify $\frac{F_{S}}{2}$
  \item Fold at $\frac{F_{S}}{2}$. (Turn the values past $\frac{F_{S}}{2}$ backwards, towards the origin).
  \item Add the turned value to the original
  \item If your folded signal goes past the origin, fold the remainaing signal at the origin.
  \item Add this secondly-turned value to the current amount
  \item Repeat Steps 1-6 until all of the original signal is contained between $0$ and $\frac{F_{S}}{2}$.
  \item Change the bounds from $F$ to $\frac{-1}{2} \leq f \leq \frac{1}{2}$
  \end{enumerate}
\end{definition}

\subsection{Reconstruction}\label{subsec:Reconstruction}
The equation for reconstruction of a signal from its sampled counterpart is shown in \Cref{eq:Reconstruction}
\begin{equation}\label{eq:Reconstruction}
  x(t) = \sum\limits_{n=-\infty}^{\infty} x(n) \sinc\Bigl( F_{S} \left( \frac{t-n}{F_{S}} \right) \Bigr)
\end{equation}
\begin{equation}\label{eq:Reconstruction_Sufficient_Sampling}
  x(n) = x(t \vert t = \frac{n}{F_{S}})
\end{equation}

\subsection{Interconnection of Systems with Different Sampling Frequencies}\label{subsec:Interconnect_Different_Sampling_Frequency}
Frequently, various elements in a system will have different sampling frequencies.
This means that your A/D converter and D/A converter will have different sampling frequencies.
This could affect the output of your system, by changing the frequency characteristics through \nameref{def:Aliasing}.

There are 2 main ways that an interconnection of elements in a system will affect the output:
\begin{enumerate}[noitemsep]
\item \nameref{subsubsec:Decimation}
\item \nameref{subsubsec:Interpolation}
\end{enumerate}

\subsubsection{Decimation}\label{subsubsec:Decimation}
\begin{definition}[Decimation]\label{def:Decimation}
  \emph{Decimation} takes an input signal and compresses it.
  This effectively ``downsamples'' the signal.
  Decimation uses the symbol $D \in \PositiveInts$.
  \begin{equation*}
    y(m) = x(mD)
  \end{equation*}

  If decimation occurs later in the system, then if just the input and output are compared, $y(m)$ appears it was sampled at
  \begin{equation}\label{eq:Input_Output_Decimation}
    f = \frac{F_{S}}{D}
  \end{equation}
  
  Thus, when we perform sampling on the input signal, then there is folding at
  \begin{equation}\label{eq:Decimation_Sampling_Frequency_Change}
    f = \frac{F_{S}}{2D}
  \end{equation}
\end{definition}

\subsubsection{Interpolation}\label{subsubsec:Interpolation}
\begin{definition}[Interpolation]\label{def:Interpolation}
  \emph{Interpolation} is the act of putting zeros in between each of the original signal values, while maintaining the original signal profile.
  This effectively ``upsamples'' the signal.
  Interpolation uses the symbol $I \in \PositiveInts$.

  The input/output relationship of a signal when interpolated is more easily shown in the frequency domain.
  \begin{equation}\label{eq:Interpolation}
    Y(f) = X(If)
  \end{equation}

  But, in the time domain, the interpolated output signal would look like
  \begin{equation*}
    y(n) = \lbrace \underline{x(0)}, 0, x(1), 0, x(2), 0, \ldots \rbrace
  \end{equation*}
\end{definition}

%%% Local Variables:
%%% mode: latex
%%% TeX-master: "../EITF75-Systems_and_Signals-Reference_Sheet"
%%% End:
