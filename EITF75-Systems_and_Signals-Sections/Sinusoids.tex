\section{Sinusoids}\label{sec:Sinusoids}
There are several ways to characterize \nameref{sec:Sinusoids}.
The first is by dimension:
\begin{enumerate}[noitemsep]
\item Multidimensional/Multichannel Signals
\item Monodimensional/Monochannel Signals
\end{enumerate}

You can also classify sinusoids by their independent variable (usually time) and the values they take.
\begin{enumerate}[noitemsep]
\item \nameref{def:Continuous-Time Signals} or Analog Signals
\item \nameref{def:Discrete-Time Signals}
\item There is a third way to classify sinusoids and their signals: \nameref{subsec:Digital Signals}
\end{enumerate}

\begin{definition}[Continuous-Time Signals]\label{def:Continuous-Time Signals}
  \emph{Continuous-time signals} or \emph{Analog signals} are defined for every value of time and they take on values in the continuous interval $(a,b)$, where $a$ can be $-\infty$ and $b$ can be $\infty$.
  Mathematically, these signals can be described by functions of a continuous variable.

  For example,
  \begin{equation*}
    x_{1}(t) = \cos \pi t \text{, } x_{2}(t) = e^{-\lvert t \rvert} \text{, } -\infty < t < \infty
  \end{equation*}
\end{definition}

\begin{definition}[Discrete-Time Signals]\label{def:Discrete-Time Signals}
  \emph{Discrete-time signals} are defined only at certain specifed values of time.
  These time instants \textbf{\emph{need not}} be equidistant, but in practice, they are usually taken at equally speced intervals for computation convenience and mathematical tractability.

  For example,
  \begin{equation*}
    x(t_{n}) = e^{-\lvert t_{n} \rvert} \text{, } n=0, \pm 1, \pm 2, \ldots
  \end{equation*}

  A \nameref{def:Discrete-Time Signals} can be represented mathematically by a sequence of real or complex numbers.
  \begin{remark}
    To emphasize the discrete-time nature of the signal, we shall denote the signal as $x(n)$, rather than $x(t)$.
  \end{remark}
  \begin{remark}
    If the time instants $t_{n}$ are equally spaced (i.e., $t_{n}=nT$), the notation $x(nT)$ is also used.
  \end{remark}
\end{definition}

\subsection{Continuous-Time Signals}\label{subsec:Continuous-Time Signals}
\subsubsection{Frequency in Continuous-Time Signals}\label{subsubsec:Frequency in Continuous-Time Signals}
A simple harmonic oscillation is mathematically described by \Cref{eq:Simple Harmonic Oscillation}.

\begin{equation}\label{eq:Simple Harmonic Oscillation}
  x_{a}(t) = A \cos \left( \Omega t + \theta \right) \text{, } -\infty < t < \infty
\end{equation}

\begin{remark*}
  The subscript $a$ is used with $x(t)$ to denote an analog signal.
\end{remark*}

This signal is completely characterized by three parameters:
\begin{enumerate}[noitemsep]
\item $A$, the \emph{amplitude} of the sinusoid
\item $\Omega$, the \emph{frequency} in radians per second (\si{\radian / \second})
\item $\theta$, the \emph{phase} in radians.
\end{enumerate}

Instead of $\Omega$, the frequency $F$ in cycles per second or hertz (\si{\hertz}) is used.
\begin{equation}\label{eq:Continuous Angular Frequency to Frequency}
  \Omega = 2 \pi F
\end{equation}

Plugging~\eqref{eq:Continuous Angular Frequency to Frequency} into~\eqref{eq:Simple Harmonic Oscillation}, yields
\begin{equation}\label{eq:Frequency Harmonic Oscillation}
  x_{a}(t) = A \cos \left( 2 \pi Ft + \theta \right) \text{, } -\infty < t < \infty
\end{equation}

\subsubsection{Properties of Continuous-Time Sinusoidal Signals}\label{subsubsec:Properties Continuous-Time Sinusoids}
The analog sinusoidal signal in \cref{eq:Frequency Harmonic Oscillation} is characterized by the following properties:
\begin{propertylist}
\item For every fixed value of the frequency $F$, $x_{a}(t)$ is periodic.
  \begin{equation*}
    x_{a}(t+T_{p}) = x_{a}(t)
  \end{equation*}
  where $T_{p} = \frac{1}{F}$ is the fundamental period.
\item Continuous-time sinusoidal signals with distinct (different) frequencies are themselves distinct.
\item Increasing the frequency $F$ results in an increase in the rate of oscillation of the signal, in the sense that more periods are included in the given time interval.
\end{propertylist}

\subsection{Discrete-Time Signals}\label{subsec:Discrete-Time Signals}
\subsubsection{Frequency in Discrete-Time Signals}\label{subsubsec:Frequency in Discrete-Time Signals}
A discrete-time sinusoidal signal may be expressed as
\begin{equation}\label{eq:Discrete Time Sinusoid}
  x(n) = A \cos \left( \omega n + \theta \right) \text{, } n \in \AllIntegers \text{, } -\infty < n < \infty
\end{equation}

The signal is characterized by these parameters:
\begin{enumerate}[noitemsep]
\item $n$, the sample number. MUST be an integer.
\item $A$, the \emph{amplitude} of the sinusoid
\item $\omega$, the \emph{angular frequency} in radians per sample
\item $\theta$, is the \emph{phase}, in radians.
\end{enumerate}

Instead of $\omega$, we use the frequency variable $f$ defined by
\begin{equation}\label{eq:Discrete Angular Frequency to Frequency}
  \omega \equiv 2 \pi f
\end{equation}

Using~\eqref{eq:Discrete Time Sinusoid} and~\eqref{eq:Discrete Angular Frequency to Frequency} yields
\begin{equation}\label{eq:Discrete Frequency Sinusoid}
  x(n) = A \cos \left( 2 \pi fn + \theta \right) \text{, } n \in \AllIntegers \text{, } -\infty < n < \infty
\end{equation}

\subsubsection{Properties of Discrete-Time Sinusoidal Signals}\label{subsubsec:Properties Discrete-Time Sinusoids}
\begin{propertylist}
\item A discrete-time sinusoid is periodic \textbf{\emph{ONLY}} if its frequency is a rational number.
\item Discrete-time sinusoids whose frequencies are separated by an integer multiple of $2\pi$ are identical. This leads us to the idea of a \nameref{def:Frequency Alias}.\label{prop:Discrete-Time Integer Multiple}
\item The highest rate of oscillation in a discrete-time sinusoid is attained when $\omega = \pm \pi$ or, equivalently, $f= \pm \frac{1}{2}$.\label{prop:Discrete-Time Frequency Limit}
\end{propertylist}

\subsubsection{Frequency Aliases}\label{subsubsec:Frequency Aliases}
The concept of a \nameref{def:Frequency Alias} is drawn from the idea that discrete-time sinusoids whose frequencies are separated by an integer mutliple of $2\pi$ are identical and that frequencies $\lvert f \rvert > \frac{1}{2}$ are identical.
(Properties~\ref{prop:Discrete-Time Integer Multiple} and~\ref{prop:Discrete-Time Frequency Limit})

\begin{definition}[Frequency Alias]\label{def:Frequency Alias}
  A \emph{frequency alias} is a sinusoid having a frequency $\lvert \omega \rvert > \pi$ or $\lvert f \rvert > \frac{1}{2}$.
  This is because this sinusoid is \emph{indistinguishable} (\emph{identical}) to one with  frequency $\lvert \omega \rvert < \pi$ or $\lvert f \rvert < \frac{1}{2}$.
  \newline
  \begin{blackbox}
    A \emph{frequency alias} is a sequence resulting from the
    following assertion based on the sinusoid
    $\cos(\omega_{0}n + \theta)$.

    It follows that
    \begin{equation*}
      \cos \left[ \left( \omega_{0} + 2\pi \right)n + \theta \right] = \cos \left( \omega_{0}n + 2\pi n + \theta \right) = \cos (\omega_{0}n + \theta
    \end{equation*}

    As a result, all sinusoidal sequences
    \begin{equation*}
      x_{k}(n) = A \cos (\omega_{k}n + \theta) , \> k=0, 1, 2, \ldots
    \end{equation*}

    where
    \begin{equation*}
      \omega k = \omega_{0} + 2k \pi , \> -\pi \leq \omega_{0} \leq \pi
    \end{equation*}

    are \emph{indistinguishable} (i.e., \emph{identical}). \newline
  \end{blackbox}

  Because of this, we regard frequencies in the range of $-\pi \leq \omega \leq \pi$ or $-\frac{1}{2} \leq f \leq \frac{1}{2}$ as unique, and all frequencies that fall outside of these ranges as aliases.

  \begin{remark}
    It should be noted that there is a difference between discrete-time sinusoids and continuous-time sinusoids here.
    Continuous-time sinusoids have distinct signals for $\Omega$ or $F$ in the entire range $-\infty < \Omega < \infty$ or $-\infty < F < \infty$.
  \end{remark}
\end{definition}

\subsection{Sampling Rates and Sampling Frequency}\label{subsec:Sampling Rates and Frequency}
\subsubsection{Nyquist Rate}\label{subsubsec:Nyquist Rate}
\subsubsection{Nyquist Frequency}\label{subsubsec:Nyquist Frequency}

\subsection{Digital Signals}\label{subsec:Digital Signals}
\begin{definition}[Digital Signals]\label{def:Digital Signals}
  \emph{Digital signals} are a subset of \nameref{def:Discrete-Time Signals}.
  In this case, not only are the values being measured occuring at fixed points in time, the values themselves can only take certain, fixed values.
\end{definition}

\subsubsection{Quantization}\label{subsubsec:Quantization}
\paragraph{Quantization Levels}\label{par:Quantization Levels}
\paragraph{Bit Requirements}\label{par:Quantization Bit Requirements}
\paragraph{Bit Rate}\label{par:Quantization Bit Rate}
%%% Local Variables:
%%% mode: latex
%%% TeX-master: "../EITF75-Systems_and_Signals-Reference_Sheet"
%%% End:
