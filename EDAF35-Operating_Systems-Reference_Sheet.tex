\documentclass[10pt,letterpaper,final,twoside,notitlepage]{article}
\usepackage[margin=.5in]{geometry} % 1/2 inch margins on all pages
\usepackage[utf8]{inputenc} % Define the input encoding
\usepackage[USenglish]{babel} % Define language used
\usepackage{amsmath,amsfonts,amssymb}
\usepackage{amsthm} % Gives us plain, definition, and remark to use in \theoremstyle{style}
\usepackage{mathtools} % Allow for text and math in align* environment.
\usepackage{thmtools}
\usepackage{thm-restate}
\usepackage{graphicx}

\usepackage[
backend=biber,
style=alphabetic,
citestyle=authoryear]{biblatex} % Must include citation somewhere in document to print bibliography
\usepackage{hyperref} % Generate hyperlinks to referenced items
\usepackage[nottoc]{tocbibind} % Prints the Reference/Bibliography in TOC as well
\usepackage[noabbrev,nameinlink]{cleveref} % Fancy cross-references in the document everywhere
\usepackage{nameref} % Can make references by name to places
\usepackage{caption} % Allows for greater control over captions in figure, algorithm, table, etc. environments
\usepackage{subcaption} % Allows for multiple figures in one Figure environment
\usepackage[binary-units=true]{siunitx} % Gives us ways to typeset units for stuff
\usepackage{csquotes} % Context-sensitive quotation facilities
\usepackage{enumitem} % Provides [noitemsep, nolistsep] for more compact lists
\usepackage{chngcntr} % Allows us to tamper with the counter a little more
\usepackage{empheq} % Allow boxing of equations in special math environments
\usepackage[x11names]{xcolor} % Gives access to coloring text in environments or just text, MUST be before tikz
\usepackage{tcolorbox} % Allows us to create boxes of various types for examples
\usepackage{tikz} % Allows us to create TikZ and PGF Pictures
\usepackage{ctable} % Greater control over tables and how they look
\usepackage{diagbox} % Allow us to have shared diagonal cells in tables
\usepackage{multirow} % Allow us to have a single cell in a table span multiple rows
\usepackage{titling} % Put document information throughout the document programmatically
\usepackage[linesnumbered,ruled,vlined]{algorithm2e} % Allows us to write algorithms in a nice style.

\counterwithin{figure}{section}
\counterwithin{table}{section}
\counterwithin{equation}{section}
\counterwithin{algocf}{section}
\crefname{algocf}{algorithm}{algorithms}
\Crefname{algocf}{Algorithm}{Algorithms}
\setcounter{secnumdepth}{4}
\setcounter{tocdepth}{4} % Include \paragraph in toc
\crefname{paragraph}{paragraph}{paragraphs}
\Crefname{paragraph}{Paragraph}{Paragraphs}

% Create a theorem environment
\theoremstyle{plain}
\newtheorem{theorem}{Theorem}[section]
% Create a numbered theorem-like environment for lemmas
\newtheorem{lemma}{Lemma}[theorem]

% Create a definition environment
\theoremstyle{definition}
\newtheorem{definition}{Defn}
\newtheorem{corollary}{Corollary}[section]
% \begin{definition}[Term] \label{def:}
%   Make sure the term is emphasized with \emph{term}.
%   This ensures that if \emph is changed, it shows up everywhere
% \end{definition}

% Create a numbered remark environment numbered based on definition
% NOTE: This version of remark MUST go inside a definition environment
\theoremstyle{remark}
\newtheorem{remark}{Remark}[definition]
%\counterwithin{definition}{subsection} % Uncomment to have definitions use section.subsection numbering

% Create an unnumbered remark environment for general use
% NOTE: This version of remark has NO restrictions on placement
\newtheorem*{remark*}{Remark}

% Create a special list that handles properties. It can be broken and restarted
\newlist{propertylist}{enumerate}{1} % {Name}{Template}{Max Depth}
% [newlistname, LevelsToApplyTo]{formatting options}
\setlist[propertylist, 1]{label=\textbf{(\roman*)}, ref=\textbf{(\roman*)}, noitemsep, nolistsep}
\crefname{propertylisti}{property}{properties}
\Crefname{propertylisti}{Property}{Properties}

% Create a special list that handles enumerate starting with lower letters. Breakable/Restartable.
\newlist{boldalphlist}{enumerate}{1} % {Name}{Template}{Max Depth}
% [newlistname, LevelsToApplyTo]{formatting options}
\setlist[boldalphlist, 1]{label=\textbf{(\alph*)}, ref=\alph*, noitemsep, nolistsep} % Set options

\newlist{nocrefenumerate}{enumerate}{1} % {Name}{Template}{Max Depth}
% [newlistname, LevelsToApplyTo]{formatting options}
\setlist[nocrefenumerate, 1]{label=(\arabic*), ref=(\arabic*), noitemsep, nolistsep}

% Create a list that allows for deeper nesting of numbers. By default enumerate only allows depth=4.
\newlist{nestednums}{enumerate}{6}
% [newlistname, LevelsToApplyTo]{formatting options}
\setlist[nestednums]{noitemsep, label*=\arabic*.}

\tcbuselibrary{breakable} % Allow tcolorboxes to be broken across pages
% Create a tcolorbox for examples
% /begin{example}[extra name]{NAME}
% Create a tcolorbox for examples
% Argument #1 is optional, given by [], that is the textbook's problem number
% Argument #2 is mandatory, given by {}, that is the title for the example
% Avoid putting special characters, (), [], {}, ",", etc. in the title.
\newtcolorbox[auto counter,
number within=section,
number format=\arabic,
crefname={example}{examples}, % Define reference format for cref (No Capitals)
Crefname={Example}{Examples}, % Reference format for cleveref (With Capitals)
]{example}[2][]{ % The [2][] Means the first argument is optional
  width=\textwidth,
  title={Example \thetcbcounter: #2. #1}, % Parentheses and commas are not well supported
  fonttitle=\bfseries,
  label={ex:#2},
  nameref=#2,
  colbacktitle=white!100!black,
  coltitle=black!100!white,
  colback=white!100!black,
  upperbox=visible,
  lowerbox=visible,
  sharp corners=all,
  breakable
}

% Create a tcolorbox for general use
\newtcolorbox[% auto counter,
% number within=section,
% number format=\arabic,
% crefname={example}{examples}, % Define reference format for cref (No Capitals)
% Crefname={Example}{Examples}, % Reference format for cleveref (With Capitals)
]{blackbox}{
  width=\textwidth,
  % title={},
  fonttitle=\bfseries,
  % label={},
  % nameref=,
  colbacktitle=white!100!black,
  coltitle=black!100!white,
  colback=white!100!black,
  upperbox=visible,
  lowerbox=visible,
  sharp corners=all
}

% Redefine the 'end of proof' symbol to be a black square, not blank
\renewcommand{\qedsymbol}{$\blacksquare$} % Change proofs to have black square at end

% Common Mathematical Stuff
\newcommand{\Abs}[1]{\ensuremath{\lvert #1 \rvert}}
\newcommand{\DNE}{\ensuremath{\mathrm{DNE}}} % Used when limit of function Does Not Exist

% Complex Numbers functions
\renewcommand{\Re}{\operatorname{Re}} % Redefine to use the command, but not the fraktur version
\renewcommand{\Im}{\operatorname{Im}} % Redefine to use the command, but not the fraktur version
\newcommand{\Real}[1]{\ensuremath{\Re \lbrace #1 \rbrace}}
\newcommand{\Imag}[1]{\ensuremath{\Im \lbrace #1 \rbrace}}
\newcommand{\Conjugate}[1]{\ensuremath{\overline{#1}}}
\newcommand{\Modulus}[1]{\ensuremath{\lvert #1 \rvert}}
\DeclareMathOperator{\PrincipalArg}{\ensuremath{Arg}}

% Math Operators that are useful to abstract the written math away to one spot
% Number Sets
\DeclareMathOperator{\RealNumbers}{\ensuremath{\mathbb{R}}}
\DeclareMathOperator{\AllIntegers}{\ensuremath{\mathbb{Z}}}
\DeclareMathOperator{\PositiveInts}{\ensuremath{\mathbb{Z}^{+}}}
\DeclareMathOperator{\NegativeInts}{\ensuremath{\mathbb{Z}^{-}}}
\DeclareMathOperator{\NaturalNumbers}{\ensuremath{\mathbb{N}}}
\DeclareMathOperator{\ComplexNumbers}{\ensuremath{\mathbb{C}}}
\DeclareMathOperator{\RationalNumbers}{\ensuremath{\mathbb{Q}}}

% Calculus operators
\DeclareMathOperator*{\argmax}{argmax} % Thin Space and subscripts are UNDER in display

% Signal and System Functions
\DeclareMathOperator{\UnitStep}{\mathcal{U}}
\DeclareMathOperator{\sinc}{sinc} % sinc(x) = (sin(pi x)/(pi x))

% Transformations
\DeclareMathOperator{\Lapl}{\mathcal{L}} % Declare a Laplace symbol to be used

% Logical Operators
\DeclareMathOperator{\XOR}{\oplus}

% x86 CPU Registers
\newcommand{\rbpRegister}{\texttt{\%rbp}}
\newcommand{\rspRegister}{\texttt{\%rsp}}
\newcommand{\ripRegister}{\texttt{\%rip}}
\newcommand{\raxRegister}{\texttt{\%rax}}
\newcommand{\rbxRegister}{\texttt{\%rbx}}

%%% Local Variables:
%%% mode: latex
%%% TeX-master: shared
%%% End:


% These packages are more specific to certain documents, but will be availabe in the template
% \usepackage{esint} % Provides us with more types of integral symbols to use
\usepackage[outputdir=./TeX_Output]{minted} % Allow us to nicely typeset 300+ programming languages
% This document must be compiled with the -shell-escape flag if the packages above are uncommented

% \graphicspath{{./Drawings/Course}} % Uncomment this to use pictures in this document
\addbibresource{./Bibliographies/EDAF35-Operating_Systems.bib}

% Math Operators that are useful to abstract the written math away to one spot
% These are supposed to be document-specific mathematical operators that will make life easier
% Many fundamental operators are defined in Reference_Sheet_Preamble.tex

\begin{titlepage}
  \title{EDAF35: Operating Systems --- Reference Sheet \\ Lund University}
  \author{Karl Hallsby}
  \date{Last Edited: \today} % We want to inform people when this document was last edited
\end{titlepage}

\begin{document}
\pagenumbering{gobble}
\maketitle
\pagenumbering{roman} % i, ii, iii on beginning pages, that don't have content
\tableofcontents
\clearpage
\listoftheorems[ignoreall, show={definition, Definition}]
\clearpage
\pagenumbering{arabic} % 1,2,3 on content pages

\nocite{*}

\section{Operating System Introduction}\label{sec:OS_Intro}
A computer system can be roughly divided into 4 parts.
\begin{itemize}[noitemsep]
\item The \nameref{def:Hardware}
\item The \nameref{def:Operating_System}
\item The \nameref{def:Application_Program}s
\item The Users
\end{itemize}

\begin{definition}[Hardware]\label{def:Hardware}
  \emph{Hardware} is the physical components of the system and provide the basic computing resources for the system..
  Hardware includes the \nameref{def:CPU}, \nameref{def:Memory}, and all I/O devices (monitor, keyboard, mouse, etc.).

  \begin{remark}[How to Differentiate]\label{rmk:Hardware_Differentiate}
    If you are finding it difficult to tell \nameref{def:Hardware}, \nameref{def:Software}, and \nameref{def:Firmware} apart, answer this simple question.
    Can you hit it with a hammer and break the thing?
    \begin{description}[noitemsep]
    \item[Yes] Then it is \nameref{def:Hardware}.
    \item[No] Then it is \nameref{def:Software}.
    \item[Yes and No] Then it is \nameref{def:Firmware}.
    \end{description}
  \end{remark}
\end{definition}

\begin{definition}[Software]\label{def:Software}
  \emph{Software} is the code that is used to build the system and make it perform operations.
  Technically, it is the electrical signals that represent \texttt{0} or \texttt{1} and makes the \nameref{def:Hardware} act in a specific, desired fashion to produce some result.

  On a higher level, this can be though of as computer code.

  \begin{remark}[How to Differentiate]\label{rmk:Software_Differentiate}
    If you are finding it difficult to tell \nameref{def:Hardware}, \nameref{def:Software}, and \nameref{def:Firmware} apart, answer this simple question.
    Can you hit it with a hammer and break the thing?
    \begin{description}[noitemsep]
    \item[Yes] Then it is \nameref{def:Hardware}.
    \item[No] Then it is \nameref{def:Software}.
    \item[Yes and No] Then it is \nameref{def:Firmware}.
    \end{description}
  \end{remark}
\end{definition}

\begin{restatable}[Operating System]{definition}{defOperatingSystem}\label{def:Operating_System}
  An \emph{operating system} is a large piece of software that controls the \nameref{def:Hardware} and coordinates the many \nameref{def:Application_Program}s various numbers of \nameref{def:User}s may use.
  It provides the means for proper use of these resources to allow the computer to run.

  By itself, an operating system does nothing useful.
  It simply provides an \textbf{environment} within which other programs can perform useful work.

  The fundamental goal of computer systems is to execute user programs and to make solving user problems easier.
  These programs require certain common operations, such as those controlling the I/O devices.

  In addition, there is no universally accepted definition of what is part of the operating system.
  A simple definition is that it includes everything a vendor ships when you order ``the operating system.''
  The features included, however, vary greatly across systems.
  Some systems take up less than a megabyte of space and lack even a full-screen editor, whereas others require gigabytes of space and are based entirely on graphical windowing systems.
  A more common definition, and the one that we usually follow, is that the operating system is the one program running at all times on the computer—usually called the \nameref{def:Kernel}.

  \begin{remark}[Kernel-Level Non-Kernal Programs]\label{rmk:Kernel_Level_Non_Kernel_Programs}
    Along with the \nameref{def:Kernel}, there are two other types of programs:
    \begin{enumerate}[noitemsep]
    \item System Programs,
      \begin{itemize}[noitemsep]
      \item Associated with the \nameref{def:Operating_System} but are not necessarily part of the \nameref{def:Kernel}.
      \end{itemize}
    \item Application Programs
      \begin{itemize}[noitemsep]
      \item Includes all programs not associated with the operation of the system
      \end{itemize}
    \end{enumerate}
  \end{remark}
\end{restatable}

\begin{definition}[Kernel]\label{def:Kernel}
  The kernel is a computer program at the core of a computer's operating system with complete control over everything in the system.
  It is the ``portion of the operating system code that is always resident in memory''.
  It facilitates interactions between hardware and software components.
  On most systems, it is one of the first programs loaded on startup (after the bootloader).
  It handles input/output requests from software, translating them into data-processing instructions for the central processing unit.
  It handles memory and its mapping, peripherals like: keyboards, monitors, printers, and speakers.
  A kernel connects the application software to the hardware of a computer.

  The critical code of the kernel is usually loaded into a separate area of memory, which is protected from access by application programs or other, less critical parts of the operating system.
  The kernel performs its tasks, such as running processes, managing hardware devices such as the hard disk, and handling interrupts, in this protected kernel space.
\end{definition}

\begin{definition}[Application Program]\label{def:Application_Program}
  An \emph{application program} is a tool used by a \nameref{def:User} to solve some problem.
  This is the main thing a normal person will interact with.
  These pieces of software can include:
  \begin{itemize}[noitemsep]
  \item Text editors
  \item Compilers
  \item Web browsers
  \item Word Processors
  \item Spreadsheets
  \item etc.
  \end{itemize}
\end{definition}

\begin{definition}[User]\label{def:User}
  A \emph{user} is the person and/or thing that is running some \nameref{def:Application_Program}s.

  \begin{remark}[Thing Users]\label{rmk:Thing_Users}
    Not all \nameref{def:User}s are required to be people.
    The automated tasks a computer may do to provide a seamless experience for the person may be done by other users in the system.
  \end{remark}
\end{definition}

\subsection{User View}\label{subsec:User_View}
The user's view of the computer varies according to the interface they are using.

In modern times, most people are using computers with a monitor that provides a GUI, a keyboard, mouse, and the physical system itself.
These are designed for one user to use the system at a time, allowing that user to monopolize the system's resources.
The \nameref{def:Operating_System} is designed for \textbf{ease of use} in this case, with relatively little attention paid to performance and resource utilization.

More old-school, but stil in use, a \nameref{def:User} sits at a terminal connected to a mainframe or a minicomputer.
Other users are accessing the same computer through other terminals.
These users share resources and may exchange information.
The operating system in such cases is designed to maximize resource utilization, to assure that all available CPU time, memory, and I/O are used efficiently and that no individual user takes more than their fair share.

In still other cases, \nameref{def:User}s sit at workstations connected to networks of other workstations and servers.
These users have dedicated resources at their disposal, but they also share resources such as networking and servers, including file, compute, and print servers.
Therefore, their operating system is designed to compromise between individual usability and resource utilization.

Lastly, there are \nameref{def:Operating_System}s that are designed to have little to no \nameref{def:User} view.
These are typically embedded systems with very limited input/output.

\subsection{System View}\label{subsec:System_View}
From the computer’s point of view, the \nameref{def:Operating_System} is the program that interacts the most with the hardware.
A computer system has many resources that can be used to solve a problem:
\begin{itemize}[noitemsep]
\item CPU time
\item Memory space
\item File-storage space
\item I/O devices
\item etc.
\end{itemize}

The operating system acts as the manager of these resources.
Facing numerous and possibly conflicting requests for resources, the operating system must decide how to allocate them so that it can operate the computer system efficiently and fairly.
As we have seen, resource allocation is especially important where many \nameref{def:User}s access the same system.

Another, slightly different, view of an operating system emphasizes the need to control the various I/O devices and user programs.
An operating system is a control program.
A control program manages the execution of user programs to prevent errors and improper use of the computer.
It is especially concerned with the operation and control of I/O devices.

\subsection{Computer Organization}\label{subsec:Computer_Organization}
The initial program, run \textbf{\emph{RIGHT}} when the computer starts is typically kept onboard the computer \nameref{def:Hardware}, on ROMs or EEPROMs.

\begin{definition}[Firmware]\label{def:Firmware}
  \emph{Firmware} is software that is written for a specific piece of hardware in mind.
  Its characteristics fall somewhere between those of \nameref{def:Hardware} and those of software.
  It is almost always stored in the \nameref{def:Hardware}'s onboard storage.
  Typically it is stored in ROM~(Read-Only Memory) or EEPROM~(Electrically Erasable Programmable Read-Only Memory).
  It initializes all aspects of the system, from \nameref{def:CPU} \nameref{def:Register}s to device controllers, to memory contents.

  \begin{remark}[How to Differentiate]\label{rmk:Firmware_Differentiate}
    If you are finding it difficult to tell \nameref{def:Hardware}, \nameref{def:Software}, and \nameref{def:Firmware} apart, answer this simple question.
    Can you hit it with a hammer and break the thing?
    \begin{description}[noitemsep]
    \item[Yes] Then it is \nameref{def:Hardware}.
    \item[No] Then it is \nameref{def:Software}.
    \item[Yes and No] Then it is \nameref{def:Firmware}.
    \end{description}
  \end{remark}
\end{definition}

A \nameref{def:CPU} will continue its boot process, until it reaches the \texttt{init} phase, where many other system processes or \nameref{def:Daemon}s start.
Once the computer finishes going through all its \texttt{init} phases, it is ready for use, waiting for some event to occur.
These events can be a \nameref{def:Hardware} \nameref{def:Interrupt} or a software \nameref{def:System_Call}.

\begin{definition}[Daemon]\label{def:Daemon}
  In UNIX and UNIX-like \nameref{def:Operating_System}s, a \emph{daemon} is a \nameref{def:System_Program} process that runs in the ``background'', is started, stopped, and handled by the system, rather than the \nameref{def:User}.
  Daemons run constantly, from the time they are started (potentially the computer's boot) to the time they are killed (potentially when the computer shuts down).
  Typical systems are running dozens, possibly hundreds, of daemons constantly.

  Some examples of daemons are:
  \begin{itemize}[noitemsep]
  \item Network daemons to listen for network connections to connect those requests to the correct processes.
  \item Process schedulers that start processes according to a specified schedule
  \item System error monitoring services
  \item Print servers
  \end{itemize}

  \begin{remark}[Other Names]\label{rmk:Daemon_Other_Names}
    On other, non-UNIX systems, \nameref{def:Daemon}s are called other names.
    They can be called \emph{services}, \emph{subsystems}, or anything of that nature.
  \end{remark}
\end{definition}

\begin{definition}[Interrupt]\label{def:Interrupt}
  An \emph{interrupt} is a special event that the \nameref{def:CPU} \textbf{MUST} handle.
  These could be system errors, or just a button on the keyboard was pressed.
  Hardware may trigger an interrupt at any time by sending a signal to the CPU, usually by way of the system bus.

  When a CPU receives an interrupt, it immediately stops what it is doing and transfers execution to some fixed address.
  To ensure that this happens as quickly as possible, a \nameref{def:Interrupt_Vector} is created.
\end{definition}

\begin{definition}[Trap]\label{def:Trap}
  A \emph{trap} or \emph{exception} is a software-generated \nameref{def:Interrupt} caused by:
  \begin{itemize}[noitemsep]
  \item A program execution error (Division-by-zero or Invalid Memory Access).
  \item A specific request from a user program that an operating-system service be performed (Print to screen).
  \end{itemize}
\end{definition}

\begin{definition}[Interrupt Vector]\label{def:Interrupt_Vector}
  The \emph{interrupt vector} is a table/list of addresses that redirect the \nameref{def:CPU} to the location of the instructions for how to handle that particular \nameref{def:Interrupt}.
  Since only a predefined number of interrupts is possible, a table of pointers to interrupt routines is used to provide the necessary speed.
  These locations hold the addresses of the interrupt service routines for the various devices.
  This array, or interrupt vector, of addresses is then indexed by a unique device number, given with the interrupt request, to provide the address of the interrupt service routine for the interrupting device.
  The interrupt routine is called indirectly through the table, with no intermediate routine needed.
  Generally, this is stored in low memory (the first hundred or so locations).
\end{definition}

\subsection{Storage Management}\label{subsec:Storage_Management}
\begin{definition}[File]\label{def:File}
  The \nameref{def:Operating_System} abstracts from the physical properties of its storage devices to define a logical storage unit, the \emph{file}.
  The operating system maps files onto physical media and accesses these files via the storage devices.
\end{definition}

\subsection{System Calls}\label{subsec:System_Calls}
\begin{definition}[System Call]\label{def:System_Call}
  Software may trigger an interrupt by executing a special operation called a \emph{system call}.
  This can also be called a monitor call.

  System calls provide an interface to the services made available by an \nameref{def:Operating_System}.
  These calls are generally available as routines written in C and C++.
  Some of the lowest-level tasks (for example, tasks where hardware must be accessed directly) may have to be written using assembly instructions.

  There are roughly 6 different types of system calls:
  \begin{enumerate}[noitemsep]
  \item \nameref{subsubsec:Process_Control}
    \begin{itemize}[noitemsep]
    \item End, Abort
    \item Load, Execute
    \item Create process, Terminate process
    \item Get process attributes, Set process attributes
    \item Wait for time
    \item Wait event, Signal event
    \item Allocate and Free memory
    \end{itemize}
  \item \nameref{subsubsec:File_Manipulation}
    \begin{itemize}[noitemsep]
    \item Create file, Delete file
    \item Open, Close
    \item Read, Write, Reposition
    \item Get file attributes, Set file attributes
    \end{itemize}
  \item \nameref{subsubsec:Device_Manipulation}
    \begin{itemize}[noitemsep]
    \item Request device, Release device
    \item Read, Write, Reposition
    \item Get device attributes, Set device attributes
    \item Logically attach or detach devices
    \end{itemize}
  \item \nameref{subsubsec:Information_Maintenance}
    \begin{itemize}[noitemsep]
    \item Get time or date, Set time or date
    \item Get system data, Set system data
    \item Get Process, File, or Device attributes
    \item Set Process, File, or Device attributes
    \end{itemize}
  \item \nameref{subsubsec:Communications}
    \begin{itemize}[noitemsep]
    \item Create, Delete communication connection
    \item Send, Receive messages
    \item Transfer status information
    \item Attach or Detach remote devices
    \end{itemize}
  \item \nameref{subsubsec:Protection}
  \end{enumerate}
\end{definition}

\begin{table}[h!tbp]
  \centering
  \begin{tabular}{lll}
    \toprule
    & \textbf{Windows} & \textbf{Unix} \\
    \midrule
    \nameref{subsubsec:Process_Control} & \mintinline{cpp}{CreateProcess()} & \mintinline{c}{fork()} \\
    & \mintinline{cpp}{ExitProcess()} & \mintinline{c}{exit()} \\
    & \mintinline{cpp}{WaitForSingleObject()} & \mintinline{c}{wait()} \\
    \midrule
    \nameref{subsubsec:File_Manipulation} & \mintinline{cpp}{CreateFile()} & \mintinline{c}{open()} \\
    & \mintinline{cpp}{ReadFile()} & \mintinline{c}{read()} \\
    & \mintinline{cpp}{WriteFile()} & \mintinline{c}{write()} \\
    & \mintinline{cpp}{CloseHandle()} & \mintinline{c}{close()} \\
    \midrule
    \nameref{subsubsec:Device_Manipulation} & \mintinline{cpp}{SetConsoleMode()} & \mintinline{c}{ioctl()} \\
    & \mintinline{cpp}{ReadConsole()} & \mintinline{c}{read()} \\
    & \mintinline{cpp}{WriteConsole()} & \mintinline{c}{write()} \\
    \midrule
    \nameref{subsubsec:Information_Maintenance} & \mintinline{cpp}{GetCurrentProcessID()} & \mintinline{c}{getpid()} \\
    & \mintinline{cpp}{SetTimer()} & \mintinline{c}{alarm()} \\
    & \mintinline{cpp}{Sleep()} & \mintinline{c}{sleep()} \\
    \midrule
    \nameref{subsubsec:Communications} & \mintinline{cpp}{CreatePipe()} & \mintinline{c}{pipe()} \\
    & \mintinline{cpp}{CreateFileMapping()} & \mintinline{c}{shm_open()} \\
    & \mintinline{cpp}{MapViewOfFile()} & \mintinline{c}{mmap()} \\
    \midrule
    \nameref{subsubsec:Protection} & \mintinline{cpp}{SetFileSecurity()} & \mintinline{c}{chmod()} \\
    & \mintinline{cpp}{InitializeSecurityDescriptor()} & \mintinline{c}{umask()} \\
    & \mintinline{cpp}{SetSecurityDescriptorGroup()} & \mintinline{c}{chown()} \\
    \bottomrule
  \end{tabular}
  \caption{System Calls in Unix and Windows}
  \label{tab:System_Calls_Examples}
\end{table}

\nameref{def:System_Call}s are exposed to the programmer by an \nameref{def:API}.
\begin{definition}[Application Programming Interface]\label{def:API}
  An \emph{Application Programming Interface} (\emph{API}) specifies a set of functions that are available to an application programmer.
  They specify the parameters that are passed to each function and the return values the programmer can expect.

  Typically, API calls perform \nameref{def:System_Call}s in the background, without the programmer knowing about them.
\end{definition}

\subsubsection{Process Control}\label{subsubsec:Process_Control}
A running program needs to be able to halt its own execution, either normally or abnormally.
If a system call is made to terminate the currently running program abnormally, or if the program runs into a problem and causes an error \nameref{def:Trap}, a dump of memory is sometimes taken and an error message generated.
The dump is written to disk and may be examined by a debugger—a system program designed to aid the programmer in finding and correcting errors, or bugs—to determine the cause of the problem.

Under either normal or abnormal circumstances, the operating system must transfer control to the invoking command interpreter.
The command interpreter then reads the next command.

To determine how bad the execution halt was, when the program ceases execution, it will return an exit code.
By convention, and for no other reason, an exit code of \texttt{0} is considered to be the program completed execution successfully.
Otherwise, the greater the return value, the greater the severity of the error.

\subsubsection{File Manipulation}\label{subsubsec:File_Manipulation}
We first need to be able to \mintinline{c}{create()} and \mintinline{c}{delete()} files.
Either system call requires the name of the file and perhaps some of the file’s attributes.
Once the file is created, we need to \mintinline{c}{open()} it and to use it.
We may then \mintinline{c}{read()}, \mintinline{c}{write()}, or perform any other \nameref{def:API}-defined action(s).
Finally, we need to \mintinline{c}{close()} the file, indicating that we are no longer using it.

We may need these same sets of operations for directories if we have a directory structure for organizing files in the file system.
In addition, for either files or directories, we need to be able to determine the values of various attributes and perhaps to reset them if necessary.

\begin{definition}[File Attribute]\label{def:File_Attribute}
  A \emph{file attribute} contains metadata about the file.
  This includes the file's name, type, protection codes, accounting information, and so on.
\end{definition}

\begin{remark*}
  If the system programs are callable by other programs, then each can be considered an \nameref{def:API} by other system programs.
\end{remark*}

\subsubsection{Device Manipulation}\label{subsubsec:Device_Manipulation}
\begin{definition}[Device]\label{def:Device}
  A \emph{device} in an \nameref{def:Operating_System} is a resource that must be controlled.
  Some of these devices are physical devices (for example, disk drives), while others can be thought of as abstract or virtual devices (for example, files).
\end{definition}

A system with multiple users may require us to first \mintinline{c}{request()} a device, to ensure exclusive use of it.
After we are finished with the device, we \mintinline{c}{release()} it.
These functions are similar to the \mintinline{c}{open()} and \mintinline{c}{close()} system calls for files.
Other operating systems allow unmanaged access to devices.
The hazard then is the potential for device contention and perhaps \nameref{def:Deadlock}.

Once the device has been requested (and allocated to us), we can \mintinline{c}{read()}, \mintinline{c}{write()}, just as we can with files.
In fact, the similarity between I/O devices and files is so great that many operating systems, including UNIX, merge the two into a combined file–device structure.
In this case, a set of system calls can be shared between both files and \nameref{def:Device}s.
Sometimes, I/O devices are identified by special file names, directory placement, or file attributes.

\subsubsection{Information Maintenance}\label{subsubsec:Information_Maintenance}
Many system calls exist simply for the purpose of transferring information between the \nameref{def:User} program and the \nameref{def:Operating_System}.
For example, most systems have a system call to return the current \mintinline{c}{time()} and \mintinline{c}{date()}.
Other system calls may return information about the system, such as the number of current users, the version number of the operating system, the amount of free memory or disk space, and so on.

Another set of system calls is helpful in debugging a program.
Many systems provide system calls to \mintinline{c}{dump()} memory.
A program \texttt{trace} lists each system call as it is executed.
In addition, the \nameref{def:Operating_System} keeps information about all its processes, and \nameref{def:System_Call}s are used to access this information.

\subsubsection{Communications}\label{subsubsec:Communications}
\subsubsection{Protection}\label{subsubsec:Protection}


%%% Local Variables:
%%% mode: latex
%%% TeX-master: "../../EDAF35-Operating_Systems-Reference_Sheet"
%%% End:


\subsection{System Programs}\label{subsec:System_Programs}
Another aspect of a modern system is its collection of system programs.
\begin{definition}[System Program]\label{def:System_Program}
  \emph{System programs}, also known as \emph{system utilities}, provide a convenient environment for program development and execution.
  Some of them are simply user interfaces to system calls.
  Others are considerably more complex.
  They can be divided into these categories:
  \begin{description}
  \item[File Management] These programs create, delete, copy, rename, print, dump, list, and generally manipulate files and directories.
  \item[Status Information] Some programs simply ask the system for the date, time, amount of available memory or disk space, number of users, or similar status information.
    Others are more complex, providing detailed performance, logging, and debugging information.
    Typically, these programs format and print the output to the terminal or other output devices or files or display it in a window of the GUI.\@
    Some systems also support a registry, which is used to store and retrieve configuration information.
  \item[File Modification] Several text editors may be available to create and modify the content of files stored on disk or other storage devices.
    There may also be special commands to search contents of files or perform transformations of the text.
  \item[Programming-Language Support] Compilers, assemblers, debuggers, and interpreters for common programming languages (such as C, C++, Java, and PERL) are often provided with the operating system or available as a separate download.
  \item[Program Loading and Execution] Once a program is assembled or compiled, it must be loaded into memory to be executed.
    Debugging systems for either higher-level languages or machine language are needed as well.
  \item[Communications] These programs provide the mechanism for creating virtual connections among processes, users, and computer systems.
    They allow users to send messages to one another’s screens, to browse Web pages, to send e-mail messages, to log in remotely, or to transfer files from one machine to another.
  \item[Background Services] All general-purpose systems have methods for launching certain \nameref{def:System_Program} processes at boot time.
    Some of these processes terminate after completing their tasks, while others continue to run until the system is halted.
    These are typically called \nameref{def:Daemon}s, and systems have dozens of them.
    In addition, operating systems that run important activities in user context rather than in kernel context may use \nameref{def:Daemon}s to run these activities.
  \end{description}
\end{definition}

%%% Local Variables:
%%% mode: latex
%%% TeX-master: "../../EDAF35-Operating_Systems-Reference_Sheet"
%%% End:


\subsection{Operating System Design and Implementation}\label{subsec:OS_Design_Implementation}
One important principle is the separation of \nameref{def:Policy} from \nameref{def:Mechanism}.
\begin{definition}[Mechanism]\label{def:Mechanism}
  A \emph{mechanism} determines how to do something.
\end{definition}

\begin{definition}[Policy]\label{def:Policy}
  A \emph{policy} determines \textbf{what} will be done given the \nameref{def:Mechanism} works correctly.
\end{definition}

The separation of \nameref{def:Policy} and \nameref{def:Mechanism} is important for system flexibility.
Policies are likely to change across places or over time.
In the worst case, each change in policy would require a change in the underlying mechanism.
A general mechanism insensitive to changes in policy would be more desirable.
A change in policy would then require redefinition of only certain parameters of the system.
For instance, consider a mechanism for giving priority to certain types of programs over others.
If the mechanism is properly separated from policy, it can be used either to support a policy decision that I/O-intensive programs should have priority over CPU-intensive ones or to support the opposite policy.

The advantages of using a higher-level language, or at least a systems-implementation language, for implementing operating systems are the same as those gained when the language is used for application programs:
\begin{itemize}[noitemsep]
\item The code can be written faster
\item Is more compact
\item Is easier to understand and debug
\end{itemize}

In addition, improvements in compiler technology will improve the generated code for the entire operating system by simple recompilation.
Finally, an \nameref{def:Operating_System} is far easier to port—to move to some other hardware —
if it is written in a higher-level language.

\begin{definition}[Port]\label{def:Software_Port}
  A \emph{port} is the process of moving a piece of software that was written for one piece of \nameref{def:Hardware} to another.
  In some cases, this only requires a recompilation of the higher-level software.
  In others, it may require completely rewriting the program.

  \begin{remark}[Port Confusion]\label{rmk:Software_Port_Confusion}
    It is important to note that the \nameref{def:Software_Port} is \textbf{\emph{NOT}} the same thing as a \nameref{def:Network_Port}.
  \end{remark}
\end{definition}

%%% Local Variables:
%%% mode: latex
%%% TeX-master: "../../EDAF35-Operating_Systems-Reference_Sheet"
%%% End:


\subsection{Operating System Structure}\label{subsec:OS_Structure}
A system as large and complex as a modern operating system must be engineered carefully if it is to function properly and be modified easily.

\subsubsection{Monolithic Approach}\label{subsubsec:Monolithic_Approach}
Many operating systems do not have well-defined structures.
Frequently, such systems started as small, simple, and limited systems and then grew beyond their original scope.
MS-DOS is an example of such a system.
It was originally designed and implemented by a few people who had no idea that it would become so popular.
It was written to provide the most functionality in the least space, so it was not carefully divided into modules.

\begin{definition}[Monolithic Kernel]\label{def:Monolithic_Kernel}
  A \emph{monolithic kernel} is an \nameref{def:Operating_System} architecture where the entire operating system is working in \nameref{def:Kernel} space, and typically uses only its own memory space to run.
  The monolithic model differs from other operating system architectures (such as the \nameref{def:Microkernel}) in that it alone defines a high-level virtual interface over computer hardware.
  A set of \nameref{def:System_Call}s implement all \nameref{def:Operating_System} services such as process management, concurrency, and memory management.

  Device drivers can be added to the \nameref{def:Kernel} as \nameref{def:Kernel_Module}s.
\end{definition}

In MS-DOS, the interfaces and levels of functionality are not well separated.
For instance, application programs are able to access the basic I/O routines to write directly to the display and disk drives.
Such freedom leaves MS-DOS vulnerable to errant (or malicious) programs, causing entire system crashes when user programs fail.

However, this was partly because MS-DOS was also limited by the hardware of its era.
Because the Intel~8088 for which it was written provides no dual mode and no hardware protection, the designers of MS-DOS had no choice but to leave the base hardware accessible.

\subsubsection{Layered Approach}\label{subsubsec:Layered_Approach}
With proper hardware support, \nameref{def:Operating_System}s can be broken into pieces that are smaller and more appropriate than those allowed by the original MS-DOS and UNIX systems.
The \nameref{def:Operating_System} can then retain much greater control over the computer and over the applications that make use of that computer.
Implementers have more freedom in changing the inner workings of the system and in creating modular \nameref{def:Operating_System}s.
Under a top-down approach, the overall functionality and features are determined and are separated into components.
Information hiding is also important, because it leaves programmers free to implement the low-level routines as they see fit, provided that the external interface of the routine stays unchanged and that the routine itself performs the advertised task.

A system can be made modular in many ways.
One method is the layered approach, in which the \nameref{def:Operating_System} is broken into a number of layers.
The bottom layer (layer 0) is the hardware; the highest (layer N) is the user interface.

A typical operating-system layer, layer $M$ consists of data structures and a set of routines that can be invoked by higher-level layers.
Layer $M$, in turn, can \textbf{\emph{ONLY}} invoke operations on lower-level layers and itself.

The main advantage of the layered approach is simplicity of construction and debugging.
The layers are selected so that each uses functions and services of only lower-level layers.
This approach simplifies debugging and system verification.
The first layer can be debugged without any concern for the rest of the system.
Once the first layer is debugged, its correct functioning can be assumed while the second layer is debugged, and so on.
If an error is found during the debugging of a particular layer, the error must be on that layer, because the layers below it are already debugged.
Thus, the design and implementation of the system are simplified.
Each layer is implemented only with operations provided by lower-level layers.
A layer does not need to know how these operations are implemented; it needs to know only what these operations do.

The major difficulty with the layered approach involves appropriately defining the various layers.
Because a layer can use only lower-level layers, careful planning is necessary.
Even with planning, there can be circular dependencies created between layers.
For example, the backing-store driver would normally be above the CPU scheduler, because the driver may need to wait for I/O and the CPU can be rescheduled during this time.
However, the CPU scheduler may have more information about all the active processes than can fit in memory.
Therefore, this information may need to be swapped in and out of memory, requiring the backing-store driver routine to be below the CPU scheduler.

A final problem with layered implementations is that they tend to be less efficient than other types.

\subsubsection{Microkernels}\label{subsubsec:Microkernels}
This method structures the \nameref{def:Operating_System} by removing all nonessential components from the \nameref{def:Kernel} and implementing them as system and user-level programs, resulting in a smaller \nameref{def:Kernel}.
There is little consensus regarding which services should remain in the kernel and which should be implemented in user space.
Typically, however, microkernels provide minimal process and memory management, in addition to a communication facility.

\begin{definition}[Microkernel]\label{def:Microkernel}
  A \emph{microkernel} (often abbreviated as $\mu$-kernel) is the near-minimum amount of software that can provide the mechanisms needed to implement an \nameref{def:Operating_System}.
  These mechanisms include:
  \begin{itemize}[noitemsep]
  \item Low-level address space management
  \item Thread management
  \item Inter-Process Communication (IPC)
  \end{itemize}

  If the hardware provides multiple rings or CPU modes, the microkernel may be the only software executing at the most privileged level, which is generally referred to as supervisor or kernel mode.
  Traditional \nameref{def:Operating_System} functions, such as device drivers, protocol stacks and file systems, are typically removed from the microkernel itself and are instead run in user space.

  In terms of the source code size, microkernels are often smaller than monolithic kernels.
\end{definition}

The main function of the \nameref{def:Microkernel} is to provide communication between the client program and the various services that are also running in user space.
Communication is provided through \nameref{par:Message_Passing}.

\subsubsection{Kernel Modules}\label{subsubsec:Kernel_Modules}
\subsubsection{Hybrid Systems}\label{subsubsec:Hybrid_Systems}

%%% Local Variables:
%%% mode: latex
%%% TeX-master: "../../EDAF35-Operating_Systems-Reference_Sheet"
%%% End:


\subsection{Operating System Debugging}\label{subsec:OS_Debugging}
\subsubsection{Failure Analysis}\label{subsubsec:Failure_Analysis}
If a process fails, most \nameref{def:Operating_System}s write the error information to a log file to alert \nameref{def:User}s that the problem occurred.
The operating system can also take a \nameref{def:Core_Dump}—— and store it in a file for later analysis.

\begin{definition}[Core Dump]\label{def:Core_Dump}
  A \emph{core dump} captures the memory of the process right as it fails and writes it to a disk.

  \begin{remark}[Why Core?]\label{rmk:Why_Core_Dump}
    The reason a \nameref{def:Core_Dump} is named the way it is is because memory was referred to as the ``core'' in the early days of computing.
  \end{remark}
\end{definition}

Running programs and \nameref{def:Core_Dump}s can be probed by a debugger, which allows a programmer to explore the code and memory of a process.
Operating-system kernel debugging is more complex than usual because of:
\begin{itemize}[noitemsep]
\item The size of the \nameref{def:Kernel}
\item The complexity of the \nameref{def:Kernel}
\item The \nameref{def:Kernel}'s control of the hardware
\item The lack of user-level debugging tools.
\end{itemize}


\begin{definition}[Crash]\label{def:Crash}
  A failure in the \nameref{def:Kernel} is called a \emph{crash}.
\end{definition}

When a \nameref{def:Crash} occurs, error information is saved to a log file, and the memory state is saved to a \nameref{def:Crash_Dump}.

\begin{definition}[Crash Dump]\label{def:Crash_Dump}
  When a \nameref{def:Crash} occurs in the \nameref{def:Kernel}, a \emph{crash dump} is generated.
  This is like a \nameref{def:Core_Dump}, in that the entire contents of that process's \nameref{def:Memory} is written to disk, except the \nameref{def:Crash}ed \nameref{def:Kernel} process is written, instead of a \nameref{def:User} program.
\end{definition}

Operating-system debugging and process debugging frequently use different tools and techniques due to the very different nature of these two tasks.

\subsubsection{Performance Tuning}\label{subsec:Performance_Tuning}
Performance tuning seeks to improve performance by removing processing bottlenecks.
To identify bottlenecks, we must be able to monitor system performance.
Thus, the \nameref{def:Operating_System} must have some means of computing and displaying measures of system behavior.
In a number of systems, the operating system does this by producing trace listings of system behavior.
All interesting events are logged with their time and important parameters and are written to a file.
Later, an analysis program can process the log file to determine system performance and to identify bottlenecks and inefficiencies.
Traces also can help people to find errors in operating-system behavior.

Another approach to performance tuning uses single-purpose, interactive tools that allow users and administrators to question the state of various system components to look for bottlenecks.
One such tool employs the UNIX command \texttt{top} to display the resources used on the system, as well as a sorted list of the ``top'' resource-using processes.

%%% Local Variables:
%%% mode: latex
%%% TeX-master: "../../EDAF35-Operating_Systems-Reference_Sheet"
%%% End:


\subsection{System Boot}\label{subsec:System_Boot}
The procedure of starting a computer by loading the \nameref{def:Kernel} is known as booting the system.
On most computer systems, a small piece of code known as the \nameref{def:Bootloader} is the first thing that runs.

\begin{definition}[Bootloader]\label{def:Bootloader}
  The \emph{bootloader} (or bootstrap loader) is a bootstrap program that:
  \begin{enumerate}[noitemsep]
  \item Locates the kernel
  \item Loads it into main memory
  \item Starts its execution
  \end{enumerate}
\end{definition}

Some computer systems, such as PCs, use a two-step process in which a simple \nameref{def:Bootloader} fetches a more complex boot program from disk, which in turn loads the \nameref{def:Kernel}.

When a CPU receives a reset event, the instruction register is loaded with a predefined memory location, and execution starts there.
At that location is the initial \nameref{def:Bootloader} program.
This program is in the form of read-only memory (ROM), because the RAM is in an unknown state at system startup.
ROM is convenient because it needs no initialization and cannot easily be infected by a computer virus.

\begin{remark*}
  A reset event on the CPU can be the computer having just booted, or it has been restarted, or the reset switched was flipped.
\end{remark*}

The \nameref{def:Bootloader} can perform a variety of tasks.
Usually, one task is to run diagnostics to determine the state of the machine.
If the diagnostics pass, the program can continue with the booting steps.
It also initializes all aspects of the system, from CPU registers to device controllers and the contents of main memory.
Sooner or later, it starts the \nameref{def:Operating_System}.
Cellular phones, tablets, and game consoles store the entire operating system in ROM.\@
Storing the operating system in ROM is suitable only for:
\begin{itemize}[noitemsep]
\item Small operating systems
\item Simple supporting hardware
\item Ensuring rugged operation
\end{itemize}

A problem with this approach is that changing the bootstrap code requires changing the ROM hardware chips.

All forms of ROM are also known as \nameref{def:Firmware}.
A problem with firmware in general is that executing code there is slower than executing code in RAM.\@
Some systems store the \nameref{def:Operating_System} in firmware and copy it to RAM for fast execution.

A final issue with \nameref{def:Firmware} is that it is relatively expensive, so usually only small amounts are available.
For large operating systems, or for systems that change frequently, the \nameref{def:Bootloader} is stored in \nameref{def:Firmware}, and the \nameref{def:Operating_System} is on disk.
In this case, the bootstrap runs diagnostics and has a bit of code that can read a single block at a fixed location (say block zero) from disk into memory and execute the code from that boot block.
The program stored in the boot block may be sophisticated enough to load the entire operating system into memory and begin its execution.

More typically, it is simple code (as it fits in a single disk block) and knows only the address on disk and length of the remainder of the bootstrap program.
GRUB is an example of an open-source \nameref{def:Bootloader} program for Linux systems.
All of the disk-bound bootstrap, and the \nameref{def:Operating_System} that is loaded, can be easily changed by writing new versions to disk.
A disk that has a boot partition is called a boot disk or system disk.
Now that the full bootstrap program has been loaded, it can traverse the file system to find the \nameref{def:Operating_System}'s \nameref{def:Kernel}, load it into \nameref{def:Memory}, and start its execution.
It is only at this point that the system is said to be running.

%%% Local Variables:
%%% mode: latex
%%% TeX-master: "../../EDAF35-Operating_Systems-Reference_Sheet"
%%% End:


%%% Local Variables:
%%% mode: latex
%%% TeX-master: "../EDAF35-Operating_Systems-Reference_Sheet"
%%% End:


\section{CPU Scheduling and Synchronization}\label{sec:CPU_Scheduling_Synchronization}
The growing importance of multicore systems has brought an increased emphasis on developing multithreaded applications.
In such applications, several threads, which may be sharing data, are running in parallel on different processing cores.
These \nameref{def:Process}es are called \nameref{def:Cooperating_Process}es.

\begin{definition}[Cooperating Process]\label{def:Cooperating_Process}
  A \emph{cooperating process} is one that can affect or be affected by other \nameref{def:Process}es executing on a system.
  They can share a logical address space (code and data), \textbf{\nameref{def:Thread}s}, or can share data through files and/or messages, \textbf{\nameref{subsubsec:Communications}}.
\end{definition}

Given the way that multiple \nameref{def:Thread}s can be scheduled, namely in any order (relatively speaking), as programmers, we cannot be certain about which thread will be scheduled first.
This leads to all sorts of problems because of sharing information between multiple users.
The largest, and likely the most common, error in a multi\nameref{def:Thread}ed program is the \nameref{def:Race_Condition}.

\begin{definition}[Race Condition]\label{def:Race_Condition}
  A \emph{race condition} is when several processes access and manipulate the same data concurrently and the outcome of the execution depends on the particular order in which the access takes place.
  The only way to prevent a race condition is to ensure that \textbf{only one \nameref{def:Thread} can change the value at a time}.
\end{definition}

\subsection{Process/Thread Synchronization}\label{subsec:Synchronization}
The main problem that occurs in multi\nameref{def:Thread}ed programs is that there is a small portion of code that is a \nameref{def:Critical_Section}.
This leads to the development of the \nameref{subsubsec:Critical_Section_Problem}.

\begin{definition}[Critical Section]\label{def:Critical_Section}
  The \emph{critical section} of a \nameref{def:Process} is a portion where the \nameref{def:Thread} and/or \nameref{def:Process} is changing common variables, updating a table, writing a file, or other global state changes.
\end{definition}

\subsubsection{Critical Section Problem}\label{subsubsec:Critical_Section_Problem}
The \emph{Critical Section Problem} is the issue of coordinating multiple \nameref{def:Thread}s about a \nameref{def:Critical_Section} of the code.
The problem is to design a protocol that the \nameref{def:Process}es/\nameref{def:Thread}s can use to cooperate.
Each \nameref{def:Process} must request permission to enter its critical section.
The section of code implementing this request is the \nameref{def:Entry_Section}.
The critical section may be followed by an \nameref{def:Exit_Section}.
The remaining code is the \nameref{def:Remainder_Section}.

\begin{definition}[Entry Section]\label{def:Entry_Section}
  The \emph{entry section} of a \nameref{def:Process} is the portion where the request to execute the \nameref{def:Critical_Section} occurs.
  In the case of a \nameref{def:Mutex}, this is the process of aquiring the it.
  For a \nameref{def:Semaphore}, it is the process of manipulating the value it currently contains.
\end{definition}

\begin{definition}[Exit Section]\label{def:Exit_Section}
  In the \emph{exit section}, the constructs used to ensure coordination in the \nameref{def:Critical_Section} are freed.
  In the case of a \nameref{def:Mutex}, this is the process of releasing the it.
  For a \nameref{def:Semaphore}, it is the process of manipulating the value it currently contains in the opposite direction it was initially manipulated by.
\end{definition}

\begin{definition}[Remainder Section]\label{def:Remainder_Section}
  The \emph{remainder section} is the rest of the code, after this \nameref{def:Critical_Section}.
  This code may be parallelized, or not.
  It could contain further \nameref{def:Critical_Section}s.
\end{definition}

Any solution to this problem \textbf{MUST} satisfy one of the following 3 requirements:
\begin{enumerate}[noitemsep]
\item \textbf{Mutual Exclusion}.
  If \nameref{def:Process} $P_{i}$ is executing its \nameref{def:Critical_Section}, then \textbf{no other} processes can execute their critical sections.
\item \textbf{Progress}.
  If no \nameref{def:Process} is executing its \nameref{def:Critical_Section}, and some processes wish to enter their critical sections, then only those processes that \textbf{are not executing} in their \nameref{def:Remainder_Section}s can decide which will enter the \nameref{def:Critical_Section} next.
  Essentially, the only way a process gets a voice in the choice is by not having executed the critical section yet.
\item \textbf{Bounded Waiting}.
  There exists a bound on the number of times that other \nameref{def:Process}es are allowed to enter their \nameref{def:Critical_Section}s after a process has made a request to enter its critical section and before that request is granted.
\end{enumerate}

To handle the \nameref{subsubsec:Critical_Section_Problem}, there are 2 main types of \nameref{def:Kernel}s that present solutions.
\begin{enumerate}[noitemsep]
\item \nameref{def:Nonpreemptive_Kernel}s. Not used frequently today.
\item \nameref{def:Preemptive_Kernel}s. The most common type today.
\end{enumerate}

\begin{definition}[Nonpreemptive Kernel]\label{def:Nonpreemptive_Kernel}
  A \emph{nonpreemptive kernel} is a \nameref{def:Kernel} that does \textbf{NOT} use \nameref{def:Preemption} on \nameref{def:Process}es or \nameref{def:Thread}s running in kernel-mode.
\end{definition}

\begin{definition}[Preemptive Kernel]\label{def:Preemptive_Kernel}
  A \emph{preemptive kernel} is a \nameref{def:Kernel} that uses \nameref{def:Preemption} on \nameref{def:Process}es or \nameref{def:Thread}s running in kernel-mode.
  This means that we cannot say anything definitive about the state of the \nameref{def:Kernel}'s data structures at a given time, because we cannot say which process/thread is running at that time.
\end{definition}

\subsubsection{Hardware Support for Synchronization}\label{subsubsec:Hardware_Support_Synchronization}
Software-based solutions to handling multithreading and multiprocessing tends to be better than hardware-based solutions, as they are more flexible.
Many of the solutions that will be presented here are based on the idea of \textbf{\nameref{def:Lock}ing}.

\begin{definition}[Lock]\label{def:Lock}
  A \emph{lock} allows \textbf{only one} \nameref{def:Thread} to enter the portion of code that is locked.
  While a thread holds this lock no other \nameref{def:Thread} can execute on this code portion.

  \begin{remark}[Binary Semaphore]\label{rmk:Binary_Semaphore}
    Locks can be represented as \emph{binary \nameref{def:Semaphore}}s.
  \end{remark}
\end{definition}

In a single-processor system, we can solve the \nameref{subsubsec:Critical_Section_Problem} by preventing interrupts from being handled.
This would prevent the currently running instruction from being preempted in any way, and allow it to finish.
However, this does not really work on a multiprocessor system, because disabling interrupts and their handling on all processors is time consuming.

However, the idea of certain instructions being \nameref{def:Atomic} is an elegant solution to the \nameref{subsubsec:Critical_Section_Problem}.
So, most computer systems provide special hardware-level instructions that allow us to test and modify the contents of a word, or swap the contents of 2 words \nameref{def:Atomic}ally.

\begin{definition}[Atomic]\label{def:Atomic}
  An \emph{atomic} operation is one that cannot be interrupted, preempted, or altered in any way.
  As soon as an atomic operation begins, the system \textbf{MUST} finish handling it before it may do anything else.
\end{definition}

Some operations on data are possible to do at any given point in time, without affecting the potential outcome.
One example of this is \textbf{reading} from a location in memory.
However, if this location can also be written to, we need to limit the number of writers.
Additionally, if someone is waiting to write, they should get some priority over anything waiting to read.
Thus, the \nameref{def:Read_Write_Lock} was created.

\begin{definition}[Read/Write Lock]\label{def:Read_Write_Lock}
  \emph{Read/Write Lock}s allow either an unlimited number of readers \textbf{OR} 1 writer at any given time.
  Writers will be scheduled to use the lock sooner than readers, so the value is updated first, before anyone reads it again.
  But, the writer will have to wait until everyone currently reading the value is done reading, otherwise the value in memory will change underneath the readers.
\end{definition}

\subsubsection{Mutex Locks}\label{subsubsec:Mutex_Locks}
The hardware-based solutions presented in \Cref{subsubsec:Hardware_Support_Synchronization} are typically not available to application programmers.
Instead, operating system designers build software tools to handle the \nameref{subsubsec:Critical_Section_Problem}.
The simplest tool is that of a \nameref{def:Mutex}.

%%% Local Variables:
%%% mode: latex
%%% TeX-master: "../../EDAF35-Operating_Systems-Reference_Sheet"
%%% End:


\subsection{Scheduling}\label{subsec:Scheduling}
CPU scheduling is the basis of multiprogrammed operating systems.
In a single-processor system, only one process can run at a time.
Others must wait until the CPU is free and can be rescheduled.
By switching the CPU among processes, the operating system can maximize CPU utilization.

For example, a \nameref{def:Process} is executed until it must wait.
Typically the process waits for the completion of some I/O request.
In a simple computer system, the CPU then just sits idle.
All this waiting time is wasted; no useful work is accomplished.

\begin{blackbox}
  \textbf{On operating systems that support them, \nameref{def:Kernel_Thread}s, not \nameref{def:Process}es are scheduled by the operating system.}
  However, the terms ``process scheduling'' and ``thread scheduling'' are often used interchangeably.
  Process scheduling is used when discussing general scheduling concepts and thread scheduling to refer to thread-specific ideas.
\end{blackbox}

Scheduling of this kind is a fundamental operating-system function.
Almost all computer resources are scheduled before use.

\subsubsection{CPU and I/O Bursts}\label{subsubsec:CPU_IO_Bursts}
To properly schedule a \nameref{def:Process}, its \nameref{def:CPU_Burst}s and \nameref{def:I/O_Burst}s need to observed.

\begin{definition}[CPU Burst]\label{def:CPU_Burst}
  A \emph{CPU burst} is one of the states of execution for a \nameref{def:Process}.
  This is the state when the process is actively using the CPU to perform computations.
  In this state, the CPU is performing activity for \textbf{this} \nameref{def:Process}, and \textbf{IS NOT} waiting for an I/O device to perform some action or return information.
\end{definition}

\begin{definition}[I/O Burst]\label{def:I/O_Burst}
  An \emph{I/O burst} is one of the states of execution for a \nameref{def:Process}.
  This is the state when the process is waiting on the I/O device to return the requested information or perform the desired action.
  In this state, the CPU is doing no activity for \textbf{this} \nameref{def:Process}.
\end{definition}

A \nameref{def:Process} alternates between these two bursts, with the final \nameref{def:CPU_Burst} terminating this \nameref{def:Process}'s execution.
The distribution of length of CPU bursts is an exponential or hyperexponential graph.
This means:
\begin{itemize}[noitemsep]
\item There is a large number of short duration CPU bursts.
\item There is a small number of long duration CPU bursts.
\end{itemize}

We can categorize these into either \nameref{def:CPU_Bound} programs or \nameref{def:IO_Bound} programs.
\begin{itemize}[noitemsep]
\item I/O-bound programs have a small number of CPU bursts which have a relatively short duration relative to the I/O operations.
  The I/O operations take up a majority of the time the \nameref{def:Process} executes.
\item CPU-bound programs have a large number of CPU bursts, which have a relatively long duration relative to the I/O operations.
  The CPU operatiosn take up a majority of the time the \nameref{def:Process} executes.
\end{itemize}

\subsubsection{CPU Scheduler}\label{subsubsec:CPU_Scheduler}
Whenever the CPU becomes idle, i.e.\ it has finished the current CPU burst early, or there is an I/O operation, the \nameref{def:Operating_System} must select the next \nameref{def:Process} and/or \nameref{def:Thread} to schedule.
This is handled by the \nameref{def:Short_Term_Scheduler}.

\begin{definition}[Short-Term Scheduler]\label{def:Short_Term_Scheduler}
  The \emph{short-term scheduler} is responsible for scheduling either the next \nameref{def:Process} or \nameref{def:Thread} for execution on the \textbf{CPU} from all the possible ones in memory.
  This is run quite frequently, every couple hundred milliseconds, usually.

  \begin{remark}[CPU Scheduler]\label{rmk:CPU_Scheduler}
    Because the \nameref{def:Short_Term_Scheduler} only schedules tasks for the CPU, it is also called the \emph{CPU Scheduler}.
  \end{remark}
\end{definition}

\paragraph{Preemption and Scheduling}\label{par:Preemption_Scheduling}
There are 4 times when CPU scheduling occurs:
\begin{enumerate}[noitemsep]
\item When a process switches from the \texttt{RUNNING} state to the \texttt{WAITING} state.
\item When a process switches from the \texttt{RUNNING} state to the \texttt{READY} state (for example, when an interrupt occurs).
\item When a process switches from the \texttt{WAITING} state to the \texttt{READY} state (for example, at completion of I/O).
\item When a process terminates.
\end{enumerate}

%%% Local Variables:
%%% mode: latex
%%% TeX-master: "../../EDAF35-Operating_Systems-Reference_Sheet"
%%% End:


\subsection{Thread Scheduling}\label{subsec:Thread_Scheduling}

%%% Local Variables:
%%% mode: latex
%%% TeX-master: "../../EDAF35-Operating_Systems-Reference_Sheet"
%%% End:


\subsection{Real-Time Scheduling}\label{subsec:Real_Time_Scheduling}
Real-time operating systems have their own class of scheduling issues.
This depends on whether the \nameref{def:Operating_System} is a \nameref{def:Soft_Real_Time_System} or a \nameref{def:Hard_Real_Time_System}.

\begin{definition}[Soft Real-Time System]\label{def:Soft_Real_Time_System}
  \emph{Soft real-time system}s do not provide a guarantee about the scheduling of a critical real-time process.
\end{definition}

\begin{definition}[Hard Real-Time System]\label{def:Hard_Real_Time_System}
  \emph{Hard real-time system}s guarantee the execution time of a real-time process.
  These tasks will be serviced by its deadline, otherwise the process will not be executed at all.
\end{definition}

POSIX also provides support for real-time scheduling through 2 functions with 2 scheduling types.
\begin{enumerate}[noitemsep]
\item \kernelinline{pthread_attr_getsched_policy(pthread_attr_t *attr, int *policy)}
\item \kernelinline{pthread_attr_setsched_policy(pthread_attr_t *attr, int policy)}
\end{enumerate}
\begin{enumerate}[noitemsep]
\item \texttt{SCHED\_FIFO}
\item \texttt{SCHED\_RR}
\end{enumerate}

\subsubsection{Minimizing Latency}\label{subsubsec:Minimizing_Latency}
The key aspect here is the amount of time it takes for a system to respond to an event.
This is called \nameref{def:Event_Latency}.

\begin{definition}[Event Latency]\label{def:Event_Latency}
  \emph{Event latency} is the amount of time that elapses from when an event occurs to when it is serviced.
  Different events can have different event latency requirements.
\end{definition}

There are 2 factors that affect \nameref{def:Event_Latency}.
\begin{enumerate}[noitemsep]
\item Interrupt Latency.
  The amount of time from the arrival of an \nameref{def:Interrupt} to the start of the Interrupt Service Routine (ISR).
  This includes the amount of time needed to get the currently running instruction to a point where it can be switched.
  Also included is the amount of time needed to perform the switch.
\item Dispatch Latency the amount of time the scheduler needs to stop one process and start another.
  There are 2 parts that affect the value of the dispatch latency:
  \begin{enumerate}[noitemsep]
  \item \nameref{def:Preemption} of \textbf{ANY} process running tin the kernel.
  \item Release of resources used by low-priority process for higher-priority processes.
  \end{enumerate}
\end{enumerate}

\subsubsection{Scheduling}\label{subsubsec:Real_Time_Scheduling}
In this case, there are not as many choices of \nameref{def:Scheduling_Algorithm} for real-time systems as other systems.
All algorithms must be roughly based on a priority-based system all of which must support \nameref{def:Preemption}.

Most modern \nameref{def:Operating_System}s offer support for \nameref{def:Soft_Real_Time_System}s with their scheduling priorities.
Note however, that pure priority-based algorithms only guarantee soft real-time functionality, not hard.

Processes are considered periodic; they require the CPU at constant intervals.
Once a periodic process has acquired the CPU, it has a fixed processing time $t$, a deadline $d$ by which it must be serviced by the CPU, and a period $p$.
The relationship of the processing time, the deadline, and the period can be expressed as $0 \leq t \leq d \leq p$.
The rate of a periodic task is $\frac{1}{p}$.
Schedulers can take advantage of these characteristics and assign priorities according to a process’s deadline or rate requirements.
What is different about this algorithm is a process may have to announce its deadline to the scheduler.
Using an admission-control algorithm, the scheduler either admits the process, guaranteeing that the process will complete on time, or rejects the request if it cannot guarantee that the task will be serviced by its deadline.

\paragraph{Rate-Monotonic Scheduling}\label{par:Rate_Monotonic_Scheduling}
\begin{definition}[Rate-Monotonic Scheduling]\label{def:Rate_Monotonic_Scheduling}
  \emph{Rate-monotonic scheduling} is a \nameref{def:Scheduling_Algorithm} for periodic tasks that uses a static priority policy with preemption.
  If a higher-priority process arrives, and a lower priority one is running, it is immediately preempted.
  The priority is statically calculated based on the inverse of the period of the task.
  Less frequent (longer period) tasks have a lower priority, and more frequent ones have higher priority.
  Additionally, the size of the CPU burst is assumed to be constant during every period.
\end{definition}

%%% Local Variables:
%%% mode: latex
%%% TeX-master: "../../EDAF35-Operating_Systems-Reference_Sheet"
%%% End:


\subsection{Algorithm Evaluation}\label{subsec:Algorithm_Evaluation}
Now that we have selected a \nameref{def:Scheduling_Algorithm} to use, how do we know that it was the right choice?
First, we need to know what our criteria were.
Some systems might have multiple criteria at a time, such as:
\begin{itemize}[noitemsep]
\item Maximum CPU response time is 1 second.
\item Turnaround time is (on average) linearly proportional to total execution time.
\end{itemize}

To do this, there are 4 main ways to do this:
\begin{enumerate}[noitemsep]
\item \nameref{par:Deterministic_Modeling}
\item \nameref{par:Queuing_Models}
\item \nameref{par:Simulations}
\item \nameref{par:Implementation}
\end{enumerate}

\paragraph{Deterministic Modeling}\label{par:Deterministic_Modeling}
Deterministic modeling is simple and fast.
It gives us exact numbers, allowing us to compare the algorithms.
However, it requires \textbf{exact numbers for input}, and its answers apply \emph{only} to those cases.

This method takes a particular predetermined workload and defines the performance of each algorithm for that workload.

The main uses of deterministic modeling are in describing \nameref{def:Scheduling_Algorithm}s and providing examples.
In cases where we are running the same program repeatedly, we can measure the program’s processing requirements exactly, allowing us to select a \nameref{def:Scheduling_Algorithm}.
Furthermore, over a set of examples, deterministic modeling may indicate trends that can then be analyzed and proved separately.


%%% Local Variables:
%%% mode: latex
%%% TeX-master: "../../EDAF35-Operating_Systems-Reference_Sheet"
%%% End:


\begin{definition}[Deadlock]\label{def:Deadlock}
  \emph{Deadlock} is when 2 processes require information from each other to continue running.
  If this happens, neither process will provide the other with its required information, so they will both wait for each other, forever.
\end{definition}
\subsection{Deadlocks}\label{subsec:Deadlocks}
\nameref{def:Deadlock} is a serious issue in \nameref{rmk:CPU_Scheduler}s because \nameref{def:Process}es lock resources for themselves.
A good example of a deadlock is ``When two trains approach each other at a crossing, both shall come to a full stop and neither shall start up again until the other has gone.''

\begin{definition}[Deadlock]\label{def:Deadlock}
  \emph{Deadlock} is when 2 processes require information or resources from each other to continue running.
  If this happens, neither process will provide the other with its required information, so they will both wait for each other, forever.
\end{definition}

There are only 2 options for handling \nameref{def:Deadlock}s:
\begin{enumerate}[noitemsep]
\item Prevent them from happening in the first place.
\item Identify them and fix the problem that is causing them.
\item Hope they don't happen and consider them as unlikely events to occur.
  \begin{itemize}[noitemsep]
  \item This is what most desktop \nameref{def:Operating_System}s do.
  \end{itemize}
\end{enumerate}

Most \nameref{def:Operating_System}s do \textbf{NOT} provide functionality to identify \nameref{def:Deadlock}s and correct them.

%%% Local Variables:
%%% mode: latex
%%% TeX-master: "../../EDAF35-Operating_Systems-Reference_Sheet"
%%% End:



%%% Local Variables:
%%% mode: latex
%%% TeX-master: "../EDAF35-Operating_Systems-Reference_Sheet"
%%% End:

%====================================APPENDIX====================================
\appendix
\counterwithin{definition}{subsection}

\clearpage
\section{Computer Components}\label{app:Computer_Components}
\subsection{Central Processing Unit}\label{subsec:CPU}
\begin{definition}[Central Processing Unit]\label{def:CPU}
  The \emph{Central Processing Unit}, \emph{CPU}, is a chip that performs all actions in the computer.
  It calculates mathematical and logical values and acts based on them.
  It has several components built onto it, and can be thought of as the ``brain'' of the computer.

  The design of a CPU determines some of the functionality it has.
  Therefore, more specialized processors can be made for special tasks, and more general processors can be built to handle a wide variety of calculations.
\end{definition}

\subsubsection{Registers}\label{subsubsec:Registers}
\begin{definition}[Register]\label{def:Register}
  A \emph{register} is a data storage mechanism built directly onto the \nameref{def:CPU}.
  It is several hundred times faster than the system \nameref{def:Memory}.
  Registers are generally used when the currently running program is performing calculations.
  Since they are so fast, they are used as both source and destination operands in instructions.

  \begin{remark}
    Depending on the \nameref{def:CPU} architecture, there may be cases when \nameref{def:Register}s behave slightly differently between processors.
    This is something that can only be found by checking the \nameref{def:CPU} manufacturer's documentation.
  \end{remark}
\end{definition}

\subsubsection{Program Counter}\label{subsubsec:Program_Counter}
\subsubsection{Arithmetic Logic Unit}\label{subsubsec:ALU}
\subsubsection{Cache}\label{subsubsec:CPU_Cache}

\subsection{Memory}\label{subsec:Memory}
\begin{definition}[Memory]\label{def:Memory}
  \emph{Memory}, or \emph{RAM} (\emph{Random Access Memory}), is a \nameref{def:Volatile} data storage mechanism.
  It is directly connected to the \nameref{def:CPU}.
  This is the location that the \nameref{def:CPU} writes to when it cannot or should not store something in the \nameref{def:CPU}'s \nameref{def:Register}s.

  \begin{remark}[Volatility]
    \nameref{def:Memory} is volatile because each of the cells is a small capacitor.
    In between the clock cycles on the \nameref{def:CPU} and \nameref{def:Memory}, the capacitors discharge.
    On the clock cycle, the capacitors are refreshed with electrical power, which does one of 2 things:
    \begin{enumerate}[noitemsep]
    \item Keep the data bits the same, 1 to 1.
    \item Update the data bits from 0 to 1.
    \end{enumerate}
  \end{remark}
\end{definition}

\begin{definition}[Volatile]\label{def:Volatile}
  If a data storage mechanism is called \emph{volatile}, it means that once the storage mechanism loses power, the data is lost.
  This is in contrast to \nameref{def:Non-Volatile} data storage mechanisms.
\end{definition}

\subsection{Disk}\label{subsec:Disk}
\begin{definition}[Non-Volatile]\label{def:Non-Volatile}
  If a data storage mechanism is called \emph{non-volatile}, it means that once the storage device loses power, the data is still safely stored.
  This is in contrast to \nameref{def:Volatile} data storage mechanisms.
\end{definition}

\subsection{Fetch-Execute Cycle}\label{subsec:Fetch_Execute_Cycle}
%%% Local Variables:
%%% mode: latex
%%% TeX-master: shared
%%% End:


\clearpage
\section{Complex Numbers}\label{sec:Complex_Numbers}
\begin{definition}[Complex Number]\label{def:Complex_Number}
  A \emph{complex number} is a hyper real number system.
  This means that two real numbers, $a, b \in \RealNumbers$, are used to construct the set of complex numbers, denoted $\ComplexNumbers$.

  A complex number is written, in Cartesian form, as shown in \Cref{eq:Complex_Number} below.
  \begin{equation}\label{eq:Complex_Number}
    z = a \pm ib
  \end{equation}
  where
  \begin{equation}\label{eq:Imaginary_Value}
    i = \sqrt{-1}
  \end{equation}

  \begin{remark*}[$i$ vs. $j$ for Imaginary Numbers]
    Complex numbers are generally denoted with either $i$ or $j$.
    Electrical engineering regularly makes use of $j$ as the imaginary value.
    This is because alternating current $i$ is already taken, so $j$ is used as the imaginary value instad.
  \end{remark*}
\end{definition}

\subsection{Parts of a Complex Number}\label{subsec:Complex_Number_Parts}
A \nameref{def:Complex_Number} is made of up 2 parts:
\begin{enumerate}[noitemsep]
\item \nameref{def:Real_Part}
\item \nameref{def:Imaginary_Part}
\end{enumerate}

\begin{definition}[Real Part]\label{def:Real_Part}
  The \emph{real part} of an imaginary number, denoted with the $\Re$ operator, is the portion of the \nameref{def:Complex_Number} with no part of the imaginary value $i$ present.

  If $z = x + iy$, then
  \begin{equation}\label{eq:Real_Part}
    \Real{z} = x
  \end{equation}

  \begin{remark}[Alternative Notation]\label{rmk:Real_Part_Alternative_Notation}
    The \nameref{def:Real_Part} of a number sometimes uses a slightly different symbol for denoting the operation.
    It is:
    \begin{equation*}
      \mathfrak{Re}
    \end{equation*}
  \end{remark}
\end{definition}

\begin{definition}[Imaginary Part]\label{def:Imaginary_Part}
  The \emph{imaginary part} of an imaginary number, denoted with the $\Im$ operator, is the portion of the \nameref{def:Complex_Number} where the imaginary value $i$ is present.

  If $z = x + iy$, then
  \begin{equation}\label{eq:Imaginary_Part}
    \Imag{z} = y
  \end{equation}

  \begin{remark}[Alternative Notation]\label{rmk:Imaginary_Part_Alternative_Notation}
    The \nameref{def:Imaginary_Part} of a number sometimes uses a slightly different symbol for denoting the operation.
    It is:
    \begin{equation*}
      \mathfrak{Im}
    \end{equation*}
  \end{remark}
\end{definition}

\subsection{Binary Operations}\label{subsec:Binary_Operations}

%%% Local Variables:
%%% mode: latex
%%% TeX-master: shared
%%% End:


\subsection{Complex Conjugates}\label{app:Complex_Conjugates}
\begin{definition}[Complex Conjugate]\label{def:Complex_Conjugate}
  The conjugate of a complex number is called its \emph{complex conjugate}.
  The complex conjugate of a complex number is the number with an equal real part and an imaginary part equal in magnitude but opposite in sign.
  If we have a complex number as shown below,
  \begin{equation*}
    z = a \pm bi
  \end{equation*}

  then, the conjugate is denoted and calculated as shown below.
  \begin{equation}\label{eq:Complex_Conjugates}
    \Conjugate{z} = a \mp bi
  \end{equation}
\end{definition}

The \nameref{def:Complex_Conjugate} can also be denoted with an asterisk ($*$).
This is generally done for complex functions, rather than single variables.
\begin{equation}\label{eq:Complex_Conjugates_Asterisk}
  z^{*} = \Conjugate{z}
\end{equation}

%%% Local Variables:
%%% mode: latex
%%% TeX-master: shared
%%% End:


\subsection{Geometry of Complex Numbers}\label{subsec:Geometry_Complex_Numbers}
So far, we have viewed \nameref{def:Complex_Number}s only algebraically.
However, we can also view them geometrically as points on a 2 dimensional \nameref{def:Argand_Plane}.

\begin{definition}[Argand Plane]\label{def:Argand_Plane}
  An \emph{Argane Plane} is a standard two dimensional plane whose points are all elements of the complex numbers, $z \in \ComplexNumbers$.
  This is taken from Descarte's definition of a completely real plane.

  The Argand plane contains 2 lines that form the axes, that indicate the real component and the imaginary component of the complex number specified.
\end{definition}

A \nameref{def:Complex_Number} can be viewed as a point in the \nameref{def:Argand_Plane}, where the \nameref{def:Real_Part} is the ``$x$''-component and the \nameref{def:Imaginary_Part} is the ``$y$''-component.

By plotting this, you see that we form a right triangle, so we can find the hypotenuse of that triangle.
This hypotenuse is the distance the point $p$ is from the origin, refered to as the \nameref{def:Complex_Number_Modulus}.
\begin{remark*}
  When working with \nameref{def:Complex_Number}s geometrically, we refer to the points, where they are defined like so:
  \begin{equation*}
    z = x + iy = p(x, y)
  \end{equation*}

  Note that $p$ is \textbf{not} a function of $x$ and $y$.
  Those are the values that inform us \textbf{where} $p$ is located on the \nameref{def:Argand_Plane}.
\end{remark*}

\subsubsection{Modulus of a Complex Number}\label{subsubsec:Complex_Number_Modulus}
\begin{definition}[Modulus]\label{def:Complex_Number_Modulus}
  The \emph{modulus} of a \nameref{def:Complex_Number} is the distance from the origin to the complex point $p$.
  This is based off the Pythagorean Theorem.
  \begin{equation}\label{eq:Complex_Number_Modulus}
    \begin{aligned}
      {\lvert z \rvert}^{2} = x^{2} + y^{2} &= z \Conjugate{z} \\
      \lvert z \rvert &= \sqrt{x^{2} + y^{2}}
    \end{aligned}
  \end{equation}
\end{definition}

\begin{propertylist}
\item The \emph{Law of Moduli} states that $\lvert z w \rvert = \lvert z \rvert \lvert w \rvert$.\label{prop:Law_of_Moduli}.
\end{propertylist}

We can prove \Cref{prop:Law_of_Moduli} using an algebraic identity.
\begin{proof}[Prove \Cref*{prop:Law_of_Moduli}]
  Let $z$ and $w$ be complex numbers ($z, w \in \ComplexNumbers$).
  We are asked to prove
  \begin{equation*}
    \lvert z w \rvert = \lvert z \rvert \lvert w \rvert
  \end{equation*}

  But, it is actually easier to prove
  \begin{equation*}
    {\lvert z w \rvert}^{2} = {\lvert z \rvert}^{2} {\lvert w \rvert}^{2}
  \end{equation*}

  We start by simplifying the ${\lvert z w \rvert}^{2}$ equation above.
  \begin{align*}
    {\lvert z w \rvert}^{2} &= {\lvert z \rvert}^{2} {\lvert w \rvert}^{2} \\
    \intertext{Using the definition of the \nameref{def:Complex_Number_Modulus} of a \nameref{def:Complex_Number} in \Cref{eq:Complex_Number_Modulus}, we can expand the modulus.}
                            &= (z w) (\Conjugate{z w}) \\
    \intertext{Using \Cref{prop:Complex_Conjugate_Split} for multiplication allows us to do the next step.}
                            &= (z w) (\Conjugate{z} \Conjugate{w}) \\
    \intertext{Using Multiplicative Associativity and Multiplicative Commutativity, we can simplify this further.}
                            &= (z \Conjugate{z}) (w \Conjugate{w}) \\
                            &= {\lvert z \rvert}^{2} {\lvert w \rvert}^{2}
  \end{align*}

  Note how we never needed to define $z$ or $w$, so this is as general a result as possible.
\end{proof}

\paragraph{Algebraic Effects of the Modulus' \Cref*{prop:Law_of_Moduli}}\label{par:Law_of_Moduli-Algebraic_Effects}
For this section, let $z = x_{1} + iy_{1}$ and $w = x_{2} + iy_{2}$.
Now,
\begin{align*}
  z w &= (x_{1}x_{2} - y_{1}y_{2}) + i(x_{1}y_{2} + x_{2}y_{1}) \\
  {\lvert z w \rvert}^{2} &= {(x_{1}x_{2} - y_{1}y_{2})}^{2} + {(x_{1}y_{2} + x_{2}y_{1})}^{2} \\
      &= \left( x_{1}^{2} + x_{2}^{2} \right) \left( x_{2}^{2} + y_{2}^{2} \right) \\
      &= {\lvert z \rvert}^{2} {\lvert w \rvert}^{2}
\end{align*}

However, the Law of Moduli (\Cref{prop:Law_of_Moduli}) does \textbf{not} hold for a hyper complex number system one that uses 2 or more imaginaries, i.e.\ $z = a + iy + jz$.
But, the Law of Moduli (\Cref{prop:Law_of_Moduli}) \textbf{does} hold for hyper complex number system that uses 3 imaginaries, $a = z + iy + jz + k \ell$.

\paragraph{Conceptual Effects of the Modulus' \Cref*{prop:Law_of_Moduli}}\label{par:Law_of_Moduli-Conceptual_Effects}
We are interested in seeing if $\lvert z w \rvert = (x_{1}^{2} + y_{1}^{2})(x_{2}^{2}+y_{2}^{2})$ can be extended to more complex terms (3 terms in the complex number).

However, Langrange proved that the equation below \textbf{always} holds.
Note that the $z$ below has no relation to the $z$ above.
\begin{equation*}
  (x_{1} + y_{1} + z_{1}) \neq X^{2} + Y^{2} + Z^{2}
\end{equation*}

%%% Local Variables:
%%% mode: latex
%%% TeX-master: shared
%%% End:


\subsection{Circles and Complex Numbers}\label{subsec:Circles_Complex_Numbers}
We need to define both a center and a radius, just like with regular purely real values.
\Cref{eq:Circles_Complex_Numbers} defines the relation required for a circle using \nameref{def:Complex_Number}s.
\begin{equation}\label{eq:Circles_Complex_Numbers}
  \lvert z - a \rvert = r
\end{equation}

\begin{example}[Lecture 2, Example 1]{Convert to Circle}
  Given the expression below, find the location of the center of the circle and the radius of the circle?
  \begin{equation*}
    \lvert 5 iz + 10 \rvert = 7
  \end{equation*}
  \tcblower{}
  This is just a matter of simplification and moving terms around.
  \begin{align*}
    \lvert 5 iz + 10 \rvert &= 7 \\
    \lvert 5i (z + \frac{10}{5i}) \rvert &= 7 \\
    \lvert 5i (z + \frac{2}{i}) \rvert &= 7 \\
    \lvert 5i (z + \frac{2}{i} \frac{-i}{-i}) \rvert &= 7 \\
    \lvert 5i (z - 2i) \rvert &= 7 \\
    \intertext{Now using the Law of Moduli (\Cref{prop:Law_of_Moduli}) $\lvert a b \rvert = \lvert a \rvert \lvert b \rvert$, we can simplify out the extra imaginary term.}
    \lvert 5i \rvert \lvert z-2i \rvert &= 7 \\
    5 \lvert z - 2i \rvert &= 7 \\
    \lvert z - 2i \rvert = \frac{7}{5}
  \end{align*}

  Thus, the circle formed by the equation $\lvert 5 iz + 10 \rvert = 7$ is actually $\lvert z - 2i \rvert = \frac{7}{5}$, with a center at $a = 2i$ and a radius of $\frac{7}{5}$.
\end{example}

\subsubsection{Annulus}\label{subsubsec:Annulus}
\begin{definition}[Annulus]\label{def:Annulus}
  An \emph{annulus} is a region that is bounded by 2 concentric circles.
  This takes the form of \Cref{eq:Annulus}.
  \begin{equation}\label{eq:Annulus}
    r_{1} \leq \lvert z - a \rvert \leq r_{2}
  \end{equation}

  In \Cref{eq:Annulus}, each of the $\leq$ symbols could also be replaced with $<$.
  This leads to 3 different possibilities for the annulus:
  \begin{enumerate}[noitemsep]
  \item If both inequality symbols are $\leq$, then it is a \textbf{Closed Annulus}.
  \item If both inequality symbols are $<$, then it is an \textbf{Open Annulus}.
  \item If \textbf{only one} inequality symbol $<$ and the other $\leq$, then it is not an \textbf{Open Annulus}.
  \end{enumerate}
\end{definition}


%%% Local Variables:
%%% mode: latex
%%% TeX-master: shared
%%% End:



%%% Local Variables:
%%% mode: latex
%%% TeX-master: shared
%%% End:

\clearpage
\subsection{Trigonometry} \label{app:Trig}
	\subsubsection{Trigonometric Formulas} \label{subsubsec:Trig Formulas}
		\begin{equation} \label{eq:Sin plus Sin with diff Angles}
			\sin \left( \alpha \right) + \sin \left( \beta \right) = 2 \sin \left( \frac{\alpha + \beta}{2} \right) \cos\left( \frac{\alpha - \beta}{2} \right)  
		\end{equation}
		\begin{equation} \label{eq:Cosine-Sine Product}
			\cos \left( \theta \right) \sin \left( \theta \right) = \frac{1}{2} \sin \left( 2 \theta \right)
		\end{equation}
	
	\subsubsection{Euler Equivalents of Trigonometric Functions} \label{subsubsec:Euler Equivalents}
		\begin{equation} \label{eq:Euler Sin}
			\sin \left( x \right) = \frac{e^{\imath x} + e^{-\imath x}}{2}
		\end{equation}
		\begin{equation} \label{eq:Euler Cos}
			\cos \left( x \right) = \frac{e^{\imath x} - e^{-\imath x}}{2 \imath}
		\end{equation}
		\begin{equation} \label{eq:Euler Sinh}
			\sinh \left( x \right) = \frac{e^{x} - e^{-x}}{2}
		\end{equation}
		\begin{equation} \label{eq:Euler Cosh}
			\cosh \left( x \right) = \frac{e^{x} + e^{-x}}{2}
		\end{equation}

\clearpage
\section{Calculus}\label{app:Calculus}
\subsection{L'Hopital's Rule}\label{subsec:LHopitals_Rule}
L'Hopital's Rule can be used to simplify and solve expressions regarding limits that yield irreconcialable results.
\begin{lemma}[L'Hopital's Rule]\label{lemma:LHopitals_Rule}
  If the equation
  \begin{equation*}
    \lim\limits_{x \rightarrow a} \frac{f(x)}{g(x)} =
    \begin{cases}
      \frac{0}{0} \\
      \frac{\infty}{\infty} \\
    \end{cases}
  \end{equation*}
  then \Cref{eq:LHopitals_Rule} holds.
  \begin{equation}\label{eq:LHopitals_Rule}
    \lim\limits_{x \rightarrow a} \frac{f(x)}{g(x)} = \lim\limits_{x \rightarrow a} \frac{f'(x)}{g'(x)}
  \end{equation}
\end{lemma}

\subsection{Fundamental Theorems of Calculus}\label{subsec:Fundamental Theorem of Calculus}
\begin{definition}[First Fundamental Theorem of Calculus]\label{def:1st Fundamental Theorem of Calculus}
  The \emph{first fundamental theorem of calculus} states that, if $f$ is continuous on the closed interval $\left[ a,b \right]$ and $F$ is the indefinite integral of $f$ on $\left[ a,b \right]$, then

  \begin{equation}\label{eq:1st Fundamental Theorem of Calculus}
    \int_{a}^{b}f \left( x \right) dx = F \left( b \right) - F \left( a \right)
  \end{equation}
\end{definition}

\begin{definition}[Second Fundamental Theorem of Calculus]\label{def:2nd Fundamental Theorem of Calculus}
  The \emph{second fundamental theorem of calculus} holds for $f$ a continuous function on an open interval $I$ and $a$ any point in $I$, and states that if $F$ is defined by

  \begin{equation*}
    F \left( x \right) = \int_{a}^{x} f \left( t \right) dt,
  \end{equation*}
  then
  \begin{equation}\label{eq:2nd Fundamental Theorem of Calculus}
    \begin{aligned}
      \frac{d}{dx} \int_{a}^{x} f \left( t \right) dt &= f \left( x \right) \\
      F' \left( x \right) &= f \left( x \right) \\
    \end{aligned}
  \end{equation}
\end{definition}

\begin{definition}[argmax]\label{def:argmax}
  The arguments to the \emph{argmax} function are to be maximized by using their derivatives.
  You must take the derivative of the function, find critical points, then determine if that critical point is a global maxima.
  This is denoted as
  \begin{equation*}\label{eq:argmax}
    \argmax_{x}
  \end{equation*}
\end{definition}

\subsection{Rules of Calculus}\label{subsec:Rules of Calculus}
\subsubsection{Chain Rule}\label{subsubsec:Chain Rule}
\begin{definition}[Chain Rule]\label{def:Chain Rule}
  The \emph{chain rule} is a way to differentiate a function that has 2 functions multiplied together.

  If
  \begin{equation*}
    f(x) = g(x) \cdot h(x)
  \end{equation*}
  then,
  \begin{equation}\label{eq:Chain Rule}
    \begin{aligned}
      f'(x) &= g'(x) \cdot h(x) + g(x) \cdot h'(x) \\
      \frac{df(x)}{dx} &= \frac{dg(x)}{dx} \cdot g(x) + g(x) \cdot \frac{dh(x)}{dx} \\
    \end{aligned}
  \end{equation}
\end{definition}

\subsection{Useful Integrals}\label{subsec:Useful_Integrals}
\begin{equation}\label{eq:Cosine_Indefinite_Integral}
  \int \cos(x) \; dx = \sin(x)
\end{equation}

\begin{equation}\label{eq:Sine_Indefinite_Integral}
  \int \sin(x) \; dx = -\cos(x)
\end{equation}

\begin{equation}\label{eq:x_Cosine_Indefinite_Integral}
  \int x \cos(x) \; dx = \cos(x) + x \sin(x)
\end{equation}
\Cref{eq:x_Cosine_Indefinite_Integral} simplified with Integration by Parts.

\begin{equation}\label{eq:x_Sine_Indefinite_Integral}
  \int x \sin(x) \; dx = \sin(x) - x \cos(x)
\end{equation}
\Cref{eq:x_Sine_Indefinite_Integral} simplified with Integration by Parts.

\begin{equation}\label{eq:x_Squared_Cosine_Indefinite_Integral}
  \int x^{2} \cos(x) \; dx = 2x \cos(x) + (x^{2} - 2) \sin(x)
\end{equation}
\Cref{eq:x_Squared_Cosine_Indefinite_Integral} simplified by using Integration by Parts twice.

\begin{equation}\label{eq:x_Squared_Sine_Indefinite_Integral}
  \int x^{2} \sin(x) \; dx = 2x \sin(x) - (x^{2} - 2) \cos(x)
\end{equation}
\Cref{eq:x_Squared_Sine_Indefinite_Integral} simplified by using Integration by Parts twice.

\begin{equation}\label{eq:Exponential_Cosine_Indefinite_Integral}
  \int e^{\alpha x} \cos(\beta x) \; dx = \frac{e^{\alpha x} \bigl( \alpha \cos(\beta x) + \beta \sin(\beta x) \bigr)}{\alpha^{2} + \beta^{2}} + C
\end{equation}

\begin{equation}\label{eq:Exponential_Sine_Indefinite_Integral}
  \int e^{\alpha x} \sin(\beta x) \; dx = \frac{e^{\alpha x} \bigl( \alpha \sin(\beta x) - \beta \cos(\beta x) \bigr)}{\alpha^{2}+\beta^{2}} + C
\end{equation}

\begin{equation}\label{eq:Exponential_Indefinite_Integral}
  \int e^{\alpha x} \; dx = \frac{e^{\alpha x}}{\alpha}
\end{equation}

\begin{equation}\label{eq:x_Exponential_Indefinite_Integral}
  \int x e^{\alpha x} \; dx = e^{\alpha x} \left( \frac{x}{\alpha} - \frac{1}{\alpha^{2}} \right)
\end{equation}
\Cref{eq:x_Exponential_Indefinite_Integral} simplified with Integration by Parts.

\begin{equation}\label{eq:Inverse_x_Indefinite_Integral}
  \int \frac{dx}{\alpha + \beta x} = \int \frac{1}{\alpha + \beta x} \; dx = \frac{1}{\beta} \ln (\alpha + \beta x)
\end{equation}

\begin{equation}\label{eq:Inverse_x_Squared_Indefinite_Integral}
  \int \frac{dx}{\alpha^{2} + \beta^{2} x^{2}} = \int \frac{1}{\alpha^{2} + \beta^{2} x^{2}} \; dx = \frac{1}{\alpha \beta} \arctan \left( \frac{\beta x}{\alpha} \right)
\end{equation}

\begin{equation}\label{eq:a_Exponential_Indefinite_Integral}
  \int \alpha^{x} \; dx = \frac{\alpha^{x}}{\ln(\alpha)}
\end{equation}

\begin{equation}\label{eq:a_Exponential_Derivative}
  \frac{d}{dx} \alpha^{x} = \frac{d\alpha^{x}}{dx} = \alpha^{x} \ln(x)
\end{equation}

\subsection{Leibnitz's Rule}\label{subsec:Leibnitzs_Rule}
\begin{lemma}[Leibnitz's Rule]\label{lemma:Leibnitzs_Rule}
  Given
  \begin{equation*}
    g(t) = \int_{a(t)}^{b(t)} f(x, t) \, dx
  \end{equation*}
  with $a(t)$ and $b(t)$ differentiable in $t$ and $\frac{\partial f(x, t)}{\partial t}$ continuous in both $t$ and $x$, then
  \begin{equation}\label{eq:Leibnitzs_Rule}
    \frac{d}{dt} g(t) = \frac{d g(t)}{dt} = \int_{a(t)}^{b(t)} \frac{\partial f(x, t)}{\partial t} \, dx + f \bigl[ b(t), t \bigr] \, \frac{d b(t)}{dt} - f \bigl[ a(t), t \bigr] \, \frac{d a(t)}{dt}
  \end{equation}
\end{lemma}



\clearpage
\section{Laplace Transform}\label{app:Laplace_Transform}
\subsection{Laplace Transform}\label{subsec:Laplace_Transform}
\begin{definition}[Laplace Transform]\label{def:Laplace_Transform}
  The \emph{Laplace transformation} operation is denoted as $\Lapl \lbrace x(t) \rbrace$ and is defined as
  \begin{equation}\label{eq:Laplace_Transform}
    X(s) = \int\limits_{-\infty}^{\infty} x(t) e^{-st} dt
  \end{equation}
\end{definition}

\subsection{Inverse Laplace Transform}\label{subsec:Inverse_Laplace_Transform}
\begin{definition}[Inverse Laplace Transform]\label{def:Inverse_Laplace_Transform}
  The \emph{inverse Laplace transformation} operation is denoted as $\Lapl^{-1} \lbrace X(s) \rbrace$ and is defined as
  \begin{equation}\label{eq:Inverse_Laplace_Transform}
    x(t) = \frac{1}{2j \pi} \int_{\sigma-\infty}^{\sigma+\infty} X(s) e^{st} \, ds
  \end{equation}
\end{definition}

\subsection{Properties of the Laplace Transform}\label{subsec:Laplace_Transform_Properties}
\subsubsection{Linearity}\label{subsubsec:Laplace_Linearity}
The \nameref{def:Laplace_Transform} is a linear operation, meaning it obeys the laws of linearity.
This means \Cref{eq:Laplace_Linearity} must hold.
\begin{subequations}\label{eq:Laplace_Linearity}
  \begin{equation}\label{eq:Laplace_Linearity_Time}
    x(t) = \alpha_{1} x_{1}(t) + \alpha_{2} x_{2}(t)
  \end{equation}
  \begin{equation}\label{eq:Laplace_Linearity_Frequency}
    X(s) = \alpha_{1} X_{1}(s) + \alpha_{2} X_{2}(s)
  \end{equation}
\end{subequations}

\subsubsection{Time Scaling}\label{subsubsec:Laplace_Time_Scaling}
Scaling in the time domain (expanding or contracting) yields a slightly different transform.
However, this only makes sense for $\alpha > 0$ in this case.
This is seen in \Cref{eq:Laplace_Time_Scaling}.
\begin{equation}\label{eq:Laplace_Time_Scaling}
  \Lapl \bigl\lbrace x(\alpha t) \bigr\rbrace = \frac{1}{\alpha} X \left( \frac{s}{\alpha} \right)
\end{equation}

\subsubsection{Time Shift}\label{subsubsec:Laplace_Time_Shift}
Shifting in the time domain means to change the point at which we consider $t=0$.
\Cref{eq:Laplace_Time_Shifting} below holds for shifting both forward in time and backward.
\begin{equation}\label{eq:Laplace_Time_Shifting}
  \Lapl \bigl\lbrace x(t-a) \bigr\rbrace = X(s) e^{-a s}
\end{equation}

\subsubsection{Frequency Shift}\label{subsubsec:Laplace_Frequency_Shift}
Shifting in the frequency domain means to change the complex exponential in the time domain.
\begin{equation}\label{eq:Laplace_Frequency_Shift}
  \Lapl^{-1} \bigl\lbrace X(s-a) \bigr\rbrace = x(t)e^{at}
\end{equation}

\subsubsection{Integration in Time}\label{subsubsec:Laplace_Time_Integration}
Integrating in time is equivalent to scaling in the frequency domain.
\begin{equation}\label{eq:Laplace_Time_Integration}
  \Lapl \left\lbrace \int_{0}^{t} x(\lambda) \, d\lambda \right\rbrace = \frac{1}{s} X(s)
\end{equation}

\subsubsection{Frequency Multiplication}\label{subsubsec:Laplace_Frequency_Multiplication}
Multiplication of two signals in the frequency domain is equivalent to a convolution of the signals in the time domain.
\begin{equation}\label{eq:Laplace_Frequency_Multiplication}
  \Lapl \bigl\lbrace x(t) * v(t) \bigr\rbrace = X(s) V(s)
\end{equation}

\subsubsection{Relation to Fourier Transform}\label{subsubsec:Fourier_Transform_Relation}
The Fourier transform looks and behaves very similarly to the Laplace transform.
In fact, if $X(\omega)$ exists, then \Cref{eq:Fourier_Laplace_Transform_Relation} holds.
\begin{equation}\label{eq:Fourier_Laplace_Transform_Relation}
  X(s) = X(\omega) \vert_{\omega = \frac{s}{j}}
\end{equation}

\subsection{Theorems}\label{subsec:Laplace_Theorems}
There are 2 theorems that are most useful here:
\begin{enumerate}[noitemsep]
\item \nameref{thm:Laplace_Initial_Value_Theorem}
\item \nameref{thm:Laplace_Final_Value_Theorem}
\end{enumerate}


%%% Local Variables:
%%% mode: latex
%%% TeX-master: shared
%%% End:


% To make this print, you must include a citation somewhere in the document
\clearpage
\printbibliography{}
\end{document}