\documentclass[10pt,letterpaper,final,twoside,notitlepage]{article}
\usepackage[margin=.5in]{geometry} % 1/2 inch margins on all pages
\usepackage[utf8]{inputenc} % Define the input encoding
\usepackage[USenglish]{babel} % Define language used
\usepackage{amsmath,amsfonts,amssymb}
\usepackage{amsthm} % Gives us plain, definition, and remark to use in \theoremstyle{style}
\usepackage{mathtools} % Allow for text and math in align* environment.
\usepackage{thmtools}
\usepackage{thm-restate}
\usepackage{graphicx}

\usepackage[
backend=biber,
style=alphabetic,
citestyle=authoryear]{biblatex} % Must include citation somewhere in document to print bibliography
\usepackage{hyperref} % Generate hyperlinks to referenced items
\usepackage[nottoc]{tocbibind} % Prints the Reference/Bibliography in TOC as well
\usepackage[noabbrev,nameinlink]{cleveref} % Fancy cross-references in the document everywhere
\usepackage{nameref} % Can make references by name to places
\usepackage{caption} % Allows for greater control over captions in figure, algorithm, table, etc. environments
\usepackage{subcaption} % Allows for multiple figures in one Figure environment
\usepackage[binary-units=true]{siunitx} % Gives us ways to typeset units for stuff
\usepackage{csquotes} % Context-sensitive quotation facilities
\usepackage{enumitem} % Provides [noitemsep, nolistsep] for more compact lists
\usepackage{chngcntr} % Allows us to tamper with the counter a little more
\usepackage{empheq} % Allow boxing of equations in special math environments
\usepackage[x11names]{xcolor} % Gives access to coloring text in environments or just text, MUST be before tikz
\usepackage{tcolorbox} % Allows us to create boxes of various types for examples
\usepackage{tikz} % Allows us to create TikZ and PGF Pictures
\usepackage{ctable} % Greater control over tables and how they look
\usepackage{diagbox} % Allow us to have shared diagonal cells in tables
\usepackage{multirow} % Allow us to have a single cell in a table span multiple rows
\usepackage{titling} % Put document information throughout the document programmatically
\usepackage[linesnumbered,ruled,vlined]{algorithm2e} % Allows us to write algorithms in a nice style.

\counterwithin{figure}{section}
\counterwithin{table}{section}
\counterwithin{equation}{section}
\counterwithin{algocf}{section}
\crefname{algocf}{algorithm}{algorithms}
\Crefname{algocf}{Algorithm}{Algorithms}
\setcounter{secnumdepth}{4}
\setcounter{tocdepth}{4} % Include \paragraph in toc
\crefname{paragraph}{paragraph}{paragraphs}
\Crefname{paragraph}{Paragraph}{Paragraphs}

% Create a theorem environment
\theoremstyle{plain}
\newtheorem{theorem}{Theorem}[section]
% Create a numbered theorem-like environment for lemmas
\newtheorem{lemma}{Lemma}[theorem]

% Create a definition environment
\theoremstyle{definition}
\newtheorem{definition}{Defn}
\newtheorem{corollary}{Corollary}[section]
% \begin{definition}[Term] \label{def:}
%   Make sure the term is emphasized with \emph{term}.
%   This ensures that if \emph is changed, it shows up everywhere
% \end{definition}

% Create a numbered remark environment numbered based on definition
% NOTE: This version of remark MUST go inside a definition environment
\theoremstyle{remark}
\newtheorem{remark}{Remark}[definition]
%\counterwithin{definition}{subsection} % Uncomment to have definitions use section.subsection numbering

% Create an unnumbered remark environment for general use
% NOTE: This version of remark has NO restrictions on placement
\newtheorem*{remark*}{Remark}

% Create a special list that handles properties. It can be broken and restarted
\newlist{propertylist}{enumerate}{1} % {Name}{Template}{Max Depth}
% [newlistname, LevelsToApplyTo]{formatting options}
\setlist[propertylist, 1]{label=\textbf{(\roman*)}, ref=\textbf{(\roman*)}, noitemsep, nolistsep}
\crefname{propertylisti}{property}{properties}
\Crefname{propertylisti}{Property}{Properties}

% Create a special list that handles enumerate starting with lower letters. Breakable/Restartable.
\newlist{boldalphlist}{enumerate}{1} % {Name}{Template}{Max Depth}
% [newlistname, LevelsToApplyTo]{formatting options}
\setlist[boldalphlist, 1]{label=\textbf{(\alph*)}, ref=\alph*, noitemsep, nolistsep} % Set options

\newlist{nocrefenumerate}{enumerate}{1} % {Name}{Template}{Max Depth}
% [newlistname, LevelsToApplyTo]{formatting options}
\setlist[nocrefenumerate, 1]{label=(\arabic*), ref=(\arabic*), noitemsep, nolistsep}

% Create a list that allows for deeper nesting of numbers. By default enumerate only allows depth=4.
\newlist{nestednums}{enumerate}{6}
% [newlistname, LevelsToApplyTo]{formatting options}
\setlist[nestednums]{noitemsep, label*=\arabic*.}

\tcbuselibrary{breakable} % Allow tcolorboxes to be broken across pages
% Create a tcolorbox for examples
% /begin{example}[extra name]{NAME}
% Create a tcolorbox for examples
% Argument #1 is optional, given by [], that is the textbook's problem number
% Argument #2 is mandatory, given by {}, that is the title for the example
% Avoid putting special characters, (), [], {}, ",", etc. in the title.
\newtcolorbox[auto counter,
number within=section,
number format=\arabic,
crefname={example}{examples}, % Define reference format for cref (No Capitals)
Crefname={Example}{Examples}, % Reference format for cleveref (With Capitals)
]{example}[2][]{ % The [2][] Means the first argument is optional
  width=\textwidth,
  title={Example \thetcbcounter: #2. #1}, % Parentheses and commas are not well supported
  fonttitle=\bfseries,
  label={ex:#2},
  nameref=#2,
  colbacktitle=white!100!black,
  coltitle=black!100!white,
  colback=white!100!black,
  upperbox=visible,
  lowerbox=visible,
  sharp corners=all,
  breakable
}

% Create a tcolorbox for general use
\newtcolorbox[% auto counter,
% number within=section,
% number format=\arabic,
% crefname={example}{examples}, % Define reference format for cref (No Capitals)
% Crefname={Example}{Examples}, % Reference format for cleveref (With Capitals)
]{blackbox}{
  width=\textwidth,
  % title={},
  fonttitle=\bfseries,
  % label={},
  % nameref=,
  colbacktitle=white!100!black,
  coltitle=black!100!white,
  colback=white!100!black,
  upperbox=visible,
  lowerbox=visible,
  sharp corners=all
}

% Redefine the 'end of proof' symbol to be a black square, not blank
\renewcommand{\qedsymbol}{$\blacksquare$} % Change proofs to have black square at end

% Common Mathematical Stuff
\newcommand{\Abs}[1]{\ensuremath{\lvert #1 \rvert}}
\newcommand{\DNE}{\ensuremath{\mathrm{DNE}}} % Used when limit of function Does Not Exist

% Complex Numbers functions
\renewcommand{\Re}{\operatorname{Re}} % Redefine to use the command, but not the fraktur version
\renewcommand{\Im}{\operatorname{Im}} % Redefine to use the command, but not the fraktur version
\newcommand{\Real}[1]{\ensuremath{\Re \lbrace #1 \rbrace}}
\newcommand{\Imag}[1]{\ensuremath{\Im \lbrace #1 \rbrace}}
\newcommand{\Conjugate}[1]{\ensuremath{\overline{#1}}}
\newcommand{\Modulus}[1]{\ensuremath{\lvert #1 \rvert}}
\DeclareMathOperator{\PrincipalArg}{\ensuremath{Arg}}

% Math Operators that are useful to abstract the written math away to one spot
% Number Sets
\DeclareMathOperator{\RealNumbers}{\ensuremath{\mathbb{R}}}
\DeclareMathOperator{\AllIntegers}{\ensuremath{\mathbb{Z}}}
\DeclareMathOperator{\PositiveInts}{\ensuremath{\mathbb{Z}^{+}}}
\DeclareMathOperator{\NegativeInts}{\ensuremath{\mathbb{Z}^{-}}}
\DeclareMathOperator{\NaturalNumbers}{\ensuremath{\mathbb{N}}}
\DeclareMathOperator{\ComplexNumbers}{\ensuremath{\mathbb{C}}}
\DeclareMathOperator{\RationalNumbers}{\ensuremath{\mathbb{Q}}}

% Calculus operators
\DeclareMathOperator*{\argmax}{argmax} % Thin Space and subscripts are UNDER in display

% Signal and System Functions
\DeclareMathOperator{\UnitStep}{\mathcal{U}}
\DeclareMathOperator{\sinc}{sinc} % sinc(x) = (sin(pi x)/(pi x))

% Transformations
\DeclareMathOperator{\Lapl}{\mathcal{L}} % Declare a Laplace symbol to be used

% Logical Operators
\DeclareMathOperator{\XOR}{\oplus}

% x86 CPU Registers
\newcommand{\rbpRegister}{\texttt{\%rbp}}
\newcommand{\rspRegister}{\texttt{\%rsp}}
\newcommand{\ripRegister}{\texttt{\%rip}}
\newcommand{\raxRegister}{\texttt{\%rax}}
\newcommand{\rbxRegister}{\texttt{\%rbx}}

%%% Local Variables:
%%% mode: latex
%%% TeX-master: shared
%%% End:


% These packages are more specific to certain documents, but will be availabe in the template
% \usepackage{esint} % Provides us with more types of integral symbols to use
% \usepackage[outputdir=./TeX_Output]{minted} % Allow us to nicely typeset 300+ programming languages
% \crefname{lstlisting}{listing}{listings}
% \Crefname{lstlisting}{Listing}{Listings}
% This document must be compiled with the -shell-escape flag if the packages above are uncommented

% \graphicspath{{./Drawings/PSYC_303-Intro_Psychopathology/}} % Uncomment this to use pictures in this document
\addbibresource{./Bibliographies/PSYC_303-Intro_Psychopathology.bib}
% % Define English Imperial units.
% Length
\DeclareSIUnit\inch{in}
\DeclareSIUnit\in{in}

\DeclareSIUnit\feet{ft}
\DeclareSIUnit\ft{ft}

\DeclareSIUnit\yard{yd}
\DeclareSIUnit\yd{yd}

\DeclareSIUnit\mile{mi}
\DeclareSIUnit\mi{mi}

% Volume
\DeclareSIUnit\fluidOunce{fl oz}
\DeclareSIUnit\floz{fl oz}

\DeclareSIUnit\pint{pt}
\DeclareSIUnit\pt{pt}

\DeclareSIUnit\quart{qt}
\DeclareSIUnit\qt{qt}

\DeclareSIUnit\gallon{gal}
\DeclareSIUnit\gal{gal}

% Mass
\DeclareSIUnit\grain{gr}
\DeclareSIUnit\gr{gr}

\DeclareSIUnit\ounce{oz}
\DeclareSIUnit\oz{oz}

\DeclareSIUnit\pound{lb}
\DeclareSIUnit\lb{lb}

\DeclareSIUnit\poundMass{lbm}
\DeclareSIUnit\lbm{lbm}

\DeclareSIUnit\ton{t}
\DeclareSIUnit\slug{slug}

% Temperature
\DeclareSIUnit\rankine{R}
\DeclareSIUnit[number-unit-product={}]\degreeF{\degree{}F}
\DeclareSIUnit[number-unit-product={}]\dF{\degree{}F}
\DeclareSIUnit[number-unit-product={}]\degF{\degree{}F}

\DeclareSIUnit[number-unit-product={}]\degreeR{\degree{}R}
\DeclareSIUnit[number-unit-product={}]\dR{\degree{}R}
\DeclareSIUnit[number-unit-product={}]\degR{\degree{}R}
% \DeclareSIUnit[number-unit-product={}]\Rankine{\degree{}R}
% \DeclareSIUnit[number-unit-product={}]\rankine{\degree{}R}
\DeclareSIUnit[number-unit-product={}]\degreeRankine{\degree{}R}

% Pressure
\DeclareSIUnit\bar{bar}
\DeclareSIUnit\atm{atm}
\DeclareSIUnit\psia{psia}
\DeclareSIUnit\psig{psig}
\DeclareSIUnit\psi{psi}

% Energy
\DeclareSIUnit\btu{btu}
\DeclareSIUnit\BTU{BTU}

%Force
\DeclareSIUnit\poundForce{lbf}
\DeclareSIUnit\lbf{lbf}

% volumetric flow
\DeclareSIUnit\cfm{cfm}
\DeclareSIUnit\CFM{CFM}

% Moles
\DeclareSIUnit\lbmol{lbmol}

%%% Local Variables:
%%% mode: latex
%%% TeX-master: shared
%%% End:


% Math Operators that are useful to abstract the written math away to one spot
% These are supposed to be document-specific mathematical operators that will make life easier
% Many fundamental operators are defined in Reference_Sheet_Preamble.tex

\begin{titlepage}
  \title{PSYC 303: Introduction to Psychopathology --- Reference Sheet \\ Illinois Institute of Technology}
  \author{Karl Hallsby}
  \date{Last Edited: \today} % We want to inform people when this document was last edited
\end{titlepage}

\begin{document}
\pagenumbering{gobble}
\maketitle
\pagenumbering{roman} % i, ii, iii on beginning pages, that don't have content
\tableofcontents
\clearpage
\listoftheorems[ignoreall, show={definition, Definition}]
\clearpage
\pagenumbering{arabic} % 1,2,3 on content pages

\section*{Forewarning}\label{sec:Forewarning}
Many of the definitions in this document will come from \citetitle{textbook}.

Diagnoses used here are drawn from \citetitle{dsm5}.

\section{Historical Context}\label{sec:Historical_Context}
We are concerned with both \nameref{def:Psychological_Disorder}s and \nameref{def:Psychological_Dysfunction}s.

\begin{definition}[Psychological Disorder]\label{def:Psychological_Disorder}
  There are 3 criteria for a \emph{psychological disorder}:
  \begin{enumerate}[noitemsep]
  \item \nameref{def:Psychological_Dysfunction}
  \item Distress or Impairment.
    The \nameref{def:Psychological_Dysfunction} must cause distress to the individual, or impair their ability to function ``normally''
  \item The response to the \nameref{def:Psychological_Dysfunction} must be \emph{atypical} or \emph{not culturally expected}.
    Some people with disorders that do not affect their life terribly strongly might be considered ``eccentric'' or ``talented''.
    In addition, the more productive you are, the more abnormal you can be without society making a big fuss about it.
    \begin{itemize}[noitemsep]
    \item Note that ``normal'' does \textbf{not} refer to, for example, political dissidents.
    \end{itemize}
  \end{enumerate}
\end{definition}

\begin{definition}[Psychological Dysfunction]\label{def:Psychological_Dysfunction}
  \emph{Psychological dysfunction} refers to a breakdown in cognitive, emotional, or behavioral functioning.
  For example, if you are out on a date, it should be fun.
  But if you experience severe fear all evening and just want to go home, even though there is nothing to be afraid of, and the severe fear happens on every date, your emotions are not functioning properly.

  \begin{remark}[Harmful Dysfunction]\label{rmk:Harmful_Dysfunction}
    \emph{Harmful dysfunction} is a narrowing of the idea of a \nameref{def:Psychological_Dysfunction}.
    Namely, it is concerned with whether the behavior is out of the individual's control.
  \end{remark}
\end{definition}

But it is never easy to decide what represents a \nameref{def:Psychological_Dysfunction}, and we may never be able to satisfactorily define disease or \nameref{def:Psychological_Disorder}.
The best we may be able to do is to consider how the apparent disease or disorder matches our current understanding of its \nameref{def:Prototype}.

\begin{definition}[Prototype]\label{def:Prototype}
  A \emph{prototype} is a ``typical'' profile of a disorder.
\end{definition}

A patient may present only some features or symptoms of the disorder (there exists a minimum number) and still meet criteria for the \nameref{def:Psychological_Disorder} because their set of symptoms is close to the \nameref{def:Prototype}.

\subsection{Psychopathology}\label{subsec:Psychopathology}
\begin{definition}[Psychopathology]\label{def:Psychopathology}
  \emph{Psychopathology} is the scientific study of psychological disorders.
  This field contains many different branches, including:
  \begin{itemize}[noitemsep]
  \item \nameref{def:Clinical_Psychologist}s
  \item \nameref{def:Counseling_Psychologist}s
  \item \nameref{def:Psychiatrist}s
  \item \nameref{def:Psychiatric_Social_Worker}s
  \item \nameref{def:Psychiatric_Nurse}s
  \item \nameref{def:Marriage_Family_Therapist}s
  \item \nameref{def:Mental_Health_Counselor}s
  \end{itemize}
\end{definition}

They study patients that are \nameref{def:Present}ing a problem.

\begin{definition}[Present]\label{def:Present}
  \emph{Present} is a traditional shorthand way of indicating why the person came to the clinic.
\end{definition}

The effects that the patient \nameref{def:Present}s are used to form a \nameref{def:Clinical_Description}.
\begin{definition}[Clinical Description]\label{def:Clinical_Description}
  A \emph{clinical description} represents the unique combination of behaviors, thoughts, and feelings that make up a specific disorder.
  The clinical portion of the phrase refers both to the types of problems or disorders that you would find in a clinic or hospital and to the activities connected with assessment and treatment.
\end{definition}

Both the course of the \nameref{def:Psychological_Disorder} and the onset are used to determine the \nameref{def:Prognosis}.

\begin{definition}[Prognosis]\label{def:Prognosis}
  The anticipated course of a disorder is called the \emph{prognosis}.
\end{definition}

\subsubsection{Branches of Psychopathology}\label{subsubsec:Psychopathology_Branches}
\begin{definition}[Clinical Psychologist]\label{def:Clinical_Psychologist}
  \emph{Clinical psychologists} receive the Ph.D., doctor of philosophy, degree (or sometimes an Ed.D., doctor of education, or Psy.D., doctor of psychology) and follow a course of graduate-level study lasting approximately 5 years, which prepares them to conduct research into the causes and treatment of psychological disorders and to diagnose, assess, and treat these disorders.
  Clinical psychologists usually concentrate on more severe \nameref{def:Psychological_Disorder}s.
\end{definition}

\begin{definition}[Counseling Psychologist]\label{def:Counseling_Psychologist}
  \emph{Counseling psychologists} receive the Ph.D., doctor of philosophy, degree (or sometimes an Ed.D., doctor of education, or Psy.D., doctor of psychology) and follow a course of graduate-level study lasting approximately 5 years, which prepares them to conduct research into the causes and treatment of psychological disorders and to diagnose, assess, and treat these disorders.
  Counseling psychologists tend to study and treat adjustment and vocational issues encountered by relatively healthy individuals.
\end{definition}

\begin{definition}[Psychiatrist]\label{def:Psychiatrist}
  \emph{Psychiatrists} first earn an M.D.\ degree in medical school and then specialize in psychiatry during residency training that lasts 3 to 4 years.
  Psychiatrists also investigate the nature and causes of psychological disorders, often from a biological point of view; make diagnoses; and offer treatments.
  Many psychiatrists emphasize drugs or other biological treatments, although most use psychosocial treatments as well.
\end{definition}

\begin{definition}[Psychiatric Social Worker]\label{def:Psychiatric_Social_Worker}
  \emph{Psychiatric social workers} typically earn a master's degree in social work as they develop expertise in collecting information relevant to the social and family situation of the individual with a \nameref{def:Psychological_Disorder}.
  Social workers also treat \nameref{def:Psychological_Disorder}s, often concentrating on family problems associated with them.
\end{definition}

\begin{definition}[Psychiatric Nurse]\label{def:Psychiatric_Nurse}
  \emph{Psychiatric nurses} have advanced degrees, such as a master's or even a Ph.D., and specialize in the care and treatment of patients with \nameref{def:Psychological_Disorder}s, usually in hospitals as part of a treatment team.
\end{definition}

\begin{definition}[Marriage and Family Therapist]\label{def:Marriage_Family_Therapist}
  \emph{Marriage and family therapists} typically spend 1 to 2 years earning a master’s degree and are employed to provide clinical services by hospitals or clinics, usually under the supervision of a doctoral-level clinician.
\end{definition}

\begin{definition}[Mental Health Counselor]\label{def:Mental_Health_Counselor}
  \emph{Mental health counselors} typically spend 1 to 2 years earning a master’s degree and are employed to provide clinical services by hospitals or clin ics, usually under the supervision of a doctoral-level clinician.
\end{definition}

\subsubsection{Some Concerns of Psychopathology}\label{subsubsec:Psychopathology_Concerns}
\nameref{def:Psychopathology} works with individuals to improve their particular issue.
But, they also have some other concerns as well.

\begin{definition}[Prevalence]\label{def:Prevalence}
  \emph{Prevalence} is how many people in the population as a whole have the disorder.
\end{definition}

\begin{definition}[Incidence]\label{def:Incidence}
  \emph{Incidence} is the statistic on how many new cases occur during a given period.
\end{definition}

Some other statistics include:
\begin{itemize}[noitemsep]
\item The sex ratio, what percentage of males and females have the disorder.
\item The typical age of onset, which often differs from one disorder to another.
\item The course of the \nameref{def:Psychological_Disorder}.
\end{itemize}

\subsection{Courses of Psychological Disorders}
\begin{definition}[Chronic Course]\label{def:Chronic_Course}
  A \emph{chronic course}, means that the \nameref{def:Psychological_Disorder} tends to last a long time, sometimes a lifetime.
\end{definition}

\begin{definition}[Episodic Course]
  An \emph{episodic course}, means that the \nameref{def:Psychological_Disorder} will have a sequence of episodes in which the individual recovers, only to suffer a recurrence of the disorder.
  This may occur throughout the individual's life.
\end{definition}

\begin{definition}[Time-Limited Course]\label{def:Time_Limited_Course}
  \emph{Time-limited course}, meaning the disorder will improve without treatment in a relatively short period with little or no risk of recurrence.
\end{definition}

\subsection{Onset of Psychological Disorders}
The onset of particular \nameref{def:Psychological_Disorder}s can vary quite widely.

\begin{definition}[Acute Onset]\label{def:Acute_Onset}
  An \emph{acute onset} \nameref{def:Psychological_Disorder} is one where the disorder begins quite suddenly.
\end{definition}

\begin{definition}[Insidious Onset]\label{def:Insidious_Onset}
  An \emph{acute onset} \nameref{def:Psychological_Disorder} is one where the disorder begins quite suddenly.
\end{definition}

\subsection{Models of Abnormal Behavior}\label{subsec:Models_Abnormal_Behavior}
There are three main models that have been and continue to be used today:
\begin{enumerate}[noitemsep]
\item The \nameref{def:Supernatural_Model}
\item The \nameref{def:Biological_Model}
\item The \nameref{def:Psychological_Model}
\end{enumerate}

\subsubsection{The Supernatural Model}\label{subsubsec:Supernatural_Model}
Deviant behavior has been considered a reflection of the battle between good and evil.
When confronted with unexplainable, irrational behavior and by suffering and upheaval, people have perceived evil.

\begin{definition}[Supernatural Model]\label{def:Supernatural_Model}
  The driving factors in the \emph{supernatural model} are:
  \begin{itemize}[noitemsep]
  \item Divinities
  \item Demons
  \item Spirits
  \item Other phenomena such as:
    \begin{itemize}[noitemsep]
    \item Magnetic fields
    \item The moon
    \item The stars
    \end{itemize}
  \end{itemize}
\end{definition}

While this model is still alive today, it is limited to small religious sects and primitive cultures.
Most members of organized religions today turn to psychology and medical science for help with \nameref{def:Psychological_Disorder}s.

\subsubsection{The Biological Model}\label{subsubsec:Biological_Model}
The \nameref{def:Biological_Model} was initially developed by Hippocrates his associates left a body of work called the \textit{Hippocratic Corpus}, written between 450 and 350 \textsc{b.c.}.
In this work, they suggested that \nameref{def:Psychological_Disorder}s could be treated like any other disease.
In addition, they did not limit their search for the causes of \nameref{def:Psychopathology} to the general area of ``disease,'' because they believed that psychological disorders might also be caused by brain pathology or head trauma and could be influenced by heredity (genetics).

\begin{definition}[Biological Model]\label{def:Biological_Model}
  The \emph{biological model} states that abnormal behavior arises from the body influencing the mind.
\end{definition}

This model has gone through phases of interest and disinterest.
When it becomes the only explanation for mental health issues, it tends to lead to focusing solely on improving the life of the patient through rest, relaxation, proper diet, proper exercise, etc.
However, when taken to the extreme, this leads to a complete neglect and disinterest in the potential mental and emotional explanations for various \nameref{def:Psychological_Disorder}s.

\subsubsection{The Psychological Model}\label{subsubsec:Psychological_Model}
Plato was one of the first to explain that mental health issues were caused by poor environments and learning centers.
For example, Plato thought that the two causes of maladaptive behavior were the social and cultural influences in one’s life and the learning that took place in that environment.
If something was wrong in the environment, such as abusive parents, one’s impulses and emotions would overcome reason.
In his mind, the best treatment was to reeducate the individual through rational discussion so that the power of reason would predominate.

This formed the precursor to modern psychosocial treatment.

\begin{definition}[Psychological Model]\label{def:Psychological_Model}
  The \emph{psychological model} states that abnormal behavior arises from the mind influencing the body.

  In history, there were two movements in this model:
  \begin{enumerate}[noitemsep]
  \item Psychoanalytic, which involved analyzing the mind and the patient's history to determine their treatment.
  \item Behavioral, which involved identifying the issue and changing the patient's behavior to the issue through slow changes to the situation intended to show there is not inherent discomfort.
  \end{enumerate}
\end{definition}

Freud stated that the mind has three components:
\begin{enumerate}[noitemsep]
\item The \nameref{def:Id}.
\item The \nameref{def:Ego}.
\item The \nameref{def:Superego}.
\end{enumerate}

\begin{definition}[Id]\label{def:Id}
  The \emph{id} is the source of our strong sexual and aggressive feelings or energies.
  It is, basically, the animal within us; if totally unchecked, it would make us all rapists or killers.

  The id processes information according to the \emph{primary process}, which is emotional, irrational, illogical, fantastical, and preoccupied with sex, aggression, selfishness, and envy.
\end{definition}

\begin{definition}[Ego]\label{def:Ego}
  The \emph{ego} is the part of our mind that ensures that we act realistically.
  It  operates according to the according to the reality principle instead of the pleasure principle of the \nameref{def:Id}.

  The cognitive operations or thinking styles of the ego are characterized by logic and reason and are referred to as the \emph{secondary process}.

  The ego is responsible for ensuring the \nameref{def:Superego} and \nameref{def:Id} are in balance.
\end{definition}

\begin{definition}[Superego]\label{def:Superego}
  The \emph{superego}, represents the moral principles instilled in us by our parents and our culture.
  It is the voice within us that nags at us when we know we’re doing something wrong.

  It is fundamentally at odds with the \nameref{def:Id}.
\end{definition}

\subsection{Scientific Method and Integrative Approach}\label{subsec:Scientific_Method_Integrative_Approach}
This is the modern approach to psychopathology.
It is founded on the principle that any time a person does something, both the brain and the body are working toegether.
In addition, the person's thoughts are influencing their actions, and together they form our response.

%%% Local Variables:
%%% mode: latex
%%% TeX-master: "../PSYC_303-Intro_Psychopathology-Reference_Sheet"
%%% End:


\section{Integrative Approach to Psychopathology}\label{sec:Integrative_Approach}
We prefer to use a \emph{multidimensional integrative approach} to \nameref{def:Psychopathology}.
These dimensions include:
\begin{description}[noitemsep]
\item[Biological] Causal factors from genetics and neuroscience.
\item[Psychological] Causal factors from behavioral and cognitive processes, including learned helplessness, social learning, prepared learning, and unconscious processes.
\item[Emotional] These contribute in a variety of ways.
  Emotions play a substantial role in the development of many disorders.
\item[Developmental] These factors are always present.
\end{description}

Each of the above dimensions are \textbf{never} in isolation, each one is strongly influenced by the others.

\subsection{Genetic Contributions to Psychopathology}\label{subsec:Genetic_Contributions}
Genetics influence our physical characteristics such as hair color and eye color.
However, genetics only provide boundaries for our body to work with.
The environment influences our physical appearance too.
To some extent, our weight and even our height are affected by nutritional, social, and cultural factors.

Much of our development and, interestingly, most of our behavior, our personality, and even our intelligence are probably \nameref{def:Polygenic}.

\begin{definition}[Polygenic]\label{def:Polygenic}
  For something to be \emph{polygenic}, the end result must be reached when multiple genes are being expressed.
\end{definition}

For psychological disorders, the evidence indicates that genetic factors make some contribution to all disorders but account for less than half of the explanation.

\subsubsection{The Diathesis-Stress Model}\label{subsubsec:Diathesis-Stress_Model}
\begin{definition}[Diathesis-Stress Model]\label{def:Diathesis-Stress_Model}
  The \emph{diathesis-stress model} states that individuals inherit tendencies to express certain traits or behaviors, which may then be activated under conditions of stress.
\end{definition}

This model of gene–environment interactions has been popular, although, in view of the relationship of the environment to the structure and function of the brain, it is greatly oversimplified.

\subsubsection{The Gene-Environment Correlation Model}\label{subsubsec:Gene-Environment_Correlation_Model}
\begin{definition}[Gene-Environment Correlation Model]\label{def:Gene-Environment_Correlation_Model}
  The \emph{gene-environment correlation model} states that a person may have a tendency towards a particular psychological issue, but also have genes that help incite that disorder.
  To that end, the genetic predisposition helps cause stresses that add to the person's predisposition to the \nameref{def:Psychological_Disorder}, eventually causing issues.
\end{definition}

Although the environment cannot change our DNA, it can change the gene expression.
Genes are turned on or off by cellular material that is located just outside of the genome (``epi,'' as in the word epigenetics, means on or around) and that stress, nutrition, or other factors can affect this epigenome, which is then immediately passed down to the next generation and maybe for several generations.

\subsection{Neuroscience and its Contributions to Psychopathology}\label{subsec:Neuroscience_Contributions_Psychopathology}
The brain is made up of nerve cells called \nameref{def:Neuron}.

\begin{definition}[Neuron]\label{def:Neuron}
  \emph{Neuron}s are the nerve cells in the brain.
  They are composed of a branch called a \nameref{def:Dendrite} and an \nameref{def:Axon}.
\end{definition}

\begin{definition}[Dendrite]\label{def:Dendrite}
  \emph{Dendrites} \textbf{receive} messages from other neurons.
\end{definition}

\begin{definition}[Axon]\label{def:Axon}
  \emph{Axon} is responsible for \textbf{sending} messages to other neurons.
\end{definition}

\begin{definition}[Synapse]\label{def:Synapse}
  The \nameref{def:Dendrite}-\nameref{def:Axon} connections between neurons are \emph{synapses}.
\end{definition}

The brain creates \emph{neurotransmitters} to communicate between \nameref{def:Neuron}s.
When communicating with the rest of the body, the brain and endocrine system work together to communicate with hormones.
Many of these hormones can be related (the extent is unknown) to many \nameref{def:Psychological_Disorder}s that we have identified.
However, our current understanding is that a lack of a particular neurotransmitter will alter the way we process information and the way we exhibit certain kinds of behavior.

\subsubsection{Neurotransmitter Changes}\label{subsubsec:Neurotransmitter_Changes}
\begin{definition}[Agonist]\label{def:Agonist}
  \emph{Agonists} effectively increase the activity of a neurotransmitter by mimicking its effects.
\end{definition}

\begin{definition}[Antagonist]\label{def:Agonist}
  \emph{Antagonists} that decrease, or block, a neurotransmitter.
\end{definition}

\begin{definition}[Inverse Agonist]\label{def:Inverse_Agonist}
  \emph{Inverse agonists} that produce effects opposite to those produced by the neurotransmitter.
\end{definition}

\subsubsection{Psychosocial Influences on Brain Structure and Function}\label{subsubsec:Psychosocial_Influences_Brain_Structure_Function}
The initiating factors (reason why a problem develops) are not necessarily the same as the maintaining factors (reason why a problem persists).
In order to treat the problem effectively, it is typically more important to know and target the maintaining factors than the initiating factors.

\subsection{Behavioral and Cognitive Science}\label{subsec:Behavioral_Cognitive_Science}
\begin{definition}[Cognitive Science]\label{def:Cognitive_Science}
  \emph{Cognitive science} is concerned with how we acquire and process information and how we store and ultimately retrieve it.
\end{definition}

Complex cognitive processing of information, as well as emotional processing, is involved when conditioning occurs.

\begin{definition}[Learned Helplessness]\label{def:Learned_Helplessness}
  \emph{Learned helplessness} occurs when rats or other animals encounter conditions over which they have no control.
  The animal gives up attempting to cope and seem to develop the animal equivalent of depression.
\end{definition}

Organisms do not have to experience certain events in their environment to learn effectively.
Rather, they can learn just as much by observing what happens to someone else in a given situation.
This fairly obvious discovery came to be known as \emph{modeling} or \emph{observational learning}.

\subsection{Emotions}\label{subsec:Emotions}
The alarm reaction that activates during potentially life-threatening emergencies is called the \emph{flight or fight response}.

To define ``emotion'' is difficult, but most theorists agree that it is linked to an action tendency (a tendency to behave in a certain way), elicited by an external event (a threat) and a feeling state (terror) and accompanied by a (possibly) characteristic physiological response.
Emotions are usually short-lived, temporary states lasting from several minutes to several hours, occurring in response to an external event.

\begin{definition}[Mood]\label{def:Mood}
  \emph{Mood} is a more persistent period of affect or emotionality.
\end{definition}

\begin{definition}[Affect]\label{def:Affect}
  \emph{Affect} is the response to a given situation at a given point in time.
  For example, we laugh when we say something funny or look sad when we talk about something sad.
\end{definition}

Emotion scientists now agree that emotion is composed of three related components-behavior, physiology, and cognition-but most emotion scientists tend to concentrate on one component or another.

\subsection{Cultural, Social, Interpersonal Factors}\label{subsec:Cultural_Social_Interpersonal_Factors}
In many cultures around the world, individuals may suffer from fright disorders, which are characterized by exaggerated startle responses, and other observable fear and anxiety reactions.

Fear and phobias are universal, occurring across all cultures.
But what we fear is strongly influenced by our social environment.

Our gender doesn’t cause psychopathology.
We think these substantial differences have to do with, at least in part, cultural expectations of men and women, or our gender roles.
But because gender role is a social and cultural factor that influences the form and content of a disorder.

Many studies have demonstrated that the greater the number and frequency of social relationships and contacts, the longer you are likely to live.
Interestingly, it is not just the absolute number of social contacts that is important.
It is the actual perception of loneliness.

Psychological disorders continue to carry a substantial stigma in our society.
To be anxious or depressed is to be weak and cowardly.
To be schizophrenic is to be unpredictable and crazy.

\subsection{Life-Span Development}\label{subsec:Life-Span_Development}
We tend to look at psychological disorders from a snapshot perspective: we focus on a particular point in a person’s life and assume it represents the whole person.

For example, in depressive (mood) disorders, children and adolescents do not receive the same benefit from antidepressant drugs as do adults, and for many of them these drugs pose risks that are not present in adults.

%%% Local Variables:
%%% mode: latex
%%% TeX-master: "../PSYC_303-Intro_Psychopathology-Reference_Sheet"
%%% End:


\section{Clinical Assessment}\label{sec:Clinical_Assessment}
\begin{definition}[Clinical Assessment]\label{def:Clinical_Assessment}
  \emph{Clinical assessment} is the systematic evaluation and measurement of psychological, biological, and social factors in an individual presenting with a possible psychological disorder.
\end{definition}

A \nameref{def:Clinical_Assessment} is used to help make a \nameref{def:Diagnosis}.

\begin{definition}[Diagnosis]\label{def:Diagnosis}
  \emph{Diagnosis} is the process of determining whether the particular problem afflicting the individual meets all criteria for a psychological disorder, as set forth in the fifth edition of the Diagnostic and Statistical Manual of Mental Disorders, or DSM-5.
\end{definition}

\subsection{Key Concepts}\label{subsec:Key_Clinical_Assessment_Concepts}
There are three major concepts that are incredibly important to know about when discussing \nameref{def:Clinical_Assessment}s and diagnoses.

\begin{enumerate}[noitemsep]
\item \nameref{def:Reliability}
\item \nameref{def:Validity}
\item \nameref{def:Standardization}
\end{enumerate}

\begin{definition}[Reliability]\label{def:Reliability}
  \emph{Reliability} is the degree to which a measurement is consistent.

  This means that running a particular test multiple times will yield similar, if not identical, diagnoses.
\end{definition}

\begin{definition}[Validity]\label{def:Validity}
  \emph{Validity} is whether something measures what it is designed to measure.

  This means that the test actual does what it says it does.
\end{definition}

\begin{definition}[Standardization]\label{def:Standardization}
  \emph{Standardization} is the process by which a certain set of standards or norms is determined for a technique to make its use consistent across different measurements.

  For example, when conducting a test, the proctor might have to give the directionsin a certain order, speak in at a certain pace, and then cannot communicate any further.
  This would be followed exactly the same way by every person every time the test is administered.
\end{definition}

\subsection{The Clinical Interview}\label{subsec:The_Clinical_Interview}
This is the opportunity for the health professional to gather:
\begin{itemize}[noitemsep]
\item The patient's personal information
\item The patient's current and past behavior, attitudes, and emotions
\item The patient's family information
\item The patient's friend group
\item Information on sexual development, religious attitudes, relevant cultural concerns, and educational history.
\end{itemize}

To achieve this, many clinicians use a \nameref{def:Mental_Status_Exam}.

\begin{definition}[Mental Status Exam]\label{def:Mental_Status_Exam}
  A \emph{mental status exam} involves the systematic observation of an individual's behavior.
  This type of observation occurs when any one person interacts with another.
  The clinician is trained to organize these observations such that they help provide information about the patient and their condition.

  The exam covers 5 categories:
  \begin{enumerate}[noitemsep]
  \item Appearance and behavior
  \item Thought processes --- Including rate of speech, speech patterns, etc.
  \item \nameref{def:Mood} and \nameref{def:Affect}
  \item Intellectual functioning
  \item Sensorium --- The general awareness of our surroundings, namely, is the patient \nameref{def:Oriented_Times_Three}.
  \end{enumerate}
\end{definition}

\begin{definition}[Oriented Times Three]\label{def:Oriented_Times_Three}
  To be \emph{oriented times three} requires that the patient is aware of who they are and who they are communicating with (person), where they are (place), and when it is (time).
\end{definition}

Such a \nameref{def:Mental_Status_Exam} can help identify various \nameref{def:Delusion}s.

\begin{definition}[Delusion]\label{def:Delusion}
  A \emph{delusion} is a distorted view of reality.

  There are several general types of delusion:
  \begin{itemize}[noitemsep]
  \item \nameref{def:Delusion_of_Persecution}
  \item \nameref{def:Delusion_of_Grandeur}
  \item \nameref{def:Ideas_of_Reference}
  \item \nameref{def:Hallucination}s
  \end{itemize}
\end{definition}

\begin{definition}[Delusion of Persection]\label{def:Delusion_of_Persecution}
  A \emph{delusion of persecution} is where the patient believes people are after them and out to get them all the time.
\end{definition}

\begin{definition}[Delusion of Grandeur]\label{def:Delusion_of_Grandeur}
  A \emph{delusion of grandeur} is where an individual thinks they are all-powerful in some way.
\end{definition}

\begin{definition}[Ideas of Reference]\label{def:Ideas_of_Reference}
   \emph{Ideas of reference} is where everything everyone else does somehow relates back to the individual.
\end{definition}

\begin{definition}[Hallucination]\label{def:Hallucination}
  \emph{Hallucinations} are things a person sees or hears when those things really aren't there.
\end{definition}

\begin{remark*}
  Patients tend to have a good general idea of the underlying issue they are facing.
\end{remark*}

\subsubsection{Interview Structure}\label{subsubsec:Interview_Structure}
Clinical interviews have three general structures:
\begin{description}[noitemsep]
\item[Structured] In a structured clinical interview, the clinician asks a set of questions from a script in an exact order and reocrds the patient's answers.
  This has the benefit of being very reliable, but not flexible.
  This can also lead to very jarring interviews, as topics that may be discussed in-depth are abruptly switched to or from.
\item[Unstructured] In an unstructured clinical interview, there is no script whatsoever.
  The clinician is free to ask questionsin any way they see fit.
  This allows for great flexibility, but next to no reliability.
  Because this is more of a conversation than an interview, both the patient and clinician may act/react differently each time.
\item[Semistructured] In a semistructured interview, the interview takes on a mix of both structured and unstructured interviews.
  The clinician has a script of questions they must ask, but they are also free to follow any information the client brings up.
  This offers a good balance of reliability and flexibility.
\end{description}

\subsection{Physical Examination}\label{subsec:Physical_Examination}
If the patient is presenting symptoms of a mental condition that could also be caused by physical factors, the clinician will typically suggest a physical examination to rule out those possibilities.
A short list of physical items that can cause mental issues are:
\begin{itemize}[noitemsep]
\item Bad food
\item Wrong amount of medication
\item Wrong type of medication
\item Onset of a medical condition (hyper-/hypo-thyroidism can cause symptoms that mimic mental issues)
\item Substances (alcohol, marijuana, other drugs)
\end{itemize}

\subsection{Behavioral Assessment}\label{subsec:Behavioral_Assessment}
\nameref{def:Behavioral_Assessment} may be used in conjunction with clinical interviews.c

\begin{definition}[Behavioral Assessment]\label{def:Behavioral_Assessment}
  \emph{Behavioral assessment} takes the observation process from clinical interviews and \nameref{def:Mental_Status_Exam}s one step further by using direct observation to formally assess an individual's thoughts, feelings, and behavior in specific situations or contexts.
\end{definition}

In a \nameref{def:Behavioral_Assessment}, target behaviors are identified and observed with the goal of determining the factors that seem to influence them.

There is a short-hand to remember the method of \textbf{formal} observation used in \nameref{def:Behavorial_Assessment}s:
\begin{description}[noitemsep]
\item[A] \nameref{def:Antecedent}
\item[B] \nameref{def:Behavior}
\item[C] \nameref{def:Consequences}
\end{description}

\begin{definition}[Antecedent]\label{def:Antecedent}
  The \emph{antecedent} is the situation that leads up to a formally observable behavior in a \nameref{def:Behavior_Assessment}.
\end{definition}

\begin{definition}[Behavior]\label{def:Behavior}
  The \emph{behavior} is the formally observable behavior in a \nameref{def:Behavior_Assessment}.
\end{definition}

\begin{definition}[Consequences]\label{def:Consequences}
  The \emph{consequences} are the consequences that follow due to a formally observable behavior in a \nameref{def:Behavior_Assessment}.
\end{definition}

%%% Local Variables:
%%% mode: latex
%%% TeX-master: "../PSYC_303-Intro_Psychopathology-Reference_Sheet"
%%% End:


%====================================APPENDIX====================================
\appendix
\counterwithin{definition}{subsection}

% To make this print, you must include a citation somewhere in the document
\clearpage
\printbibliography{}
\end{document}

%%% Local Variables:
%%% mode: latex
%%% TeX-master: t
%%% End:
